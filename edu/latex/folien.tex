%
%	LaTeX - SCHABLONE fuer die VORTRAGSFOLIEN
%	©egst · 12.09.2oo8
%
\documentclass[a4paper]{slides}

\usepackage[ngerman]{babel}
\usepackage[utf8]{inputenc}
\usepackage{graphicx}
\usepackage{color}
\usepackage{amsmath}
\usepackage{amsfonts}

\newcommand\lvtyp{PROSEMINAR}
\newcommand\lvname{Algorithmen der Mustererkennung und Künstlichen Intelligenz}
\newcommand\lvinst{Institut für Informatik FSU Jena}

%
%	HIER WERDEN TITEL REFERENT UND DATUM EINGETRAGEN
%
\newcommand\svthema{Kaltblüter in großstätischer Heimtierhaltung}
\newcommand\svperson{Steven Spielberg}
\newcommand\svdatum{14.November 2oo8}

\begin{document}

\title{ \textbf{\color{blue}\svthema} }
\author{ \emph{\color{red}\svperson} }
\date{ \textrm{\lvtyp} \\ \emph{"`\lvname"'} \\ \svdatum }
\maketitle

\begin{slide}
\textbf{\Large Inhalt}
	\vfill
\begin{itemize}
\item	ÃÜbersicht
\item	Entwicklungsgeschichte
\item	Theoretische Grundlagen
\item	Algorithmen \& Systemarchitektur
\item	Stand der Technik
\item	Anwendungsbeispiele \& Produkte
\item	Zusammenfassung und Ausblick
\end{itemize}
	\vfill
\end{slide}

\begin{slide}
\textbf{\Large Theoretische Grundlagen}
	\vfill
Univariate Normalverteilung
	\[
	f (x \mid \mu, \sigma^2) ~=~ \frac 1 {\sigma\sqrt{2\pi}}
		\cdot \exp\left\{
			-\frac {(x-\mu)^2} {2\sigma^2}
			\right\}
	\]
	\par
Multivariate Normalverteilung
	\[
	f (\vec x \mid \vec\mu, S) ~=~ f (\vec x - \vec\mu \mid \vec0, S)
	\]
mit
	\[
	f (\vec x \mid \vec0, S) ~=~ \det(2\pi S)^{-1/2}
		\cdot \exp \left\{
			-\frac 12 \cdot \vec x^\top S^{-1} \vec x
			\right\}
	\]
	\par
Definition der quadratischen Form:
	\[
	\vec x^\top A \vec x ~:=~ \sum_i \sum_j x_i A_{ij} x_j
	\]
	\vfill
\end{slide}

\begin{slide}
\textbf{\Large Systemarchitektur}
	\vfill
\includegraphics [width=\textwidth] {bildchen}
	\vfill
\includegraphics [width=\textwidth,height=0.3\textheight] {bildchen}
	\vfill
\end{slide}

\end{document}
