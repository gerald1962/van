\documentclass[10pt,a4paper]{article}

\usepackage[ngerman]{babel}		 % Deutsche Namen/Umlaute
\usepackage[utf8]{inputenc}		 % Zeichensatzkodierung
\usepackage[usenames,dvipsnames]{color}  % Using the color package, not xcolor
\usepackage{makeidx}
\makeindex

\setcounter{secnumdepth}{4}

\renewcommand{\theparagraph}{\arabic{section}.\arabic{paragraph}.}
% \renewcommand{\theparagraph}{\arabic{paragraph}.}

\usepackage{chngcntr}
\counterwithout*{paragraph}{section}  %% stop resetting paragraph number with each new subsubsection
% \counterwithin*{paragraph}{section}   %% reset paragraph number for each section; only works with the preceding line!

\definecolor{coquelicot}{rgb}{1.0, 0.22, 0.0}

%	HIER WERDEN TITEL REFERENT UND DATUM EINGETRAGEN
%
\newcommand\svthema{Pädagogik und Psychologie}
\newcommand\svperson{Leoni Heber, Gerald Schüller}
\newcommand\svdatum{\today}
\newcommand\lvtyp{Schuljahr 2022/23}
\newcommand\lvinst{Berufsschule für Kinderpflege - Höchstadt a. d. Aisch}

\begin{document}

\title{ \textbf{\color{blue}\svthema} }
\author{ \textsl{\color{red}\svperson} --- \svdatum }
\date{ \small {\lvtyp} - {\lvinst} }
\maketitle

\tableofcontents

\newpage
\section{Das aktuelle Bild vom Kind}

\paragraph{\color{coquelicot} {Definition:}} \bf{Bindung}
\footnote{\label{Bindung} PP, S. 14} \index{Bindung}
\sl{ist ein lang anhaltendes, gefühlsmäßiges Band zu einer spezifischen Person,
die nicht ausgetauscht werden kann.}

\newpage
\section{Das Wesen der Erziehung}

\paragraph{\color{coquelicot} {Definition:}} \bf{Intentionale Erziehung}
\footnote{\label{Intensionale} PP, S. 35} \index{Intentionale Erziehung}
\sl{umfaßt die zielgerichteten Handlungen des Erziehers, die absichtsvoll und
geplant durchgeführt werden.}

\printindex

\end{document}
