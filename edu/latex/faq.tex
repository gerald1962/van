% This is the preamble of a LaTeX source code.
\documentclass[10pt,a4paper]{article}

% Import a package for a specific purpose.
\usepackage{color, soul}  % Underlining.
\usepackage{stackengine}  % Stacking of objects, see \cul.
\usepackage{ulem}         % Underline text.
\usepackage{minted}       % Syntax highlighting.
\usepackage{tikz}         % Creating graphics.
\usepackage{tikzsymbols}  % Various emoticon, cooking symbols and trees.
\usepackage{scrextend}    % KOMA-Script class: labeling 

% Initial distance of a paragraphs like enumerate
\setlength{\parskip}{-4pt}%

% All paragraphs are without indentation.
\setlength{\parindent}{0pt}%

% Force the numbering of a paragraph text.
\setcounter{secnumdepth}{4}

% Introduce a paragraph section with a number.
\renewcommand{\theparagraph}{\large \arabic{paragraph}.}

% Default settings for a paragraph segment.
\newcommand\p[1] {\paragraph{\large \hskip -10pt #1} \ \hskip -8pt}

% Subparagraph counters.
\newcounter{subc}
\renewcommand{\thesubc}{\bf\arabic{subc}.}

\newcounter{subsubc}
\renewcommand{\thesubsubc}{\bf\arabic{subsubc}.}

% Settings for a subparagraph segment.
\newcommand\subp[1] {%
  \stepcounter{subc}
  {\bf\theparagraph\thesubc}\hskip 3pt\rm #1\hskip 4pt
}

\newcommand\subsubp[1] {%
  \stepcounter{subsubc}
  {\bf\theparagraph\thesubc\thesubsubc}\hskip 3pt\rm #1\hskip 4pt
}

% Allocate a new length register to save the distance between underlines.
\newlength\lunderset

% munderline - draw multiple underlines.
% #1: variable argument.
% #2: text.
% #3: underline color.
% #4: line width
\newcommand\cul[4][1]{%
  \setbox0 = \hbox{#2}%
  \stackunder[#1 \lunderset - #4]
             {\strut #2}
             {\color{#3} \rule{\wd0} {#4}}%
}

% Elements of a title page.
\newcommand\svthema{Frequently Asked Questions}
\newcommand\svperson{Gerald Schüller}
\newcommand\svdatum{\today}

% Color list:
\definecolor{amber}{rgb}{1.0, 0.75, 0.0}
\definecolor{amber(sae/ece)}{rgb}{1.0, 0.49, 0.0}
\definecolor{americanrose}{rgb}{1.0, 0.01, 0.24}
\definecolor{amethyst}{rgb}{0.6, 0.4, 0.8}
\definecolor{armygreen}{rgb}{0.29, 0.33, 0.13}
\definecolor{english}{rgb}{0.0, 0.5, 0.0}

% Start of a LaTeX text, that is to be printed.
\begin{document}

% Front page
\title{ \textbf{\color{blue}\svthema} \Summertree [1.5] }
\author{ \textsl{\color{red}\svperson} --- \svdatum }
\date{}

% You tell LaTeX the information used to produce the title page.
\maketitle

% \newpage
% Initialize the subparagraph counters.
\setcounter{subc}{0}
\setcounter{subsubc}{0}

% Define the distance between underlines.
\lunderset = 0.75pt\relax

% Start of a new paragraph.
\p{\cul[1] {LaTeX:} {amber}{2pt}}  \LaTeX\ is a typesetting system.

\vskip 7.5pt
\subp {\cul[2] {\cul[1] {Template:} {amber}{2pt}} {amber(sae/ece)}{1.5pt}}
      template is a pattern used as a guide to make something.

\vskip 7.5pt
\subsubp{\cul[2] {%
    \cul[1] {Expose} {amber(sae/ece)}{2pt}} {americanrose}{1.5pt}}
\begin{enumerate}
  \setlength\itemsep{0em}
\item Interest a sponsor for a project: {\color{english} $s := subject$}
  
\item Convince the sponsor, that you are well informed -
  {\color{english} $k := level\ of\ knowledge$}
  - and sketch your ideas: {\color{english} $h := hypothesis$}.
    
\item Describe clearly, how do you want to realize the project -
  {\color{english} $m := method$} - and which result do you strive:
  {\color{english} $t := target$}.

\item Express your ideas:  {\color{english} $p := propositions$}

\item Specify the final delivery -
  {\color{english} $d := deadline$} - and what do you want to show
  regularly - daily, weekly, $\ldots$: {\color{english} $m := milestons$}.
\end{enumerate}

\vskip 7.5pt
tbd.: Diagramm

Prototype:

% Needed for busy beaver Turing machine, see
% https://mirror.dogado.de/tex-archive/graphics/pgf/base/doc/pgfmanual.pd
% p. 102

% Turingmaschine:
% https://de.wikipedia.org/wiki/Turingmaschine
%
% \begin{verbatim}


\vskip 7.5pt

\vskip 7.5pt
\subsubp{\cul[2] {%
    \cul[1] {Used websites} {amber(sae/ece)}{2pt}} {americanrose}{1.5pt}}

\vskip 7.5pt
\begin{verbatim}
https://www.linguee.de/
https://en.wikipedia.org/wiki/LaTeX
https://latexcolor.com/
https://www.techtarget.com/whatis/definition/template
https://en.wikipedia.org/wiki/Help:A_quick_guide_to_templates
https://de.wikipedia.org/wiki/Expos%C3%A9_(Wissenschaft)
https://latex-tutorial.com/underline-latex/
https://www.ctan.org/pkg/pgf
\end{verbatim}

\vskip 7.55pt
\subsubp{\cul[2] {%
    \cul[1] {Example} {amber(sae/ece)}{2pt}} {americanrose}{1.5pt}}

% Start of the syntax highlighting.
\begin{minted}{tex}
  
\vskip 7.5pt
{\bf i. \uuline{\large Field}} \vskip 7.5pt Summary. \vskip 10pt
{\bf i.j.} \underline{Plot} \vskip 7.5pt Introduction. \vskip 10pt
{\bf i.j.1.} \dashuline{Reason} \vskip 7.5pt Motivation. \vskip 10pt
{\bf i.j.2.} \dashuline{Reference} \vskip 7.5pt Inet address. \vskip 10pt
{\bf i.j.3.} \dashuline{Pattern} \vskip 7.5pt TeX/LateX source text. \vskip 10pt
{\bf i.j.4.} \dashuline{Result} \vskip 7.5pt PDF document. \vskip 10pt
{\bf i.j.5.} \dashuline{Backflow} \vskip 7.5pt Field extension.
  
\end{minted}
% The syntax highlighting ends her.

\vskip 5pt
\subsubp{\cul[2] {%
    \cul[1] {Printout} {amber(sae/ece)}{2pt}} {americanrose}{1.5pt}}

\vskip 10pt
{\bf i. \uuline{\large Field}} \vskip 7.5pt Definition. \vskip 10pt
{\bf i.j.} \underline{Plot} \vskip 7.5pt Summary. \vskip 10pt
{\bf i.j.1.} \dashuline{Reason} \vskip 7.5pt Motivation. \vskip 10pt
{\bf i.j.2.} \dashuline{Reference} \vskip 7.5pt Internet address. \vskip 10pt
{\bf i.j.3.} \dashuline{Pattern} \vskip 7.5pt TeX/LateX source text. \vskip 10pt
{\bf i.j.4.} \dashuline{Result} \vskip 7.5pt PDF format. \vskip 10pt
{\bf i.j.5.} \dashuline{Backflow} \vskip 7.5pt Field design.


% \newpage
\vskip 16pt \hrule \vskip 20pt

% Initialize the subparagraph counters.
\setcounter{subsubc}{0}

% Plot: introduction.
\subp {\underline{Multiple colored underlines}} \vskip 7.5pt
We show, how you can highlight text through stacked colored underlines. \vskip 7.5pt

{\color{amethyst} Synopsis:} \vskip 7.5pt

{\color{armygreen} cul} - draw multiple underlines. \vskip 7.5pt

\begin{description}
  \setlength\itemsep{-3pt}
  \item[\#1:] variable argument.
  \item[\#2:] text.
  \item[\#3:] underline color.
  \item[\#4:] line width.
\end{description}
    
% Reason: motivation.
\vskip 2.5pt
\subsubp {\dashuline{Multidimensional text}} \vskip 7.5pt
With colors you are able to design a text multidimensionally. \vskip 10pt

% Reference: Inet address.
\subsubp {\dashuline{Looking up information}} \vskip 7.5pt
Search the Internet.

\begin{labeling}{Search:} 
  \setlength\itemsep{-3pt}
  \item[Tool:]   \verb+https://duckduckgo.com+
  \item[Search:] \verb+latex many coloured underlines+
  \item[Hits:]   \verb+https://tex.stackexchange.com/questions+ \\
    \verb+/169189/highlighting-text-through-stacked-colored-+ \\
    \verb+underlines.+
\end{labeling}

% Pattern: TeX/LateX source text.
\vskip 2.5pt
\subsubp {\dashuline{Implementation}} \vskip 7.5pt
% Start of the syntax highlighting.
\begin{minted}{tex}
\documentclass{article}
\usepackage[dvipsnames]{xcolor}  % Using the color package.
\usepackage{stackengine}         % Stacking of objects.

\newlength\lunderset
\newcommand\cul[4][1]{%
  \setbox0 = \hbox{#2}%
  \stackunder[#1 \lunderset - #4]
             {\strut #2}
             {\color{#3} \rule{\wd0} {#4}}%
}

\newcommand\world[1] {#1}
\newcommand\wcase[1] {#1}
\newcommand\wspace[1] {#1}

\begin{document}
\lunderset = 1.25pt\relax
\cul[1] {\world World: education} {red} {1pt} \vskip 7.5pt
\cul[2]{\cul[1]{\wcase Case: x} {red}{1pt}}{green}{2pt} \vskip 7.5pt
\cul[3]{\cul[2]{\cul[1]{\wspace Space: y}{red}{1pt}}{green}{2pt}}{blue}{4pt}
\vskip 7.5pt
\end{document}
\end{minted}

% Result: PDF document.
\vskip 7.5pt
\subsubp {\dashuline{Simplification and generalization}} \vskip 7.5pt

\newcommand\world[1] {#1}
\newcommand\wcase[1] {#1}
\newcommand\wspace[1] {#1}

\lunderset = 1.25pt\relax
\cul[1] {\world World: Psychologie} {red} {1pt} \vskip 7.5pt
\cul[2]{\cul[1]{\wcase Case: Wahrnehmung} {red}{1pt}}{green}{2pt} \vskip 7.5pt
\cul[3]{\cul[2]{\cul[1]{\wspace Space: Schwitzen}{red}{1pt}}{green}{2pt}}{blue}{4pt}
\vskip 7.5pt

% Backflow: Field extension.
\vskip 7.5pt
\subsubp {\dashuline{Backflow}} \vskip 7.5pt
Decoration of headings.


% \newpage
\vskip 16pt \hrule \vskip 20pt

% Initialize the subparagraph counters.
\setcounter{subsubc}{0}

% Plot: Summary.
\subp {\underline{Plot}} \vskip 7.5pt
Summary. \vskip 10pt

% Reason: Motivation.
\subsubp {\dashuline{Reason}} \vskip 7.5pt
Motivation. \vskip 10pt

% Reference: Internet address.
\subsubp {\dashuline{Reference}} \vskip 7.5pt
Internet address. \vskip 10pt

% Pattern: TeX/LateX source text.
\subsubp {\dashuline{Pattern}} \vskip 7.5pt
TeX/LateX source text. \vskip 10pt

% Result: PDF document.
\subsubp {\dashuline{Result}} \vskip 7.5pt
PDF format. \vskip 10pt

% Backflow: Field design.
\subsubp {\dashuline{Backflow}} \vskip 7.5pt
Field design.

% End of a LaTeX text, that is to be printed.
\end{document}
