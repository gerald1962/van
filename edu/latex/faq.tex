% This is the preamble of a LaTeX source code.
\documentclass[10pt,a4paper]{article}

% Import a package for a specific purpose.
\usepackage{color, soul}  % Underlining
\usepackage{minted}       % Syntax highlighting.

% Force the numbering of a paragraph text.
\setcounter{secnumdepth}{4}

% Introduce a paragraph section with a number.
\renewcommand{\theparagraph}{\arabic{paragraph}.}

% Subparagraph counter.
\newcounter{subParagraphC}
\renewcommand{\thesubParagraphC}{\bf\arabic{subParagraphC}.}

% Begin of a subparagraph
\newcommand\subParagraph[1] {%
  \stepcounter{subParagraphC}
  \bf{\theparagraph}\thesubParagraphC \hskip 6pt \rm #1
}

% Elements of a title page.
\newcommand\svthema{Frequently Asked Questions}
\newcommand\svperson{Gerald Schüller}
\newcommand\svdatum{\today}

% General conditions for a paragraph. 
\newcommand\p[1]{\paragraph\rm{\large\hskip -20pt #1}}

% Start of a LaTeX text, that is to be printed.
\begin{document}

% Front page
\title{ \textbf{\color{blue}\svthema} }
\author{ \textsl{\color{red}\svperson} --- \svdatum }
\date{}

% You tell LaTeX the information used to produce the title page.
\maketitle

% \newpage
% Start of a new paragraph.
\p{{\setulcolor{red} \ul{\LaTeX}}}

% Initialize the subparagraph counter.
\setcounter{subParagraphC}{0}

\vskip 20pt
\subParagraph{Template}

\begin{verbatim}
124.28 Template
---------------
A template is a pattern used as a guide to make something.

124.28.2 Expose
----------------
1. In einem Expose möchte man einen Manger für etwas interessieren (Projekt).
2. Dann spricht man über irgendwas (Wissenstand) und skizziert eine Idee (Hypthesen).
3. Wie das geht (Methode) und was dabei herauskommt (Ziel).
4. Und wie lange etwas dauert (Termin) und was man jeden Tagen zeigen
kann (Plan), hängt vom Vertrauensverhältnis ab.

124.28.1 Reference
..................
https://www.techtarget.com/whatis/definition/template
https://en.wikipedia.org/wiki/Help:A_quick_guide_to_templates
https://de.wikipedia.org/wiki/Expos%C3%A9_(Wissenschaft)

124.28.3 Example
.................
i.j Title
------------
Introduction

i.j.1 Reason
---------------

i.j.2 Reference
..................
Inet address

i.j.3 Example
.................
TeX/LateX source text

i.j.4 Result
.................
LateX output
TeX/LateX source text

124.28.4 Result
.................
LateX output

\end{verbatim}

% Start of the syntax highlighting.
\begin{minted}{tex}
  xxx
  
\end{minted}
% The syntax highlighting ends her.

% End of a LaTeX text, that is to be printed.
\end{document}
