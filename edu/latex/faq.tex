% This is the preamble of a LaTeX source code.
\documentclass[10pt,a4paper]{article}

% Import a package for a specific purpose.
\usepackage{color, soul}  % Underlining
\usepackage{stackengine}  % Stacking of objects, see \munderline.
\usepackage{ulem}         % Underline text.
\usepackage{minted}       % Syntax highlighting.
\usepackage{tikz}         % Creating graphics
\usepackage{tikzsymbols}  % Various emoticon, cooking symbols and trees.

% Initial distance of a paragraphs like enumerate
\setlength{\parskip}{-4pt}%

% All paragraphs are without indentation.
\setlength{\parindent}{0pt}%

% Force the numbering of a paragraph text.
\setcounter{secnumdepth}{4}

% Introduce a paragraph section with a number.
\renewcommand{\theparagraph}{\large \arabic{paragraph}.}

% Default settings for a paragraph segment.
\newcommand\p[1] {\paragraph{\large \hskip -10pt #1} \ \hskip -8pt}

% Subparagraph counters.
\newcounter{subc}
\renewcommand{\thesubc}{\bf\arabic{subc}.}

\newcounter{subsubc}
\renewcommand{\thesubsubc}{\bf\arabic{subsubc}.}

% Settings for a subparagraph segment.
\newcommand\subp[1] {%
  \stepcounter{subc}
  {\bf\theparagraph\thesubc}\hskip 3pt\rm #1\hskip 4pt
}

\newcommand\subsubp[1] {%
  \stepcounter{subsubc}
  {\bf\theparagraph\thesubc\thesubsubc}\hskip 3pt\rm #1\hskip 4pt
}

% Allocate a new length register to save the distance between underlines.
\newlength\lunderset

% munderline - draw multiple underlines.
% #1: variable argument.
% #2: text.
% #3: underline color.
% #4: line width
\newcommand\munderline[4][1]{%
  \setbox0 = \hbox{#2}%
  \stackunder[#1 \lunderset - #4]
             {\strut #2}
             {\color{#3} \rule{\wd0} {#4}}%
}

% Elements of a title page.
\newcommand\svthema{Frequently Asked Questions}
\newcommand\svperson{Gerald Schüller}
\newcommand\svdatum{\today}

% Color list:
\definecolor{amber}{rgb}{1.0, 0.75, 0.0}
\definecolor{amber(sae/ece)}{rgb}{1.0, 0.49, 0.0}
\definecolor{americanrose}{rgb}{1.0, 0.01, 0.24}
\definecolor{english}{rgb}{0.0, 0.5, 0.0}


% Start of a LaTeX text, that is to be printed.
\begin{document}

% Front page
\title{ \textbf{\color{blue}\svthema} \Summertree [1.5] }
\author{ \textsl{\color{red}\svperson} --- \svdatum }
\date{}

% You tell LaTeX the information used to produce the title page.
\maketitle

% \newpage
% Initialize the subparagraph counters.
\setcounter{subc}{0}
\setcounter{subsubc}{0}

% Define the distance between underlines.
\lunderset = 0.75pt\relax

% Start of a new paragraph.
\p{\munderline[1] {LaTeX:} {amber}{2pt}}  \LaTeX\ is a typesetting system.

\vskip 7.5pt
\subp {\munderline[2] {\munderline[1] {Template:} {amber}{2pt}} {amber(sae/ece)}{1.5pt}}
      template is a pattern used as a guide to make something.

\vskip 7.5pt
\subsubp{\munderline[2] {%
    \munderline[1] {Expose} {amber(sae/ece)}{2pt}} {americanrose}{1.5pt}}
\begin{enumerate}
  \setlength\itemsep{0em}
\item Interest a sponsor for a subject: {\color{english} $p := project$}
  
\item Convince the sponsor, that you are well informed -
  {\color{english} $k := level\ of\ knowledge$}
  - and sketch your ideas: {\color{english} $h := hypothesis$}.
    
\item Describe clearly, how do you want to realize the project -
  {\color{english} $m := method$} - and which result do you strive:
  {\color{english} $t := target$}.
    
\item Specify the final result -
  {\color{english} $d := delivery\ date$} - and what do you want to show
  regularly - daily, weekly, $\ldots$: {\color{english} $p := plan$}.
\end{enumerate}

\vskip 7.5pt
tbd.: Diagramm
\vskip 7.5pt

\vskip 7.5pt
\subsubp{\munderline[2] {%
    \munderline[1] {Reference} {amber(sae/ece)}{2pt}} {americanrose}{1.5pt}}

\vskip 7.5pt
\begin{verbatim}
https://www.linguee.de/
https://en.wikipedia.org/wiki/LaTeX
https://latexcolor.com/
https://www.techtarget.com/whatis/definition/template
https://en.wikipedia.org/wiki/Help:A_quick_guide_to_templates
https://de.wikipedia.org/wiki/Expos%C3%A9_(Wissenschaft)
https://latex-tutorial.com/underline-latex/
https://www.ctan.org/pkg/pgf
\end{verbatim}

\vskip 7.55pt
\subsubp{\munderline[2] {%
    \munderline[1] {Pattern} {amber(sae/ece)}{2pt}} {americanrose}{1.5pt}}

% Start of the syntax highlighting.
\begin{minted}{tex}
  
\vskip 7.5pt
{\bf i. \uuline{\large Field}} \vskip 7.5pt 
{\bf i.j.} \underline{Plot} \vskip 7.5pt 
{\bf i.j.1.} \dashuline{Reason} \vskip 7.5pt 
{\bf i.j.2.} \dashuline{Reference} \vskip 7.5pt
{\bf i.j.3.} \dashuline{Pattern} \vskip 7.5pt
{\bf i.j.4.} \dashuline{Result} \vskip 7.5pt
  
\end{minted}
% The syntax highlighting ends her.

\vskip 5pt
\subsubp{\munderline[2] {%
    \munderline[1] {Result} {amber(sae/ece)}{2pt}} {americanrose}{1.5pt}}

\vskip 10pt
{\bf i. \uuline{\large Field}} \vskip 7.5pt
{\bf i.j.} \underline{Plot} \vskip 7.5pt
{\bf i.j.1.} \dashuline{Reason} \vskip 7.5pt
{\bf i.j.2.} \dashuline{Reference} \vskip 7.5pt
{\bf i.j.3.} \dashuline{Pattern} \vskip 7.5pt
{\bf i.j.4.} \dashuline{Result}

% End of a LaTeX text, that is to be printed.
\end{document}
