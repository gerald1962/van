\documentclass[12pt,a4paper]{article}

\usepackage[textwidth=17cm,textheight=26cm]{geometry}

\setlength{\parskip}{-4pt}%
\setlength{\parindent}{0pt}%

\usepackage[ngerman]{babel}	 % Deutsche Namen/Umlaute
\usepackage[utf8]{inputenc}	 % Zeichensatzkodierung
\usepackage[dvipsnames]{xcolor}  % Using the color package
\usepackage{color,soul}          % Underlining
\usepackage{makeidx}
\usepackage{parallel}            % 2003/04/13
\usepackage{stackengine}         % Stacking of objects:
\usepackage{amsmath}		 % AMS-Mathe: roman digits
\usepackage{tikz}                % Creating graphics
\usepackage{tikzsymbols}         % Various emoticon, cooking symbols and trees.
\usepackage{ulem}                % Underlining for empahsis: \coloreduwave

\makeindex

\setcounter{secnumdepth}{4}

\renewcommand{\theparagraph}{\arabic{paragraph}.}
% \renewcommand{\theparagraph}{\arabic{paragraph}.}

\usepackage{chngcntr}
\counterwithout*{paragraph}{section}  %% stop resetting paragraph number with each new subsubsection
% \counterwithin*{paragraph}{section}   %% reset paragraph number for each section; only works with the preceding line!

\definecolor{amethyst}{rgb}{0.6, 0.4, 0.8}
\definecolor{applegreen}{rgb}{0.55, 0.71, 0.0}
\definecolor{aqua}{rgb}{0.0, 1.0, 1.0}
\definecolor{brinkpink}{rgb}{0.98, 0.38, 0.5}
\definecolor{cadmiumorange}{rgb}{0.93, 0.53, 0.18}
\definecolor{capri}{rgb}{0.0, 0.75, 1.0}
\definecolor{carnationpink}{rgb}{1.0, 0.65, 0.79}
\definecolor{cerisepink}{rgb}{0.93, 0.23, 0.51}
\definecolor{coolblack}{rgb}{0.0, 0.18, 0.39}
\definecolor{emerald}{rgb}{0.31, 0.78, 0.47}
\definecolor{english}{rgb}{0.0, 0.5, 0.0}
\definecolor{lemonchiffon}{rgb}{1.0, 0.98, 0.8}
\definecolor{lightcyan}{rgb}{0.88, 1.0, 1.0}

\usepackage{titlesec}
\newcommand\smallsection{%
  \titleformat{\section}
    {\normalfont\normalsize\bfseries}{\thesection}{1em}{}
}

%	HIER WERDEN TITEL REFERENT UND DATUM EINGETRAGEN
%
\newcommand\svthema{Pädagogik und Psychologie}
\newcommand\svperson{Leonie Heber}
\newcommand\svdatum{\today}
\newcommand\lvtyp{Schuljahr 2021/22}
\newcommand\lvinst{Berufsschule für Kinderpflege - Höchstadt a. d. Aisch}

\usepackage{verbatim}

\begin{comment}
:Title: Diagram of the sensory stimulus and reaction or response cycle
:Tags: Diagrams;Flowcharts;Charts;Styles;Computer science
:Author: Gerald Schüller
:Slug: linux

A flow diagram of a sensory stimulus and reaction or response cycle.
It uses basic nodes and arrows and defines node styles.
\end{comment}

\usetikzlibrary{arrows, arrows.meta, decorations.markings}

% for double arrows adapt line thickness and line width, if needed
\tikzstyle{vecArrow} = [
  thick,
  decoration = {
    markings, mark=at position
    1 with {\arrow[semithick]{open triangle 60}}
  },
  double distance =1.4pt,
  shorten >= 5.5pt,
  preaction = {decorate},
  postaction = {
    draw, line width=1.4pt, white,shorten >= 4.5pt}
]

\tikzset{%
  >={Latex[width=2mm,length=2mm]},
  % Specifications for style of nodes:
  base/.style = {
    rectangle,
    draw = black,
    double = black,           %% here
    double distance = 0.5pt,  %% here
    minimum width = 3cm,
    minimum height = 0.5cm,
    text centered,
    font = \sffamily,
    text = violet!80},
  processA/.style = {
    rectangle,
    draw = aqua,
    double = aqua,           %% here
    double distance = 0.75pt,  %% here
    minimum width = 3cm,
    minimum height = 0.5cm,
    text centered,
    font = \sffamily,
    text = black},
  processO/.style = {
    rectangle,
    draw = orange,
    double = orange,           %% here
    double distance = 0.75pt,  %% here
    minimum width = 3cm,
    minimum height = 0.5cm,
    text centered,
    font = \sffamily,
    text = black},
}


\newlength\lunderset
\newlength\rulethick
\lunderset = 1.5pt\relax
\rulethick = .8pt\relax
\def\stackalignment{l}

\newcommand\nunderline[3][1]{%
  \setbox0 = \hbox{#2}%
  \stackunder[#1 \lunderset - \rulethick]
             {\strut #2}
             {\color{#3} \rule{\wd0} {\rulethick}}%
}


% Subparagraph counter.
\newcounter{subParagraphC}
\renewcommand{\thesubParagraphC}{\bf\arabic{subParagraphC}.}

% Begin of a subparagraph
\newcommand\subParagraph[1] {%
  \stepcounter{subParagraphC}
  \bf{\theparagraph}\thesubParagraphC \hskip 6pt \rm #1
}

% Description counter.
\newcommand\desccount{%
  \stepcounter{subParagraphC}\theparagraph\thesubParagraphC
}


\begin{document}

\title{ \textbf{\color{blue}\svthema} }
\author{ \textsl{\color{red}\svperson} --- \svdatum }
\date{ \small {\lvtyp} - {\lvinst} }
\maketitle

\smallsection

% \newpage
\paragraph\rm{\large\hskip -20pt {Unser neues Fach
    \setulcolor{red} \ul{P\"adagogik} und \setulcolor{english} \ul{Psychologie}}}

\begin{Parallel}[c]{0.475\textwidth}{0.475\textwidth}%
  \ParallelLText{\center {\color{english} {Pädagogik}}}%
  \ParallelRText{\center {\color{red} {Psychologie}}}%
  \ParallelPar%
  
  \vskip 4pt
  \ParallelLText{Diese Wissenschaft beschäftigt sich mit der \\
    {\color{red} Erziehung} und {\color{red} Bildung} eines Menschen.}

  \ParallelRText{Diese Wissenschaft beschäft sich mit dem
    {\color{english} Verhältnis} von {\color{english} Erleben}
    (von außen nicht beobachtbar) und {\color{english} Entwicklung} oder
    {\color{english} Verhalten} (von außen beobachtbar) eines Menschen. }
  \ParallelPar%
\end{Parallel}%

% \newpage
\vskip 16pt \hrule \vskip 20pt 
\paragraph\rm{\large\hskip -20pt {\setulcolor{english}\setul{.65ex}{2pt}
    \ul{Das Fach P\"adagogik und Psychologie}}}

% Initialize the subparagraph counter.
\setcounter{subParagraphC}{0}

\vskip 20pt

\fboxrule = 2pt % Sets the width of the frame.
\fboxsep  = 4pt  % Set the separation between the text and the frame.

\subParagraph{Arbeitsauftrag}

\vskip 16pt
\fcolorbox{red}{white}{
  \begin{minipage}{0.95\textwidth}    
    \begin{enumerate}
    \item Lies den Text aufmerksam durch und tausche dich über Unklarheiten
      mit deinem Banknachbarn aus.

    \item Markiert euch wichtige Merkmale zu den Begriffen Pädagogik,
      Psychologie und Alltagspsychologie.

    \item Erklärt euch gegenseitig die Begriffe in eigenen Worten.
    \end{enumerate}
    
  \end{minipage}
}

\vskip 20pt
\subParagraph{\rm{\colorbox{yellow}{Pädagogik}} ist die Bezeichnung für
  die \colorbox{yellow}{wissenschaftliche Disziplin}, die sich mit der Theorie
  und Praxis von \colorbox{yellow}{Bildung und Erziehung} auseinandersetzt. Die
  Pädagogik nimmt dabei ein Doppelrolle ein: sie erforscht Bildungs- und
  Erziehungszusammenhänge (Theorie), macht aber auch Vorschläge, wie Bildungs-
  und Erziehungspraxis\footnote{\fcolorbox{aqua}{white}{
    \begin{minipage}{0.90\textwidth} Mit Erziehungspraxis wird das Handeln in
      konkreten erzieherischen Situationen, das ein bestimmtes Ziel verfolgt,
      bezeichnet.
  \end{minipage}}}
  gestaltet und verbessert werden kann (Praxis).}

\vskip 16pt
\subParagraph{\rm{\colorbox{yellow}{Psychologie}} ist eine weitere
  Wissenschaft, die das \colorbox{yellow}{Erleben und Verhalten} des Menschen,
  seine \colorbox{yellow}{Entwicklung} im Lauf des Lebens und alle dafür
  maßgeblichen inneren und äußeren Ursachen und Bedingungen beschreibt und
  erklärt. Jeder Mensch ist psychologisch tätig, indem er Menschenkenntnis
  besitzt, andere Menschen durchschaut oder Urteile über den Charakter anderer
  fällt. Dieses aufgrund persönlicher Erfahrungen gewonnene Wissen bezeichnen wir
  als Alltagspsychologie im Gegensatz zur wissenschaftlichen Psychologie, die
  sich auf wissenschaftliche Methoden (Experiment, Test, Befragung, $\ldots$)
  stützt. Bei der Alltagspsychologie orientiert man sich vorwiegend an den
  eigenen Erfahrungen und an denen anderer oder auch an Tradition und
  Alltagsweisheiten. }

% \newpage
% \vskip 16pt \hrule \vskip 20pt 
\paragraph\rm{\large\hskip -20pt {\setulcolor{amethyst}\setul{.65ex}{2pt}
    \ul{Was ist Psychologie?}}}

\vskip 20pt

Folgende zwei Fälle können uns bei der Beantwortung dieser Fragen helfen:

\vskip 16pt

\fcolorbox{amethyst}{white}{
  \begin{minipage}{0.95\textwidth}    
    \begin{description}
    \item[\theparagraph 1.] Die Kinder des Kindergartens Sonnenblume sind beim Picknick im Walt.
      Die kleine Maria lässt die Hand der Kinderpflegerin nicht los. Auch als
      Peter kommt, um sich von der Kinderpflegerin beim Öffnen der Flasche
      helfen zu lassen, lässt sie die Hand nicht lost. Die Kinderpflegerin
      fragt: ,,Maria, kannst du bitte kurz loslassen?'' Maria hält die Hand aber
      nur noch fester und ruft: ,,Nein, nein. Nicht loslassen. Ich hab sonst so
      viel Angst.''
      
    \end{description}

    Inwiefern ist Maria auffällig? \\
    Habt ihr eine mögliche Erklärung dafür?
      
  \end{minipage}
}

\begin{description}
\item[\theparagraph 2.] Auszug aus einem Elterngespräch zwischen Frau Schmitt,
  Mutter der sechsjährigen Kerstin und der Kinderpflegerin:
\end {description}

\begin{description}
\item[{\setulcolor{emerald}\setul{.65ex}{2pt} \ul{\rm Frau Schmitt}:}] ,,Und wie geht es
  eigentlich mit Kerstin im Kindergarten?''

\item[{\setulcolor{aqua}\setul{.65ex}{2pt} \ul{\rm Kinderpflegerin}:}] ,,Also, ich kann
  mich überhaupt nicht beschweren, Ihre Tochter ist sehr höflich,
  kommt gut mit den anderen Kindern klar und ist gerade beim Bastelen
  sehr geschickt.''
  
\item[{\setulcolor{emerald}\setul{.65ex}{2pt} \ul{\rm Frau Schmitt}:}] ,,Hm.
  Also, ich frage mal ganz direkt: gibt es irgendwelche Probleme, wenn sie auf
  die Toilette muss?''

\item[{\setulcolor{aqua}\setul{.65ex}{2pt} \ul{\rm Kinderpflegerin (erstaunt)}:}]
  ,,Nein, überhaupt nicht. Weshalb fragen Sie?''

\item[{\setulcolor{emerald}\setul{.65ex}{2pt} \ul{\rm Frau Schmitt}:}] ,,Kerstin
  macht fast jede Nacht ins Bett. Sie weint dann immer und meint: ,, Ich will
  nicht, dass das passiert.'' Der Kinderarzt konnte keinen Grund finden. Ich
  mach mir Sorgen.''
\end {description}

Beschreibe, wie die Kinderpflegerin Kerstin wahrnimmt. Die Mutter hat eine
andere Wahrnehmung. Woran könnte das liegen?


% \newpage
\vskip 16pt \hrule \vskip 20pt
\paragraph\rm{\large\hskip -20pt {\setulcolor{brinkpink}\setul{.65ex}{2pt}
    \ul{Unser Fach P\"adagogik und Psychologie}}}

\vskip 20pt

Kläre bei den folgenden Fällen, ob es sich um {\setulcolor{red} \ul{p\"adagogische}}
oder {\setulcolor{english} \ul{psychologische}} Fallbeispiele handelt und
begründe deine Entscheidung!

% Initialize the subparagraph counter.
\setcounter{subParagraphC}{0}

\begin{description}
\item[\desccount] Frau Schmitt traut sich wegen ihrer Spinnenphobie ($=$ große
  Angst vor Spinnen) nicht mehr in den Keller. \\
  {\color{english} {Psychologisch}}, denn Frau Schmitt könnte ein unliebsames
  Erlebnis gehabt haben mit den Spinnen.

\item[\desccount] Die Kinderpflegerin übt mit den Kindern das Zählen. \\
  {\color{red} {Pädagogisch}}, denn die Kinderpflegerin bildet die Kinder, indem
  sie das Zählen übt.

\item[\desccount] Karsten, 18 Jahre, wacht oft mit schlimmen Alpträumen auf. \\
  {\color{english} {Psychologisch}}, $\ldots$

\item[\desccount] Helga, 16 Jahre, verletzt sich mit Absicht selbst am Arm. \\
  {\color{english} {Psychologisch}}, denn Helga versucht ihr schlechtes Erlebnis
  auf ihre Art zu verdrängen / vergessen.

\item[\desccount] Die Hauswirtschaftslehrerin Frau N erklärt ihren Schülern, wie
  man eine Lasagne zubereitet. \\
  {\color{red} {Pädagogisch}}, denn die Lehrerin bildet die Schüler. Sie zeigt
  ihnen, wie man eine Lasagne zubereitet.

\item[\desccount] Herr Huber zeigt seinem achtjährigen Sohn, wie man einen
  Köpfer vom Beckenrand macht. \\
  {\color{red} {Pädagogisch}}, denn Herr Huber bildet seinen Sohn, indem er
  einen Kopfsprung vom Beckenrand aus vorführt.

\item[\desccount] Gisela, 23 Jahre, traut sich nach einem Autounfall nicht
  mehr selbst ans Steuer. \\
  {\color{english} {Psychologisch}}, denn Gisela hat ein schlechtes Erlebnis
  erlebt, durch das sie traumatisiert ist.

\item[\desccount] Lara, 3 Jahre, kommt vom Kindergarten nach Hause und zählt
  beim Abendessen die neuesten Wörter auf, die sie heute gelernt hat: ,,Du Sau!
  Hau ab!'' \\
  {\color{red} {Pädagogisch}}, denn Lara hat was neues gelernt und wurde somit
  gebildet.
  
\end{description}

% \newpage
\vskip 2pt \hrule \vskip 20pt
\paragraph\rm{\large\hskip -20pt {\setulcolor{amethyst}\setul{.65ex}{2pt}
    \ul{Wahrnehmung - Ich sehe, was du nicht siehst!}}}

% Initialize the subparagraph counter.
\setcounter{subParagraphC}{0}

\begin{description}

\item[\desccount] \nunderline[2]{\nunderline{Definition:}{english}}{red}
  Wahrnehmung ist der Prozess der Reizaufnahme und Reizverarbeitung.

\item[\desccount] Der Prozess der Wahrnehmung:

% Drawing part, node distance is 2 cm and every node
% is prefilled with white background
\begin{tikzpicture}[
    node distance = 1.75cm,
    every node/.style = {fill = white, font = \sffamily}, align = center
  ]
  
\newcommand\laArrow{%
  [-{Latex[open, line width=3mm, width=3mm]}]
}

  % Specification of nodes (position, etc.)
  \node (start)        [base]                      {Reiz};
  \node (senses)       [base, below of = start]    {Sinnesorgan};
  \node (brain)        [base, below of = senses]   {Gehirn};
  \node (experiences)  [base, left of  = brain, xshift=-3.5cm]
        {Bisherige Erfahrungen \\ {\scriptsize {\color{black} {(Schmerz, Gefühle, Bedürfnis)}} }};
  \node (feeling)      [base, below of = brain]    {Empfindung \& \\ Bewertung};
  \node (stop)         [base, below of = feeling]  {Reaktion};

  % Specification of lines between nodes specified above
  % with aditional nodes for description 
  \draw[vecArrow]  (start)        --  (senses);
  \draw\laArrow    (senses)       --  (brain);
  \draw\laArrow    (experiences)  |-  node [yshift=-2cm, text width=3.1cm] { }
                                      (brain);
  \draw\laArrow    (brain)        --  (feeling);
  \draw[vecArrow]  (feeling)      --  (stop);

\end{tikzpicture}

Reize aus der Umwelt oder dem Körperinneren werden von den Sinnesorganen
aufgenommen und von dort über das Nervensystem an das Gehirn weitergeleitet,
in welchem sie verarbeitet werden. Sie werden mit bisherigen Erfahrungen in
Beziehung gesetzt und lösen eine bestimmte Empfindung aus. Diese Empfindung
wird bewertet. Die Empfindung und die bisherige, subjektive Erfahrung führen
letztendlich zu einer Reaktion.

\end{description}

\vskip 4pt
\subParagraph{\setulcolor{amethyst}\setul{.65ex}{2pt}
  \ul{Beispiel Fliege}}

\vskip 16pt
Der Reiz kommt aus der Umwelt, die Fliege auf dem Käsebrot und wird vom
Sinnesorgan Auge aufgenommen. Von dort wird der Reiz über das Nervensystem an
das Gehirn weitergeleitet, in welchem er verarbeitet wird. Ich setze den Reiz
mit meinen bisherigen Erfahrungen in Beziehung, nämlich dass ich solche Fliegen
beim Essen kenne. Dies löst eine Empfindung und Bewertung aus, ich empfinde Ekel
und bewerte die Situation als unangenehm. Auf diese Empfindung reagiere ich,
indem ich die Fliege wegscheuche.


\newpage
\rulethick = 3pt\relax

\paragraph\rm{%
  \large\hskip -20pt {%
    \nunderline[2]{\nunderline{Jeder sieht die Welt mit anderen Augen!}
      {cerisepink}}{aqua}
}}

% Initialize the subparagraph counter.
\setcounter{subParagraphC}{0}

\vskip 16pt
\fcolorbox{aqua}{white}{%
  \begin{minipage}{0.95\textwidth} 
    Unsere Wahrnehmung wird immer von individuellen und sozialen Faktoren sowie
    durch die Beschaffenheit unserer Sinnesorgrane bestimmt. So wird sie teilweise
    stark verzerrt, verändert und verfälscht, was sich in der Wahrnehmung von
    anderen Personen bzw. Gruppen schicksalhaft auswirken kann.
  \end{minipage}
}

\vskip 16pt
\fcolorbox{cerisepink}{white}{
  \begin{minipage}{0.95\textwidth}    
    \begin{description}
    \item[Arbeitsauftrag:] Überlege dir gemeinsam mit deinem Banknachbarn, was
      in den einzelnen Beispielen / Situationen die Wahrnehmung der jeweiligen
      Persion beeinflußt. Notiere diesen Einflußfaktor!
      
    \end{description}
  \end{minipage}
}

\begin{description}

  \newcommand\answer[1]{%
    \begin{tikzpicture}
      \draw[cerisepink, -{Triangle}, line width=1.5pt] (0, 0)--(0.5, 0);
      \node[text width=12cm, left] at (12.75, 0) {\color{cerisepink} {#1}};
    \end{tikzpicture}
  }

\item[\desccount] Du triffst im Kaufhaus einen Menschen, von dem deine Freundin
  sagt, dass er Zahnarzt sei. Er erscheint dir sehr arrogant und unsympathisch. \\
  \answer{Schlechte Erfahrung, Vorurteil}

\item[\desccount] Sara geht mit knurrenden Magen durch den Supermarkt. Fast
  alle Produkte erscheinen ihr sehr lecker. Auch die Fertignudeln, die sie
  sonst nicht mag, wegen ihr Interesse. Dafür vergisst sie den Schreibblock. \\
    \begin{tikzpicture}
      \draw[cerisepink,-{Triangle}, line width=1.5pt] (0, 0)--(0.5, 0);
      \node[text width=6cm, left] at (6.75, 0) {\color{cerisepink} {Hunger}};
      \draw[vecArrow] (2.1, 0)--(2.6, 0);
      \node[text width=6cm, left] at (8.9, 0) {\color{cerisepink} {Bedürfnis}};
    \end{tikzpicture}

  \item[\desccount] Karl hat den Führerschein bestanden und ist darüber sehr
    glückllich. Heute schmeckt ihm sogar der Gemüseeintopf seiner Mutter. \\
    \answer{Gefühle, Stimmung, Freude}

  \item[\desccount] Sandra trifft auf der Straße einen jungen Mann, der mit
    typischen Punk-Outfit auf sie zukommt und sie denkt, er wollte sie um Geld
    anbetteln. \\
    \answer{Vorurteil}

  \item[\desccount] Deine Freundin hat dich überredet, einen Film in
    englischsprachiger Originalfassung im Kino anzusehen. Leider verstehts du
    nur die Hälfte, während sich deine Freundin prächtig amüsiert. \\
    \answer{Interessen, Fähigkeiten}

  \item[\desccount] In Peters Clique rauche alle. Er beginnt nun auch damit,
    weil es anscheindend angesagt ist. Nichtraucher findet er ab jetzt uncool. \\
    \answer{Freunde, Gruppenzwang, Vorstellungen der Gruppe}

  \item[\desccount] Auf einer Hochzeitsfeier trägt ein Gast ein Kleid von Gucci
    im Wert von 900 \texteuro. Dir ist dieses Kleid viel zu teuer und es gefällt
    dir nicht. Die Frau ist Vorstandsvorsitzende bei Bosch und hat eigenen
    Angaben zufolge mit dem Kleid bei Sommersale ein Schnäppchen gemacht. \\
    \answer{Neid, Erwartungen, Einstellungen}

\end{description}


% \newpage
\vskip 4pt \hrule \vskip 20pt
\paragraph\rm{\large\hskip -20pt {\setulcolor{english}\setul{.65ex}{2pt}
    \ul{Die Wahrnehmung wird von verschiedenen Faktoren beeinflusst:}}}

% Initialize the subparagraph counter.
\setcounter{subParagraphC}{0}

\newcommand\rectItem[1]{%
  \vskip 6pt \hskip #1
  \tikz \fill [black] (0.1,0.1) rectangle (0.3,0.3);
}

\vskip 20pt
\subParagraph{\colorbox{lightcyan}{Individuelle Faktoren} \tikz \Strichmaxerl[2]; :}

\rectItem{165pt} Erwartungen
\rectItem{165pt} Erfahrungen
\rectItem{165pt} Befürfnisse, Motive
\rectItem{165pt} Gefühle und Stimmungen
\rectItem{165pt} Wertvorstellungen, Einstellungen
\rectItem{165pt} Vorteile
\rectItem{165pt} Interessen, Fähigkeiten

\vskip 30pt
\subParagraph{%
  \colorbox{lemonchiffon}{Soziale Faktoren}
  \tikz \Strichmaxerl[2]; \hskip 8pt - \tikz \Strichmaxerl[2]; \hskip 8pt :}

\rectItem{160pt} Wert- und Normvorstellungen
\rectItem{160pt} Gesellschaft / Kultur
\rectItem{160pt} Personengruppen (z.B. Clique)
\rectItem{160pt} Andere Einzelpersonen


% \newpage
\vskip 20pt \hrule \vskip 20pt

\lunderset = 2pt\relax
\rulethick = 3pt\relax

\paragraph\rm{%
  \large\hskip -20pt {%
    \nunderline[2]{\nunderline{Die 7 Sinne des Menschen}
      {magenta}}{red}
}}

% Initialize the subparagraph counter.
\setcounter{subParagraphC}{0}

\vskip 16pt
\begin{description}

\item[\rm Tastsinn]

\item[\rm Geschmacksinn]

\item[\rm Gleichgewichtssinn]
  
\item[\rm Lage - und Bewegungssinn]
  
\end{description}

\vskip 4pt
\subParagraph{%
  \lunderset = 1pt\relax
  \rulethick = 1pt\relax
  \nunderline[2]{\nunderline{Merke:} {red}}{red} \ {\color {magenta} {Nahsinne}}
}

\vskip 16pt \hskip 77.5 pt
Mit Nahsinne nehmen wir körpereigene, nahe Reize auf.

\vskip 8pt \hskip 80pt
\tikz {%
  \draw [line width=1pt, -] (0, 0) -- (0, 0.25);
  \draw[-{Triangle[open]}, line width=1pt] (0, 0)--(0.5, 0); }
$=$ Signale aus der Umwelt

\vskip 8 pt\hskip 77.5 pt
z.B. Geschmack des Kuchens

\vskip 16pt
\begin{description}

\item[\rm Sehsinn]

\item[\rm Hörsinn]

\item[\rm Geruchsinn]
  
\end{description}

\vskip 4pt
\subParagraph{%
  \lunderset = 1pt\relax
  \rulethick = 1pt\relax
  \nunderline[2]{\nunderline{Merke:} {red}}{red} \ {\color {magenta} {Fernsinne}}
}

\vskip 16pt \hskip 77.5 pt
\begin{minipage}{0.75\textwidth}  
  Mit Fersinne können wir Reize aus der Umwelt aufnehmen, die weiter entfernt
  sind, ohne dass man direkten Kontakt mit ihnen hat.
\end{minipage}


% \newpage
\vskip 20pt \hrule \vskip 20pt

\lunderset = 1pt\relax
\rulethick = 3pt\relax

\paragraph\rm{%
  \large\hskip -20pt {%
    \nunderline[2]{\nunderline{Bestandteile der Gruppenarbeit}
      {magenta}}{capri}
}}

% Roman digits.
\newcommand{\rom}[1]{$\underline{\overline{\text{#1}}}$}

% Initialize the subparagraph counter.
\setcounter{subParagraphC}{0}

\vskip 20 pt
Bei Gruppenarbeiten gibt es wichtige Bestandteile, die zu Vorbereitung,
Durführung und Nachbearbeitung gehören. Auch im Unterrichtung führen wir
Gruppenarbeiten durch.

\vskip 16pt
\rom{I}. {\setulcolor{yellow} \setul{.65ex}{2pt} \ul{Vorbereitung}}

\vskip 16pt
\subParagraph{%
  \lunderset = 1pt\relax
  \rulethick = 1pt\relax
  \nunderline[2]{\nunderline{Arbeitsplatz:} {magenta}}{magenta}
}

\vskip 16pt \hskip 90pt
\begin{minipage}{0.75\textwidth}
  \begin{enumerate}
    \setlength\itemsep{0em}

  \item Tisch und Sitzordnung stellen (genügend Platz)
  \item Auf guten Lichteinfall achten
  \item Auf guten Blickkontakt achten
  \item Arbeitsmaterial bereit halten / legen
    
  \end{enumerate}
\end{minipage}

\vskip 16pt
\subParagraph{%
  \lunderset = 1pt\relax
  \rulethick = 1pt\relax
  \nunderline[2]{\nunderline{Arbeitsplanung:} {magenta}}{magenta}
}

\vskip 16pt \hskip 90pt
\begin{minipage}{0.75\textwidth}
  \begin{enumerate}

  \item Brainstorming
    \setlength\itemsep{0em}
    
    \tikz {%
      \draw [line width=1pt, -] (0, 0) -- (0, 0.25);
      \draw[-{Triangle[open]}, line width=1pt] (0, 0)--(0.5, 0); }
    Thema genau erfassen

  \item Zügig arbeiten

    \item Klärung der Vorgehensweise und der Arbeitsvorgaben
    
  \end{enumerate}
\end{minipage}

\vskip 16pt
\subParagraph{%
  \lunderset = 1pt\relax
  \rulethick = 1pt\relax
  \nunderline[2]{\nunderline{Gruppenregeln:} {magenta}}{magenta}
}

\vskip 16pt \hskip 90pt
\begin{minipage}{0.75\textwidth}
  \begin{itemize}
    \setlength\itemsep{0em}

  \item Jeder ist gleichberechtigt und kann seine Meinung äußern
  \item Kritik konstruktiv äußern
  \item Höflicher Umgangston
  \item Jeder arbeitet mit
  \item Fair bleiben
  \item Nicht vom Thema abweichen
  \item Gruppenrollen einhalten
  \item Ziel verfolgen
    
  \end{itemize}
\end{minipage}

\vskip 16pt
\lunderset = 1pt\relax
\rulethick = 3pt\relax
\rom{II}. \nunderline[2]{\nunderline{Durchf\"uhrung}
  {yellow}}{brown}


% \newpage
\vskip 20pt \hrule \vskip 20pt

\lunderset = 1pt\relax
\rulethick = 3pt\relax

\paragraph\rm{%
  \large\hskip -20pt {%
    \nunderline[2]{\nunderline{Respektvoller Umgang}
      {emerald}}{capri}
}}

% Initialize the subparagraph counter.
\setcounter{subParagraphC}{0}

\vskip 16pt
\subParagraph{%
  \lunderset = 1pt\relax
  \rulethick = 1pt\relax
  \nunderline[2]{\nunderline{sagen} {magenta}}{magenta}
} \tikz {%
  \draw [line width=1pt, -to] (0,0) -- (0.5,0);
  
}

\vskip -10pt \hskip 75pt
\begin{minipage}{0.75\textwidth}
  \begin{itemize}

  \item Höflichkeit
    
  \end{itemize}
\end{minipage}

\vskip 16pt
\subParagraph{%
  \lunderset = 1pt\relax
  \rulethick = 1pt\relax
  \nunderline[2]{\nunderline{tun} {magenta}}{magenta}
} \tikz {%
  \draw [line width=1pt, -to] (0,0) -- (0.5,0);
  
}

\newcommand{\skipitems}[1]{%
  \addtocounter{\@enumctr}{#1}%
}
\makeatother

\vskip -10pt \hskip 70pt
\begin{minipage}{0.75\textwidth}
  \begin{itemize}
    \setlength\itemsep{0em}
    
  \item Konflikt suchen und ansprechen
  \item Begrüßen
  \item Anlächeln
  \item Bei Problemen helfen
  \item Jeden akzeptieren
  \item Erst überlegen, dann reden / handeln
  \item Sichtweise überdenken
  \item Bei Problemen miteinander reden
    
  \end{itemize}
\end{minipage}

% \vskip 20pt
\newpage

\lunderset = 1pt\relax
\rulethick = 2pt\relax

\subParagraph\rm{%
  \hskip -2pt {%
    \nunderline[2]{\nunderline{Empathie $=$ sich in den anderen hineinversetzen}
      {yellow}}{magenta}
}}

\vskip 20pt
\hskip -20pt
\begin{minipage}{0.75\textwidth}

  \begin{itemize}
    \setlength\itemsep{0em}
    
  \item Man muss sich in den anderen heinversetzen.
  \item Seine Sichtweise ändern oder eine andere Sichtweise einnehmen.
  \item Sich Vorschläge, Meinungen und Tipps anhören.
  \item Eine gemeinsame Lösung finden.
  \item Evtl. eine dritte Persion dazu holen.
    
  \end{itemize}
  
\end{minipage}


% \newpage
\vskip 16pt \hrule \vskip 20pt

\lunderset = 1pt\relax
\rulethick = 3pt\relax

\paragraph\rm{%
  \large\hskip -20pt {%
    \nunderline[2]{\nunderline{Respektvolle Umgangsformen}
      {yellow}}{magenta}
}}

% Initialize the subparagraph counter.
\setcounter{subParagraphC}{0}

\vskip 20pt
\hskip -20pt
\begin{minipage}{0.75\textwidth}
  \begin{itemize}
    \setlength\itemsep{0em}
    
  \item Freundlich grüßen und anlächeln
  \item Bitte und danke sage \\
    ,,Darf ich bitte $\ldots$'', ,,Könntest du bitte $\ldots$''
  \item Niemanden auslachen
    
  \end{itemize}
  
\end{minipage}

\vskip 20pt
\subParagraph{\setulcolor{yellow}\setul{.65ex}{2pt}
  \ul{In Konfliktsituationen:}}

\vskip 12pt
\begin{description}
  \setlength\itemsep{0em}
  
  \item[-] Die eigene Sichtweise verändern und versuchen, sich in den anderen
    hineinzuversetzen
    
  \item[-] Ehrlich und fair miteinander reden, ohne zu verletzen

  \item[-] Argumente und Meinungen austauschen

  \item[-] Kompromisse finden

  \item[-] Evtl. eine dritte Person hinzuholen
    
\end{description}


% \newpage
\vskip 12pt \hrule \vskip 20pt 
\paragraph\rm{\large\hskip -20pt {\setulcolor{magenta}\setul{.65ex}{2pt}
    \ul{Methodische Prinzipien}}}

\vskip 16pt
\begin{description}
  \setlength\itemsep{0em}
  
\item[-] Lebensnähe
\item[-] Kindgemäßheit
\item[-] Individualisierung
\item[-] Anschaulichkeit
\item[-] Teilschritte
\item[-] Handlungsorientierung
\item[-] Freiwilligkeit

\end{description}


% \newpage
\vskip 12pt \hrule \vskip 20pt

\lunderset = 1.5pt\relax
\rulethick = 2pt\relax

\paragraph\rm{%
  \large\hskip -20pt {%
    \nunderline[2]{\nunderline{{\color{magenta} Ein Steckbrief $\ldots$}}
      {yellow}}{coolblack}
}}

% Initialize the subparagraph counter.
\setcounter{subParagraphC}{0}

\vskip 20pt
$\ldots$ dient dazu sich selbst anderen Personen mit Hilfe con Stichpunkten
vorzustellen. Man sollte also keine vollständigen Sätze verwenden.

\newpage
% \vskip 20pt
\lunderset = 1.5pt\relax
\rulethick = 2pt\relax
\subParagraph{%
  \large {%
    \nunderline[2]{\nunderline{{\color{magenta} Wichtige Angaben}}
      {yellow}}{coolblack}
}}
\normalsize
  
\vskip 16pt
\begin{description}
  \setlength\itemsep{0em}
  
\item[-] Vollständiger Name
\item[-] Geburtstag / Alter / evtl. Geschwister, Kinder, $\ldots$
\item[-] Wohnort
\item[-] Besuchte Schule (Berufsfachschule für Kinderpflege in Höchstadt) \\
  $+$ Warum ihr in die Einrichtung kommt
\item[-] Hobbys / Besonderheiten / Lieblingstier $\ldots$
\item[-] ,,Warum gefält mir der Beruf?''
  
\end{description}

\vskip 20pt
\lunderset = 1.5pt\relax
\rulethick = 2pt\relax
\subParagraph{%
  \large {%
    \nunderline[2]{\nunderline{{\color{magenta} Tipps für die Gestaltung}}
      {yellow}}{coolblack}
}}
\normalsize

% Through this transition, there is a puzzling LaTex warning:
% pdfTeX warning: pdflatex: pop empty color page stack 0

\vskip 16pt
\begin{description}
  \setlength\itemsep{0em}
  
\item[-] Foto (Gesicht gut erkennbar)
\item[-] Bezug zum Gruppennamen: Motiv / Farbgestaltung
\item[-] Farbenfroh / kinderfreundlich
\item[-] Übersichtlich
\item[-] Mit Bildern und Motiven arbeiten \\
  \tikz {%
    \draw[-{Triangle[open]}, line width=1pt] (0, 0)--(0.5, 0);
  }
  Entwurf auf Rechtschreibung, Grammatik von einer anderen Person
  kontrollieren lassen.
  
\end{description}


% \newpage
\newcommand{\mydash}[1]{%
  \tikz[baseline=(dashed.base)]{
    \node[inner sep=1pt, outer sep=0pt] (dashed) {#1};
    \draw[dashed, line width=2pt, color=magenta]
    ([yshift=-2pt]dashed.south west) -- ([yshift=-2pt]dashed.south east);
    }%
}%

\vskip 12pt \hrule \vskip 20pt 
\paragraph\rm{\large\hskip -20pt {\setulcolor{carnationpink}\setul{.65ex}{4pt}
    \ul{Lerninhalte f\"ur die 1. Schulaufgabe}}}

% Initialize the subparagraph counter.
\setcounter{subParagraphC}{0}

\vskip 20pt
\subParagraph{\mydash{- Begriffe Pädagogik und Psychologie}}

\vskip 20pt
\subParagraph{\mydash{- Wahrnehmung und Beobachtung}}

\vskip 15pt
\hskip -20pt
\begin{minipage}{0.75\textwidth}
  \begin{itemize}
    \setlength\itemsep{0em}
    
  \item Unser Sinnessystem
  \item Wahrnehmung: Begriff und Prozess
  \item Faktoren, welche die Wahrnehmung beeinflussen
  \item Abgrenzung Wahrnehmung von Beobachtung
  \item Begriff: subjektive und selektive Wahrnehmung
  \item Bedeutung von Beobachtung
  \item Beobachtungsfehler
    
  \end{itemize}
  
\end{minipage}

\vskip 20pt
\subParagraph{\mydash{- Erziehung}}

\vskip 15pt
\hskip -20pt
\begin{minipage}{0.75\textwidth}
  \begin{itemize}
    \setlength\itemsep{0em}
    
  \item Begriff Erziehung
  \item Erziehungsprozess
  \item intensional und funktionale Erziehung
  \item Anlange, Umwelt und Selbststeuerung
    
  \end{itemize}
  
\end{minipage}

\vskip 20pt
{\large Viel Erfolg und Vergnügen beim Vorbereiten} \dSmiley [2][yellow]


% \newpage
\newcommand{\mywave}[2]{%
  \tikz[baseline=(decorate.base)]{
    \node[inner sep=1pt, outer sep=0pt] (decorate) {#2};
    \draw[decorate, decoration={snake}, color=#1, line width=1.5pt]
    ([yshift=-2pt] decorate.south west) -- ([yshift=-2pt] decorate.south east);
    }%
}%

\vskip 12pt \hrule \vskip 20pt 
\paragraph\rm{\large\hskip -20pt {\mywave{orange}{Der Erziehungsprozess}}}

\vskip 20pt 
\begin{tikzpicture}[
    node distance = 2.5cm,
    every node/.style = {fill = white, font = \sffamily}, align = center
  ]
  
\newcommand\LatexA{%
  [-{Latex}, line width=1.25pt]
}
\newcommand\latexlatexA{%
  [latex-latex, line width=1.25pt]
}

% Specification of nodes (position, etc.)
\node (start)        [processA]
      {Erziehungsziel \\
        {\scriptsize {\color{red} {Hygiene/Gesundheit}}} \\
        {\scriptsize {\color{emerald} {Selbstst\"andigkeit}}}
      };
      
                    
\node (procedure)    [processO, right of=start, xshift=2.5cm]
      {Erziehungsmaßnahme \\
        {\scriptsize {\color{red} {Drohung}}} \\
        {\scriptsize {\color{emerald} {Marion beruhigt und ermutigt}}}
      };

\node (change)        [processA, below of=procedure]
      {dauerhafte Verhaltensänderung \\
        {\scriptsize {\color{red} {Händewaschen vor dem Mittagessen}}} \\
        {\scriptsize {\color{emerald} {Paul kann selbständig seine Jacke schließen}}}
      };

\node (teacher)  [processO, left of=change, xshift=-5cm]
      {Erzieher \\
        {\scriptsize {\color{red} {Mutter}}} \\
        {\scriptsize {\color{emerald} {Kinderpflegerin Marion}}}
      };
      
\node (pupil)        [processA, below of=change]
      {Zuerziehender \\
        {\scriptsize {\color{red} {Martin}}} \\
        {\scriptsize {\color{emerald} {Paul}}}
      };

\node (basic)        [processA, below of=pupil]
      {Rahmenbedingung \\
        {\scriptsize {\color{red} {Beeinflusung durch Freund Jens}}}
      };

  % Specification of lines between nodes specified above
  % with aditional nodes for description
  \draw\LatexA       (start)    |-  (change);
  \draw\LatexA       (teacher)  |-  (start);
  \draw\LatexA       (pupil)    --  (change);
  \draw\latexlatexA  (pupil)    -|  (teacher);
  \draw\LatexA       (basic)    --  (pupil);

\end{tikzpicture}

\vskip 20pt

\fboxrule = 4pt % Sets the width of the frame.
\fboxsep  = 4pt  % Set the separation between the text and the frame.

\vskip 20pt
\fcolorbox{emerald}{white}{
  \begin{minipage}{0.95\textwidth}    
    \begin{Parallel}{0.475\textwidth}{0.75\textwidth}%

      \ParallelLText{%
        {\setulcolor{red} \setul{.65ex}{2pt} \ul{Definition Erziehung:\ }}
      }

      \ParallelRText{%
        Erziehung ist das zielgerichtete und beabsichtigte Einwirken des
        Erziehers anf den Zuerziehenden, um eine dauerhafte Verhaltensänderung
        herbeizuführen.
      }
      
      \ParallelPar%
    \end{Parallel}%
  \end{minipage}
}


% \newpage
\vskip 16pt \hrule \vskip 20pt

\lunderset = 1pt\relax
\rulethick = 2pt\relax

\paragraph\rm{%
  \large\hskip -20pt {%
    \nunderline[2]{\nunderline{Beobachtungsfehler}
      {yellow}}{magenta}
}}

\printindex

\end{document}
