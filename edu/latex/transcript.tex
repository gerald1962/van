\documentclass[12pt,a4paper]{article}

\usepackage[textwidth=17cm,textheight=26cm]{geometry}

\setlength{\parskip}{-4pt}%
\setlength{\parindent}{0pt}%

\usepackage[ngerman]{babel}		 % Deutsche Namen/Umlaute
\usepackage[utf8]{inputenc}		 % Zeichensatzkodierung
\usepackage[dvipsnames]{xcolor}          % Using the color package
\usepackage{color,soul}                  % Underlining
\usepackage{makeidx}
\usepackage{parallel}                    % 2003/04/13
\usepackage{stackengine}                 % Stacking of objects: 
\makeindex

\setcounter{secnumdepth}{4}

\renewcommand{\theparagraph}{\arabic{paragraph}.}
% \renewcommand{\theparagraph}{\arabic{paragraph}.}

\usepackage{chngcntr}
\counterwithout*{paragraph}{section}  %% stop resetting paragraph number with each new subsubsection
% \counterwithin*{paragraph}{section}   %% reset paragraph number for each section; only works with the preceding line!

\definecolor{amethyst}{rgb}{0.6, 0.4, 0.8}
\definecolor{applegreen}{rgb}{0.55, 0.71, 0.0}
\definecolor{aqua}{rgb}{0.0, 1.0, 1.0}
\definecolor{brinkpink}{rgb}{0.98, 0.38, 0.5}
\definecolor{coolblack}{rgb}{0.0, 0.18, 0.39}
\definecolor{emerald}{rgb}{0.31, 0.78, 0.47}
\definecolor{english}{rgb}{0.0, 0.5, 0.0}

\usepackage{titlesec}
\newcommand\smallsection{%
  \titleformat{\section}
    {\normalfont\normalsize\bfseries}{\thesection}{1em}{}
}

%	HIER WERDEN TITEL REFERENT UND DATUM EINGETRAGEN
%
\newcommand\svthema{Pädagogik und Psychologie}
\newcommand\svperson{Leonie Heber}
\newcommand\svdatum{\today}
\newcommand\lvtyp{Schuljahr 2021/22}
\newcommand\lvinst{Berufsschule für Kinderpflege - Höchstadt a. d. Aisch}

\usepackage{verbatim}

\begin{comment}
:Title: Diagram of the sensory stimulus and reaction or response cycle
:Tags: Diagrams;Flowcharts;Charts;Styles;Computer science
:Author: Gerald Schüller
:Slug: linux

A flow diagram of a sensory stimulus and reaction or response cycle.
It uses basic nodes and arrows and defines node styles.
\end{comment}

\usepackage{tikz}
\usetikzlibrary{arrows, arrows.meta, decorations.markings}

% for double arrows adapt line thickness and line width, if needed
\tikzstyle{vecArrow} = [
  thick,
  decoration = {
    markings, mark=at position
    1 with {\arrow[semithick]{open triangle 60}}
  },
  double distance =1.4pt,
  shorten >= 5.5pt,
  preaction = {decorate},
  postaction = {
    draw, line width=1.4pt, white,shorten >= 4.5pt}
]

\tikzset{%
  >={Latex[width=2mm,length=2mm]},
  % Specifications for style of nodes:
  base/.style = {
    rectangle,
    draw = black,
    double = black,           %% here
    double distance = 0.5pt,  %% here
    minimum width = 3cm,
    minimum height = 0.5cm,
    text centered,
    font = \sffamily,
    text = violet!80},
}

\begin{document}

\title{ \textbf{\color{blue}\svthema} }
\author{ \textsl{\color{red}\svperson} --- \svdatum }
\date{ \small {\lvtyp} - {\lvinst} }
\maketitle

\smallsection

% \newpage
\paragraph\rm{\large\hskip -20pt {Unser neues Fach
    \setulcolor{red} \ul{P\"adagogik} und \setulcolor{english} \ul{Psychologie}}}

\begin{Parallel}[c]{0.475\textwidth}{0.475\textwidth}%
  \ParallelLText{\center {\color{english} {Pädagogik}}}%
  \ParallelRText{\center {\color{red} {Psychologie}}}%
  \ParallelPar%
  
  \vskip 4pt
  \ParallelLText{Diese Wissenschaft beschäftigt sich mit der \\
    {\color{red} Erziehung} und {\color{red} Bildung} eines Menschen.}

  \ParallelRText{Diese Wissenschaft beschäft sich mit dem
    {\color{english} Verhältnis} von {\color{english} Erleben}
    (von außen nicht beobachtbar) und {\color{english} Entwicklung} oder
    {\color{english} Verhalten} (von außen beobachtbar) eines Menschen. }
  \ParallelPar%
\end{Parallel}%

% \newpage
\vskip 16pt \hrule \vskip 20pt 
\paragraph\rm{\large\hskip -20pt {\setulcolor{english}\setul{.65ex}{2pt}
    \ul{Das Fach P\"adagogik und Psychologie}}}

\vskip 20pt

\fboxrule = 2pt % Sets the width of the frame.
\fboxsep  = 4pt  % Set the separation between the text and the frame.

\fcolorbox{red}{white}{
  \begin{minipage}{0.95\textwidth}    
    Arbeitsauftrag:
    \begin{enumerate}
    \item Lies den Text aufmerksam durch und tausche dich über Unklarheiten
      mit deinem Banknachbarn aus.

    \item Markiert euch wichtige Merkmale zu den Begriffen Pädagogik,
      Psychologie und Alltagspsychologie.

    \item Erklärt euch gegenseitig die Begriffe in eigenen Worten.
    \end{enumerate}
    
  \end{minipage}
}

\vskip 20pt
\paragraph\rm{\hskip -20pt \colorbox{yellow}{Pädagogik} ist die Bezeichnung für
  die \colorbox{yellow}{wissenschaftliche Disziplin}, die sich mit der Theorie
  und Praxis von \colorbox{yellow}{Bildung und Erziehung} auseinandersetzt. Die
  Pädagogik nimmt dabei ein Doppelrolle ein: sie erforscht Bildungs- und
  Erziehungszusammenhänge (Theorie), macht aber auch Vorschläge, wie Bildungs-
  und Erziehungspraxis\footnote{\fcolorbox{aqua}{white}{
    \begin{minipage}{0.90\textwidth} Mit Erziehungspraxis wird das Handeln in
      konkreten erzieherischen Situationen, das ein bestimmtes Ziel verfolgt,
      bezeichnet.
  \end{minipage}}}
  gestaltet und verbessert werden kann (Praxis).}

\vskip 16pt
\paragraph\rm{\hskip -20pt \colorbox{yellow}{Psychologie} ist eine weitere
  Wissenschaft, die das \colorbox{yellow}{Erleben und Verhalten} des Menschen,
  seine \colorbox{yellow}{Entwicklung} im Lauf des Lebens und alle dafür
  maßgeblichen inneren und äußeren Ursachen und Bedingungen beschreibt und
  erklärt. Jeder Mensch ist psychologisch tätig, indem er Menschenkenntnis
  besitzt, andere Menschen durchschaut oder Urteile über den Charakter anderer
  fällt. Dieses aufgrund persönlicher Erfahrungen gewonnene Wissen bezeichnen wir
  als Alltagspsychologie im Gegensatz zur wissenschaftlichen Psychologie, die
  sich auf wissenschaftliche Methoden (Experiment, Test, Befragung, $\ldots$)
  stützt. Bei der Alltagspsychologie orientiert man sich vorwiegend an den
  eigenen Erfahrungen und an denen anderer oder auch an Tradition und
  Alltagsweisheiten. }

% \vskip 16pt \hrule \vskip 20pt 
\paragraph\rm{\large\hskip -20pt {\setulcolor{amethyst}\setul{.65ex}{2pt}
    \ul{Was ist Psychologie?}}}

\vskip 20pt

Folgende zwei Fälle können uns bei der Beantwortung dieser Fragen helfen:

\vskip 16pt

\fcolorbox{amethyst}{white}{
  \begin{minipage}{0.95\textwidth}    
    \begin{description}
    \item[\theparagraph 1.] Die Kinder des Kindergartens Sonnenblume sind beim Picknick im Walt.
      Die kleine Maria lässt die Hand der Kinderpflegerin nicht los. Auch als
      Peter kommt, um sich von der Kinderpflegerin beim Öffnen der Flasche
      helfen zu lassen, lässt sie die Hand nicht lost. Die Kinderpflegerin
      fragt: ,,Maria, kannst du bitte kurz loslassen?'' Maria hält die Hand aber
      nur noch fester und ruft: ,,Nein, nein. Nicht loslassen. Ich hab sonst so
      viel Angst.''
      
    \end{description}

    Inwiefern ist Maria auffällig? \\
    Habt ihr eine mögliche Erklärung dafür?
      
  \end{minipage}
}

\begin{description}
\item[\theparagraph 2.] Auszug aus einem Elterngespräch zwischen Frau Schmitt,
  Mutter der sechsjährigen Kerstin und der Kinderpflegerin:
\end {description}

\begin{description}
\item[{\setulcolor{emerald}\setul{.65ex}{2pt} \ul{\rm Frau Schmitt}:}] ,,Und wie geht es
  eigentlich mit Kerstin im Kindergarten?''

\item[{\setulcolor{aqua}\setul{.65ex}{2pt} \ul{\rm Kinderpflegerin}:}] ,,Also, ich kann
  mich überhaupt nicht beschweren, Ihre Tochter ist sehr höflich,
  kommt gut mit den anderen Kindern klar und ist gerade beim Bastelen
  sehr geschickt.''
  
\item[{\setulcolor{emerald}\setul{.65ex}{2pt} \ul{\rm Frau Schmitt}:}] ,,Hm.
  Also, ich frage mal ganz direkt: gibt es irgendwelche Probleme, wenn sie auf
  die Toilette muss?''

\item[{\setulcolor{aqua}\setul{.65ex}{2pt} \ul{\rm Kinderpflegerin (erstaunt)}:}]
  ,,Nein, überhaupt nicht. Weshalb fragen Sie?''

\item[{\setulcolor{emerald}\setul{.65ex}{2pt} \ul{\rm Frau Schmitt}:}] ,,Kerstin
  macht fast jede Nacht ins Bett. Sie weint dann immer und meint: ,, Ich will
  nicht, dass das passiert.'' Der Kinderarzt konnte keinen Grund finden. Ich
  mach mir Sorgen.''
\end {description}

Beschreibe, wie die Kinderpflegerin Kerstin wahrnimmt. Die Mutter hat eine
andere Wahrnehmung. Woran könnte das liegen?

% \newpage
\vskip 16pt \hrule \vskip 20pt
\paragraph\rm{\large\hskip -20pt {\setulcolor{brinkpink}\setul{.65ex}{2pt}
    \ul{Unser Fach P\"adagogik und Psychologie}}}

\vskip 20pt

xxx

\vskip 16pt \hrule \vskip 20pt
\paragraph\rm{\large\hskip -20pt {\setulcolor{amethyst}\setul{.65ex}{2pt}
    \ul{Wahrnehmung - Ich sehe, was du nicht siehst!}}}

\begin{description}

\newlength\lunderset
\newlength\rulethick
\lunderset = 1.5pt\relax
\rulethick = .8pt\relax
\def\stackalignment{l}

\newcommand\nunderline[3][1]{%
  \setbox0 = \hbox{#2}%
  \stackunder[#1 \lunderset - \rulethick]
             {\strut #2}
             {\color{#3} \rule{\wd0} {\rulethick}}%
}

\item[\theparagraph 1.] \nunderline[2]{\nunderline{Definition:}{english}}{red}
  Wahrnehmung ist der Prozess der Reizaufnahme und Reizverarbeitung.

\item[\theparagraph 2.] Der Prozess der Wahrnehmung:

% Drawing part, node distance is 2 cm and every node
% is prefilled with white background
\begin{tikzpicture}[
    node distance = 1.75cm,
    every node/.style = {fill = white, font = \sffamily}, align = center
  ]
  
\newcommand\laArrow{%
  [-{Latex[open, line width=3mm, width=3mm]}]
}

  % Specification of nodes (position, etc.)
  \node (start)        [base]                      {Reiz};
  \node (senses)       [base, below of = start]    {Sinnesorgan};
  \node (brain)        [base, below of = senses]   {Gehirn};
  \node (experiences)  [base, left of  = brain, xshift=-3.5cm]
        {Bisherige Erfahrungen \\ {\scriptsize {\color{black} {(Schmerz, Gefühle, Bedürfnis)}} }};
  \node (feeling)      [base, below of = brain]    {Empfindung \& \\ Bewertung};
  \node (stop)         [base, below of = feeling]  {Reaktion};

  % Specification of lines between nodes specified above
  % with aditional nodes for description 
  \draw[vecArrow]  (start)        --  (senses);
  \draw\laArrow    (senses)       --  (brain);
  \draw\laArrow    (experiences)  |-  node [yshift=-2cm, text width=3.1cm] { }
                                      (brain);
  \draw\laArrow    (brain)        --  (feeling);
  \draw[vecArrow]  (feeling)      --  (stop);

\end{tikzpicture}

\end{description}

\printindex

\end{document}
