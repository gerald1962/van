\documentclass[12pt,a4paper]{article}

\usepackage[textwidth=17cm,textheight=26cm]{geometry}

\setlength{\parskip}{-4pt}%
\setlength{\parindent}{0pt}%

\usepackage[ngerman]{babel}		 % Deutsche Namen/Umlaute
\usepackage[utf8]{inputenc}		 % Zeichensatzkodierung
\usepackage[usenames,dvipsnames]{color}  % Using the color package, not xcolor
\usepackage{makeidx}
\makeindex

\setcounter{secnumdepth}{4}

\renewcommand{\theparagraph}{\arabic{section}.\arabic{paragraph}.}
% \renewcommand{\theparagraph}{\arabic{paragraph}.}

\usepackage{chngcntr}
\counterwithout*{paragraph}{section}  %% stop resetting paragraph number with each new subsubsection
% \counterwithin*{paragraph}{section}   %% reset paragraph number for each section; only works with the preceding line!

\definecolor{amethyst}{rgb}{0.6, 0.4, 0.8}
\definecolor{applegreen}{rgb}{0.55, 0.71, 0.0}
\definecolor{coolblack}{rgb}{0.0, 0.18, 0.39}

\usepackage{titlesec}
\newcommand\smallsection{%
  \titleformat{\section}
    {\normalfont\normalsize\bfseries}{\thesection}{1em}{}
}

\newcommand\einstieg{\paragraph{\color{amethyst} {Einstiegssituation:}}}
\newcommand\definition{\paragraph{\color{red} {Definition:}}}
\newcommand\fallbeispiel{\paragraph{\color{applegreen} {Fallbeispiel:}}}

%	HIER WERDEN TITEL REFERENT UND DATUM EINGETRAGEN
%
\newcommand\svthema{Pädagogik und Psychologie}
\newcommand\svperson{Leoni Heber, Gerald Schüller}
\newcommand\svdatum{\today}
\newcommand\lvtyp{Schuljahr 2022/23}
\newcommand\lvinst{Berufsschule für Kinderpflege - Höchstadt a. d. Aisch}

\begin{document}

\title{ \textbf{\color{blue}\svthema} }
\author{ \textsl{\color{red}\svperson} --- \svdatum }
\date{ \small {\lvtyp} - {\lvinst} }
\maketitle

\tableofcontents

\smallsection

% \newpage
\vskip 16pt \hrule
{ \color{coolblack}
  \section{Das aktuelle Bild vom Kind} }

\einstieg {\sl Lucca, ein Jahr und vier Monate, hat die Kinderpflegerin an die
  Hand genommen und zum Wickeltisch geführt, weil seine Windel voll
  war.}\footnote{\label{E-Lucca} PP, S. 13}

\vskip 10pt
{\rm Die Bindung bildet die Basis für die soziale Entwicklung des
  Kindes.}\footnote{\label{E-Bindung} PP, S. 14}

\definition \bf{Bindung}\index{Bindung}
\sl{ist ein lang anhaltendes, gefühlsmäßiges Band zu einer spezifischen Person,
  die nicht ausgetauscht werden kann.}\footnote{\label{Bindung} PP, S. 14}

% \newpage
\vskip 16pt \hrule
{ \color{coolblack}
  \section{Das Wesen der Erziehung} }

\einstieg {\sl Melanie, fünf Jahre, malt und ihre Nase läuft. Die
  Kinderpflegerin fordert sie auf, sich ein Taschentuch zu holen. Melanie kommt
  der Aufforderung nach.}\footnote{\label{E-Melanie} PP, S. 29}

\fallbeispiel \sl{Der dreijährige Linus
  klettert über den Zaun und läuft weg. Eine Kinderpflegerin erklärt ihm, warum
  er nicht über den Zaun steigen und weglaufen darf.}\footnote{\label{F-Linus} PP, S. 35}

\vskip 10pt
{\rm Da die Erziehung hier beabsichtigt geschieht, nennt man sie}
\bf{intentionale Erziehung}.

\definition \bf{Intentionale Erziehung}\index{Intentionale Erziehung} \sl{umfaßt
  die zielgerichteten Handlungen des Erziehers, die absichtsvoll und geplant
  durchgeführt werden.}\footnote{\label{Intensionale} PP, S. 35}

% \newpage
\vskip 16pt \hrule
{ \color{coolblack}
  \section{Erziehung in Familie und sozalpädagogischen Einrichtungen} }

\einstieg {\sl Ein Kindergarten wird von Kindern besucht wie Thomas oder Anna.
  Thomas Eltern - Spätaussieder mit einem niedrigen Bildungsniveau - leben in
  einer Sozialwohnung und verdienen wenig. Annas Eltern - Akademiker - wohnen
  im Eigenheim und verfügen über ein hohes Einkommen.}\footnote{\label{F-Anna} PP, S. 38}

\definition \bf{Familie}\index{Familie} \sl{bezeichnet eine Lebensform, die
  mindestens ein Kind und ein Elternteil umfasst, dauerhaft ist und im Innern
  durch Solidarität und persönliche Verbundenheit zusammengehalten
  wird.}\footnote{\label{Familie} PP, S. 39}

% \newpage
\vskip 16pt \hrule
{ \color{coolblack}
  \section{Personen und Situationen wahrnehmen und beobachten} }

\einstieg {\sl Erzieherin Theresa beobachtet Emely und
  beschreibt sie als \\ übertrieben
  bestimmend. Deswegen führt die Kinderpflegerin
  eine systematische Beobachtung durch und bestätigt ihre gegenteilige
  Wahrnehmung: Emely spielt kooperativ mit anderen Kindern.}\footnote{\label{F-Emily} PP, S. 67}

\definition \bf{Wahrnehmung}\index{Wahrnehmung} \sl{ist der Prozess der
  Reizaufname und Reizverarbeitung.}\footnote{\label{Wahrnehmung} PP, S. 71}

\definition \bf{Beobachtung}\index{Beobachtung} \sl{ist die bewusste und
  planvolle Wahrnung von Ereignissen und Verhaltensweisen.}\footnote{\label{Beobachtung} PP, S. 71}

% \newpage
\vskip 16pt \hrule
{ \color{coolblack}
  \section{Bedürfnisse wahrnehmen, erkennen und angemessen befriedigen} }

\einstieg {\sl Das Team des Kindergartens ,,Arche'' beschließt, ihn um eine
  Kindertagesstätte mit Krippe zu erweitern. Sie müssen nun anfangen, die Bedürnisse
  und Fähigkeiten aller Altersgruppen zu analysieren.}\footnote{\label{F-Arche} PP, S. 85}

\definition \sl{Der Begriff} {\bf ,,Bedürfnis''}\index{Bedürfnis} \sl{bezeichnet
  einen physischen oder psychischen Mangelzustand.}\footnote{\label{Bedürfnis} PP, S. 86}

\definition \bf{Motive}\index{Motiv} \sl{sind Begeggründe. Sie treiben den
  Menschen an, ein bestimmtes Ziel zu erreichen, bestimmte Bedürnisse zu
  befriedigen. Motivation ist der Vorgang, bei dem Motive den Menschen
  antreiben.}\footnote{\label{Motive} PP, S. 86}

\newpage
{ \color{coolblack}
  \section{Werte und Ziele in der Erziehung} }

\einstieg {\sl Pia aus dem Team der Krippengruppe hat beobachtet, dass sich Kinder für
  Körper und Krankheit oder Gesundheit interessieren. Das Team überlegt, welche
  Kompetenzen gestärkt werden können.}\footnote{\label{F-Pia} PP, S. 106}

\definition {\bf Kompetenz}\index{Kompetenz} \sl{bezeichnet die Fähigkeit und
  Bereitschaft des Einzelnen, Kenntnisse und Fertigkeiten sowie persönliche und
  methodische Fähigkeiten zu nutzen und sich in verschiedenen Lebenssituationen
  durchdacht sowie individuell und sozial verantwortlich zu
  verhalten.}\footnote{\label{Kompetenz} PP, S. 107}

% \newpage
\vskip 16pt \hrule
{ \color{coolblack}
  \section{Lernen und erziehen} }

\einstieg {\sl Rita, heute sieben Jahre alt, ist eine begabte
  Zeichnerin. Nach wenige Wochen in der ersten Klasse erlosch bei ihr die Freude
  am Zeichnen. Die Erzieherin überlegt, wie man die Freude am Zeichnen
  wiederbeleben könnte.}\footnote{\label{F-Rita} PP, S. 129}

\definition {\bf Erziehung}\index{Erziehung} \sl{ist das beabsichtigte und
  zielgerichtete Einwirken des Erziehenden auf das Kind, um das Verhalten oder
  Erleben des Kindes zu stärken oder zu ändern. Hat das Kind dieses Verhalten
  oder Erlben verändert oder eine neue Verhaltens- oder Erlebensweise erworben,
  so hat das Kind gelernt.}\footnote{\label{Erziehung} PP, S. 130}

% \newpage
\vskip 16pt \hrule
{ \color{coolblack}
  \section{Erziehungsstile} }

\einstieg {\sl Die Geschwister Martin, neun Jahre, und Maria, sieben Jahre,
  wohnen im Kinderheim. Sie trödeln morgens und kommen deshalb nicht
  rechtzeitig zum Frühstück. Die Erzieherin Sabine greift nicht ein, obwohl die
  Kinder zu spät in die Schule kommen. Anders fordert sie die Praktikantin Katja
  erst freundlich auf, ermahnt sie dann und schimpft. Martin beschwert sich:
  ,,Sabine lässt uns in Ruhe!'' Sie will die Angelegenheit mit dem Team
  besprechen.}\footnote{\label{F-Katja} PP, S. 158}

\definition {\sl Unter} {\bf ,,Erziehungsstiel''}\index{Erziehungsstiel}
\sl{versteht man die Grundhaltung, eine charakteristische Art und Weise, die
  Erziehende den Kindern gegenüber einnehmen. Es sind Muster von Einstellungen,
  Handlungsweisen, sprachlichen und nicht sprachlichen Äußerungen, die die Art
  des Umgangs von Erziehenden mit Kindern kennzeichnen.}\footnote{\label{Erziehungsstiel} PP, S. 159}

% \newpage
\vskip 16pt \hrule
\section{A}

\einstieg {\sl x.}\footnote{\label{F-x} B, S. i}

\definition {\bf y}\footnote{\label{y} B, S. i}
\index{y} \sl{z.}

\printindex

\end{document}
