\input cwebmac
% This file is part of CWEB.
% This program by Silvio Levy and Donald E. Knuth
% is based on a program by Knuth.
% It is distributed WITHOUT ANY WARRANTY, express or implied.
% Version 4.7 --- February 2022

% Copyright (C) 1987,1990,1993,2000 Silvio Levy and Donald E. Knuth

% Permission is granted to make and distribute verbatim copies of this
% document provided that the copyright notice and this permission notice
% are preserved on all copies.

% Permission is granted to copy and distribute modified versions of this
% document under the conditions for verbatim copying, provided that the
% entire resulting derived work is given a different name and distributed
% under the terms of a permission notice identical to this one.

% Amendments to 'cweave.w' resulting in this updated version were created
% by numerous collaborators over the course of many years.

% Please send comments, suggestions, etc. to tex-k@tug.org.

% Here is TeX material that gets inserted after \input cwebmac
\def\hang{\hangindent 3em\indent\ignorespaces}
\def\pb{$\.|\ldots\.|$} % C brackets (|...|)
\def\v{\char'174} % vertical (|) in typewriter font
\def\dleft{[\![} \def\dright{]\!]} % double brackets
\mathchardef\RA="3221 % right arrow
\mathchardef\BA="3224 % double arrow
\def\({} % ) kludge for alphabetizing certain section names
\def\TeXxstring{\\{\TEX/\_string}}
\def\skipxTeX{\\{skip\_\TEX/}}
\def\copyxTeX{\\{copy\_\TEX/}}

\def\title{CWEAVE (Version 4.7)}
\def\topofcontents{\null\vfill
  \centerline{\titlefont The {\ttitlefont CWEAVE} processor}
  \vskip 15pt
  \centerline{(Version 4.7)}
  \vfill}
\def\botofcontents{\vfill
\noindent
Copyright \copyright\ 1987, 1990, 1993, 2000 Silvio Levy and Donald E. Knuth
\bigskip\noindent
Permission is granted to make and distribute verbatim copies of this
document provided that the copyright notice and this permission notice
are preserved on all copies.

\smallskip\noindent
Permission is granted to copy and distribute modified versions of this
document under the conditions for verbatim copying, provided that the
entire resulting derived work is given a different name and distributed
under the terms of a permission notice identical to this one.
}
\pageno=\contentspagenumber \advance\pageno by 1
\let\maybe=\iftrue


\N{0}{1}Introduction.
This is the \.{CWEAVE} program by Silvio Levy and Donald E. Knuth,
based on \.{WEAVE} by Knuth.
We are thankful to Steve Avery,
Nelson Beebe, Hans-Hermann Bode (to whom the original \CPLUSPLUS/ adaptation
is due), Klaus Guntermann, Norman Ramsey, Tomas Rokicki, Joachim Schnitter,
Joachim Schrod, Lee Wittenberg, Saroj Mahapatra, Cesar Augusto Rorato
Crusius, and others who have contributed improvements.

The ``banner line'' defined here should be changed whenever \.{CWEAVE}
is modified.

\Y\B\4\D\\{banner}\5
\.{"This\ is\ CWEAVE\ (Ver}\)\.{sion\ 4.7)"}\par
\Y\B\X4:Include files\X\6
\ATH\6
\X3:Common code for \.{CWEAVE} and \.{CTANGLE}\X\6
\X22:Typedef declarations\X\6
\X21:Private variables\X\6
\X8:Predeclaration of procedures\X\par
\fi

\M{2}\.{CWEAVE} has a fairly straightforward outline.  It operates in
three phases: First it inputs the source file and stores cross-reference
data, then it inputs the source once again and produces the \TEX/ output
file, finally it sorts and outputs the index.

Please read the documentation for \.{COMMON}, the set of routines common
to \.{CTANGLE} and \.{CWEAVE}, before proceeding further.

\Y\B\1\1\&{int} \\{main}(\&{int} \\{ac}${},{}$\C{ argument count }\6
\&{char} ${}{*}{*}\\{av}{}$)\C{ argument values }\2\2\6
${}\{{}$\1\6
${}\\{argc}\K\\{ac};{}$\6
${}\\{argv}\K\\{av};{}$\6
${}\\{program}\K\\{cweave};{}$\6
\X24:Set initial values\X\6
\\{common\_init}(\,);\6
\X89:Start \TEX/ output\X\6
\&{if} (\\{show\_banner})\1\5
\\{puts}(\\{banner});\C{ print a ``banner line'' }\2\6
\X34:Store all the reserved words\X\6
\\{phase\_one}(\,);\C{ read all the user's text and store the cross-references
}\6
\\{phase\_two}(\,);\C{ read all the text again and translate it to \TEX/ form }%
\6
\\{phase\_three}(\,);\C{ output the cross-reference index }\6
\&{if} ${}(\\{tracing}\E\\{fully}\W\R\\{show\_progress}){}$\1\5
\\{new\_line};\2\6
\&{return} \\{wrap\_up}(\,);\C{ and exit gracefully }\6
\4${}\}{}$\2\par
\fi

\M{3}The next few sections contain stuff from the file \PB{\.{"common.w"}} that
must
be included in both \PB{\.{"ctangle.w"}} and \PB{\.{"cweave.w"}}. It appears in
file \PB{\.{"common.h"}}, which is also included in \PB{\.{"common.w"}} to
propagate
possible changes from this \.{COMMON} interface consistently.

% This file is part of CWEB.
% This program by Silvio Levy and Donald E. Knuth
% is based on a program by Knuth.
% It is distributed WITHOUT ANY WARRANTY, express or implied.
% Version 4.7 --- February 2022 (works also with later versions)

% Copyright (C) 1987,1990,1993 Silvio Levy and Donald E. Knuth

% Permission is granted to make and distribute verbatim copies of this
% document provided that the copyright notice and this permission notice
% are preserved on all copies.

% Permission is granted to copy and distribute modified versions of this
% document under the conditions for verbatim copying, provided that the
% entire resulting derived work is given a different name and distributed
% under the terms of a permission notice identical to this one.

% Amendments to 'common.h' resulting in this updated version were created
% by numerous collaborators over the course of many years.

% Please send comments, suggestions, etc. to tex-k@tug.org.

% The next few sections contain stuff from the file \PB{\.{"common.w"}} that
%has
% to be included in both \PB{\.{"ctangle.w"}} and \PB{\.{"cweave.w"}}. It
%appears in this
% file \PB{\.{"common.h"}}, which is also included in \PB{\.{"common.w"}} to
%propagate
% possible changes from this single source consistently.

First comes general stuff:



\Y\B\4\D\\{ctangle}\5
\\{false}\par
\B\4\D\\{cweave}\5
\\{true}\par
\Y\B\4\X3:Common code for \.{CWEAVE} and \.{CTANGLE}\X${}\E{}$\6
\&{typedef} \&{bool} \&{boolean};\6
\&{typedef} \&{uint8\_t} \&{eight\_bits};\6
\&{typedef} \&{uint16\_t} \&{sixteen\_bits};\6
\&{extern} \&{boolean} \\{program};\C{ \.{CWEAVE} or \.{CTANGLE}? }\6
\&{extern} \&{int} \\{phase};\C{ which phase are we in? }\par
\As5, 6, 7, 9, 10, 12, 14\ETs15.
\U1.\fi

\M{4}Interface to the standard \CEE/ library:

\Y\B\4\X4:Include files\X${}\E{}$\6
\8\#\&{include} \.{<ctype.h>}\C{ definition of \PB{\\{isalpha}}, \PB{%
\\{isdigit}} and so on }\6
\8\#\&{include} \.{<stdbool.h>}\C{ definition of \PB{\&{bool}}, \PB{\\{true}}
and \PB{\\{false}} }\6
\8\#\&{include} \.{<stddef.h>}\C{ definition of \PB{\&{ptrdiff\_t}} }\6
\8\#\&{include} \.{<stdint.h>}\C{ definition of \PB{\&{uint8\_t}} and \PB{%
\&{uint16\_t}} }\6
\8\#\&{include} \.{<stdio.h>}\C{ definition of \PB{\\{printf}} and friends }\6
\8\#\&{include} \.{<stdlib.h>}\C{ definition of \PB{\\{getenv}} and \PB{%
\\{exit}} }\6
\8\#\&{include} \.{<string.h>}\C{ definition of \PB{\\{strlen}}, \PB{%
\\{strcmp}} and so on }\par
\U1.\fi

\M{5}Code related to the character set:

\Y\B\4\D\\{and\_and}\5
\T{\~4}\C{ `\.{\&\&}'\,; corresponds to MIT's {\tentex\char'4} }\par
\B\4\D\\{lt\_lt}\5
\T{\~20}\C{ `\.{<<}'\,; corresponds to MIT's {\tentex\char'20} }\par
\B\4\D\\{gt\_gt}\5
\T{\~21}\C{ `\.{>>}'\,; corresponds to MIT's {\tentex\char'21} }\par
\B\4\D\\{plus\_plus}\5
\T{\~13}\C{ `\.{++}'\,; corresponds to MIT's {\tentex\char'13} }\par
\B\4\D\\{minus\_minus}\5
\T{\~1}\C{ `\.{--}'\,; corresponds to MIT's {\tentex\char'1} }\par
\B\4\D\\{minus\_gt}\5
\T{\~31}\C{ `\.{->}'\,; corresponds to MIT's {\tentex\char'31} }\par
\B\4\D\\{non\_eq}\5
\T{\~32}\C{ `\.{!=}'\,; corresponds to MIT's {\tentex\char'32} }\par
\B\4\D\\{lt\_eq}\5
\T{\~34}\C{ `\.{<=}'\,; corresponds to MIT's {\tentex\char'34} }\par
\B\4\D\\{gt\_eq}\5
\T{\~35}\C{ `\.{>=}'\,; corresponds to MIT's {\tentex\char'35} }\par
\B\4\D\\{eq\_eq}\5
\T{\~36}\C{ `\.{==}'\,; corresponds to MIT's {\tentex\char'36} }\par
\B\4\D\\{or\_or}\5
\T{\~37}\C{ `\.{\v\v}'\,; corresponds to MIT's {\tentex\char'37} }\par
\B\4\D\\{dot\_dot\_dot}\5
\T{\~16}\C{ `\.{...}'\,; corresponds to MIT's {\tentex\char'16} }\par
\B\4\D\\{colon\_colon}\5
\T{\~6}\C{ `\.{::}'\,; corresponds to MIT's {\tentex\char'6} }\par
\B\4\D\\{period\_ast}\5
\T{\~26}\C{ `\.{.*}'\,; corresponds to MIT's {\tentex\char'26} }\par
\B\4\D\\{minus\_gt\_ast}\5
\T{\~27}\C{ `\.{->*}'\,; corresponds to MIT's {\tentex\char'27} }\Y\par
\B\4\D\\{compress}$(\|c)$\5
\&{if} ${}(\\{loc}\PP\Z\\{limit})$ \&{return} \|c\par
\Y\B\4\X3:Common code for \.{CWEAVE} and \.{CTANGLE}\X${}\mathrel+\E{}$\6
\&{extern} \&{char} \\{section\_text}[\,];\C{ text being sought for }\6
\&{extern} \&{char} ${}{*}\\{section\_text\_end}{}$;\C{ end of \PB{\\{section%
\_text}} }\6
\&{extern} \&{char} ${}{*}\\{id\_first}{}$;\C{ where the current identifier
begins in the buffer }\6
\&{extern} \&{char} ${}{*}\\{id\_loc}{}$;\C{ just after the current identifier
in the buffer }\par
\fi

\M{6}Code related to input routines:
\Y\B\4\D\\{xisalpha}$(\|c)$\5
$(\\{isalpha}((\&{int})(\|c))\W((\&{eight\_bits})(\|c)<\T{\~200}){}$)\par
\B\4\D\\{xisdigit}$(\|c)$\5
$(\\{isdigit}((\&{int})(\|c))\W((\&{eight\_bits})(\|c)<\T{\~200}){}$)\par
\B\4\D\\{xisspace}$(\|c)$\5
$(\\{isspace}((\&{int})(\|c))\W((\&{eight\_bits})(\|c)<\T{\~200}){}$)\par
\B\4\D\\{xislower}$(\|c)$\5
$(\\{islower}((\&{int})(\|c))\W((\&{eight\_bits})(\|c)<\T{\~200}){}$)\par
\B\4\D\\{xisupper}$(\|c)$\5
$(\\{isupper}((\&{int})(\|c))\W((\&{eight\_bits})(\|c)<\T{\~200}){}$)\par
\B\4\D\\{xisxdigit}$(\|c)$\5
$(\\{isxdigit}((\&{int})(\|c))\W((\&{eight\_bits})(\|c)<\T{\~200}){}$)\par
\B\4\D\\{isxalpha}$(\|c)$\5
$((\|c)\E\.{'\_'}\V(\|c)\E\.{'\$'}{}$)\C{ non-alpha characters allowed in
identifier }\par
\B\4\D\\{ishigh}$(\|c)$\5
$((\&{eight\_bits})(\|c)>\T{\~177}{}$)\par
\Y\B\4\X3:Common code for \.{CWEAVE} and \.{CTANGLE}\X${}\mathrel+\E{}$\6
\&{extern} \&{char} \\{buffer}[\,];\C{ where each line of input goes }\6
\&{extern} \&{char} ${}{*}\\{buffer\_end}{}$;\C{ end of \PB{\\{buffer}} }\6
\&{extern} \&{char} ${}{*}\\{loc}{}$;\C{ points to the next character to be
read from the buffer }\6
\&{extern} \&{char} ${}{*}\\{limit}{}$;\C{ points to the last character in the
buffer }\par
\fi

\M{7}Code related to file handling:
\Y\B\F\\{line}\5
\|x\C{ make \PB{\\{line}} an unreserved word }\par
\B\4\D\\{max\_include\_depth}\5
\T{10}\C{ maximum number of source files open   simultaneously, not counting
the change file }\par
\B\4\D\\{max\_file\_name\_length}\5
\T{60}\par
\B\4\D\\{cur\_file}\5
\\{file}[\\{include\_depth}]\C{ current file }\par
\B\4\D\\{cur\_file\_name}\5
\\{file\_name}[\\{include\_depth}]\C{ current file name }\par
\B\4\D\\{cur\_line}\5
\\{line}[\\{include\_depth}]\C{ number of current line in current file }\par
\B\4\D\\{web\_file}\5
\\{file}[\T{0}]\C{ main source file }\par
\B\4\D\\{web\_file\_name}\5
\\{file\_name}[\T{0}]\C{ main source file name }\par
\Y\B\4\X3:Common code for \.{CWEAVE} and \.{CTANGLE}\X${}\mathrel+\E{}$\6
\&{extern} \&{int} \\{include\_depth};\C{ current level of nesting }\6
\&{extern} \&{FILE} ${}{*}\\{file}[\,]{}$;\C{ stack of non-change files }\6
\&{extern} \&{FILE} ${}{*}\\{change\_file}{}$;\C{ change file }\6
\&{extern} \&{char} \\{file\_name}[\,][\\{max\_file\_name\_length}];\C{ stack
of non-change file names }\6
\&{extern} \&{char} \\{change\_file\_name}[\,];\C{ name of change file }\6
\&{extern} \&{int} \\{line}[\,];\C{ number of current line in the stacked files
}\6
\&{extern} \&{int} \\{change\_line};\C{ number of current line in change file }%
\6
\&{extern} \&{int} \\{change\_depth};\C{ where \.{@y} originated during a
change }\6
\&{extern} \&{boolean} \\{input\_has\_ended};\C{ if there is no more input }\6
\&{extern} \&{boolean} \\{changing};\C{ if the current line is from \PB{%
\\{change\_file}} }\6
\&{extern} \&{boolean} \\{web\_file\_open};\C{ if the web file is being read }%
\par
\fi

\M{8}\B\X8:Predeclaration of procedures\X${}\E{}$\6
\&{extern} \&{boolean} \\{get\_line}(\&{void});\C{ inputs the next line }\6
\&{extern} \&{void} \\{check\_complete}(\&{void});\C{ checks that all changes
were picked up }\6
\&{extern} \&{void} \\{reset\_input}(\&{void});\C{ initialize to read the web
file and change file }\par
\As11, 13, 16, 25, 40, 45, 65, 69, 71, 83, 86, 90, 95, 98, 116, 118, 122, 181,
189, 194, 201, 210, 214, 228, 235, 244, 248, 259\ETs268.
\U1.\fi

\M{9}Code related to section numbers:
\Y\B\4\X3:Common code for \.{CWEAVE} and \.{CTANGLE}\X${}\mathrel+\E{}$\6
\&{extern} \&{sixteen\_bits} \\{section\_count};\C{ the current section number
}\6
\&{extern} \&{boolean} \\{changed\_section}[\,];\C{ is the section changed? }\6
\&{extern} \&{boolean} \\{change\_pending};\C{ is a decision about change still
unclear? }\6
\&{extern} \&{boolean} \\{print\_where};\C{ tells \.{CTANGLE} to print line and
file info }\par
\fi

\M{10}Code related to identifier and section name storage:
\Y\B\4\D\\{length}$(\|c)$\5
$(\&{size\_t})((\|c+\T{1})\MG\\{byte\_start}-(\|c)\MG\\{byte\_start}{}$)\C{ the
length of a name }\par
\B\4\D\\{print\_id}$(\|c)$\5
$\\{term\_write}((\|c)\MG\\{byte\_start},\39\\{length}(\|c){}$)\C{ print
identifier }\par
\B\4\D\\{llink}\5
\\{link}\C{ left link in binary search tree for section names }\par
\B\4\D\\{rlink}\5
$\\{dummy}.{}$\\{Rlink}\C{ right link in binary search tree for section names }%
\par
\B\4\D\\{root}\5
$\\{name\_dir}\MG{}$\\{rlink}\C{ the root of the binary search tree   for
section names }\par
\Y\B\4\X3:Common code for \.{CWEAVE} and \.{CTANGLE}\X${}\mathrel+\E{}$\6
\&{typedef} \&{struct} \&{name\_info} ${}\{{}$\1\6
\&{char} ${}{*}\\{byte\_start}{}$;\C{ beginning of the name in \PB{\\{byte%
\_mem}} }\6
\&{struct} \&{name\_info} ${}{*}\\{link};{}$\6
\&{union} ${}\{{}$\1\6
\&{struct} \&{name\_info} ${}{*}\\{Rlink}{}$;\C{ right link in binary search
tree for section       names }\6
\&{char} \\{Ilk};\C{ used by identifiers in \.{CWEAVE} only }\2\6
${}\}{}$ \\{dummy};\6
\&{void} ${}{*}\\{equiv\_or\_xref}{}$;\C{ info corresponding to names }\2\6
${}\}{}$ \&{name\_info};\C{ contains information about an identifier or section
name }\6
\&{typedef} \&{name\_info} ${}{*}\&{name\_pointer}{}$;\C{ pointer into array of
\&{name\_info}s }\6
\&{typedef} \&{name\_pointer} ${}{*}\&{hash\_pointer};{}$\6
\&{extern} \&{char} \\{byte\_mem}[\,];\C{ characters of names }\6
\&{extern} \&{char} ${}{*}\\{byte\_mem\_end}{}$;\C{ end of \PB{\\{byte\_mem}} }%
\6
\&{extern} \&{char} ${}{*}\\{byte\_ptr}{}$;\C{ first unused position in \PB{%
\\{byte\_mem}} }\6
\&{extern} \&{name\_info} \\{name\_dir}[\,];\C{ information about names }\6
\&{extern} \&{name\_pointer} \\{name\_dir\_end};\C{ end of \PB{\\{name\_dir}} }%
\6
\&{extern} \&{name\_pointer} \\{name\_ptr};\C{ first unused position in \PB{%
\\{name\_dir}} }\6
\&{extern} \&{name\_pointer} \\{hash}[\,];\C{ heads of hash lists }\6
\&{extern} \&{hash\_pointer} \\{hash\_end};\C{ end of \PB{\\{hash}} }\6
\&{extern} \&{hash\_pointer} \|h;\C{ index into hash-head array }\par
\fi

\M{11}\B\X8:Predeclaration of procedures\X${}\mathrel+\E{}$\6
\&{extern} \&{boolean} ${}\\{names\_match}(\&{name\_pointer},\39{}$\&{const} %
\&{char} ${}{*},\39\&{size\_t},\39\&{eight\_bits}){}$;\6
\&{extern} \&{name\_pointer} \\{id\_lookup}(\&{const} \&{char} ${}{*},\39{}$%
\&{const} \&{char} ${}{*},\39\&{eight\_bits}){}$;\C{ looks up a string in the
identifier table }\6
\&{extern} \&{name\_pointer} \\{section\_lookup}(\&{char} ${}{*},\39{}$\&{char}
${}{*},\39\&{boolean}){}$;\C{ finds section name }\6
\&{extern} \&{void} \\{init\_node}(\&{name\_pointer});\6
\&{extern} \&{void} ${}\\{init\_p}(\&{name\_pointer},\39\&{eight\_bits}){}$;\6
\&{extern} \&{void} \\{print\_prefix\_name}(\&{name\_pointer});\6
\&{extern} \&{void} \\{print\_section\_name}(\&{name\_pointer});\6
\&{extern} \&{void} \\{sprint\_section\_name}(\&{char} ${}{*},\39\&{name%
\_pointer}){}$;\par
\fi

\M{12}Code related to error handling:
\Y\B\4\D\\{spotless}\5
\T{0}\C{ \PB{\\{history}} value for normal jobs }\par
\B\4\D\\{harmless\_message}\5
\T{1}\C{ \PB{\\{history}} value when non-serious info was printed }\par
\B\4\D\\{error\_message}\5
\T{2}\C{ \PB{\\{history}} value when an error was noted }\par
\B\4\D\\{fatal\_message}\5
\T{3}\C{ \PB{\\{history}} value when we had to stop prematurely }\par
\B\4\D\\{mark\_harmless}\5
\&{if} ${}(\\{history}\E\\{spotless})$ $\\{history}\K{}$\\{harmless\_message}%
\par
\B\4\D\\{mark\_error}\5
$\\{history}\K{}$\\{error\_message}\par
\B\4\D\\{confusion}$(\|s)$\5
$\\{fatal}(\.{"!\ This\ can't\ happen}\)\.{:\ "},\39\|s{}$)\par
\Y\B\4\X3:Common code for \.{CWEAVE} and \.{CTANGLE}\X${}\mathrel+\E{}$\6
\&{extern} \&{int} \\{history};\C{ indicates how bad this run was }\par
\fi

\M{13}\B\X8:Predeclaration of procedures\X${}\mathrel+\E{}$\6
\&{extern} \&{int} \\{wrap\_up}(\&{void});\C{ indicate \PB{\\{history}} and
exit }\6
\&{extern} \&{void} \\{err\_print}(\&{const} \&{char} ${}{*}){}$;\C{ print
error message and context }\6
\&{extern} \&{void} \\{fatal}(\&{const} \&{char} ${}{*},\39{}$\&{const} %
\&{char} ${}{*}){}$;\C{ issue error message and die }\6
\&{extern} \&{void} \\{overflow}(\&{const} \&{char} ${}{*}){}$;\C{ succumb
because a table has overflowed }\par
\fi

\M{14}Code related to command line arguments:
\Y\B\4\D\\{show\_banner}\5
\\{flags}[\.{'b'}]\C{ should the banner line be printed? }\par
\B\4\D\\{show\_progress}\5
\\{flags}[\.{'p'}]\C{ should progress reports be printed? }\par
\B\4\D\\{show\_happiness}\5
\\{flags}[\.{'h'}]\C{ should lack of errors be announced? }\par
\B\4\D\\{show\_stats}\5
\\{flags}[\.{'s'}]\C{ should statistics be printed at end of run? }\par
\B\4\D\\{make\_xrefs}\5
\\{flags}[\.{'x'}]\C{ should cross references be output? }\par
\Y\B\4\X3:Common code for \.{CWEAVE} and \.{CTANGLE}\X${}\mathrel+\E{}$\6
\&{extern} \&{int} \\{argc};\C{ copy of \PB{\\{ac}} parameter to \PB{\\{main}}
}\6
\&{extern} \&{char} ${}{*}{*}\\{argv}{}$;\C{ copy of \PB{\\{av}} parameter to %
\PB{\\{main}} }\6
\&{extern} \&{char} \\{C\_file\_name}[\,];\C{ name of \PB{\\{C\_file}} }\6
\&{extern} \&{char} \\{tex\_file\_name}[\,];\C{ name of \PB{\\{tex\_file}} }\6
\&{extern} \&{char} \\{idx\_file\_name}[\,];\C{ name of \PB{\\{idx\_file}} }\6
\&{extern} \&{char} \\{scn\_file\_name}[\,];\C{ name of \PB{\\{scn\_file}} }\6
\&{extern} \&{boolean} \\{flags}[\,];\C{ an option for each 7-bit code }\par
\fi

\M{15}Code related to output:
\Y\B\4\D\\{update\_terminal}\5
\\{fflush}(\\{stdout})\C{ empty the terminal output buffer }\par
\B\4\D\\{new\_line}\5
\\{putchar}(\.{'\\n'})\par
\B\4\D\\{term\_write}$(\|a,\|b)$\5
$\\{fflush}(\\{stdout}),\39\\{fwrite}(\|a,\39\&{sizeof}(\&{char}),\39\|b,\39%
\\{stdout}{}$)\par
\Y\B\4\X3:Common code for \.{CWEAVE} and \.{CTANGLE}\X${}\mathrel+\E{}$\6
\&{extern} \&{FILE} ${}{*}\\{C\_file}{}$;\C{ where output of \.{CTANGLE} goes }%
\6
\&{extern} \&{FILE} ${}{*}\\{tex\_file}{}$;\C{ where output of \.{CWEAVE} goes
}\6
\&{extern} \&{FILE} ${}{*}\\{idx\_file}{}$;\C{ where index from \.{CWEAVE} goes
}\6
\&{extern} \&{FILE} ${}{*}\\{scn\_file}{}$;\C{ where list of sections from %
\.{CWEAVE} goes }\6
\&{extern} \&{FILE} ${}{*}\\{active\_file}{}$;\C{ currently active file for %
\.{CWEAVE} output }\par
\fi

\M{16}The procedure that gets everything rolling:
\Y\B\4\X8:Predeclaration of procedures\X${}\mathrel+\E{}$\6
\&{extern} \&{void} \\{common\_init}(\&{void});\6
\&{extern} \&{void} \\{print\_stats}(\&{void});\par
\fi

\M{17}The following parameters are sufficient to handle \TEX/ (converted to
\.{CWEB}), so they should be sufficient for most applications of \.{CWEB}.

\Y\B\4\D\\{buf\_size}\5
\T{200}\C{ maximum length of input line, plus one }\par
\B\4\D\\{longest\_name}\5
\T{10000}\C{ file names, section names, and section texts    shouldn't be
longer than this }\par
\B\4\D\\{long\_buf\_size}\5
$(\\{buf\_size}+\\{longest\_name}{}$)\C{ for \.{CWEAVE} }\par
\B\4\D\\{max\_bytes}\5
\T{100000}\C{ the number of bytes in identifiers,   index entries, and section
names; must be less than $2^{24}$ }\par
\B\4\D\\{max\_names}\5
\T{5000}\C{ number of identifiers, strings, section names;   must be less than
10240 }\par
\B\4\D\\{max\_sections}\5
\T{2000}\C{ greater than the total number of sections }\par
\fi

\M{18}End of \.{COMMON} interface.

\fi

\M{19}The following parameters are sufficient to handle \TEX/ (converted to
\.{CWEB}), so they should be sufficient for most applications of \.{CWEAVE}.

\Y\B\4\D\\{line\_length}\5
\T{80}\C{ lines of \TEX/ output have at most this many characters;   should be
less than 256 }\par
\B\4\D\\{max\_refs}\5
\T{30000}\C{ number of cross-references; must be less than 65536 }\par
\B\4\D\\{max\_scraps}\5
\T{5000}\C{ number of tokens in \CEE/ texts being parsed }\par
\fi

\N{1}{20}Data structures exclusive to {\tt CWEAVE}.
As explained in \.{common.w}, the field of a \PB{\&{name\_info}} structure
that contains the \PB{\\{rlink}} of a section name is used for a completely
different purpose in the case of identifiers.  It is then called the
\PB{\\{ilk}} of the identifier, and it is used to
distinguish between various types of identifiers, as follows:

\yskip\hang \PB{\\{normal}} and \PB{\\{func\_template}} identifiers are part of
the
\CEE/ program that will  appear in italic type (or in typewriter type
if all uppercase).

\yskip\hang \PB{\\{custom}} identifiers are part of the \CEE/ program that
will be typeset in special ways.

\yskip\hang \PB{\\{roman}} identifiers are index entries that appear after
\.{@\^} in the \.{CWEB} file.

\yskip\hang \PB{\\{wildcard}} identifiers are index entries that appear after
\.{@:} in the \.{CWEB} file.

\yskip\hang \PB{\\{typewriter}} identifiers are index entries that appear after
\.{@.} in the \.{CWEB} file.

\yskip\hang \PB{\\{alfop}}, \dots, \PB{\\{attr}}
identifiers are \CEE/ or \CPLUSPLUS/ reserved words whose \PB{\\{ilk}}
explains how they are to be treated when \CEE/ code is being
formatted.

\Y\B\4\D\\{ilk}\5
$\\{dummy}.{}$\\{Ilk}\par
\B\4\D\\{normal}\5
\T{0}\C{ ordinary identifiers have \PB{\\{normal}} ilk }\par
\B\4\D\\{roman}\5
\T{1}\C{ normal index entries have \PB{\\{roman}} ilk }\par
\B\4\D\\{wildcard}\5
\T{2}\C{ user-formatted index entries have \PB{\\{wildcard}} ilk }\par
\B\4\D\\{typewriter}\5
\T{3}\C{ `typewriter type' entries have \PB{\\{typewriter}} ilk }\par
\B\4\D\\{abnormal}$(\|a)$\5
$((\|a)\MG\\{ilk}>\\{typewriter}{}$)\C{ tells if a name is special }\par
\B\4\D\\{func\_template}\5
\T{4}\C{ identifiers that can be followed by optional template }\par
\B\4\D\\{custom}\5
\T{5}\C{ identifiers with user-given control sequence }\par
\B\4\D\\{alfop}\5
\T{22}\C{ alphabetic operators like \&{and} or \&{not\_eq} }\par
\B\4\D\\{else\_like}\5
\T{26}\C{ \&{else} }\par
\B\4\D\\{public\_like}\5
\T{40}\C{ \&{public}, \&{private}, \&{protected} }\par
\B\4\D\\{operator\_like}\5
\T{41}\C{ \&{operator} }\par
\B\4\D\\{new\_like}\5
\T{42}\C{ \&{new} }\par
\B\4\D\\{catch\_like}\5
\T{43}\C{ \&{catch} }\par
\B\4\D\\{for\_like}\5
\T{45}\C{ \&{for}, \&{switch}, \&{while} }\par
\B\4\D\\{do\_like}\5
\T{46}\C{ \&{do} }\par
\B\4\D\\{if\_like}\5
\T{47}\C{ \&{if}, \&{ifdef}, \&{endif}, \&{pragma}, \dots }\par
\B\4\D\\{delete\_like}\5
\T{48}\C{ \&{delete} }\par
\B\4\D\\{raw\_ubin}\5
\T{49}\C{ `\.\&' or `\.*' when looking for \&{const} following }\par
\B\4\D\\{const\_like}\5
\T{50}\C{ \&{const}, \&{volatile} }\par
\B\4\D\\{raw\_int}\5
\T{51}\C{ \&{int}, \&{char}, \dots; also structure and class names }\par
\B\4\D\\{int\_like}\5
\T{52}\C{ same, when not followed by left parenthesis or \DC\ }\par
\B\4\D\\{case\_like}\5
\T{53}\C{ \&{case}, \&{return}, \&{goto}, \&{break}, \&{continue} }\par
\B\4\D\\{sizeof\_like}\5
\T{54}\C{ \&{sizeof} }\par
\B\4\D\\{struct\_like}\5
\T{55}\C{ \&{struct}, \&{union}, \&{enum}, \&{class} }\par
\B\4\D\\{typedef\_like}\5
\T{56}\C{ \&{typedef} }\par
\B\4\D\\{define\_like}\5
\T{57}\C{ \&{define} }\par
\B\4\D\\{template\_like}\5
\T{58}\C{ \&{template} }\par
\B\4\D\\{alignas\_like}\5
\T{59}\C{ \&{alignas} }\par
\B\4\D\\{using\_like}\5
\T{60}\C{ \&{using} }\par
\B\4\D\\{default\_like}\5
\T{61}\C{ \&{default} }\par
\B\4\D\\{attr}\5
\T{62}\C{ \&{noexcept} and attributes }\par
\fi

\M{21}We keep track of the current section number in \PB{\\{section\_count}},
which
is the total number of sections that have started.  Sections which have
been altered by a change file entry have their \PB{\\{changed\_section}} flag
turned on during the first phase.

\Y\B\4\X21:Private variables\X${}\E{}$\6
\&{static} \&{boolean} \\{change\_exists};\C{ has any section changed? }\par
\As23, 30, 37, 43, 46, 48, 67, 76, 81, 85, 106, 113, 119, 186, 208, 213, 229,
238, 249, 251, 254, 256\ETs265.
\U1.\fi

\M{22}The other large memory area in \.{CWEAVE} keeps the cross-reference data.
All uses of the name \PB{\|p} are recorded in a linked list beginning at
\PB{$\|p\MG\\{xref}$}, which points into the \PB{\\{xmem}} array. The elements
of \PB{\\{xmem}}
are structures consisting of an integer, \PB{\\{num}}, and a pointer \PB{%
\\{xlink}}
to another element of \PB{\\{xmem}}.  If \PB{$\|x\K\|p\MG\\{xref}$} is a
pointer into \PB{\\{xmem}},
the value of \PB{$\|x\MG\\{num}$} is either a section number where \PB{\|p} is
used,
or \PB{\\{cite\_flag}} plus a section number where \PB{\|p} is mentioned,
or \PB{\\{def\_flag}} plus a section number where \PB{\|p} is defined;
and \PB{$\|x\MG\\{xlink}$} points to the next such cross-reference for \PB{%
\|p},
if any. This list of cross-references is in decreasing order by
section number. The next unused slot in \PB{\\{xmem}} is \PB{\\{xref\_ptr}}.
The linked list ends at \PB{${\AND}\\{xmem}[\T{0}]$}.

The global variable \PB{\\{xref\_switch}} is set either to \PB{\\{def\_flag}}
or to zero,
depending on whether the next cross-reference to an identifier is to be
underlined or not in the index. This switch is set to \PB{\\{def\_flag}} when
\.{@!} or \.{@d} is scanned, and it is cleared to zero when
the next identifier or index entry cross-reference has been made.
Similarly, the global variable \PB{\\{section\_xref\_switch}} is either
\PB{\\{def\_flag}} or \PB{\\{cite\_flag}} or zero, depending
on whether a section name is being defined, cited or used in \CEE/ text.

\Y\B\4\X22:Typedef declarations\X${}\E{}$\6
\&{typedef} \&{struct} \&{xref\_info} ${}\{{}$\1\6
\&{sixteen\_bits} \\{num};\C{ section number plus zero or \PB{\\{def\_flag}} }\6
\&{struct} \&{xref\_info} ${}{*}\\{xlink}{}$;\C{ pointer to the previous
cross-reference }\2\6
${}\}{}$ \&{xref\_info};\6
\&{typedef} \&{xref\_info} ${}{*}\&{xref\_pointer}{}$;\par
\As29, 112\ETs207.
\U1.\fi

\M{23}\B\X21:Private variables\X${}\mathrel+\E{}$\6
\&{static} \&{xref\_info} \\{xmem}[\\{max\_refs}];\C{ contains cross-reference
information }\6
\&{static} \&{xref\_pointer} \\{xmem\_end}${}\K\\{xmem}+\\{max\_refs}-\T{1};{}$%
\6
\&{static} \&{xref\_pointer} \\{xref\_ptr};\C{ the largest occupied position in
\PB{\\{xmem}} }\6
\&{static} \&{sixteen\_bits} \\{xref\_switch}${},{}$ \\{section\_xref\_switch};%
\C{ either zero or \PB{\\{def\_flag}} }\par
\fi

\M{24}A section that is used for multi-file output (with the \.{@(} feature)
has a special first cross-reference whose \PB{\\{num}} field is \PB{\\{file%
\_flag}}.

\Y\B\4\D\\{file\_flag}\5
$(\T{3}*\\{cite\_flag}{}$)\par
\B\4\D\\{def\_flag}\5
$(\T{2}*\\{cite\_flag}{}$)\par
\B\4\D\\{cite\_flag}\5
\T{10240}\C{ must be strictly larger than \PB{\\{max\_sections}} }\par
\B\4\D\\{xref}\5
\\{equiv\_or\_xref}\par
\Y\B\4\X24:Set initial values\X${}\E{}$\6
$\\{xref\_ptr}\K\\{xmem};{}$\6
\\{init\_node}(\\{name\_dir});\6
${}\\{xref\_switch}\K\\{section\_xref\_switch}\K\T{0};{}$\6
${}\\{xmem}\MG\\{num}\K\T{0}{}$;\C{ sentinel value }\par
\As31, 38, 61, 92, 107, 114, 155, 204, 209, 255\ETs257.
\U2.\fi

\M{25}A new cross-reference for an identifier is formed by calling \PB{\\{new%
\_xref}},
which discards duplicate entries and ignores non-underlined references
to one-letter identifiers or \CEE/'s reserved words.

If the user has sent the \PB{\\{no\_xref}} flag (the \.{-x} option of the
command line),
it is unnecessary to keep track of cross-references for identifiers.
If one were careful, one could probably make more changes around section
115 to avoid a lot of identifier looking up.

\Y\B\4\D\\{append\_xref}$(\|c)$\6
\&{if} ${}(\\{xref\_ptr}\E\\{xmem\_end}){}$\1\5
\\{overflow}(\.{"cross-reference"});\2\6
\&{else} $(\PP\\{xref\_ptr})\MG\\{num}\K{}$\|c\par
\B\4\D\\{no\_xref}\5
$\R{}$\\{make\_xrefs}\par
\B\4\D\\{is\_tiny}$(\|p)$\5
$\\{length}(\|p)\E{}$\T{1}\par
\B\4\D\\{unindexed}$(\|a)$\5
$((\|a)<\\{res\_wd\_end}\W(\|a)\MG\\{ilk}\G\\{custom}{}$)\C{ tells if uses of a
name are to be indexed }\par
\Y\B\4\X8:Predeclaration of procedures\X${}\mathrel+\E{}$\6
\&{static} \&{void} \\{new\_xref}(\&{name\_pointer});\6
\&{static} \&{void} \\{new\_section\_xref}(\&{name\_pointer});\6
\&{static} \&{void} \\{set\_file\_flag}(\&{name\_pointer});\par
\fi

\M{26}\B\1\1\&{static} \&{void} \\{new\_xref}(\&{name\_pointer} \|p)\2\2\6
${}\{{}$\1\6
\&{xref\_pointer} \|q;\C{ pointer to previous cross-reference }\6
\&{sixteen\_bits} \|m${},{}$ \|n;\C{ new and previous cross-reference value }\7
\&{if} (\\{no\_xref})\1\5
\&{return};\2\6
\&{if} ${}((\\{unindexed}(\|p)\V\\{is\_tiny}(\|p))\W\\{xref\_switch}\E\T{0}){}$%
\1\5
\&{return};\2\6
${}\|m\K\\{section\_count}+\\{xref\_switch};{}$\6
${}\\{xref\_switch}\K\T{0};{}$\6
${}\|q\K{}$(\&{xref\_pointer}) \|p${}\MG\\{xref};{}$\6
\&{if} ${}(\|q\I\\{xmem}){}$\5
${}\{{}$\1\6
${}\|n\K\|q\MG\\{num};{}$\6
\&{if} ${}(\|n\E\|m\V\|n\E\|m+\\{def\_flag}){}$\1\5
\&{return};\2\6
\&{else} \&{if} ${}(\|m\E\|n+\\{def\_flag}){}$\5
${}\{{}$\1\6
${}\|q\MG\\{num}\K\|m;{}$\6
\&{return};\6
\4${}\}{}$\2\6
\4${}\}{}$\2\6
\\{append\_xref}(\|m);\6
${}\\{xref\_ptr}\MG\\{xlink}\K\|q;{}$\6
\\{update\_node}(\|p);\6
\4${}\}{}$\2\par
\fi

\M{27}The cross-reference lists for section names are slightly different.
Suppose that a section name is defined in sections $m_1$, \dots,
$m_k$, cited in sections $n_1$, \dots, $n_l$, and used in sections
$p_1$, \dots, $p_j$.  Then its list will contain $m_1+\PB{\\{def\_flag}}$,
\dots, $m_k+\PB{\\{def\_flag}}$, $n_1+\PB{\\{cite\_flag}}$, \dots,
$n_l+\PB{\\{cite\_flag}}$, $p_1$, \dots, $p_j$, in this order.

Although this method of storage takes quadratic time with respect to
the length of the list, under foreseeable uses of \.{CWEAVE} this inefficiency
is insignificant.

\Y\B\1\1\&{static} \&{void} \\{new\_section\_xref}(\&{name\_pointer} \|p)\2\2\6
${}\{{}$\1\6
\&{xref\_pointer} \|q${}\K{}$(\&{xref\_pointer}) \|p${}\MG\\{xref};{}$\6
\&{xref\_pointer} \|r${}\K\\{xmem}{}$;\C{ pointers to previous cross-references
}\7
\&{if} ${}(\|q>\|r){}$\1\6
\&{while} ${}(\|q\MG\\{num}>\\{section\_xref\_switch}){}$\5
${}\{{}$\1\6
${}\|r\K\|q;{}$\6
${}\|q\K\|q\MG\\{xlink};{}$\6
\4${}\}{}$\2\2\6
\&{if} ${}(\|r\MG\\{num}\E\\{section\_count}+\\{section\_xref\_switch}){}$\1\5
\&{return};\C{ don't duplicate entries }\2\6
${}\\{append\_xref}(\\{section\_count}+\\{section\_xref\_switch});{}$\6
${}\\{xref\_ptr}\MG\\{xlink}\K\|q;{}$\6
${}\\{section\_xref\_switch}\K\T{0};{}$\6
\&{if} ${}(\|r\E\\{xmem}){}$\1\5
\\{update\_node}(\|p);\2\6
\&{else}\1\5
${}\|r\MG\\{xlink}\K\\{xref\_ptr};{}$\2\6
\4${}\}{}$\2\par
\fi

\M{28}The cross-reference list for a section name may also begin with
\PB{\\{file\_flag}}. Here's how that flag gets put~in.

\Y\B\1\1\&{static} \&{void} \\{set\_file\_flag}(\&{name\_pointer} \|p)\2\2\6
${}\{{}$\1\6
\&{xref\_pointer} \|q${}\K{}$(\&{xref\_pointer}) \|p${}\MG\\{xref};{}$\7
\&{if} ${}(\|q\MG\\{num}\E\\{file\_flag}){}$\1\5
\&{return};\2\6
\\{append\_xref}(\\{file\_flag});\6
${}\\{xref\_ptr}\MG\\{xlink}\K\|q;{}$\6
\\{update\_node}(\|p);\6
\4${}\}{}$\2\par
\fi

\M{29}A third large area of memory is used for sixteen-bit `tokens', which
appear
in short lists similar to the strings of characters in \PB{\\{byte\_mem}}.
Token lists
are used to contain the result of \CEE/ code translated into \TEX/ form;
further details about them will be explained later. A \&{text\_pointer}
variable is an index into \PB{\\{tok\_start}}.

\Y\B\4\X22:Typedef declarations\X${}\mathrel+\E{}$\6
\&{typedef} \&{sixteen\_bits} \&{token};\6
\&{typedef} \&{token} ${}{*}\&{token\_pointer};{}$\6
\&{typedef} \&{token\_pointer} ${}{*}\&{text\_pointer}{}$;\par
\fi

\M{30}The first position of \PB{\\{tok\_mem}}
that is unoccupied by replacement text is called \PB{\\{tok\_ptr}}, and the
first
unused location of \PB{\\{tok\_start}} is called \PB{\\{text\_ptr}}.
Thus, we usually have \PB{${*}\\{text\_ptr}\E\\{tok\_ptr}$}.

\Y\B\4\D\\{max\_toks}\5
\T{30000}\C{ number of symbols in \CEE/ texts being parsed;   must be less than
65536 }\par
\B\4\D\\{max\_texts}\5
\T{8000}\C{ number of phrases in \CEE/ texts being parsed;   must be less than
10240 }\par
\Y\B\4\X21:Private variables\X${}\mathrel+\E{}$\6
\&{static} \&{token} \\{tok\_mem}[\\{max\_toks}];\C{ tokens }\6
\&{static} \&{token\_pointer} \\{tok\_mem\_end}${}\K\\{tok\_mem}+\\{max\_toks}-%
\T{1}{}$;\C{ end of \PB{\\{tok\_mem}} }\6
\&{static} \&{token\_pointer} \\{tok\_ptr};\C{ first unused position in \PB{%
\\{tok\_mem}} }\6
\&{static} \&{token\_pointer} \\{max\_tok\_ptr};\C{ largest value of \PB{\\{tok%
\_ptr}} }\6
\&{static} \&{token\_pointer} \\{tok\_start}[\\{max\_texts}];\C{ directory into
\PB{\\{tok\_mem}} }\6
\&{static} \&{text\_pointer} \\{tok\_start\_end}${}\K\\{tok\_start}+\\{max%
\_texts}-\T{1}{}$;\C{ end of \PB{\\{tok\_start}} }\6
\&{static} \&{text\_pointer} \\{text\_ptr};\C{ first unused position in \PB{%
\\{tok\_start}} }\6
\&{static} \&{text\_pointer} \\{max\_text\_ptr};\C{ largest value of \PB{%
\\{text\_ptr}} }\par
\fi

\M{31}\B\X24:Set initial values\X${}\mathrel+\E{}$\6
$\\{tok\_ptr}\K\\{max\_tok\_ptr}\K\\{tok\_mem}+\T{1}{}$;\6
${}\\{tok\_start}[\T{0}]\K\\{tok\_start}[\T{1}]\K\\{tok\_mem}+\T{1}{}$;\6
${}\\{text\_ptr}\K\\{max\_text\_ptr}\K\\{tok\_start}+\T{1}{}$;\par
\fi

\M{32}Here are the three procedures needed to complete \PB{\\{id\_lookup}}:
\Y\B\1\1\&{boolean} \\{names\_match}(\&{name\_pointer} \|p${},{}$\C{ points to
the proposed match }\6
\&{const} \&{char} ${}{*}\\{first},{}$\C{ position of first character of string
}\6
\&{size\_t} \|l${},{}$\C{ length of identifier }\6
\&{eight\_bits} \|t)\C{ desired \PB{\\{ilk}} }\2\2\6
${}\{{}$\1\6
\&{if} ${}(\\{length}(\|p)\I\|l){}$\1\5
\&{return} \\{false};\2\6
\&{if} ${}(\|p\MG\\{ilk}\I\|t\W\R(\|t\E\\{normal}\W\\{abnormal}(\|p))){}$\1\5
\&{return} \\{false};\2\6
\&{return} ${}\R\\{strncmp}(\\{first},\39\|p\MG\\{byte\_start},\39\|l);{}$\6
\4${}\}{}$\2\7
\1\1\&{void} \\{init\_p}(\&{name\_pointer} \|p${},\39{}$\&{eight\_bits} \|t)\2%
\2\6
${}\{{}$\1\6
${}\|p\MG\\{ilk}\K\|t;{}$\6
\\{init\_node}(\|p);\6
\4${}\}{}$\2\7
\1\1\&{void} \\{init\_node}(\&{name\_pointer} \|p)\2\2\6
${}\{{}$\1\6
${}\|p\MG\\{xref}\K{}$(\&{void} ${}{*}){}$ \\{xmem};\6
\4${}\}{}$\2\par
\fi

\M{33}And here's a small helper function to simplify the code.

\Y\B\4\D\\{update\_node}$(\|p)$\5
$(\|p)\MG\\{xref}\K{}$(\&{void} ${}{*}){}$ \\{xref\_ptr}\par
\fi

\M{34}We have to get \CEE/'s and \CPLUSPLUS/'s
reserved words into the hash table, and the simplest way to do this is
to insert them every time \.{CWEAVE} is run.  Fortunately there are relatively
few reserved words. (Some of these are not strictly ``reserved,'' but
are defined in header files of the ISO Standard \CEE/ Library.
An ever growing list of \CPLUSPLUS/ keywords can be found here:
\.{https://en.cppreference.com/w/cpp/keyword}.)

\Y\B\4\X34:Store all the reserved words\X${}\E{}$\6
$\\{id\_lookup}(\.{"alignas"},\39\NULL,\39\\{alignas\_like});{}$\6
${}\\{id\_lookup}(\.{"alignof"},\39\NULL,\39\\{sizeof\_like});{}$\6
${}\\{id\_lookup}(\.{"and"},\39\NULL,\39\\{alfop});{}$\6
${}\\{id\_lookup}(\.{"and\_eq"},\39\NULL,\39\\{alfop});{}$\6
${}\\{id\_lookup}(\.{"asm"},\39\NULL,\39\\{sizeof\_like});{}$\6
${}\\{id\_lookup}(\.{"auto"},\39\NULL,\39\\{int\_like});{}$\6
${}\\{id\_lookup}(\.{"bitand"},\39\NULL,\39\\{alfop});{}$\6
${}\\{id\_lookup}(\.{"bitor"},\39\NULL,\39\\{alfop});{}$\6
${}\\{id\_lookup}(\.{"bool"},\39\NULL,\39\\{raw\_int});{}$\6
${}\\{id\_lookup}(\.{"break"},\39\NULL,\39\\{case\_like});{}$\6
${}\\{id\_lookup}(\.{"case"},\39\NULL,\39\\{case\_like});{}$\6
${}\\{id\_lookup}(\.{"catch"},\39\NULL,\39\\{catch\_like});{}$\6
${}\\{id\_lookup}(\.{"char"},\39\NULL,\39\\{raw\_int});{}$\6
${}\\{id\_lookup}(\.{"char8\_t"},\39\NULL,\39\\{raw\_int});{}$\6
${}\\{id\_lookup}(\.{"char16\_t"},\39\NULL,\39\\{raw\_int});{}$\6
${}\\{id\_lookup}(\.{"char32\_t"},\39\NULL,\39\\{raw\_int});{}$\6
${}\\{id\_lookup}(\.{"class"},\39\NULL,\39\\{struct\_like});{}$\6
${}\\{id\_lookup}(\.{"clock\_t"},\39\NULL,\39\\{raw\_int});{}$\6
${}\\{id\_lookup}(\.{"compl"},\39\NULL,\39\\{alfop});{}$\6
${}\\{id\_lookup}(\.{"concept"},\39\NULL,\39\\{int\_like});{}$\6
${}\\{id\_lookup}(\.{"const"},\39\NULL,\39\\{const\_like});{}$\6
${}\\{id\_lookup}(\.{"consteval"},\39\NULL,\39\\{const\_like});{}$\6
${}\\{id\_lookup}(\.{"constexpr"},\39\NULL,\39\\{const\_like});{}$\6
${}\\{id\_lookup}(\.{"constinit"},\39\NULL,\39\\{const\_like});{}$\6
${}\\{id\_lookup}(\.{"const\_cast"},\39\NULL,\39\\{raw\_int});{}$\6
${}\\{id\_lookup}(\.{"continue"},\39\NULL,\39\\{case\_like});{}$\6
${}\\{id\_lookup}(\.{"co\_await"},\39\NULL,\39\\{case\_like});{}$\6
${}\\{id\_lookup}(\.{"co\_return"},\39\NULL,\39\\{case\_like});{}$\6
${}\\{id\_lookup}(\.{"co\_yield"},\39\NULL,\39\\{case\_like});{}$\6
${}\\{id\_lookup}(\.{"decltype"},\39\NULL,\39\\{sizeof\_like});{}$\6
${}\\{id\_lookup}(\.{"default"},\39\NULL,\39\\{default\_like});{}$\6
${}\\{id\_lookup}(\.{"define"},\39\NULL,\39\\{define\_like});{}$\6
${}\\{id\_lookup}(\.{"defined"},\39\NULL,\39\\{sizeof\_like});{}$\6
${}\\{id\_lookup}(\.{"delete"},\39\NULL,\39\\{delete\_like});{}$\6
${}\\{id\_lookup}(\.{"div\_t"},\39\NULL,\39\\{raw\_int});{}$\6
${}\\{id\_lookup}(\.{"do"},\39\NULL,\39\\{do\_like});{}$\6
${}\\{id\_lookup}(\.{"double"},\39\NULL,\39\\{raw\_int});{}$\6
${}\\{id\_lookup}(\.{"dynamic\_cast"},\39\NULL,\39\\{raw\_int});{}$\6
${}\\{id\_lookup}(\.{"elif"},\39\NULL,\39\\{if\_like});{}$\6
${}\\{id\_lookup}(\.{"else"},\39\NULL,\39\\{else\_like});{}$\6
${}\\{id\_lookup}(\.{"endif"},\39\NULL,\39\\{if\_like});{}$\6
${}\\{id\_lookup}(\.{"enum"},\39\NULL,\39\\{struct\_like});{}$\6
${}\\{id\_lookup}(\.{"error"},\39\NULL,\39\\{if\_like});{}$\6
${}\\{id\_lookup}(\.{"explicit"},\39\NULL,\39\\{int\_like});{}$\6
${}\\{id\_lookup}(\.{"export"},\39\NULL,\39\\{int\_like});{}$\6
${}\\{id\_lookup}(\.{"extern"},\39\NULL,\39\\{int\_like});{}$\6
${}\\{id\_lookup}(\.{"FILE"},\39\NULL,\39\\{raw\_int});{}$\6
${}\\{id\_lookup}(\.{"false"},\39\NULL,\39\\{normal});{}$\6
${}\\{id\_lookup}(\.{"float"},\39\NULL,\39\\{raw\_int});{}$\6
${}\\{id\_lookup}(\.{"for"},\39\NULL,\39\\{for\_like});{}$\6
${}\\{id\_lookup}(\.{"fpos\_t"},\39\NULL,\39\\{raw\_int});{}$\6
${}\\{id\_lookup}(\.{"friend"},\39\NULL,\39\\{int\_like});{}$\6
${}\\{id\_lookup}(\.{"goto"},\39\NULL,\39\\{case\_like});{}$\6
${}\\{id\_lookup}(\.{"if"},\39\NULL,\39\\{if\_like});{}$\6
${}\\{id\_lookup}(\.{"ifdef"},\39\NULL,\39\\{if\_like});{}$\6
${}\\{id\_lookup}(\.{"ifndef"},\39\NULL,\39\\{if\_like});{}$\6
${}\\{id\_lookup}(\.{"include"},\39\NULL,\39\\{if\_like});{}$\6
${}\\{id\_lookup}(\.{"inline"},\39\NULL,\39\\{int\_like});{}$\6
${}\\{id\_lookup}(\.{"int"},\39\NULL,\39\\{raw\_int});{}$\6
${}\\{id\_lookup}(\.{"jmp\_buf"},\39\NULL,\39\\{raw\_int});{}$\6
${}\\{id\_lookup}(\.{"ldiv\_t"},\39\NULL,\39\\{raw\_int});{}$\6
${}\\{id\_lookup}(\.{"line"},\39\NULL,\39\\{if\_like});{}$\6
${}\\{id\_lookup}(\.{"long"},\39\NULL,\39\\{raw\_int});{}$\6
${}\\{id\_lookup}(\.{"mutable"},\39\NULL,\39\\{int\_like});{}$\6
${}\\{id\_lookup}(\.{"namespace"},\39\NULL,\39\\{struct\_like});{}$\6
${}\\{id\_lookup}(\.{"new"},\39\NULL,\39\\{new\_like});{}$\6
${}\\{id\_lookup}(\.{"noexcept"},\39\NULL,\39\\{attr});{}$\6
${}\\{id\_lookup}(\.{"not"},\39\NULL,\39\\{alfop});{}$\6
${}\\{id\_lookup}(\.{"not\_eq"},\39\NULL,\39\\{alfop});{}$\6
${}\\{id\_lookup}(\.{"NULL"},\39\NULL,\39\\{custom});{}$\6
${}\\{id\_lookup}(\.{"nullptr"},\39\NULL,\39\\{custom});{}$\6
${}\\{id\_lookup}(\.{"offsetof"},\39\NULL,\39\\{raw\_int});{}$\6
${}\\{id\_lookup}(\.{"operator"},\39\NULL,\39\\{operator\_like});{}$\6
${}\\{id\_lookup}(\.{"or"},\39\NULL,\39\\{alfop});{}$\6
${}\\{id\_lookup}(\.{"or\_eq"},\39\NULL,\39\\{alfop});{}$\6
${}\\{id\_lookup}(\.{"pragma"},\39\NULL,\39\\{if\_like});{}$\6
${}\\{id\_lookup}(\.{"private"},\39\NULL,\39\\{public\_like});{}$\6
${}\\{id\_lookup}(\.{"protected"},\39\NULL,\39\\{public\_like});{}$\6
${}\\{id\_lookup}(\.{"ptrdiff\_t"},\39\NULL,\39\\{raw\_int});{}$\6
${}\\{id\_lookup}(\.{"public"},\39\NULL,\39\\{public\_like});{}$\6
${}\\{id\_lookup}(\.{"register"},\39\NULL,\39\\{int\_like});{}$\6
${}\\{id\_lookup}(\.{"reinterpret\_cast"},\39\NULL,\39\\{raw\_int});{}$\6
${}\\{id\_lookup}(\.{"requires"},\39\NULL,\39\\{int\_like});{}$\6
${}\\{id\_lookup}(\.{"restrict"},\39\NULL,\39\\{int\_like});{}$\6
${}\\{id\_lookup}(\.{"return"},\39\NULL,\39\\{case\_like});{}$\6
${}\\{id\_lookup}(\.{"short"},\39\NULL,\39\\{raw\_int});{}$\6
${}\\{id\_lookup}(\.{"sig\_atomic\_t"},\39\NULL,\39\\{raw\_int});{}$\6
${}\\{id\_lookup}(\.{"signed"},\39\NULL,\39\\{raw\_int});{}$\6
${}\\{id\_lookup}(\.{"size\_t"},\39\NULL,\39\\{raw\_int});{}$\6
${}\\{id\_lookup}(\.{"sizeof"},\39\NULL,\39\\{sizeof\_like});{}$\6
${}\\{id\_lookup}(\.{"static"},\39\NULL,\39\\{int\_like});{}$\6
${}\\{id\_lookup}(\.{"static\_assert"},\39\NULL,\39\\{sizeof\_like});{}$\6
${}\\{id\_lookup}(\.{"static\_cast"},\39\NULL,\39\\{raw\_int});{}$\6
${}\\{id\_lookup}(\.{"struct"},\39\NULL,\39\\{struct\_like});{}$\6
${}\\{id\_lookup}(\.{"switch"},\39\NULL,\39\\{for\_like});{}$\6
${}\\{id\_lookup}(\.{"template"},\39\NULL,\39\\{template\_like});{}$\6
${}\\{id\_lookup}(\.{"this"},\39\NULL,\39\\{custom});{}$\6
${}\\{id\_lookup}(\.{"thread\_local"},\39\NULL,\39\\{raw\_int});{}$\6
${}\\{id\_lookup}(\.{"throw"},\39\NULL,\39\\{case\_like});{}$\6
${}\\{id\_lookup}(\.{"time\_t"},\39\NULL,\39\\{raw\_int});{}$\6
${}\\{id\_lookup}(\.{"true"},\39\NULL,\39\\{normal});{}$\6
${}\\{id\_lookup}(\.{"try"},\39\NULL,\39\\{else\_like});{}$\6
${}\\{id\_lookup}(\.{"typedef"},\39\NULL,\39\\{typedef\_like});{}$\6
${}\\{id\_lookup}(\.{"typeid"},\39\NULL,\39\\{sizeof\_like});{}$\6
${}\\{id\_lookup}(\.{"typename"},\39\NULL,\39\\{struct\_like});{}$\6
${}\\{id\_lookup}(\.{"undef"},\39\NULL,\39\\{if\_like});{}$\6
${}\\{id\_lookup}(\.{"union"},\39\NULL,\39\\{struct\_like});{}$\6
${}\\{id\_lookup}(\.{"unsigned"},\39\NULL,\39\\{raw\_int});{}$\6
${}\\{id\_lookup}(\.{"using"},\39\NULL,\39\\{using\_like}){}$;\6
${}\\{id\_lookup}(\.{"va\_dcl"},\39\NULL,\39\\{decl}){}$;\C{ Berkeley's
variable-arg-list convention }\6
${}\\{id\_lookup}(\.{"va\_list"},\39\NULL,\39\\{raw\_int}){}$;\C{ ditto }\6
${}\\{id\_lookup}(\.{"virtual"},\39\NULL,\39\\{int\_like});{}$\6
${}\\{id\_lookup}(\.{"void"},\39\NULL,\39\\{raw\_int});{}$\6
${}\\{id\_lookup}(\.{"volatile"},\39\NULL,\39\\{const\_like});{}$\6
${}\\{id\_lookup}(\.{"wchar\_t"},\39\NULL,\39\\{raw\_int});{}$\6
${}\\{id\_lookup}(\.{"while"},\39\NULL,\39\\{for\_like});{}$\6
${}\\{id\_lookup}(\.{"xor"},\39\NULL,\39\\{alfop});{}$\6
${}\\{id\_lookup}(\.{"xor\_eq"},\39\NULL,\39\\{alfop}){}$;\5
${}\\{res\_wd\_end}\K\\{name\_ptr};{}$\6
${}\\{id\_lookup}(\.{"TeX"},\39\NULL,\39\\{custom});{}$\6
${}\\{id\_lookup}(\.{"complex"},\39\NULL,\39\\{int\_like});{}$\6
${}\\{id\_lookup}(\.{"imaginary"},\39\NULL,\39\\{int\_like});{}$\6
${}\\{id\_lookup}(\.{"make\_pair"},\39\NULL,\39\\{func\_template}){}$;\par
\U2.\fi

\N{1}{35}Lexical scanning.
Let us now consider the subroutines that read the \.{CWEB} source file
and break it into meaningful units. There are four such procedures:
One simply skips to the next `\.{@\ }' or `\.{@*}' that begins a
section; another passes over the \TEX/ text at the beginning of a
section; the third passes over the \TEX/ text in a \CEE/ comment;
and the last, which is the most interesting, gets the next token of
a \CEE/ text.  They all use the pointers \PB{\\{limit}} and \PB{\\{loc}} into
the line of input currently being studied.

\fi

\M{36}Control codes in \.{CWEB}, which begin with `\.{@}', are converted
into a numeric code designed to simplify \.{CWEAVE}'s logic; for example,
larger numbers are given to the control codes that denote more significant
milestones, and the code of \PB{\\{new\_section}} should be the largest of
all. Some of these numeric control codes take the place of \PB{\&{char}}
control codes that will not otherwise appear in the output of the
scanning routines.

\Y\B\4\D\\{ignore}\5
\T{\~0}\C{ control code of no interest to \.{CWEAVE} }\par
\B\4\D\\{verbatim}\5
\T{\~2}\C{ takes the place of ASCII \.{STX} }\par
\B\4\D\\{begin\_short\_comment}\5
\T{\~3}\C{ \CPLUSPLUS/ short comment }\par
\B\4\D\\{begin\_comment}\5
\.{'\\t'}\C{ tab marks will not appear }\par
\B\4\D\\{underline}\5
\.{'\\n'}\C{ this code will be intercepted without confusion }\par
\B\4\D\\{noop}\5
\T{\~177}\C{ takes the place of ASCII \.{DEL} }\par
\B\4\D\\{xref\_roman}\5
\T{\~203}\C{ control code for `\.{@\^}' }\par
\B\4\D\\{xref\_wildcard}\5
\T{\~204}\C{ control code for `\.{@:}' }\par
\B\4\D\\{xref\_typewriter}\5
\T{\~205}\C{ control code for `\.{@.}' }\par
\B\4\D\TeXxstring\5
\T{\~206}\C{ control code for `\.{@t}' }\par
\B\F\\{TeX\_string}\5
\\{TeX}\par
\B\4\D\\{ord}\5
\T{\~207}\C{ control code for `\.{@'}' }\par
\B\4\D\\{join}\5
\T{\~210}\C{ control code for `\.{@\&}' }\par
\B\4\D\\{thin\_space}\5
\T{\~211}\C{ control code for `\.{@,}' }\par
\B\4\D\\{math\_break}\5
\T{\~212}\C{ control code for `\.{@\v}' }\par
\B\4\D\\{line\_break}\5
\T{\~213}\C{ control code for `\.{@/}' }\par
\B\4\D\\{big\_line\_break}\5
\T{\~214}\C{ control code for `\.{@\#}' }\par
\B\4\D\\{no\_line\_break}\5
\T{\~215}\C{ control code for `\.{@+}' }\par
\B\4\D\\{pseudo\_semi}\5
\T{\~216}\C{ control code for `\.{@;}' }\par
\B\4\D\\{macro\_arg\_open}\5
\T{\~220}\C{ control code for `\.{@[}' }\par
\B\4\D\\{macro\_arg\_close}\5
\T{\~221}\C{ control code for `\.{@]}' }\par
\B\4\D\\{trace}\5
\T{\~222}\C{ control code for `\.{@0}', `\.{@1}' and `\.{@2}' }\par
\B\4\D\\{translit\_code}\5
\T{\~223}\C{ control code for `\.{@l}' }\par
\B\4\D\\{output\_defs\_code}\5
\T{\~224}\C{ control code for `\.{@h}' }\par
\B\4\D\\{format\_code}\5
\T{\~225}\C{ control code for `\.{@f}' and `\.{@s}' }\par
\B\4\D\\{definition}\5
\T{\~226}\C{ control code for `\.{@d}' }\par
\B\4\D\\{begin\_C}\5
\T{\~227}\C{ control code for `\.{@c}' }\par
\B\4\D\\{section\_name}\5
\T{\~230}\C{ control code for `\.{@<}' }\par
\B\4\D\\{new\_section}\5
\T{\~231}\C{ control code for `\.{@\ }' and `\.{@*}' }\par
\fi

\M{37}Control codes are converted to \.{CWEAVE}'s internal
representation by means of the table \PB{\\{ccode}}.

\Y\B\4\X21:Private variables\X${}\mathrel+\E{}$\6
\&{static} \&{eight\_bits} \\{ccode}[\T{256}];\C{ meaning of a char following %
\.{@} }\par
\fi

\M{38}\B\X24:Set initial values\X${}\mathrel+\E{}$\6
${}\{{}$\1\6
\&{int} \|c;\C{ must be \PB{\&{int}} so the \PB{\&{for}} loop will end }\7
\&{for} ${}(\|c\K\T{0};{}$ ${}\|c<\T{256};{}$ ${}\|c\PP){}$\1\5
${}\\{ccode}[\|c]\K\\{ignore};{}$\2\6
\4${}\}{}$\2\6
${}\\{ccode}[\.{'\ '}]\K\\{ccode}[\.{'\\t'}]\K\\{ccode}[\.{'\\n'}]\K\\{ccode}[%
\.{'\\v'}]\K\\{ccode}[\.{'\\r'}]\K\\{ccode}[\.{'\\f'}]\K\\{ccode}[\.{'*'}]\K%
\\{new\_section};{}$\6
${}\\{ccode}[\.{'@'}]\K\.{'@'}{}$;\C{ `quoted' at sign }\6
${}\\{ccode}[\.{'='}]\K\\{verbatim};{}$\6
${}\\{ccode}[\.{'d'}]\K\\{ccode}[\.{'D'}]\K\\{definition};{}$\6
${}\\{ccode}[\.{'f'}]\K\\{ccode}[\.{'F'}]\K\\{ccode}[\.{'s'}]\K\\{ccode}[%
\.{'S'}]\K\\{format\_code};{}$\6
${}\\{ccode}[\.{'c'}]\K\\{ccode}[\.{'C'}]\K\\{ccode}[\.{'p'}]\K\\{ccode}[%
\.{'P'}]\K\\{begin\_C};{}$\6
${}\\{ccode}[\.{'t'}]\K\\{ccode}[\.{'T'}]\K\TeXxstring;{}$\6
${}\\{ccode}[\.{'l'}]\K\\{ccode}[\.{'L'}]\K\\{translit\_code};{}$\6
${}\\{ccode}[\.{'q'}]\K\\{ccode}[\.{'Q'}]\K\\{noop};{}$\6
${}\\{ccode}[\.{'h'}]\K\\{ccode}[\.{'H'}]\K\\{output\_defs\_code};{}$\6
${}\\{ccode}[\.{'\&'}]\K\\{join};{}$\6
${}\\{ccode}[\.{'<'}]\K\\{ccode}[\.{'('}]\K\\{section\_name};{}$\6
${}\\{ccode}[\.{'!'}]\K\\{underline};{}$\6
${}\\{ccode}[\.{'\^'}]\K\\{xref\_roman};{}$\6
${}\\{ccode}[\.{':'}]\K\\{xref\_wildcard};{}$\6
${}\\{ccode}[\.{'.'}]\K\\{xref\_typewriter};{}$\6
${}\\{ccode}[\.{','}]\K\\{thin\_space};{}$\6
${}\\{ccode}[\.{'|'}]\K\\{math\_break};{}$\6
${}\\{ccode}[\.{'/'}]\K\\{line\_break};{}$\6
${}\\{ccode}[\.{'\#'}]\K\\{big\_line\_break};{}$\6
${}\\{ccode}[\.{'+'}]\K\\{no\_line\_break};{}$\6
${}\\{ccode}[\.{';'}]\K\\{pseudo\_semi};{}$\6
${}\\{ccode}[\.{'['}]\K\\{macro\_arg\_open};{}$\6
${}\\{ccode}[\.{']'}]\K\\{macro\_arg\_close};{}$\6
${}\\{ccode}[\.{'\\''}]\K\\{ord};{}$\6
\X39:Special control codes for debugging\X\par
\fi

\M{39}Users can write
\.{@2}, \.{@1}, and \.{@0} to turn tracing \PB{\\{fully}} on, \PB{\\{partly}}
on,
and \PB{\\{off}}, respectively.

\Y\B\4\X39:Special control codes for debugging\X${}\E{}$\6
$\\{ccode}[\.{'0'}]\K\\{ccode}[\.{'1'}]\K\\{ccode}[\.{'2'}]\K\\{trace}{}$;\par
\U38.\fi

\M{40}The \PB{\\{skip\_limbo}} routine is used on the first pass to skip
through
portions of the input that are not in any sections, i.e., that precede
the first section. After this procedure has been called, the value of
\PB{\\{input\_has\_ended}} will tell whether or not a section has actually been
found.

There's a complication that we will postpone until later: If the \.{@s}
operation appears in limbo, we want to use it to adjust the default
interpretation of identifiers.

\Y\B\4\X8:Predeclaration of procedures\X${}\mathrel+\E{}$\6
\&{static} \&{void} \\{skip\_limbo}(\&{void});\6
\&{static} \&{eight\_bits} ${}\skipxTeX(\&{void}){}$;\par
\fi

\M{41}\B\1\1\&{static} \&{void} \\{skip\_limbo}(\&{void})\2\2\6
${}\{{}$\1\6
\&{while} (\\{true})\5
${}\{{}$\1\6
\&{if} ${}(\\{loc}>\\{limit}\W\\{get\_line}(\,)\E\\{false}){}$\1\5
\&{return};\2\6
${}{*}(\\{limit}+\T{1})\K\.{'@'};{}$\6
\&{while} ${}({*}\\{loc}\I\.{'@'}){}$\1\5
${}\\{loc}\PP{}$;\C{ look for `\.{@}', then skip two chars }\2\6
\&{if} ${}(\\{loc}\PP\Z\\{limit}){}$\1\6
\&{switch} (\\{ccode}[(\&{eight\_bits}) ${}{*}\\{loc}\PP]){}$\5
${}\{{}$\1\6
\4\&{case} \\{new\_section}:\5
\&{return};\6
\4\&{case} \\{noop}:\5
\\{skip\_restricted}(\,);\6
\&{break};\6
\4\&{case} \\{format\_code}:\5
\X79:Process simple format in limbo\X\6
\4${}\}{}$\2\2\6
\4${}\}{}$\2\6
\4${}\}{}$\2\par
\fi

\M{42}The \PB{$\skipxTeX$} routine is used on the first pass to skip through
the \TEX/ code at the beginning of a section. It returns the next
control code or `\.{\v}' found in the input. A \PB{\\{new\_section}} is
assumed to exist at the very end of the file.

\Y\B\F\\{skip\_TeX}\5
\\{TeX}\par
\Y\B\1\1\&{static} \&{eight\_bits} ${}\skipxTeX{}$(\&{void})\C{ skip past pure %
\TEX/ code }\2\2\6
${}\{{}$\1\6
\&{while} (\\{true})\5
${}\{{}$\1\6
\&{if} ${}(\\{loc}>\\{limit}\W\\{get\_line}(\,)\E\\{false}){}$\1\5
\&{return} \\{new\_section};\2\6
${}{*}(\\{limit}+\T{1})\K\.{'@'};{}$\6
\&{while} ${}({*}\\{loc}\I\.{'@'}\W{*}\\{loc}\I\.{'|'}){}$\1\5
${}\\{loc}\PP;{}$\2\6
\&{if} ${}({*}\\{loc}\PP\E\.{'|'}){}$\1\5
\&{return} (\&{eight\_bits}) \.{'|'};\2\6
\&{if} ${}(\\{loc}\Z\\{limit}){}$\1\5
\&{return} \\{ccode}[(\&{eight\_bits}) ${}{*}(\\{loc}\PP)];{}$\2\6
\4${}\}{}$\2\6
\4${}\}{}$\2\par
\fi

\N{2}{43}Inputting the next token.
As stated above, \.{CWEAVE}'s most interesting lexical scanning routine is the
\PB{\\{get\_next}} function that inputs the next token of \CEE/ input. However,
\PB{\\{get\_next}} is not especially complicated.

The result of \PB{\\{get\_next}} is either a \PB{\&{char}} code for some
special character,
or it is a special code representing a pair of characters (e.g., `\.{!=}'),
or it is the numeric value computed by the \PB{\\{ccode}}
table, or it is one of the following special codes:

\yskip\hang \PB{\\{identifier}}: In this case the global variables \PB{\\{id%
\_first}} and
\PB{\\{id\_loc}} will have been set to the beginning and ending-plus-one
locations
in the buffer, as required by the \PB{\\{id\_lookup}} routine.

\yskip\hang \PB{\\{string}}: The string will have been copied into the array
\PB{\\{section\_text}}; \PB{\\{id\_first}} and \PB{\\{id\_loc}} are set as
above (now they are
pointers into \PB{\\{section\_text}}).

\yskip\hang \PB{\\{constant}}: The constant is copied into \PB{\\{section%
\_text}}, with
slight modifications; \PB{\\{id\_first}} and \PB{\\{id\_loc}} are set.

\yskip\noindent Furthermore, some of the control codes cause
\PB{\\{get\_next}} to take additional actions:

\yskip\hang \PB{\\{xref\_roman}}, \PB{\\{xref\_wildcard}}, \PB{\\{xref%
\_typewriter}}, \PB{$\TeXxstring$},
\PB{\\{verbatim}}: The values of \PB{\\{id\_first}} and \PB{\\{id\_loc}} will
have been set to
the beginning and ending-plus-one locations in the buffer.

\yskip\hang \PB{\\{section\_name}}: In this case the global variable \PB{\\{cur%
\_section}} will
point to the \PB{\\{byte\_start}} entry for the section name that has just been
scanned.
The value of \PB{\\{cur\_section\_char}} will be \PB{\.{'('}} if the section
name was
preceded by \.{@(} instead of \.{@<}.

\yskip\noindent If \PB{\\{get\_next}} sees `\.{@!}'
it sets \PB{\\{xref\_switch}} to \PB{\\{def\_flag}} and goes on to the next
token.

\Y\B\4\D\\{constant}\5
\T{\~200}\C{ \CEE/ constant }\par
\B\4\D\\{string}\5
\T{\~201}\C{ \CEE/ string }\par
\B\4\D\\{identifier}\5
\T{\~202}\C{ \CEE/ identifier or reserved word }\par
\Y\B\4\X21:Private variables\X${}\mathrel+\E{}$\6
\&{static} \&{name\_pointer} \\{cur\_section};\C{ name of section just scanned
}\6
\&{static} \&{char} \\{cur\_section\_char};\C{ the character just before that
name }\par
\fi

\M{44}As one might expect, \PB{\\{get\_next}} consists mostly of a big switch
that branches to the various special cases that can arise.

\Y\B\1\1\&{static} \&{eight\_bits} \\{get\_next}(\&{void})\C{ produces the next
input token }\2\2\6
${}\{{}$\1\6
\&{eight\_bits} \|c;\C{ the current character }\7
\&{while} (\\{true})\5
${}\{{}$\1\6
\X50:Check if we're at the end of a preprocessor command\X\6
\&{if} ${}(\\{loc}>\\{limit}\W\\{get\_line}(\,)\E\\{false}){}$\1\5
\&{return} \\{new\_section};\2\6
${}\|c\K{*}(\\{loc}\PP);{}$\6
\&{if} (\\{xisdigit}((\&{int}) \|c)${}\V\|c\E\.{'.'}){}$\1\5
\X53:Get a constant\X\2\6
\&{else} \&{if} ${}(\|c\E\.{'\\''}\V\|c\E\.{'"'}\3{-1}\V((\|c\E\.{'L'}\V\|c\E%
\.{'u'}\V\|c\E\.{'U'})\W({*}\\{loc}\E\.{'\\''}\V{*}\\{loc}\E\.{'"'}))\3{-1}\V((%
\|c\E\.{'u'}\W{*}\\{loc}\E\.{'8'})\W({*}(\\{loc}+\T{1})\E\.{'\\''}\V{*}(%
\\{loc}+\T{1})\E\.{'"'}))\3{-1}\V(\|c\E\.{'<'}\W\\{sharp\_include\_line}\E%
\\{true})){}$\1\5
\X57:Get a string\X\2\6
\&{else} \&{if} (\\{isalpha}((\&{int}) \|c)${}\V\\{isxalpha}(\|c)\V\\{ishigh}(%
\|c)){}$\1\5
\X52:Get an identifier\X\2\6
\&{else} \&{if} ${}(\|c\E\.{'@'}){}$\1\5
\X59:Get control code and possible section name\X\2\6
\&{else} \&{if} (\\{xisspace}(\|c))\1\5
\&{continue};\C{ ignore spaces and tabs }\2\6
\&{if} ${}(\|c\E\.{'\#'}\W\\{loc}\E\\{buffer}+\T{1}){}$\1\5
\X47:Raise preprocessor flag\X\2\6
\4\\{mistake}:\5
\X51:Compress two-symbol operator\X\6
\&{return} \|c;\6
\4${}\}{}$\2\6
\4${}\}{}$\2\par
\fi

\M{45}\B\X8:Predeclaration of procedures\X${}\mathrel+\E{}$\5
\&{static} \&{eight\_bits} \\{get\_next}(\&{void});\par
\fi

\M{46}Because preprocessor commands do not fit in with the rest of the syntax
of \CEE/,
we have to deal with them separately.  One solution is to enclose such
commands between special markers.  Thus, when a \.\# is seen as the
first character of a line, \PB{\\{get\_next}} returns a special code
\PB{\\{left\_preproc}} and raises a flag \PB{\\{preprocessing}}.

We can use the same internal code number for \PB{\\{left\_preproc}} as we do
for \PB{\\{ord}}, since \PB{\\{get\_next}} changes \PB{\\{ord}} into a string.

\Y\B\4\D\\{left\_preproc}\5
\\{ord}\C{ begins a preprocessor command }\par
\B\4\D\\{right\_preproc}\5
\T{\~217}\C{ ends a preprocessor command }\par
\Y\B\4\X21:Private variables\X${}\mathrel+\E{}$\6
\&{static} \&{boolean} \\{preprocessing}${}\K\\{false}{}$;\C{ are we scanning a
preprocessor command? }\par
\fi

\M{47}\B\X47:Raise preprocessor flag\X${}\E{}$\6
${}\{{}$\1\6
${}\\{preprocessing}\K\\{true};{}$\6
\X49:Check if next token is \PB{\&{include}}\X\6
\&{return} \\{left\_preproc};\6
\4${}\}{}$\2\par
\U44.\fi

\M{48}An additional complication is the freakish use of \.< and \.> to delimit
a file name in lines that start with \.{\#include}.  We must treat this file
name as a string.

\Y\B\4\X21:Private variables\X${}\mathrel+\E{}$\6
\&{static} \&{boolean} \\{sharp\_include\_line}${}\K\\{false}{}$;\C{ are we
scanning a \#\&{include} line? }\par
\fi

\M{49}\B\X49:Check if next token is \PB{\&{include}}\X${}\E{}$\6
\&{while} ${}(\\{loc}\Z\\{buffer\_end}-\T{7}\W\\{xisspace}({*}\\{loc})){}$\1\5
${}\\{loc}\PP;{}$\2\6
\&{if} ${}(\\{loc}\Z\\{buffer\_end}-\T{6}\W\\{strncmp}(\\{loc},\39%
\.{"include"},\39\T{7})\E\T{0}){}$\1\5
${}\\{sharp\_include\_line}\K\\{true}{}$;\2\par
\U47.\fi

\M{50}When we get to the end of a preprocessor line,
we lower the flag and send a code \PB{\\{right\_preproc}}, unless
the last character was a \.\\.

\Y\B\4\X50:Check if we're at the end of a preprocessor command\X${}\E{}$\6
\&{while} ${}(\\{loc}\E\\{limit}-\T{1}\W\\{preprocessing}\W{*}\\{loc}\E\.{'%
\\\\'}){}$\1\6
\&{if} ${}(\\{get\_line}(\,)\E\\{false}){}$\1\5
\&{return} \\{new\_section};\C{ still in preprocessor mode }\2\2\6
\&{if} ${}(\\{loc}\G\\{limit}\W\\{preprocessing}){}$\5
${}\{{}$\1\6
${}\\{preprocessing}\K\\{sharp\_include\_line}\K\\{false};{}$\6
\&{return} \\{right\_preproc};\6
\4${}\}{}$\2\par
\U44.\fi

\M{51}The following code assigns values to the combinations \.{++},
\.{--}, \.{->}, \.{>=}, \.{<=}, \.{==}, \.{<<}, \.{>>}, \.{!=}, %\.{\PB{}}
\.{\v\v} and~\.{\&\&}, and to the \CPLUSPLUS/
combinations \.{...}, \.{::}, \.{.*} and \.{->*}.
The compound assignment operators (e.g., \.{+=}) are
treated as separate tokens.

\Y\B\4\X51:Compress two-symbol operator\X${}\E{}$\6
\&{switch} (\|c)\5
${}\{{}$\1\6
\4\&{case} \.{'/'}:\6
\&{if} ${}({*}\\{loc}\E\.{'*'}){}$\5
${}\{{}$\5
\1\\{compress}(\\{begin\_comment});\5
${}\}{}$\2\6
\&{else} \&{if} ${}({*}\\{loc}\E\.{'/'}){}$\1\5
\\{compress}(\\{begin\_short\_comment});\2\6
\&{break};\6
\4\&{case} \.{'+'}:\6
\&{if} ${}({*}\\{loc}\E\.{'+'}){}$\1\5
\\{compress}(\\{plus\_plus});\2\6
\&{break};\6
\4\&{case} \.{'-'}:\6
\&{if} ${}({*}\\{loc}\E\.{'-'}){}$\5
${}\{{}$\5
\1\\{compress}(\\{minus\_minus});\5
${}\}{}$\2\6
\&{else} \&{if} ${}({*}\\{loc}\E\.{'>'}){}$\5
${}\{{}$\1\6
\&{if} ${}({*}(\\{loc}+\T{1})\E\.{'*'}){}$\5
${}\{{}$\1\6
${}\\{loc}\PP{}$;\5
\\{compress}(\\{minus\_gt\_ast});\6
\4${}\}{}$\2\6
\&{else}\1\5
\\{compress}(\\{minus\_gt});\2\6
\4${}\}{}$\2\6
\&{break};\6
\4\&{case} \.{'.'}:\6
\&{if} ${}({*}\\{loc}\E\.{'*'}){}$\5
${}\{{}$\5
\1\\{compress}(\\{period\_ast});\5
${}\}{}$\2\6
\&{else} \&{if} ${}({*}\\{loc}\E\.{'.'}\W{*}(\\{loc}+\T{1})\E\.{'.'}){}$\5
${}\{{}$\1\6
${}\\{loc}\PP{}$;\5
\\{compress}(\\{dot\_dot\_dot});\6
\4${}\}{}$\2\6
\&{break};\6
\4\&{case} \.{':'}:\6
\&{if} ${}({*}\\{loc}\E\.{':'}){}$\1\5
\\{compress}(\\{colon\_colon});\2\6
\&{break};\6
\4\&{case} \.{'='}:\6
\&{if} ${}({*}\\{loc}\E\.{'='}){}$\1\5
\\{compress}(\\{eq\_eq});\2\6
\&{break};\6
\4\&{case} \.{'>'}:\6
\&{if} ${}({*}\\{loc}\E\.{'='}){}$\5
${}\{{}$\5
\1\\{compress}(\\{gt\_eq});\5
${}\}{}$\2\6
\&{else} \&{if} ${}({*}\\{loc}\E\.{'>'}){}$\1\5
\\{compress}(\\{gt\_gt});\2\6
\&{break};\6
\4\&{case} \.{'<'}:\6
\&{if} ${}({*}\\{loc}\E\.{'='}){}$\5
${}\{{}$\5
\1\\{compress}(\\{lt\_eq});\5
${}\}{}$\2\6
\&{else} \&{if} ${}({*}\\{loc}\E\.{'<'}){}$\1\5
\\{compress}(\\{lt\_lt});\2\6
\&{break};\6
\4\&{case} \.{'\&'}:\6
\&{if} ${}({*}\\{loc}\E\.{'\&'}){}$\1\5
\\{compress}(\\{and\_and});\2\6
\&{break};\6
\4\&{case} \.{'|'}:\6
\&{if} ${}({*}\\{loc}\E\.{'|'}){}$\1\5
\\{compress}(\\{or\_or});\2\6
\&{break};\6
\4\&{case} \.{'!'}:\6
\&{if} ${}({*}\\{loc}\E\.{'='}){}$\1\5
\\{compress}(\\{non\_eq});\2\6
\&{break};\6
\4${}\}{}$\2\par
\U44.\fi

\M{52}\B\X52:Get an identifier\X${}\E{}$\6
${}\{{}$\1\6
${}\\{id\_first}\K\MM\\{loc};{}$\6
\&{do}\5
${}\PP\\{loc};{}$\5
\&{while} (\\{isalpha}((\&{int}) ${}{*}\\{loc})\V\\{isdigit}{}$((\&{int})
${}{*}\\{loc})\3{-1}\V\\{isxalpha}({*}\\{loc})\V\\{ishigh}({*}\\{loc}));{}$\6
${}\\{id\_loc}\K\\{loc};{}$\6
\&{return} \\{identifier};\6
\4${}\}{}$\2\par
\U44.\fi

\M{53}Different conventions are followed by \TEX/ and \CEE/ to express octal
and hexadecimal numbers; it is reasonable to stick to each convention
within its realm.  Thus the \CEE/ part of a \.{CWEB} file has octals
introduced by \.0 and hexadecimals by \.{0x}, but \.{CWEAVE} will print
with \TEX/ macros that the user can redefine to fit the context.
In order to simplify such macros, we replace some of the characters.

On output, the \.{\ } that replaces \.{'} in \CPLUSPLUS/ literals will become
``\.{\\\ }''.

Notice that in this section and the next, \PB{\\{id\_first}} and \PB{\\{id%
\_loc}}
are pointers into the array \PB{\\{section\_text}}, not into \PB{\\{buffer}}.

\Y\B\4\D\\{gather\_digits\_while}$(\|t)$\5
\&{while} ${}((\|t)\V{*}\\{loc}\E\.{'\\''})$ \6
\&{if} ${}({*}\\{loc}\E\.{'\\''}){}$\5
${}\{{}$\C{ \CPLUSPLUS/-style digit separator }\1\6
${}{*}\\{id\_loc}\PP\K\.{'\ '}{}$;\5
${}\\{loc}\PP{}$;\C{ insert a little white space }\6
\4${}\}{}$\5
\2\&{else} ${*}\\{id\_loc}\PP\K{*}\\{loc}\PP{}$\par
\Y\B\4\X53:Get a constant\X${}\E{}$\6
${}\{{}$\1\6
${}\\{id\_first}\K\\{id\_loc}\K\\{section\_text}+\T{1};{}$\6
\&{if} ${}({*}(\\{loc}-\T{1})\E\.{'.'}\W\R\\{xisdigit}({*}\\{loc})){}$\1\5
\&{goto} \\{mistake};\C{ not a constant }\2\6
\&{if} ${}({*}(\\{loc}-\T{1})\E\.{'0'}){}$\5
${}\{{}$\1\6
\&{if} ${}({*}\\{loc}\E\.{'x'}\V{*}\\{loc}\E\.{'X'}){}$\1\5
\X54:Get a hexadecimal constant\X\2\6
\&{else} \&{if} ${}({*}\\{loc}\E\.{'b'}\V{*}\\{loc}\E\.{'B'}){}$\1\5
\X55:Get a binary constant\X\2\6
\&{else} \&{if} ${}(\\{xisdigit}({*}\\{loc})){}$\1\5
\X56:Get an octal constant\X\2\6
\4${}\}{}$\2\6
${}{*}\\{id\_loc}\PP\K{*}(\\{loc}-\T{1}){}$;\C{ decimal constant }\6
${}\\{gather\_digits\_while}(\\{xisdigit}({*}\\{loc})\V{*}\\{loc}\E\.{'.'});{}$%
\6
\4\\{get\_exponent}:\6
\&{if} ${}({*}\\{loc}\E\.{'e'}\V{*}\\{loc}\E\.{'E'}){}$\1\5
${}{*}\\{id\_loc}\PP\K\.{'\_'};{}$\2\6
\&{else} \&{if} ${}({*}\\{loc}\E\.{'p'}\V{*}\\{loc}\E\.{'P'}){}$\1\5
${}{*}\\{id\_loc}\PP\K\.{'\%'};{}$\2\6
\&{else}\1\5
\&{goto} \\{digit\_suffix};\2\6
${}\\{loc}\PP;{}$\6
\&{if} ${}({*}\\{loc}\E\.{'+'}\V{*}\\{loc}\E\.{'-'}){}$\1\5
${}{*}\\{id\_loc}\PP\K{*}\\{loc}\PP;{}$\2\6
${}\\{gather\_digits\_while}(\\{xisdigit}({*}\\{loc}));{}$\6
\4\\{digit\_suffix}:\6
\&{while} ${}({*}\\{loc}\E\.{'u'}\V{*}\\{loc}\E\.{'U'}\V{*}\\{loc}\E\.{'l'}%
\V{*}\\{loc}\E\.{'L'}\V{*}\\{loc}\E\.{'f'}\V{*}\\{loc}\E\.{'F'}){}$\5
${}\{{}$\1\6
${}{*}\\{id\_loc}\PP\K\.{'\$'};{}$\6
${}{*}\\{id\_loc}\PP\K\\{toupper}{}$((\&{int}) ${}{*}\\{loc});{}$\6
${}\\{loc}\PP;{}$\6
\4${}\}{}$\2\6
\&{return} \\{constant};\6
\4${}\}{}$\2\par
\U44.\fi

\M{54}\B\X54:Get a hexadecimal constant\X${}\E{}$\6
${}\{{}$\1\6
${}{*}\\{id\_loc}\PP\K\.{'\^'};{}$\6
${}\\{loc}\PP;{}$\6
${}\\{gather\_digits\_while}(\\{xisxdigit}({*}\\{loc})\V{*}\\{loc}\E%
\.{'.'});{}$\6
\&{goto} \\{get\_exponent};\6
\4${}\}{}$\2\par
\U53.\fi

\M{55}\B\X55:Get a binary constant\X${}\E{}$\6
${}\{{}$\1\6
${}{*}\\{id\_loc}\PP\K\.{'\\\\'};{}$\6
${}\\{loc}\PP;{}$\6
${}\\{gather\_digits\_while}({*}\\{loc}\E\.{'0'}\V{*}\\{loc}\E\.{'1'});{}$\6
\&{goto} \\{digit\_suffix};\6
\4${}\}{}$\2\par
\U53.\fi

\M{56}\B\X56:Get an octal constant\X${}\E{}$\6
${}\{{}$\1\6
${}{*}\\{id\_loc}\PP\K\.{'\~'};{}$\6
${}\\{gather\_digits\_while}(\\{xisdigit}({*}\\{loc}));{}$\6
\&{goto} \\{digit\_suffix};\6
\4${}\}{}$\2\par
\U53.\fi

\M{57}\CEE/ strings and character constants, delimited by double and single
quotes, respectively, can contain newlines or instances of their own
delimiters if they are protected by a backslash.  We follow this
convention, but do not allow the string to be longer than \PB{\\{longest%
\_name}}.

\Y\B\4\X57:Get a string\X${}\E{}$\6
${}\{{}$\5
\1\&{char} \\{delim}${}\K\|c{}$;\C{ what started the string }\7
${}\\{id\_first}\K\\{section\_text}+\T{1};{}$\6
${}\\{id\_loc}\K\\{section\_text};{}$\6
\&{if} ${}(\\{delim}\E\.{'\\''}\W{*}(\\{loc}-\T{2})\E\.{'@'}){}$\5
${}\{{}$\1\6
${}{*}\PP\\{id\_loc}\K\.{'@'};{}$\6
${}{*}\PP\\{id\_loc}\K\.{'@'};{}$\6
\4${}\}{}$\2\6
${}{*}\PP\\{id\_loc}\K\\{delim};{}$\6
\&{if} ${}(\\{delim}\E\.{'L'}\V\\{delim}\E\.{'u'}\V\\{delim}\E\.{'U'}){}$\1\5
\X58:Get a wide character constant\X\2\6
\&{if} ${}(\\{delim}\E\.{'<'}){}$\1\5
${}\\{delim}\K\.{'>'}{}$;\C{ for file names in \#\&{include} lines }\2\6
\&{while} (\\{true})\5
${}\{{}$\1\6
\&{if} ${}(\\{loc}\G\\{limit}){}$\5
${}\{{}$\1\6
\&{if} ${}({*}(\\{limit}-\T{1})\I\.{'\\\\'}){}$\5
${}\{{}$\1\6
\\{err\_print}(\.{"!\ String\ didn't\ end}\)\.{"});\6
${}\\{loc}\K\\{limit};{}$\6
\&{break};\6
\4${}\}{}$\2\6
\&{if} ${}(\\{get\_line}(\,)\E\\{false}){}$\5
${}\{{}$\1\6
\\{err\_print}(\.{"!\ Input\ ended\ in\ mi}\)\.{ddle\ of\ string"});\6
${}\\{loc}\K\\{buffer};{}$\6
\&{break};\6
\4${}\}{}$\2\6
\4${}\}{}$\2\6
\&{if} ${}((\|c\K{*}\\{loc}\PP)\E\\{delim}){}$\5
${}\{{}$\1\6
\&{if} ${}(\PP\\{id\_loc}\Z\\{section\_text\_end}){}$\1\5
${}{*}\\{id\_loc}\K\|c;{}$\2\6
\&{break};\6
\4${}\}{}$\2\6
\&{if} ${}(\|c\E\.{'\\\\'}){}$\5
${}\{{}$\1\6
\&{if} ${}(\\{loc}\G\\{limit}){}$\1\5
\&{continue};\2\6
\&{else}\5
${}\{{}$\1\6
\&{if} ${}(\PP\\{id\_loc}\Z\\{section\_text\_end}){}$\5
${}\{{}$\1\6
${}{*}\\{id\_loc}\K\.{'\\\\'};{}$\6
${}\|c\K{*}\\{loc}\PP;{}$\6
\4${}\}{}$\2\6
\4${}\}{}$\2\6
\4${}\}{}$\2\6
\&{if} ${}(\PP\\{id\_loc}\Z\\{section\_text\_end}){}$\1\5
${}{*}\\{id\_loc}\K\|c;{}$\2\6
\4${}\}{}$\2\6
\&{if} ${}(\\{id\_loc}\G\\{section\_text\_end}){}$\5
${}\{{}$\1\6
${}\\{fputs}(\.{"\\n!\ String\ too\ long}\)\.{:\ "},\39\\{stdout});{}$\6
${}\\{term\_write}(\\{section\_text}+\T{1},\39\T{25});{}$\6
\\{printf}(\.{"..."});\6
\\{mark\_error};\6
\4${}\}{}$\2\6
${}\\{id\_loc}\PP;{}$\6
\&{return} \\{string};\6
\4${}\}{}$\2\par
\Us44\ET59.\fi

\M{58}\B\X58:Get a wide character constant\X${}\E{}$\6
${}\{{}$\1\6
\&{if} ${}(\\{delim}\E\.{'u'}\W{*}\\{loc}\E\.{'8'}){}$\1\5
${}{*}\PP\\{id\_loc}\K{*}\\{loc}\PP;{}$\2\6
${}\\{delim}\K{*}\\{loc}\PP;{}$\6
${}{*}\PP\\{id\_loc}\K\\{delim};{}$\6
\4${}\}{}$\2\par
\U57.\fi

\M{59}After an \.{@} sign has been scanned, the next character tells us
whether there is more work to do.

\Y\B\4\X59:Get control code and possible section name\X${}\E{}$\6
\&{switch} ${}(\\{ccode}[\|c\K{*}\\{loc}\PP]){}$\5
${}\{{}$\1\6
\4\&{case} \\{translit\_code}:\5
\\{err\_print}(\.{"!\ Use\ @l\ in\ limbo\ o}\)\.{nly"});\6
\&{continue};\6
\4\&{case} \\{underline}:\5
${}\\{xref\_switch}\K\\{def\_flag};{}$\6
\&{continue};\6
\4\&{case} \\{trace}:\5
${}\\{tracing}\K\|c-\.{'0'};{}$\6
\&{continue};\6
\4\&{case} \\{section\_name}:\5
\X60:Scan the section name and make \PB{\\{cur\_section}} point to it\X\6
\4\&{case} \\{verbatim}:\5
\X66:Scan a verbatim string\X\6
\4\&{case} \\{ord}:\5
\X57:Get a string\X\6
\4\&{case} \\{xref\_roman}:\5
\&{case} \\{xref\_wildcard}:\5
\&{case} \\{xref\_typewriter}:\5
\&{case} \\{noop}:\5
\&{case} ${}\TeXxstring{}$:\5
\\{skip\_restricted}(\,);\C{ fall through }\6
\4\&{default}:\5
\&{return} \\{ccode}[\|c];\6
\4${}\}{}$\2\par
\U44.\fi

\M{60}The occurrence of a section name sets \PB{\\{xref\_switch}} to zero,
because the section name might (for example) follow \&{int}.

\Y\B\4\X60:Scan the section name and make \PB{\\{cur\_section}} point to it%
\X${}\E{}$\6
${}\{{}$\1\6
\&{char} ${}{*}\|k\K\\{section\_text}{}$;\C{ pointer into \PB{\\{section%
\_text}} }\7
${}\\{cur\_section\_char}\K{*}(\\{loc}-\T{1});{}$\6
\X62:Put section name into \PB{\\{section\_text}}\X\6
\&{if} ${}(\|k-\\{section\_text}>\T{3}\W\\{strncmp}(\|k-\T{2},\39\.{"..."},\39%
\T{3})\E\T{0}){}$\1\5
${}\\{cur\_section}\K\\{section\_lookup}(\\{section\_text}+\T{1},\39\|k-\T{3},%
\39\\{true}){}$;\C{ \PB{\\{true}} indicates a prefix }\2\6
\&{else}\1\5
${}\\{cur\_section}\K\\{section\_lookup}(\\{section\_text}+\T{1},\39\|k,\39%
\\{false});{}$\2\6
${}\\{xref\_switch}\K\T{0};{}$\6
\&{return} \\{section\_name};\6
\4${}\}{}$\2\par
\U59.\fi

\M{61}Section names are placed into the \PB{\\{section\_text}} array with
consecutive spaces,
tabs, and carriage-returns replaced by single spaces. There will be no
spaces at the beginning or the end. (We set \PB{$\\{section\_text}[\T{0}]\K\.{'%
\ '}$} to facilitate
this, since the \PB{\\{section\_lookup}} routine uses \PB{\\{section\_text}[%
\T{1}]} as the first
character of the name.)

\Y\B\4\X24:Set initial values\X${}\mathrel+\E{}$\6
$\\{section\_text}[\T{0}]\K\.{'\ '}{}$;\par
\fi

\M{62}\B\X62:Put section name into \PB{\\{section\_text}}\X${}\E{}$\6
\&{while} (\\{true})\5
${}\{{}$\1\6
\&{if} ${}(\\{loc}>\\{limit}\W\\{get\_line}(\,)\E\\{false}){}$\5
${}\{{}$\1\6
\\{err\_print}(\.{"!\ Input\ ended\ in\ se}\)\.{ction\ name"});\6
${}\\{loc}\K\\{buffer}+\T{1};{}$\6
\&{break};\6
\4${}\}{}$\2\6
${}\|c\K{*}\\{loc};{}$\6
\X63:If end of name or erroneous control code, \PB{\&{break}}\X\6
${}\\{loc}\PP;{}$\6
\&{if} ${}(\|k<\\{section\_text\_end}){}$\1\5
${}\|k\PP;{}$\2\6
\&{if} (\\{xisspace}(\|c))\5
${}\{{}$\1\6
${}\|c\K\.{'\ '};{}$\6
\&{if} ${}({*}(\|k-\T{1})\E\.{'\ '}){}$\1\5
${}\|k\MM;{}$\2\6
\4${}\}{}$\2\6
${}{*}\|k\K\|c;{}$\6
\4${}\}{}$\2\6
\&{if} ${}(\|k\G\\{section\_text\_end}){}$\5
${}\{{}$\1\6
${}\\{fputs}(\.{"\\n!\ Section\ name\ to}\)\.{o\ long:\ "},\39\\{stdout});{}$\6
${}\\{term\_write}(\\{section\_text}+\T{1},\39\T{25});{}$\6
\\{printf}(\.{"..."});\6
\\{mark\_harmless};\6
\4${}\}{}$\2\6
\&{if} ${}({*}\|k\E\.{'\ '}\W\|k>\\{section\_text}){}$\1\5
${}\|k\MM{}$;\2\par
\U60.\fi

\M{63}\B\X63:If end of name or erroneous control code, \PB{\&{break}}\X${}\E{}$%
\6
\&{if} ${}(\|c\E\.{'@'}){}$\5
${}\{{}$\1\6
${}\|c\K{*}(\\{loc}+\T{1});{}$\6
\&{if} ${}(\|c\E\.{'>'}){}$\5
${}\{{}$\1\6
${}\\{loc}\MRL{+{\K}}\T{2};{}$\6
\&{break};\6
\4${}\}{}$\2\6
\&{if} ${}(\\{ccode}[\|c]\E\\{new\_section}){}$\5
${}\{{}$\1\6
\\{err\_print}(\.{"!\ Section\ name\ didn}\)\.{'t\ end"});\6
\&{break};\6
\4${}\}{}$\2\6
\&{if} ${}(\|c\I\.{'@'}){}$\5
${}\{{}$\1\6
\\{err\_print}(\.{"!\ Control\ codes\ are}\)\.{\ forbidden\ in\ sectio}\)\.{n\
name"});\6
\&{break};\6
\4${}\}{}$\2\6
${}{*}(\PP\|k)\K\.{'@'};{}$\6
${}\\{loc}\PP{}$;\C{ now \PB{$\|c\E{*}\\{loc}$} again }\6
\4${}\}{}$\2\par
\U62.\fi

\M{64}This function skips over a restricted context at relatively high speed.

\Y\B\1\1\&{static} \&{void} \\{skip\_restricted}(\&{void})\2\2\6
${}\{{}$\1\6
${}\\{id\_first}\K\\{loc};{}$\6
${}{*}(\\{limit}+\T{1})\K\.{'@'};{}$\6
\4\\{false\_alarm}:\6
\&{while} ${}({*}\\{loc}\I\.{'@'}){}$\1\5
${}\\{loc}\PP;{}$\2\6
${}\\{id\_loc}\K\\{loc};{}$\6
\&{if} ${}(\\{loc}\PP>\\{limit}){}$\5
${}\{{}$\1\6
\\{err\_print}(\.{"!\ Control\ text\ didn}\)\.{'t\ end"});\6
${}\\{loc}\K\\{limit};{}$\6
\4${}\}{}$\2\6
\&{else}\5
${}\{{}$\1\6
\&{if} ${}({*}\\{loc}\E\.{'@'}\W\\{loc}\Z\\{limit}){}$\5
${}\{{}$\1\6
${}\\{loc}\PP;{}$\6
\&{goto} \\{false\_alarm};\6
\4${}\}{}$\2\6
\&{if} ${}({*}\\{loc}\PP\I\.{'>'}){}$\1\5
\\{err\_print}(\.{"!\ Control\ codes\ are}\)\.{\ forbidden\ in\ contro}\)\.{l\
text"});\2\6
\4${}\}{}$\2\6
\4${}\}{}$\2\par
\fi

\M{65}\B\X8:Predeclaration of procedures\X${}\mathrel+\E{}$\5
\&{static} \&{void} \\{skip\_restricted}(\&{void});\par
\fi

\M{66}At the present point in the program we
have \PB{${*}(\\{loc}-\T{1})\E\\{verbatim}$}; we set \PB{\\{id\_first}} to the
beginning
of the string itself, and \PB{\\{id\_loc}} to its ending-plus-one location in
the
buffer.  We also set \PB{\\{loc}} to the position just after the ending
delimiter.

\Y\B\4\X66:Scan a verbatim string\X${}\E{}$\6
$\\{id\_first}\K\\{loc}\PP;{}$\6
${}{*}(\\{limit}+\T{1})\K\.{'@'};{}$\6
${}{*}(\\{limit}+\T{2})\K\.{'>'};{}$\6
\&{while} ${}({*}\\{loc}\I\.{'@'}\V{*}(\\{loc}+\T{1})\I\.{'>'}){}$\1\5
${}\\{loc}\PP;{}$\2\6
\&{if} ${}(\\{loc}\G\\{limit}){}$\1\5
\\{err\_print}(\.{"!\ Verbatim\ string\ d}\)\.{idn't\ end"});\2\6
${}\\{id\_loc}\K\\{loc};{}$\6
${}\\{loc}\MRL{+{\K}}\T{2};{}$\6
\&{return} \\{verbatim};\par
\U59.\fi

\N{0}{67}Phase one processing.
We now have accumulated enough subroutines to make it possible to carry out
\.{CWEAVE}'s first pass over the source file. If everything works right,
both phase one and phase two of \.{CWEAVE} will assign the same numbers to
sections, and these numbers will agree with what \.{CTANGLE} does.

The global variable \PB{\\{next\_control}} often contains the most recent
output of
\PB{\\{get\_next}}; in interesting cases, this will be the control code that
ended a section or part of a section.

\Y\B\4\X21:Private variables\X${}\mathrel+\E{}$\6
\&{static} \&{eight\_bits} \\{next\_control};\C{ control code waiting to be
acting upon }\par
\fi

\M{68}The overall processing strategy in phase one has the following
straightforward outline.

\Y\B\1\1\&{static} \&{void} \\{phase\_one}(\&{void})\2\2\6
${}\{{}$\1\6
${}\\{phase}\K\T{1};{}$\6
\\{reset\_input}(\,);\6
${}\\{section\_count}\K\T{0};{}$\6
\\{skip\_limbo}(\,);\6
${}\\{change\_exists}\K\\{false};{}$\6
\&{while} ${}(\R\\{input\_has\_ended}){}$\1\5
\X70:Store cross-reference data for the current section\X\2\6
${}\\{changed\_section}[\\{section\_count}]\K\\{change\_exists}{}$;\C{ the
index changes if anything does }\6
\X84:Print error messages about unused or undefined section names\X\6
\4${}\}{}$\2\par
\fi

\M{69}\B\X8:Predeclaration of procedures\X${}\mathrel+\E{}$\5
\&{static} \&{void} \\{phase\_one}(\&{void});\par
\fi

\M{70}\B\X70:Store cross-reference data for the current section\X${}\E{}$\6
${}\{{}$\1\6
\&{if} ${}(\PP\\{section\_count}\E\\{max\_sections}){}$\1\5
\\{overflow}(\.{"section\ number"});\2\6
${}\\{changed\_section}[\\{section\_count}]\K\\{changing}{}$;\C{ it will become
\PB{\\{true}} if any line changes }\6
\&{if} ${}({*}(\\{loc}-\T{1})\E\.{'*'}\W\\{show\_progress}){}$\5
${}\{{}$\1\6
${}\\{printf}(\.{"*\%d"},\39{}$(\&{int}) \\{section\_count});\6
\\{update\_terminal};\C{ print a progress report }\6
\4${}\}{}$\2\6
\X74:Store cross-references in the \TEX/ part of a section\X\6
\X77:Store cross-references in the definition part of a section\X\6
\X80:Store cross-references in the \CEE/ part of a section\X\6
\&{if} (\\{changed\_section}[\\{section\_count}])\1\5
${}\\{change\_exists}\K\\{true};{}$\2\6
\4${}\}{}$\2\par
\U68.\fi

\M{71}The \PB{\\{C\_xref}} subroutine stores references to identifiers in
\CEE/ text material beginning with the current value of \PB{\\{next\_control}}
and continuing until \PB{\\{next\_control}} is `\.\{' or `\.{\v}', or until the
next
``milestone'' is passed (i.e., \PB{$\\{next\_control}\G\\{format\_code}$}). If
\PB{$\\{next\_control}\G\\{format\_code}$} when \PB{\\{C\_xref}} is called,
nothing will happen;
but if \PB{$\\{next\_control}\E\.{'|'}$} upon entry, the procedure assumes that
this is
the `\.{\v}' preceding \CEE/ text that is to be processed.

The parameter \PB{\\{spec\_ctrl}} is used to change this behavior. In most
cases
\PB{\\{C\_xref}} is called with \PB{$\\{spec\_ctrl}\E\\{ignore}$}, which
triggers the default
processing described above. If \PB{$\\{spec\_ctrl}\E\\{section\_name}$},
section names will
be gobbled. This is used when \CEE/ text in the \TEX/ part or inside comments
is parsed: It allows for section names to appear in \pb, but these
strings will not be entered into the cross reference lists since they are not
definitions of section names.

The program uses the fact that our internal code numbers satisfy
the relations \PB{$\\{xref\_roman}\E\\{identifier}+\\{roman}$} and \PB{$\\{xref%
\_wildcard}\E\\{identifier}+\\{wildcard}$} and \PB{$\\{xref\_typewriter}\E%
\\{identifier}+\\{typewriter}$},
as well as \PB{$\\{normal}\E\T{0}$}.

\Y\B\4\X8:Predeclaration of procedures\X${}\mathrel+\E{}$\6
\&{static} \&{void} \\{C\_xref}(\&{eight\_bits});\6
\&{static} \&{void} \\{outer\_xref}(\&{void});\par
\fi

\M{72}\B\1\1\&{static} \&{void} \\{C\_xref}(\C{ makes cross-references for %
\CEE/ identifiers }\6
\&{eight\_bits} \\{spec\_ctrl})\2\2\6
${}\{{}$\1\6
\&{while} ${}(\\{next\_control}<\\{format\_code}\V\\{next\_control}\E\\{spec%
\_ctrl}){}$\5
${}\{{}$\1\6
\&{if} ${}(\\{next\_control}\G\\{identifier}\W\\{next\_control}\Z\\{xref%
\_typewriter}){}$\5
${}\{{}$\1\6
\&{if} ${}(\\{next\_control}>\\{identifier}){}$\1\5
\X75:Replace `\.{@@}' by `\.{@}'\X\2\6
${}\\{new\_xref}(\\{id\_lookup}(\\{id\_first},\39\\{id\_loc},\39\\{next%
\_control}-\\{identifier}));{}$\6
\4${}\}{}$\2\6
\&{if} ${}(\\{next\_control}\E\\{section\_name}){}$\5
${}\{{}$\1\6
${}\\{section\_xref\_switch}\K\\{cite\_flag};{}$\6
\\{new\_section\_xref}(\\{cur\_section});\6
\4${}\}{}$\2\6
${}\\{next\_control}\K\\{get\_next}(\,);{}$\6
\&{if} ${}(\\{next\_control}\E\.{'|'}\V\\{next\_control}\E\\{begin\_comment}\V%
\\{next\_control}\E\\{begin\_short\_comment}){}$\1\5
\&{return};\2\6
\4${}\}{}$\2\6
\4${}\}{}$\2\par
\fi

\M{73}The \PB{\\{outer\_xref}} subroutine is like \PB{\\{C\_xref}} except that
it begins
with \PB{$\\{next\_control}\I\.{'|'}$} and ends with \PB{$\\{next\_control}\G%
\\{format\_code}$}. Thus, it
handles \CEE/ text with embedded comments.

\Y\B\1\1\&{static} \&{void} \\{outer\_xref}(\&{void})\C{ extension of \PB{\\{C%
\_xref}} }\2\2\6
${}\{{}$\1\6
\&{int} \\{bal};\C{ brace level in comment }\7
\&{while} ${}(\\{next\_control}<\\{format\_code}){}$\1\6
\&{if} ${}(\\{next\_control}\I\\{begin\_comment}\W\\{next\_control}\I\\{begin%
\_short\_comment}){}$\1\5
\\{C\_xref}(\\{ignore});\2\6
\&{else}\5
${}\{{}$\1\6
\&{boolean} \\{is\_long\_comment}${}\K(\\{next\_control}\E\\{begin%
\_comment});{}$\7
${}\\{bal}\K\\{copy\_comment}(\\{is\_long\_comment},\39\T{1});{}$\6
${}\\{next\_control}\K\.{'|'};{}$\6
\&{while} ${}(\\{bal}>\T{0}){}$\5
${}\{{}$\1\6
\\{C\_xref}(\\{section\_name});\C{ do not reference section names in comments }%
\6
\&{if} ${}(\\{next\_control}\E\.{'|'}){}$\1\5
${}\\{bal}\K\\{copy\_comment}(\\{is\_long\_comment},\39\\{bal});{}$\2\6
\&{else}\1\5
${}\\{bal}\K\T{0}{}$;\C{ an error message will occur in phase two }\2\6
\4${}\}{}$\2\6
\4${}\}{}$\2\2\6
\4${}\}{}$\2\par
\fi

\M{74}In the \TEX/ part of a section, cross-reference entries are made only for
the identifiers in \CEE/ texts enclosed in \pb, or for control texts
enclosed in \.{@\^}$\,\ldots\,$\.{@>} or \.{@.}$\,\ldots\,$\.{@>}
or \.{@:}$\,\ldots\,$\.{@>}.

\Y\B\4\X74:Store cross-references in the \TEX/ part of a section\X${}\E{}$\6
\&{while} (\\{true})\5
${}\{{}$\1\6
\&{switch} ${}(\\{next\_control}\K\skipxTeX(\,)){}$\5
${}\{{}$\1\6
\4\&{case} \\{translit\_code}:\5
\\{err\_print}(\.{"!\ Use\ @l\ in\ limbo\ o}\)\.{nly"});\6
\&{continue};\6
\4\&{case} \\{underline}:\5
${}\\{xref\_switch}\K\\{def\_flag};{}$\6
\&{continue};\6
\4\&{case} \\{trace}:\5
${}\\{tracing}\K{*}(\\{loc}-\T{1})-\.{'0'};{}$\6
\&{continue};\6
\4\&{case} \.{'|'}:\5
\\{C\_xref}(\\{section\_name});\6
\&{break};\6
\4\&{case} \\{xref\_roman}:\5
\&{case} \\{xref\_wildcard}:\5
\&{case} \\{xref\_typewriter}:\5
\&{case} \\{noop}:\5
\&{case} \\{section\_name}:\5
${}\\{loc}\MRL{-{\K}}\T{2};{}$\6
${}\\{next\_control}\K\\{get\_next}(\,){}$;\C{ scan to \.{@>} }\6
\&{if} ${}(\\{next\_control}\G\\{xref\_roman}\W\\{next\_control}\Z\\{xref%
\_typewriter}){}$\5
${}\{{}$\1\6
\X75:Replace `\.{@@}' by `\.{@}'\X\6
${}\\{new\_xref}(\\{id\_lookup}(\\{id\_first},\39\\{id\_loc},\39\\{next%
\_control}-\\{identifier}));{}$\6
\4${}\}{}$\2\6
\&{break};\6
\4${}\}{}$\2\6
\&{if} ${}(\\{next\_control}\G\\{format\_code}){}$\1\5
\&{break};\2\6
\4${}\}{}$\2\par
\U70.\fi

\M{75}\B\X75:Replace `\.{@@}' by `\.{@}'\X${}\E{}$\6
${}\{{}$\1\6
\&{char} ${}{*}\\{src}\K\\{id\_first},{}$ ${}{*}\\{dst}\K\\{id\_first};{}$\7
\&{while} ${}(\\{src}<\\{id\_loc}){}$\5
${}\{{}$\1\6
\&{if} ${}({*}\\{src}\E\.{'@'}){}$\1\5
${}\\{src}\PP;{}$\2\6
${}{*}\\{dst}\PP\K{*}\\{src}\PP;{}$\6
\4${}\}{}$\2\6
${}\\{id\_loc}\K\\{dst};{}$\6
\&{while} ${}(\\{dst}<\\{src}){}$\1\5
${}{*}\\{dst}\PP\K\.{'\ '}{}$;\C{ clean up in case of error message display }\2%
\6
\4${}\}{}$\2\par
\Us72\ET74.\fi

\M{76}During the definition and \CEE/ parts of a section, cross-references
are made for all identifiers except reserved words. However, the right
identifier in a format definition is not referenced, and the left
identifier is referenced only if it has been explicitly
underlined (preceded by \.{@!}).
The \TEX/ code in comments is, of course, ignored, except for
\CEE/ portions enclosed in \pb; the text of a section name is skipped
entirely, even if it contains \pb\ constructions.

The variables \PB{\\{lhs}} and \PB{\\{rhs}} point to the respective identifiers
involved
in a format definition.

\Y\B\4\X21:Private variables\X${}\mathrel+\E{}$\6
\&{static} \&{name\_pointer} \\{lhs}${},{}$ \\{rhs};\C{ pointers to \PB{\\{byte%
\_start}} for format identifiers }\6
\&{static} \&{name\_pointer} \\{res\_wd\_end};\C{ pointer to the first
nonreserved identifier }\par
\fi

\M{77}When we get to the following code we have \PB{$\\{next\_control}\G%
\\{format\_code}$}.

\Y\B\4\X77:Store cross-references in the definition part of a section\X${}\E{}$%
\6
\&{while} ${}(\\{next\_control}\Z\\{definition}){}$\5
${}\{{}$\C{ \PB{\\{format\_code}} or \PB{\\{definition}} }\1\6
\&{if} ${}(\\{next\_control}\E\\{definition}){}$\5
${}\{{}$\1\6
${}\\{xref\_switch}\K\\{def\_flag}{}$;\C{ implied \.{@!} }\6
${}\\{next\_control}\K\\{get\_next}(\,);{}$\6
\4${}\}{}$\2\6
\&{else}\1\5
\X78:Process a format definition\X\2\6
\\{outer\_xref}(\,);\6
\4${}\}{}$\2\par
\U70.\fi

\M{78}Error messages for improper format definitions will be issued in phase
two. Our job in phase one is to define the \PB{\\{ilk}} of a properly formatted
identifier, and to remove cross-references to identifiers that we now
discover should be unindexed.

\Y\B\4\X78:Process a format definition\X${}\E{}$\6
${}\{{}$\1\6
${}\\{next\_control}\K\\{get\_next}(\,);{}$\6
\&{if} ${}(\\{next\_control}\E\\{identifier}){}$\5
${}\{{}$\1\6
${}\\{lhs}\K\\{id\_lookup}(\\{id\_first},\39\\{id\_loc},\39\\{normal});{}$\6
${}\\{lhs}\MG\\{ilk}\K\\{normal};{}$\6
\&{if} (\\{xref\_switch})\1\5
\\{new\_xref}(\\{lhs});\2\6
${}\\{next\_control}\K\\{get\_next}(\,);{}$\6
\&{if} ${}(\\{next\_control}\E\\{identifier}){}$\5
${}\{{}$\1\6
${}\\{rhs}\K\\{id\_lookup}(\\{id\_first},\39\\{id\_loc},\39\\{normal});{}$\6
${}\\{lhs}\MG\\{ilk}\K\\{rhs}\MG\\{ilk};{}$\6
\&{if} (\\{unindexed}(\\{lhs}))\5
${}\{{}$\C{ retain only underlined entries }\1\6
\&{xref\_pointer} \|q${},{}$ \|r${}\K\NULL;{}$\7
\&{for} ${}(\|q\K{}$(\&{xref\_pointer}) \\{lhs}${}\MG\\{xref};{}$ ${}\|q>%
\\{xmem};{}$ ${}\|q\K\|q\MG\\{xlink}){}$\1\6
\&{if} ${}(\|q\MG\\{num}<\\{def\_flag}){}$\1\6
\&{if} (\|r)\1\5
${}\|r\MG\\{xlink}\K\|q\MG\\{xlink};{}$\2\6
\&{else}\1\5
${}\\{lhs}\MG\\{xref}\K{}$(\&{void} ${}{*}){}$ \|q${}\MG\\{xlink};{}$\2\2\6
\&{else}\1\5
${}\|r\K\|q;{}$\2\2\6
\4${}\}{}$\2\6
${}\\{next\_control}\K\\{get\_next}(\,);{}$\6
\4${}\}{}$\2\6
\4${}\}{}$\2\6
\4${}\}{}$\2\par
\U77.\fi

\M{79}A much simpler processing of format definitions occurs when the
definition is found in limbo.

\Y\B\4\X79:Process simple format in limbo\X${}\E{}$\6
\&{if} ${}(\\{get\_next}(\,)\I\\{identifier}){}$\1\5
\\{err\_print}(\.{"!\ Missing\ left\ iden}\)\.{tifier\ of\ @s"});\2\6
\&{else}\5
${}\{{}$\1\6
${}\\{lhs}\K\\{id\_lookup}(\\{id\_first},\39\\{id\_loc},\39\\{normal});{}$\6
\&{if} ${}(\\{get\_next}(\,)\I\\{identifier}){}$\1\5
\\{err\_print}(\.{"!\ Missing\ right\ ide}\)\.{ntifier\ of\ @s"});\2\6
\&{else}\5
${}\{{}$\1\6
${}\\{rhs}\K\\{id\_lookup}(\\{id\_first},\39\\{id\_loc},\39\\{normal});{}$\6
${}\\{lhs}\MG\\{ilk}\K\\{rhs}\MG\\{ilk};{}$\6
\4${}\}{}$\2\6
\4${}\}{}$\2\par
\U41.\fi

\M{80}Finally, when the \TEX/ and definition parts have been treated, we have
\PB{$\\{next\_control}\G\\{begin\_C}$}.

\Y\B\4\X80:Store cross-references in the \CEE/ part of a section\X${}\E{}$\6
\&{if} ${}(\\{next\_control}\Z\\{section\_name}){}$\5
${}\{{}$\C{ \PB{\\{begin\_C}} or \PB{\\{section\_name}} }\1\6
\&{if} ${}(\\{next\_control}\E\\{begin\_C}){}$\1\5
${}\\{section\_xref\_switch}\K\T{0};{}$\2\6
\&{else}\5
${}\{{}$\1\6
${}\\{section\_xref\_switch}\K\\{def\_flag};{}$\6
\&{if} ${}(\\{cur\_section\_char}\E\.{'('}\W\\{cur\_section}\I\\{name\_dir}){}$%
\1\5
\\{set\_file\_flag}(\\{cur\_section});\2\6
\4${}\}{}$\2\6
\&{do}\5
${}\{{}$\1\6
\&{if} ${}(\\{next\_control}\E\\{section\_name}\W\\{cur\_section}\I\\{name%
\_dir}){}$\1\5
\\{new\_section\_xref}(\\{cur\_section});\2\6
${}\\{next\_control}\K\\{get\_next}(\,);{}$\6
\\{outer\_xref}(\,);\6
\4${}\}{}$\2\5
\&{while} ${}(\\{next\_control}\Z\\{section\_name});{}$\6
\4${}\}{}$\2\par
\U70.\fi

\M{81}After phase one has looked at everything, we want to check that each
section name was both defined and used.  The variable \PB{\\{cur\_xref}} will
point
to cross-references for the current section name of interest.

\Y\B\4\X21:Private variables\X${}\mathrel+\E{}$\6
\&{static} \&{xref\_pointer} \\{cur\_xref};\C{ temporary cross-reference
pointer }\6
\&{static} \&{boolean} \\{an\_output};\C{ did \PB{\\{file\_flag}} precede \PB{%
\\{cur\_xref}}? }\par
\fi

\M{82}The following recursive procedure
walks through the tree of section names and prints out anomalies.

\Y\B\1\1\&{static} \&{void} \\{section\_check}(\&{name\_pointer} \|p)\C{ print
anomalies in subtree \PB{\|p} }\2\2\6
${}\{{}$\1\6
\&{if} (\|p)\5
${}\{{}$\1\6
${}\\{section\_check}(\|p\MG\\{llink});{}$\6
${}\\{cur\_xref}\K{}$(\&{xref\_pointer}) \|p${}\MG\\{xref};{}$\6
\&{if} ${}((\\{an\_output}\K(\\{cur\_xref}\MG\\{num}\E\\{file\_flag}))\E%
\\{true}){}$\1\5
${}\\{cur\_xref}\K\\{cur\_xref}\MG\\{xlink};{}$\2\6
\&{if} ${}(\\{cur\_xref}\MG\\{num}<\\{def\_flag}){}$\5
${}\{{}$\1\6
${}\\{fputs}(\.{"\\n!\ Never\ defined:\ }\)\.{<"},\39\\{stdout});{}$\6
\\{print\_section\_name}(\|p);\6
\\{putchar}(\.{'>'});\6
\\{mark\_harmless};\6
\4${}\}{}$\2\6
\&{while} ${}(\\{cur\_xref}\MG\\{num}\G\\{cite\_flag}){}$\1\5
${}\\{cur\_xref}\K\\{cur\_xref}\MG\\{xlink};{}$\2\6
\&{if} ${}(\\{cur\_xref}\E\\{xmem}\W\R\\{an\_output}){}$\5
${}\{{}$\1\6
${}\\{fputs}(\.{"\\n!\ Never\ used:\ <"},\39\\{stdout});{}$\6
\\{print\_section\_name}(\|p);\6
\\{putchar}(\.{'>'});\6
\\{mark\_harmless};\6
\4${}\}{}$\2\6
${}\\{section\_check}(\|p\MG\\{rlink});{}$\6
\4${}\}{}$\2\6
\4${}\}{}$\2\par
\fi

\M{83}\B\X8:Predeclaration of procedures\X${}\mathrel+\E{}$\5
\&{static} \&{void} \\{section\_check}(\&{name\_pointer});\par
\fi

\M{84}\B\X84:Print error messages about unused or undefined section names\X${}%
\E{}$\6
\\{section\_check}(\\{root});\par
\U68.\fi

\N{1}{85}Low-level output routines.
The \TEX/ output is supposed to appear in lines at most \PB{\\{line\_length}}
characters long, so we place it into an output buffer. During the output
process, \PB{\\{out\_line}} will hold the current line number of the line about
to
be output.

\Y\B\4\X21:Private variables\X${}\mathrel+\E{}$\6
\&{static} \&{char} ${}\\{out\_buf}[\\{line\_length}+\T{1}]{}$;\C{ assembled
characters }\6
\&{static} \&{char} ${}{*}\\{out\_buf\_end}\K\\{out\_buf}+\\{line\_length}{}$;%
\C{ end of \PB{\\{out\_buf}} }\6
\&{static} \&{char} ${}{*}\\{out\_ptr}{}$;\C{ last character in \PB{\\{out%
\_buf}} }\6
\&{static} \&{int} \\{out\_line};\C{ number of next line to be output }\par
\fi

\M{86}The \PB{\\{flush\_buffer}} routine empties the buffer up to a given
breakpoint,
and moves any remaining characters to the beginning of the next line.
If the \PB{\\{per\_cent}} parameter is \PB{\\{true}}, a \PB{\.{'\%'}} is
appended to the line
that is being output; in this case the breakpoint \PB{\|b} should be strictly
less than \PB{\\{out\_buf\_end}}. If the \PB{\\{per\_cent}} parameter is \PB{%
\\{false}},
trailing blanks are suppressed.
The characters emptied from the buffer form a new line of output;
if the \PB{\\{carryover}} parameter is \PB{\\{true}}, a \PB{\.{"\%"}} in that
line will be
carried over to the next line (so that \TEX/ will ignore the completion
of commented-out text).

\Y\B\4\D\\{c\_line\_write}$(\|c)$\5
$\\{fflush}(\\{active\_file}),\39\\{fwrite}(\\{out\_buf}+\T{1},\39\&{sizeof}(%
\&{char}),\39\|c,\39\\{active\_file}{}$)\par
\B\4\D\\{tex\_putc}$(\|c)$\5
$\\{putc}(\|c,\39\\{active\_file}{}$)\par
\B\4\D\\{tex\_new\_line}\5
$\\{putc}(\.{'\\n'},\39\\{active\_file}{}$)\par
\B\4\D\\{tex\_printf}$(\|c)$\5
$\\{fprintf}(\\{active\_file},\39\.{"\%s"},\39\|c{}$)\par
\B\4\D\\{tex\_puts}$(\|c)$\5
$\\{fputs}(\|c,\39\\{active\_file}{}$)\par
\Y\B\4\X8:Predeclaration of procedures\X${}\mathrel+\E{}$\6
\&{static} \&{void} \\{flush\_buffer}(\&{char} ${}{*},\39\&{boolean},\39%
\&{boolean}){}$;\6
\&{static} \&{void} \\{finish\_line}(\&{void});\par
\fi

\M{87}\B\1\1\&{static} \&{void} \\{flush\_buffer}(\&{char} ${}{*}\|b,{}$\C{
outputs from \PB{$\\{out\_buf}+\T{1}$} to \PB{\|b}, where \PB{$\|b\Z\\{out%
\_ptr}$} }\6
\&{boolean} \\{per\_cent}${},\39{}$\&{boolean} \\{carryover})\2\2\6
${}\{{}$\1\6
\&{char} ${}{*}\|j\K\|b{}$;\C{ pointer into \PB{\\{out\_buf}} }\7
\&{if} ${}(\R\\{per\_cent}{}$)\C{ remove trailing blanks }\1\6
\&{while} ${}(\|j>\\{out\_buf}\W{*}\|j\E\.{'\ '}){}$\1\5
${}\|j\MM;{}$\2\2\6
${}\\{c\_line\_write}(\|j-\\{out\_buf});{}$\6
\&{if} (\\{per\_cent})\1\5
\\{tex\_putc}(\.{'\%'});\2\6
\\{tex\_new\_line};\6
${}\\{out\_line}\PP;{}$\6
\&{if} (\\{carryover})\1\6
\&{while} ${}(\|j>\\{out\_buf}){}$\1\6
\&{if} ${}({*}\|j\MM\E\.{'\%'}\W(\|j\E\\{out\_buf}\V{*}\|j\I\.{'\\\\'})){}$\5
${}\{{}$\1\6
${}{*}\|b\MM\K\.{'\%'};{}$\6
\&{break};\6
\4${}\}{}$\2\2\2\6
\&{if} ${}(\|b<\\{out\_ptr}){}$\1\5
${}\\{memcpy}(\\{out\_buf}+\T{1},\39\|b+\T{1},\39(\&{size\_t})(\\{out\_ptr}-%
\|b));{}$\2\6
${}\\{out\_ptr}\MRL{-{\K}}\|b-\\{out\_buf};{}$\6
\4${}\}{}$\2\par
\fi

\M{88}When we are copying \TEX/ source material, we retain line breaks
that occur in the input, except that an empty line is not
output when the \TEX/ source line was nonempty. For example, a line
of the \TEX/ file that contains only an index cross-reference entry
will not be copied. The \PB{\\{finish\_line}} routine is called just before
\PB{\\{get\_line}} inputs a new line, and just after a line break token has
been emitted during the output of translated \CEE/ text.

\Y\B\1\1\&{static} \&{void} \\{finish\_line}(\&{void})\C{ do this at the end of
a line }\2\2\6
${}\{{}$\1\6
\&{char} ${}{*}\|k{}$;\C{ pointer into \PB{\\{buffer}} }\7
\&{if} ${}(\\{out\_ptr}>\\{out\_buf}){}$\1\5
${}\\{flush\_buffer}(\\{out\_ptr},\39\\{false},\39\\{false});{}$\2\6
\&{else}\5
${}\{{}$\1\6
\&{for} ${}(\|k\K\\{buffer};{}$ ${}\|k\Z\\{limit};{}$ ${}\|k\PP){}$\1\6
\&{if} ${}(\R(\\{xisspace}({*}\|k))){}$\1\5
\&{return};\2\2\6
${}\\{flush\_buffer}(\\{out\_buf},\39\\{false},\39\\{false});{}$\6
\4${}\}{}$\2\6
\4${}\}{}$\2\par
\fi

\M{89}In particular, the \PB{\\{finish\_line}} procedure is called near the
very
beginning of phase two. We initialize the output variables in a slightly
tricky way so that the first line of the output file will be
`\.{\\input cwebmac}'.

\Y\B\4\X89:Start \TEX/ output\X${}\E{}$\6
$\\{out\_ptr}\K\\{out\_buf}+\T{1};{}$\6
${}\\{out\_line}\K\T{1};{}$\6
${}\\{active\_file}\K\\{tex\_file};{}$\6
\\{tex\_printf}(\.{"\\\\input\ cwebma"});\6
${}{*}\\{out\_ptr}\K\.{'c'}{}$;\par
\U2.\fi

\M{90}When we wish to append one character \PB{\|c} to the output buffer, we
write
`\PB{\\{out}(\|c)}'; this will cause the buffer to be emptied if it was already
full.  If we want to append more than one character at once, we say
\PB{\\{out\_str}(\|s)}, where \PB{\|s} is a string containing the characters.

A line break will occur at a space or after a single-nonletter
\TEX/ control sequence.

\Y\B\4\D\\{out}$(\|c)$\6
${}\{{}$\1\6
\&{if} ${}(\\{out\_ptr}\G\\{out\_buf\_end}){}$\1\5
\\{break\_out}(\,);\2\6
${}{*}(\PP\\{out\_ptr})\K\|c;{}$\6
\4${}\}{}$\2\par
\Y\B\4\X8:Predeclaration of procedures\X${}\mathrel+\E{}$\6
\&{static} \&{void} \\{out\_str}(\&{const} \&{char} ${}{*}){}$;\6
\&{static} \&{void} \\{break\_out}(\&{void});\par
\fi

\M{91}\B\1\1\&{static} \&{void} \\{out\_str}(\C{ output characters from \PB{%
\|s} to end of string }\6
\&{const} \&{char} ${}{*}\|s)\2\2{}$\6
${}\{{}$\1\6
\&{while} ${}({*}\|s){}$\1\5
${}\\{out}({*}\|s\PP);{}$\2\6
\4${}\}{}$\2\par
\fi

\M{92}The \PB{\\{break\_out}} routine is called just before the output buffer
is about
to overflow. To make this routine a little faster, we initialize position
0 of the output buffer to `\.\\'; this character isn't really output.

\Y\B\4\X24:Set initial values\X${}\mathrel+\E{}$\6
$\\{out\_buf}[\T{0}]\K\.{'\\\\'}{}$;\par
\fi

\M{93}A long line is broken at a blank space or just before a backslash that
isn't
preceded by another backslash. In the latter case, a \PB{\.{'\%'}} is output at
the break.

\Y\B\1\1\&{static} \&{void} \\{break\_out}(\&{void})\C{ finds a way to break
the output line }\2\2\6
${}\{{}$\1\6
\&{char} ${}{*}\|k\K\\{out\_ptr}{}$;\C{ pointer into \PB{\\{out\_buf}} }\7
\&{while} (\\{true})\5
${}\{{}$\1\6
\&{if} ${}(\|k\E\\{out\_buf}){}$\1\5
\X94:Print warning message, break the line, \PB{\&{return}}\X\2\6
\&{if} ${}({*}\|k\E\.{'\ '}){}$\5
${}\{{}$\1\6
${}\\{flush\_buffer}(\|k,\39\\{false},\39\\{true});{}$\6
\&{return};\6
\4${}\}{}$\2\6
\&{if} ${}({*}(\|k\MM)\E\.{'\\\\'}\W{*}\|k\I\.{'\\\\'}){}$\5
${}\{{}$\C{ we've decreased \PB{\|k} }\1\6
${}\\{flush\_buffer}(\|k,\39\\{true},\39\\{true});{}$\6
\&{return};\6
\4${}\}{}$\2\6
\4${}\}{}$\2\6
\4${}\}{}$\2\par
\fi

\M{94}We get to this section only in the unusual case that the entire output
line
consists of a string of backslashes followed by a string of nonblank
non-backslashes. In such cases it is almost always safe to break the
line by putting a \PB{\.{'\%'}} just before the last character.

\Y\B\4\X94:Print warning message, break the line, \PB{\&{return}}\X${}\E{}$\6
${}\{{}$\1\6
${}\\{printf}(\.{"\\n!\ Line\ had\ to\ be\ }\)\.{broken\ (output\ l.\ \%d}\)%
\.{):\\n"},\39\\{out\_line});{}$\6
${}\\{term\_write}(\\{out\_buf}+\T{1},\39\\{out\_ptr}-\\{out\_buf}-\T{1});{}$\6
\\{new\_line};\6
\\{mark\_harmless};\6
${}\\{flush\_buffer}(\\{out\_ptr}-\T{1},\39\\{true},\39\\{true});{}$\6
\&{return};\6
\4${}\}{}$\2\par
\U93.\fi

\M{95}Here is a macro that outputs a section number in decimal notation.
The number to be converted by \PB{\\{out\_section}} is known to be less than
\PB{\\{def\_flag}}, so it cannot have more than five decimal digits.  If
the section is changed, we output `\.{\\*}' just after the number.

\Y\B\4\X8:Predeclaration of procedures\X${}\mathrel+\E{}$\6
\&{static} \&{void} \\{out\_section}(\&{sixteen\_bits});\6
\&{static} \&{void} ${}\\{out\_name}(\&{name\_pointer},\39\&{boolean}){}$;\par
\fi

\M{96}\B\1\1\&{static} \&{void} \\{out\_section}(\&{sixteen\_bits} \|n)\2\2\6
${}\{{}$\1\6
\&{char} \|s[\T{6}];\7
${}\\{sprintf}(\|s,\39\.{"\%d"},\39{}$(\&{int}) \|n);\6
\\{out\_str}(\|s);\6
\&{if} (\\{changed\_section}[\|n])\1\5
\\{out\_str}(\.{"\\\\*"});\2\6
\4${}\}{}$\2\par
\fi

\M{97}The \PB{\\{out\_name}} procedure is used to output an identifier or index
entry, enclosing it in braces.

\Y\B\1\1\&{static} \&{void} \\{out\_name}(\&{name\_pointer} \|p${},\39{}$%
\&{boolean} \\{quote\_xalpha})\2\2\6
${}\{{}$\1\6
\&{char} ${}{*}\|k,{}$ ${}{*}\\{k\_end}\K(\|p+\T{1})\MG\\{byte\_start}{}$;\C{
pointers into \PB{\\{byte\_mem}} }\7
\\{out}(\.{'\{'});\6
\&{for} ${}(\|k\K\|p\MG\\{byte\_start};{}$ ${}\|k<\\{k\_end};{}$ ${}\|k\PP){}$\5
${}\{{}$\1\6
\&{if} ${}(\\{isxalpha}({*}\|k)\W\\{quote\_xalpha}){}$\1\5
\\{out}(\.{'\\\\'});\2\6
${}\\{out}({*}\|k);{}$\6
\4${}\}{}$\2\6
\\{out}(\.{'\}'});\6
\4${}\}{}$\2\par
\fi

\N{1}{98}Routines that copy \TEX/ material.
During phase two, we use the subroutines \PB{\\{copy\_limbo}} and \PB{$%
\copyxTeX$} (and
\PB{\\{copy\_comment}}) in place of the analogous \PB{\\{skip\_limbo}} and %
\PB{$\skipxTeX$}
that were used in phase one.

The \PB{\\{copy\_limbo}} routine, for example, takes \TEX/ material that is not
part of any section and transcribes it almost verbatim to the output file.
The use of `\.{@}' signs is severely restricted in such material:
`\.{@@}' pairs are replaced by singletons; `\.{@l}' and `\.{@q}' and
`\.{@s}' are interpreted.

\Y\B\4\X8:Predeclaration of procedures\X${}\mathrel+\E{}$\6
\&{static} \&{void} \\{copy\_limbo}(\&{void});\6
\&{static} \&{eight\_bits} ${}\copyxTeX(\&{void}){}$;\6
\&{static} \&{int} ${}\\{copy\_comment}(\&{boolean},\39\&{int}){}$;\par
\fi

\M{99}\B\1\1\&{static} \&{void} \\{copy\_limbo}(\&{void})\2\2\6
${}\{{}$\1\6
\&{while} (\\{true})\5
${}\{{}$\1\6
\&{if} ${}(\\{loc}>\\{limit}\W(\\{finish\_line}(\,),\39\\{get\_line}(\,)\E%
\\{false})){}$\1\5
\&{return};\2\6
${}{*}(\\{limit}+\T{1})\K\.{'@'};{}$\6
\&{while} ${}({*}\\{loc}\I\.{'@'}){}$\1\5
${}\\{out}({*}(\\{loc}\PP));{}$\2\6
\&{if} ${}(\\{loc}\PP\Z\\{limit}){}$\5
${}\{{}$\1\6
\&{switch} (\\{ccode}[(\&{eight\_bits}) ${}{*}\\{loc}\PP]){}$\5
${}\{{}$\1\6
\4\&{case} \\{new\_section}:\5
\&{return};\6
\4\&{case} \\{translit\_code}:\5
\\{out\_str}(\.{"\\\\ATL"});\6
\&{break};\6
\4\&{case} \.{'@'}:\5
\\{out}(\.{'@'});\6
\&{break};\6
\4\&{case} \\{noop}:\5
\\{skip\_restricted}(\,);\6
\&{break};\6
\4\&{case} \\{format\_code}:\6
\&{if} ${}(\\{get\_next}(\,)\E\\{identifier}){}$\1\5
\\{get\_next}(\,);\2\6
\&{if} ${}(\\{loc}\G\\{limit}){}$\1\5
\\{get\_line}(\,);\C{ avoid blank lines in output }\2\6
\&{break};\C{ the operands of \.{@s} are ignored on this pass }\6
\4\&{default}:\5
\\{err\_print}(\.{"!\ Double\ @\ should\ b}\)\.{e\ used\ in\ limbo"});\6
\\{out}(\.{'@'});\6
\4${}\}{}$\2\6
\4${}\}{}$\2\6
\4${}\}{}$\2\6
\4${}\}{}$\2\par
\fi

\M{100}The \PB{$\copyxTeX$} routine processes the \TEX/ code at the beginning
of a
section; for example, the words you are now reading were copied in this
way. It returns the next control code or `\.{\v}' found in the input.
We don't copy spaces or tab marks into the beginning of a line. This
makes the test for empty lines in \PB{\\{finish\_line}} work.

\Y\B\F\\{copy\_TeX}\5
\\{TeX}\par
\Y\B\1\1\&{static} \&{eight\_bits} ${}\copyxTeX(\&{void})\2\2{}$\6
${}\{{}$\1\6
\&{char} \|c;\C{ current character being copied }\7
\&{while} (\\{true})\5
${}\{{}$\1\6
\&{if} ${}(\\{loc}>\\{limit}\W(\\{finish\_line}(\,),\39\\{get\_line}(\,)\E%
\\{false})){}$\1\5
\&{return} \\{new\_section};\2\6
${}{*}(\\{limit}+\T{1})\K\.{'@'};{}$\6
\&{while} ${}((\|c\K{*}(\\{loc}\PP))\I\.{'|'}\W\|c\I\.{'@'}){}$\5
${}\{{}$\1\6
\\{out}(\|c);\6
\&{if} ${}(\\{out\_ptr}\E\\{out\_buf}+\T{1}\W(\\{xisspace}(\|c))){}$\1\5
${}\\{out\_ptr}\MM;{}$\2\6
\4${}\}{}$\2\6
\&{if} ${}(\|c\E\.{'|'}){}$\1\5
\&{return} \.{'|'};\2\6
\&{if} ${}(\\{loc}\Z\\{limit}){}$\1\5
\&{return} \\{ccode}[(\&{eight\_bits}) ${}{*}(\\{loc}\PP)];{}$\2\6
\4${}\}{}$\2\6
\4${}\}{}$\2\par
\fi

\M{101}The \PB{\\{copy\_comment}} function issues a warning if more braces are
opened than
closed, and in the case of a more serious error it supplies enough
braces to keep \TEX/ from complaining about unbalanced braces.
Instead of copying the \TEX/ material
into the output buffer, this function copies it into the token memory
(in phase two only).
The abbreviation \PB{\\{app\_tok}(\|t)} is used to append token \PB{\|t} to the
current
token list, and it also makes sure that it is possible to append at least
one further token without overflow.

\Y\B\4\D\\{app\_tok}$(\|c)$\6
${}\{{}$\1\6
\&{if} ${}(\\{tok\_ptr}+\T{2}>\\{tok\_mem\_end}){}$\1\5
\\{overflow}(\.{"token"});\2\6
${}{*}(\\{tok\_ptr}\PP)\K\|c;{}$\6
\4${}\}{}$\2\par
\Y\B\1\1\&{static} \&{int} \\{copy\_comment}(\C{ copies \TeX\ code in comments
}\6
\&{boolean} \\{is\_long\_comment}${},{}$\C{ is this a traditional \CEE/
comment? }\6
\&{int} \\{bal})\C{ brace balance }\2\2\6
${}\{{}$\1\6
\&{char} \|c;\C{ current character being copied }\7
\&{while} (\\{true})\5
${}\{{}$\1\6
\&{if} ${}(\\{loc}>\\{limit}){}$\5
${}\{{}$\1\6
\&{if} (\\{is\_long\_comment})\5
${}\{{}$\1\6
\&{if} ${}(\\{get\_line}(\,)\E\\{false}){}$\5
${}\{{}$\1\6
\\{err\_print}(\.{"!\ Input\ ended\ in\ mi}\)\.{d-comment"});\6
${}\\{loc}\K\\{buffer}+\T{1};{}$\6
\&{goto} \\{done};\6
\4${}\}{}$\2\6
\4${}\}{}$\2\6
\&{else}\5
${}\{{}$\1\6
\&{if} ${}(\\{bal}>\T{1}){}$\1\5
\\{err\_print}(\.{"!\ Missing\ \}\ in\ comm}\)\.{ent"});\2\6
\&{goto} \\{done};\6
\4${}\}{}$\2\6
\4${}\}{}$\2\6
${}\|c\K{*}(\\{loc}\PP);{}$\6
\&{if} ${}(\|c\E\.{'|'}){}$\1\5
\&{return} \\{bal};\2\6
\&{if} (\\{is\_long\_comment})\1\5
\X102:Check for end of comment\X\2\6
\&{if} ${}(\\{phase}\E\T{2}){}$\5
${}\{{}$\1\6
\&{if} (\\{ishigh}(\|c))\1\5
\\{app\_tok}(\\{quoted\_char});\2\6
\\{app\_tok}(\|c);\6
\4${}\}{}$\2\6
\X103:Copy special things when \PB{$\|c\E\.{'@'},\.{'\\\\'}$}\X\6
\&{if} ${}(\|c\E\.{'\{'}){}$\1\5
${}\\{bal}\PP;{}$\2\6
\&{else} \&{if} ${}(\|c\E\.{'\}'}){}$\5
${}\{{}$\1\6
\&{if} ${}(\\{bal}>\T{1}){}$\1\5
${}\\{bal}\MM;{}$\2\6
\&{else}\5
${}\{{}$\1\6
\\{err\_print}(\.{"!\ Extra\ \}\ in\ commen}\)\.{t"});\6
\&{if} ${}(\\{phase}\E\T{2}){}$\1\5
${}\\{tok\_ptr}\MM;{}$\2\6
\4${}\}{}$\2\6
\4${}\}{}$\2\6
\4${}\}{}$\2\6
\4\\{done}:\5
\X104:Clear \PB{\\{bal}} and \PB{\&{return}}\X\6
\4${}\}{}$\2\par
\fi

\M{102}\B\X102:Check for end of comment\X${}\E{}$\6
\&{if} ${}(\|c\E\.{'*'}\W{*}\\{loc}\E\.{'/'}){}$\5
${}\{{}$\1\6
${}\\{loc}\PP;{}$\6
\&{if} ${}(\\{bal}>\T{1}){}$\1\5
\\{err\_print}(\.{"!\ Missing\ \}\ in\ comm}\)\.{ent"});\2\6
\&{goto} \\{done};\6
\4${}\}{}$\2\par
\U101.\fi

\M{103}\B\X103:Copy special things when \PB{$\|c\E\.{'@'},\.{'\\\\'}$}\X${}%
\E{}$\6
\&{if} ${}(\|c\E\.{'@'}){}$\5
${}\{{}$\1\6
\&{if} ${}({*}(\\{loc}\PP)\I\.{'@'}){}$\5
${}\{{}$\1\6
\\{err\_print}(\.{"!\ Illegal\ use\ of\ @\ }\)\.{in\ comment"});\6
${}\\{loc}\MRL{-{\K}}\T{2};{}$\6
\&{if} ${}(\\{phase}\E\T{2}){}$\1\5
${}{*}(\\{tok\_ptr}-\T{1})\K\.{'\ '};{}$\2\6
\&{goto} \\{done};\6
\4${}\}{}$\2\6
\4${}\}{}$\2\6
\&{else}\5
${}\{{}$\1\6
\&{if} ${}(\|c\E\.{'\\\\'}\W{*}\\{loc}\I\.{'@'}){}$\5
${}\{{}$\1\6
\&{if} ${}(\\{phase}\E\T{2}){}$\1\5
${}\\{app\_tok}({*}(\\{loc}\PP))\hbox{;}{}$\2\6
\&{else}\1\5
${}\\{loc}\PP;{}$\2\6
\4${}\}{}$\2\6
\4${}\}{}$\2\par
\U101.\fi

\M{104}We output
enough right braces to keep \TEX/ happy.

\Y\B\4\X104:Clear \PB{\\{bal}} and \PB{\&{return}}\X${}\E{}$\6
\&{if} ${}(\\{phase}\E\T{2}){}$\1\6
\&{while} ${}(\\{bal}\MM>\T{0}){}$\1\5
\\{app\_tok}(\.{'\}'});\2\2\6
\&{return} \T{0};\par
\U101.\fi

\N{0}{105}Parsing.
The most intricate part of \.{CWEAVE} is its mechanism for converting
\CEE/-like code into \TEX/ code, and we might as well plunge into this
aspect of the program now. A ``bottom up'' approach is used to parse the
\CEE/-like material, since \.{CWEAVE} must deal with fragmentary
constructions whose overall ``part of speech'' is not known.

At the lowest level, the input is represented as a sequence of entities
that we shall call {\it scraps}, where each scrap of information consists
of two parts, its {\it category} and its {\it translation}. The category
is essentially a syntactic class, and the translation is a token list that
represents \TEX/ code. Rules of syntax and semantics tell us how to
combine adjacent scraps into larger ones, and if we are lucky an entire
\CEE/ text that starts out as hundreds of small scraps will join
together into one gigantic scrap whose translation is the desired \TEX/
code. If we are unlucky, we will be left with several scraps that don't
combine; their translations will simply be output, one by one.

The combination rules are given as context-sensitive productions that are
applied from left to right. Suppose that we are currently working on the
sequence of scraps $s_1\,s_2\ldots s_n$. We try first to find the longest
production that applies to an initial substring $s_1\,s_2\ldots\,$; but if
no such productions exist, we try to find the longest production
applicable to the next substring $s_2\,s_3\ldots\,$; and if that fails, we
try to match $s_3\,s_4\ldots\,$, etc.

A production applies if the category codes have a given pattern. For
example, one of the productions (see rule~3) is
$$\hbox{\PB{\\{exp}} }\left\{\matrix{\hbox{\PB{\\{binop}}}\cr\hbox{\PB{%
\\{ubinop}}}}\right\}
\hbox{ \PB{\\{exp}} }\RA\hbox{ \PB{\\{exp}}}$$
and it means that three consecutive scraps whose respective categories are
\PB{\\{exp}}, \PB{\\{binop}} (or \PB{\\{ubinop}}),
and \PB{\\{exp}} are converted to one scrap whose category
is \PB{\\{exp}}.  The translations of the original
scraps are simply concatenated.  The case of
$$\hbox{\PB{\\{exp}} \PB{\\{comma}} \PB{\\{exp}} $\RA$ \PB{\\{exp}}} %
\hskip4emE_1C\,\\{opt}9\,E_2$$
(rule 4) is only slightly more complicated:
Here the resulting \PB{\\{exp}} translation
consists not only of the three original translations, but also of the
tokens \PB{\\{opt}} and 9 between the translations of the
\PB{\\{comma}} and the following \PB{\\{exp}}.
In the \TEX/ file, this will specify an optional line break after the
comma, with penalty 90.

At each opportunity the longest possible production is applied.  For
example, if the current sequence of scraps is \PB{\\{if\_clause}} \PB{\\{stmt}}
\PB{\\{else\_like}} \PB{\\{if\_like}}, rule 63 is applied; but if the sequence
is
\PB{\\{if\_clause}} \PB{\\{stmt}} \PB{\\{else\_like}} followed by anything
other than
\PB{\\{if\_like}}, rule 64 takes effect; and if the sequence is \PB{\\{if%
\_clause}}
\PB{\\{stmt}} followed by anything other than \PB{\\{else\_like}}, rule 65
takes
effect.

Translation rules such as `$E_1C\,\\{opt}9\,E_2$' above use subscripts
to distinguish between translations of scraps whose categories have the
same initial letter; these subscripts are assigned from left to right.

\fi

\M{106}Here is a list of the category codes that scraps can have.
(A few others, like \PB{\\{int\_like}}, have already been defined; the
\PB{\\{cat\_name}} array contains a complete list.)

\Y\B\4\D\\{exp}\5
\T{1}\C{ denotes an expression, including perhaps a single identifier }\par
\B\4\D\\{unop}\5
\T{2}\C{ denotes a unary operator }\par
\B\4\D\\{binop}\5
\T{3}\C{ denotes a binary operator }\par
\B\4\D\\{ubinop}\5
\T{4}\C{ denotes an operator that can be unary or binary, depending on context
}\par
\B\4\D\\{cast}\5
\T{5}\C{ denotes a cast }\par
\B\4\D\\{question}\5
\T{6}\C{ denotes a question mark and possibly the expressions flanking it }\par
\B\4\D\\{lbrace}\5
\T{7}\C{ denotes a left brace }\par
\B\4\D\\{rbrace}\5
\T{8}\C{ denotes a right brace }\par
\B\4\D\\{decl\_head}\5
\T{9}\C{ denotes an incomplete declaration }\par
\B\4\D\\{comma}\5
\T{10}\C{ denotes a comma }\par
\B\4\D\\{lpar}\5
\T{11}\C{ denotes a left parenthesis }\par
\B\4\D\\{rpar}\5
\T{12}\C{ denotes a right parenthesis }\par
\B\4\D\\{prelangle}\5
\T{13}\C{ denotes `$<$' before we know what it is }\par
\B\4\D\\{prerangle}\5
\T{14}\C{ denotes `$>$' before we know what it is }\par
\B\4\D\\{langle}\5
\T{15}\C{ denotes `$<$' when it's used as angle bracket in a template }\par
\B\4\D\\{colcol}\5
\T{18}\C{ denotes `::' }\par
\B\4\D\\{base}\5
\T{19}\C{ denotes a colon that introduces a base specifier }\par
\B\4\D\\{decl}\5
\T{20}\C{ denotes a complete declaration }\par
\B\4\D\\{struct\_head}\5
\T{21}\C{ denotes the beginning of a structure specifier }\par
\B\4\D\\{stmt}\5
\T{23}\C{ denotes a complete statement }\par
\B\4\D\\{function}\5
\T{24}\C{ denotes a complete function }\par
\B\4\D\\{fn\_decl}\5
\T{25}\C{ denotes a function declarator }\par
\B\4\D\\{semi}\5
\T{27}\C{ denotes a semicolon }\par
\B\4\D\\{colon}\5
\T{28}\C{ denotes a colon }\par
\B\4\D\\{tag}\5
\T{29}\C{ denotes a statement label }\par
\B\4\D\\{if\_head}\5
\T{30}\C{ denotes the beginning of a compound conditional }\par
\B\4\D\\{else\_head}\5
\T{31}\C{ denotes a prefix for a compound statement }\par
\B\4\D\\{if\_clause}\5
\T{32}\C{ pending \.{if} together with a condition }\par
\B\4\D\\{lproc}\5
\T{35}\C{ begins a preprocessor command }\par
\B\4\D\\{rproc}\5
\T{36}\C{ ends a preprocessor command }\par
\B\4\D\\{insert}\5
\T{37}\C{ a scrap that gets combined with its neighbor }\par
\B\4\D\\{section\_scrap}\5
\T{38}\C{ section name }\par
\B\4\D\\{dead}\5
\T{39}\C{ scrap that won't combine }\par
\B\4\D\\{ftemplate}\5
\T{63}\C{ \\{make\_pair} }\par
\B\4\D\\{new\_exp}\5
\T{64}\C{ \&{new} and a following type identifier }\par
\B\4\D\\{begin\_arg}\5
\T{65}\C{ \.{@[} }\par
\B\4\D\\{end\_arg}\5
\T{66}\C{ \.{@]} }\par
\B\4\D\\{lbrack}\5
\T{67}\C{ denotes a left bracket }\par
\B\4\D\\{rbrack}\5
\T{68}\C{ denotes a right bracket }\par
\B\4\D\\{attr\_head}\5
\T{69}\C{ denotes beginning of attribute }\par
\Y\B\4\X21:Private variables\X${}\mathrel+\E{}$\6
\&{static} \&{char} \\{cat\_name}[\T{256}][\T{12}];\C{ \PB{$\T{12}\E\\{strlen}(%
\.{"struct\_head"})+\T{1}$} }\par
\fi

\M{107}\B\X24:Set initial values\X${}\mathrel+\E{}$\6
${}\{{}$\1\6
\&{int} \|c;\7
\&{for} ${}(\|c\K\T{0};{}$ ${}\|c<\T{256};{}$ ${}\|c\PP){}$\1\5
${}\\{strcpy}(\\{cat\_name}[\|c],\39\.{"UNKNOWN"});{}$\2\6
\4${}\}{}$\2\6
${}\\{strcpy}(\\{cat\_name}[\\{exp}],\39\.{"exp"});{}$\6
${}\\{strcpy}(\\{cat\_name}[\\{unop}],\39\.{"unop"});{}$\6
${}\\{strcpy}(\\{cat\_name}[\\{binop}],\39\.{"binop"});{}$\6
${}\\{strcpy}(\\{cat\_name}[\\{ubinop}],\39\.{"ubinop"});{}$\6
${}\\{strcpy}(\\{cat\_name}[\\{cast}],\39\.{"cast"});{}$\6
${}\\{strcpy}(\\{cat\_name}[\\{question}],\39\.{"?"});{}$\6
${}\\{strcpy}(\\{cat\_name}[\\{lbrace}],\39\.{"\{"});{}$\6
${}\\{strcpy}(\\{cat\_name}[\\{rbrace}],\39\.{"\}"});{}$\6
${}\\{strcpy}(\\{cat\_name}[\\{decl\_head}],\39\.{"decl\_head"});{}$\6
${}\\{strcpy}(\\{cat\_name}[\\{comma}],\39\.{","});{}$\6
${}\\{strcpy}(\\{cat\_name}[\\{lpar}],\39\.{"("});{}$\6
${}\\{strcpy}(\\{cat\_name}[\\{rpar}],\39\.{")"});{}$\6
${}\\{strcpy}(\\{cat\_name}[\\{prelangle}],\39\.{"<"});{}$\6
${}\\{strcpy}(\\{cat\_name}[\\{prerangle}],\39\.{">"});{}$\6
${}\\{strcpy}(\\{cat\_name}[\\{langle}],\39\.{"\\\\<"});{}$\6
${}\\{strcpy}(\\{cat\_name}[\\{colcol}],\39\.{"::"});{}$\6
${}\\{strcpy}(\\{cat\_name}[\\{base}],\39\.{"\\\\:"});{}$\6
${}\\{strcpy}(\\{cat\_name}[\\{decl}],\39\.{"decl"});{}$\6
${}\\{strcpy}(\\{cat\_name}[\\{struct\_head}],\39\.{"struct\_head"});{}$\6
${}\\{strcpy}(\\{cat\_name}[\\{alfop}],\39\.{"alfop"});{}$\6
${}\\{strcpy}(\\{cat\_name}[\\{stmt}],\39\.{"stmt"});{}$\6
${}\\{strcpy}(\\{cat\_name}[\\{function}],\39\.{"function"});{}$\6
${}\\{strcpy}(\\{cat\_name}[\\{fn\_decl}],\39\.{"fn\_decl"});{}$\6
${}\\{strcpy}(\\{cat\_name}[\\{else\_like}],\39\.{"else\_like"});{}$\6
${}\\{strcpy}(\\{cat\_name}[\\{semi}],\39\.{";"});{}$\6
${}\\{strcpy}(\\{cat\_name}[\\{colon}],\39\.{":"});{}$\6
${}\\{strcpy}(\\{cat\_name}[\\{tag}],\39\.{"tag"});{}$\6
${}\\{strcpy}(\\{cat\_name}[\\{if\_head}],\39\.{"if\_head"});{}$\6
${}\\{strcpy}(\\{cat\_name}[\\{else\_head}],\39\.{"else\_head"});{}$\6
${}\\{strcpy}(\\{cat\_name}[\\{if\_clause}],\39\.{"if()"});{}$\6
${}\\{strcpy}(\\{cat\_name}[\\{lproc}],\39\.{"\#\{"});{}$\6
${}\\{strcpy}(\\{cat\_name}[\\{rproc}],\39\.{"\#\}"});{}$\6
${}\\{strcpy}(\\{cat\_name}[\\{insert}],\39\.{"insert"});{}$\6
${}\\{strcpy}(\\{cat\_name}[\\{section\_scrap}],\39\.{"section"});{}$\6
${}\\{strcpy}(\\{cat\_name}[\\{dead}],\39\.{"@d"});{}$\6
${}\\{strcpy}(\\{cat\_name}[\\{public\_like}],\39\.{"public"});{}$\6
${}\\{strcpy}(\\{cat\_name}[\\{operator\_like}],\39\.{"operator"});{}$\6
${}\\{strcpy}(\\{cat\_name}[\\{new\_like}],\39\.{"new"});{}$\6
${}\\{strcpy}(\\{cat\_name}[\\{catch\_like}],\39\.{"catch"});{}$\6
${}\\{strcpy}(\\{cat\_name}[\\{for\_like}],\39\.{"for"});{}$\6
${}\\{strcpy}(\\{cat\_name}[\\{do\_like}],\39\.{"do"});{}$\6
${}\\{strcpy}(\\{cat\_name}[\\{if\_like}],\39\.{"if"});{}$\6
${}\\{strcpy}(\\{cat\_name}[\\{delete\_like}],\39\.{"delete"});{}$\6
${}\\{strcpy}(\\{cat\_name}[\\{raw\_ubin}],\39\.{"ubinop?"});{}$\6
${}\\{strcpy}(\\{cat\_name}[\\{const\_like}],\39\.{"const"});{}$\6
${}\\{strcpy}(\\{cat\_name}[\\{raw\_int}],\39\.{"raw"});{}$\6
${}\\{strcpy}(\\{cat\_name}[\\{int\_like}],\39\.{"int"});{}$\6
${}\\{strcpy}(\\{cat\_name}[\\{case\_like}],\39\.{"case"});{}$\6
${}\\{strcpy}(\\{cat\_name}[\\{sizeof\_like}],\39\.{"sizeof"});{}$\6
${}\\{strcpy}(\\{cat\_name}[\\{struct\_like}],\39\.{"struct"});{}$\6
${}\\{strcpy}(\\{cat\_name}[\\{typedef\_like}],\39\.{"typedef"});{}$\6
${}\\{strcpy}(\\{cat\_name}[\\{define\_like}],\39\.{"define"});{}$\6
${}\\{strcpy}(\\{cat\_name}[\\{template\_like}],\39\.{"template"});{}$\6
${}\\{strcpy}(\\{cat\_name}[\\{ftemplate}],\39\.{"ftemplate"});{}$\6
${}\\{strcpy}(\\{cat\_name}[\\{new\_exp}],\39\.{"new\_exp"});{}$\6
${}\\{strcpy}(\\{cat\_name}[\\{begin\_arg}],\39\.{"@["});{}$\6
${}\\{strcpy}(\\{cat\_name}[\\{end\_arg}],\39\.{"@]"});{}$\6
${}\\{strcpy}(\\{cat\_name}[\\{lbrack}],\39\.{"["});{}$\6
${}\\{strcpy}(\\{cat\_name}[\\{rbrack}],\39\.{"]"});{}$\6
${}\\{strcpy}(\\{cat\_name}[\\{attr\_head}],\39\.{"attr\_head"});{}$\6
${}\\{strcpy}(\\{cat\_name}[\\{attr}],\39\.{"attr"});{}$\6
${}\\{strcpy}(\\{cat\_name}[\\{alignas\_like}],\39\.{"alignas"});{}$\6
${}\\{strcpy}(\\{cat\_name}[\\{using\_like}],\39\.{"using"});{}$\6
${}\\{strcpy}(\\{cat\_name}[\\{default\_like}],\39\.{"default"});{}$\6
${}\\{strcpy}(\\{cat\_name}[\T{0}],\39\.{"zero"}){}$;\par
\fi

\M{108}This code allows \.{CWEAVE} to display its parsing steps.

\Y\B\4\D\\{print\_cat}$(\|c)$\5
$\\{fputs}(\\{cat\_name}[\|c],\39\\{stdout}{}$)\C{ symbolic printout of a
category }\par
\fi

\M{109}The token lists for translated \TEX/ output contain some special control
symbols as well as ordinary characters. These control symbols are
interpreted by \.{CWEAVE} before they are written to the output file.

\yskip\hang \PB{\\{break\_space}} denotes an optional line break or an en
space;

\yskip\hang \PB{\\{force}} denotes a line break;

\yskip\hang \PB{\\{big\_force}} denotes a line break with additional vertical
space;

\yskip\hang \PB{\\{preproc\_line}} denotes that the line will be printed flush
left;

\yskip\hang \PB{\\{opt}} denotes an optional line break (with the continuation
line indented two ems with respect to the normal starting position)---this
code is followed by an integer \PB{\|n}, and the break will occur with penalty
$10n$;

\yskip\hang \PB{\\{backup}} denotes a backspace of one em;

\yskip\hang \PB{\\{cancel}} obliterates any \PB{\\{break\_space}}, \PB{%
\\{opt}}, \PB{\\{force}}, or
\PB{\\{big\_force}} tokens that immediately precede or follow it and also
cancels any
\PB{\\{backup}} tokens that follow it;

\yskip\hang \PB{\\{indent}} causes future lines to be indented one more em;

\yskip\hang \PB{\\{outdent}} causes future lines to be indented one less em;

\yskip\hang \PB{\\{dindent}} causes future lines to be indented two more ems.

\yskip\noindent All of these tokens are removed from the \TEX/ output that
comes from \CEE/ text between \pb\ signs; \PB{\\{break\_space}} and \PB{%
\\{force}} and
\PB{\\{big\_force}} become single spaces in this mode. The translation of other
\CEE/ texts results in \TEX/ control sequences \.{\\1}, \.{\\2},
\.{\\3}, \.{\\4}, \.{\\5}, \.{\\6}, \.{\\7}, \.{\\8}
corresponding respectively to
\PB{\\{indent}}, \PB{\\{outdent}}, \PB{\\{opt}}, \PB{\\{backup}}, \PB{\\{break%
\_space}}, \PB{\\{force}},
\PB{\\{big\_force}} and \PB{\\{preproc\_line}}.
However, a sequence of consecutive `\.\ ', \PB{\\{break\_space}},
\PB{\\{force}}, and/or \PB{\\{big\_force}} tokens is first replaced by a single
token
(the maximum of the given ones).

A \PB{\\{dindent}} token becomes \.{\\1\\1}. It is equivalent to a pair of \PB{%
\\{indent}}
tokens. However, if \PB{\\{dindent}} immediately precedes \PB{\\{big\_force}},
the two tokens
are swapped, so that the indentation happens after the line break.

The token \PB{\\{math\_rel}} will be translated into
\.{\\MRL\{}, and it will get a matching \.\} later.
Other control sequences in the \TEX/ output will be
`\.{\\\\\{}$\,\ldots\,$\.\}'
surrounding identifiers, `\.{\\\&\{}$\,\ldots\,$\.\}' surrounding
reserved words, `\.{\\.\{}$\,\ldots\,$\.\}' surrounding strings,
`\.{\\C\{}$\,\ldots\,$\.\}$\,$\PB{\\{force}}' surrounding comments, and
`\.{\\X$n$:}$\,\ldots\,$\.{\\X}' surrounding section names, where
\PB{\|n} is the section number.

\Y\B\4\D\\{math\_rel}\5
\T{\~206}\par
\B\4\D\\{big\_cancel}\5
\T{\~210}\C{ like \PB{\\{cancel}}, also overrides spaces }\par
\B\4\D\\{cancel}\5
\T{\~211}\C{ overrides \PB{\\{backup}}, \PB{\\{break\_space}}, \PB{\\{force}}, %
\PB{\\{big\_force}} }\par
\B\4\D\\{indent}\5
\T{\~212}\C{ one more tab (\.{\\1}) }\par
\B\4\D\\{outdent}\5
\T{\~213}\C{ one less tab (\.{\\2}) }\par
\B\4\D\\{opt}\5
\T{\~214}\C{ optional break in mid-statement (\.{\\3}) }\par
\B\4\D\\{backup}\5
\T{\~215}\C{ stick out one unit to the left (\.{\\4}) }\par
\B\4\D\\{break\_space}\5
\T{\~216}\C{ optional break between statements (\.{\\5}) }\par
\B\4\D\\{force}\5
\T{\~217}\C{ forced break between statements (\.{\\6}) }\par
\B\4\D\\{big\_force}\5
\T{\~220}\C{ forced break with additional space (\.{\\7}) }\par
\B\4\D\\{preproc\_line}\5
\T{\~221}\C{ begin line without indentation (\.{\\8}) }\par
\B\4\D\\{quoted\_char}\5
\T{\~222}\C{ introduces a character token in the range \PB{\T{\~200}}--\PB{\T{%
\~377}} }\par
\B\4\D\\{end\_translation}\5
\T{\~223}\C{ special sentinel token at end of list }\par
\B\4\D\\{inserted}\5
\T{\~224}\C{ sentinel to mark translations of inserts }\par
\B\4\D\\{qualifier}\5
\T{\~225}\C{ introduces an explicit namespace qualifier }\par
\B\4\D\\{dindent}\5
\T{\~226}\C{ two more tabs (\.{\\1\\1}) }\par
\fi

\M{110}The raw input is converted into scraps according to the following table,
which gives category codes followed by the translations.
\def\stars {\.{**}}%
The symbol `\stars' stands for `\.{\\\&\{{\rm identifier}\}}',
i.e., the identifier itself treated as a reserved word.
The right-hand column is the so-called \PB{\\{mathness}}, which is explained
further below.

An identifier \PB{\|c} of length 1 is translated as \.{\\\v c} instead of
as \.{\\\\\{c\}}. An identifier \.{CAPS} in all caps is translated as
\.{\\.\{CAPS\}} instead of as \.{\\\\\{CAPS\}}. An identifier that has
become a reserved word via \PB{\&{typedef}} is translated with \.{\\\&}
replacing
\.{\\\\} and \PB{\\{raw\_int}} replacing \PB{\\{exp}}.

A string of length greater than 20 is broken into pieces of size at most~20
with discretionary breaks in between.

\yskip\halign{\quad#\hfil&\quad#\hfil&\quad\hfil#\hfil\cr
\.{!=}&\PB{\\{binop}}: \.{\\I}&yes\cr
\.{<=}&\PB{\\{binop}}: \.{\\Z}&yes\cr
\.{>=}&\PB{\\{binop}}: \.{\\G}&yes\cr
\.{==}&\PB{\\{binop}}: \.{\\E}&yes\cr
\.{\&\&}&\PB{\\{binop}}: \.{\\W}&yes\cr
\.{\v\v}&\PB{\\{binop}}: \.{\\V}&yes\cr
\.{++}&\PB{\\{unop}}: \.{\\PP}&yes\cr
\.{--}&\PB{\\{unop}}: \.{\\MM}&yes\cr
\.{->}&\PB{\\{binop}}: \.{\\MG}&yes\cr
\.{>>}&\PB{\\{binop}}: \.{\\GG}&yes\cr
\.{<<}&\PB{\\{binop}}: \.{\\LL}&yes\cr
\.{::}&\PB{\\{colcol}}: \.{\\DC}&maybe\cr
\.{.*}&\PB{\\{binop}}: \.{\\PA}&yes\cr
\.{->*}&\PB{\\{binop}}: \.{\\MGA}&yes\cr
\.{...}&\PB{\\{raw\_int}}: \.{\\,\\ldots\\,}&yes\cr
\."string\."&\PB{\\{exp}}: \.{\\.\{}string with special characters quoted\.%
\}&maybe\cr
\.{@=}string\.{@>}&\PB{\\{exp}}: \.{\\vb\{}string with special characters
quoted\.\}&maybe\cr
\.{@'7'}&\PB{\\{exp}}: \.{\\.\{@'7'\}}&maybe\cr
\.{077} or \.{\\77}&\PB{\\{exp}}: \.{\\T\{\\\~77\}}&maybe\cr
\.{0x7f}&\PB{\\{exp}}: \.{\\T\{\\\^7f\}}&maybe\cr
\.{0b10111}&\PB{\\{exp}}: \.{\\T\{\\\\10111\}}&maybe\cr
\.{77}&\PB{\\{exp}}: \.{\\T\{77\}}&maybe\cr
\.{77L}&\PB{\\{exp}}: \.{\\T\{77\\\$L\}}&maybe\cr
\.{0.1E5}&\PB{\\{exp}}: \.{\\T\{0.1\\\_5\}}&maybe\cr
\.{0x10p3}&\PB{\\{exp}}: \.{\\T\{\\\^10\}\\p\{3\}}&maybe\cr
\.{1'000'000}&\PB{\\{exp}}: \.{\\T\{1\\\ 000\\\ 000\}}&maybe\cr
\.+&\PB{\\{ubinop}}: \.+&yes\cr
\.-&\PB{\\{ubinop}}: \.-&yes\cr
\.*&\PB{\\{raw\_ubin}}: \.*&yes\cr
\./&\PB{\\{binop}}: \./&yes\cr
\.<&\PB{\\{prelangle}}: \.{\\langle}&yes\cr
\.=&\PB{\\{binop}}: \.{\\K}&yes\cr
\.>&\PB{\\{prerangle}}: \.{\\rangle}&yes\cr
\..&\PB{\\{binop}}: \..&yes\cr
\.{\v}&\PB{\\{binop}}: \.{\\OR}&yes\cr
\.\^&\PB{\\{binop}}: \.{\\XOR}&yes\cr
\.\%&\PB{\\{binop}}: \.{\\MOD}&yes\cr
\.?&\PB{\\{question}}: \.{\\?}&yes\cr
\.!&\PB{\\{unop}}: \.{\\R}&yes\cr
\.\~&\PB{\\{unop}}: \.{\\CM}&yes\cr
\.\&&\PB{\\{raw\_ubin}}: \.{\\AND}&yes\cr
\.(&\PB{\\{lpar}}: \.(&maybe\cr
\.)&\PB{\\{rpar}}: \.)&maybe\cr
\.[&\PB{\\{lbrack}}: \.[&maybe\cr
\.]&\PB{\\{rbrack}}: \.]&maybe\cr
\.\{&\PB{\\{lbrace}}: \.\{&yes\cr
\.\}&\PB{\\{lbrace}}: \.\}&yes\cr
\.,&\PB{\\{comma}}: \.,&yes\cr
\.;&\PB{\\{semi}}: \.;&maybe\cr
\.:&\PB{\\{colon}}: \.:&no\cr
\.\# (within line)&\PB{\\{ubinop}}: \.{\\\#}&yes\cr
\.\# (at beginning)&\PB{\\{lproc}}: \PB{\\{force}} \PB{\\{preproc\_line}} \.{%
\\\#}&no\cr
end of \.\# line&\PB{\\{rproc}}: \PB{\\{force}}&no\cr
identifier&\PB{\\{exp}}: \.{\\\\\{}identifier with underlines and
dollar signs quoted\.\}&maybe\cr
\.{alignas}&\PB{\\{alignas\_like}}: \stars&maybe\cr
\.{alignof}&\PB{\\{sizeof\_like}}: \stars&maybe\cr
\.{and}&\PB{\\{alfop}}: \stars&yes\cr
\.{and\_eq}&\PB{\\{alfop}}: \stars&yes\cr
\.{asm}&\PB{\\{sizeof\_like}}: \stars&maybe\cr
\.{auto}&\PB{\\{int\_like}}: \stars&maybe\cr
\.{bitand}&\PB{\\{alfop}}: \stars&yes\cr
\.{bitor}&\PB{\\{alfop}}: \stars&yes\cr
\.{bool}&\PB{\\{raw\_int}}: \stars&maybe\cr
\.{break}&\PB{\\{case\_like}}: \stars&maybe\cr
\.{case}&\PB{\\{case\_like}}: \stars&maybe\cr
\.{catch}&\PB{\\{catch\_like}}: \stars&maybe\cr
\.{char}&\PB{\\{raw\_int}}: \stars&maybe\cr
\.{char8\_t}&\PB{\\{raw\_int}}: \stars&maybe\cr
\.{char16\_t}&\PB{\\{raw\_int}}: \stars&maybe\cr
\.{char32\_t}&\PB{\\{raw\_int}}: \stars&maybe\cr
\.{class}&\PB{\\{struct\_like}}: \stars&maybe\cr
\.{clock\_t}&\PB{\\{raw\_int}}: \stars&maybe\cr
\.{compl}&\PB{\\{alfop}}: \stars&yes\cr
\.{complex}&\PB{\\{int\_like}}: \stars&yes\cr
\.{concept}&\PB{\\{int\_like}}: \stars&maybe\cr
\.{const}&\PB{\\{const\_like}}: \stars&maybe\cr
\.{consteval}&\PB{\\{const\_like}}: \stars&maybe\cr
\.{constexpr}&\PB{\\{const\_like}}: \stars&maybe\cr
\.{constinit}&\PB{\\{const\_like}}: \stars&maybe\cr
\.{const\_cast}&\PB{\\{raw\_int}}: \stars&maybe\cr
\.{continue}&\PB{\\{case\_like}}: \stars&maybe\cr
\.{co\_await}&\PB{\\{case\_like}}: \stars&maybe\cr
\.{co\_return}&\PB{\\{case\_like}}: \stars&maybe\cr
\.{co\_yield}&\PB{\\{case\_like}}: \stars&maybe\cr
\.{decltype}&\PB{\\{sizeof\_like}}: \stars&maybe\cr
\.{default}&\PB{\\{default\_like}}: \stars&maybe\cr
\.{define}&\PB{\\{define\_like}}: \stars&maybe\cr
\.{defined}&\PB{\\{sizeof\_like}}: \stars&maybe\cr
\.{delete}&\PB{\\{delete\_like}}: \stars&maybe\cr
\.{div\_t}&\PB{\\{raw\_int}}: \stars&maybe\cr
\.{do}&\PB{\\{do\_like}}: \stars&maybe\cr
\.{double}&\PB{\\{raw\_int}}: \stars&maybe\cr
\.{dynamic\_cast}&\PB{\\{raw\_int}}: \stars&maybe\cr
\.{elif}&\PB{\\{if\_like}}: \stars&maybe\cr
\.{else}&\PB{\\{else\_like}}: \stars&maybe\cr
\.{endif}&\PB{\\{if\_like}}: \stars&maybe\cr
\.{enum}&\PB{\\{struct\_like}}: \stars&maybe\cr
\.{error}&\PB{\\{if\_like}}: \stars&maybe\cr
\.{explicit}&\PB{\\{int\_like}}: \stars&maybe\cr
\.{export}&\PB{\\{int\_like}}: \stars&maybe\cr
\.{extern}&\PB{\\{int\_like}}: \stars&maybe\cr
\.{FILE}&\PB{\\{raw\_int}}: \stars&maybe\cr
\.{false}&\PB{\\{normal}}: \stars&maybe\cr
\.{float}&\PB{\\{raw\_int}}: \stars&maybe\cr
\.{for}&\PB{\\{for\_like}}: \stars&maybe\cr
\.{fpos\_t}&\PB{\\{raw\_int}}: \stars&maybe\cr
\.{friend}&\PB{\\{int\_like}}: \stars&maybe\cr
\.{goto}&\PB{\\{case\_like}}: \stars&maybe\cr
\.{if}&\PB{\\{if\_like}}: \stars&maybe\cr
\.{ifdef}&\PB{\\{if\_like}}: \stars&maybe\cr
\.{ifndef}&\PB{\\{if\_like}}: \stars&maybe\cr
\.{imaginary}&\PB{\\{int\_like}}: \stars&maybe\cr
\.{include}&\PB{\\{if\_like}}: \stars&maybe\cr
\.{inline}&\PB{\\{int\_like}}: \stars&maybe\cr
\.{int}&\PB{\\{raw\_int}}: \stars&maybe\cr
\.{jmp\_buf}&\PB{\\{raw\_int}}: \stars&maybe\cr
\.{ldiv\_t}&\PB{\\{raw\_int}}: \stars&maybe\cr
\.{line}&\PB{\\{if\_like}}: \stars&maybe\cr
\.{long}&\PB{\\{raw\_int}}: \stars&maybe\cr
\.{make\_pair}&\PB{\\{ftemplate}}: \.{\\\\\{make\\\_pair\}}&maybe\cr
\.{mutable}&\PB{\\{int\_like}}: \stars&maybe\cr
\.{namespace}&\PB{\\{struct\_like}}: \stars&maybe\cr
\.{new}&\PB{\\{new\_like}}: \stars&maybe\cr
\.{noexcept}&\PB{\\{attr}}: \stars&maybe\cr
\.{not}&\PB{\\{alfop}}: \stars&yes\cr
\.{not\_eq}&\PB{\\{alfop}}: \stars&yes\cr
\.{NULL}&\PB{\\{exp}}: \.{\\NULL}&yes\cr
\.{nullptr}&\PB{\\{exp}}: \.{\\NULL}&yes\cr
\.{offsetof}&\PB{\\{raw\_int}}: \stars&maybe\cr
\.{operator}&\PB{\\{operator\_like}}: \stars&maybe\cr
\.{or}&\PB{\\{alfop}}: \stars&yes\cr
\.{or\_eq}&\PB{\\{alfop}}: \stars&yes\cr
\.{pragma}&\PB{\\{if\_like}}: \stars&maybe\cr
\.{private}&\PB{\\{public\_like}}: \stars&maybe\cr
\.{protected}&\PB{\\{public\_like}}: \stars&maybe\cr
\.{ptrdiff\_t}&\PB{\\{raw\_int}}: \stars&maybe\cr
\.{public}&\PB{\\{public\_like}}: \stars&maybe\cr
\.{register}&\PB{\\{int\_like}}: \stars&maybe\cr
\.{reinterpret\_cast}&\PB{\\{raw\_int}}: \stars&maybe\cr
\.{requires}&\PB{\\{int\_like}}: \stars&maybe\cr
\.{restrict}&\PB{\\{int\_like}}: \stars&maybe\cr
\.{return}&\PB{\\{case\_like}}: \stars&maybe\cr
\.{short}&\PB{\\{raw\_int}}: \stars&maybe\cr
\.{sig\_atomic\_t}&\PB{\\{raw\_int}}: \stars&maybe\cr
\.{signed}&\PB{\\{raw\_int}}: \stars&maybe\cr
\.{size\_t}&\PB{\\{raw\_int}}: \stars&maybe\cr
\.{sizeof}&\PB{\\{sizeof\_like}}: \stars&maybe\cr
\.{static}&\PB{\\{int\_like}}: \stars&maybe\cr
\.{static\_assert}&\PB{\\{sizeof\_like}}: \stars&maybe\cr
\.{static\_cast}&\PB{\\{raw\_int}}: \stars&maybe\cr
\.{struct}&\PB{\\{struct\_like}}: \stars&maybe\cr
\.{switch}&\PB{\\{for\_like}}: \stars&maybe\cr
\.{template}&\PB{\\{template\_like}}: \stars&maybe\cr
\.{TeX}&\PB{\\{exp}}: \.{\\TeX}&yes\cr
\.{this}&\PB{\\{exp}}: \.{\\this}&yes\cr
\.{thread\_local}&\PB{\\{raw\_int}}: \stars&maybe\cr
\.{throw}&\PB{\\{case\_like}}: \stars&maybe\cr
\.{time\_t}&\PB{\\{raw\_int}}: \stars&maybe\cr
\.{try}&\PB{\\{else\_like}}: \stars&maybe\cr
\.{typedef}&\PB{\\{typedef\_like}}: \stars&maybe\cr
\.{typeid}&\PB{\\{sizeof\_like}}: \stars&maybe\cr
\.{typename}&\PB{\\{struct\_like}}: \stars&maybe\cr
\.{undef}&\PB{\\{if\_like}}: \stars&maybe\cr
\.{union}&\PB{\\{struct\_like}}: \stars&maybe\cr
\.{unsigned}&\PB{\\{raw\_int}}: \stars&maybe\cr
\.{using}&\PB{\\{using\_like}}: \stars&maybe\cr
\.{va\_dcl}&\PB{\\{decl}}: \stars&maybe\cr
\.{va\_list}&\PB{\\{raw\_int}}: \stars&maybe\cr
\.{virtual}&\PB{\\{int\_like}}: \stars&maybe\cr
\.{void}&\PB{\\{raw\_int}}: \stars&maybe\cr
\.{volatile}&\PB{\\{const\_like}}: \stars&maybe\cr
\.{wchar\_t}&\PB{\\{raw\_int}}: \stars&maybe\cr
\.{while}&\PB{\\{for\_like}}: \stars&maybe\cr
\.{xor}&\PB{\\{alfop}}: \stars&yes\cr
\.{xor\_eq}&\PB{\\{alfop}}: \stars&yes\cr
\.{@,}&\PB{\\{insert}}: \.{\\,}&maybe\cr
\.{@\v}&\PB{\\{insert}}: \PB{\\{opt}} \.0&maybe\cr
\.{@/}&\PB{\\{insert}}: \PB{\\{force}}&no\cr
\.{@\#}&\PB{\\{insert}}: \PB{\\{big\_force}}&no\cr
\.{@+}&\PB{\\{insert}}: \PB{\\{big\_cancel}} \.{\{\}} \PB{\\{break\_space}}
\.{\{\}} \PB{\\{big\_cancel}}&no\cr
\.{@;}&\PB{\\{semi}}: &maybe\cr
\.{@[}&\PB{\\{begin\_arg}}: &maybe\cr
\.{@]}&\PB{\\{end\_arg}}: &maybe\cr
\.{@\&}&\PB{\\{insert}}: \.{\\J}&maybe\cr
\.{@h}&\PB{\\{insert}}: \PB{\\{force}} \.{\\ATH} \PB{\\{force}}&no\cr
\.{@<}\thinspace section name\thinspace\.{@>}&\PB{\\{section\_scrap}}:
\.{\\X}$n$\.:translated section name\.{\\X}&maybe\cr
\.{@(}\thinspace section name\thinspace\.{@>}&\PB{\\{section\_scrap}}:
\.{\\X}$n$\.{:\\.\{}section name with special characters
quoted\.{\ \}\\X}&maybe\cr
\.{/*}\thinspace comment\thinspace\.{*/}&\PB{\\{insert}}: \PB{\\{cancel}}
\.{\\C\{}translated comment\.\} \PB{\\{force}}&no\cr
\.{//}\thinspace comment&\PB{\\{insert}}: \PB{\\{cancel}}
\.{\\SHC\{}translated comment\.\} \PB{\\{force}}&no\cr
}

\smallskip
The construction \.{@t}\thinspace stuff\/\thinspace\.{@>} contributes
\.{\\hbox\{}\thinspace stuff\/\thinspace\.\} to the following scrap.

% This file is part of CWEB.
% This program by Silvio Levy and Donald E. Knuth
% is based on a program by Knuth.
% It is distributed WITHOUT ANY WARRANTY, express or implied.
% Version 4.7 --- February 2022
%
\fi

\M{111}Here is a table of all the productions.  Each production that
combines two or more consecutive scraps implicitly inserts a {\tt \$}
where necessary, that is, between scraps whose abutting boundaries
have different \PB{\\{mathness}}.  In this way we never get double {\tt\$\$}.

% The following kludge is needed because \newcount, \newdimen, and \+
% are "\outer" control sequences that cannot be used in skipped text!
\fi \newcount\prodno \newdimen\midcol \let\+\relax \ifon

\def\v{\char'174}
\mathchardef\RA="3221 % right arrow
\mathchardef\BA="3224 % double arrow

A translation is provided when the resulting scrap is not merely a
juxtaposition of the scraps it comes from.  An asterisk$^*$ next to a scrap
means that its first identifier gets an underlined entry in the index,
via the function \PB{\\{make\_underlined}}.  Two asterisks$^{**}$ means that
both
\PB{\\{make\_underlined}} and \PB{\\{make\_reserved}} are called; that is, the
identifier's ilk becomes \PB{\\{raw\_int}}.  A dagger \dag\ before the
production number refers to the notes at the end of this section,
which deal with various exceptional cases.

We use \\{in}, \\{out}, \\{back}, \\{bsp}, and \\{din} as shorthands for
\PB{\\{indent}}, \PB{\\{outdent}}, \PB{\\{backup}}, \PB{\\{break\_space}}, and %
\PB{\\{dindent}}, respectively.

\begingroup \lineskip=4pt
\def\alt #1 #2
{$\displaystyle\Bigl\{\!\matrix{\strut\hbox{#1}\cr
\strut\hbox{#2}\cr}\!\Bigr\}$ }
\def\altt #1 #2 #3
{$\displaystyle\Biggl\{\!\matrix{\strut\hbox{#1}\cr\hbox{#2}\cr
\strut\hbox{#3}\cr}\!\Biggr\}$ }
\def\malt #1 #2
{$\displaystyle\!\matrix{\strut\hbox{#1}\hfill\cr\strut\hbox{#2}\hfill\cr}$}
\def\maltt #1 #2 #3
{$\displaystyle\!\matrix{\strut\hbox{#1}\hfill\cr\hbox{#2}\hfill\cr
\strut\hbox{#3}\hfill\cr}$}
\yskip
\prodno=0 \midcol=2.5in
\def\theprodno{\number\prodno \global\advance\prodno by1\enspace}
\def\dagit{\dag\theprodno}
\def\+#1&#2&#3&#4\cr{\def\next{#1}%
\line{\hbox to 2em{\hss
\ifx\next\empty\theprodno\else\next\fi}\strut
\ignorespaces#2\hfil\hbox to\midcol{$\RA$
\ignorespaces#3\hfil}\quad \hbox to1.45in{\ignorespaces#4\hfil}}}
\+\relax & LHS & RHS \hfill Translation & Example\cr
\yskip
\+& \altt\\{any} {\\{any} \\{any}} {\\{any} \\{any} \\{any}}
\PB{\\{insert}} & \altt\\{any} {\\{any} \\{any}} {\\{any} \\{any} \\{any}}
& stmt; \4\4 \C{comment}\cr
\+& \PB{\\{exp}} \altt\PB{\\{lbrace}} \PB{\\{int\_like}} \PB{\\{decl}}
& \PB{\\{fn\_decl}} \altt\PB{\\{lbrace}} \PB{\\{int\_like}} \PB{\\{decl}} %
\hfill $F=E^*\,\\{din}$
& \malt {\\{main}(\,) $\{$}
{$\\{main}(\\{ac},\\{av}){}$ \&{int} \\{ac};} \cr
\+& \PB{\\{exp}} \PB{\\{unop}} & \PB{\\{exp}} & $x\PP$ \cr
\+& \PB{\\{exp}} \alt \PB{\\{binop}} \PB{\\{ubinop}} \PB{\\{exp}} & \PB{%
\\{exp}} & \malt {$x/y$} {$x+y$} \cr
\+& \PB{\\{exp}} \PB{\\{comma}} \PB{\\{exp}} & \PB{\\{exp}} \hfill $E_1C\,\PB{%
\\{opt}}9\,E_2$ & $f(x,y)$ \cr
\+& \PB{\\{exp}} \alt {\PB{\\{lpar}} \PB{\\{rpar}}} \PB{\\{cast}} \PB{%
\\{colon}}
& \PB{\\{exp}} \alt {\PB{\\{lpar}} \PB{\\{rpar}}} \PB{\\{cast}} \PB{\\{base}}
& \malt {\&C(\,) :} {\&C(\&{int} $i$) :} \cr
\+& \PB{\\{exp}} \PB{\\{semi}} & \PB{\\{stmt}} & $x=0;$ \cr
\+& \PB{\\{exp}} \PB{\\{colon}} & \PB{\\{tag}} \hfill $E^*C$ & \\{found}: \cr
\+& \PB{\\{exp}} \PB{\\{rbrace}} & \PB{\\{stmt}} \PB{\\{rbrace}} & end of %
\&{enum} list\cr
\+& \PB{\\{exp}} \alt {\PB{\\{lpar}} \PB{\\{rpar}}} \PB{\\{cast}} \alt\PB{%
\\{const\_like}} \PB{\\{case\_like}}
& \PB{\\{exp}} \alt {\PB{\\{lpar}} \PB{\\{rpar}}} \PB{\\{cast}} \hfill
\alt $R=R\.\ C$ $C_1=C_1\.\ C_2$
& \malt {$f$(\,) \&{const}} {$f$(\&{int}) \&{throw}} \cr
\+& \PB{\\{exp}} \alt \PB{\\{exp}} \PB{\\{cast}} & \PB{\\{exp}} & \\{time}(\,) %
\cr
\+& \PB{\\{lpar}} \alt \PB{\\{exp}} \PB{\\{ubinop}} \PB{\\{rpar}} & \PB{%
\\{exp}} & \malt{($x$)} {$(*)$} \cr
\+& \PB{\\{lpar}} \PB{\\{rpar}} & \PB{\\{exp}} \hfill $L\.{\\,}R$ & functions,
declarations\cr
\+& \PB{\\{lpar}} \altt \PB{\\{decl\_head}} \PB{\\{int\_like}} \PB{\\{cast}} %
\PB{\\{rpar}} & \PB{\\{cast}} & \PB{(\&{char} ${}{*})$}\cr
\+& \PB{\\{lpar}} \altt \PB{\\{decl\_head}} \PB{\\{int\_like}} \PB{\\{exp}} %
\PB{\\{comma}} & \PB{\\{lpar}} \hfill
$L$\,\altt $D$ $I$ $E$ \unskip $C$\,\PB{\\{opt}}9 & \PB{$(\&{int},$}\cr
\+& \PB{\\{lpar}} \alt \PB{\\{stmt}} \PB{\\{decl}} & \PB{\\{lpar}} \hfill \alt
{$LS\.\ $} {$LD\.\ $}
& \malt {$(k=5;$} {(\&{int} $k=5;$} \cr
\+& \PB{\\{unop}} \alt \PB{\\{exp}} \PB{\\{int\_like}} & \PB{\\{exp}}
& \malt {$\R x$} {$\CM\&C$} \cr
\+& \PB{\\{ubinop}} \PB{\\{cast}} \PB{\\{rpar}} & \PB{\\{cast}} \PB{\\{rpar}} %
\hfill
$C=\.\{U\.\}C$ & \PB{$*$}\&{CPtr}) \cr
\+& \PB{\\{ubinop}} \alt\PB{\\{exp}} \PB{\\{int\_like}} & \alt\PB{\\{exp}} \PB{%
\\{int\_like}} \hfill
\.\{$U$\.\}\alt$E$ $I$ & \malt {${*}x$} {${*}\&{CPtr}$} \cr
\+& \PB{\\{ubinop}} \PB{\\{binop}} & \PB{\\{binop}} \hfill $\PB{\\{math\_rel}}%
\,U\.\{B\.\}\.\}$ & \PB{$\MRL{*{\K}}$}\cr
\+& \PB{\\{binop}} \PB{\\{binop}} & \PB{\\{binop}} \hfill
$\PB{\\{math\_rel}}\,\.\{B_1\.\}\.\{B_2\.\}\.\}$ & \PB{$\MRL{{\GG}{\K}}$}\cr
\+& \PB{\\{cast}} \alt \PB{\\{lpar}} \PB{\\{exp}} & \alt \PB{\\{lpar}} \PB{%
\\{exp}} \hfill
\alt $CL$ $C\.\ E$ & \malt {$(\&{double})(x+2)$} {(\&{double}) $x$} \cr
\+& \PB{\\{cast}} \PB{\\{semi}} & \PB{\\{exp}} \PB{\\{semi}} & \PB{(\&{int});}%
\cr
\+& \PB{\\{sizeof\_like}} \PB{\\{cast}} & \PB{\\{exp}} & \PB{\&{sizeof}(%
\&{double})}\cr
\+& \PB{\\{sizeof\_like}} \PB{\\{exp}} & \PB{\\{exp}} \hfill $S\.\ E$ & %
\&{sizeof} $x$\cr
\+& \PB{\\{int\_like}} \alt\PB{\\{int\_like}} \PB{\\{struct\_like}} &
\alt\PB{\\{int\_like}} \PB{\\{struct\_like}} \hfill $I\.\ $\alt $I$ $S$
\unskip & \PB{\&{extern} \&{char}}\cr
\+& \PB{\\{int\_like}} \PB{\\{exp}} \alt\PB{\\{raw\_int}} \PB{\\{struct\_like}}
&
\PB{\\{int\_like}} \alt\PB{\\{raw\_int}} \PB{\\{struct\_like}} & \PB{\&{extern}%
\.{"Ada"} \&{int}}\cr
\+& \PB{\\{int\_like}} \altt\PB{\\{exp}} \PB{\\{ubinop}} \PB{\\{colon}} &
\PB{\\{decl\_head}} \altt\PB{\\{exp}} \PB{\\{ubinop}} \PB{\\{colon}} \hfill
$D=I$\.\ %
& \maltt {\&{int} $x$} {\&{int} ${}{*}x$} {\&{unsigned} :} \cr
\+& \PB{\\{int\_like}} \alt \PB{\\{semi}} \PB{\\{binop}} & \PB{\\{decl\_head}} %
\alt \PB{\\{semi}} \PB{\\{binop}}
& \malt {\&{int} $x$;} {\&{int} $f(\&{int}=4)$} \cr
\+& \PB{\\{public\_like}} \PB{\\{colon}} & \PB{\\{tag}} & \PB{\&{private}:}\cr
\+& \PB{\\{public\_like}} & \PB{\\{int\_like}} & \PB{\&{private}}\cr
\+& \PB{\\{colcol}} \alt\PB{\\{exp}} \PB{\\{int\_like}} & \alt\PB{\\{exp}} \PB{%
\\{int\_like}} \hfill
\PB{\\{qualifier}} $C$\alt$E$ $I$ & \&C\DC$x$\cr
\+& \PB{\\{colcol}} \PB{\\{colcol}} & \PB{\\{colcol}} & \&C\DC\&B\DC\cr
\+& \PB{\\{decl\_head}} \PB{\\{comma}} & \PB{\\{decl\_head}} \hfill $DC\.\ $ & %
\&{int} $x,{}$ \cr
\+& \PB{\\{decl\_head}} \PB{\\{ubinop}} & \PB{\\{decl\_head}} \hfill $D\.\{U\.%
\}$ & \PB{\&{int} ${}{*}$}\cr
\+\dagit& \PB{\\{decl\_head}} \PB{\\{exp}} & \PB{\\{decl\_head}} \hfill $DE^*$
& \&{int} $x$ \cr
\+& \PB{\\{decl\_head}} \alt\PB{\\{binop}} \PB{\\{colon}} \PB{\\{exp}} \altt%
\PB{\\{comma}} \PB{\\{semi}} \PB{\\{rpar}} &
\PB{\\{decl\_head}} \altt\PB{\\{comma}} \PB{\\{semi}} \PB{\\{rpar}} \hfill
$D=D$\alt $B$ $C$ \unskip $E$
& \malt {\&{int} $f(\&{int}\ x=2)$} {\&{int} $b$ : 1} \cr
\+& \PB{\\{decl\_head}} \PB{\\{cast}} & \PB{\\{decl\_head}} & \&{int} $f$(%
\&{int})\cr
\+& \PB{\\{decl\_head}} \altt\PB{\\{int\_like}} \PB{\\{lbrace}} \PB{\\{decl}} &
\PB{\\{fn\_decl}}
\altt\PB{\\{int\_like}} \PB{\\{lbrace}} \PB{\\{decl}} \hfill $F=D\,\\{din}$
& \&{long} \\{time}(\,) $\{$\cr
\+& \PB{\\{decl\_head}} \PB{\\{semi}} & \PB{\\{decl}} & \&{int} $n$;\cr
\+& \PB{\\{decl}} \PB{\\{decl}} & \PB{\\{decl}} \hfill $D_1\,\PB{\\{force}}%
\,D_2$
& \&{int} $n$; \&{double} $x$;\cr
\+& \PB{\\{decl}} \alt\PB{\\{stmt}} \PB{\\{function}} & \alt\PB{\\{stmt}} \PB{%
\\{function}}
\hfill $D\,\PB{\\{big\_force}}\,$\alt $S$ $F$ \unskip
& \&{extern} $n$; \\{main}(\,) $\{\,\}$\cr
\+& \PB{\\{base}} \alt \PB{\\{int\_like}} \PB{\\{exp}} \PB{\\{comma}} & \PB{%
\\{base}} \hfill
$B$\.\ \alt $I$ $E$ \unskip $C$\,\PB{\\{opt}}9
& \malt {: \&{public} \&A,} {: $i$(5),} \cr
\+& \PB{\\{base}} \alt \PB{\\{int\_like}} \PB{\\{exp}} \PB{\\{lbrace}} & \PB{%
\\{lbrace}} \hfill
$B$\.\ \alt $I$ $E$ \unskip \.\ $L$ & \&D : \&{public} \&A $\{$\cr
\+& \PB{\\{struct\_like}} \PB{\\{lbrace}} & \PB{\\{struct\_head}} \hfill $S\.\
L$ & \PB{\&{struct} ${}\{$}\cr
\+& \PB{\\{struct\_like}} \alt\PB{\\{exp}} \PB{\\{int\_like}} \PB{\\{semi}} & %
\PB{\\{decl\_head}} \PB{\\{semi}}
\hfill $S\.\ $\alt $E^{**}$ $I^{**}$ & \&{struct} \&{forward};\cr
\+& \PB{\\{struct\_like}} \alt\PB{\\{exp}} \PB{\\{int\_like}} \PB{\\{lbrace}} &
\PB{\\{struct\_head}} \hfill
$S\.\ $\alt $E^{**}$ $I^{**}$ \unskip $\.\ L$
& \&{struct} \&{name\_info} $\{$\cr
\+& \PB{\\{struct\_like}} \alt\PB{\\{exp}} \PB{\\{int\_like}} \PB{\\{colon}} &
\PB{\\{struct\_like}} \alt\PB{\\{exp}} \PB{\\{int\_like}} \PB{\\{base}} & \PB{%
\&{class}} \&C :\cr
\+\dagit& \PB{\\{struct\_like}} \alt\PB{\\{exp}} \PB{\\{int\_like}} & \PB{%
\\{int\_like}}
\hfill $S\.\ $\alt$E$ $I$ \unskip & \&{struct} \&{name\_info} $z$;\cr
\+& \PB{\\{struct\_head}} \altt\PB{\\{decl}} \PB{\\{stmt}} \PB{\\{function}} %
\PB{\\{rbrace}} & \PB{\\{int\_like}}\hfill
$S\,\\{in}\,\PB{\\{force}}$\altt$D$ $S$ $F$ \unskip $\\{out}\,\PB{\\{force}}%
\,R$
& \PB{\&{struct} ${}\{$} declaration \PB{$\}$}\cr
\+& \PB{\\{struct\_head}} \PB{\\{rbrace}} & \PB{\\{int\_like}}\hfill $S\.{%
\\,}R$
& \&{class} \&C $\{\,\}$\cr
\+& \PB{\\{fn\_decl}} \PB{\\{decl}} & \PB{\\{fn\_decl}} \hfill $F\,\PB{%
\\{force}}\,D$
& $f(z)$ \&{double} $z$; \cr
\+& \PB{\\{fn\_decl}} \PB{\\{stmt}} & \PB{\\{function}} \hfill $F\,\PB{\\{out}}%
\,\PB{\\{out}}\,\PB{\\{force}}\,S$
& \\{main}() {\dots}\cr
\+& \PB{\\{function}} \altt\PB{\\{stmt}} \PB{\\{decl}} \PB{\\{function}} & %
\altt \PB{\\{stmt}} \PB{\\{decl}} \PB{\\{function}}
\hfill $F\,\PB{\\{big\_force}}\,$\altt $S$ $D$ $F$ \unskip & outer block\cr
\+& \PB{\\{lbrace}} \PB{\\{rbrace}} & \PB{\\{stmt}} \hfill $L\.{\\,}R$ & empty
statement\cr
\advance\midcol35pt
\+& \PB{\\{lbrace}} \altt\PB{\\{stmt}} \PB{\\{decl}} \PB{\\{function}} \PB{%
\\{rbrace}} & \PB{\\{stmt}} \hfill
$\PB{\\{force}}\,L\,\\{in}\,\PB{\\{force}}\,S\,
\PB{\\{force}}\,\\{back}\,R\,\\{out}\,\PB{\\{force}}$ & compound statement\cr
\advance\midcol-20pt
\+& \PB{\\{lbrace}} \PB{\\{exp}} [\PB{\\{comma}}] \PB{\\{rbrace}} & \PB{%
\\{exp}} & initializer\cr
\+& \PB{\\{if\_like}} \PB{\\{exp}} & \PB{\\{if\_clause}} \hfill $I\.{\ }E$ & %
\&{if} ($z$)\cr
\+& \PB{\\{else\_like}} \PB{\\{colon}} & \PB{\\{else\_like}} \PB{\\{base}} & %
\PB{\&{try} :}\cr
\+& \PB{\\{else\_like}} \PB{\\{lbrace}} & \PB{\\{else\_head}} \PB{\\{lbrace}} &
\PB{\&{else} $\{$}\cr
\+& \PB{\\{else\_like}} \PB{\\{stmt}} & \PB{\\{stmt}} \hfill
$\PB{\\{force}}\,E\,\\{in}\,\\{bsp}\,S\,\\{out}\,\PB{\\{force}}$
& \&{else} $x=0;$\cr
\+& \PB{\\{else\_head}} \alt\PB{\\{stmt}} \PB{\\{exp}} & \PB{\\{stmt}} \hfill
$\PB{\\{force}}\,E\,\\{bsp}\,\PB{\\{noop}}\,\PB{\\{cancel}}\,S\,\\{bsp}$
& $\!\!$ \&{else} $\{x=0;\}$\cr
\+& \PB{\\{if\_clause}} \PB{\\{lbrace}} & \PB{\\{if\_head}} \PB{\\{lbrace}} & %
\&{if} ($x$) $\{$\cr
\+& \PB{\\{if\_clause}} \PB{\\{stmt}} \PB{\\{else\_like}} \PB{\\{if\_like}} & %
\PB{\\{if\_like}} \hfill
$\PB{\\{force}}\,I\,\\{in}\,\\{bsp}\,S\,\\{out}\,\PB{\\{force}}\,E\,\.\ I$
& $\!\!$ \&{if} ($x$) $y$; \&{else} \&{if}\cr
\+& \PB{\\{if\_clause}} \PB{\\{stmt}} \PB{\\{else\_like}} & \PB{\\{else\_like}}
\hfill
$\PB{\\{force}}\,I\,\\{in}\,\\{bsp}\,S\,\\{out}\,\PB{\\{force}}\,E$
& $\!\!$ \&{if} ($x$) $y$; \&{else}\cr
\+& \PB{\\{if\_clause}} \PB{\\{stmt}} & \PB{\\{else\_like}} \PB{\\{stmt}} & $\!%
\!$ \&{if} ($x$) $y$;\cr
\+& \PB{\\{if\_head}} \alt\PB{\\{stmt}} \PB{\\{exp}} \PB{\\{else\_like}} \PB{%
\\{if\_like}} & \PB{\\{if\_like}} \hfill
$\PB{\\{force}}\,I\,\\{bsp}\,\PB{\\{noop}}\,\PB{\\{cancel}}\,S\,\PB{\\{force}}%
\,E\,\.\ I$
& $\!\!$ \&{if} ($x$) $\{\,y;\,\}$ \&{else} \&{if}\cr
\+& \PB{\\{if\_head}} \alt\PB{\\{stmt}} \PB{\\{exp}} \PB{\\{else\_like}} & \PB{%
\\{else\_like}} \hfill
$\PB{\\{force}}\,I\,\\{bsp}\,\PB{\\{noop}}\,\PB{\\{cancel}}\,S\,\PB{\\{force}}%
\,E$
& $\!\!$ \&{if} ($x$) $\{\,y;\,\}$ \&{else}\cr
\+& \PB{\\{if\_head}} \alt\PB{\\{stmt}} \PB{\\{exp}} & \PB{\\{else\_head}} \alt%
\PB{\\{stmt}} \PB{\\{exp}}
& $\!\!$ \&{if} ($x$) ${}\{\,y\,\}{}$\cr
\advance\midcol20pt
\+& \PB{\\{do\_like}} \PB{\\{stmt}} \PB{\\{else\_like}} \PB{\\{semi}} & \PB{%
\\{stmt}} \hfill
$D\,\\{bsp}\,\PB{\\{noop}}\,\PB{\\{cancel}}\,S\,\PB{\\{cancel}}\,\PB{\\{noop}}%
\,\\{bsp}\,ES$%
& \&{do} $f$($x$); \&{while} ($g$($x$));\cr
\advance\midcol-20pt
\+& \PB{\\{case\_like}} \PB{\\{semi}} & \PB{\\{stmt}} & \PB{\&{return};}\cr
\+& \PB{\\{case\_like}} \PB{\\{colon}} & \PB{\\{tag}} & \PB{\&{default}:}\cr
\+& \PB{\\{case\_like}} \PB{\\{exp}} & \PB{\\{exp}} \hfill $C\.\ E$ & \PB{%
\&{return} \T{0}}\cr
\+& \PB{\\{catch\_like}} \alt\PB{\\{cast}} \PB{\\{exp}} & \PB{\\{fn\_decl}} %
\hfill
$C$\alt $C$ $E$ \unskip \\{din} & \PB{\&{catch}\1\1${}(\,\ldots\,)$}\cr
\+& \PB{\\{tag}} \PB{\\{tag}} & \PB{\\{tag}} \hfill $T_1\,\\{bsp}\,T_2$ & \PB{%
\&{case} \T{0}: \&{case} \T{1}:}\cr
\+& \PB{\\{tag}} \altt\PB{\\{stmt}} \PB{\\{decl}} \PB{\\{function}} & \altt\PB{%
\\{stmt}} \PB{\\{decl}} \PB{\\{function}}
\hfill $\PB{\\{force}}\,\\{back}\,T\,\\{bsp}\,S$
& $\!\!$ \&{case} 0: $z=0;$\cr
\+\dagit& \PB{\\{stmt}} \altt\PB{\\{stmt}} \PB{\\{decl}} \PB{\\{function}} &
\altt\PB{\\{stmt}} \PB{\\{decl}} \PB{\\{function}}
\hfill $S\,$\altt$\PB{\\{force}}\,S$ $\PB{\\{big\_force}}\,D$ $\PB{\\{big%
\_force}}\,F$ \unskip
& $x=1;$ $y=2;$\cr
\+& \PB{\\{semi}} & \PB{\\{stmt}} \hfill \.\ $S$& empty statement\cr
\+\dagit& \PB{\\{lproc}} \altt \PB{\\{if\_like}} \PB{\\{else\_like}} \PB{%
\\{define\_like}} & \PB{\\{lproc}} &
\maltt {\#\&{include}} \#\&{else} \#\&{define} \cr
\+& \PB{\\{lproc}} \PB{\\{rproc}} & \PB{\\{insert}} & \#\&{endif} \cr
\+& \PB{\\{lproc}} \alt {\PB{\\{exp}} [\PB{\\{exp}}]} \PB{\\{function}} \PB{%
\\{rproc}} & \PB{\\{insert}} \hfill
$I$\.\ \alt {$E{[\.{\ \\5}E]}$} {$F$} &
\malt{\#\&{define} $a$\enspace 1} {\#\&{define} $a$\enspace$\{\,b;\,\}$} \cr
\+& \PB{\\{section\_scrap}} \PB{\\{semi}} & \PB{\\{stmt}}\hfill $MS$ \PB{%
\\{force}}
&$\langle\,$section name$\,\rangle$;\cr
\+& \PB{\\{section\_scrap}} & \PB{\\{exp}} &$\langle\,$section name$\,\rangle$%
\cr
\+& \PB{\\{insert}} \\{any} & \\{any} & \.{\v\#include\v}\cr
\+& \PB{\\{prelangle}} & \PB{\\{binop}} \hfill \.< & $<$ not in template\cr
\+& \PB{\\{prerangle}} & \PB{\\{binop}} \hfill \.> & $>$ not in template\cr
\+& \PB{\\{langle}} \PB{\\{prerangle}} & \PB{\\{cast}} \hfill $L\.{\\,}P$ & $%
\langle\,\rangle$\cr
\+& \PB{\\{langle}} \altt\PB{\\{decl\_head}} \PB{\\{int\_like}} \PB{\\{exp}} %
\PB{\\{prerangle}} & \PB{\\{cast}} &
$\langle\&{class}\,\&C\rangle$\cr
\+& \PB{\\{langle}} \altt\PB{\\{decl\_head}} \PB{\\{int\_like}} \PB{\\{exp}} %
\PB{\\{comma}} & \PB{\\{langle}} \hfill
$L$\,\altt $D$ $I$ $E$ \unskip $C$\,\PB{\\{opt}}9 & $\langle\&{class}\,\&C,$\cr
\+& \PB{\\{template\_like}} \PB{\\{exp}} \PB{\\{prelangle}} & \PB{\\{template%
\_like}} \PB{\\{exp}} \PB{\\{langle}} &
\&{template} $a\langle100\rangle$\cr
\+& \PB{\\{template\_like}} \alt\PB{\\{exp}} \PB{\\{raw\_int}} & \alt\PB{%
\\{exp}} \PB{\\{raw\_int}} \hfill
$T$\.\ \alt$E$ $R$ & \&C\DC\&{template} $a$(\,)\cr
\+& \PB{\\{template\_like}} & \PB{\\{raw\_int}} & \&{template}$\langle\&{class}%
\,\&T\rangle$\cr
\+& \PB{\\{new\_like}} \PB{\\{lpar}} \PB{\\{exp}} \PB{\\{rpar}} & \PB{\\{new%
\_like}} & \&{new}(\\{nothrow})\cr
\+& \PB{\\{new\_like}} \PB{\\{cast}} & \PB{\\{exp}} \hfill $N\.\ C$ & \PB{%
\&{new} (\&{int} ${}{*})$}\cr
\+\dagit& \PB{\\{new\_like}} & \PB{\\{new\_exp}} & \&{new} \&C(\,)\cr
\+& \PB{\\{new\_exp}} \alt\PB{\\{int\_like}} \PB{\\{const\_like}} & \PB{\\{new%
\_exp}} \hfill
$N$\.\ \alt $I$ $C$ & \PB{\&{new} \&{const} \&{int}}\cr
\+& \PB{\\{new\_exp}} \PB{\\{struct\_like}} \alt \PB{\\{exp}} \PB{\\{int%
\_like}} & \PB{\\{new\_exp}} \hfill
$N\.\ S$\.\ \alt $E$ $I$ & \&{new} \&{struct} \&S\cr
\+& \PB{\\{new\_exp}} \PB{\\{raw\_ubin}} & \PB{\\{new\_exp}} \hfill $N\.\{R\.%
\}$ & \PB{\&{new} \&{int}${}{*}[\T{2}]$}\cr
\+& \PB{\\{new\_exp}} \alt \PB{\\{lpar}} \PB{\\{exp}} & \PB{\\{exp}} \alt \PB{%
\\{lpar}} \PB{\\{exp}} \hfill
$E=N$\,\alt {} {\.\ } & \malt {\PB{\&{operator}[\,](\&{int})}} {\PB{\&{new} %
\&{int}(\T{2})}} \cr
\+\dagit& \PB{\\{new\_exp}} & \PB{\\{exp}} & \PB{\&{new} \&{int};}\cr
\+& \PB{\\{ftemplate}} \PB{\\{prelangle}} & \PB{\\{ftemplate}} \PB{\\{langle}}
& \PB{$\\{make\_pair}\langle\&{int},\&{int}\rangle$}\cr
\+& \PB{\\{ftemplate}} & \PB{\\{exp}} & \PB{$\\{make\_pair}(\T{1},\T{2})$}\cr
\+& \PB{\\{for\_like}} \PB{\\{exp}} & \PB{\\{else\_like}} \hfill $F\.\ E$ & %
\PB{\&{while} (\T{1})}\cr
\+& \PB{\\{raw\_ubin}} \PB{\\{const\_like}} & \PB{\\{raw\_ubin}} \hfill $RC$\.{%
\\\ }
& $*$\&{const} $x$\cr
\+& \PB{\\{raw\_ubin}} & \PB{\\{ubinop}} & $*$ $x$\cr
\+& \PB{\\{const\_like}} & \PB{\\{int\_like}} & \&{const} $x$\cr
\+& \PB{\\{raw\_int}} \PB{\\{prelangle}} & \PB{\\{raw\_int}} \PB{\\{langle}} & %
\&C$\langle$\cr
\+& \PB{\\{raw\_int}} \PB{\\{colcol}} & \PB{\\{colcol}} & \&C\DC\cr
\+& \PB{\\{raw\_int}} \PB{\\{cast}} & \PB{\\{raw\_int}} & \&C$\langle\&{class}\
\&T\rangle$\cr
\+& \PB{\\{raw\_int}} \PB{\\{lpar}} & \PB{\\{exp}} \PB{\\{lpar}} & %
\&{complex}$(x,y)$\cr
\+\dagit& \PB{\\{raw\_int}} & \PB{\\{int\_like}}   & \&{complex} $z$\cr
\+\dagit& \PB{\\{operator\_like}} \altt\PB{\\{binop}} \PB{\\{unop}} \PB{%
\\{ubinop}} & \PB{\\{exp}}
\hfill $O$\.\{\altt $B$ $U$ $U$ \unskip \.\} & \PB{$\&{operator}{+}$}\cr
\+& \PB{\\{operator\_like}} \alt\PB{\\{new\_like}} \PB{\\{delete\_like}} & \PB{%
\\{exp}} \hfill
$O$\.\ \alt $N$ $S$ & \PB{\&{operator} \&{delete}}\cr
\+& \PB{\\{operator\_like}} \PB{\\{comma}} & \PB{\\{exp}} & \&{operator},\cr
\+\dagit& \PB{\\{operator\_like}} & \PB{\\{new\_exp}} & \PB{\&{operator} %
\&{char}${}{*}$}\cr
\advance\midcol-8pt
\+& \PB{\\{typedef\_like}} \alt\PB{\\{int\_like}} \PB{\\{cast}} \alt\PB{%
\\{comma}} \PB{\\{semi}} &
\PB{\\{typedef\_like}} \PB{\\{exp}} \alt\PB{\\{comma}} \PB{\\{semi}} & %
\&{typedef} \&{int} \&I,\cr
\advance\midcol+8pt
\+& \PB{\\{typedef\_like}} \PB{\\{int\_like}} & \PB{\\{typedef\_like}} \hfill
$T\.\ I$ &
\&{typedef} \&{char}\cr
\+\dagit& \PB{\\{typedef\_like}} \PB{\\{exp}} & \PB{\\{typedef\_like}} \hfill
$T\.\ E^{**}$ &
\&{typedef} \&I \.{@[@]} (\PB{$*$}\&P)\cr
\+& \PB{\\{typedef\_like}} \PB{\\{comma}} & \PB{\\{typedef\_like}} \hfill $TC\.%
\ $ &
\&{typedef} \&{int} \&x,\cr
\+& \PB{\\{typedef\_like}} \PB{\\{semi}} & \PB{\\{decl}} & \&{typedef} \&{int}
$\&x,\&y$;\cr
\+& \PB{\\{typedef\_like}} \PB{\\{ubinop}} \alt \PB{\\{cast}} \PB{\\{ubinop}} &
\PB{\\{typedef\_like}} \alt \PB{\\{cast}} \PB{\\{ubinop}} \hfill
\alt $C=\.\{U\.\}C$ $U_2=\.\{U_1\.\}U_2$ \unskip &
\&{typedef} \PB{$*$}{}\PB{$*$}(\&{CPtr})\cr
\+& \PB{\\{delete\_like}} \PB{\\{lpar}} \PB{\\{rpar}} & \PB{\\{delete\_like}}%
\hfill $DL\.{\\,}R$
& \&{delete}[\,] \cr
\+& \PB{\\{delete\_like}} \PB{\\{exp}} & \PB{\\{exp}} \hfill $D\.\ E$ & %
\&{delete} $p$ \cr
\+\dagit& \PB{\\{question}} \PB{\\{exp}} \alt \PB{\\{colon}} \PB{\\{base}} & %
\PB{\\{binop}}
& \malt {$\?x:$} {$\?f(\,):$} \cr
\+& \PB{\\{begin\_arg}} \PB{\\{end\_arg}} & \PB{\\{exp}} & \.{@[}\&{char}$*$%
\.{@]}\cr
\+& \\{any\_other} \PB{\\{end\_arg}} & \PB{\\{end\_arg}} &    \&{char}$*$\.{@]}%
\cr
\+& \PB{\\{alignas\_like}} \PB{\\{decl\_head}} & \PB{\\{attr}} & \&{alignas}(%
\&{struct} $s$ ${*})$ \cr
\+& \PB{\\{alignas\_like}} \PB{\\{exp}} & \PB{\\{attr}} & \PB{\&{alignas}(%
\T{8})} \cr
\+& \PB{\\{lbrack}} \PB{\\{lbrack}} & \PB{\\{attr\_head}} & attribute begins %
\cr
\+& \PB{\\{lbrack}} & \PB{\\{lpar}} & \PB{[} elsewhere \cr
\+& \PB{\\{rbrack}} & \PB{\\{rpar}} & \PB{]} elsewhere \cr
\+& \PB{\\{attr\_head}} \PB{\\{rbrack}} \PB{\\{rbrack}} & \PB{\\{attr}} & %
\PB{[[\hbox{\dots}]]} \cr
\+& \PB{\\{attr\_head}} \PB{\\{exp}} & \PB{\\{attr\_head}} & \PB{[[%
\\{deprecated}} \cr
\+& \PB{\\{attr\_head}} \PB{\\{using\_like}} \PB{\\{exp}} \PB{\\{colon}} & \PB{%
\\{attr\_head}}
& [[\&{using} \.{NS}: \cr
\+& \PB{\\{attr}} \alt\PB{\\{lbrace}} \PB{\\{stmt}} & \alt\PB{\\{lbrace}} \PB{%
\\{stmt}} \hfill
$A$\.\ \alt $S$ $L$ & \PB{[[\\{likely}]] ${}\{$}\cr
\+& \PB{\\{attr}} \PB{\\{tag}} & \PB{\\{tag}} \hfill $A\.\ T$ & \PB{[[%
\\{likely}]] \&{case} \T{0}:} \cr
\+& \PB{\\{attr}} \PB{\\{semi}} & \PB{\\{stmt}} & \PB{[[\\{fallthrough}]];} \cr
\+& \PB{\\{attr}} \PB{\\{attr}} & \PB{\\{attr}} \hfill $A_1\.\ A_2$
& \&{alignas}($x$) [[\hbox{\dots}]] \cr
\+& \PB{\\{attr}} \PB{\\{decl\_head}} & \PB{\\{decl\_head}} & [[\\{nodiscard}]]
$f$(\,) \cr
\+& \PB{\\{decl\_head}} \PB{\\{attr}} & \PB{\\{decl\_head}} & (\&{int} $x$ [[%
\\{deprecated}]])\cr
\+& \PB{\\{using\_like}} & \PB{\\{int\_like}} & \&{using} not in attributes \cr
\+& \PB{\\{struct\_like}} \PB{\\{attr}} & \PB{\\{struct\_like}} \hfill $S\.\ A$
& \&{struct} [[\\{deprecated}]]\cr
\+& \PB{\\{exp}} \PB{\\{attr}} & \PB{\\{attr}} \hfill $E\.\ A$ & \&{enum} $\{x\
[[\ldots]]\}$ \cr
\+& \PB{\\{attr}} \PB{\\{typedef\_like}} & \PB{\\{typedef\_like}} \hfill $A\.\
T$
& \PB{[[\\{deprecated}]] \&{typedef}} \cr
\+& \PB{\\{raw\_int}} \PB{\\{lbrack}} & \PB{\\{exp}} & \PB{\&{int}[\T{3}]} \cr
\+& \PB{\\{attr\_head}} \PB{\\{comma}} & \PB{\\{attr\_head}} & $[[x,y$ \cr
\+& \PB{\\{if\_head}} \PB{\\{attr}} & \PB{\\{if\_head}} \hfill $I\.\ A$
& \&{if} ($x$) [[\\{unlikely}]] $\{$ \cr
\+& \PB{\\{lbrack}} \PB{\\{lbrack}} \PB{\\{rbrack}} \PB{\\{rbrack}} & \PB{%
\\{exp}} & \PB{[[]]} \cr
\+& \PB{\\{attr}} \PB{\\{function}} & \PB{\\{function}} \hfill $A\.\ F$
& attribute and function \cr
\+& \PB{\\{default\_like}} \PB{\\{colon}} & \PB{\\{case\_like}} \PB{\\{colon}}
& \PB{\&{default}:} \cr
\+& \PB{\\{default\_like}} & \PB{\\{exp}} & $f(\,)=\&{default};$ \cr
\+& \PB{\\{struct\_like}} \PB{\\{struct\_like}} & \PB{\\{struct\_like}} \hfill
$S_1\.\ S_2$
& \PB{\&{enum} \&{class}} \cr
\+& \PB{\\{exp}} \PB{\\{colcol}} \PB{\\{int\_like}} & \PB{\\{int\_like}} & $%
\\{std}\DC\&{atomic}$ \cr
\advance\midcol-30pt
\+\dagit& \PB{\\{langle}} \PB{\\{struct\_like}} \alt \PB{\\{exp}} \PB{\\{int%
\_like}} \PB{\\{comma}} &
\PB{\\{langle}} \hfill $LS$\alt $E^{**}$ $I^{**}$ \unskip $C$
& $\langle$\&{typename} $t,$\cr
\+\dagit& \PB{\\{langle}} \PB{\\{struct\_like}} \alt \PB{\\{exp}} \PB{\\{int%
\_like}} \PB{\\{prerangle}} &
\PB{\\{cast}} \hfill $LS$\alt $E^{**}$ $I^{**}$ \unskip $P$
& $\langle$\&{typename} $t\rangle$ \cr
\advance\midcol30pt
\+& \PB{\\{template\_like}} \PB{\\{cast}} \PB{\\{struct\_like}} & \PB{\\{struct%
\_like}} \hfill $T\.\ CS$ &
\PB{$\&{template}\langle\,\hbox{\dots}\rangle{}$ \&{class}} \cr
\+& \PB{\\{tag}} \PB{\\{rbrace}} & \PB{\\{decl}} \PB{\\{rbrace}} & \PB{%
\&{public}: $\}$} \cr
\+& \PB{\\{fn\_decl}} \PB{\\{attr}} & \PB{\\{fn\_decl}} \hfill $F\.\ A$
& \&{void} $f$(\,) \&{noexcept} \cr
\+& \PB{\\{alignas\_like}} \PB{\\{cast}} & \PB{\\{attr}} & \PB{\&{alignas}(%
\&{int})} \cr
\vfill\break
\parindent=0pt
\dag{\bf Notes}
\yskip
Rule 35: The \PB{\\{exp}} must not be immediately followed by \PB{\\{lpar}}, %
\PB{\\{lbrack}},
\PB{\\{exp}}, or~\PB{\\{cast}}.

Rule 48: The \PB{\\{exp}} or \PB{\\{int\_like}} must not be immediately
followed by \PB{\\{base}}.

Rule 76: The \PB{\\{force}} in the \PB{\\{stmt}} line becomes \\{bsp} if %
\.{CWEAVE} has
been invoked with the \.{-f} option.

Rule 78: The \PB{\\{define\_like}} case calls \PB{\\{make\_underlined}} on the
following scrap.

Rule 94: The \PB{\\{new\_like}} must not be immediately followed by \PB{%
\\{lpar}}.

Rule 99: The \PB{\\{new\_exp}} must not be immediately followed by \PB{\\{raw%
\_int}},
\PB{\\{struct\_like}}, or \PB{\\{colcol}}.

Rule 110: The \PB{\\{raw\_int}} must not be immediately followed by \PB{%
\\{langle}}.

Rule 111: The operator after \PB{\\{operator\_like}}
must not be immediately followed by a \PB{\\{binop}}.

Rule 114: The \PB{\\{operator\_like}} must not be immediately followed by
\PB{\\{raw\_ubin}}.

Rule 117: The \PB{\\{exp}} must not be immediately followed by \PB{\\{lpar}}, %
\PB{\\{exp}},
or \PB{\\{cast}}.

Rule 123: The mathness of the \PB{\\{colon}} or \PB{\\{base}} changes to `yes'.

Rules 153, 154: \PB{\\{make\_reserved}} is called only if \.{CWEAVE} has been
invoked
with the \.{+t} option.

\endgroup

\fi

\N{1}{112}Implementing the productions.
More specifically, a scrap is a structure consisting of a category
\PB{\\{cat}} and a \PB{\&{text\_pointer}} \PB{\\{trans}}, which points to the
translation in
\PB{\\{tok\_start}}.  When \CEE/ text is to be processed with the grammar
above,
we form an array \PB{\\{scrap\_info}} containing the initial scraps.
Our production rules have the nice property that the right-hand side is never
longer than the left-hand side. Therefore it is convenient to use sequential
allocation for the current sequence of scraps. Five pointers are used to
manage the parsing:

\yskip\hang \PB{\\{pp}} is a pointer into \PB{\\{scrap\_info}}.  We will try to
match
the category codes \PB{$\\{pp}\MG\\{cat},\,\,(\\{pp}+\T{1})\MG\\{cat}$}$,\,\,%
\ldots\,$
to the left-hand sides of productions.

\yskip\hang \PB{\\{scrap\_base}}, \PB{\\{lo\_ptr}}, \PB{\\{hi\_ptr}}, and \PB{%
\\{scrap\_ptr}} are such that
the current sequence of scraps appears in positions \PB{\\{scrap\_base}}
through
\PB{\\{lo\_ptr}} and \PB{\\{hi\_ptr}} through \PB{\\{scrap\_ptr}}, inclusive,
in the \PB{\\{cat}} and
\PB{\\{trans}} arrays. Scraps located between \PB{\\{scrap\_base}} and \PB{%
\\{lo\_ptr}} have
been examined, while those in positions \PB{$\G$ \\{hi\_ptr}} have not yet been
looked at by the parsing process.

\yskip\noindent Initially \PB{\\{scrap\_ptr}} is set to the position of the
final
scrap to be parsed, and it doesn't change its value. The parsing process
makes sure that \PB{$\\{lo\_ptr}\G\\{pp}+\T{3}$}, since productions have as
many as four terms,
by moving scraps from \PB{\\{hi\_ptr}} to \PB{\\{lo\_ptr}}. If there are
fewer than \PB{$\\{pp}+\T{3}$} scraps left, the positions up to \PB{$\\{pp}+%
\T{3}$} are filled with
blanks that will not match in any productions. Parsing stops when
\PB{$\\{pp}\E\\{lo\_ptr}+\T{1}$} and \PB{$\\{hi\_ptr}\E\\{scrap\_ptr}+\T{1}$}.

Since the \PB{\\{scrap}} structure will later be used for other purposes, we
declare its second element as a union.

\Y\B\4\X22:Typedef declarations\X${}\mathrel+\E{}$\6
\&{typedef} \&{struct} ${}\{{}$\1\6
\&{eight\_bits} \\{cat};\6
\&{eight\_bits} \\{mathness};\6
\&{union} ${}\{{}$\1\6
\&{text\_pointer} \\{Trans};\5
\hbox{}\6{}\X253:Rest of \PB{\\{trans\_plus}} union\X\2\6
${}\}{}$ \\{trans\_plus};\2\6
${}\}{}$ \&{scrap};\6
\&{typedef} \&{scrap} ${}{*}\&{scrap\_pointer}{}$;\par
\fi

\M{113}\B\D\\{trans}\5
$\\{trans\_plus}.{}$\\{Trans}\C{ translation texts of scraps }\par
\Y\B\4\X21:Private variables\X${}\mathrel+\E{}$\6
\&{static} \&{scrap} \\{scrap\_info}[\\{max\_scraps}];\C{ memory array for
scraps }\6
\&{static} \&{scrap\_pointer} \\{scrap\_info\_end}${}\K\\{scrap\_info}+\\{max%
\_scraps}-\T{1}{}$;\C{ end of \PB{\\{scrap\_info}} }\6
\&{static} \&{scrap\_pointer} \\{scrap\_base};\C{ beginning of the current
scrap sequence }\6
\&{static} \&{scrap\_pointer} \\{scrap\_ptr};\C{ ending of the current scrap
sequence }\6
\&{static} \&{scrap\_pointer} \\{max\_scr\_ptr};\C{ largest value assumed by %
\PB{\\{scrap\_ptr}} }\6
\&{static} \&{scrap\_pointer} \\{pp};\C{ current position for reducing
productions }\6
\&{static} \&{scrap\_pointer} \\{lo\_ptr};\C{ last scrap that has been examined
}\6
\&{static} \&{scrap\_pointer} \\{hi\_ptr};\C{ first scrap that has not been
examined }\par
\fi

\M{114}\B\X24:Set initial values\X${}\mathrel+\E{}$\6
$\\{scrap\_base}\K\\{scrap\_info}+\T{1};{}$\6
${}\\{max\_scr\_ptr}\K\\{scrap\_ptr}\K\\{scrap\_info}{}$;\par
\fi

\M{115}Token lists in \PB{\\{tok\_mem}} are composed of the following kinds of
items for \TEX/ output.

\yskip\item{$\bullet$}Character codes and special codes like \PB{\\{force}} and
\PB{\\{math\_rel}} represent themselves;

\item{$\bullet$}\PB{$\\{id\_flag}+\|p$} represents \.{\\\\\{{\rm identifier
$p$}\}};

\item{$\bullet$}\PB{$\\{res\_flag}+\|p$} represents \.{\\\&\{{\rm identifier
$p$}\}};

\item{$\bullet$}\PB{$\\{section\_flag}+\|p$} represents section name \PB{\|p};

\item{$\bullet$}\PB{$\\{tok\_flag}+\|p$} represents token list number \PB{\|p};

\item{$\bullet$}\PB{$\\{inner\_tok\_flag}+\|p$} represents token list number %
\PB{\|p}, to be
translated without line-break controls.

\Y\B\4\D\\{id\_flag}\5
\T{10240}\C{ signifies an identifier }\par
\B\4\D\\{res\_flag}\5
$(\T{2}*\\{id\_flag}{}$)\C{ signifies a reserved word }\par
\B\4\D\\{section\_flag}\5
$(\T{3}*\\{id\_flag}{}$)\C{ signifies a section name }\par
\B\4\D\\{tok\_flag}\5
$(\T{4}*\\{id\_flag}{}$)\C{ signifies a token list }\par
\B\4\D\\{inner\_tok\_flag}\5
$(\T{5}*\\{id\_flag}{}$)\C{ signifies a token list in `\pb' }\par
\Y\B\1\1\&{static} \&{void} \\{print\_text}(\C{ prints a token list for
debugging; not used in \PB{\\{main}} }\6
\&{text\_pointer} \|p)\2\2\6
${}\{{}$\1\6
\&{token\_pointer} \|j;\C{ index into \PB{\\{tok\_mem}} }\6
\&{sixteen\_bits} \|r;\C{ remainder of token after the flag has been stripped
off }\7
\&{if} ${}(\|p\G\\{text\_ptr}){}$\1\5
\\{printf}(\.{"BAD"});\2\6
\&{else}\1\6
\&{for} ${}(\|j\K{*}\|p;{}$ ${}\|j<{*}(\|p+\T{1});{}$ ${}\|j\PP){}$\5
${}\{{}$\1\6
${}\|r\K{*}\|j\MOD\\{id\_flag};{}$\6
\&{switch} ${}({*}\|j){}$\5
${}\{{}$\1\6
\4\&{case} \\{id\_flag}:\5
\\{printf}(\.{"\\\\\\\\\{"});\6
${}\\{print\_id}((\\{name\_dir}+\|r));{}$\6
\\{putchar}(\.{'\}'});\6
\&{break};\6
\4\&{case} \\{res\_flag}:\5
\\{printf}(\.{"\\\\\&\{"});\6
${}\\{print\_id}((\\{name\_dir}+\|r));{}$\6
\\{putchar}(\.{'\}'});\6
\&{break};\6
\4\&{case} \\{section\_flag}:\5
\\{putchar}(\.{'<'});\6
${}\\{print\_section\_name}((\\{name\_dir}+\|r));{}$\6
\\{putchar}(\.{'>'});\6
\&{break};\6
\4\&{case} \\{tok\_flag}:\5
${}\\{printf}(\.{"[[\%d]]"},\39{}$(\&{int}) \|r);\6
\&{break};\6
\4\&{case} \\{inner\_tok\_flag}:\5
${}\\{printf}(\.{"|[[\%d]]|"},\39{}$(\&{int}) \|r);\6
\&{break};\6
\4\&{default}:\5
\X117:Print token \PB{\|r} in symbolic form\X\6
\4${}\}{}$\2\6
\4${}\}{}$\2\2\6
\\{update\_terminal};\6
\4${}\}{}$\2\par
\fi

\M{116}\B\X8:Predeclaration of procedures\X${}\mathrel+\E{}$\5
\&{static} \&{void} \\{print\_text}(\&{text\_pointer} \|p);\par
\fi

\M{117}\B\X117:Print token \PB{\|r} in symbolic form\X${}\E{}$\6
\&{switch} (\|r)\5
${}\{{}$\1\6
\4\&{case} \\{math\_rel}:\5
\\{printf}(\.{"\\\\mathrel\{"});\6
\&{break};\6
\4\&{case} \\{big\_cancel}:\5
\\{printf}(\.{"[ccancel]"});\6
\&{break};\6
\4\&{case} \\{cancel}:\5
\\{printf}(\.{"[cancel]"});\6
\&{break};\6
\4\&{case} \\{indent}:\5
\\{printf}(\.{"[indent]"});\6
\&{break};\6
\4\&{case} \\{outdent}:\5
\\{printf}(\.{"[outdent]"});\6
\&{break};\6
\4\&{case} \\{dindent}:\5
\\{printf}(\.{"[dindent]"});\6
\&{break};\6
\4\&{case} \\{backup}:\5
\\{printf}(\.{"[backup]"});\6
\&{break};\6
\4\&{case} \\{opt}:\5
\\{printf}(\.{"[opt]"});\6
\&{break};\6
\4\&{case} \\{break\_space}:\5
\\{printf}(\.{"[break]"});\6
\&{break};\6
\4\&{case} \\{force}:\5
\\{printf}(\.{"[force]"});\6
\&{break};\6
\4\&{case} \\{big\_force}:\5
\\{printf}(\.{"[fforce]"});\6
\&{break};\6
\4\&{case} \\{preproc\_line}:\5
\\{printf}(\.{"[preproc]"});\6
\&{break};\6
\4\&{case} \\{quoted\_char}:\5
${}\|j\PP;{}$\6
${}\\{printf}(\.{"[\%o]"},\39{}$(\&{unsigned} \&{int}) ${}{*}\|j);{}$\6
\&{break};\6
\4\&{case} \\{end\_translation}:\5
\\{printf}(\.{"[quit]"});\6
\&{break};\6
\4\&{case} \\{inserted}:\5
\\{printf}(\.{"[inserted]"});\6
\&{break};\6
\4\&{default}:\5
\\{putchar}((\&{int}) \|r);\6
\4${}\}{}$\2\par
\U115.\fi

\M{118}The production rules listed above are embedded directly into \.{CWEAVE},
since it is easier to do this than to write an interpretive system
that would handle production systems in general. Several macros are defined
here so that the program for each production is fairly short.

All of our productions conform to the general notion that some \PB{\|k}
consecutive scraps starting at some position \PB{\|j} are to be replaced by a
single scrap of some category \PB{\|c} whose translation is composed from the
translations of the disappearing scraps. After this production has been
applied, the production pointer \PB{\\{pp}} should change by an amount \PB{%
\|d}. Such
a production can be represented by the quadruple \PB{$(\|j,\|k,\|c,\|d)$}. For
example,
the production `\PB{\\{exp}\,\\{comma}\,\\{exp}} $\RA$ \PB{\\{exp}}' would be
represented by
`\PB{$(\\{pp},\T{3},\\{exp},{-}\T{2})$}'; in this case the pointer \PB{\\{pp}}
should decrease by 2
after the production has been applied, because some productions with
\PB{\\{exp}} in their second or third positions might now match,
but no productions have
\PB{\\{exp}} in the fourth position of their left-hand sides. Note that
the value of \PB{\|d} is determined by the whole collection of productions, not
by an individual one.
The determination of \PB{\|d} has been
done by hand in each case, based on the full set of productions but not on
the grammar of \CEE/ or on the rules for constructing the initial
scraps.

We also attach a serial number to each production, so that additional
information is available when debugging. For example, the program below
contains the statement `\PB{$\\{reduce}(\\{pp},\T{3},\\{exp},{-}\T{2},\T{4})$}'
when it implements
the production just mentioned.

Before calling \PB{\\{reduce}}, the program should have appended the tokens of
the new translation to the \PB{\\{tok\_mem}} array. We commonly want to append
copies of several existing translations, and macros are defined to
simplify these common cases. For example, \PB{\\{big\_app2}(\\{pp})} will
append the
translations of two consecutive scraps, \PB{$\\{pp}\MG\\{trans}$} and \PB{$(%
\\{pp}+\T{1})\MG\\{trans}$}, to
the current token list. If the entire new translation is formed in this
way, we write `\PB{$\\{squash}(\|j,\|k,\|c,\|d,\|n)$}' instead of `\PB{$%
\\{reduce}(\|j,\|k,\|c,\|d,\|n)$}'. For
example, `\PB{$\\{squash}(\\{pp},\T{3},\\{exp},{-}\T{2},\T{3})$}' is an
abbreviation for `\PB{\\{big\_app3}(\\{pp}); $\\{reduce}(\\{pp},\T{3},%
\\{exp},{-}\T{2},\T{3})$}'.

A couple more words of explanation:
Both \PB{\\{big\_app}} and \PB{\\{app}} append a token (while \PB{\\{big%
\_app1}} to \PB{\\{big\_app4}}
append the specified number of scrap translations) to the current token list.
The difference between \PB{\\{big\_app}} and \PB{\\{app}} is simply that \PB{%
\\{big\_app}}
checks whether there can be a conflict between math and non-math
tokens, and intercalates a `\.{\$}' token if necessary.  When in
doubt what to use, use \PB{\\{big\_app}}.

\Y\B\4\D\\{app}$(\|a)$\5
${*}(\\{tok\_ptr}\PP)\K{}$(\&{token})(\|a)\par
\B\4\D\\{big\_app2}$(\|a)$\5
\\{big\_app1}(\|a);\5
$\\{big\_app1}(\|a+\T{1}{}$)\par
\B\4\D\\{big\_app3}$(\|a)$\5
\\{big\_app2}(\|a);\5
$\\{big\_app1}(\|a+\T{2}{}$)\par
\B\4\D\\{big\_app4}$(\|a)$\5
\\{big\_app3}(\|a);\5
$\\{big\_app1}(\|a+\T{3}{}$)\par
\B\4\D\\{big\_app1\_insert}$(\|p,\|c)$\5
\\{big\_app1}(\|p);\5
\\{big\_app}(\|c);\5
$\\{big\_app1}(\|p+\T{1}{}$)\par
\Y\B\4\X8:Predeclaration of procedures\X${}\mathrel+\E{}$\6
\&{static} \&{void} \\{app\_str}(\&{const} \&{char} ${}{*}){}$;\6
\&{static} \&{void} \\{big\_app}(\&{token});\6
\&{static} \&{void} \\{big\_app1}(\&{scrap\_pointer});\par
\fi

\M{119}The \PB{\\{mathness}} is an attribute of scraps that says whether they
are
to be printed in a math mode context or not.  It is separate from the
``part of speech'' (the \PB{\\{cat}}) because to make each \PB{\\{cat}} have
a fixed \PB{\\{mathness}} (as in the original \.{WEAVE}) would multiply the
number of necessary production rules.

The low two bits (i.e., \PB{$\\{mathness}\MOD\T{4}$}) control the left
boundary.
(We need two bits because we allow cases \PB{\\{yes\_math}}, \PB{\\{no\_math}}
and
\PB{\\{maybe\_math}}, which can go either way.)
The next two bits (i.e., \PB{$\\{mathness}/\T{4}$}) control the right boundary.
If we combine two scraps and the right boundary of the first has
a different mathness from the left boundary of the second, we
insert a \.{\$} in between.  Similarly, if at printing time some
irreducible scrap has a \PB{\\{yes\_math}} boundary the scrap gets preceded
or followed by a~\.{\$}. The left boundary is \PB{\\{maybe\_math}} if and
only if the right boundary is.

\Y\B\4\D\\{no\_math}\5
\T{2}\C{ should be in horizontal mode }\par
\B\4\D\\{yes\_math}\5
\T{1}\C{ should be in math mode }\par
\B\4\D\\{maybe\_math}\5
\T{0}\C{ works in either horizontal or math mode }\par
\Y\B\4\X21:Private variables\X${}\mathrel+\E{}$\6
\&{static} \&{int} \\{cur\_mathness}${},{}$ \\{init\_mathness};\par
\fi

\M{120}The code below is an exact translation of the production rules into
\CEE/, using such macros, and the reader should have no difficulty
understanding the format by comparing the code with the symbolic
productions as they were listed earlier.

\Y\B\1\1\&{static} \&{void} \\{app\_str}(\&{const} \&{char} ${}{*}\|s)\2\2{}$\6
${}\{{}$\1\6
\&{while} ${}({*}\|s){}$\1\5
${}\\{app\_tok}({*}\|s\PP);{}$\2\6
\4${}\}{}$\2\7
\1\1\&{static} \&{void} \\{big\_app}(\&{token} \|a)\2\2\6
${}\{{}$\1\6
\&{if} ${}(\|a\E\.{'\ '}\V(\|a\G\\{big\_cancel}\W\|a\Z\\{big\_force})\V\|a\E%
\\{dindent}{}$)\C{ non-math token }\6
${}\{{}$\1\6
\&{if} ${}(\\{cur\_mathness}\E\\{maybe\_math}){}$\1\5
${}\\{init\_mathness}\K\\{no\_math};{}$\2\6
\&{else} \&{if} ${}(\\{cur\_mathness}\E\\{yes\_math}){}$\1\5
\\{app\_str}(\.{"\{\}\$"});\2\6
${}\\{cur\_mathness}\K\\{no\_math};{}$\6
\4${}\}{}$\2\6
\&{else}\5
${}\{{}$\1\6
\&{if} ${}(\\{cur\_mathness}\E\\{maybe\_math}){}$\1\5
${}\\{init\_mathness}\K\\{yes\_math};{}$\2\6
\&{else} \&{if} ${}(\\{cur\_mathness}\E\\{no\_math}){}$\1\5
\\{app\_str}(\.{"\$\{\}"});\2\6
${}\\{cur\_mathness}\K\\{yes\_math};{}$\6
\4${}\}{}$\2\6
\\{app}(\|a);\6
\4${}\}{}$\2\7
\1\1\&{static} \&{void} \\{big\_app1}(\&{scrap\_pointer} \|a)\2\2\6
${}\{{}$\1\6
\&{switch} ${}(\|a\MG\\{mathness}\MOD\T{4}){}$\5
${}\{{}$\C{ left boundary }\1\6
\4\&{case} (\\{no\_math}):\6
\&{if} ${}(\\{cur\_mathness}\E\\{maybe\_math}){}$\1\5
${}\\{init\_mathness}\K\\{no\_math};{}$\2\6
\&{else} \&{if} ${}(\\{cur\_mathness}\E\\{yes\_math}){}$\1\5
\\{app\_str}(\.{"\{\}\$"});\2\6
${}\\{cur\_mathness}\K\|a\MG\\{mathness}/\T{4}{}$;\C{ right boundary }\6
\&{break};\6
\4\&{case} (\\{yes\_math}):\6
\&{if} ${}(\\{cur\_mathness}\E\\{maybe\_math}){}$\1\5
${}\\{init\_mathness}\K\\{yes\_math};{}$\2\6
\&{else} \&{if} ${}(\\{cur\_mathness}\E\\{no\_math}){}$\1\5
\\{app\_str}(\.{"\$\{\}"});\2\6
${}\\{cur\_mathness}\K\|a\MG\\{mathness}/\T{4}{}$;\C{ right boundary }\6
\&{break};\6
\4\&{case} (\\{maybe\_math}):\C{ no changes }\6
\&{break};\6
\4${}\}{}$\2\6
${}\\{app}(\\{tok\_flag}+(\&{int})((\|a)\MG\\{trans}-\\{tok\_start}));{}$\6
\4${}\}{}$\2\par
\fi

\M{121}Let us consider the big switch for productions now, before looking
at its context. We want to design the program so that this switch
works, so we might as well not keep ourselves in suspense about exactly what
code needs to be provided with a proper environment.

\Y\B\4\D\\{cat1}\5
$(\\{pp}+\T{1})\MG{}$\\{cat}\par
\B\4\D\\{cat2}\5
$(\\{pp}+\T{2})\MG{}$\\{cat}\par
\B\4\D\\{cat3}\5
$(\\{pp}+\T{3})\MG{}$\\{cat}\par
\B\4\D\\{lhs\_not\_simple}\5
$(\\{pp}\MG\\{cat}\I\\{public\_like}\W\\{pp}\MG\\{cat}\I\\{semi}\W\\{pp}\MG%
\\{cat}\I\\{prelangle}\W\\{pp}\MG\\{cat}\I\\{prerangle}\3{-1}\W\\{pp}\MG\\{cat}%
\I\\{template\_like}\W\\{pp}\MG\\{cat}\I\\{new\_like}\W\\{pp}\MG\\{cat}\I\\{new%
\_exp}\W\\{pp}\MG\\{cat}\I\\{ftemplate}\3{-1}\W\\{pp}\MG\\{cat}\I\\{raw\_ubin}%
\W\\{pp}\MG\\{cat}\I\\{const\_like}\W\\{pp}\MG\\{cat}\I\\{raw\_int}\W\\{pp}\MG%
\\{cat}\I\\{operator\_like}{}$)\C{ not a production with left side length 1 }%
\par
\Y\B\4\X121:Match a production at \PB{\\{pp}}, or increase \PB{\\{pp}} if there
is no match\X${}\E{}$\6
\&{if} ${}(\\{cat1}\E\\{end\_arg}\W\\{lhs\_not\_simple}){}$\1\6
\&{if} ${}(\\{pp}\MG\\{cat}\E\\{begin\_arg}){}$\1\5
${}\\{squash}(\\{pp},\39\T{2},\39\\{exp},\39{-}\T{2},\39\T{124});{}$\2\6
\&{else}\1\5
${}\\{squash}(\\{pp},\39\T{2},\39\\{end\_arg},\39{-}\T{1},\39\T{125});{}$\2\2\6
\&{else} \&{if} ${}(\\{pp}\MG\\{cat}\E\\{rbrack}){}$\1\5
${}\\{reduce}(\\{pp},\39\T{0},\39\\{rpar},\39{-}\T{3},\39\T{130});{}$\2\6
\&{else} \&{if} ${}(\\{pp}\MG\\{cat}\E\\{using\_like}){}$\1\5
${}\\{reduce}(\\{pp},\39\T{0},\39\\{int\_like},\39{-}\T{3},\39\T{140});{}$\2\6
\&{else} \&{if} ${}(\\{cat1}\E\\{insert}){}$\1\5
${}\\{squash}(\\{pp},\39\T{2},\39\\{pp}\MG\\{cat},\39{-}\T{2},\39\T{0});{}$\2\6
\&{else} \&{if} ${}(\\{cat2}\E\\{insert}){}$\1\5
${}\\{squash}(\\{pp}+\T{1},\39\T{2},\39(\\{pp}+\T{1})\MG\\{cat},\39{-}\T{1},\39%
\T{0});{}$\2\6
\&{else} \&{if} ${}(\\{cat3}\E\\{insert}){}$\1\5
${}\\{squash}(\\{pp}+\T{2},\39\T{2},\39(\\{pp}+\T{2})\MG\\{cat},\39\T{0},\39%
\T{0});{}$\2\6
\&{else}\1\6
\&{switch} ${}(\\{pp}\MG\\{cat}){}$\5
${}\{{}$\1\6
\4\&{case} \\{exp}:\5
\X128:Cases for \PB{\\{exp}}\X\5
\&{break};\6
\4\&{case} \\{lpar}:\5
\X129:Cases for \PB{\\{lpar}}\X\5
\&{break};\6
\4\&{case} \\{unop}:\5
\X130:Cases for \PB{\\{unop}}\X\5
\&{break};\6
\4\&{case} \\{ubinop}:\5
\X131:Cases for \PB{\\{ubinop}}\X\5
\&{break};\6
\4\&{case} \\{binop}:\5
\X132:Cases for \PB{\\{binop}}\X\5
\&{break};\6
\4\&{case} \\{cast}:\5
\X133:Cases for \PB{\\{cast}}\X\5
\&{break};\6
\4\&{case} \\{sizeof\_like}:\5
\X134:Cases for \PB{\\{sizeof\_like}}\X\5
\&{break};\6
\4\&{case} \\{int\_like}:\5
\X135:Cases for \PB{\\{int\_like}}\X\5
\&{break};\6
\4\&{case} \\{public\_like}:\5
\X136:Cases for \PB{\\{public\_like}}\X\5
\&{break};\6
\4\&{case} \\{colcol}:\5
\X137:Cases for \PB{\\{colcol}}\X\5
\&{break};\6
\4\&{case} \\{decl\_head}:\5
\X138:Cases for \PB{\\{decl\_head}}\X\5
\&{break};\6
\4\&{case} \\{decl}:\5
\X139:Cases for \PB{\\{decl}}\X\5
\&{break};\6
\4\&{case} \\{base}:\5
\X140:Cases for \PB{\\{base}}\X\5
\&{break};\6
\4\&{case} \\{struct\_like}:\5
\X141:Cases for \PB{\\{struct\_like}}\X\5
\&{break};\6
\4\&{case} \\{struct\_head}:\5
\X142:Cases for \PB{\\{struct\_head}}\X\5
\&{break};\6
\4\&{case} \\{fn\_decl}:\5
\X143:Cases for \PB{\\{fn\_decl}}\X\5
\&{break};\6
\4\&{case} \\{function}:\5
\X144:Cases for \PB{\\{function}}\X\5
\&{break};\6
\4\&{case} \\{lbrace}:\5
\X145:Cases for \PB{\\{lbrace}}\X\5
\&{break};\6
\4\&{case} \\{if\_like}:\5
\X146:Cases for \PB{\\{if\_like}}\X\5
\&{break};\6
\4\&{case} \\{else\_like}:\5
\X147:Cases for \PB{\\{else\_like}}\X\5
\&{break};\6
\4\&{case} \\{else\_head}:\5
\X148:Cases for \PB{\\{else\_head}}\X\5
\&{break};\6
\4\&{case} \\{if\_clause}:\5
\X149:Cases for \PB{\\{if\_clause}}\X\5
\&{break};\6
\4\&{case} \\{if\_head}:\5
\X150:Cases for \PB{\\{if\_head}}\X\5
\&{break};\6
\4\&{case} \\{do\_like}:\5
\X151:Cases for \PB{\\{do\_like}}\X\5
\&{break};\6
\4\&{case} \\{case\_like}:\5
\X152:Cases for \PB{\\{case\_like}}\X\5
\&{break};\6
\4\&{case} \\{catch\_like}:\5
\X153:Cases for \PB{\\{catch\_like}}\X\5
\&{break};\6
\4\&{case} \\{tag}:\5
\X154:Cases for \PB{\\{tag}}\X\5
\&{break};\6
\4\&{case} \\{stmt}:\5
\X156:Cases for \PB{\\{stmt}}\X\5
\&{break};\6
\4\&{case} \\{semi}:\5
\X157:Cases for \PB{\\{semi}}\X\5
\&{break};\6
\4\&{case} \\{lproc}:\5
\X158:Cases for \PB{\\{lproc}}\X\5
\&{break};\6
\4\&{case} \\{section\_scrap}:\5
\X159:Cases for \PB{\\{section\_scrap}}\X\5
\&{break};\6
\4\&{case} \\{insert}:\5
\X160:Cases for \PB{\\{insert}}\X\5
\&{break};\6
\4\&{case} \\{prelangle}:\5
\X161:Cases for \PB{\\{prelangle}}\X\5
\&{break};\6
\4\&{case} \\{prerangle}:\5
\X162:Cases for \PB{\\{prerangle}}\X\5
\&{break};\6
\4\&{case} \\{langle}:\5
\X163:Cases for \PB{\\{langle}}\X\5
\&{break};\6
\4\&{case} \\{template\_like}:\5
\X164:Cases for \PB{\\{template\_like}}\X\5
\&{break};\6
\4\&{case} \\{new\_like}:\5
\X165:Cases for \PB{\\{new\_like}}\X\5
\&{break};\6
\4\&{case} \\{new\_exp}:\5
\X166:Cases for \PB{\\{new\_exp}}\X\5
\&{break};\6
\4\&{case} \\{ftemplate}:\5
\X167:Cases for \PB{\\{ftemplate}}\X\5
\&{break};\6
\4\&{case} \\{for\_like}:\5
\X168:Cases for \PB{\\{for\_like}}\X\5
\&{break};\6
\4\&{case} \\{raw\_ubin}:\5
\X169:Cases for \PB{\\{raw\_ubin}}\X\5
\&{break};\6
\4\&{case} \\{const\_like}:\5
\X170:Cases for \PB{\\{const\_like}}\X\5
\&{break};\6
\4\&{case} \\{raw\_int}:\5
\X171:Cases for \PB{\\{raw\_int}}\X\5
\&{break};\6
\4\&{case} \\{operator\_like}:\5
\X172:Cases for \PB{\\{operator\_like}}\X\5
\&{break};\6
\4\&{case} \\{typedef\_like}:\5
\X173:Cases for \PB{\\{typedef\_like}}\X\5
\&{break};\6
\4\&{case} \\{delete\_like}:\5
\X174:Cases for \PB{\\{delete\_like}}\X\5
\&{break};\6
\4\&{case} \\{question}:\5
\X175:Cases for \PB{\\{question}}\X\5
\&{break};\6
\4\&{case} \\{alignas\_like}:\5
\X176:Cases for \PB{\\{alignas\_like}}\X\5
\&{break};\6
\4\&{case} \\{lbrack}:\5
\X177:Cases for \PB{\\{lbrack}}\X\5
\&{break};\6
\4\&{case} \\{attr\_head}:\5
\X178:Cases for \PB{\\{attr\_head}}\X\5
\&{break};\6
\4\&{case} \\{attr}:\5
\X179:Cases for \PB{\\{attr}}\X\5
\&{break};\6
\4\&{case} \\{default\_like}:\5
\X180:Cases for \PB{\\{default\_like}}\X\5
\&{break};\6
\4${}\}{}$\2\2\6
${}\\{pp}\PP{}$;\C{ if no match was found, we move to the right }\par
\U184.\fi

\M{122}In \CEE/, new specifier names can be defined via \PB{\&{typedef}}, and
we want
to make the parser recognize future occurrences of the identifier thus
defined as specifiers.  This is done by the procedure \PB{\\{make\_reserved}},
which changes the \PB{\\{ilk}} of the relevant identifier.

We first need a procedure to recursively seek the first
identifier in a token list, because the identifier might
be enclosed in parentheses, as when one defines a function
returning a pointer.

If the first identifier found is a keyword like `\&{case}', we
return the special value \PB{\\{case\_found}}; this prevents underlining
of identifiers in case labels.

If the first identifier is the keyword `\&{operator}', we give up;
users who want to index definitions of overloaded \CPLUSPLUS/ operators
should say, for example, `\.{@!@\^\\\&\{operator\} \$+\{=\}\$@>}' (or,
more properly alpha\-betized,
`\.{@!@:operator+=\}\{\\\&\{operator\} \$+\{=\}\$@>}').

\Y\B\4\D\\{no\_ident\_found}\5
(\&{token\_pointer}) \T{0}\C{ distinct from any identifier token }\par
\B\4\D\\{case\_found}\5
(\&{token\_pointer}) \T{1}\C{ likewise }\par
\B\4\D\\{operator\_found}\5
(\&{token\_pointer}) \T{2}\C{ likewise }\par
\Y\B\4\X8:Predeclaration of procedures\X${}\mathrel+\E{}$\6
\&{static} \&{token\_pointer} \\{find\_first\_ident}(\&{text\_pointer});\6
\&{static} \&{void} \\{make\_reserved}(\&{scrap\_pointer});\6
\&{static} \&{void} \\{make\_underlined}(\&{scrap\_pointer});\6
\&{static} \&{void} \\{underline\_xref}(\&{name\_pointer});\par
\fi

\M{123}\B\1\1\&{static} \&{token\_pointer} \\{find\_first\_ident}(\&{text%
\_pointer} \|p)\2\2\6
${}\{{}$\1\6
\&{token\_pointer} \|q;\C{ token to be returned }\6
\&{token\_pointer} \|j;\C{ token being looked at }\6
\&{sixteen\_bits} \|r;\C{ remainder of token after the flag has been stripped
off }\7
\&{if} ${}(\|p\G\\{text\_ptr}){}$\1\5
\\{confusion}(\.{"find\_first\_ident"});\2\6
\&{for} ${}(\|j\K{*}\|p;{}$ ${}\|j<{*}(\|p+\T{1});{}$ ${}\|j\PP){}$\5
${}\{{}$\1\6
${}\|r\K{*}\|j\MOD\\{id\_flag};{}$\6
\&{switch} ${}({*}\|j/\\{id\_flag}){}$\5
${}\{{}$\1\6
\4\&{case} \T{2}:\C{ \PB{\\{res\_flag}} }\6
\&{if} ${}(\\{name\_dir}[\|r].\\{ilk}\E\\{case\_like}){}$\1\5
\&{return} \\{case\_found};\2\6
\&{if} ${}(\\{name\_dir}[\|r].\\{ilk}\E\\{operator\_like}){}$\1\5
\&{return} \\{operator\_found};\2\6
\&{if} ${}(\\{name\_dir}[\|r].\\{ilk}\I\\{raw\_int}){}$\1\5
\&{break};\2\6
\4\&{case} \T{1}:\5
\&{return} \|j;\6
\4\&{case} \T{4}:\5
\&{case} \T{5}:\C{ \PB{\\{tok\_flag}} or \PB{\\{inner\_tok\_flag}} }\6
\&{if} ${}((\|q\K\\{find\_first\_ident}(\\{tok\_start}+\|r))\I\\{no\_ident%
\_found}){}$\1\5
\&{return} \|q;\2\6
\4\&{default}:\5
;\C{ char, \PB{\\{section\_flag}}, fall thru: move on to next token }\6
\&{if} ${}({*}\|j\E\\{inserted}){}$\1\5
\&{return} \\{no\_ident\_found};\C{ ignore inserts }\2\6
\&{else} \&{if} ${}({*}\|j\E\\{qualifier}){}$\1\5
${}\|j\PP{}$;\C{ bypass namespace qualifier }\2\6
\4${}\}{}$\2\6
\4${}\}{}$\2\6
\&{return} \\{no\_ident\_found};\6
\4${}\}{}$\2\par
\fi

\M{124}The scraps currently being parsed must be inspected for any
occurrence of the identifier that we're making reserved; hence
the \PB{\&{for}} loop below.

\Y\B\1\1\&{static} \&{void} \\{make\_reserved}(\C{ make the first identifier in
\PB{$\|p\MG\\{trans}$} like \PB{\&{int}} }\6
\&{scrap\_pointer} \|p)\2\2\6
${}\{{}$\1\6
\&{sixteen\_bits} \\{tok\_value};\C{ the name of this identifier, plus its flag
}\6
\&{token\_pointer} \\{tok\_loc};\C{ pointer to \PB{\\{tok\_value}} }\7
\&{if} ${}((\\{tok\_loc}\K\\{find\_first\_ident}(\|p\MG\\{trans}))\Z\\{operator%
\_found}){}$\1\5
\&{return};\C{ this should not happen }\2\6
${}\\{tok\_value}\K{*}\\{tok\_loc};{}$\6
\&{for} ( ; ${}\|p\Z\\{scrap\_ptr};{}$ ${}\|p\E\\{lo\_ptr}\?\|p\K\\{hi\_ptr}:%
\|p\PP){}$\1\6
\&{if} ${}(\|p\MG\\{cat}\E\\{exp}){}$\1\6
\&{if} ${}({*}{*}(\|p\MG\\{trans})\E\\{tok\_value}){}$\5
${}\{{}$\1\6
${}\|p\MG\\{cat}\K\\{raw\_int};{}$\6
${}{*}{*}(\|p\MG\\{trans})\K\\{tok\_value}\MOD\\{id\_flag}+\\{res\_flag};{}$\6
\4${}\}{}$\2\2\2\6
${}(\\{name\_dir}+(\&{sixteen\_bits})(\\{tok\_value}\MOD\\{id\_flag}))\MG%
\\{ilk}\K\\{raw\_int};{}$\6
${}{*}\\{tok\_loc}\K\\{tok\_value}\MOD\\{id\_flag}+\\{res\_flag};{}$\6
\4${}\}{}$\2\par
\fi

\M{125}In the following situations we want to mark the occurrence of
an identifier as a definition: when \PB{\\{make\_reserved}} is just about to be
used; after a specifier, as in \PB{\&{char} ${}{*}{*}\\{argv}$};
before a colon, as in \\{found}:; and in the declaration of a function,
as in \\{main}()$\{\ldots;\}$.  This is accomplished by the invocation
of \PB{\\{make\_underlined}} at appropriate times.  Notice that, in the
declaration
of a function, we find out that the identifier is being defined only after
it has been swallowed up by an \PB{\\{exp}}.

\Y\B\1\1\&{static} \&{void} \\{make\_underlined}(\C{ underline the entry for
the first identifier in \PB{$\|p\MG\\{trans}$} }\6
\&{scrap\_pointer} \|p)\2\2\6
${}\{{}$\1\6
\&{token\_pointer} \\{tok\_loc};\C{ where the first identifier appears }\7
\&{if} ${}((\\{tok\_loc}\K\\{find\_first\_ident}(\|p\MG\\{trans}))\Z\\{operator%
\_found}){}$\1\5
\&{return};\C{ this happens, for example, in \PB{\&{case} \\{found}:} }\2\6
${}\\{xref\_switch}\K\\{def\_flag};{}$\6
${}\\{underline\_xref}({*}\\{tok\_loc}\MOD\\{id\_flag}+\\{name\_dir});{}$\6
\4${}\}{}$\2\par
\fi

\M{126}We cannot use \PB{\\{new\_xref}} to underline a cross-reference at this
point
because this would just make a new cross-reference at the end of the list.
We actually have to search through the list for the existing
cross-reference.

\Y\B\1\1\&{static} \&{void} \\{underline\_xref}(\&{name\_pointer} \|p)\2\2\6
${}\{{}$\1\6
\&{xref\_pointer} \|q${}\K{}$(\&{xref\_pointer}) \|p${}\MG\\{xref}{}$;\C{
pointer to cross-reference being examined }\6
\&{xref\_pointer} \|r;\C{ temporary pointer for permuting cross-references }\6
\&{sixteen\_bits} \|m;\C{ cross-reference value to be installed }\6
\&{sixteen\_bits} \|n;\C{ cross-reference value being examined }\7
\&{if} (\\{no\_xref})\1\5
\&{return};\2\6
${}\|m\K\\{section\_count}+\\{xref\_switch};{}$\6
\&{while} ${}(\|q\I\\{xmem}){}$\5
${}\{{}$\1\6
${}\|n\K\|q\MG\\{num};{}$\6
\&{if} ${}(\|n\E\|m){}$\1\5
\&{return};\2\6
\&{else} \&{if} ${}(\|m\E\|n+\\{def\_flag}){}$\5
${}\{{}$\1\6
${}\|q\MG\\{num}\K\|m;{}$\6
\&{return};\6
\4${}\}{}$\2\6
\&{else} \&{if} ${}(\|n\G\\{def\_flag}\W\|n<\|m){}$\1\5
\&{break};\2\6
${}\|q\K\|q\MG\\{xlink};{}$\6
\4${}\}{}$\2\6
\X127:Insert new cross-reference at \PB{\|q}, not at beginning of list\X\6
\4${}\}{}$\2\par
\fi

\M{127}We get to this section only when the identifier is one letter long,
so it didn't get a non-underlined entry during phase one.  But it may
have got some explicitly underlined entries in later sections, so in order
to preserve the numerical order of the entries in the index, we have
to insert the new cross-reference not at the beginning of the list
(namely, at \PB{$\|p\MG\\{xref}$}), but rather right before \PB{\|q}.

\Y\B\4\X127:Insert new cross-reference at \PB{\|q}, not at beginning of list%
\X${}\E{}$\6
\\{append\_xref}(\T{0});\C{ this number doesn't matter }\6
${}\\{xref\_ptr}\MG\\{xlink}\K{}$(\&{xref\_pointer}) \|p${}\MG\\{xref};{}$\6
${}\|r\K\\{xref\_ptr};{}$\6
\\{update\_node}(\|p);\6
\&{while} ${}(\|r\MG\\{xlink}\I\|q){}$\5
${}\{{}$\1\6
${}\|r\MG\\{num}\K\|r\MG\\{xlink}\MG\\{num};{}$\6
${}\|r\K\|r\MG\\{xlink};{}$\6
\4${}\}{}$\2\6
${}\|r\MG\\{num}\K\|m{}$;\C{ everything from \PB{\|q} on is left undisturbed }%
\par
\U126.\fi

\M{128}Now comes the code that tries to match each production starting
with a particular type of scrap. Whenever a match is discovered,
the \PB{\\{squash}} or \PB{\\{reduce}} function will cause the appropriate
action
to be performed.

\Y\B\4\X128:Cases for \PB{\\{exp}}\X${}\E{}$\6
\&{if} ${}(\\{cat1}\E\\{lbrace}\V\\{cat1}\E\\{int\_like}\V\\{cat1}\E%
\\{decl}){}$\5
${}\{{}$\1\6
\\{make\_underlined}(\\{pp});\6
\\{big\_app1}(\\{pp});\6
\\{big\_app}(\\{dindent});\6
${}\\{reduce}(\\{pp},\39\T{1},\39\\{fn\_decl},\39\T{0},\39\T{1});{}$\6
\4${}\}{}$\2\6
\&{else} \&{if} ${}(\\{cat1}\E\\{unop}){}$\1\5
${}\\{squash}(\\{pp},\39\T{2},\39\\{exp},\39{-}\T{2},\39\T{2});{}$\2\6
\&{else} \&{if} ${}((\\{cat1}\E\\{binop}\V\\{cat1}\E\\{ubinop})\W\\{cat2}\E%
\\{exp}){}$\1\5
${}\\{squash}(\\{pp},\39\T{3},\39\\{exp},\39{-}\T{2},\39\T{3});{}$\2\6
\&{else} \&{if} ${}(\\{cat1}\E\\{comma}\W\\{cat2}\E\\{exp}){}$\5
${}\{{}$\1\6
\\{big\_app2}(\\{pp});\6
\\{app}(\\{opt});\6
\\{app}(\.{'9'});\6
${}\\{big\_app1}(\\{pp}+\T{2});{}$\6
${}\\{reduce}(\\{pp},\39\T{3},\39\\{exp},\39{-}\T{2},\39\T{4});{}$\6
\4${}\}{}$\2\6
\&{else} \&{if} ${}(\\{cat1}\E\\{lpar}\W\\{cat2}\E\\{rpar}\W\\{cat3}\E%
\\{colon}){}$\1\5
${}\\{reduce}(\\{pp}+\T{3},\39\T{0},\39\\{base},\39\T{0},\39\T{5});{}$\2\6
\&{else} \&{if} ${}(\\{cat1}\E\\{cast}\W\\{cat2}\E\\{colon}){}$\1\5
${}\\{reduce}(\\{pp}+\T{2},\39\T{0},\39\\{base},\39\T{0},\39\T{5});{}$\2\6
\&{else} \&{if} ${}(\\{cat1}\E\\{semi}){}$\1\5
${}\\{squash}(\\{pp},\39\T{2},\39\\{stmt},\39{-}\T{1},\39\T{6});{}$\2\6
\&{else} \&{if} ${}(\\{cat1}\E\\{colon}){}$\5
${}\{{}$\1\6
\\{make\_underlined}(\\{pp});\6
${}\\{squash}(\\{pp},\39\T{2},\39\\{tag},\39{-}\T{1},\39\T{7});{}$\6
\4${}\}{}$\2\6
\&{else} \&{if} ${}(\\{cat1}\E\\{rbrace}){}$\1\5
${}\\{reduce}(\\{pp},\39\T{0},\39\\{stmt},\39{-}\T{1},\39\T{8});{}$\2\6
\&{else} \&{if} ${}(\\{cat1}\E\\{lpar}\W\\{cat2}\E\\{rpar}\W(\\{cat3}\E\\{const%
\_like}\V\\{cat3}\E\\{case\_like})){}$\5
${}\{{}$\1\6
${}\\{big\_app1\_insert}(\\{pp}+\T{2},\39\.{'\ '});{}$\6
${}\\{reduce}(\\{pp}+\T{2},\39\T{2},\39\\{rpar},\39\T{0},\39\T{9});{}$\6
\4${}\}{}$\2\6
\&{else} \&{if} ${}(\\{cat1}\E\\{cast}\W(\\{cat2}\E\\{const\_like}\V\\{cat2}\E%
\\{case\_like})){}$\5
${}\{{}$\1\6
${}\\{big\_app1\_insert}(\\{pp}+\T{1},\39\.{'\ '});{}$\6
${}\\{reduce}(\\{pp}+\T{1},\39\T{2},\39\\{cast},\39\T{0},\39\T{9});{}$\6
\4${}\}{}$\2\6
\&{else} \&{if} ${}(\\{cat1}\E\\{exp}\V\\{cat1}\E\\{cast}){}$\1\5
${}\\{squash}(\\{pp},\39\T{2},\39\\{exp},\39{-}\T{2},\39\T{10});{}$\2\6
\&{else} \&{if} ${}(\\{cat1}\E\\{attr}){}$\5
${}\{{}$\1\6
${}\\{big\_app1\_insert}(\\{pp},\39\.{'\ '});{}$\6
${}\\{reduce}(\\{pp},\39\T{2},\39\\{exp},\39{-}\T{2},\39\T{142});{}$\6
\4${}\}{}$\2\6
\&{else} \&{if} ${}(\\{cat1}\E\\{colcol}\W\\{cat2}\E\\{int\_like}){}$\1\5
${}\\{squash}(\\{pp},\39\T{3},\39\\{int\_like},\39{-}\T{2},\39\T{152}){}$;\2\par
\U121.\fi

\M{129}\B\X129:Cases for \PB{\\{lpar}}\X${}\E{}$\6
\&{if} ${}((\\{cat1}\E\\{exp}\V\\{cat1}\E\\{ubinop})\W\\{cat2}\E\\{rpar}){}$\1\5
${}\\{squash}(\\{pp},\39\T{3},\39\\{exp},\39{-}\T{2},\39\T{11});{}$\2\6
\&{else} \&{if} ${}(\\{cat1}\E\\{rpar}){}$\5
${}\{{}$\1\6
\\{big\_app1}(\\{pp});\6
\\{app\_str}(\.{"\\\\,"});\6
${}\\{big\_app1}(\\{pp}+\T{1});{}$\6
${}\\{reduce}(\\{pp},\39\T{2},\39\\{exp},\39{-}\T{2},\39\T{12});{}$\6
\4${}\}{}$\2\6
\&{else} \&{if} ${}((\\{cat1}\E\\{decl\_head}\V\\{cat1}\E\\{int\_like}\V%
\\{cat1}\E\\{cast})\W\\{cat2}\E\\{rpar}){}$\1\5
${}\\{squash}(\\{pp},\39\T{3},\39\\{cast},\39{-}\T{2},\39\T{13});{}$\2\6
\&{else} \&{if} ${}((\\{cat1}\E\\{decl\_head}\V\\{cat1}\E\\{int\_like}\V%
\\{cat1}\E\\{exp})\W\\{cat2}\E\\{comma}){}$\5
${}\{{}$\1\6
\\{big\_app3}(\\{pp});\6
\\{app}(\\{opt});\6
\\{app}(\.{'9'});\6
${}\\{reduce}(\\{pp},\39\T{3},\39\\{lpar},\39{-}\T{1},\39\T{14});{}$\6
\4${}\}{}$\2\6
\&{else} \&{if} ${}(\\{cat1}\E\\{stmt}\V\\{cat1}\E\\{decl}){}$\5
${}\{{}$\1\6
\\{big\_app2}(\\{pp});\6
\\{big\_app}(\.{'\ '});\6
${}\\{reduce}(\\{pp},\39\T{2},\39\\{lpar},\39{-}\T{1},\39\T{15});{}$\6
\4${}\}{}$\2\par
\U121.\fi

\M{130}\B\X130:Cases for \PB{\\{unop}}\X${}\E{}$\6
\&{if} ${}(\\{cat1}\E\\{exp}\V\\{cat1}\E\\{int\_like}){}$\1\5
${}\\{squash}(\\{pp},\39\T{2},\39\\{exp},\39{-}\T{2},\39\T{16}){}$;\2\par
\U121.\fi

\M{131}\B\X131:Cases for \PB{\\{ubinop}}\X${}\E{}$\6
\&{if} ${}(\\{cat1}\E\\{cast}\W\\{cat2}\E\\{rpar}){}$\5
${}\{{}$\1\6
\\{big\_app}(\.{'\{'});\6
${}\\{big\_app1\_insert}(\\{pp},\39\.{'\}'});{}$\6
${}\\{reduce}(\\{pp},\39\T{2},\39\\{cast},\39{-}\T{2},\39\T{17});{}$\6
\4${}\}{}$\2\6
\&{else} \&{if} ${}(\\{cat1}\E\\{exp}\V\\{cat1}\E\\{int\_like}){}$\5
${}\{{}$\1\6
\\{big\_app}(\.{'\{'});\6
${}\\{big\_app1\_insert}(\\{pp},\39\.{'\}'});{}$\6
${}\\{reduce}(\\{pp},\39\T{2},\39\\{cat1},\39{-}\T{2},\39\T{18});{}$\6
\4${}\}{}$\2\6
\&{else} \&{if} ${}(\\{cat1}\E\\{binop}){}$\5
${}\{{}$\1\6
\\{big\_app}(\\{math\_rel});\6
${}\\{big\_app1\_insert}(\\{pp},\39\.{'\{'});{}$\6
\\{big\_app}(\.{'\}'});\6
\\{big\_app}(\.{'\}'});\6
${}\\{reduce}(\\{pp},\39\T{2},\39\\{binop},\39{-}\T{1},\39\T{19});{}$\6
\4${}\}{}$\2\par
\U121.\fi

\M{132}\B\X132:Cases for \PB{\\{binop}}\X${}\E{}$\6
\&{if} ${}(\\{cat1}\E\\{binop}){}$\5
${}\{{}$\1\6
\\{big\_app}(\\{math\_rel});\6
\\{big\_app}(\.{'\{'});\6
\\{big\_app1}(\\{pp});\6
\\{big\_app}(\.{'\}'});\6
\\{big\_app}(\.{'\{'});\6
${}\\{big\_app1}(\\{pp}+\T{1});{}$\6
\\{big\_app}(\.{'\}'});\6
\\{big\_app}(\.{'\}'});\6
${}\\{reduce}(\\{pp},\39\T{2},\39\\{binop},\39{-}\T{1},\39\T{20});{}$\6
\4${}\}{}$\2\par
\U121.\fi

\M{133}\B\X133:Cases for \PB{\\{cast}}\X${}\E{}$\6
\&{if} ${}(\\{cat1}\E\\{lpar}){}$\1\5
${}\\{squash}(\\{pp},\39\T{2},\39\\{lpar},\39{-}\T{1},\39\T{21});{}$\2\6
\&{else} \&{if} ${}(\\{cat1}\E\\{exp}){}$\5
${}\{{}$\1\6
${}\\{big\_app1\_insert}(\\{pp},\39\.{'\ '});{}$\6
${}\\{reduce}(\\{pp},\39\T{2},\39\\{exp},\39{-}\T{2},\39\T{21});{}$\6
\4${}\}{}$\2\6
\&{else} \&{if} ${}(\\{cat1}\E\\{semi}){}$\1\5
${}\\{reduce}(\\{pp},\39\T{0},\39\\{exp},\39{-}\T{2},\39\T{22}){}$;\2\par
\U121.\fi

\M{134}\B\X134:Cases for \PB{\\{sizeof\_like}}\X${}\E{}$\6
\&{if} ${}(\\{cat1}\E\\{cast}){}$\1\5
${}\\{squash}(\\{pp},\39\T{2},\39\\{exp},\39{-}\T{2},\39\T{23});{}$\2\6
\&{else} \&{if} ${}(\\{cat1}\E\\{exp}){}$\5
${}\{{}$\1\6
${}\\{big\_app1\_insert}(\\{pp},\39\.{'\ '});{}$\6
${}\\{reduce}(\\{pp},\39\T{2},\39\\{exp},\39{-}\T{2},\39\T{24});{}$\6
\4${}\}{}$\2\par
\U121.\fi

\M{135}\B\X135:Cases for \PB{\\{int\_like}}\X${}\E{}$\6
\&{if} ${}(\\{cat1}\E\\{int\_like}\V\\{cat1}\E\\{struct\_like}){}$\5
${}\{{}$\1\6
${}\\{big\_app1\_insert}(\\{pp},\39\.{'\ '});{}$\6
${}\\{reduce}(\\{pp},\39\T{2},\39\\{cat1},\39{-}\T{2},\39\T{25});{}$\6
\4${}\}{}$\2\6
\&{else} \&{if} ${}(\\{cat1}\E\\{exp}\W(\\{cat2}\E\\{raw\_int}\V\\{cat2}\E%
\\{struct\_like})){}$\1\5
${}\\{squash}(\\{pp},\39\T{2},\39\\{int\_like},\39{-}\T{2},\39\T{26});{}$\2\6
\&{else} \&{if} ${}(\\{cat1}\E\\{exp}\V\\{cat1}\E\\{ubinop}\V\\{cat1}\E%
\\{colon}){}$\5
${}\{{}$\1\6
\\{big\_app1}(\\{pp});\6
\\{big\_app}(\.{'\ '});\6
${}\\{reduce}(\\{pp},\39\T{1},\39\\{decl\_head},\39{-}\T{1},\39\T{27});{}$\6
\4${}\}{}$\2\6
\&{else} \&{if} ${}(\\{cat1}\E\\{semi}\V\\{cat1}\E\\{binop}){}$\1\5
${}\\{reduce}(\\{pp},\39\T{0},\39\\{decl\_head},\39\T{0},\39\T{28}){}$;\2\par
\U121.\fi

\M{136}\B\X136:Cases for \PB{\\{public\_like}}\X${}\E{}$\6
\&{if} ${}(\\{cat1}\E\\{colon}){}$\1\5
${}\\{squash}(\\{pp},\39\T{2},\39\\{tag},\39{-}\T{1},\39\T{29});{}$\2\6
\&{else}\1\5
${}\\{reduce}(\\{pp},\39\T{0},\39\\{int\_like},\39{-}\T{2},\39\T{30}){}$;\2\par
\U121.\fi

\M{137}\B\X137:Cases for \PB{\\{colcol}}\X${}\E{}$\6
\&{if} ${}(\\{cat1}\E\\{exp}\V\\{cat1}\E\\{int\_like}){}$\5
${}\{{}$\1\6
\\{app}(\\{qualifier});\6
${}\\{squash}(\\{pp},\39\T{2},\39\\{cat1},\39{-}\T{2},\39\T{31});{}$\6
\4${}\}{}$\2\6
\&{else} \&{if} ${}(\\{cat1}\E\\{colcol}){}$\1\5
${}\\{squash}(\\{pp},\39\T{2},\39\\{colcol},\39{-}\T{1},\39\T{32}){}$;\2\par
\U121.\fi

\M{138}\B\X138:Cases for \PB{\\{decl\_head}}\X${}\E{}$\6
\&{if} ${}(\\{cat1}\E\\{comma}){}$\5
${}\{{}$\1\6
\\{big\_app2}(\\{pp});\6
\\{big\_app}(\.{'\ '});\6
${}\\{reduce}(\\{pp},\39\T{2},\39\\{decl\_head},\39{-}\T{1},\39\T{33});{}$\6
\4${}\}{}$\2\6
\&{else} \&{if} ${}(\\{cat1}\E\\{ubinop}){}$\5
${}\{{}$\1\6
${}\\{big\_app1\_insert}(\\{pp},\39\.{'\{'});{}$\6
\\{big\_app}(\.{'\}'});\6
${}\\{reduce}(\\{pp},\39\T{2},\39\\{decl\_head},\39{-}\T{1},\39\T{34});{}$\6
\4${}\}{}$\2\6
\&{else} \&{if} ${}(\\{cat1}\E\\{exp}\W\\{cat2}\I\\{lpar}\W\\{cat2}\I\\{lbrack}%
\W\\{cat2}\I\\{exp}\W\\{cat2}\I\\{cast}){}$\5
${}\{{}$\1\6
${}\\{make\_underlined}(\\{pp}+\T{1});{}$\6
${}\\{squash}(\\{pp},\39\T{2},\39\\{decl\_head},\39{-}\T{1},\39\T{35});{}$\6
\4${}\}{}$\2\6
\&{else} \&{if} ${}((\\{cat1}\E\\{binop}\V\\{cat1}\E\\{colon})\W\\{cat2}\E%
\\{exp}\W(\\{cat3}\E\\{comma}\V\\{cat3}\E\\{semi}\V\\{cat3}\E\\{rpar})){}$\1\5
${}\\{squash}(\\{pp},\39\T{3},\39\\{decl\_head},\39{-}\T{1},\39\T{36});{}$\2\6
\&{else} \&{if} ${}(\\{cat1}\E\\{cast}){}$\1\5
${}\\{squash}(\\{pp},\39\T{2},\39\\{decl\_head},\39{-}\T{1},\39\T{37});{}$\2\6
\&{else} \&{if} ${}(\\{cat1}\E\\{lbrace}\V\\{cat1}\E\\{int\_like}\V\\{cat1}\E%
\\{decl}){}$\5
${}\{{}$\1\6
\\{big\_app}(\\{dindent});\6
${}\\{squash}(\\{pp},\39\T{1},\39\\{fn\_decl},\39\T{0},\39\T{38});{}$\6
\4${}\}{}$\2\6
\&{else} \&{if} ${}(\\{cat1}\E\\{semi}){}$\1\5
${}\\{squash}(\\{pp},\39\T{2},\39\\{decl},\39{-}\T{1},\39\T{39});{}$\2\6
\&{else} \&{if} ${}(\\{cat1}\E\\{attr}){}$\5
${}\{{}$\1\6
${}\\{big\_app1\_insert}(\\{pp},\39\.{'\ '});{}$\6
${}\\{reduce}(\\{pp},\39\T{2},\39\\{decl\_head},\39{-}\T{1},\39\T{139});{}$\6
\4${}\}{}$\2\par
\U121.\fi

\M{139}\B\X139:Cases for \PB{\\{decl}}\X${}\E{}$\6
\&{if} ${}(\\{cat1}\E\\{decl}){}$\5
${}\{{}$\1\6
${}\\{big\_app1\_insert}(\\{pp},\39\\{force});{}$\6
${}\\{reduce}(\\{pp},\39\T{2},\39\\{decl},\39{-}\T{1},\39\T{40});{}$\6
\4${}\}{}$\2\6
\&{else} \&{if} ${}(\\{cat1}\E\\{stmt}\V\\{cat1}\E\\{function}){}$\5
${}\{{}$\1\6
${}\\{big\_app1\_insert}(\\{pp},\39\\{big\_force});{}$\6
${}\\{reduce}(\\{pp},\39\T{2},\39\\{cat1},\39{-}\T{1},\39\T{41});{}$\6
\4${}\}{}$\2\par
\U121.\fi

\M{140}\B\X140:Cases for \PB{\\{base}}\X${}\E{}$\6
\&{if} ${}(\\{cat1}\E\\{int\_like}\V\\{cat1}\E\\{exp}){}$\5
${}\{{}$\1\6
\&{if} ${}(\\{cat2}\E\\{comma}){}$\5
${}\{{}$\1\6
\\{big\_app1}(\\{pp});\6
\\{big\_app}(\.{'\ '});\6
${}\\{big\_app2}(\\{pp}+\T{1});{}$\6
\\{app}(\\{opt});\6
\\{app}(\.{'9'});\6
${}\\{reduce}(\\{pp},\39\T{3},\39\\{base},\39\T{0},\39\T{42});{}$\6
\4${}\}{}$\2\6
\&{else} \&{if} ${}(\\{cat2}\E\\{lbrace}){}$\5
${}\{{}$\1\6
${}\\{big\_app1\_insert}(\\{pp},\39\.{'\ '});{}$\6
\\{big\_app}(\.{'\ '});\6
${}\\{big\_app1}(\\{pp}+\T{2});{}$\6
${}\\{reduce}(\\{pp},\39\T{3},\39\\{lbrace},\39{-}\T{2},\39\T{43});{}$\6
\4${}\}{}$\2\6
\4${}\}{}$\2\par
\U121.\fi

\M{141}\B\X141:Cases for \PB{\\{struct\_like}}\X${}\E{}$\6
\&{if} ${}(\\{cat1}\E\\{lbrace}){}$\5
${}\{{}$\1\6
${}\\{big\_app1\_insert}(\\{pp},\39\.{'\ '});{}$\6
${}\\{reduce}(\\{pp},\39\T{2},\39\\{struct\_head},\39\T{0},\39\T{44});{}$\6
\4${}\}{}$\2\6
\&{else} \&{if} ${}(\\{cat1}\E\\{exp}\V\\{cat1}\E\\{int\_like}){}$\5
${}\{{}$\1\6
\&{if} ${}(\\{cat2}\E\\{lbrace}\V\\{cat2}\E\\{semi}){}$\5
${}\{{}$\1\6
${}\\{make\_underlined}(\\{pp}+\T{1});{}$\6
${}\\{make\_reserved}(\\{pp}+\T{1});{}$\6
${}\\{big\_app1\_insert}(\\{pp},\39\.{'\ '});{}$\6
\&{if} ${}(\\{cat2}\E\\{semi}){}$\1\5
${}\\{reduce}(\\{pp},\39\T{2},\39\\{decl\_head},\39\T{0},\39\T{45});{}$\2\6
\&{else}\5
${}\{{}$\1\6
\\{big\_app}(\.{'\ '});\6
${}\\{big\_app1}(\\{pp}+\T{2});{}$\6
${}\\{reduce}(\\{pp},\39\T{3},\39\\{struct\_head},\39\T{0},\39\T{46});{}$\6
\4${}\}{}$\2\6
\4${}\}{}$\2\6
\&{else} \&{if} ${}(\\{cat2}\E\\{colon}){}$\1\5
${}\\{reduce}(\\{pp}+\T{2},\39\T{0},\39\\{base},\39\T{2},\39\T{47});{}$\2\6
\&{else} \&{if} ${}(\\{cat2}\I\\{base}){}$\5
${}\{{}$\1\6
${}\\{big\_app1\_insert}(\\{pp},\39\.{'\ '});{}$\6
${}\\{reduce}(\\{pp},\39\T{2},\39\\{int\_like},\39{-}\T{2},\39\T{48});{}$\6
\4${}\}{}$\2\6
\4${}\}{}$\2\6
\&{else} \&{if} ${}(\\{cat1}\E\\{attr}){}$\5
${}\{{}$\1\6
${}\\{big\_app1\_insert}(\\{pp},\39\.{'\ '});{}$\6
${}\\{reduce}(\\{pp},\39\T{2},\39\\{struct\_like},\39{-}\T{3},\39\T{141});{}$\6
\4${}\}{}$\2\6
\&{else} \&{if} ${}(\\{cat1}\E\\{struct\_like}){}$\5
${}\{{}$\1\6
${}\\{big\_app1\_insert}(\\{pp},\39\.{'\ '});{}$\6
${}\\{reduce}(\\{pp},\39\T{2},\39\\{struct\_like},\39{-}\T{3},\39\T{151});{}$\6
\4${}\}{}$\2\par
\U121.\fi

\M{142}\B\X142:Cases for \PB{\\{struct\_head}}\X${}\E{}$\6
\&{if} ${}((\\{cat1}\E\\{decl}\V\\{cat1}\E\\{stmt}\V\\{cat1}\E\\{function})\W%
\\{cat2}\E\\{rbrace}){}$\5
${}\{{}$\1\6
\\{big\_app1}(\\{pp});\6
\\{big\_app}(\\{indent});\6
\\{big\_app}(\\{force});\6
${}\\{big\_app1}(\\{pp}+\T{1});{}$\6
\\{big\_app}(\\{outdent});\6
\\{big\_app}(\\{force});\6
${}\\{big\_app1}(\\{pp}+\T{2});{}$\6
${}\\{reduce}(\\{pp},\39\T{3},\39\\{int\_like},\39{-}\T{2},\39\T{49});{}$\6
\4${}\}{}$\2\6
\&{else} \&{if} ${}(\\{cat1}\E\\{rbrace}){}$\5
${}\{{}$\1\6
\\{big\_app1}(\\{pp});\6
\\{app\_str}(\.{"\\\\,"});\6
${}\\{big\_app1}(\\{pp}+\T{1});{}$\6
${}\\{reduce}(\\{pp},\39\T{2},\39\\{int\_like},\39{-}\T{2},\39\T{50});{}$\6
\4${}\}{}$\2\par
\U121.\fi

\M{143}\B\X143:Cases for \PB{\\{fn\_decl}}\X${}\E{}$\6
\&{if} ${}(\\{cat1}\E\\{decl}){}$\5
${}\{{}$\1\6
${}\\{big\_app1\_insert}(\\{pp},\39\\{force});{}$\6
${}\\{reduce}(\\{pp},\39\T{2},\39\\{fn\_decl},\39\T{0},\39\T{51});{}$\6
\4${}\}{}$\2\6
\&{else} \&{if} ${}(\\{cat1}\E\\{stmt}){}$\5
${}\{{}$\1\6
\\{big\_app1}(\\{pp});\6
\\{app}(\\{outdent});\6
\\{app}(\\{outdent});\6
\\{big\_app}(\\{force});\6
${}\\{big\_app1}(\\{pp}+\T{1});{}$\6
${}\\{reduce}(\\{pp},\39\T{2},\39\\{function},\39{-}\T{1},\39\T{52});{}$\6
\4${}\}{}$\2\6
\&{else} \&{if} ${}(\\{cat1}\E\\{attr}){}$\5
${}\{{}$\1\6
${}\\{big\_app1\_insert}(\\{pp},\39\.{'\ '});{}$\6
${}\\{reduce}(\\{pp},\39\T{2},\39\\{fn\_decl},\39\T{0},\39\T{157});{}$\6
\4${}\}{}$\2\par
\U121.\fi

\M{144}\B\X144:Cases for \PB{\\{function}}\X${}\E{}$\6
\&{if} ${}(\\{cat1}\E\\{function}\V\\{cat1}\E\\{decl}\V\\{cat1}\E\\{stmt}){}$\5
${}\{{}$\1\6
${}\\{big\_app1\_insert}(\\{pp},\39\\{big\_force});{}$\6
${}\\{reduce}(\\{pp},\39\T{2},\39\\{cat1},\39{-}\T{1},\39\T{53});{}$\6
\4${}\}{}$\2\par
\U121.\fi

\M{145}\B\X145:Cases for \PB{\\{lbrace}}\X${}\E{}$\6
\&{if} ${}(\\{cat1}\E\\{rbrace}){}$\5
${}\{{}$\1\6
\\{big\_app1}(\\{pp});\6
\\{app\_str}(\.{"\\\\,"});\6
${}\\{big\_app1}(\\{pp}+\T{1});{}$\6
${}\\{reduce}(\\{pp},\39\T{2},\39\\{stmt},\39{-}\T{1},\39\T{54});{}$\6
\4${}\}{}$\2\6
\&{else} \&{if} ${}((\\{cat1}\E\\{stmt}\V\\{cat1}\E\\{decl}\V\\{cat1}\E%
\\{function})\W\\{cat2}\E\\{rbrace}){}$\5
${}\{{}$\1\6
\\{big\_app}(\\{force});\6
\\{big\_app1}(\\{pp});\6
\\{big\_app}(\\{indent});\6
\\{big\_app}(\\{force});\6
${}\\{big\_app1}(\\{pp}+\T{1});{}$\6
\\{big\_app}(\\{force});\6
\\{big\_app}(\\{backup});\6
${}\\{big\_app1}(\\{pp}+\T{2});{}$\6
\\{big\_app}(\\{outdent});\6
\\{big\_app}(\\{force});\6
${}\\{reduce}(\\{pp},\39\T{3},\39\\{stmt},\39{-}\T{1},\39\T{55});{}$\6
\4${}\}{}$\2\6
\&{else} \&{if} ${}(\\{cat1}\E\\{exp}){}$\5
${}\{{}$\1\6
\&{if} ${}(\\{cat2}\E\\{rbrace}){}$\1\5
${}\\{squash}(\\{pp},\39\T{3},\39\\{exp},\39{-}\T{2},\39\T{56});{}$\2\6
\&{else} \&{if} ${}(\\{cat2}\E\\{comma}\W\\{cat3}\E\\{rbrace}){}$\1\5
${}\\{squash}(\\{pp},\39\T{4},\39\\{exp},\39{-}\T{2},\39\T{56});{}$\2\6
\4${}\}{}$\2\par
\U121.\fi

\M{146}\B\X146:Cases for \PB{\\{if\_like}}\X${}\E{}$\6
\&{if} ${}(\\{cat1}\E\\{exp}){}$\5
${}\{{}$\1\6
${}\\{big\_app1\_insert}(\\{pp},\39\.{'\ '});{}$\6
${}\\{reduce}(\\{pp},\39\T{2},\39\\{if\_clause},\39\T{0},\39\T{57});{}$\6
\4${}\}{}$\2\par
\U121.\fi

\M{147}\B\X147:Cases for \PB{\\{else\_like}}\X${}\E{}$\6
\&{if} ${}(\\{cat1}\E\\{colon}){}$\1\5
${}\\{reduce}(\\{pp}+\T{1},\39\T{0},\39\\{base},\39\T{1},\39\T{58});{}$\2\6
\&{else} \&{if} ${}(\\{cat1}\E\\{lbrace}){}$\1\5
${}\\{reduce}(\\{pp},\39\T{0},\39\\{else\_head},\39\T{0},\39\T{59});{}$\2\6
\&{else} \&{if} ${}(\\{cat1}\E\\{stmt}){}$\5
${}\{{}$\1\6
\\{big\_app}(\\{force});\6
\\{big\_app1}(\\{pp});\6
\\{big\_app}(\\{indent});\6
\\{big\_app}(\\{break\_space});\6
${}\\{big\_app1}(\\{pp}+\T{1});{}$\6
\\{big\_app}(\\{outdent});\6
\\{big\_app}(\\{force});\6
${}\\{reduce}(\\{pp},\39\T{2},\39\\{stmt},\39{-}\T{1},\39\T{60});{}$\6
\4${}\}{}$\2\par
\U121.\fi

\M{148}\B\X148:Cases for \PB{\\{else\_head}}\X${}\E{}$\6
\&{if} ${}(\\{cat1}\E\\{stmt}\V\\{cat1}\E\\{exp}){}$\5
${}\{{}$\1\6
\\{big\_app}(\\{force});\6
\\{big\_app1}(\\{pp});\6
\\{big\_app}(\\{break\_space});\6
\\{app}(\\{noop});\6
\\{big\_app}(\\{cancel});\6
${}\\{big\_app1}(\\{pp}+\T{1});{}$\6
\\{big\_app}(\\{force});\6
${}\\{reduce}(\\{pp},\39\T{2},\39\\{stmt},\39{-}\T{1},\39\T{61});{}$\6
\4${}\}{}$\2\par
\U121.\fi

\M{149}\B\X149:Cases for \PB{\\{if\_clause}}\X${}\E{}$\6
\&{if} ${}(\\{cat1}\E\\{lbrace}){}$\1\5
${}\\{reduce}(\\{pp},\39\T{0},\39\\{if\_head},\39\T{0},\39\T{62});{}$\2\6
\&{else} \&{if} ${}(\\{cat1}\E\\{stmt}){}$\5
${}\{{}$\1\6
\&{if} ${}(\\{cat2}\E\\{else\_like}){}$\5
${}\{{}$\1\6
\\{big\_app}(\\{force});\6
\\{big\_app1}(\\{pp});\6
\\{big\_app}(\\{indent});\6
\\{big\_app}(\\{break\_space});\6
${}\\{big\_app1}(\\{pp}+\T{1});{}$\6
\\{big\_app}(\\{outdent});\6
\\{big\_app}(\\{force});\6
${}\\{big\_app1}(\\{pp}+\T{2});{}$\6
\&{if} ${}(\\{cat3}\E\\{if\_like}){}$\5
${}\{{}$\1\6
\\{big\_app}(\.{'\ '});\6
${}\\{big\_app1}(\\{pp}+\T{3});{}$\6
${}\\{reduce}(\\{pp},\39\T{4},\39\\{if\_like},\39\T{0},\39\T{63});{}$\6
\4${}\}{}$\2\6
\&{else}\1\5
${}\\{reduce}(\\{pp},\39\T{3},\39\\{else\_like},\39\T{0},\39\T{64});{}$\2\6
\4${}\}{}$\2\6
\&{else}\1\5
${}\\{reduce}(\\{pp},\39\T{0},\39\\{else\_like},\39\T{0},\39\T{65});{}$\2\6
\4${}\}{}$\2\6
\&{else} \&{if} ${}(\\{cat1}\E\\{attr}){}$\5
${}\{{}$\1\6
${}\\{big\_app1\_insert}(\\{pp},\39\.{'\ '});{}$\6
${}\\{reduce}(\\{pp},\39\T{2},\39\\{if\_head},\39\T{0},\39\T{146});{}$\6
\4${}\}{}$\2\par
\U121.\fi

\M{150}\B\X150:Cases for \PB{\\{if\_head}}\X${}\E{}$\6
\&{if} ${}(\\{cat1}\E\\{stmt}\V\\{cat1}\E\\{exp}){}$\5
${}\{{}$\1\6
\&{if} ${}(\\{cat2}\E\\{else\_like}){}$\5
${}\{{}$\1\6
\\{big\_app}(\\{force});\6
\\{big\_app1}(\\{pp});\6
\\{big\_app}(\\{break\_space});\6
\\{app}(\\{noop});\6
\\{big\_app}(\\{cancel});\6
${}\\{big\_app1\_insert}(\\{pp}+\T{1},\39\\{force});{}$\6
\&{if} ${}(\\{cat3}\E\\{if\_like}){}$\5
${}\{{}$\1\6
\\{big\_app}(\.{'\ '});\6
${}\\{big\_app1}(\\{pp}+\T{3});{}$\6
${}\\{reduce}(\\{pp},\39\T{4},\39\\{if\_like},\39\T{0},\39\T{66});{}$\6
\4${}\}{}$\2\6
\&{else}\1\5
${}\\{reduce}(\\{pp},\39\T{3},\39\\{else\_like},\39\T{0},\39\T{67});{}$\2\6
\4${}\}{}$\2\6
\&{else}\1\5
${}\\{reduce}(\\{pp},\39\T{0},\39\\{else\_head},\39\T{0},\39\T{68});{}$\2\6
\4${}\}{}$\2\par
\U121.\fi

\M{151}\B\X151:Cases for \PB{\\{do\_like}}\X${}\E{}$\6
\&{if} ${}(\\{cat1}\E\\{stmt}\W\\{cat2}\E\\{else\_like}\W\\{cat3}\E\\{semi}){}$%
\5
${}\{{}$\1\6
\\{big\_app1}(\\{pp});\6
\\{big\_app}(\\{break\_space});\6
\\{app}(\\{noop});\6
\\{big\_app}(\\{cancel});\6
${}\\{big\_app1}(\\{pp}+\T{1});{}$\6
\\{big\_app}(\\{cancel});\6
\\{app}(\\{noop});\6
\\{big\_app}(\\{break\_space});\6
${}\\{big\_app2}(\\{pp}+\T{2});{}$\6
${}\\{reduce}(\\{pp},\39\T{4},\39\\{stmt},\39{-}\T{1},\39\T{69});{}$\6
\4${}\}{}$\2\par
\U121.\fi

\M{152}\B\X152:Cases for \PB{\\{case\_like}}\X${}\E{}$\6
\&{if} ${}(\\{cat1}\E\\{semi}){}$\1\5
${}\\{squash}(\\{pp},\39\T{2},\39\\{stmt},\39{-}\T{1},\39\T{70});{}$\2\6
\&{else} \&{if} ${}(\\{cat1}\E\\{colon}){}$\1\5
${}\\{squash}(\\{pp},\39\T{2},\39\\{tag},\39{-}\T{1},\39\T{71});{}$\2\6
\&{else} \&{if} ${}(\\{cat1}\E\\{exp}){}$\5
${}\{{}$\1\6
${}\\{big\_app1\_insert}(\\{pp},\39\.{'\ '});{}$\6
${}\\{reduce}(\\{pp},\39\T{2},\39\\{exp},\39{-}\T{2},\39\T{72});{}$\6
\4${}\}{}$\2\par
\U121.\fi

\M{153}\B\X153:Cases for \PB{\\{catch\_like}}\X${}\E{}$\6
\&{if} ${}(\\{cat1}\E\\{cast}\V\\{cat1}\E\\{exp}){}$\5
${}\{{}$\1\6
${}\\{big\_app1\_insert}(\\{pp},\39\\{dindent});{}$\6
${}\\{reduce}(\\{pp},\39\T{2},\39\\{fn\_decl},\39\T{0},\39\T{73});{}$\6
\4${}\}{}$\2\par
\U121.\fi

\M{154}\B\X154:Cases for \PB{\\{tag}}\X${}\E{}$\6
\&{if} ${}(\\{cat1}\E\\{tag}){}$\5
${}\{{}$\1\6
${}\\{big\_app1\_insert}(\\{pp},\39\\{break\_space});{}$\6
${}\\{reduce}(\\{pp},\39\T{2},\39\\{tag},\39{-}\T{1},\39\T{74});{}$\6
\4${}\}{}$\2\6
\&{else} \&{if} ${}(\\{cat1}\E\\{stmt}\V\\{cat1}\E\\{decl}\V\\{cat1}\E%
\\{function}){}$\5
${}\{{}$\1\6
\\{big\_app}(\\{force});\6
\\{big\_app}(\\{backup});\6
${}\\{big\_app1\_insert}(\\{pp},\39\\{break\_space});{}$\6
${}\\{reduce}(\\{pp},\39\T{2},\39\\{cat1},\39{-}\T{1},\39\T{75});{}$\6
\4${}\}{}$\2\6
\&{else} \&{if} ${}(\\{cat1}\E\\{rbrace}){}$\1\5
${}\\{reduce}(\\{pp},\39\T{0},\39\\{decl},\39{-}\T{1},\39\T{156}){}$;\2\par
\U121.\fi

\M{155}The user can decide at run-time whether short statements should be
grouped together on the same line.

\Y\B\4\D\\{force\_lines}\5
\\{flags}[\.{'f'}]\C{ should each statement be on its own line? }\par
\Y\B\4\X24:Set initial values\X${}\mathrel+\E{}$\6
$\\{force\_lines}\K\\{true}{}$;\par
\fi

\M{156}\B\X156:Cases for \PB{\\{stmt}}\X${}\E{}$\6
\&{if} ${}(\\{cat1}\E\\{stmt}\V\\{cat1}\E\\{decl}\V\\{cat1}\E\\{function}){}$\5
${}\{{}$\1\6
${}\\{big\_app1\_insert}(\\{pp},\39(\\{cat1}\E\\{function}\V\\{cat1}\E\\{decl})%
\?\\{big\_force}:\\{force\_lines}\?\\{force}:\\{break\_space});{}$\6
${}\\{reduce}(\\{pp},\39\T{2},\39\\{cat1},\39{-}\T{1},\39\T{76});{}$\6
\4${}\}{}$\2\par
\U121.\fi

\M{157}\B\X157:Cases for \PB{\\{semi}}\X${}\E{}$\6
\\{big\_app}(\.{'\ '});\6
${}\\{squash}(\\{pp},\39\T{1},\39\\{stmt},\39{-}\T{1},\39\T{77}){}$;\par
\U121.\fi

\M{158}\B\X158:Cases for \PB{\\{lproc}}\X${}\E{}$\6
\&{if} ${}(\\{cat1}\E\\{define\_like}){}$\1\5
${}\\{make\_underlined}(\\{pp}+\T{2});{}$\2\6
\&{if} ${}(\\{cat1}\E\\{else\_like}\V\\{cat1}\E\\{if\_like}\V\\{cat1}\E%
\\{define\_like}){}$\1\5
${}\\{squash}(\\{pp},\39\T{2},\39\\{lproc},\39\T{0},\39\T{78});{}$\2\6
\&{else} \&{if} ${}(\\{cat1}\E\\{rproc}){}$\5
${}\{{}$\1\6
\\{app}(\\{inserted});\6
${}\\{squash}(\\{pp},\39\T{2},\39\\{insert},\39{-}\T{1},\39\T{79});{}$\6
\4${}\}{}$\2\6
\&{else} \&{if} ${}(\\{cat1}\E\\{exp}\V\\{cat1}\E\\{function}){}$\5
${}\{{}$\1\6
\&{if} ${}(\\{cat2}\E\\{rproc}){}$\5
${}\{{}$\1\6
\\{app}(\\{inserted});\6
\\{big\_app1}(\\{pp});\6
\\{big\_app}(\.{'\ '});\6
${}\\{big\_app2}(\\{pp}+\T{1});{}$\6
${}\\{reduce}(\\{pp},\39\T{3},\39\\{insert},\39{-}\T{1},\39\T{80});{}$\6
\4${}\}{}$\2\6
\&{else} \&{if} ${}(\\{cat1}\E\\{exp}\W\\{cat2}\E\\{exp}\W\\{cat3}\E%
\\{rproc}){}$\5
${}\{{}$\1\6
\\{app}(\\{inserted});\6
${}\\{big\_app1\_insert}(\\{pp},\39\.{'\ '});{}$\6
\\{app\_str}(\.{"\\\\5"});\6
${}\\{big\_app2}(\\{pp}+\T{2});{}$\6
${}\\{reduce}(\\{pp},\39\T{4},\39\\{insert},\39{-}\T{1},\39\T{80});{}$\6
\4${}\}{}$\2\6
\4${}\}{}$\2\par
\U121.\fi

\M{159}\B\X159:Cases for \PB{\\{section\_scrap}}\X${}\E{}$\6
\&{if} ${}(\\{cat1}\E\\{semi}){}$\5
${}\{{}$\1\6
\\{big\_app2}(\\{pp});\6
\\{big\_app}(\\{force});\6
${}\\{reduce}(\\{pp},\39\T{2},\39\\{stmt},\39{-}\T{2},\39\T{81});{}$\6
\4${}\}{}$\2\6
\&{else}\1\5
${}\\{reduce}(\\{pp},\39\T{0},\39\\{exp},\39{-}\T{2},\39\T{82}){}$;\2\par
\U121.\fi

\M{160}\B\X160:Cases for \PB{\\{insert}}\X${}\E{}$\6
\&{if} (\\{cat1})\1\5
${}\\{squash}(\\{pp},\39\T{2},\39\\{cat1},\39\T{0},\39\T{83}){}$;\2\par
\U121.\fi

\M{161}\B\X161:Cases for \PB{\\{prelangle}}\X${}\E{}$\6
$\\{init\_mathness}\K\\{cur\_mathness}\K\\{yes\_math};{}$\6
\\{app}(\.{'<'});\6
${}\\{reduce}(\\{pp},\39\T{1},\39\\{binop},\39{-}\T{2},\39\T{84}){}$;\par
\U121.\fi

\M{162}\B\X162:Cases for \PB{\\{prerangle}}\X${}\E{}$\6
$\\{init\_mathness}\K\\{cur\_mathness}\K\\{yes\_math};{}$\6
\\{app}(\.{'>'});\6
${}\\{reduce}(\\{pp},\39\T{1},\39\\{binop},\39{-}\T{2},\39\T{85}){}$;\par
\U121.\fi

\M{163}\B\D\\{reserve\_typenames}\5
\\{flags}[\.{'t'}]\C{ should we treat \&{typename} in a template like %
\&{typedef}? }\par
\Y\B\4\X163:Cases for \PB{\\{langle}}\X${}\E{}$\6
\&{if} ${}(\\{cat1}\E\\{prerangle}){}$\5
${}\{{}$\1\6
\\{big\_app1}(\\{pp});\6
\\{app\_str}(\.{"\\\\,"});\6
${}\\{big\_app1}(\\{pp}+\T{1});{}$\6
${}\\{reduce}(\\{pp},\39\T{2},\39\\{cast},\39{-}\T{1},\39\T{86});{}$\6
\4${}\}{}$\2\6
\&{else} \&{if} ${}(\\{cat1}\E\\{decl\_head}\V\\{cat1}\E\\{int\_like}\V\\{cat1}%
\E\\{exp}){}$\5
${}\{{}$\1\6
\&{if} ${}(\\{cat2}\E\\{prerangle}){}$\1\5
${}\\{squash}(\\{pp},\39\T{3},\39\\{cast},\39{-}\T{1},\39\T{87});{}$\2\6
\&{else} \&{if} ${}(\\{cat2}\E\\{comma}){}$\5
${}\{{}$\1\6
\\{big\_app3}(\\{pp});\6
\\{app}(\\{opt});\6
\\{app}(\.{'9'});\6
${}\\{reduce}(\\{pp},\39\T{3},\39\\{langle},\39\T{0},\39\T{88});{}$\6
\4${}\}{}$\2\6
\4${}\}{}$\2\6
\&{else} \&{if} ${}((\\{cat1}\E\\{struct\_like})\3{-1}\W(\\{cat2}\E\\{exp}\V%
\\{cat2}\E\\{int\_like})\3{-1}\W(\\{cat3}\E\\{comma}\V\\{cat3}\E%
\\{prerangle})){}$\5
${}\{{}$\1\6
${}\\{make\_underlined}(\\{pp}+\T{2});{}$\6
\&{if} (\\{reserve\_typenames})\1\5
${}\\{make\_reserved}(\\{pp}+\T{2});{}$\2\6
\\{big\_app2}(\\{pp});\6
\\{big\_app}(\.{'\ '});\6
${}\\{big\_app2}(\\{pp}+\T{2});{}$\6
\&{if} ${}(\\{cat3}\E\\{comma}){}$\1\5
${}\\{reduce}(\\{pp},\39\T{4},\39\\{langle},\39\T{0},\39\T{153});{}$\2\6
\&{else}\1\5
${}\\{reduce}(\\{pp},\39\T{4},\39\\{cast},\39{-}\T{1},\39\T{154});{}$\2\6
\4${}\}{}$\2\par
\U121.\fi

\M{164}\B\X164:Cases for \PB{\\{template\_like}}\X${}\E{}$\6
\&{if} ${}(\\{cat1}\E\\{exp}\W\\{cat2}\E\\{prelangle}){}$\1\5
${}\\{reduce}(\\{pp}+\T{2},\39\T{0},\39\\{langle},\39\T{2},\39\T{89});{}$\2\6
\&{else} \&{if} ${}(\\{cat1}\E\\{exp}\V\\{cat1}\E\\{raw\_int}){}$\5
${}\{{}$\1\6
${}\\{big\_app1\_insert}(\\{pp},\39\.{'\ '});{}$\6
${}\\{reduce}(\\{pp},\39\T{2},\39\\{cat1},\39{-}\T{2},\39\T{90});{}$\6
\4${}\}{}$\2\6
\&{else} \&{if} ${}(\\{cat1}\E\\{cast}\W\\{cat2}\E\\{struct\_like}){}$\5
${}\{{}$\1\6
${}\\{big\_app1\_insert}(\\{pp},\39\.{'\ '});{}$\6
${}\\{reduce}(\\{pp},\39\T{2},\39\\{struct\_like},\39\T{0},\39\T{155});{}$\6
\4${}\}{}$\2\6
\&{else}\1\5
${}\\{reduce}(\\{pp},\39\T{0},\39\\{raw\_int},\39\T{0},\39\T{91}){}$;\2\par
\U121.\fi

\M{165}\B\X165:Cases for \PB{\\{new\_like}}\X${}\E{}$\6
\&{if} ${}(\\{cat1}\E\\{lpar}\W\\{cat2}\E\\{exp}\W\\{cat3}\E\\{rpar}){}$\1\5
${}\\{squash}(\\{pp},\39\T{4},\39\\{new\_like},\39\T{0},\39\T{92});{}$\2\6
\&{else} \&{if} ${}(\\{cat1}\E\\{cast}){}$\5
${}\{{}$\1\6
${}\\{big\_app1\_insert}(\\{pp},\39\.{'\ '});{}$\6
${}\\{reduce}(\\{pp},\39\T{2},\39\\{exp},\39{-}\T{2},\39\T{93});{}$\6
\4${}\}{}$\2\6
\&{else} \&{if} ${}(\\{cat1}\I\\{lpar}){}$\1\5
${}\\{reduce}(\\{pp},\39\T{0},\39\\{new\_exp},\39\T{0},\39\T{94}){}$;\2\par
\U121.\fi

\M{166}\B\X166:Cases for \PB{\\{new\_exp}}\X${}\E{}$\6
\&{if} ${}(\\{cat1}\E\\{int\_like}\V\\{cat1}\E\\{const\_like}){}$\5
${}\{{}$\1\6
${}\\{big\_app1\_insert}(\\{pp},\39\.{'\ '});{}$\6
${}\\{reduce}(\\{pp},\39\T{2},\39\\{new\_exp},\39\T{0},\39\T{95});{}$\6
\4${}\}{}$\2\6
\&{else} \&{if} ${}(\\{cat1}\E\\{struct\_like}\W(\\{cat2}\E\\{exp}\V\\{cat2}\E%
\\{int\_like})){}$\5
${}\{{}$\1\6
${}\\{big\_app1\_insert}(\\{pp},\39\.{'\ '});{}$\6
\\{big\_app}(\.{'\ '});\6
${}\\{big\_app1}(\\{pp}+\T{2});{}$\6
${}\\{reduce}(\\{pp},\39\T{3},\39\\{new\_exp},\39\T{0},\39\T{96});{}$\6
\4${}\}{}$\2\6
\&{else} \&{if} ${}(\\{cat1}\E\\{raw\_ubin}){}$\5
${}\{{}$\1\6
${}\\{big\_app1\_insert}(\\{pp},\39\.{'\{'});{}$\6
\\{big\_app}(\.{'\}'});\6
${}\\{reduce}(\\{pp},\39\T{2},\39\\{new\_exp},\39\T{0},\39\T{97});{}$\6
\4${}\}{}$\2\6
\&{else} \&{if} ${}(\\{cat1}\E\\{lpar}){}$\1\5
${}\\{reduce}(\\{pp},\39\T{0},\39\\{exp},\39{-}\T{2},\39\T{98});{}$\2\6
\&{else} \&{if} ${}(\\{cat1}\E\\{exp}){}$\5
${}\{{}$\1\6
\\{big\_app1}(\\{pp});\6
\\{big\_app}(\.{'\ '});\6
${}\\{reduce}(\\{pp},\39\T{1},\39\\{exp},\39{-}\T{2},\39\T{98});{}$\6
\4${}\}{}$\2\6
\&{else} \&{if} ${}(\\{cat1}\I\\{raw\_int}\W\\{cat1}\I\\{struct\_like}\W%
\\{cat1}\I\\{colcol}){}$\1\5
${}\\{reduce}(\\{pp},\39\T{0},\39\\{exp},\39{-}\T{2},\39\T{99}){}$;\2\par
\U121.\fi

\M{167}\B\X167:Cases for \PB{\\{ftemplate}}\X${}\E{}$\6
\&{if} ${}(\\{cat1}\E\\{prelangle}){}$\1\5
${}\\{reduce}(\\{pp}+\T{1},\39\T{0},\39\\{langle},\39\T{1},\39\T{100});{}$\2\6
\&{else}\1\5
${}\\{reduce}(\\{pp},\39\T{0},\39\\{exp},\39{-}\T{2},\39\T{101}){}$;\2\par
\U121.\fi

\M{168}\B\X168:Cases for \PB{\\{for\_like}}\X${}\E{}$\6
\&{if} ${}(\\{cat1}\E\\{exp}){}$\5
${}\{{}$\1\6
${}\\{big\_app1\_insert}(\\{pp},\39\.{'\ '});{}$\6
${}\\{reduce}(\\{pp},\39\T{2},\39\\{else\_like},\39{-}\T{2},\39\T{102});{}$\6
\4${}\}{}$\2\par
\U121.\fi

\M{169}\B\X169:Cases for \PB{\\{raw\_ubin}}\X${}\E{}$\6
\&{if} ${}(\\{cat1}\E\\{const\_like}){}$\5
${}\{{}$\1\6
\\{big\_app2}(\\{pp});\6
\\{app\_str}(\.{"\\\\\ "});\6
${}\\{reduce}(\\{pp},\39\T{2},\39\\{raw\_ubin},\39\T{0},\39\T{103});{}$\6
\4${}\}{}$\2\6
\&{else}\1\5
${}\\{reduce}(\\{pp},\39\T{0},\39\\{ubinop},\39{-}\T{2},\39\T{104}){}$;\2\par
\U121.\fi

\M{170}\B\X170:Cases for \PB{\\{const\_like}}\X${}\E{}$\6
$\\{reduce}(\\{pp},\39\T{0},\39\\{int\_like},\39{-}\T{2},\39\T{105}){}$;\par
\U121.\fi

\M{171}\B\X171:Cases for \PB{\\{raw\_int}}\X${}\E{}$\6
\&{if} ${}(\\{cat1}\E\\{prelangle}){}$\1\5
${}\\{reduce}(\\{pp}+\T{1},\39\T{0},\39\\{langle},\39\T{1},\39\T{106});{}$\2\6
\&{else} \&{if} ${}(\\{cat1}\E\\{colcol}){}$\1\5
${}\\{squash}(\\{pp},\39\T{2},\39\\{colcol},\39{-}\T{1},\39\T{107});{}$\2\6
\&{else} \&{if} ${}(\\{cat1}\E\\{cast}){}$\1\5
${}\\{squash}(\\{pp},\39\T{2},\39\\{raw\_int},\39\T{0},\39\T{108});{}$\2\6
\&{else} \&{if} ${}(\\{cat1}\E\\{lpar}){}$\1\5
${}\\{reduce}(\\{pp},\39\T{0},\39\\{exp},\39{-}\T{2},\39\T{109});{}$\2\6
\&{else} \&{if} ${}(\\{cat1}\E\\{lbrack}){}$\1\5
${}\\{reduce}(\\{pp},\39\T{0},\39\\{exp},\39{-}\T{2},\39\T{144});{}$\2\6
\&{else} \&{if} ${}(\\{cat1}\I\\{langle}){}$\1\5
${}\\{reduce}(\\{pp},\39\T{0},\39\\{int\_like},\39{-}\T{3},\39\T{110}){}$;\2\par
\U121.\fi

\M{172}\B\X172:Cases for \PB{\\{operator\_like}}\X${}\E{}$\6
\&{if} ${}(\\{cat1}\E\\{binop}\V\\{cat1}\E\\{unop}\V\\{cat1}\E\\{ubinop}){}$\5
${}\{{}$\1\6
\&{if} ${}(\\{cat2}\E\\{binop}){}$\1\5
\&{break};\2\6
${}\\{big\_app1\_insert}(\\{pp},\39\.{'\{'});{}$\6
\\{big\_app}(\.{'\}'});\6
${}\\{reduce}(\\{pp},\39\T{2},\39\\{exp},\39{-}\T{2},\39\T{111});{}$\6
\4${}\}{}$\2\6
\&{else} \&{if} ${}(\\{cat1}\E\\{new\_like}\V\\{cat1}\E\\{delete\_like}){}$\5
${}\{{}$\1\6
${}\\{big\_app1\_insert}(\\{pp},\39\.{'\ '});{}$\6
${}\\{reduce}(\\{pp},\39\T{2},\39\\{exp},\39{-}\T{2},\39\T{112});{}$\6
\4${}\}{}$\2\6
\&{else} \&{if} ${}(\\{cat1}\E\\{comma}){}$\1\5
${}\\{squash}(\\{pp},\39\T{2},\39\\{exp},\39{-}\T{2},\39\T{113});{}$\2\6
\&{else} \&{if} ${}(\\{cat1}\I\\{raw\_ubin}){}$\1\5
${}\\{reduce}(\\{pp},\39\T{0},\39\\{new\_exp},\39\T{0},\39\T{114}){}$;\2\par
\U121.\fi

\M{173}\B\X173:Cases for \PB{\\{typedef\_like}}\X${}\E{}$\6
\&{if} ${}((\\{cat1}\E\\{int\_like}\V\\{cat1}\E\\{cast})\W(\\{cat2}\E\\{comma}%
\V\\{cat2}\E\\{semi})){}$\1\5
${}\\{reduce}(\\{pp}+\T{1},\39\T{0},\39\\{exp},\39{-}\T{1},\39\T{115});{}$\2\6
\&{else} \&{if} ${}(\\{cat1}\E\\{int\_like}){}$\5
${}\{{}$\1\6
${}\\{big\_app1\_insert}(\\{pp},\39\.{'\ '});{}$\6
${}\\{reduce}(\\{pp},\39\T{2},\39\\{typedef\_like},\39\T{0},\39\T{116});{}$\6
\4${}\}{}$\2\6
\&{else} \&{if} ${}(\\{cat1}\E\\{exp}\W\\{cat2}\I\\{lpar}\W\\{cat2}\I\\{exp}\W%
\\{cat2}\I\\{cast}){}$\5
${}\{{}$\1\6
${}\\{make\_underlined}(\\{pp}+\T{1});{}$\6
${}\\{make\_reserved}(\\{pp}+\T{1});{}$\6
${}\\{big\_app1\_insert}(\\{pp},\39\.{'\ '});{}$\6
${}\\{reduce}(\\{pp},\39\T{2},\39\\{typedef\_like},\39\T{0},\39\T{117});{}$\6
\4${}\}{}$\2\6
\&{else} \&{if} ${}(\\{cat1}\E\\{comma}){}$\5
${}\{{}$\1\6
\\{big\_app2}(\\{pp});\6
\\{big\_app}(\.{'\ '});\6
${}\\{reduce}(\\{pp},\39\T{2},\39\\{typedef\_like},\39\T{0},\39\T{118});{}$\6
\4${}\}{}$\2\6
\&{else} \&{if} ${}(\\{cat1}\E\\{semi}){}$\1\5
${}\\{squash}(\\{pp},\39\T{2},\39\\{decl},\39{-}\T{1},\39\T{119});{}$\2\6
\&{else} \&{if} ${}(\\{cat1}\E\\{ubinop}\W(\\{cat2}\E\\{ubinop}\V\\{cat2}\E%
\\{cast})){}$\5
${}\{{}$\1\6
\\{big\_app}(\.{'\{'});\6
${}\\{big\_app1\_insert}(\\{pp}+\T{1},\39\.{'\}'});{}$\6
${}\\{reduce}(\\{pp}+\T{1},\39\T{2},\39\\{cat2},\39\T{0},\39\T{120});{}$\6
\4${}\}{}$\2\par
\U121.\fi

\M{174}\B\X174:Cases for \PB{\\{delete\_like}}\X${}\E{}$\6
\&{if} ${}(\\{cat1}\E\\{lpar}\W\\{cat2}\E\\{rpar}){}$\5
${}\{{}$\1\6
\\{big\_app2}(\\{pp});\6
\\{app\_str}(\.{"\\\\,"});\6
${}\\{big\_app1}(\\{pp}+\T{2});{}$\6
${}\\{reduce}(\\{pp},\39\T{3},\39\\{delete\_like},\39\T{0},\39\T{121});{}$\6
\4${}\}{}$\2\6
\&{else} \&{if} ${}(\\{cat1}\E\\{exp}){}$\5
${}\{{}$\1\6
${}\\{big\_app1\_insert}(\\{pp},\39\.{'\ '});{}$\6
${}\\{reduce}(\\{pp},\39\T{2},\39\\{exp},\39{-}\T{2},\39\T{122});{}$\6
\4${}\}{}$\2\par
\U121.\fi

\M{175}\B\X175:Cases for \PB{\\{question}}\X${}\E{}$\6
\&{if} ${}(\\{cat1}\E\\{exp}\W(\\{cat2}\E\\{colon}\V\\{cat2}\E\\{base})){}$\5
${}\{{}$\1\6
${}(\\{pp}+\T{2})\MG\\{mathness}\K\T{5}*\\{yes\_math}{}$;\C{ this colon should
be in math mode }\6
${}\\{squash}(\\{pp},\39\T{3},\39\\{binop},\39{-}\T{2},\39\T{123});{}$\6
\4${}\}{}$\2\par
\U121.\fi

\M{176}\B\X176:Cases for \PB{\\{alignas\_like}}\X${}\E{}$\6
\&{if} ${}(\\{cat1}\E\\{decl\_head}){}$\1\5
${}\\{squash}(\\{pp},\39\T{2},\39\\{attr},\39{-}\T{1},\39\T{126});{}$\2\6
\&{else} \&{if} ${}(\\{cat1}\E\\{exp}){}$\1\5
${}\\{squash}(\\{pp},\39\T{2},\39\\{attr},\39{-}\T{1},\39\T{127});{}$\2\6
\&{else} \&{if} ${}(\\{cat1}\E\\{cast}){}$\1\5
${}\\{squash}(\\{pp},\39\T{2},\39\\{attr},\39{-}\T{1},\39\T{158}){}$;\2\par
\U121.\fi

\M{177}\B\X177:Cases for \PB{\\{lbrack}}\X${}\E{}$\6
\&{if} ${}(\\{cat1}\E\\{lbrack}){}$\1\6
\&{if} ${}(\\{cat2}\E\\{rbrack}\W\\{cat3}\E\\{rbrack}){}$\1\5
${}\\{squash}(\\{pp},\39\T{4},\39\\{exp},\39{-}\T{2},\39\T{147});{}$\2\6
\&{else}\1\5
${}\\{squash}(\\{pp},\39\T{2},\39\\{attr\_head},\39{-}\T{1},\39\T{128});{}$\2\2%
\6
\&{else}\1\5
${}\\{reduce}(\\{pp},\39\T{0},\39\\{lpar},\39{-}\T{1},\39\T{129}){}$;\2\par
\U121.\fi

\M{178}\B\X178:Cases for \PB{\\{attr\_head}}\X${}\E{}$\6
\&{if} ${}(\\{cat1}\E\\{rbrack}\W\\{cat2}\E\\{rbrack}){}$\1\5
${}\\{squash}(\\{pp},\39\T{3},\39\\{attr},\39{-}\T{1},\39\T{131});{}$\2\6
\&{else} \&{if} ${}(\\{cat1}\E\\{exp}){}$\1\5
${}\\{squash}(\\{pp},\39\T{2},\39\\{attr\_head},\39\T{0},\39\T{132});{}$\2\6
\&{else} \&{if} ${}(\\{cat1}\E\\{using\_like}\W\\{cat2}\E\\{exp}\W\\{cat3}\E%
\\{colon}){}$\5
${}\{{}$\1\6
\\{big\_app2}(\\{pp});\6
\\{big\_app}(\.{'\ '});\6
${}\\{big\_app2}(\\{pp}+\T{2});{}$\6
\\{big\_app}(\.{'\ '});\6
${}\\{reduce}(\\{pp},\39\T{4},\39\\{attr\_head},\39\T{0},\39\T{133});{}$\6
\4${}\}{}$\2\6
\&{else} \&{if} ${}(\\{cat1}\E\\{comma}){}$\1\5
${}\\{squash}(\\{pp},\39\T{2},\39\\{attr\_head},\39\T{0},\39\T{145}){}$;\2\par
\U121.\fi

\M{179}\B\X179:Cases for \PB{\\{attr}}\X${}\E{}$\6
\&{if} ${}(\\{cat1}\E\\{lbrace}\V\\{cat1}\E\\{stmt}){}$\5
${}\{{}$\1\6
${}\\{big\_app1\_insert}(\\{pp},\39\.{'\ '});{}$\6
${}\\{reduce}(\\{pp},\39\T{2},\39\\{cat1},\39{-}\T{2},\39\T{134});{}$\6
\4${}\}{}$\2\6
\&{else} \&{if} ${}(\\{cat1}\E\\{tag}){}$\5
${}\{{}$\1\6
${}\\{big\_app1\_insert}(\\{pp},\39\.{'\ '});{}$\6
${}\\{reduce}(\\{pp},\39\T{2},\39\\{tag},\39{-}\T{1},\39\T{135});{}$\6
\4${}\}{}$\2\6
\&{else} \&{if} ${}(\\{cat1}\E\\{semi}){}$\1\5
${}\\{squash}(\\{pp},\39\T{2},\39\\{stmt},\39{-}\T{2},\39\T{136});{}$\2\6
\&{else} \&{if} ${}(\\{cat1}\E\\{attr}){}$\5
${}\{{}$\1\6
${}\\{big\_app1\_insert}(\\{pp},\39\.{'\ '});{}$\6
${}\\{reduce}(\\{pp},\39\T{2},\39\\{attr},\39{-}\T{1},\39\T{137});{}$\6
\4${}\}{}$\2\6
\&{else} \&{if} ${}(\\{cat1}\E\\{decl\_head}){}$\5
${}\{{}$\1\6
${}\\{big\_app1\_insert}(\\{pp},\39\.{'\ '});{}$\6
${}\\{reduce}(\\{pp},\39\T{2},\39\\{decl\_head},\39{-}\T{1},\39\T{138});{}$\6
\4${}\}{}$\2\6
\&{else} \&{if} ${}(\\{cat1}\E\\{typedef\_like}){}$\5
${}\{{}$\1\6
${}\\{big\_app1\_insert}(\\{pp},\39\.{'\ '});{}$\6
${}\\{reduce}(\\{pp},\39\T{2},\39\\{typedef\_like},\39\T{0},\39\T{143});{}$\6
\4${}\}{}$\2\6
\&{else} \&{if} ${}(\\{cat1}\E\\{function}){}$\5
${}\{{}$\1\6
${}\\{big\_app1\_insert}(\\{pp},\39\.{'\ '});{}$\6
${}\\{reduce}(\\{pp},\39\T{2},\39\\{function},\39{-}\T{1},\39\T{148});{}$\6
\4${}\}{}$\2\par
\U121.\fi

\M{180}\B\X180:Cases for \PB{\\{default\_like}}\X${}\E{}$\6
\&{if} ${}(\\{cat1}\E\\{colon}){}$\1\5
${}\\{reduce}(\\{pp},\39\T{0},\39\\{case\_like},\39{-}\T{3},\39\T{149});{}$\2\6
\&{else}\1\5
${}\\{reduce}(\\{pp},\39\T{0},\39\\{exp},\39{-}\T{2},\39\T{150}){}$;\2\par
\U121.\fi

\M{181}The `\PB{\\{freeze\_text}}' macro is used to give official status to a
token list.
Before saying \PB{\\{freeze\_text}}, items are appended to the current token
list,
and we know that the eventual number of this token list will be the current
value of \PB{\\{text\_ptr}}. But no list of that number really exists as yet,
because no ending point for the current list has been
stored in the \PB{\\{tok\_start}} array. After saying \PB{\\{freeze\_text}},
the
old current token list becomes legitimate, and its number is the current
value of \PB{$\\{text\_ptr}-\T{1}$} since \PB{\\{text\_ptr}} has been
increased. The new
current token list is empty and ready to be appended~to.
Note that \PB{\\{freeze\_text}} does not check to see that \PB{\\{text\_ptr}}
hasn't gotten
too large, since it is assumed that this test was done beforehand.

\Y\B\4\D\\{freeze\_text}\5
${*}(\PP\\{text\_ptr})\K{}$\\{tok\_ptr}\par
\Y\B\4\X8:Predeclaration of procedures\X${}\mathrel+\E{}$\6
\&{static} \&{void} ${}\\{reduce}(\&{scrap\_pointer},\39\&{short},\39\&{eight%
\_bits},\39\&{short},\39\&{short}){}$;\6
\&{static} \&{void} ${}\\{squash}(\&{scrap\_pointer},\39\&{short},\39\&{eight%
\_bits},\39\&{short},\39\&{short}){}$;\par
\fi

\M{182}Now here's the \PB{\\{reduce}} procedure used in our code for
productions,
which takes advantage of the simplification that occurs when \PB{$\|k\E\T{0}$}.

\Y\B\1\1\&{static} \&{void} \\{reduce}(\&{scrap\_pointer} \|j${},\39{}$%
\&{short} \|k${},\39{}$\&{eight\_bits} \|c${},\39{}$\&{short} \|d${},\39{}$%
\&{short} \|n)\2\2\6
${}\{{}$\1\6
\&{scrap\_pointer} \|i${},{}$ \\{i1};\C{ pointers into scrap memory }\7
${}\|j\MG\\{cat}\K\|c;{}$\6
\&{if} ${}(\|k>\T{0}){}$\5
${}\{{}$\1\6
${}\|j\MG\\{trans}\K\\{text\_ptr};{}$\6
${}\|j\MG\\{mathness}\K\T{4}*\\{cur\_mathness}+\\{init\_mathness};{}$\6
\\{freeze\_text};\6
\4${}\}{}$\2\6
\&{if} ${}(\|k>\T{1}){}$\5
${}\{{}$\1\6
\&{for} ${}(\|i\K\|j+\|k,\39\\{i1}\K\|j+\T{1};{}$ ${}\|i\Z\\{lo\_ptr};{}$ ${}%
\|i\PP,\39\\{i1}\PP){}$\5
${}\{{}$\1\6
${}\\{i1}\MG\\{cat}\K\|i\MG\\{cat};{}$\6
${}\\{i1}\MG\\{trans}\K\|i\MG\\{trans};{}$\6
${}\\{i1}\MG\\{mathness}\K\|i\MG\\{mathness};{}$\6
\4${}\}{}$\2\6
${}\\{lo\_ptr}\K\\{lo\_ptr}-\|k+\T{1};{}$\6
\4${}\}{}$\2\6
${}\\{pp}\K(\\{pp}+\|d<\\{scrap\_base}\?\\{scrap\_base}:\\{pp}+\|d);{}$\6
\X187:Print a snapshot of the scrap list if debugging\X\6
${}\\{pp}\MM{}$;\C{ we next say \PB{$\\{pp}\PP$} }\6
\4${}\}{}$\2\par
\fi

\M{183}And here's the \PB{\\{squash}} procedure, which combines \PB{\\{big%
\_app}}${}_k$ and
\PB{\\{reduce}} for matching numbers~\PB{\|k}.

\Y\B\1\1\&{static} \&{void} \\{squash}(\&{scrap\_pointer} \|j${},\39{}$%
\&{short} \|k${},\39{}$\&{eight\_bits} \|c${},\39{}$\&{short} \|d${},\39{}$%
\&{short} \|n)\2\2\6
${}\{{}$\1\6
\&{switch} (\|k)\5
${}\{{}$\1\6
\4\&{case} \T{1}:\5
\\{big\_app1}(\|j);\6
\&{break};\6
\4\&{case} \T{2}:\5
\\{big\_app2}(\|j);\6
\&{break};\6
\4\&{case} \T{3}:\5
\\{big\_app3}(\|j);\6
\&{break};\6
\4\&{case} \T{4}:\5
\\{big\_app4}(\|j);\6
\&{break};\6
\4\&{default}:\5
\\{confusion}(\.{"squash"});\6
\4${}\}{}$\2\6
${}\\{reduce}(\|j,\39\|k,\39\|c,\39\|d,\39\|n);{}$\6
\4${}\}{}$\2\par
\fi

\M{184}And here now is the code that applies productions as long as possible.
Before applying the production mechanism, we must make sure
it has good input (at least four scraps, the length of the lhs of the
longest rules), and that there is enough room in the memory arrays
to hold the appended tokens and texts.  Here we use a very
conservative test; it's more important to make sure the program
will still work if we change the production rules (within reason)
than to squeeze the last bit of space from the memory arrays.

\Y\B\4\D\\{safe\_tok\_incr}\5
\T{20}\par
\B\4\D\\{safe\_text\_incr}\5
\T{10}\par
\B\4\D\\{safe\_scrap\_incr}\5
\T{10}\par
\Y\B\4\X184:Reduce the scraps using the productions until no more rules apply%
\X${}\E{}$\6
\&{while} (\\{true})\5
${}\{{}$\1\6
\X185:Make sure the entries \PB{\\{pp}} through \PB{$\\{pp}+\T{3}$} of \PB{%
\\{cat}} are defined\X\6
\&{if} ${}(\\{tok\_ptr}+\\{safe\_tok\_incr}>\\{tok\_mem\_end}){}$\5
${}\{{}$\1\6
\&{if} ${}(\\{tok\_ptr}>\\{max\_tok\_ptr}){}$\1\5
${}\\{max\_tok\_ptr}\K\\{tok\_ptr};{}$\2\6
\\{overflow}(\.{"token"});\6
\4${}\}{}$\2\6
\&{if} ${}(\\{text\_ptr}+\\{safe\_text\_incr}>\\{tok\_start\_end}){}$\5
${}\{{}$\1\6
\&{if} ${}(\\{text\_ptr}>\\{max\_text\_ptr}){}$\1\5
${}\\{max\_text\_ptr}\K\\{text\_ptr};{}$\2\6
\\{overflow}(\.{"text"});\6
\4${}\}{}$\2\6
\&{if} ${}(\\{pp}>\\{lo\_ptr}){}$\1\5
\&{break};\2\6
${}\\{init\_mathness}\K\\{cur\_mathness}\K\\{maybe\_math};{}$\6
\X121:Match a production at \PB{\\{pp}}, or increase \PB{\\{pp}} if there is no
match\X\6
\4${}\}{}$\2\par
\U188.\fi

\M{185}If we get to the end of the scrap list, category codes equal to zero are
stored, since zero does not match anything in a production.

\Y\B\4\X185:Make sure the entries \PB{\\{pp}} through \PB{$\\{pp}+\T{3}$} of %
\PB{\\{cat}} are defined\X${}\E{}$\6
\&{if} ${}(\\{lo\_ptr}<\\{pp}+\T{3}){}$\5
${}\{{}$\1\6
\&{while} ${}(\\{hi\_ptr}\Z\\{scrap\_ptr}\W\\{lo\_ptr}\I\\{pp}+\T{3}){}$\5
${}\{{}$\1\6
${}(\PP\\{lo\_ptr})\MG\\{cat}\K\\{hi\_ptr}\MG\\{cat};{}$\6
${}\\{lo\_ptr}\MG\\{mathness}\K\\{hi\_ptr}\MG\\{mathness};{}$\6
${}\\{lo\_ptr}\MG\\{trans}\K(\\{hi\_ptr}\PP)\MG\\{trans};{}$\6
\4${}\}{}$\2\6
\&{for} ${}(\|i\K\\{lo\_ptr}+\T{1};{}$ ${}\|i\Z\\{pp}+\T{3};{}$ ${}\|i\PP){}$\1%
\5
${}\|i\MG\\{cat}\K\T{0};{}$\2\6
\4${}\}{}$\2\par
\U184.\fi

\M{186}If \.{CWEAVE} is being run in debugging mode, the production numbers and
current stack categories will be printed out when \PB{\\{tracing}} is set to %
\PB{\\{fully}};
a sequence of two or more irreducible scraps will be printed out when
\PB{\\{tracing}} is set to \PB{\\{partly}}.

\Y\B\4\D\\{off}\5
\T{0}\par
\B\4\D\\{partly}\5
\T{1}\par
\B\4\D\\{fully}\5
\T{2}\par
\Y\B\4\X21:Private variables\X${}\mathrel+\E{}$\6
\&{static} \&{int} \\{tracing}${}\K\\{off}{}$;\C{ can be used to show parsing
details }\par
\fi

\M{187}\B\X187:Print a snapshot of the scrap list if debugging\X${}\E{}$\6
\&{if} ${}(\\{tracing}\E\\{fully}){}$\5
${}\{{}$\1\6
\&{scrap\_pointer} \|k;\C{ pointer into \PB{\\{scrap\_info}}; shadows \PB{%
\&{short} \|k} }\7
${}\\{printf}(\.{"\\n\%d:"},\39\|n);{}$\6
\&{for} ${}(\|k\K\\{scrap\_base};{}$ ${}\|k\Z\\{lo\_ptr};{}$ ${}\|k\PP){}$\5
${}\{{}$\1\6
\&{if} ${}(\|k\E\\{pp}){}$\1\5
\\{putchar}(\.{'*'});\2\6
\&{else}\1\5
\\{putchar}(\.{'\ '});\2\6
\&{if} ${}(\|k\MG\\{mathness}\MOD\T{4}\E\\{yes\_math}){}$\1\5
\\{putchar}(\.{'+'});\2\6
\&{else} \&{if} ${}(\|k\MG\\{mathness}\MOD\T{4}\E\\{no\_math}){}$\1\5
\\{putchar}(\.{'-'});\2\6
${}\\{print\_cat}(\|k\MG\\{cat});{}$\6
\&{if} ${}(\|k\MG\\{mathness}/\T{4}\E\\{yes\_math}){}$\1\5
\\{putchar}(\.{'+'});\2\6
\&{else} \&{if} ${}(\|k\MG\\{mathness}/\T{4}\E\\{no\_math}){}$\1\5
\\{putchar}(\.{'-'});\2\6
\4${}\}{}$\2\6
\&{if} ${}(\\{hi\_ptr}\Z\\{scrap\_ptr}){}$\1\5
\\{printf}(\.{"..."});\C{ indicate that more is coming }\2\6
\4${}\}{}$\2\par
\U182.\fi

\M{188}The \PB{\\{translate}} function assumes that scraps have been stored in
positions \PB{\\{scrap\_base}} through \PB{\\{scrap\_ptr}} of \PB{\\{cat}} and %
\PB{\\{trans}}. It
applies productions as much as
possible. The result is a token list containing the translation of
the given sequence of scraps.

After calling \PB{\\{translate}}, we will have \PB{$\\{text\_ptr}+\T{3}\Z\\{max%
\_texts}$} and
\PB{$\\{tok\_ptr}+\T{6}\Z\\{max\_toks}$}, so it will be possible to create up
to three token
lists with up to six tokens without checking for overflow. Before calling
\PB{\\{translate}}, we should have \PB{$\\{text\_ptr}<\\{max\_texts}$} and %
\PB{$\\{scrap\_ptr}<\\{max\_scraps}$},
since \PB{\\{translate}} might add a new text and a new scrap before it checks
for overflow.

\Y\B\1\1\&{static} \&{text\_pointer} \\{translate}(\&{void})\C{ converts a
sequence of scraps }\2\2\6
${}\{{}$\1\6
\&{scrap\_pointer} \|i;\C{ index into \PB{\\{cat}} }\6
\&{scrap\_pointer} \|j;\C{ runs through final scraps }\7
${}\\{pp}\K\\{scrap\_base};{}$\6
${}\\{lo\_ptr}\K\\{pp}-\T{1};{}$\6
${}\\{hi\_ptr}\K\\{pp};{}$\6
\X192:If tracing, print an indication of where we are\X\6
\X184:Reduce the scraps using the productions until no more rules apply\X\6
\X190:Combine the irreducible scraps that remain\X\6
\4${}\}{}$\2\par
\fi

\M{189}\B\X8:Predeclaration of procedures\X${}\mathrel+\E{}$\5
\&{static} \&{text\_pointer} \\{translate}(\&{void});\par
\fi

\M{190}If the initial sequence of scraps does not reduce to a single scrap,
we concatenate the translations of all remaining scraps, separated by
blank spaces, with dollar signs surrounding the translations of scraps
where appropriate.

\Y\B\4\X190:Combine the irreducible scraps that remain\X${}\E{}$\6
\X191:If semi-tracing, show the irreducible scraps\X\6
\&{for} ${}(\|j\K\\{scrap\_base};{}$ ${}\|j\Z\\{lo\_ptr};{}$ ${}\|j\PP){}$\5
${}\{{}$\1\6
\&{if} ${}(\|j\I\\{scrap\_base}){}$\1\5
\\{app}(\.{'\ '});\2\6
\&{if} ${}(\|j\MG\\{mathness}\MOD\T{4}\E\\{yes\_math}){}$\1\5
\\{app}(\.{'\$'});\2\6
${}\\{app}(\\{tok\_flag}+(\&{int})(\|j\MG\\{trans}-\\{tok\_start}));{}$\6
\&{if} ${}(\|j\MG\\{mathness}/\T{4}\E\\{yes\_math}){}$\1\5
\\{app}(\.{'\$'});\2\6
\&{if} ${}(\\{tok\_ptr}+\T{6}>\\{tok\_mem\_end}){}$\1\5
\\{overflow}(\.{"token"});\2\6
\4${}\}{}$\2\6
\\{freeze\_text};\6
\&{return} \\{text\_ptr}${}-\T{1}{}$;\par
\U188.\fi

\M{191}\B\X191:If semi-tracing, show the irreducible scraps\X${}\E{}$\6
\&{if} ${}(\\{lo\_ptr}>\\{scrap\_base}\W\\{tracing}\E\\{partly}){}$\5
${}\{{}$\1\6
${}\\{printf}(\.{"\\nIrreducible\ scrap}\)\.{\ sequence\ in\ section}\)\.{\ %
\%d:"},\39{}$(\&{int}) \\{section\_count});\6
\\{mark\_harmless};\6
\&{for} ${}(\|j\K\\{scrap\_base};{}$ ${}\|j\Z\\{lo\_ptr};{}$ ${}\|j\PP){}$\5
${}\{{}$\1\6
\\{putchar}(\.{'\ '});\6
${}\\{print\_cat}(\|j\MG\\{cat});{}$\6
\4${}\}{}$\2\6
\4${}\}{}$\2\par
\U190.\fi

\M{192}\B\X192:If tracing, print an indication of where we are\X${}\E{}$\6
\&{if} ${}(\\{tracing}\E\\{fully}){}$\5
${}\{{}$\1\6
${}\\{printf}(\.{"\\nTracing\ after\ l.\ }\)\.{\%d:\\n"},\39\\{cur\_line});{}$\6
\\{mark\_harmless};\6
\&{if} ${}(\\{loc}>\\{buffer}+\T{50}){}$\5
${}\{{}$\1\6
\\{printf}(\.{"..."});\6
${}\\{term\_write}(\\{loc}-\T{51},\39\T{51});{}$\6
\4${}\}{}$\2\6
\&{else}\1\5
${}\\{term\_write}(\\{buffer},\39\\{loc}-\\{buffer});{}$\2\6
\4${}\}{}$\2\par
\U188.\fi

\N{1}{193}Initializing the scraps.
If we are going to use the powerful production mechanism just developed, we
must get the scraps set up in the first place, given a \CEE/ text. A table
of the initial scraps corresponding to \CEE/ tokens appeared above in the
section on parsing; our goal now is to implement that table. We shall do this
by implementing a subroutine called \PB{\\{C\_parse}} that is analogous to the
\PB{\\{C\_xref}} routine used during phase one.

Like \PB{\\{C\_xref}}, the \PB{\\{C\_parse}} procedure starts with the current
value of \PB{\\{next\_control}} and it uses the operation \PB{$\\{next%
\_control}\K\\{get\_next}(\,)$}
repeatedly to read \CEE/ text until encountering the next `\.{\v}' or
`\.{/*}', or until \PB{$\\{next\_control}\G\\{format\_code}$}. The scraps
corresponding to
what it reads are appended into the \PB{\\{cat}} and \PB{\\{trans}} arrays, and
\PB{\\{scrap\_ptr}}
is advanced.

\Y\B\1\1\&{static} \&{void} \\{C\_parse}(\C{ creates scraps from \CEE/ tokens }%
\6
\&{eight\_bits} \\{spec\_ctrl})\2\2\6
${}\{{}$\1\6
\&{while} ${}(\\{next\_control}<\\{format\_code}\V\\{next\_control}\E\\{spec%
\_ctrl}){}$\5
${}\{{}$\1\6
\X196:Append the scrap appropriate to \PB{\\{next\_control}}\X\6
${}\\{next\_control}\K\\{get\_next}(\,);{}$\6
\&{if} ${}(\\{next\_control}\E\.{'|'}\V\\{next\_control}\E\\{begin\_comment}\V%
\\{next\_control}\E\\{begin\_short\_comment}){}$\1\5
\&{return};\2\6
\4${}\}{}$\2\6
\4${}\}{}$\2\par
\fi

\M{194}\B\X8:Predeclaration of procedures\X${}\mathrel+\E{}$\5
\&{static} \&{void} \\{C\_parse}(\&{eight\_bits});\par
\fi

\M{195}The following macro is used to append a scrap whose tokens have just
been appended:

\Y\B\4\D\\{app\_scrap}$(\|c,\|b)$\6
${}\{{}$\1\6
${}(\PP\\{scrap\_ptr})\MG\\{cat}\K(\|c);{}$\6
${}\\{scrap\_ptr}\MG\\{trans}\K\\{text\_ptr};{}$\6
${}\\{scrap\_ptr}\MG\\{mathness}\K\T{5}*(\|b){}$;\C{ no no, yes yes, or maybe
maybe }\6
\\{freeze\_text};\6
\4${}\}{}$\2\par
\fi

\M{196}\B\X196:Append the scrap appropriate to \PB{\\{next\_control}}\X${}\E{}$%
\6
\X197:Make sure that there is room for the new scraps, tokens, and texts\X\6
\&{switch} (\\{next\_control})\5
${}\{{}$\1\6
\4\&{case} \\{section\_name}:\5
${}\\{app}(\\{section\_flag}+(\&{int})(\\{cur\_section}-\\{name\_dir}));{}$\6
${}\\{app\_scrap}(\\{section\_scrap},\39\\{maybe\_math}){}$;\5
${}\\{app\_scrap}(\\{exp},\39\\{yes\_math}){}$;\5
\&{break};\6
\4\&{case} \\{string}:\5
\&{case} \\{constant}:\5
\&{case} \\{verbatim}:\5
\X199:Append a string or constant\X\5
\&{break};\6
\4\&{case} \\{identifier}:\5
\\{app\_cur\_id}(\\{true});\5
\&{break};\6
\4\&{case} ${}\TeXxstring{}$:\5
\X200:Append a \TEX/ string, without forming a scrap\X\5
\&{break};\6
\4\&{case} \.{'/'}:\5
\&{case} \.{'.'}:\5
\\{app}(\\{next\_control});\5
${}\\{app\_scrap}(\\{binop},\39\\{yes\_math}){}$;\5
\&{break};\6
\4\&{case} \.{'<'}:\5
\\{app\_str}(\.{"\\\\langle"});\5
${}\\{app\_scrap}(\\{prelangle},\39\\{yes\_math}){}$;\5
\&{break};\6
\4\&{case} \.{'>'}:\5
\\{app\_str}(\.{"\\\\rangle"});\5
${}\\{app\_scrap}(\\{prerangle},\39\\{yes\_math}){}$;\5
\&{break};\6
\4\&{case} \.{'='}:\5
\\{app\_str}(\.{"\\\\K"});\5
${}\\{app\_scrap}(\\{binop},\39\\{yes\_math}){}$;\5
\&{break};\6
\4\&{case} \.{'|'}:\5
\\{app\_str}(\.{"\\\\OR"});\5
${}\\{app\_scrap}(\\{binop},\39\\{yes\_math}){}$;\5
\&{break};\6
\4\&{case} \.{'\^'}:\5
\\{app\_str}(\.{"\\\\XOR"});\5
${}\\{app\_scrap}(\\{binop},\39\\{yes\_math}){}$;\5
\&{break};\6
\4\&{case} \.{'\%'}:\5
\\{app\_str}(\.{"\\\\MOD"});\5
${}\\{app\_scrap}(\\{binop},\39\\{yes\_math}){}$;\5
\&{break};\6
\4\&{case} \.{'!'}:\5
\\{app\_str}(\.{"\\\\R"});\5
${}\\{app\_scrap}(\\{unop},\39\\{yes\_math}){}$;\5
\&{break};\6
\4\&{case} \.{'\~'}:\5
\\{app\_str}(\.{"\\\\CM"});\5
${}\\{app\_scrap}(\\{unop},\39\\{yes\_math}){}$;\5
\&{break};\6
\4\&{case} \.{'+'}:\5
\&{case} \.{'-'}:\5
\\{app}(\\{next\_control});\5
${}\\{app\_scrap}(\\{ubinop},\39\\{yes\_math}){}$;\5
\&{break};\6
\4\&{case} \.{'*'}:\5
\\{app}(\\{next\_control});\5
${}\\{app\_scrap}(\\{raw\_ubin},\39\\{yes\_math}){}$;\5
\&{break};\6
\4\&{case} \.{'\&'}:\5
\\{app\_str}(\.{"\\\\AND"});\5
${}\\{app\_scrap}(\\{raw\_ubin},\39\\{yes\_math}){}$;\5
\&{break};\6
\4\&{case} \.{'?'}:\5
\\{app\_str}(\.{"\\\\?"});\5
${}\\{app\_scrap}(\\{question},\39\\{yes\_math}){}$;\5
\&{break};\6
\4\&{case} \.{'\#'}:\5
\\{app\_str}(\.{"\\\\\#"});\5
${}\\{app\_scrap}(\\{ubinop},\39\\{yes\_math}){}$;\5
\&{break};\6
\4\&{case} \\{ignore}:\5
\&{case} \\{xref\_roman}:\5
\&{case} \\{xref\_wildcard}:\5
\&{case} \\{xref\_typewriter}:\5
\&{case} \\{noop}:\5
\&{break};\6
\4\&{case} \.{'('}:\5
\\{app}(\\{next\_control});\5
${}\\{app\_scrap}(\\{lpar},\39\\{maybe\_math}){}$;\5
\&{break};\6
\4\&{case} \.{')'}:\5
\\{app}(\\{next\_control});\5
${}\\{app\_scrap}(\\{rpar},\39\\{maybe\_math}){}$;\5
\&{break};\6
\4\&{case} \.{'['}:\5
\\{app}(\\{next\_control});\5
${}\\{app\_scrap}(\\{lbrack},\39\\{maybe\_math}){}$;\5
\&{break};\6
\4\&{case} \.{']'}:\5
\\{app}(\\{next\_control});\5
${}\\{app\_scrap}(\\{rbrack},\39\\{maybe\_math}){}$;\5
\&{break};\6
\4\&{case} \.{'\{'}:\5
\\{app\_str}(\.{"\\\\\{"});\5
${}\\{app\_scrap}(\\{lbrace},\39\\{yes\_math}){}$;\5
\&{break};\6
\4\&{case} \.{'\}'}:\5
\\{app\_str}(\.{"\\\\\}"});\5
${}\\{app\_scrap}(\\{rbrace},\39\\{yes\_math}){}$;\5
\&{break};\6
\4\&{case} \.{','}:\5
\\{app}(\.{','});\5
${}\\{app\_scrap}(\\{comma},\39\\{yes\_math}){}$;\5
\&{break};\6
\4\&{case} \.{';'}:\5
\\{app}(\.{';'});\5
${}\\{app\_scrap}(\\{semi},\39\\{maybe\_math}){}$;\5
\&{break};\6
\4\&{case} \.{':'}:\5
\\{app}(\.{':'});\5
${}\\{app\_scrap}(\\{colon},\39\\{no\_math}){}$;\5
\&{break};\6
\hbox{\4}\X198:Cases involving nonstandard characters\X\6
\4\&{case} \\{thin\_space}:\5
\\{app\_str}(\.{"\\\\,"});\5
${}\\{app\_scrap}(\\{insert},\39\\{maybe\_math}){}$;\5
\&{break};\6
\4\&{case} \\{math\_break}:\5
\\{app}(\\{opt});\5
\\{app}(\.{'0'});\5
${}\\{app\_scrap}(\\{insert},\39\\{maybe\_math}){}$;\5
\&{break};\6
\4\&{case} \\{line\_break}:\5
\\{app}(\\{force});\5
${}\\{app\_scrap}(\\{insert},\39\\{no\_math}){}$;\5
\&{break};\6
\4\&{case} \\{left\_preproc}:\5
\\{app}(\\{force});\5
\\{app}(\\{preproc\_line});\5
\\{app\_str}(\.{"\\\\\#"});\6
${}\\{app\_scrap}(\\{lproc},\39\\{no\_math}){}$;\5
\&{break};\6
\4\&{case} \\{right\_preproc}:\5
\\{app}(\\{force});\5
${}\\{app\_scrap}(\\{rproc},\39\\{no\_math}){}$;\5
\&{break};\6
\4\&{case} \\{big\_line\_break}:\5
\\{app}(\\{big\_force});\5
${}\\{app\_scrap}(\\{insert},\39\\{no\_math}){}$;\5
\&{break};\6
\4\&{case} \\{no\_line\_break}:\5
\\{app}(\\{big\_cancel});\5
\\{app}(\\{noop});\5
\\{app}(\\{break\_space});\5
\\{app}(\\{noop});\5
\\{app}(\\{big\_cancel});\6
${}\\{app\_scrap}(\\{insert},\39\\{no\_math}){}$;\5
\&{break};\6
\4\&{case} \\{pseudo\_semi}:\5
${}\\{app\_scrap}(\\{semi},\39\\{maybe\_math}){}$;\5
\&{break};\6
\4\&{case} \\{macro\_arg\_open}:\5
${}\\{app\_scrap}(\\{begin\_arg},\39\\{maybe\_math}){}$;\5
\&{break};\6
\4\&{case} \\{macro\_arg\_close}:\5
${}\\{app\_scrap}(\\{end\_arg},\39\\{maybe\_math}){}$;\5
\&{break};\6
\4\&{case} \\{join}:\5
\\{app\_str}(\.{"\\\\J"});\5
${}\\{app\_scrap}(\\{insert},\39\\{no\_math}){}$;\5
\&{break};\6
\4\&{case} \\{output\_defs\_code}:\5
\\{app}(\\{force});\5
\\{app\_str}(\.{"\\\\ATH"});\5
\\{app}(\\{force});\6
${}\\{app\_scrap}(\\{insert},\39\\{no\_math}){}$;\5
\&{break};\6
\4\&{default}:\5
\\{app}(\\{inserted});\5
\\{app}(\\{next\_control});\6
${}\\{app\_scrap}(\\{insert},\39\\{maybe\_math}){}$;\5
\&{break};\6
\4${}\}{}$\2\par
\U193.\fi

\M{197}\B\X197:Make sure that there is room for the new scraps, tokens, and
texts\X${}\E{}$\6
\&{if} ${}(\\{scrap\_ptr}+\\{safe\_scrap\_incr}>\\{scrap\_info\_end}\V\\{tok%
\_ptr}+\\{safe\_tok\_incr}>\\{tok\_mem\_end}\3{-1}\V\\{text\_ptr}+\\{safe\_text%
\_incr}>\\{tok\_start\_end}){}$\5
${}\{{}$\1\6
\&{if} ${}(\\{scrap\_ptr}>\\{max\_scr\_ptr}){}$\1\5
${}\\{max\_scr\_ptr}\K\\{scrap\_ptr};{}$\2\6
\&{if} ${}(\\{tok\_ptr}>\\{max\_tok\_ptr}){}$\1\5
${}\\{max\_tok\_ptr}\K\\{tok\_ptr};{}$\2\6
\&{if} ${}(\\{text\_ptr}>\\{max\_text\_ptr}){}$\1\5
${}\\{max\_text\_ptr}\K\\{text\_ptr};{}$\2\6
\\{overflow}(\.{"scrap/token/text"});\6
\4${}\}{}$\2\par
\Us196\ET205.\fi

\M{198}Some nonstandard characters may have entered \.{CWEAVE} by means of
standard ones. They are converted to \TEX/ control sequences so that it is
possible to keep \.{CWEAVE} from outputting unusual \PB{\&{char}} codes.

\Y\B\4\X198:Cases involving nonstandard characters\X${}\E{}$\6
\4\hbox{\1\quad}\&{case} \\{non\_eq}:\5
\\{app\_str}(\.{"\\\\I"});\5
${}\\{app\_scrap}(\\{binop},\39\\{yes\_math}){}$;\5
\&{break};\6
\4\&{case} \\{lt\_eq}:\5
\\{app\_str}(\.{"\\\\Z"});\5
${}\\{app\_scrap}(\\{binop},\39\\{yes\_math}){}$;\5
\&{break};\6
\4\&{case} \\{gt\_eq}:\5
\\{app\_str}(\.{"\\\\G"});\5
${}\\{app\_scrap}(\\{binop},\39\\{yes\_math}){}$;\5
\&{break};\6
\4\&{case} \\{eq\_eq}:\5
\\{app\_str}(\.{"\\\\E"});\5
${}\\{app\_scrap}(\\{binop},\39\\{yes\_math}){}$;\5
\&{break};\6
\4\&{case} \\{and\_and}:\5
\\{app\_str}(\.{"\\\\W"});\5
${}\\{app\_scrap}(\\{binop},\39\\{yes\_math}){}$;\5
\&{break};\6
\4\&{case} \\{or\_or}:\5
\\{app\_str}(\.{"\\\\V"});\5
${}\\{app\_scrap}(\\{binop},\39\\{yes\_math}){}$;\5
\&{break};\6
\4\&{case} \\{plus\_plus}:\5
\\{app\_str}(\.{"\\\\PP"});\5
${}\\{app\_scrap}(\\{unop},\39\\{yes\_math}){}$;\5
\&{break};\6
\4\&{case} \\{minus\_minus}:\5
\\{app\_str}(\.{"\\\\MM"});\5
${}\\{app\_scrap}(\\{unop},\39\\{yes\_math}){}$;\5
\&{break};\6
\4\&{case} \\{minus\_gt}:\5
\\{app\_str}(\.{"\\\\MG"});\5
${}\\{app\_scrap}(\\{binop},\39\\{yes\_math}){}$;\5
\&{break};\6
\4\&{case} \\{gt\_gt}:\5
\\{app\_str}(\.{"\\\\GG"});\5
${}\\{app\_scrap}(\\{binop},\39\\{yes\_math}){}$;\5
\&{break};\6
\4\&{case} \\{lt\_lt}:\5
\\{app\_str}(\.{"\\\\LL"});\5
${}\\{app\_scrap}(\\{binop},\39\\{yes\_math}){}$;\5
\&{break};\6
\4\&{case} \\{dot\_dot\_dot}:\5
\\{app\_str}(\.{"\\\\,\\\\ldots\\\\,"});\5
${}\\{app\_scrap}(\\{raw\_int},\39\\{yes\_math}){}$;\5
\&{break};\6
\4\&{case} \\{colon\_colon}:\5
\\{app\_str}(\.{"\\\\DC"});\5
${}\\{app\_scrap}(\\{colcol},\39\\{maybe\_math}){}$;\5
\&{break};\6
\4\&{case} \\{period\_ast}:\5
\\{app\_str}(\.{"\\\\PA"});\5
${}\\{app\_scrap}(\\{binop},\39\\{yes\_math}){}$;\5
\&{break};\6
\4\&{case} \\{minus\_gt\_ast}:\5
\\{app\_str}(\.{"\\\\MGA"});\5
${}\\{app\_scrap}(\\{binop},\39\\{yes\_math}){}$;\5
\&{break};\par
\U196.\fi

\M{199}The following code must use \PB{\\{app\_tok}} instead of \PB{\\{app}} in
order to
protect against overflow. Note that \PB{$\\{tok\_ptr}+\T{1}\Z\\{max\_toks}$}
after \PB{\\{app\_tok}}
has been used, so another \PB{\\{app}} is legitimate before testing again.

Many of the special characters in a string must be prefixed by `\.\\' so that
\TEX/ will print them properly.

\Y\B\4\X199:Append a string or constant\X${}\E{}$\6
${}\{{}$\5
\1\&{int} \\{count}${}\K{-}\T{1}{}$;\C{ characters remaining before string
break }\7
\&{switch} (\\{next\_control})\5
${}\{{}$\1\6
\4\&{case} \\{constant}:\5
\\{app\_str}(\.{"\\\\T\{"});\6
\&{break};\6
\4\&{case} \\{string}:\5
${}\\{count}\K\T{20};{}$\6
\\{app\_str}(\.{"\\\\.\{"});\6
\&{break};\6
\4\&{default}:\5
\\{app\_str}(\.{"\\\\vb\{"});\6
\4${}\}{}$\2\6
\&{while} ${}(\\{id\_first}<\\{id\_loc}){}$\5
${}\{{}$\1\6
\&{if} ${}(\\{count}\E\T{0}){}$\5
${}\{{}$\C{ insert a discretionary break in a long string }\1\6
\\{app\_str}(\.{"\}\\\\)\\\\.\{"});\6
${}\\{count}\K\T{20};{}$\6
\4${}\}{}$\2\6
\&{switch} ${}({*}\\{id\_first}){}$\5
${}\{{}$\1\6
\4\&{case} \.{'\ '}:\5
\&{case} \.{'\\\\'}:\5
\&{case} \.{'\#'}:\5
\&{case} \.{'\$'}:\5
\&{case} \.{'\^'}:\5
\&{case} \.{'\{'}:\5
\&{case} \.{'\}'}:\5
\&{case} \.{'\~'}:\5
\&{case} \.{'\&'}:\5
\&{case} \.{'\_'}:\5
\\{app}(\.{'\\\\'});\6
\&{break};\6
\4\&{case} \.{'\%'}:\6
\&{if} ${}(\\{next\_control}\E\\{constant}){}$\5
${}\{{}$\1\6
\\{app\_str}(\.{"\}\\\\p\{"});\C{ special macro for `hex exponent' }\6
${}\\{id\_first}\PP{}$;\C{ skip \PB{\.{'\%'}} }\6
\4${}\}{}$\2\6
\&{else}\1\5
\\{app}(\.{'\\\\'});\2\6
\&{break};\6
\4\&{case} \.{'@'}:\6
\&{if} ${}({*}(\\{id\_first}+\T{1})\E\.{'@'}){}$\1\5
${}\\{id\_first}\PP;{}$\2\6
\&{else}\1\5
\\{err\_print}(\.{"!\ Double\ @\ should\ b}\)\.{e\ used\ in\ strings"});\2\6
\&{break};\6
\4\&{default}:\C{ high-bit character handling }\6
\&{if} ${}((\&{eight\_bits})({*}\\{id\_first})>\T{\~177}){}$\1\5
\\{app\_tok}(\\{quoted\_char})\hbox{;}\2\6
\4${}\}{}$\2\6
${}\\{app\_tok}({*}\\{id\_first}\PP);{}$\6
${}\\{count}\MM;{}$\6
\4${}\}{}$\2\6
\\{app}(\.{'\}'});\6
${}\\{app\_scrap}(\\{exp},\39\\{maybe\_math});{}$\6
\4${}\}{}$\2\par
\U196.\fi

\M{200}We do not make the \TEX/ string into a scrap, because there is no
telling what the user will be putting into it; instead we leave it
open, to be picked up by the next scrap. If it comes at the end of a
section, it will be made into a scrap when \PB{\\{finish\_C}} is called.

There's a known bug here, in cases where an adjacent scrap is
\PB{\\{prelangle}} or \PB{\\{prerangle}}. Then the \TEX/ string can disappear
when the \.{\\langle} or \.{\\rangle} becomes \.{<} or \.{>}.
For example, if the user writes \.{\v x<@ty@>\v}, the \TEX/ string
\.{\\hbox\{y\}} eventually becomes part of an \PB{\\{insert}} scrap, which is
combined
with a \PB{\\{prelangle}} scrap and eventually lost. The best way to work
around
this bug is probably to enclose the \.{@t...@>} in \.{@[...@]} so that
the \TEX/ string is treated as an expression.

\Y\B\4\X200:Append a \TEX/ string, without forming a scrap\X${}\E{}$\6
\\{app\_str}(\.{"\\\\hbox\{"});\6
\&{while} ${}(\\{id\_first}<\\{id\_loc}){}$\5
${}\{{}$\1\6
\&{if} ${}((\&{eight\_bits})({*}\\{id\_first})>\T{\~177}){}$\1\5
\\{app\_tok}(\\{quoted\_char})\hbox{;}\2\6
\&{else} \&{if} ${}({*}\\{id\_first}\E\.{'@'}){}$\1\5
${}\\{id\_first}\PP;{}$\2\6
${}\\{app\_tok}({*}\\{id\_first}\PP);{}$\6
\4${}\}{}$\2\6
\\{app}(\.{'\}'});\par
\U196.\fi

\M{201}The function \PB{\\{app\_cur\_id}} appends the current identifier to the
token list; it also builds a new scrap if \PB{$\\{scrapping}\E\\{true}$}.

\Y\B\4\X8:Predeclaration of procedures\X${}\mathrel+\E{}$\6
\&{static} \&{void} \\{app\_cur\_id}(\&{boolean});\6
\&{static} \&{text\_pointer} \\{C\_translate}(\&{void});\6
\&{static} \&{void} \\{outer\_parse}(\&{void});\par
\fi

\M{202}\B\1\1\&{static} \&{void} \\{app\_cur\_id}(\&{boolean} \\{scrapping})\C{
are we making this into a scrap? }\2\2\6
${}\{{}$\1\6
\&{name\_pointer} \|p${}\K\\{id\_lookup}(\\{id\_first},\39\\{id\_loc},\39%
\\{normal});{}$\7
\&{if} ${}(\|p\MG\\{ilk}\Z\\{custom}){}$\5
${}\{{}$\C{ not a reserved word }\1\6
${}\\{app}(\\{id\_flag}+(\&{int})(\|p-\\{name\_dir}));{}$\6
\&{if} (\\{scrapping})\1\5
${}\\{app\_scrap}(\|p\MG\\{ilk}\E\\{func\_template}\?\\{ftemplate}:\\{exp},\39%
\|p\MG\\{ilk}\E\\{custom}\?\\{yes\_math}:\\{maybe\_math});{}$\2\6
\4${}\}{}$\2\6
\&{else}\5
${}\{{}$\1\6
${}\\{app}(\\{res\_flag}+(\&{int})(\|p-\\{name\_dir}));{}$\6
\&{if} (\\{scrapping})\5
${}\{{}$\1\6
\&{if} ${}(\|p\MG\\{ilk}\E\\{alfop}){}$\1\5
${}\\{app\_scrap}(\\{ubinop},\39\\{yes\_math})\hbox{;}{}$\2\6
\&{else}\1\5
${}\\{app\_scrap}(\|p\MG\\{ilk},\39\\{maybe\_math});{}$\2\6
\4${}\}{}$\2\6
\4${}\}{}$\2\6
\4${}\}{}$\2\par
\fi

\M{203}When the `\.{\v}' that introduces \CEE/ text is sensed, a call on
\PB{\\{C\_translate}} will return a pointer to the \TEX/ translation of
that text. If scraps exist in \PB{\\{scrap\_info}}, they are
unaffected by this translation process.

\Y\B\1\1\&{static} \&{text\_pointer} \\{C\_translate}(\&{void})\2\2\6
${}\{{}$\1\6
\&{text\_pointer} \|p;\C{ points to the translation }\6
\&{scrap\_pointer} \\{save\_base}${}\K\\{scrap\_base}{}$;\C{ holds original
value of \PB{\\{scrap\_base}} }\7
${}\\{scrap\_base}\K\\{scrap\_ptr}+\T{1};{}$\6
\\{C\_parse}(\\{section\_name});\C{ get the scraps together }\6
\&{if} ${}(\\{next\_control}\I\.{'|'}){}$\1\5
\\{err\_print}(\.{"!\ Missing\ '|'\ after}\)\.{\ C\ text"});\2\6
\\{app\_tok}(\\{cancel});\6
${}\\{app\_scrap}(\\{insert},\39\\{maybe\_math}){}$;\C{ place a \PB{\\{cancel}}
token as a final ``comment'' }\6
${}\|p\K\\{translate}(\,){}$;\C{ make the translation }\6
\&{if} ${}(\\{scrap\_ptr}>\\{max\_scr\_ptr}){}$\1\5
${}\\{max\_scr\_ptr}\K\\{scrap\_ptr};{}$\2\6
${}\\{scrap\_ptr}\K\\{scrap\_base}-\T{1};{}$\6
${}\\{scrap\_base}\K\\{save\_base}{}$;\C{ scrap the scraps }\6
\&{return} \|p;\6
\4${}\}{}$\2\par
\fi

\M{204}The \PB{\\{outer\_parse}} routine is to \PB{\\{C\_parse}} as \PB{%
\\{outer\_xref}}
is to \PB{\\{C\_xref}}: It constructs a sequence of scraps for \CEE/ text
until \PB{$\\{next\_control}\G\\{format\_code}$}. Thus, it takes care of
embedded comments.

The token list created from within `\pb' brackets is output as an argument
to \.{\\PB}.  Although \.{cwebmac} ignores \.{\\PB}, other macro packages
might use it to localize the special meaning of the macros that mark up
program text.

\Y\B\4\D\\{make\_pb}\5
\\{flags}[\.{'e'}]\par
\Y\B\4\X24:Set initial values\X${}\mathrel+\E{}$\6
$\\{make\_pb}\K\\{true}{}$;\par
\fi

\M{205}\B\1\1\&{static} \&{void} \\{outer\_parse}(\&{void})\C{ makes scraps
from \CEE/ tokens and comments }\2\2\6
${}\{{}$\1\6
\&{int} \\{bal};\C{ brace level in comment }\6
\&{text\_pointer} \|p${},{}$ \|q;\C{ partial comments }\7
\&{while} ${}(\\{next\_control}<\\{format\_code}){}$\1\6
\&{if} ${}(\\{next\_control}\I\\{begin\_comment}\W\\{next\_control}\I\\{begin%
\_short\_comment}){}$\1\5
\\{C\_parse}(\\{ignore});\2\6
\&{else}\5
${}\{{}$\1\6
\&{boolean} \\{is\_long\_comment}${}\K(\\{next\_control}\E\\{begin%
\_comment});{}$\7
\X197:Make sure that there is room for the new scraps, tokens, and texts\X\6
\\{app}(\\{cancel});\6
\\{app}(\\{inserted});\6
\&{if} (\\{is\_long\_comment})\1\5
\\{app\_str}(\.{"\\\\C\{"});\2\6
\&{else}\1\5
\\{app\_str}(\.{"\\\\SHC\{"});\2\6
${}\\{bal}\K\\{copy\_comment}(\\{is\_long\_comment},\39\T{1});{}$\6
${}\\{next\_control}\K\\{ignore};{}$\6
\&{while} ${}(\\{bal}>\T{0}){}$\5
${}\{{}$\1\6
${}\|p\K\\{text\_ptr};{}$\6
\\{freeze\_text};\6
${}\|q\K\\{C\_translate}(\,){}$;\C{ at this point we have \PB{$\\{tok\_ptr}+%
\T{6}\Z\\{max\_toks}$} }\6
${}\\{app}(\\{tok\_flag}+(\&{int})(\|p-\\{tok\_start}));{}$\6
\&{if} (\\{make\_pb})\1\5
\\{app\_str}(\.{"\\\\PB\{"});\2\6
${}\\{app}(\\{inner\_tok\_flag}+(\&{int})(\|q-\\{tok\_start}));{}$\6
\&{if} (\\{make\_pb})\1\5
\\{app\_tok}(\.{'\}'});\2\6
\&{if} ${}(\\{next\_control}\E\.{'|'}){}$\5
${}\{{}$\1\6
${}\\{bal}\K\\{copy\_comment}(\\{is\_long\_comment},\39\\{bal});{}$\6
${}\\{next\_control}\K\\{ignore};{}$\6
\4${}\}{}$\2\6
\&{else}\1\5
${}\\{bal}\K\T{0}{}$;\C{ an error has been reported }\2\6
\4${}\}{}$\2\6
\\{app}(\\{force});\6
${}\\{app\_scrap}(\\{insert},\39\\{no\_math}){}$;\C{ the full comment becomes a
scrap }\6
\4${}\}{}$\2\2\6
\4${}\}{}$\2\par
\fi

\N{1}{206}Output of tokens.
So far our programs have only built up multi-layered token lists in
\.{CWEAVE}'s internal memory; we have to figure out how to get them into
the desired final form. The job of converting token lists to characters in
the \TEX/ output file is not difficult, although it is an implicitly
recursive process. Four main considerations had to be kept in mind when
this part of \.{CWEAVE} was designed. (a) There are two modes of output:
\PB{\\{outer}} mode, which translates tokens like \PB{\\{force}} into
line-breaking
control sequences, and \PB{\\{inner}} mode, which ignores them except that
blank
spaces take the place of line breaks. (b) The \PB{\\{cancel}} instruction
applies
to adjacent token or tokens that are output, and this cuts across levels
of recursion since `\PB{\\{cancel}}' occurs at the beginning or end of a token
list on one level. (c) The \TEX/ output file will be semi-readable if line
breaks are inserted after the result of tokens like \PB{\\{break\_space}} and
\PB{\\{force}}. (d) The final line break should be suppressed, and there should
be no \PB{\\{force}} token output immediately after `\.{\\Y\\B}'.

\fi

\M{207}The output process uses a stack to keep track of what is going on at
different ``levels'' as the token lists are being written out. Entries on
this stack have three parts:

\yskip\hang \PB{\\{end\_field}} is the \PB{\\{tok\_mem}} location where the
token list of a
particular level will end;

\yskip\hang \PB{\\{tok\_field}} is the \PB{\\{tok\_mem}} location from which
the next token
on a particular level will be read;

\yskip\hang \PB{\\{mode\_field}} is the current mode, either \PB{\\{inner}} or %
\PB{\\{outer}}.

\yskip\noindent The current values of these quantities are referred to
quite frequently, so they are stored in a separate place instead of in the
\PB{\\{stack}} array. We call the current values \PB{\\{cur\_end}}, \PB{\\{cur%
\_tok}}, and
\PB{\\{cur\_mode}}.

The global variable \PB{\\{stack\_ptr}} tells how many levels of output are
currently in progress. The end of output occurs when an \PB{\\{end%
\_translation}}
token is found, so the stack is never empty except when we first begin the
output process.

\Y\B\4\D\\{inner}\5
\\{false}\C{ value of \PB{\\{mode}} for \CEE/ texts within \TEX/ texts }\par
\B\4\D\\{outer}\5
\\{true}\C{ value of \PB{\\{mode}} for \CEE/ texts in sections }\par
\Y\B\4\X22:Typedef declarations\X${}\mathrel+\E{}$\6
\&{typedef} \&{int} \&{mode};\6
\&{typedef} \&{struct} ${}\{{}$\1\6
\&{token\_pointer} \\{end\_field};\C{ ending location of token list }\6
\&{token\_pointer} \\{tok\_field};\C{ present location within token list }\6
\&{boolean} \\{mode\_field};\C{ interpretation of control tokens }\2\6
${}\}{}$ \&{output\_state};\6
\&{typedef} \&{output\_state} ${}{*}\&{stack\_pointer}{}$;\par
\fi

\M{208}\B\D\\{stack\_size}\5
\T{2000}\C{ number of simultaneous output levels }\par
\B\4\D\\{cur\_end}\5
$\\{cur\_state}.{}$\\{end\_field}\C{ current ending location in \PB{\\{tok%
\_mem}} }\par
\B\4\D\\{cur\_tok}\5
$\\{cur\_state}.{}$\\{tok\_field}\C{ location of next output token in \PB{%
\\{tok\_mem}} }\par
\B\4\D\\{cur\_mode}\5
$\\{cur\_state}.{}$\\{mode\_field}\C{ current mode of interpretation }\par
\B\4\D\\{init\_stack}\5
$\\{stack\_ptr}\K\\{stack};$ $\\{cur\_mode}\K{}$\\{outer}\C{ initialize the
stack }\par
\Y\B\4\X21:Private variables\X${}\mathrel+\E{}$\6
\&{static} \&{output\_state} \\{cur\_state};\C{ \PB{\\{cur\_end}}, \PB{\\{cur%
\_tok}}, \PB{\\{cur\_mode}} }\6
\&{static} \&{output\_state} \\{stack}[\\{stack\_size}];\C{ info for
non-current levels }\6
\&{static} \&{stack\_pointer} \\{stack\_end}${}\K\\{stack}+\\{stack\_size}-%
\T{1}{}$;\C{ end of \PB{\\{stack}} }\6
\&{static} \&{stack\_pointer} \\{stack\_ptr};\C{ first unused location in the
output state stack }\6
\&{static} \&{stack\_pointer} \\{max\_stack\_ptr};\C{ largest value assumed by %
\PB{\\{stack\_ptr}} }\par
\fi

\M{209}\B\X24:Set initial values\X${}\mathrel+\E{}$\6
$\\{max\_stack\_ptr}\K\\{stack}{}$;\par
\fi

\M{210}To insert token-list \PB{\|p} into the output, the \PB{\\{push\_level}}
subroutine
is called; it saves the old level of output and gets a new one going.
The value of \PB{\\{cur\_mode}} is not changed.

\Y\B\4\X8:Predeclaration of procedures\X${}\mathrel+\E{}$\6
\&{static} \&{void} \\{push\_level}(\&{text\_pointer});\6
\&{static} \&{void} \\{pop\_level}(\&{void});\par
\fi

\M{211}\B\1\1\&{static} \&{void} \\{push\_level}(\C{ suspends the current level
}\6
\&{text\_pointer} \|p)\2\2\6
${}\{{}$\1\6
\&{if} ${}(\\{stack\_ptr}\E\\{stack\_end}){}$\1\5
\\{overflow}(\.{"stack"});\2\6
\&{if} ${}(\\{stack\_ptr}>\\{stack}){}$\5
${}\{{}$\C{ save current state }\1\6
${}\\{stack\_ptr}\MG\\{end\_field}\K\\{cur\_end};{}$\6
${}\\{stack\_ptr}\MG\\{tok\_field}\K\\{cur\_tok};{}$\6
${}\\{stack\_ptr}\MG\\{mode\_field}\K\\{cur\_mode};{}$\6
\4${}\}{}$\2\6
${}\\{stack\_ptr}\PP;{}$\6
\&{if} ${}(\\{stack\_ptr}>\\{max\_stack\_ptr}){}$\1\5
${}\\{max\_stack\_ptr}\K\\{stack\_ptr};{}$\2\6
${}\\{cur\_tok}\K{*}\|p;{}$\6
${}\\{cur\_end}\K{*}(\|p+\T{1});{}$\6
\4${}\}{}$\2\par
\fi

\M{212}Conversely, the \PB{\\{pop\_level}} routine restores the conditions that
were in
force when the current level was begun. This subroutine will never be
called when \PB{$\\{stack\_ptr}\E\T{1}$}.

\Y\B\1\1\&{static} \&{void} \\{pop\_level}(\&{void})\2\2\6
${}\{{}$\1\6
${}\\{cur\_end}\K(\MM\\{stack\_ptr})\MG\\{end\_field};{}$\6
${}\\{cur\_tok}\K\\{stack\_ptr}\MG\\{tok\_field};{}$\6
${}\\{cur\_mode}\K\\{stack\_ptr}\MG\\{mode\_field};{}$\6
\4${}\}{}$\2\par
\fi

\M{213}The \PB{\\{get\_output}} function returns the next byte of output that
is not a
reference to a token list. It returns the values \PB{\\{identifier}} or \PB{%
\\{res\_word}}
or \PB{\\{section\_code}} if the next token is to be an identifier (typeset in
italics), a reserved word (typeset in boldface), or a section name (typeset
by a complex routine that might generate additional levels of output).
In these cases \PB{\\{cur\_name}} points to the identifier or section name in
question.

\Y\B\4\X21:Private variables\X${}\mathrel+\E{}$\6
\&{static} \&{name\_pointer} \\{cur\_name};\par
\fi

\M{214}\B\D\\{res\_word}\5
\T{\~201}\C{ returned by \PB{\\{get\_output}} for reserved words }\par
\B\4\D\\{section\_code}\5
\T{\~200}\C{ returned by \PB{\\{get\_output}} for section names }\par
\Y\B\4\X8:Predeclaration of procedures\X${}\mathrel+\E{}$\6
\&{static} \&{eight\_bits} \\{get\_output}(\&{void});\6
\&{static} \&{void} \\{output\_C}(\&{void});\6
\&{static} \&{void} \\{make\_output}(\&{void});\par
\fi

\M{215}\B\1\1\&{static} \&{eight\_bits} \\{get\_output}(\&{void})\C{ returns
the next token of output }\2\2\6
${}\{{}$\1\6
\&{sixteen\_bits} \|a;\C{ current item read from \PB{\\{tok\_mem}} }\7
\4\\{restart}:\6
\&{while} ${}(\\{cur\_tok}\E\\{cur\_end}){}$\1\5
\\{pop\_level}(\,);\2\6
${}\|a\K{*}(\\{cur\_tok}\PP);{}$\6
\&{if} ${}(\|a\G\T{\~400}){}$\5
${}\{{}$\1\6
${}\\{cur\_name}\K\|a\MOD\\{id\_flag}+\\{name\_dir};{}$\6
\&{switch} ${}(\|a/\\{id\_flag}){}$\5
${}\{{}$\1\6
\4\&{case} \T{2}:\5
\&{return} \\{res\_word};\C{ \PB{$\|a\E\\{res\_flag}+\\{cur\_name}$} }\6
\4\&{case} \T{3}:\5
\&{return} \\{section\_code};\C{ \PB{$\|a\E\\{section\_flag}+\\{cur\_name}$} }\6
\4\&{case} \T{4}:\5
${}\\{push\_level}(\|a\MOD\\{id\_flag}+\\{tok\_start});{}$\6
\&{goto} \\{restart};\C{ \PB{$\|a\E\\{tok\_flag}+\\{cur\_name}$} }\6
\4\&{case} \T{5}:\5
${}\\{push\_level}(\|a\MOD\\{id\_flag}+\\{tok\_start});{}$\6
${}\\{cur\_mode}\K\\{inner};{}$\6
\&{goto} \\{restart};\C{ \PB{$\|a\E\\{inner\_tok\_flag}+\\{cur\_name}$} }\6
\4\&{default}:\5
\&{return} \\{identifier};\C{ \PB{$\|a\E\\{id\_flag}+\\{cur\_name}$} }\6
\4${}\}{}$\2\6
\4${}\}{}$\2\6
\&{return} (\&{eight\_bits}) \|a;\6
\4${}\}{}$\2\par
\fi

\M{216}The real work associated with token output is done by \PB{\\{make%
\_output}}.
This procedure appends an \PB{\\{end\_translation}} token to the current token
list,
and then it repeatedly calls \PB{\\{get\_output}} and feeds characters to the
output
buffer until reaching the \PB{\\{end\_translation}} sentinel. It is possible
for
\PB{\\{make\_output}} to be called recursively, since a section name may
include
embedded \CEE/ text; however, the depth of recursion never exceeds one
level, since section names cannot be inside of section names.

A procedure called \PB{\\{output\_C}} does the scanning, translation, and
output of \CEE/ text within `\pb' brackets, and this procedure uses
\PB{\\{make\_output}} to output the current token list. Thus, the recursive
call
of \PB{\\{make\_output}} actually occurs when \PB{\\{make\_output}} calls \PB{%
\\{output\_C}}
while outputting the name of a section.

\Y\B\1\1\&{static} \&{void} \\{output\_C}(\&{void})\C{ outputs the current
token list }\2\2\6
${}\{{}$\1\6
\&{token\_pointer} \\{save\_tok\_ptr}${}\K\\{tok\_ptr};{}$\6
\&{text\_pointer} \\{save\_text\_ptr}${}\K\\{text\_ptr};{}$\6
\&{sixteen\_bits} \\{save\_next\_control}${}\K\\{next\_control}{}$;\C{ values
to be restored }\6
\&{text\_pointer} \|p;\C{ translation of the \CEE/ text }\7
${}\\{next\_control}\K\\{ignore};{}$\6
${}\|p\K\\{C\_translate}(\,);{}$\6
${}\\{app}(\\{inner\_tok\_flag}+(\&{int})(\|p-\\{tok\_start}));{}$\6
\&{if} (\\{make\_pb})\5
${}\{{}$\1\6
\\{out\_str}(\.{"\\\\PB\{"});\6
\\{make\_output}(\,);\6
\\{out}(\.{'\}'});\6
\4${}\}{}$\5
\2\&{else}\1\5
\\{make\_output}(\,);\C{ output the list }\2\6
\&{if} ${}(\\{text\_ptr}>\\{max\_text\_ptr}){}$\1\5
${}\\{max\_text\_ptr}\K\\{text\_ptr};{}$\2\6
\&{if} ${}(\\{tok\_ptr}>\\{max\_tok\_ptr}){}$\1\5
${}\\{max\_tok\_ptr}\K\\{tok\_ptr};{}$\2\6
${}\\{text\_ptr}\K\\{save\_text\_ptr};{}$\6
${}\\{tok\_ptr}\K\\{save\_tok\_ptr}{}$;\C{ forget the tokens }\6
${}\\{next\_control}\K\\{save\_next\_control}{}$;\C{ restore \PB{\\{next%
\_control}} to original state }\6
\4${}\}{}$\2\par
\fi

\M{217}Here is \.{CWEAVE}'s major output handler.

\Y\B\1\1\&{static} \&{void} \\{make\_output}(\&{void})\C{ outputs the
equivalents of tokens }\2\2\6
${}\{{}$\1\6
\&{eight\_bits} \|a${}\K\T{0}{}$;\C{ current output byte }\6
\&{eight\_bits} \|b;\C{ next output byte }\6
\&{int} \|c;\C{ count of \PB{\\{indent}} and \PB{\\{outdent}} tokens }\6
\&{char} ${}\\{scratch}[\\{longest\_name}+\T{1}]{}$;\C{ scratch area for
section names }\6
\&{char} ${}{*}\|k,{}$ ${}{*}\\{k\_limit}{}$;\C{ indices into \PB{\\{scratch}}
}\6
\&{char} ${}{*}\|j{}$;\C{ index into \PB{\\{buffer}} }\6
\&{char} ${}{*}\|p{}$;\C{ index into \PB{\\{byte\_mem}} }\6
\&{char} \\{delim};\C{ first and last character of string being copied }\6
\&{char} ${}{*}\\{save\_loc},{}$ ${}{*}\\{save\_limit}{}$;\C{ \PB{\\{loc}} and %
\PB{\\{limit}} to be restored }\6
\&{name\_pointer} \\{cur\_section\_name};\C{ name of section being output }\6
\&{boolean} \\{save\_mode};\C{ value of \PB{\\{cur\_mode}} before a sequence of
breaks }\6
\&{boolean} \\{dindent\_pending}${}\K\\{false}{}$;\C{ should a \PB{\\{dindent}}
be output? }\7
\\{app}(\\{end\_translation});\C{ append a sentinel }\6
\\{freeze\_text};\6
${}\\{push\_level}(\\{text\_ptr}-\T{1});{}$\6
\&{while} (\\{true})\5
${}\{{}$\1\6
${}\|a\K\\{get\_output}(\,);{}$\6
\4\\{reswitch}:\6
\&{switch} (\|a)\5
${}\{{}$\1\6
\4\&{case} \\{end\_translation}:\5
\&{return};\6
\4\&{case} \\{identifier}:\5
\&{case} \\{res\_word}:\5
\X218:Output an identifier\X\6
\&{break};\6
\4\&{case} \\{section\_code}:\5
\X222:Output a section name\X\6
\&{break};\6
\4\&{case} \\{math\_rel}:\5
\\{out\_str}(\.{"\\\\MRL\{"});\C{ fall through }\6
\4\&{case} \\{noop}:\5
\&{case} \\{inserted}:\5
\&{break};\6
\4\&{case} \\{cancel}:\5
\&{case} \\{big\_cancel}:\5
${}\|c\K\T{0};{}$\6
${}\|b\K\|a;{}$\6
\&{while} (\\{true})\5
${}\{{}$\1\6
${}\|a\K\\{get\_output}(\,);{}$\6
\&{if} ${}(\|a\E\\{inserted}){}$\1\5
\&{continue};\2\6
\&{if} ${}((\|a<\\{indent}\W\R(\|b\E\\{big\_cancel}\W\|a\E\.{'\ '}))\3{-1}\V(%
\|a>\\{big\_force}\W\|a\I\\{dindent})){}$\1\5
\&{break};\2\6
\&{switch} (\|a)\5
${}\{{}$\1\6
\4\&{case} \\{indent}:\5
${}\|c\PP;{}$\6
\&{break};\6
\4\&{case} \\{outdent}:\5
${}\|c\MM;{}$\6
\&{break};\6
\4\&{case} \\{dindent}:\5
${}\|c\MRL{+{\K}}\T{2};{}$\6
\&{break};\6
\4\&{case} \\{opt}:\5
${}\|a\K\\{get\_output}(\,);{}$\6
\4${}\}{}$\2\6
\4${}\}{}$\2\6
\X221:Output saved \PB{\\{indent}} or \PB{\\{outdent}} tokens\X\6
\&{goto} \\{reswitch};\6
\4\&{case} \\{dindent}:\5
${}\|a\K\\{get\_output}(\,);{}$\6
\&{if} ${}(\|a\I\\{big\_force}){}$\5
${}\{{}$\1\6
\\{out\_str}(\.{"\\\\1\\\\1"});\6
\&{goto} \\{reswitch};\6
\4${}\}{}$\2\6
\&{else}\1\5
${}\\{dindent\_pending}\K\\{true}{}$;\C{ fall through }\2\6
\4\&{case} \\{indent}:\5
\&{case} \\{outdent}:\5
\&{case} \\{opt}:\5
\&{case} \\{backup}:\5
\&{case} \\{break\_space}:\5
\&{case} \\{force}:\5
\&{case} \\{big\_force}:\5
\&{case} \\{preproc\_line}:\5
\X219:Output a control, look ahead in case of line breaks, possibly \PB{%
\&{goto} \\{reswitch}}\X\6
\&{break};\6
\4\&{case} \\{quoted\_char}:\5
${}\\{out}({*}(\\{cur\_tok}\PP)){}$;\C{ fall through }\6
\4\&{case} \\{qualifier}:\5
\&{break};\6
\4\&{default}:\5
\\{out}(\|a);\C{ otherwise \PB{\|a} is an ordinary character }\6
\4${}\}{}$\2\6
\4${}\}{}$\2\6
\4${}\}{}$\2\par
\fi

\M{218}An identifier of length one does not have to be enclosed in braces, and
it
looks slightly better if set in a math-italic font instead of a (slightly
narrower) text-italic font. Thus we output `\.{\\\v}\.{a}' but
`\.{\\\\\{aa\}}'.

\Y\B\4\X218:Output an identifier\X${}\E{}$\6
\\{out}(\.{'\\\\'});\6
\&{if} ${}(\|a\E\\{identifier}){}$\5
${}\{{}$\1\6
\&{if} ${}(\\{cur\_name}\MG\\{ilk}\E\\{custom}\W\R\\{doing\_format}){}$\5
${}\{{}$\1\6
\4\\{custom\_out}:\6
\&{for} ${}(\|p\K\\{cur\_name}\MG\\{byte\_start};{}$ ${}\|p<(\\{cur\_name}+%
\T{1})\MG\\{byte\_start};{}$ ${}\|p\PP){}$\1\5
${}\\{out}({*}\|p\E\.{'\_'}\?\.{'x'}:{*}\|p\E\.{'\$'}\?\.{'X'}:{*}\|p);{}$\2\6
\&{break};\6
\4${}\}{}$\2\6
\&{else} \&{if} (\\{is\_tiny}(\\{cur\_name}))\1\5
\\{out}(\.{'|'})\hbox{;}\2\6
\&{else}\5
${}\{{}$\1\6
${}\\{delim}\K\.{'.'};{}$\6
\&{for} ${}(\|p\K\\{cur\_name}\MG\\{byte\_start};{}$ ${}\|p<(\\{cur\_name}+%
\T{1})\MG\\{byte\_start};{}$ ${}\|p\PP){}$\1\6
\&{if} ${}(\\{xislower}({*}\|p)){}$\5
${}\{{}$\C{ not entirely uppercase }\1\6
${}\\{delim}\K\.{'\\\\'};{}$\6
\&{break};\6
\4${}\}{}$\2\2\6
\\{out}(\\{delim});\6
\4${}\}{}$\2\6
\4${}\}{}$\2\6
\&{else} \&{if} ${}(\\{cur\_name}\MG\\{ilk}\E\\{alfop}){}$\5
${}\{{}$\1\6
\\{out}(\.{'X'});\6
\&{goto} \\{custom\_out};\6
\4${}\}{}$\2\6
\&{else}\1\5
\\{out}(\.{'\&'});\C{ \PB{$\|a\E\\{res\_word}$} }\2\6
\&{if} (\\{is\_tiny}(\\{cur\_name}))\5
${}\{{}$\1\6
\&{if} ${}(\\{isxalpha}((\\{cur\_name}\MG\\{byte\_start})[\T{0}])){}$\1\5
\\{out}(\.{'\\\\'});\2\6
${}\\{out}((\\{cur\_name}\MG\\{byte\_start})[\T{0}]);{}$\6
\4${}\}{}$\2\6
\&{else}\1\5
${}\\{out\_name}(\\{cur\_name},\39\\{true}){}$;\2\par
\U217.\fi

\M{219}The current mode does not affect the behavior of \.{CWEAVE}'s output
routine
except when we are outputting control tokens.

\Y\B\4\X219:Output a control, look ahead in case of line breaks, possibly \PB{%
\&{goto} \\{reswitch}}\X${}\E{}$\6
\&{if} ${}(\|a<\\{break\_space}\V\|a\E\\{preproc\_line}){}$\5
${}\{{}$\1\6
\&{if} ${}(\\{cur\_mode}\E\\{outer}){}$\5
${}\{{}$\1\6
\\{out}(\.{'\\\\'});\6
${}\\{out}(\|a-\\{cancel}+\.{'0'});{}$\6
\&{if} ${}(\|a\E\\{opt}){}$\5
${}\{{}$\1\6
${}\|b\K\\{get\_output}(\,){}$;\C{ \PB{\\{opt}} is followed by a digit }\6
\&{if} ${}(\|b\I\.{'0'}\V\\{force\_lines}\E\\{false}){}$\1\5
\\{out}(\|b)\hbox{;}\2\6
\&{else}\1\5
\\{out\_str}(\.{"\{-1\}"});\C{ \PB{\\{force\_lines}} encourages more \.{@\v}
breaks }\2\6
\4${}\}{}$\2\6
\4${}\}{}$\2\6
\&{else} \&{if} ${}(\|a\E\\{opt}){}$\1\5
${}\|b\K\\{get\_output}(\,){}$;\C{ ignore digit following \PB{\\{opt}} }\2\6
\4${}\}{}$\2\6
\&{else} \X220:Look ahead for strongest line break, \PB{\&{goto} \\{reswitch}}%
\X\par
\U217.\fi

\M{220}If several of the tokens \PB{\\{break\_space}}, \PB{\\{force}}, \PB{%
\\{big\_force}} occur in a
row, possibly mixed with blank spaces (which are ignored),
the largest one is used. A line break also occurs in the output file,
except at the very end of the translation. The very first line break
is suppressed (i.e., a line break that follows `\.{\\Y\\B}').

\Y\B\4\X220:Look ahead for strongest line break, \PB{\&{goto} \\{reswitch}}%
\X${}\E{}$\6
${}\{{}$\1\6
${}\|b\K\|a;{}$\6
${}\\{save\_mode}\K\\{cur\_mode};{}$\6
\&{if} (\\{dindent\_pending})\5
${}\{{}$\1\6
${}\|c\K\T{2};{}$\6
${}\\{dindent\_pending}\K\\{false};{}$\6
\4${}\}{}$\2\6
\&{else}\1\5
${}\|c\K\T{0};{}$\2\6
\&{while} (\\{true})\5
${}\{{}$\1\6
${}\|a\K\\{get\_output}(\,);{}$\6
\&{if} ${}(\|a\E\\{inserted}){}$\1\5
\&{continue};\2\6
\&{if} ${}(\|a\E\\{cancel}\V\|a\E\\{big\_cancel}){}$\5
${}\{{}$\1\6
\X221:Output saved \PB{\\{indent}} or \PB{\\{outdent}} tokens\X\6
\&{goto} \\{reswitch};\C{ \PB{\\{cancel}} overrides everything }\6
\4${}\}{}$\2\6
\&{if} ${}((\|a\I\.{'\ '}\W\|a<\\{indent})\V\|a\E\\{backup}\V\|a>\\{big%
\_force}){}$\5
${}\{{}$\1\6
\&{if} ${}(\\{save\_mode}\E\\{outer}){}$\5
${}\{{}$\1\6
\&{if} ${}(\\{out\_ptr}>\\{out\_buf}+\T{3}\W\\{strncmp}(\\{out\_ptr}-\T{3},\39%
\.{"\\\\Y\\\\B"},\39\T{4})\E\T{0}){}$\1\5
\&{goto} \\{reswitch};\2\6
\X221:Output saved \PB{\\{indent}} or \PB{\\{outdent}} tokens\X\6
\\{out}(\.{'\\\\'});\6
${}\\{out}(\|b-\\{cancel}+\.{'0'});{}$\6
\&{if} ${}(\|a\I\\{end\_translation}){}$\1\5
\\{finish\_line}(\,);\2\6
\4${}\}{}$\2\6
\&{else} \&{if} ${}(\|a\I\\{end\_translation}\W\\{cur\_mode}\E\\{inner}){}$\1\5
\\{out}(\.{'\ '});\2\6
\&{goto} \\{reswitch};\6
\4${}\}{}$\2\6
\&{if} ${}(\|a\E\\{indent}){}$\1\5
${}\|c\PP;{}$\2\6
\&{else} \&{if} ${}(\|a\E\\{outdent}){}$\1\5
${}\|c\MM;{}$\2\6
\&{else} \&{if} ${}(\|a\E\\{opt}){}$\1\5
${}\|a\K\\{get\_output}(\,);{}$\2\6
\&{else} \&{if} ${}(\|a>\|b){}$\1\5
${}\|b\K\|a{}$;\C{ if \PB{$\|a\E\.{'\ '}$} we have \PB{$\|a<\|b$} }\2\6
\4${}\}{}$\2\6
\4${}\}{}$\2\par
\U219.\fi

\M{221}\B\X221:Output saved \PB{\\{indent}} or \PB{\\{outdent}} tokens\X${}%
\E{}$\6
\&{for} ( ; ${}\|c>\T{0};{}$ ${}\|c\MM){}$\1\5
\\{out\_str}(\.{"\\\\1"});\2\6
\&{for} ( ; ${}\|c<\T{0};{}$ ${}\|c\PP){}$\1\5
\\{out\_str}(\.{"\\\\2"});\2\par
\Us217\ET220.\fi

\M{222}The remaining part of \PB{\\{make\_output}} is somewhat more
complicated. When we
output a section name, we may need to enter the parsing and translation
routines, since the name may contain \CEE/ code embedded in
\pb\ constructions. This \CEE/ code is placed at the end of the active
input buffer and the translation process uses the end of the active
\PB{\\{tok\_mem}} area.

\Y\B\4\X222:Output a section name\X${}\E{}$\6
\\{out\_str}(\.{"\\\\X"});\6
${}\\{cur\_xref}\K{}$(\&{xref\_pointer}) \\{cur\_name}${}\MG\\{xref};{}$\6
\&{if} ${}((\\{an\_output}\K(\\{cur\_xref}\MG\\{num}\E\\{file\_flag}))\E%
\\{true}){}$\1\5
${}\\{cur\_xref}\K\\{cur\_xref}\MG\\{xlink};{}$\2\6
\&{if} ${}(\\{cur\_xref}\MG\\{num}\G\\{def\_flag}){}$\5
${}\{{}$\1\6
${}\\{out\_section}(\\{cur\_xref}\MG\\{num}-\\{def\_flag});{}$\6
\&{if} ${}(\\{phase}\E\T{3}){}$\5
${}\{{}$\1\6
${}\\{cur\_xref}\K\\{cur\_xref}\MG\\{xlink};{}$\6
\&{while} ${}(\\{cur\_xref}\MG\\{num}\G\\{def\_flag}){}$\5
${}\{{}$\1\6
\\{out\_str}(\.{",\ "});\6
${}\\{out\_section}(\\{cur\_xref}\MG\\{num}-\\{def\_flag});{}$\6
${}\\{cur\_xref}\K\\{cur\_xref}\MG\\{xlink};{}$\6
\4${}\}{}$\2\6
\4${}\}{}$\2\6
\4${}\}{}$\2\6
\&{else}\1\5
\\{out}(\.{'0'});\C{ output the section number, or zero if it was undefined }\2%
\6
\\{out}(\.{':'});\6
\&{if} (\\{an\_output})\1\5
\\{out\_str}(\.{"\\\\.\{"});\2\6
\X223:Output the text of the section name\X\6
\&{if} (\\{an\_output})\1\5
\\{out\_str}(\.{"\ \}"});\2\6
\\{out\_str}(\.{"\\\\X"});\par
\U217.\fi

\M{223}\B\X223:Output the text of the section name\X${}\E{}$\6
$\\{sprint\_section\_name}(\\{scratch},\39\\{cur\_name});{}$\6
${}\|k\K\\{scratch};{}$\6
${}\\{k\_limit}\K\\{scratch}+\\{strlen}(\\{scratch});{}$\6
${}\\{cur\_section\_name}\K\\{cur\_name};{}$\6
\&{while} ${}(\|k<\\{k\_limit}){}$\5
${}\{{}$\1\6
${}\|b\K{*}(\|k\PP);{}$\6
\&{if} ${}(\|b\E\.{'@'}){}$\1\5
\X224:Skip next character, give error if not `\.{@}'\X\2\6
\&{if} (\\{an\_output})\1\6
\&{switch} (\|b)\5
${}\{{}$\1\6
\4\&{case} \.{'\ '}:\5
\&{case} \.{'\\\\'}:\5
\&{case} \.{'\#'}:\5
\&{case} \.{'\%'}:\5
\&{case} \.{'\$'}:\5
\&{case} \.{'\^'}:\5
\&{case} \.{'\{'}:\5
\&{case} \.{'\}'}:\5
\&{case} \.{'\~'}:\5
\&{case} \.{'\&'}:\5
\&{case} \.{'\_'}:\5
\\{out}(\.{'\\\\'});\C{ falls through }\6
\4\&{default}:\5
\\{out}(\|b);\6
\4${}\}{}$\2\2\6
\&{else} \&{if} ${}(\|b\I\.{'|'}){}$\1\5
\\{out}(\|b)\hbox{;}\2\6
\&{else}\5
${}\{{}$\1\6
\X225:Copy the \CEE/ text into the \PB{\\{buffer}} array\X\6
${}\\{save\_loc}\K\\{loc};{}$\6
${}\\{save\_limit}\K\\{limit};{}$\6
${}\\{loc}\K\\{limit}+\T{2};{}$\6
${}\\{limit}\K\|j+\T{1};{}$\6
${}{*}\\{limit}\K\.{'|'};{}$\6
\\{output\_C}(\,);\6
${}\\{loc}\K\\{save\_loc};{}$\6
${}\\{limit}\K\\{save\_limit};{}$\6
\4${}\}{}$\2\6
\4${}\}{}$\2\par
\U222.\fi

\M{224}\B\X224:Skip next character, give error if not `\.{@}'\X${}\E{}$\6
\&{if} ${}({*}\|k\PP\I\.{'@'}){}$\5
${}\{{}$\1\6
${}\\{fputs}(\.{"\\n!\ Illegal\ control}\)\.{\ code\ in\ section\ nam}\)\.{e:\
<"},\39\\{stdout});{}$\6
\\{print\_section\_name}(\\{cur\_section\_name});\6
\\{printf}(\.{">\ "});\6
\\{mark\_error};\6
\4${}\}{}$\2\par
\U223.\fi

\M{225}The \CEE/ text enclosed in \pb\ should not contain `\.{\v}' characters,
except within strings. We put a `\.{\v}' at the front of the buffer, so that an
error message that displays the whole buffer will look a little bit sensible.
The variable \PB{\\{delim}} is zero outside of strings, otherwise it
equals the delimiter that began the string being copied.

\Y\B\4\X225:Copy the \CEE/ text into the \PB{\\{buffer}} array\X${}\E{}$\6
$\|j\K\\{limit}+\T{1};{}$\6
${}{*}\|j\K\.{'|'};{}$\6
${}\\{delim}\K\T{0};{}$\6
\&{while} (\\{true})\5
${}\{{}$\1\6
\&{if} ${}(\|k\G\\{k\_limit}){}$\5
${}\{{}$\1\6
${}\\{fputs}(\.{"\\n!\ C\ text\ in\ secti}\)\.{on\ name\ didn't\ end:\ }\)%
\.{<"},\39\\{stdout});{}$\6
\\{print\_section\_name}(\\{cur\_section\_name});\6
\\{printf}(\.{">\ "});\6
\\{mark\_error};\6
\&{break};\6
\4${}\}{}$\2\6
${}\|b\K{*}(\|k\PP);{}$\6
\&{if} ${}(\|b\E\.{'@'}\V(\|b\E\.{'\\\\'}\W\\{delim}\I\T{0})){}$\1\5
\X226:Copy a quoted character into the buffer\X\2\6
\&{else}\5
${}\{{}$\1\6
\&{if} ${}(\|b\E\.{'\\''}\V\|b\E\.{'"'}){}$\5
${}\{{}$\1\6
\&{if} ${}(\\{delim}\E\T{0}){}$\1\5
${}\\{delim}\K\|b;{}$\2\6
\&{else} \&{if} ${}(\\{delim}\E\|b){}$\1\5
${}\\{delim}\K\T{0};{}$\2\6
\4${}\}{}$\2\6
\&{if} ${}(\|b\I\.{'|'}\V\\{delim}\I\T{0}){}$\5
${}\{{}$\1\6
\&{if} ${}(\|j>\\{buffer}+\\{long\_buf\_size}-\T{3}){}$\1\5
\\{overflow}(\.{"buffer"});\2\6
${}{*}(\PP\|j)\K\|b;{}$\6
\4${}\}{}$\2\6
\&{else}\1\5
\&{break};\2\6
\4${}\}{}$\2\6
\4${}\}{}$\2\par
\U223.\fi

\M{226}\B\X226:Copy a quoted character into the buffer\X${}\E{}$\6
${}\{{}$\1\6
\&{if} ${}(\|j>\\{buffer}+\\{long\_buf\_size}-\T{4}){}$\1\5
\\{overflow}(\.{"buffer"});\2\6
${}{*}(\PP\|j)\K\|b;{}$\6
${}{*}(\PP\|j)\K{*}(\|k\PP);{}$\6
\4${}\}{}$\2\par
\U225.\fi

\N{0}{227}Phase two processing.
We have assembled enough pieces of the puzzle in order to be ready to specify
the processing in \.{CWEAVE}'s main pass over the source file. Phase two
is analogous to phase one, except that more work is involved because we must
actually output the \TEX/ material instead of merely looking at the
\.{CWEB} specifications.

\Y\B\1\1\&{static} \&{void} \\{phase\_two}(\&{void})\2\2\6
${}\{{}$\1\6
${}\\{phase}\K\T{2};{}$\6
\\{reset\_input}(\,);\6
\&{if} (\\{show\_progress})\1\5
${}\\{fputs}(\.{"\\nWriting\ the\ outpu}\)\.{t\ file..."},\39\\{stdout});{}$\2\6
${}\\{section\_count}\K\T{0};{}$\6
${}\\{format\_visible}\K\\{true};{}$\6
\\{copy\_limbo}(\,);\6
\\{finish\_line}(\,);\6
${}\\{flush\_buffer}(\\{out\_buf},\39\\{false},\39\\{false}){}$;\C{ insert a
blank line, it looks nice }\6
\&{while} ${}(\R\\{input\_has\_ended}){}$\1\5
\X230:Translate the current section\X\2\6
\4${}\}{}$\2\par
\fi

\M{228}\B\X8:Predeclaration of procedures\X${}\mathrel+\E{}$\5
\&{static} \&{void} \\{phase\_two}(\&{void});\par
\fi

\M{229}The output file will contain the control sequence \.{\\Y} between
non-null
sections of a section, e.g., between the \TEX/ and definition parts if both
are nonempty. This puts a little white space between the parts when they are
printed. However, we don't want \.{\\Y} to occur between two definitions
within a single section. The variables \PB{\\{out\_line}} or \PB{\\{out\_ptr}}
will
change if a section is non-null, so the following macros `\PB{\\{save%
\_position}}'
and `\PB{\\{emit\_space\_if\_needed}}' are able to handle the situation:

\Y\B\4\D\\{save\_position}\5
$\\{save\_line}\K\\{out\_line};$ $\\{save\_place}\K{}$\\{out\_ptr}\par
\B\4\D\\{emit\_space\_if\_needed}\6
\&{if} ${}(\\{save\_line}\I\\{out\_line}\V\\{save\_place}\I\\{out\_ptr}){}$\1\5
\\{out\_str}(\.{"\\\\Y"});\2\6
${}\\{space\_checked}\K\\{true}{}$;\par
\Y\B\4\X21:Private variables\X${}\mathrel+\E{}$\6
\&{static} \&{int} \\{save\_line};\C{ former value of \PB{\\{out\_line}} }\6
\&{static} \&{char} ${}{*}\\{save\_place}{}$;\C{ former value of \PB{\\{out%
\_ptr}} }\6
\&{static} \&{int} \\{sec\_depth};\C{ the integer, if any, following \.{@*} }\6
\&{static} \&{boolean} \\{space\_checked};\C{ have we done \PB{\\{emit\_space%
\_if\_needed}}? }\6
\&{static} \&{boolean} \\{format\_visible};\C{ should the next format
declaration be output? }\6
\&{static} \&{boolean} \\{doing\_format}${}\K\\{false}{}$;\C{ are we outputting
a format declaration? }\6
\&{static} \&{boolean} \\{group\_found}${}\K\\{false}{}$;\C{ has a starred
section occurred? }\par
\fi

\M{230}\B\X230:Translate the current section\X${}\E{}$\6
${}\{{}$\1\6
${}\\{section\_count}\PP;{}$\6
\X231:Output the code for the beginning of a new section\X\6
\\{save\_position};\6
\X232:Translate the \TEX/ part of the current section\X\6
\X233:Translate the definition part of the current section\X\6
\X239:Translate the \CEE/ part of the current section\X\6
\X242:Show cross-references to this section\X\6
\X246:Output the code for the end of a section\X\6
\4${}\}{}$\2\par
\U227.\fi

\M{231}Sections beginning with the \.{CWEB} control sequence `\.{@\ }' start in
the
output with the \TEX/ control sequence `\.{\\M}', followed by the section
number. Similarly, `\.{@*}' sections lead to the control sequence `\.{\\N}'.
In this case there's an additional parameter, representing one plus the
specified depth, immediately after the~\.{\\N}.
If the section has changed, we put \.{\\*} just after the section number.

\Y\B\4\X231:Output the code for the beginning of a new section\X${}\E{}$\6
\&{if} ${}({*}(\\{loc}-\T{1})\I\.{'*'}){}$\1\5
\\{out\_str}(\.{"\\\\M"});\2\6
\&{else}\5
${}\{{}$\1\6
\&{while} ${}({*}\\{loc}\E\.{'\ '}){}$\1\5
${}\\{loc}\PP;{}$\2\6
\&{if} ${}({*}\\{loc}\E\.{'*'}){}$\5
${}\{{}$\C{ ``top'' level }\1\6
${}\\{sec\_depth}\K{-}\T{1};{}$\6
${}\\{loc}\PP;{}$\6
\4${}\}{}$\2\6
\&{else}\5
${}\{{}$\1\6
\&{for} ${}(\\{sec\_depth}\K\T{0};{}$ ${}\\{xisdigit}({*}\\{loc});{}$ ${}%
\\{loc}\PP){}$\1\5
${}\\{sec\_depth}\K\\{sec\_depth}*\T{10}+({*}\\{loc})-\.{'0'};{}$\2\6
\4${}\}{}$\2\6
\&{while} ${}({*}\\{loc}\E\.{'\ '}){}$\1\5
${}\\{loc}\PP{}$;\C{ remove spaces before group title }\2\6
${}\\{group\_found}\K\\{true};{}$\6
\\{out\_str}(\.{"\\\\N"});\6
${}\{{}$\5
\1\&{char} \|s[\T{32}];\5
${}\\{sprintf}(\|s,\39\.{"\{\%d\}"},\39\\{sec\_depth}+\T{1}){}$;\5
\\{out\_str}(\|s);\5
${}\}{}$\2\6
\&{if} (\\{show\_progress})\1\5
${}\\{printf}(\.{"*\%d"},\39{}$(\&{int}) \\{section\_count});\2\6
\\{update\_terminal};\C{ print a progress report }\6
\4${}\}{}$\2\6
\\{out}(\.{'\{'});\6
\\{out\_section}(\\{section\_count});\6
\\{out}(\.{'\}'});\par
\U230.\fi

\M{232}In the \TEX/ part of a section, we simply copy the source text, except
that
index entries are not copied and \CEE/ text within \pb\ is translated.

\Y\B\4\X232:Translate the \TEX/ part of the current section\X${}\E{}$\6
\&{do}\5
\&{switch} ${}(\\{next\_control}\K\copyxTeX(\,)){}$\5
${}\{{}$\1\6
\4\&{case} \.{'|'}:\5
\\{init\_stack};\6
\\{output\_C}(\,);\6
\&{break};\6
\4\&{case} \.{'@'}:\5
\\{out}(\.{'@'});\6
\&{break};\6
\4\&{case} ${}\TeXxstring{}$:\5
\&{case} \\{noop}:\5
\&{case} \\{xref\_roman}:\5
\&{case} \\{xref\_wildcard}:\5
\&{case} \\{xref\_typewriter}:\5
\&{case} \\{section\_name}:\5
${}\\{loc}\MRL{-{\K}}\T{2};{}$\6
${}\\{next\_control}\K\\{get\_next}(\,){}$;\C{ skip to \.{@>} }\6
\&{if} ${}(\\{next\_control}\E\TeXxstring){}$\1\5
\\{err\_print}(\.{"!\ TeX\ string\ should}\)\.{\ be\ in\ C\ text\ only"});\2\6
\&{break};\6
\4\&{case} \\{thin\_space}:\5
\&{case} \\{math\_break}:\5
\&{case} \\{ord}:\5
\&{case} \\{line\_break}:\5
\&{case} \\{big\_line\_break}:\5
\&{case} \\{no\_line\_break}:\5
\&{case} \\{join}:\5
\&{case} \\{pseudo\_semi}:\5
\&{case} \\{macro\_arg\_open}:\5
\&{case} \\{macro\_arg\_close}:\5
\&{case} \\{output\_defs\_code}:\5
\\{err\_print}(\.{"!\ You\ can't\ do\ that}\)\.{\ in\ TeX\ text"});\6
\&{break};\6
\4${}\}{}$\2\5
\&{while} ${}(\\{next\_control}<\\{format\_code}){}$;\par
\U230.\fi

\M{233}When we get to the following code we have \PB{$\\{next\_control}\G%
\\{format\_code}$}, and
the token memory is in its initial empty state.

\Y\B\4\X233:Translate the definition part of the current section\X${}\E{}$\6
$\\{space\_checked}\K\\{false};{}$\6
\&{while} ${}(\\{next\_control}\Z\\{definition}){}$\5
${}\{{}$\C{ \PB{\\{format\_code}} or \PB{\\{definition}} }\1\6
\\{init\_stack};\6
\&{if} ${}(\\{next\_control}\E\\{definition}){}$\1\5
\X236:Start a macro definition\X\2\6
\&{else}\1\5
\X237:Start a format definition\X\2\6
\\{outer\_parse}(\,);\6
\\{finish\_C}(\\{format\_visible});\6
${}\\{format\_visible}\K\\{true};{}$\6
${}\\{doing\_format}\K\\{false};{}$\6
\4${}\}{}$\2\par
\U230.\fi

\M{234}The \PB{\\{finish\_C}} procedure outputs the translation of the current
scraps, preceded by the control sequence `\.{\\B}' and followed by the
control sequence `\.{\\par}'. It also restores the token and scrap
memories to their initial empty state.

A \PB{\\{force}} token is appended to the current scraps before translation
takes place, so that the translation will normally end with \.{\\6} or
\.{\\7} (the \TEX/ macros for \PB{\\{force}} and \PB{\\{big\_force}}). This \.{%
\\6} or
\.{\\7} is replaced by the concluding \.{\\par} or by \.{\\Y\\par}.

\Y\B\1\1\&{static} \&{void} \\{finish\_C}(\C{ finishes a definition or a \CEE/
part }\6
\&{boolean} \\{visible})\C{ \PB{\\{true}} if we should produce \TeX\ output }\2%
\2\6
${}\{{}$\1\6
\&{text\_pointer} \|p;\C{ translation of the scraps }\7
\&{if} (\\{visible})\5
${}\{{}$\1\6
\\{out\_str}(\.{"\\\\B"});\6
\\{app\_tok}(\\{force});\6
${}\\{app\_scrap}(\\{insert},\39\\{no\_math});{}$\6
${}\|p\K\\{translate}(\,);{}$\6
${}\\{app}(\\{tok\_flag}+(\&{int})(\|p-\\{tok\_start}));{}$\6
\\{make\_output}(\,);\C{ output the list }\6
\&{if} ${}(\\{out\_ptr}>\\{out\_buf}+\T{1}){}$\1\6
\&{if} ${}({*}(\\{out\_ptr}-\T{1})\E\.{'\\\\'}){}$\5
${}\{{}$\1\6
\&{if} ${}({*}\\{out\_ptr}\E\.{'6'}){}$\1\5
${}\\{out\_ptr}\MRL{-{\K}}\T{2};{}$\2\6
\&{else} \&{if} ${}({*}\\{out\_ptr}\E\.{'7'}){}$\1\5
${}{*}\\{out\_ptr}\K\.{'Y'};{}$\2\6
\4${}\}{}$\2\2\6
\\{out\_str}(\.{"\\\\par"});\6
\\{finish\_line}(\,);\6
\4${}\}{}$\2\6
\&{if} ${}(\\{text\_ptr}>\\{max\_text\_ptr}){}$\1\5
${}\\{max\_text\_ptr}\K\\{text\_ptr};{}$\2\6
\&{if} ${}(\\{tok\_ptr}>\\{max\_tok\_ptr}){}$\1\5
${}\\{max\_tok\_ptr}\K\\{tok\_ptr};{}$\2\6
\&{if} ${}(\\{scrap\_ptr}>\\{max\_scr\_ptr}){}$\1\5
${}\\{max\_scr\_ptr}\K\\{scrap\_ptr};{}$\2\6
${}\\{tok\_ptr}\K\\{tok\_mem}+\T{1};{}$\6
${}\\{text\_ptr}\K\\{tok\_start}+\T{1};{}$\6
${}\\{scrap\_ptr}\K\\{scrap\_info}{}$;\C{ forget the tokens and the scraps }\6
\4${}\}{}$\2\par
\fi

\M{235}\B\X8:Predeclaration of procedures\X${}\mathrel+\E{}$\5
\&{static} \&{void} \\{finish\_C}(\&{boolean});\par
\fi

\M{236}Keeping in line with the conventions of the \CEE/ preprocessor (and
otherwise contrary to the rules of \.{CWEB}) we distinguish here
between the case that `\.(' immediately follows an identifier and the
case that the two are separated by a space.  In the latter case, and
if the identifier is not followed by `\.(' at all, the replacement
text starts immediately after the identifier.  In the former case,
it starts after we scan the matching `\.)'.

\Y\B\4\X236:Start a macro definition\X${}\E{}$\6
${}\{{}$\1\6
\&{if} ${}(\\{save\_line}\I\\{out\_line}\V\\{save\_place}\I\\{out\_ptr}\V%
\\{space\_checked}){}$\1\5
\\{app}(\\{backup});\2\6
\&{if} ${}(\R\\{space\_checked}){}$\5
${}\{{}$\1\6
\\{emit\_space\_if\_needed};\6
\\{save\_position};\6
\4${}\}{}$\2\6
\\{app\_str}(\.{"\\\\D"});\C{ this will produce `\#\&{define }' }\6
\&{if} ${}((\\{next\_control}\K\\{get\_next}(\,))\I\\{identifier}){}$\1\5
\\{err\_print}(\.{"!\ Improper\ macro\ de}\)\.{finition"});\2\6
\&{else}\5
${}\{{}$\1\6
\\{app\_cur\_id}(\\{false});\6
\&{if} ${}({*}\\{loc}\E\.{'('}){}$\5
${}\{{}$\1\6
\\{app}(\.{'\$'});\6
\4\\{reswitch}:\6
\&{switch} ${}(\\{next\_control}\K\\{get\_next}(\,)){}$\5
${}\{{}$\1\6
\4\&{case} \.{'('}:\5
\&{case} \.{','}:\5
\\{app}(\\{next\_control});\6
\&{goto} \\{reswitch};\6
\4\&{case} \\{identifier}:\5
\\{app\_cur\_id}(\\{false});\6
\&{goto} \\{reswitch};\6
\4\&{case} \.{')'}:\5
\\{app}(\\{next\_control});\6
${}\\{next\_control}\K\\{get\_next}(\,);{}$\6
\&{break};\6
\4\&{case} \\{dot\_dot\_dot}:\5
\\{app\_str}(\.{"\\\\,\\\\ldots\\\\,"});\6
${}\\{app\_scrap}(\\{raw\_int},\39\\{no\_math});{}$\6
\&{if} ${}((\\{next\_control}\K\\{get\_next}(\,))\E\.{')'}){}$\5
${}\{{}$\1\6
\\{app}(\\{next\_control});\6
${}\\{next\_control}\K\\{get\_next}(\,);{}$\6
\&{break};\6
\4${}\}{}$\C{ otherwise fall through }\2\6
\4\&{default}:\5
\\{err\_print}(\.{"!\ Improper\ macro\ de}\)\.{finition"});\6
\&{break};\6
\4${}\}{}$\2\6
\\{app}(\.{'\$'});\6
\4${}\}{}$\2\6
\&{else}\1\5
${}\\{next\_control}\K\\{get\_next}(\,);{}$\2\6
\\{app}(\\{break\_space});\6
${}\\{app\_scrap}(\\{dead},\39\\{no\_math}){}$;\C{ scrap won't take part in the
parsing }\6
\4${}\}{}$\2\6
\4${}\}{}$\2\par
\U233.\fi

\M{237}\B\X237:Start a format definition\X${}\E{}$\6
${}\{{}$\1\6
${}\\{doing\_format}\K\\{true};{}$\6
\&{if} ${}({*}(\\{loc}-\T{1})\E\.{'s'}\V{*}(\\{loc}-\T{1})\E\.{'S'}){}$\1\5
${}\\{format\_visible}\K\\{false};{}$\2\6
\&{if} ${}(\R\\{space\_checked}){}$\5
${}\{{}$\1\6
\\{emit\_space\_if\_needed};\6
\\{save\_position};\6
\4${}\}{}$\2\6
\\{app\_str}(\.{"\\\\F"});\C{ this will produce `\&{format }' }\6
${}\\{next\_control}\K\\{get\_next}(\,);{}$\6
\&{if} ${}(\\{next\_control}\E\\{identifier}){}$\5
${}\{{}$\1\6
${}\\{app}(\\{id\_flag}+(\&{int})(\\{id\_lookup}(\\{id\_first},\39\\{id\_loc},%
\39\\{normal})-\\{name\_dir}));{}$\6
\\{app}(\\{break\_space});\C{ this is syntactically separate from what follows
}\6
${}\\{next\_control}\K\\{get\_next}(\,);{}$\6
\&{if} ${}(\\{next\_control}\E\\{identifier}){}$\5
${}\{{}$\1\6
${}\\{app}(\\{id\_flag}+(\&{int})(\\{id\_lookup}(\\{id\_first},\39\\{id\_loc},%
\39\\{normal})-\\{name\_dir}));{}$\6
${}\\{app\_scrap}(\\{exp},\39\\{maybe\_math});{}$\6
${}\\{app\_scrap}(\\{semi},\39\\{maybe\_math});{}$\6
${}\\{next\_control}\K\\{get\_next}(\,);{}$\6
\4${}\}{}$\2\6
\4${}\}{}$\2\6
\&{if} ${}(\\{scrap\_ptr}\I\\{scrap\_info}+\T{2}){}$\1\5
\\{err\_print}(\.{"!\ Improper\ format\ d}\)\.{efinition"});\2\6
\4${}\}{}$\2\par
\U233.\fi

\M{238}Finally, when the \TEX/ and definition parts have been treated, we have
\PB{$\\{next\_control}\G\\{begin\_C}$}. We will make the global variable \PB{%
\\{this\_section}}
point to the current section name, if it has a name.

\Y\B\4\X21:Private variables\X${}\mathrel+\E{}$\6
\&{static} \&{name\_pointer} \\{this\_section};\C{ the current section name, or
zero }\par
\fi

\M{239}\B\X239:Translate the \CEE/ part of the current section\X${}\E{}$\6
$\\{this\_section}\K\\{name\_dir};{}$\6
\&{if} ${}(\\{next\_control}\Z\\{section\_name}){}$\5
${}\{{}$\1\6
\\{emit\_space\_if\_needed};\6
\\{init\_stack};\6
\&{if} ${}(\\{next\_control}\E\\{begin\_C}){}$\1\5
${}\\{next\_control}\K\\{get\_next}(\,);{}$\2\6
\&{else}\5
${}\{{}$\1\6
${}\\{this\_section}\K\\{cur\_section};{}$\6
\X240:Check that `\.{=}' or `\.{==}' follows this section name, and emit the
scraps to start the section definition\X\6
\4${}\}{}$\2\6
\&{while} ${}(\\{next\_control}\Z\\{section\_name}){}$\5
${}\{{}$\1\6
\\{outer\_parse}(\,);\6
\X241:Emit the scrap for a section name if present\X\6
\4${}\}{}$\2\6
\\{finish\_C}(\\{true});\6
\4${}\}{}$\2\par
\U230.\fi

\M{240}The title of the section and an $\E$ or $\mathrel+\E$ are made
into a scrap that should not take part in the parsing.

\Y\B\4\X240:Check that `\.{=}' or `\.{==}' follows this section name, and emit
the scraps to start the section definition\X${}\E{}$\6
\&{do}\5
${}\\{next\_control}\K\\{get\_next}(\,);{}$\5
\&{while} ${}(\\{next\_control}\E\.{'+'}){}$;\C{ allow optional `\.{+=}' }\6
\&{if} ${}(\\{next\_control}\I\.{'='}\W\\{next\_control}\I\\{eq\_eq}){}$\1\5
\\{err\_print}(\.{"!\ You\ need\ an\ =\ sig}\)\.{n\ after\ the\ section\ }\)%
\.{name"});\2\6
\&{else}\1\5
${}\\{next\_control}\K\\{get\_next}(\,);{}$\2\6
\&{if} ${}(\\{out\_ptr}>\\{out\_buf}+\T{1}\W{*}\\{out\_ptr}\E\.{'Y'}\W{*}(%
\\{out\_ptr}-\T{1})\E\.{'\\\\'}){}$\1\5
\\{app}(\\{backup});\C{ the section name will be flush left }\2\6
${}\\{app}(\\{section\_flag}+(\&{int})(\\{this\_section}-\\{name\_dir}));{}$\6
${}\\{cur\_xref}\K{}$(\&{xref\_pointer}) \\{this\_section}${}\MG\\{xref};{}$\6
\&{if} ${}(\\{cur\_xref}\MG\\{num}\E\\{file\_flag}){}$\1\5
${}\\{cur\_xref}\K\\{cur\_xref}\MG\\{xlink};{}$\2\6
\\{app\_str}(\.{"\$\{\}"});\6
\&{if} ${}(\\{cur\_xref}\MG\\{num}\I\\{section\_count}+\\{def\_flag}){}$\5
${}\{{}$\1\6
\\{app\_str}(\.{"\\\\mathrel+"});\C{ section name is multiply defined }\6
${}\\{this\_section}\K\\{name\_dir}{}$;\C{ so we won't give cross-reference
info here }\6
\4${}\}{}$\2\6
\\{app\_str}(\.{"\\\\E"});\C{ output an equivalence sign }\6
\\{app\_str}(\.{"\{\}\$"});\6
\\{app}(\\{force});\6
${}\\{app\_scrap}(\\{dead},\39\\{no\_math}){}$;\C{ this forces a line break
unless `\.{@+}' follows }\par
\U239.\fi

\M{241}\B\X241:Emit the scrap for a section name if present\X${}\E{}$\6
\&{if} ${}(\\{next\_control}<\\{section\_name}){}$\5
${}\{{}$\1\6
\\{err\_print}(\.{"!\ You\ can't\ do\ that}\)\.{\ in\ C\ text"});\6
${}\\{next\_control}\K\\{get\_next}(\,);{}$\6
\4${}\}{}$\2\6
\&{else} \&{if} ${}(\\{next\_control}\E\\{section\_name}){}$\5
${}\{{}$\1\6
${}\\{app}(\\{section\_flag}+(\&{int})(\\{cur\_section}-\\{name\_dir}));{}$\6
${}\\{app\_scrap}(\\{section\_scrap},\39\\{maybe\_math});{}$\6
${}\\{next\_control}\K\\{get\_next}(\,);{}$\6
\4${}\}{}$\2\par
\U239.\fi

\M{242}Cross references relating to a named section are given
after the section ends.

\Y\B\4\X242:Show cross-references to this section\X${}\E{}$\6
\&{if} ${}(\\{this\_section}>\\{name\_dir}){}$\5
${}\{{}$\1\6
${}\\{cur\_xref}\K{}$(\&{xref\_pointer}) \\{this\_section}${}\MG\\{xref};{}$\6
\&{if} ${}((\\{an\_output}\K(\\{cur\_xref}\MG\\{num}\E\\{file\_flag}))\E%
\\{true}){}$\1\5
${}\\{cur\_xref}\K\\{cur\_xref}\MG\\{xlink};{}$\2\6
\&{if} ${}(\\{cur\_xref}\MG\\{num}>\\{def\_flag}){}$\1\5
${}\\{cur\_xref}\K\\{cur\_xref}\MG\\{xlink}{}$;\C{ bypass current section
number }\2\6
\\{footnote}(\\{def\_flag});\6
\\{footnote}(\\{cite\_flag});\6
\\{footnote}(\T{0});\6
\4${}\}{}$\2\par
\U230.\fi

\M{243}The \PB{\\{footnote}} procedure gives cross-reference information about
multiply defined section names (if the \PB{\\{flag}} parameter is
\PB{\\{def\_flag}}), or about references to a section name
(if \PB{$\\{flag}\E\\{cite\_flag}$}), or to its uses (if \PB{$\\{flag}\E%
\T{0}$}). It assumes that
\PB{\\{cur\_xref}} points to the first cross-reference entry of interest, and
it
leaves \PB{\\{cur\_xref}} pointing to the first element not printed.  Typical
outputs:
`\.{\\A101.}'; `\.{\\Us 370\\ET1009.}';
`\.{\\As 8, 27\\*\\ETs64.}'.

Note that the output of \.{CWEAVE} is not English-specific; users may
supply new definitions for the macros \.{\\A}, \.{\\As}, etc.

\Y\B\1\1\&{static} \&{void} \\{footnote}(\C{ outputs section cross-references }%
\6
\&{sixteen\_bits} \\{flag})\2\2\6
${}\{{}$\1\6
\&{xref\_pointer} \|q${}\K\\{cur\_xref}{}$;\C{ cross-reference pointer variable
}\7
\&{if} ${}(\|q\MG\\{num}\Z\\{flag}){}$\1\5
\&{return};\2\6
\\{finish\_line}(\,);\6
\\{out}(\.{'\\\\'});\6
${}\\{out}(\\{flag}\E\T{0}\?\.{'U'}:\\{flag}\E\\{cite\_flag}\?\.{'Q'}:%
\.{'A'});{}$\6
\X245:Output all the section numbers on the reference list \PB{\\{cur\_xref}}\X%
\6
\\{out}(\.{'.'});\6
\4${}\}{}$\2\par
\fi

\M{244}\B\X8:Predeclaration of procedures\X${}\mathrel+\E{}$\5
\&{static} \&{void} \\{footnote}(\&{sixteen\_bits});\par
\fi

\M{245}The following code distinguishes three cases, according as the number
of cross-references is one, two, or more than two. Variable \PB{\|q} points
to the first cross-reference, and the last link is a zero.

\Y\B\4\X245:Output all the section numbers on the reference list \PB{\\{cur%
\_xref}}\X${}\E{}$\6
\&{if} ${}(\|q\MG\\{xlink}\MG\\{num}>\\{flag}){}$\1\5
\\{out}(\.{'s'});\C{ plural }\2\6
\&{while} (\\{true})\5
${}\{{}$\1\6
${}\\{out\_section}(\\{cur\_xref}\MG\\{num}-\\{flag});{}$\6
${}\\{cur\_xref}\K\\{cur\_xref}\MG\\{xlink}{}$;\C{ point to the next
cross-reference to output }\6
\&{if} ${}(\\{cur\_xref}\MG\\{num}\Z\\{flag}){}$\1\5
\&{break};\2\6
\&{if} ${}(\\{cur\_xref}\MG\\{xlink}\MG\\{num}>\\{flag}){}$\1\5
\\{out\_str}(\.{",\ "});\C{ not the last }\2\6
\&{else}\5
${}\{{}$\1\6
\\{out\_str}(\.{"\\\\ET"});\C{ the last }\6
\&{if} ${}(\\{cur\_xref}\I\|q\MG\\{xlink}){}$\1\5
\\{out}(\.{'s'});\C{ the last of more than two }\2\6
\4${}\}{}$\2\6
\4${}\}{}$\2\par
\U243.\fi

\M{246}\B\X246:Output the code for the end of a section\X${}\E{}$\6
\\{out\_str}(\.{"\\\\fi"});\6
\\{finish\_line}(\,);\6
${}\\{flush\_buffer}(\\{out\_buf},\39\\{false},\39\\{false}){}$;\C{ insert a
blank line, it looks nice }\par
\U230.\fi

\N{0}{247}Phase three processing.
We are nearly finished! \.{CWEAVE}'s only remaining task is to write out the
index, after sorting the identifiers and index entries.

If the user has set the \PB{\\{no\_xref}} flag (the \.{-x} option on the
command line),
just finish off the page, omitting the index, section name list, and table of
contents.

\Y\B\1\1\&{static} \&{void} \\{phase\_three}(\&{void})\2\2\6
${}\{{}$\1\6
\&{if} (\\{no\_xref})\5
${}\{{}$\1\6
\\{finish\_line}(\,);\6
\\{out\_str}(\.{"\\\\end"});\6
\\{finish\_line}(\,);\6
\4${}\}{}$\2\6
\&{else}\5
${}\{{}$\1\6
${}\\{phase}\K\T{3};{}$\6
\&{if} (\\{show\_progress})\1\5
${}\\{fputs}(\.{"\\nWriting\ the\ index}\)\.{..."},\39\\{stdout});{}$\2\6
\\{finish\_line}(\,);\6
\&{if} ${}((\\{idx\_file}\K\\{fopen}(\\{idx\_file\_name},\39\.{"wb"}))\E%
\NULL){}$\1\5
${}\\{fatal}(\.{"!\ Cannot\ open\ index}\)\.{\ file\ "},\39\\{idx\_file%
\_name});{}$\2\6
\&{if} (\\{change\_exists})\5
${}\{{}$\1\6
\X250:Tell about changed sections\X\6
\\{finish\_line}(\,);\6
\\{finish\_line}(\,);\6
\4${}\}{}$\2\6
\\{out\_str}(\.{"\\\\inx"});\6
\\{finish\_line}(\,);\6
${}\\{active\_file}\K\\{idx\_file}{}$;\C{ change active file to the index file
}\6
\X252:Do the first pass of sorting\X\6
\X260:Sort and output the index\X\6
\\{finish\_line}(\,);\6
\\{fclose}(\\{active\_file});\C{ finished with \PB{\\{idx\_file}} }\6
${}\\{active\_file}\K\\{tex\_file}{}$;\C{ switch back to \PB{\\{tex\_file}} for
a tic }\6
\\{out\_str}(\.{"\\\\fin"});\6
\\{finish\_line}(\,);\6
\&{if} ${}((\\{scn\_file}\K\\{fopen}(\\{scn\_file\_name},\39\.{"wb"}))\E%
\NULL){}$\1\5
${}\\{fatal}(\.{"!\ Cannot\ open\ secti}\)\.{on\ file\ "},\39\\{scn\_file%
\_name});{}$\2\6
${}\\{active\_file}\K\\{scn\_file}{}$;\C{ change active file to section listing
file }\6
\X269:Output all the section names\X\6
\\{finish\_line}(\,);\6
\\{fclose}(\\{active\_file});\C{ finished with \PB{\\{scn\_file}} }\6
${}\\{active\_file}\K\\{tex\_file};{}$\6
\&{if} (\\{group\_found})\1\5
\\{out\_str}(\.{"\\\\con"});\5
\2\&{else}\1\5
\\{out\_str}(\.{"\\\\end"});\2\6
\\{finish\_line}(\,);\6
\\{fclose}(\\{active\_file});\6
\4${}\}{}$\2\6
\&{if} (\\{show\_happiness})\5
${}\{{}$\1\6
\&{if} (\\{show\_progress})\1\5
\\{new\_line};\2\6
${}\\{fputs}(\.{"Done."},\39\\{stdout});{}$\6
\4${}\}{}$\2\6
\\{check\_complete}(\,);\C{ was all of the change file used? }\6
\4${}\}{}$\2\par
\fi

\M{248}\B\X8:Predeclaration of procedures\X${}\mathrel+\E{}$\5
\&{static} \&{void} \\{phase\_three}(\&{void});\par
\fi

\M{249}Just before the index comes a list of all the changed sections,
including
the index section itself.

\Y\B\4\X21:Private variables\X${}\mathrel+\E{}$\6
\&{static} \&{sixteen\_bits} \\{k\_section};\C{ runs through the sections }\par
\fi

\M{250}\B\X250:Tell about changed sections\X${}\E{}$\C{ remember that the index
is already marked as changed }\6
$\\{k\_section}\K\T{0};{}$\6
\&{while} ${}(\R\\{changed\_section}[\PP\\{k\_section}]){}$\1\5
;\2\6
\\{out\_str}(\.{"\\\\ch\ "});\6
\\{out\_section}(\\{k\_section});\6
\&{while} ${}(\\{k\_section}<\\{section\_count}){}$\5
${}\{{}$\1\6
\&{while} ${}(\R\\{changed\_section}[\PP\\{k\_section}]){}$\1\5
;\2\6
\\{out\_str}(\.{",\ "});\6
\\{out\_section}(\\{k\_section});\6
\4${}\}{}$\2\6
\\{out}(\.{'.'});\par
\U247.\fi

\M{251}A left-to-right radix sorting method is used, since this makes it easy
to
adjust the collating sequence and since the running time will be at worst
proportional to the total length of all entries in the index. We put the
identifiers into different lists based on their first characters.
(Uppercase letters are put into the same list as the corresponding lowercase
letters, since we want to have `$t<\\{TeX}<\&{to}$'.) The
list for character \PB{\|c} begins at location \PB{\\{bucket}[\|c]} and
continues through
the \PB{\\{blink}} array.

\Y\B\4\X21:Private variables\X${}\mathrel+\E{}$\6
\&{static} \&{name\_pointer} \\{bucket}[\T{256}];\6
\&{static} \&{name\_pointer} \\{next\_name};\C{ successor of \PB{\\{cur\_name}}
when sorting }\6
\&{static} \&{name\_pointer} \\{blink}[\\{max\_names}];\C{ links in the buckets
}\par
\fi

\M{252}To begin the sorting, we go through all the hash lists and put each
entry
having a nonempty cross-reference list into the proper bucket.

\Y\B\4\X252:Do the first pass of sorting\X${}\E{}$\6
${}\{{}$\1\6
\&{int} \|c;\7
\&{for} ${}(\|c\K\T{0};{}$ ${}\|c<\T{256};{}$ ${}\|c\PP){}$\1\5
${}\\{bucket}[\|c]\K\NULL;{}$\2\6
\&{for} ${}(\|h\K\\{hash};{}$ ${}\|h\Z\\{hash\_end};{}$ ${}\|h\PP){}$\5
${}\{{}$\1\6
${}\\{next\_name}\K{*}\|h;{}$\6
\&{while} (\\{next\_name})\5
${}\{{}$\1\6
${}\\{cur\_name}\K\\{next\_name};{}$\6
${}\\{next\_name}\K\\{cur\_name}\MG\\{link};{}$\6
\&{if} ${}(\\{cur\_name}\MG\\{xref}\I{}$(\&{void} ${}{*}){}$ \\{xmem})\5
${}\{{}$\1\6
${}\|c\K(\\{cur\_name}\MG\\{byte\_start})[\T{0}];{}$\6
\&{if} (\\{xisupper}(\|c))\1\5
${}\|c\K\\{tolower}(\|c);{}$\2\6
${}\\{blink}[\\{cur\_name}-\\{name\_dir}]\K\\{bucket}[\|c];{}$\6
${}\\{bucket}[\|c]\K\\{cur\_name};{}$\6
\4${}\}{}$\2\6
\4${}\}{}$\2\6
\4${}\}{}$\2\6
\4${}\}{}$\2\par
\U247.\fi

\M{253}During the sorting phase we shall use the \PB{\\{cat}} and \PB{%
\\{trans}} arrays from
\.{CWEAVE}'s parsing algorithm and rename them \PB{\\{depth}} and \PB{%
\\{head}}. They now
represent a stack of identifier lists for all the index entries that have
not yet been output. The variable \PB{\\{sort\_ptr}} tells how many such lists
are
present; the lists are output in reverse order (first \PB{\\{sort\_ptr}}, then
\PB{$\\{sort\_ptr}-\T{1}$}, etc.). The \PB{\|j}th list starts at \PB{\\{head}[%
\|j]}, and if the first
\PB{\|k} characters of all entries on this list are known to be equal we have
\PB{$\\{depth}[\|j]\E\|k$}.

\Y\B\4\X253:Rest of \PB{\\{trans\_plus}} union\X${}\E{}$\6
\&{name\_pointer} \\{Head};\par
\U112.\fi

\M{254}\B\D\\{depth}\5
\\{cat}\C{ reclaims memory that is no longer needed for parsing }\par
\B\4\D\\{head}\5
$\\{trans\_plus}.{}$\\{Head}\C{ ditto }\par
\B\F\\{sort\_pointer}\5
\\{int}\par
\B\4\D\&{sort\_pointer}\5
\&{scrap\_pointer}\C{ ditto }\par
\B\4\D\\{sort\_ptr}\5
\\{scrap\_ptr}\C{ ditto }\par
\Y\B\4\X21:Private variables\X${}\mathrel+\E{}$\6
\&{static} \&{eight\_bits} \\{cur\_depth};\C{ depth of current buckets }\6
\&{static} \&{char} ${}{*}\\{cur\_byte}{}$;\C{ index into \PB{\\{byte\_mem}} }\6
\&{static} \&{sixteen\_bits} \\{cur\_val};\C{ current cross-reference number }\6
\&{static} \&{sort\_pointer} \\{max\_sort\_ptr};\C{ largest value of \PB{%
\\{sort\_ptr}} }\par
\fi

\M{255}\B\X24:Set initial values\X${}\mathrel+\E{}$\6
$\\{max\_sort\_ptr}\K\\{scrap\_info}{}$;\par
\fi

\M{256}The desired alphabetic order is specified by the \PB{\\{collate}} array;
namely,
$\PB{\\{collate}}[0]<\PB{\\{collate}}[1]<\cdots<\PB{\\{collate}}[100]$.

\Y\B\4\X21:Private variables\X${}\mathrel+\E{}$\6
\&{static} \&{eight\_bits} ${}\\{collate}[\T{101}+\T{128}]{}$;\C{ collation
order }\par
\fi

\M{257}We use the order $\hbox{null}<\.\ <\hbox{other characters}<{}$\.\_${}<
\.A=\.a<\cdots<\.Z=\.z<\.0<\cdots<\.9.$ Warning: The collation mapping
needs to be changed if ASCII code is not being used.

We initialize \PB{\\{collate}} by copying a few characters at a time, because
some \CEE/ compilers choke on long strings.

\Y\B\4\X24:Set initial values\X${}\mathrel+\E{}$\6
$\\{collate}[\T{0}]\K\T{0};{}$\6
\\{memcpy}((\&{char} ${}{*}){}$ \\{collate}${}+\T{1},\39\.{"\ \\1\\2\\3\\4\\5%
\\6\\7\\10\\}\)\.{11\\12\\13\\14\\15\\16\\17}\)\.{"},\39\T{16}){}$;\C{ 16
characters + 1 = 17 }\6
\\{memcpy}((\&{char} ${}{*}){}$ \\{collate}${}+\T{17},\39\.{"\\20\\21\\22\\23%
\\24\\25\\}\)\.{26\\27\\30\\31\\32\\33\\34}\)\.{\\35\\36\\37"},\39\T{16}){}$;%
\C{ 16 characters + 17 = 33 }\6
\\{memcpy}((\&{char} ${}{*}){}$ \\{collate}${}+\T{33},\39\.{"!\\42\#\$\%%
\&'()*+,-./:;}\)\.{<=>?@[\\\\]\^`\{|\}\~\_"},\39\T{32}){}$;\C{ 32 characters +
33 = 65 }\6
\\{memcpy}((\&{char} ${}{*}){}$ \\{collate}${}+\T{65},\39%
\.{"abcdefghijklmnopqrs}\)\.{tuvwxyz0123456789"},\39\T{36}){}$;\C{ (26 + 10)
characters + 65 = 101 }\6
\\{memcpy}((\&{char} ${}{*}){}$ \\{collate}${}+\T{101},\39\.{"\\200\\201\\202%
\\203\\20}\)\.{4\\205\\206\\207\\210\\21}\)\.{1\\212\\213\\214\\215\\21}\)\.{6%
\\217"},\39\T{16}){}$;\C{ 16 characters + 101 = 117 }\6
\\{memcpy}((\&{char} ${}{*}){}$ \\{collate}${}+\T{117},\39\.{"\\220\\221\\222%
\\223\\22}\)\.{4\\225\\226\\227\\230\\23}\)\.{1\\232\\233\\234\\235\\23}\)\.{6%
\\237"},\39\T{16}){}$;\C{ 16 characters + 117 = 133 }\6
\\{memcpy}((\&{char} ${}{*}){}$ \\{collate}${}+\T{133},\39\.{"\\240\\241\\242%
\\243\\24}\)\.{4\\245\\246\\247\\250\\25}\)\.{1\\252\\253\\254\\255\\25}\)\.{6%
\\257"},\39\T{16}){}$;\C{ 16 characters + 133 = 149 }\6
\\{memcpy}((\&{char} ${}{*}){}$ \\{collate}${}+\T{149},\39\.{"\\260\\261\\262%
\\263\\26}\)\.{4\\265\\266\\267\\270\\27}\)\.{1\\272\\273\\274\\275\\27}\)\.{6%
\\277"},\39\T{16}){}$;\C{ 16 characters + 149 = 165 }\6
\\{memcpy}((\&{char} ${}{*}){}$ \\{collate}${}+\T{165},\39\.{"\\300\\301\\302%
\\303\\30}\)\.{4\\305\\306\\307\\310\\31}\)\.{1\\312\\313\\314\\315\\31}\)\.{6%
\\317"},\39\T{16}){}$;\C{ 16 characters + 165 = 181 }\6
\\{memcpy}((\&{char} ${}{*}){}$ \\{collate}${}+\T{181},\39\.{"\\320\\321\\322%
\\323\\32}\)\.{4\\325\\326\\327\\330\\33}\)\.{1\\332\\333\\334\\335\\33}\)\.{6%
\\337"},\39\T{16}){}$;\C{ 16 characters + 181 = 197 }\6
\\{memcpy}((\&{char} ${}{*}){}$ \\{collate}${}+\T{197},\39\.{"\\340\\341\\342%
\\343\\34}\)\.{4\\345\\346\\347\\350\\35}\)\.{1\\352\\353\\354\\355\\35}\)\.{6%
\\357"},\39\T{16}){}$;\C{ 16 characters + 197 = 213 }\6
\\{memcpy}((\&{char} ${}{*}){}$ \\{collate}${}+\T{213},\39\.{"\\360\\361\\362%
\\363\\36}\)\.{4\\365\\366\\367\\370\\37}\)\.{1\\372\\373\\374\\375\\37}\)\.{6%
\\377"},\39\T{16}){}$;\C{ 16 characters + 213 = 229 }\par
\fi

\M{258}Procedure \PB{\\{unbucket}} goes through the buckets and adds nonempty
lists
to the stack, using the collating sequence specified in the \PB{\\{collate}}
array.
The parameter to \PB{\\{unbucket}} tells the current depth in the buckets.
Any two sequences that agree in their first 255 character positions are
regarded as identical.

\Y\B\4\D\\{infinity}\5
\T{255}\C{ $\infty$ (approximately) }\par
\Y\B\1\1\&{static} \&{void} \\{unbucket}(\C{ empties buckets having depth \PB{%
\|d} }\6
\&{eight\_bits} \|d)\2\2\6
${}\{{}$\1\6
\&{int} \|c;\C{ index into \PB{\\{bucket}}; cannot be a simple \PB{\&{char}}
because of sign     comparison below }\7
\&{for} ${}(\|c\K\T{100}+\T{128};{}$ ${}\|c\G\T{0};{}$ ${}\|c\MM){}$\1\6
\&{if} (\\{bucket}[\\{collate}[\|c]])\5
${}\{{}$\1\6
\&{if} ${}(\\{sort\_ptr}\G\\{scrap\_info\_end}){}$\1\5
\\{overflow}(\.{"sorting"});\2\6
${}\\{sort\_ptr}\PP;{}$\6
\&{if} ${}(\\{sort\_ptr}>\\{max\_sort\_ptr}){}$\1\5
${}\\{max\_sort\_ptr}\K\\{sort\_ptr};{}$\2\6
\&{if} ${}(\|c\E\T{0}){}$\1\5
${}\\{sort\_ptr}\MG\\{depth}\K\\{infinity};{}$\2\6
\&{else}\1\5
${}\\{sort\_ptr}\MG\\{depth}\K\|d;{}$\2\6
${}\\{sort\_ptr}\MG\\{head}\K\\{bucket}[\\{collate}[\|c]];{}$\6
${}\\{bucket}[\\{collate}[\|c]]\K\NULL;{}$\6
\4${}\}{}$\2\2\6
\4${}\}{}$\2\par
\fi

\M{259}\B\X8:Predeclaration of procedures\X${}\mathrel+\E{}$\5
\&{static} \&{void} \\{unbucket}(\&{eight\_bits});\par
\fi

\M{260}\B\X260:Sort and output the index\X${}\E{}$\6
$\\{sort\_ptr}\K\\{scrap\_info};{}$\6
\\{unbucket}(\T{1});\6
\&{while} ${}(\\{sort\_ptr}>\\{scrap\_info}){}$\5
${}\{{}$\1\6
${}\\{cur\_depth}\K\\{sort\_ptr}\MG\\{depth};{}$\6
\&{if} ${}(\\{blink}[\\{sort\_ptr}\MG\\{head}-\\{name\_dir}]\E\T{0}\V\\{cur%
\_depth}\E\\{infinity}){}$\1\5
\X262:Output index entries for the list at \PB{\\{sort\_ptr}}\X\2\6
\&{else}\1\5
\X261:Split the list at \PB{\\{sort\_ptr}} into further lists\X\2\6
\4${}\}{}$\2\par
\U247.\fi

\M{261}\B\X261:Split the list at \PB{\\{sort\_ptr}} into further lists\X${}%
\E{}$\6
${}\{{}$\1\6
\&{int} \|c;\7
${}\\{next\_name}\K\\{sort\_ptr}\MG\\{head};{}$\6
\&{do}\5
${}\{{}$\1\6
${}\\{cur\_name}\K\\{next\_name};{}$\6
${}\\{next\_name}\K\\{blink}[\\{cur\_name}-\\{name\_dir}];{}$\6
${}\\{cur\_byte}\K\\{cur\_name}\MG\\{byte\_start}+\\{cur\_depth};{}$\6
\&{if} ${}(\\{cur\_byte}\E(\\{cur\_name}+\T{1})\MG\\{byte\_start}){}$\1\5
${}\|c\K\T{0}{}$;\C{ hit end of the name }\2\6
\&{else}\5
${}\{{}$\1\6
${}\|c\K{*}\\{cur\_byte};{}$\6
\&{if} (\\{xisupper}(\|c))\1\5
${}\|c\K\\{tolower}(\|c);{}$\2\6
\4${}\}{}$\2\6
${}\\{blink}[\\{cur\_name}-\\{name\_dir}]\K\\{bucket}[\|c];{}$\6
${}\\{bucket}[\|c]\K\\{cur\_name};{}$\6
\4${}\}{}$\2\5
\&{while} (\\{next\_name});\6
${}\MM\\{sort\_ptr};{}$\6
${}\\{unbucket}(\\{cur\_depth}+\T{1});{}$\6
\4${}\}{}$\2\par
\U260.\fi

\M{262}\B\X262:Output index entries for the list at \PB{\\{sort\_ptr}}\X${}%
\E{}$\6
${}\{{}$\1\6
${}\\{cur\_name}\K\\{sort\_ptr}\MG\\{head};{}$\6
\&{do}\5
${}\{{}$\1\6
\\{out\_str}(\.{"\\\\I"});\6
\X263:Output the name at \PB{\\{cur\_name}}\X\6
\X264:Output the cross-references at \PB{\\{cur\_name}}\X\6
${}\\{cur\_name}\K\\{blink}[\\{cur\_name}-\\{name\_dir}];{}$\6
\4${}\}{}$\2\5
\&{while} (\\{cur\_name});\6
${}\MM\\{sort\_ptr};{}$\6
\4${}\}{}$\2\par
\U260.\fi

\M{263}\B\X263:Output the name at \PB{\\{cur\_name}}\X${}\E{}$\6
\&{switch} ${}(\\{cur\_name}\MG\\{ilk}){}$\5
${}\{{}$\5
\1\&{char} ${}{*}\|j{}$;\5
\hbox{}\6{\4}\&{case} \\{normal}:\5
\&{case} \\{func\_template}:\6
\&{if} (\\{is\_tiny}(\\{cur\_name}))\1\5
\\{out\_str}(\.{"\\\\|"});\2\6
\&{else}\5
${}\{{}$\1\6
\&{for} ${}(\|j\K\\{cur\_name}\MG\\{byte\_start};{}$ ${}\|j<(\\{cur\_name}+%
\T{1})\MG\\{byte\_start};{}$ ${}\|j\PP){}$\1\6
\&{if} ${}(\\{xislower}({*}\|j)){}$\1\5
\&{goto} \\{lowcase};\2\2\6
\\{out\_str}(\.{"\\\\."});\6
\&{break};\6
\4\\{lowcase}:\5
\\{out\_str}(\.{"\\\\\\\\"});\6
\4${}\}{}$\2\6
\&{break};\6
\4\&{case} \\{wildcard}:\5
\\{out\_str}(\.{"\\\\9"});\5
\&{goto} \\{not\_an\_identifier};\6
\4\&{case} \\{typewriter}:\5
\\{out\_str}(\.{"\\\\."});\6
\4\&{case} \\{roman}:\5
\\{not\_an\_identifier}:\5
${}\\{out\_name}(\\{cur\_name},\39\\{false});{}$\6
\&{goto} \\{name\_done};\6
\4\&{case} \\{custom}:\5
\\{out\_str}(\.{"\$\\\\"});\6
\&{for} ${}(\|j\K\\{cur\_name}\MG\\{byte\_start};{}$ ${}\|j<(\\{cur\_name}+%
\T{1})\MG\\{byte\_start};{}$ ${}\|j\PP){}$\1\5
${}\\{out}({*}\|j\E\.{'\_'}\?\.{'x'}:{*}\|j\E\.{'\$'}\?\.{'X'}:{*}\|j);{}$\2\6
\\{out}(\.{'\$'});\6
\&{goto} \\{name\_done};\6
\4\&{default}:\5
\\{out\_str}(\.{"\\\\\&"});\6
\4${}\}{}$\2\6
${}\\{out\_name}(\\{cur\_name},\39\\{true});{}$\6
\4\\{name\_done}:\par
\U262.\fi

\M{264}Section numbers that are to be underlined are enclosed in
`\.{\\[}$\,\ldots\,$\.]'.

\Y\B\4\X264:Output the cross-references at \PB{\\{cur\_name}}\X${}\E{}$\6
\X266:Invert the cross-reference list at \PB{\\{cur\_name}}, making \PB{\\{cur%
\_xref}} the head\X\6
\&{do}\5
${}\{{}$\1\6
\\{out\_str}(\.{",\ "});\6
${}\\{cur\_val}\K\\{cur\_xref}\MG\\{num};{}$\6
\&{if} ${}(\\{cur\_val}<\\{def\_flag}){}$\1\5
\\{out\_section}(\\{cur\_val});\2\6
\&{else}\5
${}\{{}$\1\6
\\{out\_str}(\.{"\\\\["});\6
${}\\{out\_section}(\\{cur\_val}-\\{def\_flag});{}$\6
\\{out}(\.{']'});\6
\4${}\}{}$\2\6
${}\\{cur\_xref}\K\\{cur\_xref}\MG\\{xlink};{}$\6
\4${}\}{}$\2\5
\&{while} ${}(\\{cur\_xref}\I\\{xmem});{}$\6
\\{out}(\.{'.'});\6
\\{finish\_line}(\,);\par
\U262.\fi

\M{265}List inversion is best thought of as popping elements off one stack and
pushing them onto another. In this case \PB{\\{cur\_xref}} will be the head of
the stack that we push things onto.

\Y\B\4\X21:Private variables\X${}\mathrel+\E{}$\6
\&{static} \&{xref\_pointer} \\{next\_xref}${},{}$ \\{this\_xref};\C{ pointer
variables for rearranging a list }\par
\fi

\M{266}\B\X266:Invert the cross-reference list at \PB{\\{cur\_name}}, making %
\PB{\\{cur\_xref}} the head\X${}\E{}$\6
$\\{this\_xref}\K{}$(\&{xref\_pointer}) \\{cur\_name}${}\MG\\{xref};{}$\6
${}\\{cur\_xref}\K\\{xmem};{}$\6
\&{do}\5
${}\{{}$\1\6
${}\\{next\_xref}\K\\{this\_xref}\MG\\{xlink};{}$\6
${}\\{this\_xref}\MG\\{xlink}\K\\{cur\_xref};{}$\6
${}\\{cur\_xref}\K\\{this\_xref};{}$\6
${}\\{this\_xref}\K\\{next\_xref};{}$\6
\4${}\}{}$\2\5
\&{while} ${}(\\{this\_xref}\I\\{xmem}){}$;\par
\U264.\fi

\M{267}The following recursive procedure walks through the tree of section
names and
prints them.

\Y\B\1\1\&{static} \&{void} \\{section\_print}(\C{ print all section names in
subtree \PB{\|p} }\6
\&{name\_pointer} \|p)\2\2\6
${}\{{}$\1\6
\&{if} (\|p)\5
${}\{{}$\1\6
${}\\{section\_print}(\|p\MG\\{llink});{}$\6
\\{out\_str}(\.{"\\\\I"});\6
${}\\{tok\_ptr}\K\\{tok\_mem}+\T{1};{}$\6
${}\\{text\_ptr}\K\\{tok\_start}+\T{1};{}$\6
${}\\{scrap\_ptr}\K\\{scrap\_info};{}$\6
\\{init\_stack};\6
${}\\{app}(\\{section\_flag}+(\&{int})(\|p-\\{name\_dir}));{}$\6
\\{make\_output}(\,);\6
\\{footnote}(\\{cite\_flag});\6
\\{footnote}(\T{0});\C{ \PB{\\{cur\_xref}} was set by \PB{\\{make\_output}} }\6
\\{finish\_line}(\,);\6
${}\\{section\_print}(\|p\MG\\{rlink});{}$\6
\4${}\}{}$\2\6
\4${}\}{}$\2\par
\fi

\M{268}\B\X8:Predeclaration of procedures\X${}\mathrel+\E{}$\5
\&{static} \&{void} \\{section\_print}(\&{name\_pointer});\par
\fi

\M{269}\B\X269:Output all the section names\X${}\E{}$\6
\\{section\_print}(\\{root});\par
\U247.\fi

\M{270}Because on some systems the difference between two pointers is a \PB{%
\&{ptrdiff\_t}}
rather than an \PB{\&{int}}, we use \.{\%td} to print these quantities.

\Y\B\1\1\&{void} \\{print\_stats}(\&{void})\2\2\6
${}\{{}$\1\6
\\{puts}(\.{"\\nMemory\ usage\ stat}\)\.{istics:"});\6
${}\\{printf}(\.{"\%td\ names\ (out\ of\ \%}\)\.{ld)\\n"},\39(\&{ptrdiff\_t})(%
\\{name\_ptr}-\\{name\_dir}),\39{}$(\&{long}) \\{max\_names});\6
${}\\{printf}(\.{"\%td\ cross-reference}\)\.{s\ (out\ of\ \%ld)\\n"},\39(%
\&{ptrdiff\_t})(\\{xref\_ptr}-\\{xmem}),\39{}$(\&{long}) \\{max\_refs});\6
${}\\{printf}(\.{"\%td\ bytes\ (out\ of\ \%}\)\.{ld)\\n"},\39(\&{ptrdiff\_t})(%
\\{byte\_ptr}-\\{byte\_mem}),\39{}$(\&{long}) \\{max\_bytes});\6
\\{puts}(\.{"Parsing:"});\6
${}\\{printf}(\.{"\%td\ scraps\ (out\ of\ }\)\.{\%ld)\\n"},\39(\&{ptrdiff\_t})(%
\\{max\_scr\_ptr}-\\{scrap\_info}),\39{}$(\&{long}) \\{max\_scraps});\6
${}\\{printf}(\.{"\%td\ texts\ (out\ of\ \%}\)\.{ld)\\n"},\39(\&{ptrdiff\_t})(%
\\{max\_text\_ptr}-\\{tok\_start}),\39{}$(\&{long}) \\{max\_texts});\6
${}\\{printf}(\.{"\%td\ tokens\ (out\ of\ }\)\.{\%ld)\\n"},\39(\&{ptrdiff\_t})(%
\\{max\_tok\_ptr}-\\{tok\_mem}),\39{}$(\&{long}) \\{max\_toks});\6
${}\\{printf}(\.{"\%td\ levels\ (out\ of\ }\)\.{\%ld)\\n"},\39(\&{ptrdiff\_t})(%
\\{max\_stack\_ptr}-\\{stack}),\39{}$(\&{long}) \\{stack\_size});\6
\\{puts}(\.{"Sorting:"});\6
${}\\{printf}(\.{"\%td\ levels\ (out\ of\ }\)\.{\%ld)\\n"},\39(\&{ptrdiff\_t})(%
\\{max\_sort\_ptr}-\\{scrap\_info}),\39{}$(\&{long}) \\{max\_scraps});\6
\4${}\}{}$\2\par
\fi

\N{0}{271}Index.
If you have read and understood the code for Phase III above, you know what
is in this index and how it got here. All sections in which an identifier is
used are listed with that identifier, except that reserved words are
indexed only when they appear in format definitions, and the appearances
of identifiers in section names are not indexed. Underlined entries
correspond to where the identifier was declared. Error messages, control
sequences put into the output, and a few
other things like ``recursion'' are indexed here too.
\fi

\inx
\fin
\con
