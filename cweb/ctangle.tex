\input cwebmac
% This file is part of CWEB.
% This program by Silvio Levy and Donald E. Knuth
% is based on a program by Knuth.
% It is distributed WITHOUT ANY WARRANTY, express or implied.
% Version 4.7 --- February 2022

% Copyright (C) 1987,1990,1993,2000 Silvio Levy and Donald E. Knuth

% Permission is granted to make and distribute verbatim copies of this
% document provided that the copyright notice and this permission notice
% are preserved on all copies.

% Permission is granted to copy and distribute modified versions of this
% document under the conditions for verbatim copying, provided that the
% entire resulting derived work is given a different name and distributed
% under the terms of a permission notice identical to this one.

% Amendments to 'ctangle.w' resulting in this updated version were created
% by numerous collaborators over the course of many years.

% Please send comments, suggestions, etc. to tex-k@tug.org.

% Here is TeX material that gets inserted after \input cwebmac
\def\hang{\hangindent 3em\indent\ignorespaces}
\def\pb{$\.|\ldots\.|$} % C brackets (|...|)
\def\v{\char'174} % vertical (|) in typewriter font
\mathchardef\RA="3221 % right arrow
\mathchardef\BA="3224 % double arrow

\def\title{CTANGLE (Version 4.7)}
\def\topofcontents{\null\vfill
  \centerline{\titlefont The {\ttitlefont CTANGLE} processor}
  \vskip 15pt
  \centerline{(Version 4.7)}
  \vfill}
\def\botofcontents{\vfill
\noindent
Copyright \copyright\ 1987, 1990, 1993, 2000 Silvio Levy and Donald E. Knuth
\bigskip\noindent
Permission is granted to make and distribute verbatim copies of this
document provided that the copyright notice and this permission notice
are preserved on all copies.

\smallskip\noindent
Permission is granted to copy and distribute modified versions of this
document under the conditions for verbatim copying, provided that the
entire resulting derived work is given a different name and distributed
under the terms of a permission notice identical to this one.
}
\pageno=\contentspagenumber \advance\pageno by 1
\let\maybe=\iftrue


\N{0}{1}Introduction.
This is the \.{CTANGLE} program by Silvio Levy and Donald E. Knuth,
based on \.{TANGLE} by Knuth.
We are thankful to
Nelson Beebe, Hans-Hermann Bode (to whom the \CPLUSPLUS/ adaptation is due),
Klaus Guntermann, Norman Ramsey, Tomas Rokicki, Joachim Schnitter,
Joachim Schrod, Lee Wittenberg, and others who have contributed improvements.

The ``banner line'' defined here should be changed whenever \.{CTANGLE}
is modified.

\Y\B\4\D\\{banner}\5
\.{"This\ is\ CTANGLE\ (Ve}\)\.{rsion\ 4.7)"}\par
\Y\B\X4:Include files\X\6
\ATH\6
\X3:Common code for \.{CWEAVE} and \.{CTANGLE}\X\6
\X19:Typedef declarations\X\6
\X20:Private variables\X\6
\X8:Predeclaration of procedures\X\par
\fi

\M{2}\.{CTANGLE} has a fairly straightforward outline.  It operates in
two phases: First it reads the source file, saving the \CEE/ code in
compressed form; then it shuffles and outputs the code.

Please read the documentation for \.{COMMON}, the set of routines common
to \.{CTANGLE} and \.{CWEAVE}, before proceeding further.

\Y\B\1\1\&{int} \\{main}(\&{int} \\{ac}${},\39{}$\&{char} ${}{*}{*}\\{av})\2%
\2{}$\6
${}\{{}$\1\6
${}\\{argc}\K\\{ac};{}$\6
${}\\{argv}\K\\{av};{}$\6
${}\\{program}\K\\{ctangle};{}$\6
\X21:Set initial values\X\6
\\{common\_init}(\,);\6
\&{if} (\\{show\_banner})\1\5
\\{puts}(\\{banner});\C{ print a ``banner line'' }\2\6
\\{phase\_one}(\,);\C{ read all the user's text and compress it into \PB{\\{tok%
\_mem}} }\6
\\{phase\_two}(\,);\C{ output the contents of the compressed tables }\6
\&{return} \\{wrap\_up}(\,);\C{ and exit gracefully }\6
\4${}\}{}$\2\par
\fi

\M{3}The next few sections contain stuff from the file \PB{\.{"common.w"}} that
must
be included in both \PB{\.{"ctangle.w"}} and \PB{\.{"cweave.w"}}. It appears in
file \PB{\.{"common.h"}}, which is also included in \PB{\.{"common.w"}} to
propagate
possible changes from this \.{COMMON} interface consistently.

% This file is part of CWEB.
% This program by Silvio Levy and Donald E. Knuth
% is based on a program by Knuth.
% It is distributed WITHOUT ANY WARRANTY, express or implied.
% Version 4.7 --- February 2022 (works also with later versions)

% Copyright (C) 1987,1990,1993 Silvio Levy and Donald E. Knuth

% Permission is granted to make and distribute verbatim copies of this
% document provided that the copyright notice and this permission notice
% are preserved on all copies.

% Permission is granted to copy and distribute modified versions of this
% document under the conditions for verbatim copying, provided that the
% entire resulting derived work is given a different name and distributed
% under the terms of a permission notice identical to this one.

% Amendments to 'common.h' resulting in this updated version were created
% by numerous collaborators over the course of many years.

% Please send comments, suggestions, etc. to tex-k@tug.org.

% The next few sections contain stuff from the file \PB{\.{"common.w"}} that
%has
% to be included in both \PB{\.{"ctangle.w"}} and \PB{\.{"cweave.w"}}. It
%appears in this
% file \PB{\.{"common.h"}}, which is also included in \PB{\.{"common.w"}} to
%propagate
% possible changes from this single source consistently.

First comes general stuff:



\Y\B\4\D\\{ctangle}\5
\\{false}\par
\B\4\D\\{cweave}\5
\\{true}\par
\Y\B\4\X3:Common code for \.{CWEAVE} and \.{CTANGLE}\X${}\E{}$\6
\&{typedef} \&{bool} \&{boolean};\6
\&{typedef} \&{uint8\_t} \&{eight\_bits};\6
\&{typedef} \&{uint16\_t} \&{sixteen\_bits};\6
\&{extern} \&{boolean} \\{program};\C{ \.{CWEAVE} or \.{CTANGLE}? }\6
\&{extern} \&{int} \\{phase};\C{ which phase are we in? }\par
\As5, 6, 7, 9, 10, 12, 14\ETs15.
\U1.\fi

\M{4}Interface to the standard \CEE/ library:

\Y\B\4\X4:Include files\X${}\E{}$\6
\8\#\&{include} \.{<ctype.h>}\C{ definition of \PB{\\{isalpha}}, \PB{%
\\{isdigit}} and so on }\6
\8\#\&{include} \.{<stdbool.h>}\C{ definition of \PB{\&{bool}}, \PB{\\{true}}
and \PB{\\{false}} }\6
\8\#\&{include} \.{<stddef.h>}\C{ definition of \PB{\&{ptrdiff\_t}} }\6
\8\#\&{include} \.{<stdint.h>}\C{ definition of \PB{\&{uint8\_t}} and \PB{%
\&{uint16\_t}} }\6
\8\#\&{include} \.{<stdio.h>}\C{ definition of \PB{\\{printf}} and friends }\6
\8\#\&{include} \.{<stdlib.h>}\C{ definition of \PB{\\{getenv}} and \PB{%
\\{exit}} }\6
\8\#\&{include} \.{<string.h>}\C{ definition of \PB{\\{strlen}}, \PB{%
\\{strcmp}} and so on }\par
\U1.\fi

\M{5}Code related to the character set:

\Y\B\4\D\\{and\_and}\5
\T{\~4}\C{ `\.{\&\&}'\,; corresponds to MIT's {\tentex\char'4} }\par
\B\4\D\\{lt\_lt}\5
\T{\~20}\C{ `\.{<<}'\,; corresponds to MIT's {\tentex\char'20} }\par
\B\4\D\\{gt\_gt}\5
\T{\~21}\C{ `\.{>>}'\,; corresponds to MIT's {\tentex\char'21} }\par
\B\4\D\\{plus\_plus}\5
\T{\~13}\C{ `\.{++}'\,; corresponds to MIT's {\tentex\char'13} }\par
\B\4\D\\{minus\_minus}\5
\T{\~1}\C{ `\.{--}'\,; corresponds to MIT's {\tentex\char'1} }\par
\B\4\D\\{minus\_gt}\5
\T{\~31}\C{ `\.{->}'\,; corresponds to MIT's {\tentex\char'31} }\par
\B\4\D\\{non\_eq}\5
\T{\~32}\C{ `\.{!=}'\,; corresponds to MIT's {\tentex\char'32} }\par
\B\4\D\\{lt\_eq}\5
\T{\~34}\C{ `\.{<=}'\,; corresponds to MIT's {\tentex\char'34} }\par
\B\4\D\\{gt\_eq}\5
\T{\~35}\C{ `\.{>=}'\,; corresponds to MIT's {\tentex\char'35} }\par
\B\4\D\\{eq\_eq}\5
\T{\~36}\C{ `\.{==}'\,; corresponds to MIT's {\tentex\char'36} }\par
\B\4\D\\{or\_or}\5
\T{\~37}\C{ `\.{\v\v}'\,; corresponds to MIT's {\tentex\char'37} }\par
\B\4\D\\{dot\_dot\_dot}\5
\T{\~16}\C{ `\.{...}'\,; corresponds to MIT's {\tentex\char'16} }\par
\B\4\D\\{colon\_colon}\5
\T{\~6}\C{ `\.{::}'\,; corresponds to MIT's {\tentex\char'6} }\par
\B\4\D\\{period\_ast}\5
\T{\~26}\C{ `\.{.*}'\,; corresponds to MIT's {\tentex\char'26} }\par
\B\4\D\\{minus\_gt\_ast}\5
\T{\~27}\C{ `\.{->*}'\,; corresponds to MIT's {\tentex\char'27} }\Y\par
\B\4\D\\{compress}$(\|c)$\5
\&{if} ${}(\\{loc}\PP\Z\\{limit})$ \&{return} \|c\par
\Y\B\4\X3:Common code for \.{CWEAVE} and \.{CTANGLE}\X${}\mathrel+\E{}$\6
\&{extern} \&{char} \\{section\_text}[\,];\C{ text being sought for }\6
\&{extern} \&{char} ${}{*}\\{section\_text\_end}{}$;\C{ end of \PB{\\{section%
\_text}} }\6
\&{extern} \&{char} ${}{*}\\{id\_first}{}$;\C{ where the current identifier
begins in the buffer }\6
\&{extern} \&{char} ${}{*}\\{id\_loc}{}$;\C{ just after the current identifier
in the buffer }\par
\fi

\M{6}Code related to input routines:
\Y\B\4\D\\{xisalpha}$(\|c)$\5
$(\\{isalpha}((\&{int})(\|c))\W((\&{eight\_bits})(\|c)<\T{\~200}){}$)\par
\B\4\D\\{xisdigit}$(\|c)$\5
$(\\{isdigit}((\&{int})(\|c))\W((\&{eight\_bits})(\|c)<\T{\~200}){}$)\par
\B\4\D\\{xisspace}$(\|c)$\5
$(\\{isspace}((\&{int})(\|c))\W((\&{eight\_bits})(\|c)<\T{\~200}){}$)\par
\B\4\D\\{xislower}$(\|c)$\5
$(\\{islower}((\&{int})(\|c))\W((\&{eight\_bits})(\|c)<\T{\~200}){}$)\par
\B\4\D\\{xisupper}$(\|c)$\5
$(\\{isupper}((\&{int})(\|c))\W((\&{eight\_bits})(\|c)<\T{\~200}){}$)\par
\B\4\D\\{xisxdigit}$(\|c)$\5
$(\\{isxdigit}((\&{int})(\|c))\W((\&{eight\_bits})(\|c)<\T{\~200}){}$)\par
\B\4\D\\{isxalpha}$(\|c)$\5
$((\|c)\E\.{'\_'}\V(\|c)\E\.{'\$'}{}$)\C{ non-alpha characters allowed in
identifier }\par
\B\4\D\\{ishigh}$(\|c)$\5
$((\&{eight\_bits})(\|c)>\T{\~177}{}$)\par
\Y\B\4\X3:Common code for \.{CWEAVE} and \.{CTANGLE}\X${}\mathrel+\E{}$\6
\&{extern} \&{char} \\{buffer}[\,];\C{ where each line of input goes }\6
\&{extern} \&{char} ${}{*}\\{buffer\_end}{}$;\C{ end of \PB{\\{buffer}} }\6
\&{extern} \&{char} ${}{*}\\{loc}{}$;\C{ points to the next character to be
read from the buffer }\6
\&{extern} \&{char} ${}{*}\\{limit}{}$;\C{ points to the last character in the
buffer }\par
\fi

\M{7}Code related to file handling:
\Y\B\F\\{line}\5
\|x\C{ make \PB{\\{line}} an unreserved word }\par
\B\4\D\\{max\_include\_depth}\5
\T{10}\C{ maximum number of source files open   simultaneously, not counting
the change file }\par
\B\4\D\\{max\_file\_name\_length}\5
\T{60}\par
\B\4\D\\{cur\_file}\5
\\{file}[\\{include\_depth}]\C{ current file }\par
\B\4\D\\{cur\_file\_name}\5
\\{file\_name}[\\{include\_depth}]\C{ current file name }\par
\B\4\D\\{cur\_line}\5
\\{line}[\\{include\_depth}]\C{ number of current line in current file }\par
\B\4\D\\{web\_file}\5
\\{file}[\T{0}]\C{ main source file }\par
\B\4\D\\{web\_file\_name}\5
\\{file\_name}[\T{0}]\C{ main source file name }\par
\Y\B\4\X3:Common code for \.{CWEAVE} and \.{CTANGLE}\X${}\mathrel+\E{}$\6
\&{extern} \&{int} \\{include\_depth};\C{ current level of nesting }\6
\&{extern} \&{FILE} ${}{*}\\{file}[\,]{}$;\C{ stack of non-change files }\6
\&{extern} \&{FILE} ${}{*}\\{change\_file}{}$;\C{ change file }\6
\&{extern} \&{char} \\{file\_name}[\,][\\{max\_file\_name\_length}];\C{ stack
of non-change file names }\6
\&{extern} \&{char} \\{change\_file\_name}[\,];\C{ name of change file }\6
\&{extern} \&{int} \\{line}[\,];\C{ number of current line in the stacked files
}\6
\&{extern} \&{int} \\{change\_line};\C{ number of current line in change file }%
\6
\&{extern} \&{int} \\{change\_depth};\C{ where \.{@y} originated during a
change }\6
\&{extern} \&{boolean} \\{input\_has\_ended};\C{ if there is no more input }\6
\&{extern} \&{boolean} \\{changing};\C{ if the current line is from \PB{%
\\{change\_file}} }\6
\&{extern} \&{boolean} \\{web\_file\_open};\C{ if the web file is being read }%
\par
\fi

\M{8}\B\X8:Predeclaration of procedures\X${}\E{}$\6
\&{extern} \&{boolean} \\{get\_line}(\&{void});\C{ inputs the next line }\6
\&{extern} \&{void} \\{check\_complete}(\&{void});\C{ checks that all changes
were picked up }\6
\&{extern} \&{void} \\{reset\_input}(\&{void});\C{ initialize to read the web
file and change file }\par
\As11, 13, 16, 30, 35, 39, 44, 49, 53, 65, 70, 84, 91, 99\ETs101.
\U1.\fi

\M{9}Code related to section numbers:
\Y\B\4\X3:Common code for \.{CWEAVE} and \.{CTANGLE}\X${}\mathrel+\E{}$\6
\&{extern} \&{sixteen\_bits} \\{section\_count};\C{ the current section number
}\6
\&{extern} \&{boolean} \\{changed\_section}[\,];\C{ is the section changed? }\6
\&{extern} \&{boolean} \\{change\_pending};\C{ is a decision about change still
unclear? }\6
\&{extern} \&{boolean} \\{print\_where};\C{ tells \.{CTANGLE} to print line and
file info }\par
\fi

\M{10}Code related to identifier and section name storage:
\Y\B\4\D\\{length}$(\|c)$\5
$(\&{size\_t})((\|c+\T{1})\MG\\{byte\_start}-(\|c)\MG\\{byte\_start}{}$)\C{ the
length of a name }\par
\B\4\D\\{print\_id}$(\|c)$\5
$\\{term\_write}((\|c)\MG\\{byte\_start},\39\\{length}(\|c){}$)\C{ print
identifier }\par
\B\4\D\\{llink}\5
\\{link}\C{ left link in binary search tree for section names }\par
\B\4\D\\{rlink}\5
$\\{dummy}.{}$\\{Rlink}\C{ right link in binary search tree for section names }%
\par
\B\4\D\\{root}\5
$\\{name\_dir}\MG{}$\\{rlink}\C{ the root of the binary search tree   for
section names }\par
\Y\B\4\X3:Common code for \.{CWEAVE} and \.{CTANGLE}\X${}\mathrel+\E{}$\6
\&{typedef} \&{struct} \&{name\_info} ${}\{{}$\1\6
\&{char} ${}{*}\\{byte\_start}{}$;\C{ beginning of the name in \PB{\\{byte%
\_mem}} }\6
\&{struct} \&{name\_info} ${}{*}\\{link};{}$\6
\&{union} ${}\{{}$\1\6
\&{struct} \&{name\_info} ${}{*}\\{Rlink}{}$;\C{ right link in binary search
tree for section       names }\6
\&{char} \\{Ilk};\C{ used by identifiers in \.{CWEAVE} only }\2\6
${}\}{}$ \\{dummy};\6
\&{void} ${}{*}\\{equiv\_or\_xref}{}$;\C{ info corresponding to names }\2\6
${}\}{}$ \&{name\_info};\C{ contains information about an identifier or section
name }\6
\&{typedef} \&{name\_info} ${}{*}\&{name\_pointer}{}$;\C{ pointer into array of
\&{name\_info}s }\6
\&{typedef} \&{name\_pointer} ${}{*}\&{hash\_pointer};{}$\6
\&{extern} \&{char} \\{byte\_mem}[\,];\C{ characters of names }\6
\&{extern} \&{char} ${}{*}\\{byte\_mem\_end}{}$;\C{ end of \PB{\\{byte\_mem}} }%
\6
\&{extern} \&{char} ${}{*}\\{byte\_ptr}{}$;\C{ first unused position in \PB{%
\\{byte\_mem}} }\6
\&{extern} \&{name\_info} \\{name\_dir}[\,];\C{ information about names }\6
\&{extern} \&{name\_pointer} \\{name\_dir\_end};\C{ end of \PB{\\{name\_dir}} }%
\6
\&{extern} \&{name\_pointer} \\{name\_ptr};\C{ first unused position in \PB{%
\\{name\_dir}} }\6
\&{extern} \&{name\_pointer} \\{hash}[\,];\C{ heads of hash lists }\6
\&{extern} \&{hash\_pointer} \\{hash\_end};\C{ end of \PB{\\{hash}} }\6
\&{extern} \&{hash\_pointer} \|h;\C{ index into hash-head array }\par
\fi

\M{11}\B\X8:Predeclaration of procedures\X${}\mathrel+\E{}$\6
\&{extern} \&{boolean} ${}\\{names\_match}(\&{name\_pointer},\39{}$\&{const} %
\&{char} ${}{*},\39\&{size\_t},\39\&{eight\_bits}){}$;\6
\&{extern} \&{name\_pointer} \\{id\_lookup}(\&{const} \&{char} ${}{*},\39{}$%
\&{const} \&{char} ${}{*},\39\&{eight\_bits}){}$;\C{ looks up a string in the
identifier table }\6
\&{extern} \&{name\_pointer} \\{section\_lookup}(\&{char} ${}{*},\39{}$\&{char}
${}{*},\39\&{boolean}){}$;\C{ finds section name }\6
\&{extern} \&{void} \\{init\_node}(\&{name\_pointer});\6
\&{extern} \&{void} ${}\\{init\_p}(\&{name\_pointer},\39\&{eight\_bits}){}$;\6
\&{extern} \&{void} \\{print\_prefix\_name}(\&{name\_pointer});\6
\&{extern} \&{void} \\{print\_section\_name}(\&{name\_pointer});\6
\&{extern} \&{void} \\{sprint\_section\_name}(\&{char} ${}{*},\39\&{name%
\_pointer}){}$;\par
\fi

\M{12}Code related to error handling:
\Y\B\4\D\\{spotless}\5
\T{0}\C{ \PB{\\{history}} value for normal jobs }\par
\B\4\D\\{harmless\_message}\5
\T{1}\C{ \PB{\\{history}} value when non-serious info was printed }\par
\B\4\D\\{error\_message}\5
\T{2}\C{ \PB{\\{history}} value when an error was noted }\par
\B\4\D\\{fatal\_message}\5
\T{3}\C{ \PB{\\{history}} value when we had to stop prematurely }\par
\B\4\D\\{mark\_harmless}\5
\&{if} ${}(\\{history}\E\\{spotless})$ $\\{history}\K{}$\\{harmless\_message}%
\par
\B\4\D\\{mark\_error}\5
$\\{history}\K{}$\\{error\_message}\par
\B\4\D\\{confusion}$(\|s)$\5
$\\{fatal}(\.{"!\ This\ can't\ happen}\)\.{:\ "},\39\|s{}$)\par
\Y\B\4\X3:Common code for \.{CWEAVE} and \.{CTANGLE}\X${}\mathrel+\E{}$\6
\&{extern} \&{int} \\{history};\C{ indicates how bad this run was }\par
\fi

\M{13}\B\X8:Predeclaration of procedures\X${}\mathrel+\E{}$\6
\&{extern} \&{int} \\{wrap\_up}(\&{void});\C{ indicate \PB{\\{history}} and
exit }\6
\&{extern} \&{void} \\{err\_print}(\&{const} \&{char} ${}{*}){}$;\C{ print
error message and context }\6
\&{extern} \&{void} \\{fatal}(\&{const} \&{char} ${}{*},\39{}$\&{const} %
\&{char} ${}{*}){}$;\C{ issue error message and die }\6
\&{extern} \&{void} \\{overflow}(\&{const} \&{char} ${}{*}){}$;\C{ succumb
because a table has overflowed }\par
\fi

\M{14}Code related to command line arguments:
\Y\B\4\D\\{show\_banner}\5
\\{flags}[\.{'b'}]\C{ should the banner line be printed? }\par
\B\4\D\\{show\_progress}\5
\\{flags}[\.{'p'}]\C{ should progress reports be printed? }\par
\B\4\D\\{show\_happiness}\5
\\{flags}[\.{'h'}]\C{ should lack of errors be announced? }\par
\B\4\D\\{show\_stats}\5
\\{flags}[\.{'s'}]\C{ should statistics be printed at end of run? }\par
\B\4\D\\{make\_xrefs}\5
\\{flags}[\.{'x'}]\C{ should cross references be output? }\par
\Y\B\4\X3:Common code for \.{CWEAVE} and \.{CTANGLE}\X${}\mathrel+\E{}$\6
\&{extern} \&{int} \\{argc};\C{ copy of \PB{\\{ac}} parameter to \PB{\\{main}}
}\6
\&{extern} \&{char} ${}{*}{*}\\{argv}{}$;\C{ copy of \PB{\\{av}} parameter to %
\PB{\\{main}} }\6
\&{extern} \&{char} \\{C\_file\_name}[\,];\C{ name of \PB{\\{C\_file}} }\6
\&{extern} \&{char} \\{tex\_file\_name}[\,];\C{ name of \PB{\\{tex\_file}} }\6
\&{extern} \&{char} \\{idx\_file\_name}[\,];\C{ name of \PB{\\{idx\_file}} }\6
\&{extern} \&{char} \\{scn\_file\_name}[\,];\C{ name of \PB{\\{scn\_file}} }\6
\&{extern} \&{boolean} \\{flags}[\,];\C{ an option for each 7-bit code }\par
\fi

\M{15}Code related to output:
\Y\B\4\D\\{update\_terminal}\5
\\{fflush}(\\{stdout})\C{ empty the terminal output buffer }\par
\B\4\D\\{new\_line}\5
\\{putchar}(\.{'\\n'})\par
\B\4\D\\{term\_write}$(\|a,\|b)$\5
$\\{fflush}(\\{stdout}),\39\\{fwrite}(\|a,\39\&{sizeof}(\&{char}),\39\|b,\39%
\\{stdout}{}$)\par
\Y\B\4\X3:Common code for \.{CWEAVE} and \.{CTANGLE}\X${}\mathrel+\E{}$\6
\&{extern} \&{FILE} ${}{*}\\{C\_file}{}$;\C{ where output of \.{CTANGLE} goes }%
\6
\&{extern} \&{FILE} ${}{*}\\{tex\_file}{}$;\C{ where output of \.{CWEAVE} goes
}\6
\&{extern} \&{FILE} ${}{*}\\{idx\_file}{}$;\C{ where index from \.{CWEAVE} goes
}\6
\&{extern} \&{FILE} ${}{*}\\{scn\_file}{}$;\C{ where list of sections from %
\.{CWEAVE} goes }\6
\&{extern} \&{FILE} ${}{*}\\{active\_file}{}$;\C{ currently active file for %
\.{CWEAVE} output }\par
\fi

\M{16}The procedure that gets everything rolling:
\Y\B\4\X8:Predeclaration of procedures\X${}\mathrel+\E{}$\6
\&{extern} \&{void} \\{common\_init}(\&{void});\6
\&{extern} \&{void} \\{print\_stats}(\&{void});\par
\fi

\M{17}The following parameters are sufficient to handle \TEX/ (converted to
\.{CWEB}), so they should be sufficient for most applications of \.{CWEB}.

\Y\B\4\D\\{buf\_size}\5
\T{200}\C{ maximum length of input line, plus one }\par
\B\4\D\\{longest\_name}\5
\T{10000}\C{ file names, section names, and section texts    shouldn't be
longer than this }\par
\B\4\D\\{long\_buf\_size}\5
$(\\{buf\_size}+\\{longest\_name}{}$)\C{ for \.{CWEAVE} }\par
\B\4\D\\{max\_bytes}\5
\T{100000}\C{ the number of bytes in identifiers,   index entries, and section
names; must be less than $2^{24}$ }\par
\B\4\D\\{max\_names}\5
\T{5000}\C{ number of identifiers, strings, section names;   must be less than
10240 }\par
\B\4\D\\{max\_sections}\5
\T{2000}\C{ greater than the total number of sections }\par
\fi

\M{18}End of \.{COMMON} interface.

\fi

\N{1}{19}Data structures exclusive to {\tt CTANGLE}.
We've already seen that the \PB{\\{byte\_mem}} array holds the names of
identifiers,
strings, and sections;
the \PB{\\{tok\_mem}} array holds the replacement texts
for sections. Allocation is sequential, since things are deleted only
during Phase II, and only in a last-in-first-out manner.

A \&{text} variable is a structure containing a pointer into
\PB{\\{tok\_mem}}, which tells where the corresponding text starts, and an
integer \PB{\\{text\_link}}, which, as we shall see later, is used to connect
pieces of text that have the same name.  All the \&{text}s are stored in
the array \PB{\\{text\_info}}, and we use a \&{text\_pointer} variable to refer
to them.

The first position of \PB{\\{tok\_mem}} that is unoccupied by
replacement text is called \PB{\\{tok\_ptr}}, and the first unused location of
\PB{\\{text\_info}} is called \PB{\\{text\_ptr}}.  Thus we usually have the
identity
\PB{$\\{text\_ptr}\MG\\{tok\_start}\E\\{tok\_ptr}$}.

\Y\B\4\X19:Typedef declarations\X${}\E{}$\6
\&{typedef} \&{struct} ${}\{{}$\1\6
\&{eight\_bits} ${}{*}\\{tok\_start}{}$;\C{ pointer into \PB{\\{tok\_mem}} }\6
\&{sixteen\_bits} \\{text\_link};\C{ relates replacement texts }\2\6
${}\}{}$ \&{text};\6
\&{typedef} \&{text} ${}{*}\&{text\_pointer}{}$;\par
\A31.
\U1.\fi

\M{20}\B\D\\{max\_texts}\5
\T{4000}\C{ number of replacement texts, must be less than 10240 }\par
\B\4\D\\{max\_toks}\5
\T{270000}\C{ number of bytes in compressed \CEE/ code }\par
\Y\B\4\X20:Private variables\X${}\E{}$\6
\&{static} \&{text} \\{text\_info}[\\{max\_texts}];\6
\&{static} \&{text\_pointer} \\{text\_info\_end}${}\K\\{text\_info}+\\{max%
\_texts}-\T{1};{}$\6
\&{static} \&{text\_pointer} \\{text\_ptr};\C{ first unused position in \PB{%
\\{text\_info}} }\6
\&{static} \&{eight\_bits} \\{tok\_mem}[\\{max\_toks}];\6
\&{static} \&{eight\_bits} ${}{*}\\{tok\_mem\_end}\K\\{tok\_mem}+\\{max\_toks}-%
\T{1};{}$\6
\&{static} \&{eight\_bits} ${}{*}\\{tok\_ptr}{}$;\C{ first unused position in %
\PB{\\{tok\_mem}} }\par
\As26, 32, 37, 42, 45, 52, 57, 62, 66, 68\ETs82.
\U1.\fi

\M{21}\B\X21:Set initial values\X${}\E{}$\6
$\\{text\_info}\MG\\{tok\_start}\K\\{tok\_ptr}\K\\{tok\_mem};{}$\6
${}\\{text\_ptr}\K\\{text\_info}+\T{1};{}$\6
${}\\{text\_ptr}\MG\\{tok\_start}\K\\{tok\_mem}{}$;\C{ this makes replacement
text 0 of length zero }\par
\As23, 27, 46, 58, 63\ETs78.
\U2.\fi

\M{22}If \PB{\|p} is a pointer to a section name, \PB{$\|p\MG\\{equiv}$} is a
pointer to its
replacement text, an element of the array \PB{\\{text\_info}}.

\Y\B\4\D\\{equiv}\5
\\{equiv\_or\_xref}\C{ info corresponding to names }\par
\fi

\M{23}\B\X21:Set initial values\X${}\mathrel+\E{}$\6
\\{init\_node}(\\{name\_dir});\C{ the undefined section has no replacement text
}\par
\fi

\M{24}Here's the procedure that decides whether a name of length \PB{\|l}
starting at position \PB{\\{first}} equals the identifier pointed to by \PB{%
\|p}:

\Y\B\1\1\&{boolean} \\{names\_match}(\&{name\_pointer} \|p${},{}$\C{ points to
the proposed match }\6
\&{const} \&{char} ${}{*}\\{first},{}$\C{ position of first character of string
}\6
\&{size\_t} \|l${},{}$\C{ length of identifier }\6
\&{eight\_bits} \|t)\C{ not used by \.{CTANGLE} }\2\2\6
${}\{{}$\5
\1(\&{void}) \|t;\6
\&{return} \\{length}(\|p)${}\E\|l\W\\{strncmp}(\\{first},\39\|p\MG\\{byte%
\_start},\39\|l)\E\T{0};{}$\6
\4${}\}{}$\2\par
\fi

\M{25}The common lookup routine refers to separate routines \PB{\\{init\_node}}
and
\PB{\\{init\_p}} when the data structure grows. Actually \PB{\\{init\_p}} is
called only by
\.{CWEAVE}, but we need to declare a dummy version so that
the loader won't complain of its absence.

\Y\B\1\1\&{void} \\{init\_node}(\&{name\_pointer} \\{node})\2\2\6
${}\{{}$\1\6
${}\\{node}\MG\\{equiv}\K{}$(\&{void} ${}{*}){}$ \\{text\_info};\6
\4${}\}{}$\2\7
\1\1\&{void} \\{init\_p}(\&{name\_pointer} \|p${},\39{}$\&{eight\_bits} \|t)\2%
\2\6
${}\{{}$\5
\1(\&{void}) \|p;\5
(\&{void}) \|t;\5
${}\}{}$\2\par
\fi

\N{1}{26}Tokens.
Replacement texts, which represent \CEE/ code in a compressed format,
appear in \PB{\\{tok\_mem}} as mentioned above. The codes in
these texts are called `tokens'; some tokens occupy two consecutive
eight-bit byte positions, and the others take just one byte.

If $p$ points to a replacement text, \PB{$\|p\MG\\{tok\_start}$} is the \PB{%
\\{tok\_mem}} position
of the first eight-bit code of that text. If \PB{$\|p\MG\\{text\_link}\E%
\T{0}$},
this is the replacement text for a macro, otherwise it is the replacement
text for a section. In the latter case \PB{$\|p\MG\\{text\_link}$} is either
equal to
\PB{\\{section\_flag}}, which means that there is no further text for this
section, or
\PB{$\|p\MG\\{text\_link}$} points to a continuation of this replacement text;
such
links are created when several sections have \CEE/ texts with the same
name, and they also tie together all the \CEE/ texts of unnamed sections.
The replacement text pointer for the first unnamed section appears in
\PB{$\\{text\_info}\MG\\{text\_link}$}, and the most recent such pointer is %
\PB{\\{last\_unnamed}}.

\Y\B\4\D\\{macro}\5
\T{0}\par
\B\4\D\\{section\_flag}\5
\\{max\_texts}\C{ final \PB{\\{text\_link}} in section replacement texts }\par
\Y\B\4\X20:Private variables\X${}\mathrel+\E{}$\6
\&{static} \&{text\_pointer} \\{last\_unnamed};\C{ most recent replacement text
of unnamed section }\par
\fi

\M{27}\B\X21:Set initial values\X${}\mathrel+\E{}$\6
$\\{last\_unnamed}\K\\{text\_info};{}$\6
${}\\{text\_info}\MG\\{text\_link}\K\\{macro}{}$;\par
\fi

\M{28}If the first byte of a token is less than \PB{\T{\~200}}, the token
occupies a
single byte. Otherwise we make a sixteen-bit token by combining two consecutive
bytes \PB{\|a} and \PB{\|b}. If \PB{$\T{\~200}\Z\|a<\T{\~250}$}, then \PB{$(%
\|a-\T{\~200})\hbox{${}\times2^8$}+\|b$}
points to an identifier; if \PB{$\T{\~250}\Z\|a<\T{\~320}$}, then
\PB{$(\|a-\T{\~250})\hbox{${}\times2^8$}+\|b$} points to a section name
(or, if it has the special value \PB{\\{output\_defs\_flag}},
to the area where the preprocessor definitions are stored); and if
\PB{$\T{\~320}\Z\|a<\T{\~400}$}, then \PB{$(\|a-\T{\~320})\hbox{${}\times2^8$}+%
\|b$} is the number of the section
in which the current replacement text appears.

Codes less than \PB{\T{\~200}} are 7-bit \PB{\&{char}} codes that represent
themselves.
Some of the 7-bit codes will not be present, however, so we can
use them for special purposes. The following symbolic names are used:

\yskip \hang \PB{\\{string}} denotes the beginning or end of a string
or a verbatim construction.

\hang \PB{\\{constant}} denotes a numerical constant.

\hang \PB{\\{join}} denotes the concatenation of adjacent items with no space
or line breaks allowed between them (the \.{@\&} operation of \.{CWEB}).

\Y\B\4\D\\{string}\5
\T{\~2}\C{ takes the place of ASCII \.{STX} }\par
\B\4\D\\{constant}\5
\T{\~3}\C{ takes the place of ASCII \.{ETX} }\par
\B\4\D\\{join}\5
\T{\~177}\C{ takes the place of ASCII \.{DEL} }\par
\B\4\D\\{output\_defs\_flag}\5
$(\T{2}*\T{\~24000}-\T{1}{}$)\C{ \PB{$\T{\~24000}\E(\T{\~250}-\T{\~200})*\T{%
\~400}$} }\par
\fi

\M{29}The following procedure is used to enter a two-byte value into
\PB{\\{tok\_mem}} when a replacement text is being generated.

\Y\B\1\1\&{static} \&{void} \\{store\_two\_bytes}(\&{sixteen\_bits} \|x)\2\2\6
${}\{{}$\1\6
\&{if} ${}(\\{tok\_ptr}+\T{2}>\\{tok\_mem\_end}){}$\1\5
\\{overflow}(\.{"token"});\2\6
${}{*}\\{tok\_ptr}\PP\K\|x\GG\T{8}{}$;\C{ store high byte }\6
${}{*}\\{tok\_ptr}\PP\K\|x\AND\T{\~377}{}$;\C{ store low byte }\6
\4${}\}{}$\2\par
\fi

\M{30}\B\X8:Predeclaration of procedures\X${}\mathrel+\E{}$\5
\&{static} \&{void} \\{store\_two\_bytes}(\&{sixteen\_bits});\par
\fi

\N{0}{31}Stacks for output.  The output process uses a stack to keep track
of what is going on at different ``levels'' as the sections are being
written out.  Entries on this stack have five parts:

\yskip\hang \PB{\\{end\_field}} is the \PB{\\{tok\_mem}} location where the
replacement
text of a particular level will end;

\hang \PB{\\{byte\_field}} is the \PB{\\{tok\_mem}} location from which the
next token
on a particular level will be read;

\hang \PB{\\{name\_field}} points to the name corresponding to a particular
level;

\hang \PB{\\{repl\_field}} points to the replacement text currently being read
at a particular level;

\hang \PB{\\{section\_field}} is the section number, or zero if this is a
macro.

\yskip\noindent The current values of these five quantities are referred to
quite frequently, so they are stored in a separate place instead of in
the \PB{\\{stack}} array. We call the current values \PB{\\{cur\_end}}, \PB{%
\\{cur\_byte}},
\PB{\\{cur\_name}}, \PB{\\{cur\_repl}}, and \PB{\\{cur\_section}}.

The global variable \PB{\\{stack\_ptr}} tells how many levels of output are
currently in progress. The end of all output occurs when the stack is
empty, i.e., when \PB{$\\{stack\_ptr}\E\\{stack}$}.

\Y\B\4\X19:Typedef declarations\X${}\mathrel+\E{}$\6
\&{typedef} \&{struct} ${}\{{}$\1\6
\&{eight\_bits} ${}{*}\\{end\_field}{}$;\C{ ending location of replacement text
}\6
\&{eight\_bits} ${}{*}\\{byte\_field}{}$;\C{ present location within
replacement text }\6
\&{name\_pointer} \\{name\_field};\C{ \PB{\\{byte\_start}} index for text being
output }\6
\&{text\_pointer} \\{repl\_field};\C{ \PB{\\{tok\_start}} index for text being
output }\6
\&{sixteen\_bits} \\{section\_field};\C{ section number or zero if not a
section }\2\6
${}\}{}$ \&{output\_state};\6
\&{typedef} \&{output\_state} ${}{*}\&{stack\_pointer}{}$;\par
\fi

\M{32}\B\D\\{stack\_size}\5
\T{50}\C{ number of simultaneous levels of macro expansion }\par
\B\4\D\\{cur\_end}\5
$\\{cur\_state}.{}$\\{end\_field}\C{ current ending location in \PB{\\{tok%
\_mem}} }\par
\B\4\D\\{cur\_byte}\5
$\\{cur\_state}.{}$\\{byte\_field}\C{ location of next output byte in \PB{%
\\{tok\_mem}}}\par
\B\4\D\\{cur\_name}\5
$\\{cur\_state}.{}$\\{name\_field}\C{ pointer to current name being expanded }%
\par
\B\4\D\\{cur\_repl}\5
$\\{cur\_state}.{}$\\{repl\_field}\C{ pointer to current replacement text }\par
\B\4\D\\{cur\_section}\5
$\\{cur\_state}.{}$\\{section\_field}\C{ current section number being expanded
}\par
\Y\B\4\X20:Private variables\X${}\mathrel+\E{}$\6
\&{static} \&{output\_state} \\{cur\_state};\C{ \PB{\\{cur\_end}}, \PB{\\{cur%
\_byte}}, \PB{\\{cur\_name}}, \PB{\\{cur\_repl}},   and \PB{\\{cur\_section}} }%
\6
\&{static} \&{output\_state} ${}\\{stack}[\\{stack\_size}+\T{1}]{}$;\C{ info
for non-current levels }\6
\&{static} \&{stack\_pointer} \\{stack\_end}${}\K\\{stack}+\\{stack\_size}{}$;%
\C{ end of \PB{\\{stack}} }\6
\&{static} \&{stack\_pointer} \\{stack\_ptr};\C{ first unused location in the
output state stack }\par
\fi

\M{33}To get the output process started, we will perform the following
initialization steps. We may assume that \PB{$\\{text\_info}\MG\\{text\_link}$}
is nonzero,
since it points to the \CEE/ text in the first unnamed section that generates
code; if there are no such sections, there is nothing to output, and an
error message will have been generated before we do any of the initialization.

\Y\B\4\X33:Initialize the output stacks\X${}\E{}$\6
$\\{stack\_ptr}\K\\{stack}+\T{1};{}$\6
${}\\{cur\_name}\K\\{name\_dir};{}$\6
${}\\{cur\_repl}\K\\{text\_info}\MG\\{text\_link}+\\{text\_info};{}$\6
${}\\{cur\_byte}\K\\{cur\_repl}\MG\\{tok\_start};{}$\6
${}\\{cur\_end}\K(\\{cur\_repl}+\T{1})\MG\\{tok\_start};{}$\6
${}\\{cur\_section}\K\T{0}{}$;\par
\U48.\fi

\M{34}When the replacement text for name \PB{\|p} is to be inserted into the
output,
the following subroutine is called to save the old level of output and get
the new one going.

We assume that the \CEE/ compiler can copy structures.

\Y\B\1\1\&{static} \&{void} \\{push\_level}(\C{ suspends the current level }\6
\&{name\_pointer} \|p)\2\2\6
${}\{{}$\1\6
\&{if} ${}(\\{stack\_ptr}\E\\{stack\_end}){}$\1\5
\\{overflow}(\.{"stack"});\2\6
${}{*}\\{stack\_ptr}\K\\{cur\_state};{}$\6
${}\\{stack\_ptr}\PP;{}$\6
\&{if} ${}(\|p\I\NULL){}$\5
${}\{{}$\C{ \PB{$\|p\E\NULL$} means we are in \PB{\\{output\_defs}} }\1\6
${}\\{cur\_name}\K\|p;{}$\6
${}\\{cur\_repl}\K{}$(\&{text\_pointer}) \|p${}\MG\\{equiv};{}$\6
${}\\{cur\_byte}\K\\{cur\_repl}\MG\\{tok\_start};{}$\6
${}\\{cur\_end}\K(\\{cur\_repl}+\T{1})\MG\\{tok\_start};{}$\6
${}\\{cur\_section}\K\T{0};{}$\6
\4${}\}{}$\2\6
\4${}\}{}$\2\par
\fi

\M{35}\B\X8:Predeclaration of procedures\X${}\mathrel+\E{}$\6
\&{static} \&{void} \\{push\_level}(\&{name\_pointer});\6
\&{static} \&{void} \\{pop\_level}(\&{boolean});\par
\fi

\M{36}When we come to the end of a replacement text, the \PB{\\{pop\_level}}
subroutine
does the right thing: It either moves to the continuation of this replacement
text or returns the state to the most recently stacked level.

\Y\B\1\1\&{static} \&{void} \\{pop\_level}(\C{ do this when \PB{\\{cur\_byte}}
reaches \PB{\\{cur\_end}} }\6
\&{boolean} \\{flag})\C{ \PB{$\\{flag}\E\\{false}$} means we are in \PB{%
\\{output\_defs}} }\2\2\6
${}\{{}$\1\6
\&{if} ${}(\\{flag}\W\\{cur\_repl}\MG\\{text\_link}<\\{section\_flag}){}$\5
${}\{{}$\C{ link to a continuation }\1\6
${}\\{cur\_repl}\K\\{cur\_repl}\MG\\{text\_link}+\\{text\_info}{}$;\C{ stay on
the same level }\6
${}\\{cur\_byte}\K\\{cur\_repl}\MG\\{tok\_start};{}$\6
${}\\{cur\_end}\K(\\{cur\_repl}+\T{1})\MG\\{tok\_start};{}$\6
\&{return};\6
\4${}\}{}$\2\6
${}\\{stack\_ptr}\MM{}$;\C{ go down to the previous level }\6
\&{if} ${}(\\{stack\_ptr}>\\{stack}){}$\1\5
${}\\{cur\_state}\K{*}\\{stack\_ptr};{}$\2\6
\4${}\}{}$\2\par
\fi

\M{37}The heart of the output procedure is the function \PB{\\{get\_output}},
which produces the next token of output and sends it on to the lower-level
function \PB{\\{out\_char}}. The main purpose of \PB{\\{get\_output}} is to
handle the
necessary stacking and unstacking. It sends the value \PB{\\{section\_number}}
if the next output begins or ends the replacement text of some section,
in which case \PB{\\{cur\_val}} is that section's number (if beginning) or the
negative of that value (if ending). (A section number of 0 indicates
not the beginning or ending of a section, but a \#\&{line} command.)
And it sends the value \PB{\\{identifier}}
if the next output is an identifier, in which case
\PB{\\{cur\_val}} points to that identifier name.

\Y\B\4\D\\{section\_number}\5
\T{\~201}\C{ code returned by \PB{\\{get\_output}} for section numbers }\par
\B\4\D\\{identifier}\5
\T{\~202}\C{ code returned by \PB{\\{get\_output}} for identifiers }\par
\Y\B\4\X20:Private variables\X${}\mathrel+\E{}$\6
\&{static} \&{int} \\{cur\_val};\C{ additional information corresponding to
output token }\par
\fi

\M{38}If \PB{\\{get\_output}} finds that no more output remains, it returns
with
\PB{$\\{stack\_ptr}\E\\{stack}$}.

\Y\B\1\1\&{static} \&{void} \\{get\_output}(\&{void})\C{ sends next token to %
\PB{\\{out\_char}} }\2\2\6
${}\{{}$\1\6
\&{sixteen\_bits} \|a;\C{ value of current byte }\7
\4\\{restart}:\6
\&{if} ${}(\\{stack\_ptr}\E\\{stack}){}$\1\5
\&{return};\2\6
\&{if} ${}(\\{cur\_byte}\E\\{cur\_end}){}$\5
${}\{{}$\1\6
${}\\{cur\_val}\K{-}{}$((\&{int}) \\{cur\_section});\C{ cast needed because of
sign extension }\6
\\{pop\_level}(\\{true});\6
\&{if} ${}(\\{cur\_val}\E\T{0}){}$\1\5
\&{goto} \\{restart};\2\6
\\{out\_char}(\\{section\_number});\6
\&{return};\6
\4${}\}{}$\2\6
${}\|a\K{*}\\{cur\_byte}\PP;{}$\6
\&{if} ${}(\\{out\_state}\E\\{verbatim}\W\|a\I\\{string}\W\|a\I\\{constant}\W%
\|a\I\.{'\\n'}){}$\1\5
\\{C\_putc}(\|a);\C{ a high-bit character can occur in a string }\2\6
\&{else} \&{if} ${}(\|a<\T{\~200}){}$\1\5
\\{out\_char}(\|a);\C{ one-byte token }\2\6
\&{else}\5
${}\{{}$\1\6
${}\|a\K(\|a-\T{\~200})*\T{\~400}+{*}\\{cur\_byte}\PP;{}$\6
\&{switch} ${}(\|a/\T{\~24000}){}$\5
${}\{{}$\C{ \PB{$\T{\~24000}\E(\T{\~250}-\T{\~200})*\T{\~400}$} }\1\6
\4\&{case} \T{0}:\5
${}\\{cur\_val}\K{}$(\&{int}) \|a;\6
\\{out\_char}(\\{identifier});\6
\&{break};\6
\4\&{case} \T{1}:\6
\&{if} ${}(\|a\E\\{output\_defs\_flag}){}$\1\5
\\{output\_defs}(\,);\2\6
\&{else}\1\5
\X40:Expand section \PB{$\|a-\T{\~24000}$}, \PB{\&{goto} \\{restart}}\X\2\6
\&{break};\6
\4\&{default}:\5
${}\\{cur\_val}\K{}$(\&{int}) \|a${}-\T{\~50000};{}$\6
\&{if} ${}(\\{cur\_val}>\T{0}){}$\1\5
${}\\{cur\_section}\K{}$(\&{sixteen\_bits}) \\{cur\_val};\2\6
\\{out\_char}(\\{section\_number});\6
\4${}\}{}$\2\6
\4${}\}{}$\2\6
\4${}\}{}$\2\par
\fi

\M{39}\B\X8:Predeclaration of procedures\X${}\mathrel+\E{}$\5
\&{static} \&{void} \\{get\_output}(\&{void});\par
\fi

\M{40}The user may have forgotten to give any \CEE/ text for a section name,
or the \CEE/ text may have been associated with a different name by mistake.

\Y\B\4\X40:Expand section \PB{$\|a-\T{\~24000}$}, \PB{\&{goto} \\{restart}}%
\X${}\E{}$\6
${}\{{}$\1\6
${}\|a\MRL{-{\K}}\T{\~24000};{}$\6
\&{if} ${}((\|a+\\{name\_dir})\MG\\{equiv}\I{}$(\&{void} ${}{*}){}$ \\{text%
\_info})\1\5
${}\\{push\_level}(\|a+\\{name\_dir});{}$\2\6
\&{else} \&{if} ${}(\|a\I\T{0}){}$\5
${}\{{}$\1\6
${}\\{fputs}(\.{"\\n!\ Not\ present:\ <"},\39\\{stdout});{}$\6
${}\\{print\_section\_name}(\|a+\\{name\_dir});{}$\6
\\{err\_print}(\.{">"});\6
\4${}\}{}$\2\6
\&{goto} \\{restart};\6
\4${}\}{}$\2\par
\U38.\fi

\N{1}{41}Producing the output.
The \PB{\\{get\_output}} routine above handles most of the complexity of output
generation, but there are two further considerations that have a nontrivial
effect on \.{CTANGLE}'s algorithms.

\fi

\M{42}First,
we want to make sure that the output has spaces and line breaks in
the right places (e.g., not in the middle of a string or a constant or an
identifier, not at a `\.{@\&}' position
where quantities are being joined together, and certainly after an \.=
because the \CEE/ compiler thinks \.{=-} is ambiguous).

The output process can be in one of following states:

\yskip\hang \PB{\\{num\_or\_id}} means that the last item in the buffer is a
number or
identifier, hence a blank space or line break must be inserted if the next
item is also a number or identifier.

\yskip\hang \PB{\\{unbreakable}} means that the last item in the buffer was
followed
by the \.{@\&} operation that inhibits spaces between it and the next item.

\yskip\hang \PB{\\{verbatim}} means we're copying only character tokens, and
that they are to be output exactly as stored.  This is the case during
strings, verbatim constructions and numerical constants.

\yskip\hang \PB{\\{post\_slash}} means we've just output a slash.

\yskip\hang \PB{\\{normal}} means none of the above.

\yskip\noindent Furthermore, if the variable \PB{\\{protect}} is \PB{\\{true}},
newlines
are preceded by a `\.\\'.

\Y\B\4\D\\{normal}\5
\T{0}\C{ non-unusual state }\par
\B\4\D\\{num\_or\_id}\5
\T{1}\C{ state associated with numbers and identifiers }\par
\B\4\D\\{post\_slash}\5
\T{2}\C{ state following a \./ }\par
\B\4\D\\{unbreakable}\5
\T{3}\C{ state associated with \.{@\&} }\par
\B\4\D\\{verbatim}\5
\T{4}\C{ state in the middle of a string }\par
\Y\B\4\X20:Private variables\X${}\mathrel+\E{}$\6
\&{static} \&{eight\_bits} \\{out\_state};\C{ current status of partial output
}\6
\&{static} \&{boolean} \\{protect};\C{ should newline characters be quoted? }%
\par
\fi

\M{43}Here is a routine that is invoked when we want to output the current
line.
During the output process, \PB{\\{cur\_line}} equals the number of the next
line
to be output.

\Y\B\1\1\&{static} \&{void} \\{flush\_buffer}(\&{void})\C{ writes one line to
output file }\2\2\6
${}\{{}$\1\6
\\{C\_putc}(\.{'\\n'});\6
\&{if} ${}(\\{cur\_line}\MOD\T{100}\E\T{0}\W\\{show\_progress}){}$\5
${}\{{}$\1\6
\\{putchar}(\.{'.'});\6
\&{if} ${}(\\{cur\_line}\MOD\T{500}\E\T{0}){}$\1\5
${}\\{printf}(\.{"\%d"},\39\\{cur\_line});{}$\2\6
\\{update\_terminal};\C{ progress report }\6
\4${}\}{}$\2\6
${}\\{cur\_line}\PP;{}$\6
\4${}\}{}$\2\par
\fi

\M{44}\B\X8:Predeclaration of procedures\X${}\mathrel+\E{}$\5
\&{static} \&{void} \\{flush\_buffer}(\&{void});\par
\fi

\M{45}Second, we have modified the original \.{TANGLE} so that it will write
output
on multiple files.
If a section name is introduced in at least one place by \.{@(}
instead of \.{@<}, we treat it as the name of a file.
All these special sections are saved on a stack, \PB{\\{output\_files}}.
We write them out after we've done the unnamed section.

\Y\B\4\D\\{max\_files}\5
\T{256}\par
\Y\B\4\X20:Private variables\X${}\mathrel+\E{}$\6
\&{static} \&{name\_pointer} \\{output\_files}[\\{max\_files}];\6
\&{static} \&{name\_pointer} ${}{*}\\{cur\_out\_file},{}$ ${}{*}\\{end\_output%
\_files},{}$ ${}{*}\\{an\_output\_file};{}$\6
\&{static} \&{char} \\{cur\_section\_name\_char};\C{ is it \PB{\.{'<'}} or \PB{%
\.{'('}} }\6
\&{static} \&{char} ${}\\{output\_file\_name}[\\{longest\_name}+\T{1}]{}$;\C{
name of the file }\par
\fi

\M{46}We make \PB{\\{end\_output\_files}} point just beyond the end of
\PB{\\{output\_files}}. The stack pointer
\PB{\\{cur\_out\_file}} starts out there. Every time we see a new file, we
decrement \PB{\\{cur\_out\_file}} and then write it in.
\Y\B\4\X21:Set initial values\X${}\mathrel+\E{}$\6
$\\{cur\_out\_file}\K\\{end\_output\_files}\K\\{output\_files}+\\{max%
\_files}{}$;\par
\fi

\M{47}\B\X47:If it's not there, add \PB{\\{cur\_section\_name}} to the output
file stack, or complain we're out of room\X${}\E{}$\6
${}\{{}$\1\6
\&{for} ${}(\\{an\_output\_file}\K\\{cur\_out\_file};{}$ ${}\\{an\_output%
\_file}<\\{end\_output\_files};{}$ ${}\\{an\_output\_file}\PP){}$\1\6
\&{if} ${}({*}\\{an\_output\_file}\E\\{cur\_section\_name}){}$\1\5
\&{break};\2\2\6
\&{if} ${}(\\{an\_output\_file}\E\\{end\_output\_files}){}$\5
${}\{{}$\1\6
\&{if} ${}(\\{cur\_out\_file}>\\{output\_files}){}$\1\5
${}{*}\MM\\{cur\_out\_file}\K\\{cur\_section\_name};{}$\2\6
\&{else}\1\5
\\{overflow}(\.{"output\ files"});\2\6
\4${}\}{}$\2\6
\4${}\}{}$\2\par
\U77.\fi

\N{1}{48}The big output switch.  Here then is the routine that does the
output.

\Y\B\1\1\&{static} \&{void} \\{phase\_two}(\&{void})\2\2\6
${}\{{}$\1\6
${}\\{phase}\K\T{2};{}$\6
${}\\{web\_file\_open}\K\\{false};{}$\6
${}\\{cur\_line}\K\T{1};{}$\6
\X33:Initialize the output stacks\X\6
\X51:Output macro definitions if appropriate\X\6
\&{if} ${}(\\{text\_info}\MG\\{text\_link}\E\\{macro}\W\\{cur\_out\_file}\E%
\\{end\_output\_files}){}$\5
${}\{{}$\1\6
${}\\{fputs}(\.{"\\n!\ No\ program\ text}\)\.{\ was\ specified."},\39%
\\{stdout});{}$\6
\\{mark\_harmless};\6
\4${}\}{}$\2\6
\&{else}\5
${}\{{}$\1\6
\&{if} ${}(\\{cur\_out\_file}\E\\{end\_output\_files}){}$\5
${}\{{}$\1\6
\&{if} (\\{show\_progress})\5
${}\{{}$\1\6
${}\\{printf}(\.{"\\nWriting\ the\ outpu}\)\.{t\ file\ (\%s):"},\39\\{C\_file%
\_name});{}$\6
\\{update\_terminal};\6
\4${}\}{}$\2\6
\4${}\}{}$\2\6
\&{else}\5
${}\{{}$\1\6
\&{if} (\\{show\_progress})\5
${}\{{}$\1\6
${}\\{fputs}(\.{"\\nWriting\ the\ outpu}\)\.{t\ files:"},\39\\{stdout});{}$\6
${}\\{printf}(\.{"\ (\%s)"},\39\\{C\_file\_name});{}$\6
\\{update\_terminal};\6
\4${}\}{}$\2\6
\&{if} ${}(\\{text\_info}\MG\\{text\_link}\E\\{macro}){}$\1\5
\&{goto} \\{writeloop};\2\6
\4${}\}{}$\2\6
\&{while} ${}(\\{stack\_ptr}>\\{stack}){}$\1\5
\\{get\_output}(\,);\2\6
\\{flush\_buffer}(\,);\6
\4\\{writeloop}:\5
\X50:Write all the named output files\X\6
\&{if} (\\{show\_happiness})\5
${}\{{}$\1\6
\&{if} (\\{show\_progress})\1\5
\\{new\_line};\2\6
${}\\{fputs}(\.{"Done."},\39\\{stdout});{}$\6
\4${}\}{}$\2\6
\4${}\}{}$\2\6
\4${}\}{}$\2\par
\fi

\M{49}\B\X8:Predeclaration of procedures\X${}\mathrel+\E{}$\5
\&{static} \&{void} \\{phase\_two}(\&{void});\par
\fi

\M{50}To write the named output files, we proceed as for the unnamed
section.
The only subtlety is that we have to open each one.

\Y\B\4\X50:Write all the named output files\X${}\E{}$\6
\&{for} ${}(\\{an\_output\_file}\K\\{end\_output\_files};{}$ ${}\\{an\_output%
\_file}>\\{cur\_out\_file};{}$ \,)\5
${}\{{}$\1\6
${}\\{an\_output\_file}\MM;{}$\6
${}\\{sprint\_section\_name}(\\{output\_file\_name},\39{*}\\{an\_output%
\_file});{}$\6
\\{fclose}(\\{C\_file});\6
\&{if} ${}((\\{C\_file}\K\\{fopen}(\\{output\_file\_name},\39\.{"wb"}))\E%
\NULL){}$\1\5
${}\\{fatal}(\.{"!\ Cannot\ open\ outpu}\)\.{t\ file\ "},\39\\{output\_file%
\_name});{}$\2\6
\&{if} (\\{show\_progress})\5
${}\{{}$\1\6
${}\\{printf}(\.{"\\n(\%s)"},\39\\{output\_file\_name});{}$\6
\\{update\_terminal};\6
\4${}\}{}$\2\6
${}\\{cur\_line}\K\T{1};{}$\6
${}\\{stack\_ptr}\K\\{stack}+\T{1};{}$\6
${}\\{cur\_name}\K{*}\\{an\_output\_file};{}$\6
${}\\{cur\_repl}\K{}$(\&{text\_pointer}) \\{cur\_name}${}\MG\\{equiv};{}$\6
${}\\{cur\_byte}\K\\{cur\_repl}\MG\\{tok\_start};{}$\6
${}\\{cur\_end}\K(\\{cur\_repl}+\T{1})\MG\\{tok\_start};{}$\6
\&{while} ${}(\\{stack\_ptr}>\\{stack}){}$\1\5
\\{get\_output}(\,);\2\6
\\{flush\_buffer}(\,);\6
\4${}\}{}$\2\par
\U48.\fi

\M{51}If a \.{@h} was not encountered in the input,
we go through the list of replacement texts and copy the ones
that refer to macros, preceded by the \.{\#define} preprocessor command.

\Y\B\4\X51:Output macro definitions if appropriate\X${}\E{}$\6
\&{if} ${}(\R\\{output\_defs\_seen}){}$\1\5
\\{output\_defs}(\,);\2\par
\U48.\fi

\M{52}\B\X20:Private variables\X${}\mathrel+\E{}$\6
\&{static} \&{boolean} \\{output\_defs\_seen}${}\K\\{false}{}$;\par
\fi

\M{53}\B\X8:Predeclaration of procedures\X${}\mathrel+\E{}$\6
\&{static} \&{void} \\{output\_defs}(\&{void});\6
\&{static} \&{void} \\{out\_char}(\&{eight\_bits});\par
\fi

\M{54}\B\D\\{C\_printf}$(\|c,\|a)$\5
$\\{fprintf}(\\{C\_file},\39\|c,\39\|a{}$)\par
\B\4\D\\{C\_putc}$(\|c)$\5
$\\{putc}((\&{int})(\|c),\39\\{C\_file}{}$)\C{ isn't \CEE/ wonderfully
consistent? }\par
\Y\B\1\1\&{static} \&{void} \\{output\_defs}(\&{void})\2\2\6
${}\{{}$\1\6
\&{sixteen\_bits} \|a;\7
${}\\{push\_level}(\NULL);{}$\6
\&{for} ${}(\\{cur\_text}\K\\{text\_info}+\T{1};{}$ ${}\\{cur\_text}<\\{text%
\_ptr};{}$ ${}\\{cur\_text}\PP){}$\1\6
\&{if} ${}(\\{cur\_text}\MG\\{text\_link}\E\\{macro}){}$\5
${}\{{}$\C{ \PB{\\{cur\_text}} is the text for a \PB{\\{macro}} }\1\6
${}\\{cur\_byte}\K\\{cur\_text}\MG\\{tok\_start};{}$\6
${}\\{cur\_end}\K(\\{cur\_text}+\T{1})\MG\\{tok\_start};{}$\6
${}\\{C\_printf}(\.{"\%s"},\39\.{"\#define\ "});{}$\6
${}\\{out\_state}\K\\{normal};{}$\6
${}\\{protect}\K\\{true}{}$;\C{ newlines should be preceded by \PB{\.{'\\\\'}}
}\6
\&{while} ${}(\\{cur\_byte}<\\{cur\_end}){}$\5
${}\{{}$\1\6
${}\|a\K{*}\\{cur\_byte}\PP;{}$\6
\&{if} ${}(\\{cur\_byte}\E\\{cur\_end}\W\|a\E\.{'\\n'}){}$\1\5
\&{break};\C{ disregard a final newline }\2\6
\&{if} ${}(\\{out\_state}\E\\{verbatim}\W\|a\I\\{string}\W\|a\I\\{constant}\W%
\|a\I\.{'\\n'}){}$\1\5
\\{C\_putc}(\|a);\C{ a high-bit character can occur in a string }\2\6
\&{else} \&{if} ${}(\|a<\T{\~200}){}$\1\5
\\{out\_char}(\|a);\C{ one-byte token }\2\6
\&{else}\5
${}\{{}$\1\6
${}\|a\K(\|a-\T{\~200})*\T{\~400}+{*}\\{cur\_byte}\PP;{}$\6
\&{if} ${}(\|a<\T{\~24000}){}$\5
${}\{{}$\C{ \PB{$\T{\~24000}\E(\T{\~250}-\T{\~200})*\T{\~400}$} }\1\6
${}\\{cur\_val}\K{}$(\&{int}) \|a;\6
\\{out\_char}(\\{identifier});\6
\4${}\}{}$\2\6
\&{else} \&{if} ${}(\|a<\T{\~50000}){}$\1\5
\\{confusion}(\.{"macro\ defs\ have\ str}\)\.{ange\ char"});\2\6
\&{else}\5
${}\{{}$\1\6
${}\\{cur\_val}\K{}$(\&{int}) \|a${}-\T{\~50000};{}$\6
${}\\{cur\_section}\K{}$(\&{sixteen\_bits}) \\{cur\_val};\6
\\{out\_char}(\\{section\_number});\6
\4${}\}{}$\C{ no other cases }\2\6
\4${}\}{}$\2\6
\4${}\}{}$\2\6
${}\\{protect}\K\\{false};{}$\6
\\{flush\_buffer}(\,);\6
\4${}\}{}$\2\2\6
\\{pop\_level}(\\{false});\6
\4${}\}{}$\2\par
\fi

\M{55}A many-way switch is used to send the output.  Note that this function
is not called if \PB{$\\{out\_state}\E\\{verbatim}$}, except perhaps with
arguments
\PB{\.{'\\n'}} (protect the newline), \PB{\\{string}} (end the string), or \PB{%
\\{constant}}
(end the constant).

\Y\B\1\1\&{static} \&{void} \\{out\_char}(\&{eight\_bits} \\{cur\_char})\2\2\6
${}\{{}$\1\6
\&{char} ${}{*}\|j,{}$ ${}{*}\|k{}$;\C{ pointer into \PB{\\{byte\_mem}} }\7
\4\\{restart}:\6
\&{switch} (\\{cur\_char})\5
${}\{{}$\1\6
\4\&{case} \.{'\\n'}:\6
\&{if} ${}(\\{protect}\W\\{out\_state}\I\\{verbatim}){}$\1\5
\\{C\_putc}(\.{'\ '});\2\6
\&{if} ${}(\\{protect}\V\\{out\_state}\E\\{verbatim}){}$\1\5
\\{C\_putc}(\.{'\\\\'});\2\6
\\{flush\_buffer}(\,);\6
\&{if} ${}(\\{out\_state}\I\\{verbatim}){}$\1\5
${}\\{out\_state}\K\\{normal};{}$\2\6
\&{break};\6
\hbox{\4}\X59:Case of an identifier\X\6
\hbox{\4}\X60:Case of a section number\X\6
\hbox{\4}\X56:Cases like \.{!=}\X\6
\4\&{case} \.{'='}:\5
\&{case} \.{'>'}:\5
\\{C\_putc}(\\{cur\_char});\5
\\{C\_putc}(\.{'\ '});\6
${}\\{out\_state}\K\\{normal};{}$\6
\&{break};\6
\4\&{case} \\{join}:\5
${}\\{out\_state}\K\\{unbreakable};{}$\6
\&{break};\6
\4\&{case} \\{constant}:\6
\&{if} ${}(\\{out\_state}\E\\{verbatim}){}$\5
${}\{{}$\1\6
${}\\{out\_state}\K\\{num\_or\_id};{}$\6
\&{break};\6
\4${}\}{}$\2\6
\&{if} ${}(\\{out\_state}\E\\{num\_or\_id}){}$\1\5
\\{C\_putc}(\.{'\ '});\2\6
${}\\{out\_state}\K\\{verbatim};{}$\6
\&{break};\6
\4\&{case} \\{string}:\6
\&{if} ${}(\\{out\_state}\E\\{verbatim}){}$\1\5
${}\\{out\_state}\K\\{normal};{}$\2\6
\&{else}\1\5
${}\\{out\_state}\K\\{verbatim};{}$\2\6
\&{break};\6
\4\&{case} \.{'/'}:\5
\\{C\_putc}(\.{'/'});\6
${}\\{out\_state}\K\\{post\_slash};{}$\6
\&{break};\6
\4\&{case} \.{'*'}:\6
\&{if} ${}(\\{out\_state}\E\\{post\_slash}){}$\1\5
\\{C\_putc}(\.{'\ '});\C{ fall through }\2\6
\4\&{default}:\5
\\{C\_putc}(\\{cur\_char});\6
${}\\{out\_state}\K\\{normal};{}$\6
\&{break};\6
\4${}\}{}$\2\6
\4${}\}{}$\2\par
\fi

\M{56}\B\X56:Cases like \.{!=}\X${}\E{}$\6
\4\hbox{\1\quad}\&{case} \\{plus\_plus}:\5
\\{C\_putc}(\.{'+'});\5
\\{C\_putc}(\.{'+'});\6
${}\\{out\_state}\K\\{normal};{}$\6
\&{break};\6
\4\&{case} \\{minus\_minus}:\5
\\{C\_putc}(\.{'-'});\5
\\{C\_putc}(\.{'-'});\6
${}\\{out\_state}\K\\{normal};{}$\6
\&{break};\6
\4\&{case} \\{minus\_gt}:\5
\\{C\_putc}(\.{'-'});\5
\\{C\_putc}(\.{'>'});\6
${}\\{out\_state}\K\\{normal};{}$\6
\&{break};\6
\4\&{case} \\{gt\_gt}:\5
\\{C\_putc}(\.{'>'});\5
\\{C\_putc}(\.{'>'});\6
${}\\{out\_state}\K\\{normal};{}$\6
\&{break};\6
\4\&{case} \\{eq\_eq}:\5
\\{C\_putc}(\.{'='});\5
\\{C\_putc}(\.{'='});\6
${}\\{out\_state}\K\\{normal};{}$\6
\&{break};\6
\4\&{case} \\{lt\_lt}:\5
\\{C\_putc}(\.{'<'});\5
\\{C\_putc}(\.{'<'});\6
${}\\{out\_state}\K\\{normal};{}$\6
\&{break};\6
\4\&{case} \\{gt\_eq}:\5
\\{C\_putc}(\.{'>'});\5
\\{C\_putc}(\.{'='});\6
${}\\{out\_state}\K\\{normal};{}$\6
\&{break};\6
\4\&{case} \\{lt\_eq}:\5
\\{C\_putc}(\.{'<'});\5
\\{C\_putc}(\.{'='});\6
${}\\{out\_state}\K\\{normal};{}$\6
\&{break};\6
\4\&{case} \\{non\_eq}:\5
\\{C\_putc}(\.{'!'});\5
\\{C\_putc}(\.{'='});\6
${}\\{out\_state}\K\\{normal};{}$\6
\&{break};\6
\4\&{case} \\{and\_and}:\5
\\{C\_putc}(\.{'\&'});\5
\\{C\_putc}(\.{'\&'});\6
${}\\{out\_state}\K\\{normal};{}$\6
\&{break};\6
\4\&{case} \\{or\_or}:\5
\\{C\_putc}(\.{'|'});\5
\\{C\_putc}(\.{'|'});\6
${}\\{out\_state}\K\\{normal};{}$\6
\&{break};\6
\4\&{case} \\{dot\_dot\_dot}:\5
\\{C\_putc}(\.{'.'});\5
\\{C\_putc}(\.{'.'});\5
\\{C\_putc}(\.{'.'});\6
${}\\{out\_state}\K\\{normal};{}$\6
\&{break};\6
\4\&{case} \\{colon\_colon}:\5
\\{C\_putc}(\.{':'});\5
\\{C\_putc}(\.{':'});\6
${}\\{out\_state}\K\\{normal};{}$\6
\&{break};\6
\4\&{case} \\{period\_ast}:\5
\\{C\_putc}(\.{'.'});\5
\\{C\_putc}(\.{'*'});\6
${}\\{out\_state}\K\\{normal};{}$\6
\&{break};\6
\4\&{case} \\{minus\_gt\_ast}:\5
\\{C\_putc}(\.{'-'});\5
\\{C\_putc}(\.{'>'});\5
\\{C\_putc}(\.{'*'});\6
${}\\{out\_state}\K\\{normal};{}$\6
\&{break};\par
\U55.\fi

\M{57}When an identifier is output to the \CEE/ file, characters in the
range 128--255 must be changed into something else, so the \CEE/
compiler won't complain.  By default, \.{CTANGLE} converts the
character with code $16 x+y$ to the three characters `\.X$xy$', but
a different transliteration table can be specified.  Thus a German
might want {\it gr\"un\/} to appear as a still readable \.{gruen}.
This makes debugging a lot less confusing.

\Y\B\4\D\\{translit\_length}\5
\T{10}\par
\Y\B\4\X20:Private variables\X${}\mathrel+\E{}$\6
\&{static} \&{char} \\{translit}[\T{128}][\\{translit\_length}];\par
\fi

\M{58}\B\X21:Set initial values\X${}\mathrel+\E{}$\6
${}\{{}$\1\6
\&{int} \|i;\7
\&{for} ${}(\|i\K\T{0};{}$ ${}\|i<\T{128};{}$ ${}\|i\PP){}$\1\5
${}\\{sprintf}(\\{translit}[\|i],\39\.{"X\%02X"},\39{}$(\&{unsigned} %
\&{int})(${}\T{128}+\|i));{}$\2\6
\4${}\}{}$\2\par
\fi

\M{59}\B\X59:Case of an identifier\X${}\E{}$\6
\4\hbox{\1\quad}\&{case} \\{identifier}:\6
\&{if} ${}(\\{out\_state}\E\\{num\_or\_id}){}$\1\5
\\{C\_putc}(\.{'\ '});\2\6
\&{for} ${}(\|j\K(\\{cur\_val}+\\{name\_dir})\MG\\{byte\_start},\39\|k\K(\\{cur%
\_val}+\\{name\_dir}+\T{1})\MG\\{byte\_start};{}$ ${}\|j<\|k;{}$ ${}\|j\PP){}$%
\1\6
\&{if} ${}((\&{eight\_bits})({*}\|j)<\T{\~200}){}$\1\5
${}\\{C\_putc}({*}\|j);{}$\2\6
\&{else}\1\5
${}\\{C\_printf}(\.{"\%s"},\39\\{translit}[(\&{eight\_bits})({*}\|j)-\T{%
\~200}]);{}$\2\2\6
${}\\{out\_state}\K\\{num\_or\_id};{}$\6
\&{break};\par
\U55.\fi

\M{60}\B\X60:Case of a section number\X${}\E{}$\6
\4\hbox{\1\quad}\&{case} \\{section\_number}:\6
\&{if} ${}(\\{cur\_val}>\T{0}){}$\1\5
${}\\{C\_printf}(\.{"/*\%d:*/"},\39\\{cur\_val});{}$\2\6
\&{else} \&{if} ${}(\\{cur\_val}<\T{0}){}$\1\5
${}\\{C\_printf}(\.{"/*:\%d*/"},\39{-}\\{cur\_val});{}$\2\6
\&{else} \&{if} (\\{protect})\5
${}\{{}$\1\6
${}\\{cur\_byte}\MRL{+{\K}}\T{4}{}$;\C{ skip line number and file name }\6
${}\\{cur\_char}\K{}$(\&{eight\_bits}) \.{'\\n'};\6
\&{goto} \\{restart};\6
\4${}\}{}$\2\6
\&{else}\5
${}\{{}$\1\6
\&{sixteen\_bits} \|a;\7
${}\|a\K{*}\\{cur\_byte}\PP*\T{\~400};{}$\6
${}\|a\MRL{+{\K}}{*}\\{cur\_byte}\PP{}$;\C{ gets the line number }\6
${}\\{C\_printf}(\.{"\\n\#line\ \%d\ \\""},\39{}$(\&{int}) \|a);\6
${}\\{cur\_val}\K(\&{int})({*}\\{cur\_byte}\PP-\T{\~200})*\T{\~400};{}$\6
${}\\{cur\_val}\MRL{+{\K}}{*}\\{cur\_byte}\PP{}$;\C{ points to the file name }\6
\&{for} ${}(\|j\K(\\{cur\_val}+\\{name\_dir})\MG\\{byte\_start},\39\|k\K(\\{cur%
\_val}+\\{name\_dir}+\T{1})\MG\\{byte\_start};{}$ ${}\|j<\|k;{}$ ${}\|j\PP){}$\5
${}\{{}$\1\6
\&{if} ${}({*}\|j\E\.{'\\\\'}\V{*}\|j\E\.{'"'}){}$\1\5
\\{C\_putc}(\.{'\\\\'});\2\6
${}\\{C\_putc}({*}\|j);{}$\6
\4${}\}{}$\2\6
\\{C\_putc}(\.{'"'});\5
\\{C\_putc}(\.{'\\n'});\6
\4${}\}{}$\2\6
\&{break};\par
\U55.\fi

\N{0}{61}Introduction to the input phase.
We have now seen that \.{CTANGLE} will be able to output the full
\CEE/ program, if we can only get that program into the byte memory in
the proper format. The input process is something like the output process
in reverse, since we compress the text as we read it in and we expand it
as we write it out.

There are three main input routines. The most interesting is the one that gets
the next token of a \CEE/ text; the other two are used to scan rapidly past
\TEX/ text in the \.{CWEB} source code. One of the latter routines will jump to
the next token that starts with `\.{@}', and the other skips to the end
of a \CEE/ comment.

\fi

\M{62}Control codes in \.{CWEB} begin with `\.{@}', and the next character
identifies the code. Some of these are of interest only to \.{CWEAVE},
so \.{CTANGLE} ignores them; the others are converted by \.{CTANGLE} into
internal code numbers by the \PB{\\{ccode}} table below. The ordering
of these internal code numbers has been chosen to simplify the program logic;
larger numbers are given to the control codes that denote more significant
milestones.

\Y\B\4\D\\{ignore}\5
\T{\~0}\C{ control code of no interest to \.{CTANGLE} }\par
\B\4\D\\{ord}\5
\T{\~302}\C{ control code for `\.{@'}' }\par
\B\4\D\\{control\_text}\5
\T{\~303}\C{ control code for `\.{@t}', `\.{@\^}', etc. }\par
\B\4\D\\{translit\_code}\5
\T{\~304}\C{ control code for `\.{@l}' }\par
\B\4\D\\{output\_defs\_code}\5
\T{\~305}\C{ control code for `\.{@h}' }\par
\B\4\D\\{format\_code}\5
\T{\~306}\C{ control code for `\.{@f}' }\par
\B\4\D\\{definition}\5
\T{\~307}\C{ control code for `\.{@d}' }\par
\B\4\D\\{begin\_C}\5
\T{\~310}\C{ control code for `\.{@c}' }\par
\B\4\D\\{section\_name}\5
\T{\~311}\C{ control code for `\.{@<}' }\par
\B\4\D\\{new\_section}\5
\T{\~312}\C{ control code for `\.{@\ }' and `\.{@*}' }\par
\Y\B\4\X20:Private variables\X${}\mathrel+\E{}$\6
\&{static} \&{eight\_bits} \\{ccode}[\T{256}];\C{ meaning of a char following %
\.{@} }\par
\fi

\M{63}\B\X21:Set initial values\X${}\mathrel+\E{}$\6
${}\{{}$\1\6
\&{int} \|c;\C{ must be \PB{\&{int}} so the \PB{\&{for}} loop will end }\7
\&{for} ${}(\|c\K\T{0};{}$ ${}\|c<\T{256};{}$ ${}\|c\PP){}$\1\5
${}\\{ccode}[\|c]\K\\{ignore};{}$\2\6
\4${}\}{}$\2\6
${}\\{ccode}[\.{'\ '}]\K\\{ccode}[\.{'\\t'}]\K\\{ccode}[\.{'\\n'}]\K\\{ccode}[%
\.{'\\v'}]\K\\{ccode}[\.{'\\r'}]\K\\{ccode}[\.{'\\f'}]\K\\{ccode}[\.{'*'}]\K%
\\{new\_section};{}$\6
${}\\{ccode}[\.{'@'}]\K{}$(\&{eight\_bits}) \.{'@'};\6
${}\\{ccode}[\.{'='}]\K\\{string};{}$\6
${}\\{ccode}[\.{'d'}]\K\\{ccode}[\.{'D'}]\K\\{definition};{}$\6
${}\\{ccode}[\.{'f'}]\K\\{ccode}[\.{'F'}]\K\\{ccode}[\.{'s'}]\K\\{ccode}[%
\.{'S'}]\K\\{format\_code};{}$\6
${}\\{ccode}[\.{'c'}]\K\\{ccode}[\.{'C'}]\K\\{ccode}[\.{'p'}]\K\\{ccode}[%
\.{'P'}]\K\\{begin\_C};{}$\6
${}\\{ccode}[\.{'\^'}]\K\\{ccode}[\.{':'}]\K\\{ccode}[\.{'.'}]\K\\{ccode}[%
\.{'t'}]\K\\{ccode}[\.{'T'}]\K\\{ccode}[\.{'q'}]\K\\{ccode}[\.{'Q'}]\K%
\\{control\_text};{}$\6
${}\\{ccode}[\.{'h'}]\K\\{ccode}[\.{'H'}]\K\\{output\_defs\_code};{}$\6
${}\\{ccode}[\.{'l'}]\K\\{ccode}[\.{'L'}]\K\\{translit\_code};{}$\6
${}\\{ccode}[\.{'\&'}]\K\\{join};{}$\6
${}\\{ccode}[\.{'<'}]\K\\{ccode}[\.{'('}]\K\\{section\_name};{}$\6
${}\\{ccode}[\.{'\\''}]\K\\{ord}{}$;\par
\fi

\M{64}The \PB{\\{skip\_ahead}} procedure reads through the input at fairly high
speed
until finding the next non-ignorable control code, which it returns.

\Y\B\1\1\&{static} \&{eight\_bits} \\{skip\_ahead}(\&{void})\C{ skip to next
control code }\2\2\6
${}\{{}$\1\6
\&{eight\_bits} \|c;\C{ control code found }\7
\&{while} (\\{true})\5
${}\{{}$\1\6
\&{if} ${}(\\{loc}>\\{limit}\W(\\{get\_line}(\,)\E\\{false})){}$\1\5
\&{return} \\{new\_section};\2\6
${}{*}(\\{limit}+\T{1})\K\.{'@'};{}$\6
\&{while} ${}({*}\\{loc}\I\.{'@'}){}$\1\5
${}\\{loc}\PP;{}$\2\6
\&{if} ${}(\\{loc}\Z\\{limit}){}$\5
${}\{{}$\1\6
${}\\{loc}\PP;{}$\6
${}\|c\K\\{ccode}{}$[(\&{eight\_bits}) ${}{*}\\{loc}];{}$\6
${}\\{loc}\PP;{}$\6
\&{if} ${}(\|c\I\\{ignore}\V{*}(\\{loc}-\T{1})\E\.{'>'}){}$\1\5
\&{return} \|c;\2\6
\4${}\}{}$\2\6
\4${}\}{}$\2\6
\4${}\}{}$\2\par
\fi

\M{65}\B\X8:Predeclaration of procedures\X${}\mathrel+\E{}$\6
\&{static} \&{eight\_bits} \\{skip\_ahead}(\&{void});\6
\&{static} \&{boolean} \\{skip\_comment}(\&{boolean});\par
\fi

\M{66}The \PB{\\{skip\_comment}} procedure reads through the input at somewhat
high
speed in order to pass over comments, which \.{CTANGLE} does not transmit
to the output. If the comment is introduced by \.{/*}, \PB{\\{skip\_comment}}
proceeds until finding the end-comment token \.{*/} or a newline; in the
latter case \PB{\\{skip\_comment}} will be called again by \PB{\\{get\_next}},
since the
comment is not finished.  This is done so that each newline in the
\CEE/ part of a section is copied to the output; otherwise the \#\&{line}
commands inserted into the \CEE/ file by the output routines become useless.
On the other hand, if the comment is introduced by \.{//} (i.e., if it
is a \CPLUSPLUS/ ``short comment''), it always is simply delimited by the next
newline. The boolean argument \PB{\\{is\_long\_comment}} distinguishes between
the two types of comments.

If \PB{\\{skip\_comment}} comes to the end of the section, it prints an error
message.
No comment, long or short, is allowed to contain `\.{@\ }' or `\.{@*}'.

\Y\B\4\X20:Private variables\X${}\mathrel+\E{}$\6
\&{static} \&{boolean} \\{comment\_continues}${}\K\\{false}{}$;\C{ are we
scanning a comment? }\par
\fi

\M{67}\B\1\1\&{static} \&{boolean} \\{skip\_comment}(\C{ skips over comments }\6
\&{boolean} \\{is\_long\_comment})\2\2\6
${}\{{}$\1\6
\&{char} \|c;\C{ current character }\7
\&{while} (\\{true})\5
${}\{{}$\1\6
\&{if} ${}(\\{loc}>\\{limit}){}$\5
${}\{{}$\1\6
\&{if} (\\{is\_long\_comment})\5
${}\{{}$\1\6
\&{if} (\\{get\_line}(\,))\1\5
\&{return} \\{comment\_continues}${}\K\\{true};{}$\2\6
\&{else}\5
${}\{{}$\1\6
\\{err\_print}(\.{"!\ Input\ ended\ in\ mi}\)\.{d-comment"});\6
\&{return} \\{comment\_continues}${}\K\\{false};{}$\6
\4${}\}{}$\2\6
\4${}\}{}$\2\6
\&{else}\1\5
\&{return} \\{comment\_continues}${}\K\\{false};{}$\2\6
\4${}\}{}$\2\6
${}\|c\K{*}(\\{loc}\PP);{}$\6
\&{if} ${}(\\{is\_long\_comment}\W\|c\E\.{'*'}\W{*}\\{loc}\E\.{'/'}){}$\5
${}\{{}$\1\6
${}\\{loc}\PP;{}$\6
\&{return} \\{comment\_continues}${}\K\\{false};{}$\6
\4${}\}{}$\2\6
\&{if} ${}(\|c\E\.{'@'}){}$\5
${}\{{}$\1\6
\&{if} (\\{ccode}[(\&{eight\_bits}) ${}{*}\\{loc}]\E\\{new\_section}){}$\5
${}\{{}$\1\6
\\{err\_print}(\.{"!\ Section\ name\ ende}\)\.{d\ in\ mid-comment"});\6
${}\\{loc}\MM;{}$\6
\&{return} \\{comment\_continues}${}\K\\{false};{}$\6
\4${}\}{}$\2\6
\&{else}\1\5
${}\\{loc}\PP;{}$\2\6
\4${}\}{}$\2\6
\4${}\}{}$\2\6
\4${}\}{}$\2\par
\fi

\N{1}{68}Inputting the next token.

\Y\B\4\X20:Private variables\X${}\mathrel+\E{}$\6
\&{static} \&{name\_pointer} \\{cur\_section\_name};\C{ name of section just
scanned }\6
\&{static} \&{boolean} \\{no\_where};\C{ suppress \PB{\\{print\_where}}? }\par
\fi

\M{69}As one might expect, \PB{\\{get\_next}} consists mostly of a big switch
that branches to the various special cases that can arise.

\Y\B\1\1\&{static} \&{eight\_bits} \\{get\_next}(\&{void})\C{ produces the next
input token }\2\2\6
${}\{{}$\1\6
\&{static} \&{boolean} \\{preprocessing}${}\K\\{false};{}$\6
\&{eight\_bits} \|c;\C{ the current character }\7
\&{while} (\\{true})\5
${}\{{}$\1\6
\&{if} ${}(\\{loc}>\\{limit}){}$\5
${}\{{}$\1\6
\&{if} ${}(\\{preprocessing}\W{*}(\\{limit}-\T{1})\I\.{'\\\\'}){}$\1\5
${}\\{preprocessing}\K\\{false};{}$\2\6
\&{if} ${}(\\{get\_line}(\,)\E\\{false}){}$\1\5
\&{return} \\{new\_section};\2\6
\&{else} \&{if} ${}(\\{print\_where}\W\R\\{no\_where}){}$\5
${}\{{}$\1\6
${}\\{print\_where}\K\\{false};{}$\6
\X85:Insert the line number into \PB{\\{tok\_mem}}\X\6
\4${}\}{}$\2\6
\&{else}\1\5
\&{return} (\&{eight\_bits}) \.{'\\n'};\2\6
\4${}\}{}$\2\6
${}\|c\K{}$(\&{eight\_bits}) ${}{*}\\{loc};{}$\6
\&{if} ${}(\\{comment\_continues}\V(\|c\E\.{'/'}\W({*}(\\{loc}+\T{1})\E\.{'*'}%
\V{*}(\\{loc}+\T{1})\E\.{'/'}))){}$\5
${}\{{}$\1\6
\&{if} ${}(\\{skip\_comment}(\\{comment\_continues}\V{*}(\\{loc}+\T{1})\E%
\.{'*'})){}$\1\5
\&{return} \.{'\\n'};\C{ scan to end of comment or newline }\2\6
\&{else}\1\5
\&{continue};\2\6
\4${}\}{}$\2\6
${}\\{loc}\PP;{}$\6
\&{if} ${}(\\{xisdigit}(\|c)\V\|c\E\.{'.'}){}$\1\5
\X73:Get a constant\X\2\6
\&{else} \&{if} ${}(\|c\E\.{'\\''}\V\|c\E\.{'"'}\3{-1}\V((\|c\E\.{'L'}\V\|c\E%
\.{'u'}\V\|c\E\.{'U'})\W({*}\\{loc}\E\.{'\\''}\V{*}\\{loc}\E\.{'"'}))\3{-1}\V((%
\|c\E\.{'u'}\W{*}\\{loc}\E\.{'8'})\W({*}(\\{loc}+\T{1})\E\.{'\\''}\V{*}(%
\\{loc}+\T{1})\E\.{'"'}))){}$\1\5
\X74:Get a string\X\2\6
\&{else} \&{if} (\\{isalpha}((\&{int}) \|c)${}\V\\{isxalpha}(\|c)\V\\{ishigh}(%
\|c)){}$\1\5
\X72:Get an identifier\X\2\6
\&{else} \&{if} ${}(\|c\E\.{'@'}){}$\1\5
\X75:Get control code and possible section name\X\2\6
\&{else} \&{if} (\\{xisspace}(\|c))\5
${}\{{}$\1\6
\&{if} ${}(\R\\{preprocessing}\V\\{loc}>\\{limit}){}$\1\5
\&{continue};\C{ we don't want a blank after a final backslash }\2\6
\&{else}\1\5
\&{return} (\&{eight\_bits}) \.{'\ '};\C{ ignore spaces and tabs, unless \PB{%
\\{preprocessing}} }\2\6
\4${}\}{}$\2\6
\&{else} \&{if} ${}(\|c\E\.{'\#'}\W\\{loc}\E\\{buffer}+\T{1}){}$\1\5
${}\\{preprocessing}\K\\{true};{}$\2\6
\4\\{mistake}:\5
\X71:Compress two-symbol operator\X\6
\&{return} \|c;\6
\4${}\}{}$\2\6
\4${}\}{}$\2\par
\fi

\M{70}\B\X8:Predeclaration of procedures\X${}\mathrel+\E{}$\5
\&{static} \&{eight\_bits} \\{get\_next}(\&{void});\par
\fi

\M{71}The following code assigns values to the combinations \.{++},
\.{--}, \.{->}, \.{>=}, \.{<=}, \.{==}, \.{<<}, \.{>>}, \.{!=}, %\.{\PB{}}
\.{\v\v} and~\.{\&\&}, and to the \CPLUSPLUS/
combinations \.{...}, \.{::}, \.{.*} and \.{->*}.
The compound assignment operators (e.g., \.{+=}) are
treated as separate tokens.

\Y\B\4\X71:Compress two-symbol operator\X${}\E{}$\6
\&{switch} (\|c)\5
${}\{{}$\1\6
\4\&{case} \.{'+'}:\6
\&{if} ${}({*}\\{loc}\E\.{'+'}){}$\1\5
\\{compress}(\\{plus\_plus});\2\6
\&{break};\6
\4\&{case} \.{'-'}:\6
\&{if} ${}({*}\\{loc}\E\.{'-'}){}$\5
${}\{{}$\5
\1\\{compress}(\\{minus\_minus});\5
${}\}{}$\2\6
\&{else} \&{if} ${}({*}\\{loc}\E\.{'>'}){}$\5
${}\{{}$\1\6
\&{if} ${}({*}(\\{loc}+\T{1})\E\.{'*'}){}$\5
${}\{{}$\1\6
${}\\{loc}\PP{}$;\5
\\{compress}(\\{minus\_gt\_ast});\6
\4${}\}{}$\2\6
\&{else}\1\5
\\{compress}(\\{minus\_gt});\2\6
\4${}\}{}$\2\6
\&{break};\6
\4\&{case} \.{'.'}:\6
\&{if} ${}({*}\\{loc}\E\.{'*'}){}$\5
${}\{{}$\5
\1\\{compress}(\\{period\_ast});\5
${}\}{}$\2\6
\&{else} \&{if} ${}({*}\\{loc}\E\.{'.'}\W{*}(\\{loc}+\T{1})\E\.{'.'}){}$\5
${}\{{}$\1\6
${}\\{loc}\PP{}$;\5
\\{compress}(\\{dot\_dot\_dot});\6
\4${}\}{}$\2\6
\&{break};\6
\4\&{case} \.{':'}:\6
\&{if} ${}({*}\\{loc}\E\.{':'}){}$\1\5
\\{compress}(\\{colon\_colon});\2\6
\&{break};\6
\4\&{case} \.{'='}:\6
\&{if} ${}({*}\\{loc}\E\.{'='}){}$\1\5
\\{compress}(\\{eq\_eq});\2\6
\&{break};\6
\4\&{case} \.{'>'}:\6
\&{if} ${}({*}\\{loc}\E\.{'='}){}$\5
${}\{{}$\5
\1\\{compress}(\\{gt\_eq});\5
${}\}{}$\2\6
\&{else} \&{if} ${}({*}\\{loc}\E\.{'>'}){}$\1\5
\\{compress}(\\{gt\_gt});\2\6
\&{break};\6
\4\&{case} \.{'<'}:\6
\&{if} ${}({*}\\{loc}\E\.{'='}){}$\5
${}\{{}$\5
\1\\{compress}(\\{lt\_eq});\5
${}\}{}$\2\6
\&{else} \&{if} ${}({*}\\{loc}\E\.{'<'}){}$\1\5
\\{compress}(\\{lt\_lt});\2\6
\&{break};\6
\4\&{case} \.{'\&'}:\6
\&{if} ${}({*}\\{loc}\E\.{'\&'}){}$\1\5
\\{compress}(\\{and\_and});\2\6
\&{break};\6
\4\&{case} \.{'|'}:\6
\&{if} ${}({*}\\{loc}\E\.{'|'}){}$\1\5
\\{compress}(\\{or\_or});\2\6
\&{break};\6
\4\&{case} \.{'!'}:\6
\&{if} ${}({*}\\{loc}\E\.{'='}){}$\1\5
\\{compress}(\\{non\_eq});\2\6
\&{break};\6
\4${}\}{}$\2\par
\U69.\fi

\M{72}\B\X72:Get an identifier\X${}\E{}$\6
${}\{{}$\1\6
${}\\{id\_first}\K\MM\\{loc};{}$\6
\&{do}\5
${}\PP\\{loc};{}$\5
\&{while} (\\{isalpha}((\&{int}) ${}{*}\\{loc})\V\\{isdigit}{}$((\&{int})
${}{*}\\{loc})\3{-1}\V\\{isxalpha}({*}\\{loc})\V\\{ishigh}({*}\\{loc}));{}$\6
${}\\{id\_loc}\K\\{loc};{}$\6
\&{return} \\{identifier};\6
\4${}\}{}$\2\par
\U69.\fi

\M{73}\B\X73:Get a constant\X${}\E{}$\6
${}\{{}$\1\6
\&{boolean} \\{hex\_flag}${}\K\\{false}{}$;\C{ are we reading a hexadecimal
literal? }\7
${}\\{id\_first}\K\\{loc}-\T{1};{}$\6
\&{if} ${}({*}\\{id\_first}\E\.{'.'}\W\R\\{xisdigit}({*}\\{loc})){}$\1\5
\&{goto} \\{mistake};\C{ not a constant }\2\6
\&{if} ${}({*}\\{id\_first}\E\.{'0'}){}$\5
${}\{{}$\1\6
\&{if} ${}({*}\\{loc}\E\.{'x'}\V{*}\\{loc}\E\.{'X'}){}$\5
${}\{{}$\C{ hex constant }\1\6
${}\\{hex\_flag}\K\\{true};{}$\6
${}\\{loc}\PP;{}$\6
\&{while} ${}(\\{xisxdigit}({*}\\{loc})\V{*}\\{loc}\E\.{'\\''}){}$\1\5
${}\\{loc}\PP;{}$\2\6
\4${}\}{}$\2\6
\&{else} \&{if} ${}({*}\\{loc}\E\.{'b'}\V{*}\\{loc}\E\.{'B'}){}$\5
${}\{{}$\C{ binary constant }\1\6
${}\\{loc}\PP;{}$\6
\&{while} ${}({*}\\{loc}\E\.{'0'}\V{*}\\{loc}\E\.{'1'}\V{*}\\{loc}\E\.{'%
\\''}){}$\1\5
${}\\{loc}\PP;{}$\2\6
\&{goto} \\{found};\6
\4${}\}{}$\2\6
\4${}\}{}$\2\6
\&{while} ${}(\\{xisdigit}({*}\\{loc})\V{*}\\{loc}\E\.{'\\''}){}$\1\5
${}\\{loc}\PP;{}$\2\6
\&{if} ${}({*}\\{loc}\E\.{'.'}){}$\5
${}\{{}$\1\6
${}\\{loc}\PP;{}$\6
\&{while} ${}((\\{hex\_flag}\W\\{xisxdigit}({*}\\{loc}))\V\\{xisdigit}({*}%
\\{loc})\V{*}\\{loc}\E\.{'\\''}){}$\1\5
${}\\{loc}\PP;{}$\2\6
\4${}\}{}$\2\6
\&{if} ${}({*}\\{loc}\E\.{'e'}\V{*}\\{loc}\E\.{'E'}){}$\5
${}\{{}$\C{ float constant }\1\6
\&{if} ${}({*}\PP\\{loc}\E\.{'+'}\V{*}\\{loc}\E\.{'-'}){}$\1\5
${}\\{loc}\PP;{}$\2\6
\&{while} ${}(\\{xisdigit}({*}\\{loc})\V{*}\\{loc}\E\.{'\\''}){}$\1\5
${}\\{loc}\PP;{}$\2\6
\4${}\}{}$\2\6
\&{else} \&{if} ${}(\\{hex\_flag}\W({*}\\{loc}\E\.{'p'}\V{*}\\{loc}\E%
\.{'P'})){}$\5
${}\{{}$\C{ hex float constant }\1\6
\&{if} ${}({*}\PP\\{loc}\E\.{'+'}\V{*}\\{loc}\E\.{'-'}){}$\1\5
${}\\{loc}\PP;{}$\2\6
\&{while} ${}(\\{xisxdigit}({*}\\{loc})\V{*}\\{loc}\E\.{'\\''}){}$\1\5
${}\\{loc}\PP;{}$\2\6
\4${}\}{}$\2\6
\4\\{found}:\6
\&{while} ${}({*}\\{loc}\E\.{'u'}\V{*}\\{loc}\E\.{'U'}\V{*}\\{loc}\E\.{'l'}%
\V{*}\\{loc}\E\.{'L'}\V{*}\\{loc}\E\.{'f'}\V{*}\\{loc}\E\.{'F'}){}$\1\5
${}\\{loc}\PP;{}$\2\6
${}\\{id\_loc}\K\\{loc};{}$\6
\&{return} \\{constant};\6
\4${}\}{}$\2\par
\U69.\fi

\M{74}\CEE/ strings and character constants, delimited by double and single
quotes, respectively, can contain newlines or instances of their own
delimiters if they are protected by a backslash.  We follow this
convention, but do not allow the string to be longer than \PB{\\{longest%
\_name}}.

\Y\B\4\X74:Get a string\X${}\E{}$\6
${}\{{}$\1\6
\&{char} \\{delim}${}\K{}$(\&{char}) \|c;\C{ what started the string }\7
${}\\{id\_first}\K\\{section\_text}+\T{1};{}$\6
${}\\{id\_loc}\K\\{section\_text};{}$\6
${}{*}\PP\\{id\_loc}\K\\{delim};{}$\6
\&{if} ${}(\\{delim}\E\.{'L'}\V\\{delim}\E\.{'u'}\V\\{delim}\E\.{'U'}){}$\5
${}\{{}$\C{ wide character constant }\1\6
\&{if} ${}(\\{delim}\E\.{'u'}\W{*}\\{loc}\E\.{'8'}){}$\1\5
${}{*}\PP\\{id\_loc}\K{*}\\{loc}\PP;{}$\2\6
${}\\{delim}\K{*}\\{loc}\PP;{}$\6
${}{*}\PP\\{id\_loc}\K\\{delim};{}$\6
\4${}\}{}$\2\6
\&{while} (\\{true})\5
${}\{{}$\1\6
\&{if} ${}(\\{loc}\G\\{limit}){}$\5
${}\{{}$\1\6
\&{if} ${}({*}(\\{limit}-\T{1})\I\.{'\\\\'}){}$\5
${}\{{}$\1\6
\\{err\_print}(\.{"!\ String\ didn't\ end}\)\.{"});\6
${}\\{loc}\K\\{limit};{}$\6
\&{break};\6
\4${}\}{}$\2\6
\&{if} ${}(\\{get\_line}(\,)\E\\{false}){}$\5
${}\{{}$\1\6
\\{err\_print}(\.{"!\ Input\ ended\ in\ mi}\)\.{ddle\ of\ string"});\6
${}\\{loc}\K\\{buffer};{}$\6
\&{break};\6
\4${}\}{}$\2\6
\&{else} \&{if} ${}(\PP\\{id\_loc}\Z\\{section\_text\_end}){}$\1\5
${}{*}\\{id\_loc}\K\.{'\\n'}{}$;\C{ will print as       \.{"\\\\\\n"} }\2\6
\4${}\}{}$\2\6
\&{if} ${}((\|c\K{}$(\&{eight\_bits}) ${}{*}\\{loc}\PP)\E\\{delim}){}$\5
${}\{{}$\1\6
\&{if} ${}(\PP\\{id\_loc}\Z\\{section\_text\_end}){}$\1\5
${}{*}\\{id\_loc}\K{}$(\&{char}) \|c;\2\6
\&{break};\6
\4${}\}{}$\2\6
\&{if} ${}(\|c\E\.{'\\\\'}){}$\5
${}\{{}$\1\6
\&{if} ${}(\\{loc}\G\\{limit}){}$\1\5
\&{continue};\2\6
\&{if} ${}(\PP\\{id\_loc}\Z\\{section\_text\_end}){}$\1\5
${}{*}\\{id\_loc}\K\.{'\\\\'};{}$\2\6
${}\|c\K{}$(\&{eight\_bits}) ${}{*}\\{loc}\PP;{}$\6
\4${}\}{}$\2\6
\&{if} ${}(\PP\\{id\_loc}\Z\\{section\_text\_end}){}$\1\5
${}{*}\\{id\_loc}\K{}$(\&{char}) \|c;\2\6
\4${}\}{}$\2\6
\&{if} ${}(\\{id\_loc}\G\\{section\_text\_end}){}$\5
${}\{{}$\1\6
${}\\{fputs}(\.{"\\n!\ String\ too\ long}\)\.{:\ "},\39\\{stdout});{}$\6
${}\\{term\_write}(\\{section\_text}+\T{1},\39\T{25});{}$\6
\\{err\_print}(\.{"..."});\6
\4${}\}{}$\2\6
${}\\{id\_loc}\PP;{}$\6
\&{return} \\{string};\6
\4${}\}{}$\2\par
\U69.\fi

\M{75}After an \.{@} sign has been scanned, the next character tells us
whether there is more work to do.

\Y\B\4\X75:Get control code and possible section name\X${}\E{}$\6
\&{switch} ${}(\|c\K\\{ccode}{}$[(\&{eight\_bits}) ${}{*}\\{loc}\PP]){}$\5
${}\{{}$\1\6
\4\&{case} \\{ignore}:\5
\&{continue};\6
\4\&{case} \\{translit\_code}:\5
\\{err\_print}(\.{"!\ Use\ @l\ in\ limbo\ o}\)\.{nly"});\6
\&{continue};\6
\4\&{case} \\{control\_text}:\6
\&{while} ${}((\|c\K\\{skip\_ahead}(\,))\E\.{'@'}){}$\1\5
;\C{ only \.{@@} and \.{@>} are expected }\2\6
\&{if} ${}({*}(\\{loc}-\T{1})\I\.{'>'}){}$\1\5
\\{err\_print}(\.{"!\ Double\ @\ should\ b}\)\.{e\ used\ in\ control\ te}\)%
\.{xt"});\2\6
\&{continue};\6
\4\&{case} \\{section\_name}:\5
${}\\{cur\_section\_name\_char}\K{*}(\\{loc}-\T{1});{}$\6
\X77:Scan the section name and make \PB{\\{cur\_section\_name}} point to it\X\6
\4\&{case} \\{string}:\5
\X81:Scan a verbatim string\X\6
\4\&{case} \\{ord}:\5
\X76:Scan an ASCII constant\X\6
\4\&{default}:\5
\&{return} \|c;\6
\4${}\}{}$\2\par
\Q92.
\U69.\fi

\M{76}After scanning a valid ASCII constant that follows
\.{@'}, this code plows ahead until it finds the next single quote.
(Special care is taken if the quote is part of the constant.)
Anything after a valid ASCII constant is ignored;
thus, \.{@'\\nopq'} gives the same result as \.{@'\\n'}.

\Y\B\4\X76:Scan an ASCII constant\X${}\E{}$\6
$\\{id\_first}\K\\{loc};{}$\6
\&{if} ${}({*}\\{loc}\E\.{'\\\\'}){}$\1\6
\&{if} ${}({*}\PP\\{loc}\E\.{'\\''}){}$\1\5
${}\\{loc}\PP;{}$\2\2\6
\&{while} ${}({*}\\{loc}\I\.{'\\''}){}$\5
${}\{{}$\1\6
\&{if} ${}({*}\\{loc}\E\.{'@'}){}$\5
${}\{{}$\1\6
\&{if} ${}({*}(\\{loc}+\T{1})\I\.{'@'}){}$\1\5
\\{err\_print}(\.{"!\ Double\ @\ should\ b}\)\.{e\ used\ in\ ASCII\ cons}\)%
\.{tant"});\2\6
\&{else}\1\5
${}\\{loc}\PP;{}$\2\6
\4${}\}{}$\2\6
${}\\{loc}\PP;{}$\6
\&{if} ${}(\\{loc}>\\{limit}){}$\5
${}\{{}$\1\6
\\{err\_print}(\.{"!\ String\ didn't\ end}\)\.{"});\6
${}\\{loc}\K\\{limit}-\T{1};{}$\6
\&{break};\6
\4${}\}{}$\2\6
\4${}\}{}$\2\6
${}\\{loc}\PP;{}$\6
\&{return} \\{ord};\par
\U75.\fi

\M{77}\B\X77:Scan the section name and make \PB{\\{cur\_section\_name}} point
to it\X${}\E{}$\6
${}\{{}$\1\6
\&{char} ${}{*}\|k\K\\{section\_text}{}$;\C{ pointer into \PB{\\{section%
\_text}} }\7
\X79:Put section name into \PB{\\{section\_text}}\X\6
\&{if} ${}(\|k-\\{section\_text}>\T{3}\W\\{strncmp}(\|k-\T{2},\39\.{"..."},\39%
\T{3})\E\T{0}){}$\1\5
${}\\{cur\_section\_name}\K\\{section\_lookup}(\\{section\_text}+\T{1},\39\|k-%
\T{3},\39\\{true}){}$;\C{ \PB{\\{true}} means it's a prefix }\2\6
\&{else}\1\5
${}\\{cur\_section\_name}\K\\{section\_lookup}(\\{section\_text}+\T{1},\39\|k,%
\39\\{false}){}$;\C{ \PB{\\{false}} means it's not }\2\6
\&{if} ${}(\\{cur\_section\_name\_char}\E\.{'('}){}$\1\5
\X47:If it's not there, add \PB{\\{cur\_section\_name}} to the output file
stack, or complain we're out of room\X\2\6
\&{return} \\{section\_name};\6
\4${}\}{}$\2\par
\U75.\fi

\M{78}Section names are placed into the \PB{\\{section\_text}} array with
consecutive spaces,
tabs, and carriage-returns replaced by single spaces. There will be no
spaces at the beginning or the end. (We set \PB{$\\{section\_text}[\T{0}]\K\.{'%
\ '}$} to facilitate
this, since the \PB{\\{section\_lookup}} routine uses \PB{\\{section\_text}[%
\T{1}]} as the first
character of the name.)

\Y\B\4\X21:Set initial values\X${}\mathrel+\E{}$\6
$\\{section\_text}[\T{0}]\K\.{'\ '}{}$;\par
\fi

\M{79}\B\X79:Put section name into \PB{\\{section\_text}}\X${}\E{}$\6
\&{while} (\\{true})\5
${}\{{}$\1\6
\&{if} ${}(\\{loc}>\\{limit}\W\\{get\_line}(\,)\E\\{false}){}$\5
${}\{{}$\1\6
\\{err\_print}(\.{"!\ Input\ ended\ in\ se}\)\.{ction\ name"});\6
${}\\{loc}\K\\{buffer}+\T{1};{}$\6
\&{break};\6
\4${}\}{}$\2\6
${}\|c\K{}$(\&{eight\_bits}) ${}{*}\\{loc};{}$\6
\X80:If end of name or erroneous nesting, \PB{\&{break}}\X\6
${}\\{loc}\PP;{}$\6
\&{if} ${}(\|k<\\{section\_text\_end}){}$\1\5
${}\|k\PP;{}$\2\6
\&{if} (\\{xisspace}(\|c))\5
${}\{{}$\1\6
${}\|c\K{}$(\&{eight\_bits}) \.{'\ '};\6
\&{if} ${}({*}(\|k-\T{1})\E\.{'\ '}){}$\1\5
${}\|k\MM;{}$\2\6
\4${}\}{}$\2\6
${}{*}\|k\K{}$(\&{char}) \|c;\6
\4${}\}{}$\2\6
\&{if} ${}(\|k\G\\{section\_text\_end}){}$\5
${}\{{}$\1\6
${}\\{fputs}(\.{"\\n!\ Section\ name\ to}\)\.{o\ long:\ "},\39\\{stdout});{}$\6
${}\\{term\_write}(\\{section\_text}+\T{1},\39\T{25});{}$\6
\\{printf}(\.{"..."});\6
\\{mark\_harmless};\6
\4${}\}{}$\2\6
\&{if} ${}({*}\|k\E\.{'\ '}\W\|k>\\{section\_text}){}$\1\5
${}\|k\MM{}$;\2\par
\U77.\fi

\M{80}\B\X80:If end of name or erroneous nesting, \PB{\&{break}}\X${}\E{}$\6
\&{if} ${}(\|c\E\.{'@'}){}$\5
${}\{{}$\1\6
${}\|c\K{}$(\&{eight\_bits}) ${}{*}(\\{loc}+\T{1});{}$\6
\&{if} ${}(\|c\E\.{'>'}){}$\5
${}\{{}$\1\6
${}\\{loc}\MRL{+{\K}}\T{2};{}$\6
\&{break};\6
\4${}\}{}$\2\6
\&{if} (\\{ccode}[(\&{eight\_bits}) \|c]${}\E\\{new\_section}){}$\5
${}\{{}$\1\6
\\{err\_print}(\.{"!\ Section\ name\ didn}\)\.{'t\ end"});\6
\&{break};\6
\4${}\}{}$\2\6
\&{if} (\\{ccode}[(\&{eight\_bits}) \|c]${}\E\\{section\_name}){}$\5
${}\{{}$\1\6
\\{err\_print}(\.{"!\ Nesting\ of\ sectio}\)\.{n\ names\ not\ allowed"});\6
\&{break};\6
\4${}\}{}$\2\6
${}{*}(\PP\|k)\K\.{'@'};{}$\6
${}\\{loc}\PP{}$;\C{ now \PB{$\|c\E{*}\\{loc}$} again }\6
\4${}\}{}$\2\par
\U79.\fi

\M{81}At the present point in the program we
have \PB{${*}(\\{loc}-\T{1})\E\\{string}$}; we set \PB{\\{id\_first}} to the
beginning
of the string itself, and \PB{\\{id\_loc}} to its ending-plus-one location in
the
buffer.  We also set \PB{\\{loc}} to the position just after the ending
delimiter.

\Y\B\4\X81:Scan a verbatim string\X${}\E{}$\6
$\\{id\_first}\K\\{loc}\PP;{}$\6
${}{*}(\\{limit}+\T{1})\K\.{'@'};{}$\6
${}{*}(\\{limit}+\T{2})\K\.{'>'};{}$\6
\&{while} ${}({*}\\{loc}\I\.{'@'}\V{*}(\\{loc}+\T{1})\I\.{'>'}){}$\1\5
${}\\{loc}\PP;{}$\2\6
\&{if} ${}(\\{loc}\G\\{limit}){}$\1\5
\\{err\_print}(\.{"!\ Verbatim\ string\ d}\)\.{idn't\ end"});\2\6
${}\\{id\_loc}\K\\{loc};{}$\6
${}\\{loc}\MRL{+{\K}}\T{2};{}$\6
\&{return} \\{string};\par
\U75.\fi

\N{1}{82}Scanning a macro definition.
The rules for generating the replacement texts corresponding to macros and
\CEE/ texts of a section are almost identical; the only differences are that

\yskip \item{a)}Section names are not allowed in macros;
in fact, the appearance of a section name terminates such macros and denotes
the name of the current section.

\item{b)}The symbols \.{@d} and \.{@f} and \.{@c} are not allowed after
section names, while they terminate macro definitions.

\item{c)}Spaces are inserted after right parentheses in macros, because the
ANSI \CEE/ preprocessor sometimes requires it.

\yskip Therefore there is a single procedure \PB{\\{scan\_repl}} whose
parameter
\PB{\|t} specifies either \PB{\\{macro}} or \PB{\\{section\_name}}. After \PB{%
\\{scan\_repl}} has
acted, \PB{\\{cur\_text}} will point to the replacement text just generated,
and
\PB{\\{next\_control}} will contain the control code that terminated the
activity.

\Y\B\4\D\\{app\_repl}$(\|c)$\6
${}\{{}$\1\6
\&{if} ${}(\\{tok\_ptr}\E\\{tok\_mem\_end}){}$\1\5
\\{overflow}(\.{"token"});\2\6
\&{else}\1\5
${}{*}(\\{tok\_ptr}\PP)\K{}$(\&{eight\_bits}) \|c;\2\6
\4${}\}{}$\2\par
\Y\B\4\X20:Private variables\X${}\mathrel+\E{}$\6
\&{static} \&{text\_pointer} \\{cur\_text};\C{ replacement text formed by \PB{%
\\{scan\_repl}} }\6
\&{static} \&{eight\_bits} \\{next\_control};\par
\fi

\M{83}\B\1\1\&{static} \&{void} \\{scan\_repl}(\C{ creates a replacement text }%
\6
\&{eight\_bits} \|t)\2\2\6
${}\{{}$\1\6
\&{sixteen\_bits} \|a;\C{ the current token }\7
\&{if} ${}(\|t\E\\{section\_name}){}$\1\5
\X85:Insert the line number into \PB{\\{tok\_mem}}\X\2\6
\&{while} (\\{true})\1\6
\&{switch} ${}(\|a\K\\{get\_next}(\,)){}$\5
${}\{{}$\1\6
\hbox{\4}\X86:In cases that \PB{\|a} is a non-\PB{\&{char}} token (\PB{%
\\{identifier}}, \PB{\\{section\_name}}, etc.), either process it and change %
\PB{\|a} to a byte that should be stored, or \PB{\&{continue}} if \PB{\|a}
should be ignored, or \PB{\&{goto} \\{done}} if \PB{\|a} signals the end of
this replacement text\X\6
\4\&{case} \.{')'}:\5
\\{app\_repl}(\|a);\6
\&{if} ${}(\|t\E\\{macro}){}$\1\5
\\{app\_repl}(\.{'\ '});\2\6
\&{break};\6
\4\&{default}:\5
\\{app\_repl}(\|a);\C{ store \PB{\|a} in \PB{\\{tok\_mem}} }\6
\4${}\}{}$\2\2\6
\4\\{done}:\5
${}\\{next\_control}\K{}$(\&{eight\_bits}) \|a;\6
\&{if} ${}(\\{text\_ptr}>\\{text\_info\_end}){}$\1\5
\\{overflow}(\.{"text"});\2\6
${}\\{cur\_text}\K\\{text\_ptr};{}$\6
${}(\PP\\{text\_ptr})\MG\\{tok\_start}\K\\{tok\_ptr};{}$\6
\4${}\}{}$\2\par
\fi

\M{84}\B\X8:Predeclaration of procedures\X${}\mathrel+\E{}$\5
\&{static} \&{void} \\{scan\_repl}(\&{eight\_bits});\par
\fi

\M{85}Here is the code for the line number: first a \PB{\&{sixteen\_bits}}
equal
to \PB{\T{\~150000}}; then the numeric line number; then a pointer to the
file name.

\Y\B\4\D\\{store\_id}$(\|a)$\5
$\|a\K\\{id\_lookup}(\\{id\_first},\39\\{id\_loc},\39\.{'\\0'})-\\{name%
\_dir}{}$;\6
${}\\{app\_repl}((\|a/\T{\~400})+\T{\~200});$ $\\{app\_repl}(\|a\MOD\T{%
\~400}{}$)\par
\Y\B\4\X85:Insert the line number into \PB{\\{tok\_mem}}\X${}\E{}$\6
${}\{{}$\1\6
\&{eight\_bits} \|a;\C{ shadow variable \PB{\|a} }\7
\\{store\_two\_bytes}(\T{\~150000});\6
\&{if} ${}(\\{changing}\W\\{include\_depth}\E\\{change\_depth}){}$\5
${}\{{}$\C{ correction made Feb 2017 }\1\6
${}\\{id\_first}\K\\{change\_file\_name};{}$\6
\\{store\_two\_bytes}((\&{sixteen\_bits}) \\{change\_line});\6
\4${}\}{}$\5
\2\&{else}\5
${}\{{}$\1\6
${}\\{id\_first}\K\\{cur\_file\_name};{}$\6
\\{store\_two\_bytes}((\&{sixteen\_bits}) \\{cur\_line});\6
\4${}\}{}$\2\6
${}\\{id\_loc}\K\\{id\_first}+\\{strlen}(\\{id\_first});{}$\6
\\{store\_id}(\|a);\6
\4${}\}{}$\2\par
\Us69, 83\ETs86.\fi

\M{86}\B\X86:In cases that \PB{\|a} is a non-\PB{\&{char}} token (\PB{%
\\{identifier}}, \PB{\\{section\_name}}, etc.), either process it and change %
\PB{\|a} to a byte that should be stored, or \PB{\&{continue}} if \PB{\|a}
should be ignored, or \PB{\&{goto} \\{done}} if \PB{\|a} signals the end of
this replacement text\X${}\E{}$\6
\4\hbox{\1\quad}\&{case} \\{identifier}:\5
\\{store\_id}(\|a);\6
\&{break};\6
\4\&{case} \\{section\_name}:\6
\&{if} ${}(\|t\I\\{section\_name}){}$\1\5
\&{goto} \\{done};\2\6
\&{else}\5
${}\{{}$\1\6
\X87:Was an `\.{@}' missed here?\X\6
${}\|a\K\\{cur\_section\_name}-\\{name\_dir};{}$\6
${}\\{app\_repl}((\|a/\T{\~400})+\T{\~250});{}$\6
${}\\{app\_repl}(\|a\MOD\T{\~400});{}$\6
\X85:Insert the line number into \PB{\\{tok\_mem}}\X\6
\4${}\}{}$\2\6
\&{break};\6
\4\&{case} \\{output\_defs\_code}:\6
\&{if} ${}(\|t\I\\{section\_name}){}$\1\5
\\{err\_print}(\.{"!\ Misplaced\ @h"});\2\6
\&{else}\5
${}\{{}$\1\6
${}\\{output\_defs\_seen}\K\\{true};{}$\6
${}\|a\K\\{output\_defs\_flag};{}$\6
${}\\{app\_repl}((\|a/\T{\~400})+\T{\~200});{}$\6
${}\\{app\_repl}(\|a\MOD\T{\~400});{}$\6
\X85:Insert the line number into \PB{\\{tok\_mem}}\X\6
\4${}\}{}$\2\6
\&{break};\6
\4\&{case} \\{constant}:\5
\&{case} \\{string}:\5
\X88:Copy a string or verbatim construction or numerical constant\X\6
\&{break};\6
\4\&{case} \\{ord}:\5
\X89:Copy an ASCII constant\X\6
\&{break};\6
\4\&{case} \\{definition}:\5
\&{case} \\{format\_code}:\5
\&{case} \\{begin\_C}:\6
\&{if} ${}(\|t\I\\{section\_name}){}$\1\5
\&{goto} \\{done};\2\6
\&{else}\5
${}\{{}$\1\6
\\{err\_print}(\.{"!\ @d,\ @f\ and\ @c\ are}\)\.{\ ignored\ in\ C\ text"});\6
\&{continue};\6
\4${}\}{}$\2\6
\4\&{case} \\{new\_section}:\5
\&{goto} \\{done};\par
\U83.\fi

\M{87}\B\X87:Was an `\.{@}' missed here?\X${}\E{}$\6
${}\{{}$\1\6
\&{char} ${}{*}\\{try\_loc}\K\\{loc};{}$\7
\&{while} ${}({*}\\{try\_loc}\E\.{'\ '}\W\\{try\_loc}<\\{limit}){}$\1\5
${}\\{try\_loc}\PP;{}$\2\6
\&{if} ${}({*}\\{try\_loc}\E\.{'+'}\W\\{try\_loc}<\\{limit}){}$\1\5
${}\\{try\_loc}\PP;{}$\2\6
\&{while} ${}({*}\\{try\_loc}\E\.{'\ '}\W\\{try\_loc}<\\{limit}){}$\1\5
${}\\{try\_loc}\PP;{}$\2\6
\&{if} ${}({*}\\{try\_loc}\E\.{'='}){}$\1\5
\\{err\_print}(\.{"!\ Missing\ `@\ '\ befo}\)\.{re\ a\ named\ section"});\C{
user who isn't defining a section should put newline after the name,      as
explained in the manual }\2\6
\4${}\}{}$\2\par
\U86.\fi

\M{88}By default, \.{CTANGLE} purges single-quote characters from %
\CPLUSPLUS/-style
literals, e.g., \.{1'000'000}, so that you can use this notation also in \CEE/
code. The \.{+k} switch will `keep' the single quotes in the output.

\Y\B\4\D\\{keep\_digit\_separators}\5
\\{flags}[\.{'k'}]\par
\Y\B\4\X88:Copy a string or verbatim construction or numerical constant\X${}%
\E{}$\6
\\{app\_repl}(\|a);\C{ \PB{\\{string}} or \PB{\\{constant}} }\6
\&{while} ${}(\\{id\_first}<\\{id\_loc}){}$\5
${}\{{}$\C{ simplify \.{@@} pairs }\1\6
\&{if} ${}({*}\\{id\_first}\E\.{'@'}){}$\5
${}\{{}$\1\6
\&{if} ${}({*}(\\{id\_first}+\T{1})\E\.{'@'}){}$\1\5
${}\\{id\_first}\PP;{}$\2\6
\&{else}\1\5
\\{err\_print}(\.{"!\ Double\ @\ should\ b}\)\.{e\ used\ in\ string"});\2\6
\4${}\}{}$\2\6
\&{else} \&{if} ${}(\|a\E\\{constant}\W{*}\\{id\_first}\E\.{'\\''}\W\R\\{keep%
\_digit\_separators}){}$\1\5
${}\\{id\_first}\PP;{}$\2\6
${}\\{app\_repl}({*}\\{id\_first}\PP);{}$\6
\4${}\}{}$\2\6
\\{app\_repl}(\|a);\par
\U86.\fi

\M{89}This section should be rewritten on machines that don't use ASCII
code internally.

\Y\B\4\X89:Copy an ASCII constant\X${}\E{}$\6
${}\{{}$\1\6
\&{int} \|c${}\K{}$(\&{int})((\&{eight\_bits}) ${}{*}\\{id\_first});{}$\7
\&{if} ${}(\|c\E\.{'\\\\'}){}$\5
${}\{{}$\1\6
${}\|c\K{}$(\&{int})((\&{eight\_bits}) ${}{*}\PP\\{id\_first});{}$\6
\&{if} ${}(\|c\G\.{'0'}\W\|c\Z\.{'7'}){}$\5
${}\{{}$\1\6
${}\|c\MRL{-{\K}}\.{'0'};{}$\6
\&{if} ${}({*}(\\{id\_first}+\T{1})\G\.{'0'}\W{*}(\\{id\_first}+\T{1})\Z%
\.{'7'}){}$\5
${}\{{}$\1\6
${}\|c\K\T{8}*\|c+{*}(\PP\\{id\_first})-\.{'0'};{}$\6
\&{if} ${}({*}(\\{id\_first}+\T{1})\G\.{'0'}\W{*}(\\{id\_first}+\T{1})\Z\.{'7'}%
\W\|c<\T{32}){}$\1\5
${}\|c\K\T{8}*\|c+{*}(\PP\\{id\_first})-\.{'0'};{}$\2\6
\4${}\}{}$\2\6
\4${}\}{}$\2\6
\&{else}\1\6
\&{switch} (\|c)\5
${}\{{}$\1\6
\4\&{case} \.{'t'}:\5
${}\|c\K\.{'\\t'}{}$;\5
\&{break};\6
\4\&{case} \.{'n'}:\5
${}\|c\K\.{'\\n'}{}$;\5
\&{break};\6
\4\&{case} \.{'b'}:\5
${}\|c\K\.{'\\b'}{}$;\5
\&{break};\6
\4\&{case} \.{'f'}:\5
${}\|c\K\.{'\\f'}{}$;\5
\&{break};\6
\4\&{case} \.{'v'}:\5
${}\|c\K\.{'\\v'}{}$;\5
\&{break};\6
\4\&{case} \.{'r'}:\5
${}\|c\K\.{'\\r'}{}$;\5
\&{break};\6
\4\&{case} \.{'a'}:\5
${}\|c\K\.{'\\7'}{}$;\5
\&{break};\6
\4\&{case} \.{'?'}:\5
${}\|c\K\.{'?'}{}$;\5
\&{break};\6
\4\&{case} \.{'x'}:\6
\&{if} ${}(\\{xisdigit}({*}(\\{id\_first}+\T{1}))){}$\1\5
${}\|c\K(\&{int})({*}(\PP\\{id\_first})-\.{'0'});{}$\2\6
\&{else} \&{if} ${}(\\{xisxdigit}({*}(\\{id\_first}+\T{1}))){}$\5
${}\{{}$\1\6
${}\PP\\{id\_first};{}$\6
${}\|c\K\\{toupper}{}$((\&{int}) ${}{*}\\{id\_first})-\.{'A'}+\T{10};{}$\6
\4${}\}{}$\2\6
\&{if} ${}(\\{xisdigit}({*}(\\{id\_first}+\T{1}))){}$\1\5
${}\|c\K\T{16}*\|c+(\&{int})({*}(\PP\\{id\_first})-\.{'0'});{}$\2\6
\&{else} \&{if} ${}(\\{xisxdigit}({*}(\\{id\_first}+\T{1}))){}$\5
${}\{{}$\1\6
${}\PP\\{id\_first};{}$\6
${}\|c\K\T{16}*\|c+\\{toupper}{}$((\&{int}) ${}{*}\\{id\_first})-{}$(\&{int}) %
\.{'A'}${}+\T{10};{}$\6
\4${}\}{}$\2\6
\&{break};\6
\4\&{case} \.{'\\\\'}:\5
${}\|c\K\.{'\\\\'}{}$;\5
\&{break};\6
\4\&{case} \.{'\\''}:\5
${}\|c\K\.{'\\''}{}$;\5
\&{break};\6
\4\&{case} \.{'\\"'}:\5
${}\|c\K\.{'\\"'}{}$;\5
\&{break};\6
\4\&{default}:\5
\\{err\_print}(\.{"!\ Unrecognized\ esca}\)\.{pe\ sequence"});\6
\4${}\}{}$\2\2\6
\4${}\}{}$\C{ at this point \PB{\|c} should have been converted to its ASCII
code number }\2\6
\\{app\_repl}(\\{constant});\6
\&{if} ${}(\|c\G\T{100}){}$\1\5
\\{app\_repl}((\&{int}) \.{'0'}${}+\|c/\T{100});{}$\2\6
\&{if} ${}(\|c\G\T{10}){}$\1\5
\\{app\_repl}((\&{int}) \.{'0'}${}+(\|c/\T{10})\MOD\T{10});{}$\2\6
\\{app\_repl}((\&{int}) \.{'0'}${}+\|c\MOD\T{10});{}$\6
\\{app\_repl}(\\{constant});\6
\4${}\}{}$\2\par
\U86.\fi

\N{1}{90}Scanning a section.
The \PB{\\{scan\_section}} procedure starts when `\.{@\ }' or `\.{@*}' has been
sensed in the input, and it proceeds until the end of that section.  It
uses \PB{\\{section\_count}} to keep track of the current section number; with
luck,
\.{CWEAVE} and \.{CTANGLE} will both assign the same numbers to sections.

The body of \PB{\\{scan\_section}} is a loop where we look for control codes
that are significant to \.{CTANGLE}: those
that delimit a definition, the \CEE/ part of a module, or a new module.

\Y\B\1\1\&{static} \&{void} \\{scan\_section}(\&{void})\2\2\6
${}\{{}$\1\6
\&{name\_pointer} \|p;\C{ section name for the current section }\6
\&{text\_pointer} \|q;\C{ text for the current section }\6
\&{sixteen\_bits} \|a;\C{ token for left-hand side of definition }\7
${}\\{section\_count}\PP{}$;\5
${}\\{no\_where}\K\\{true};{}$\6
\&{if} ${}({*}(\\{loc}-\T{1})\E\.{'*'}\W\\{show\_progress}){}$\5
${}\{{}$\C{ starred section }\1\6
${}\\{printf}(\.{"*\%d"},\39{}$(\&{int}) \\{section\_count});\6
\\{update\_terminal};\6
\4${}\}{}$\2\6
${}\\{next\_control}\K\\{ignore};{}$\6
\&{while} (\\{true})\5
${}\{{}$\1\6
\X92:Skip ahead until \PB{\\{next\_control}} corresponds to \.{@d}, \.{@<}, %
\.{@\ } or the like\X\6
\&{if} ${}(\\{next\_control}\E\\{definition}){}$\5
${}\{{}$\C{ \.{@d} }\1\6
\X93:Scan a definition\X\6
\&{continue};\6
\4${}\}{}$\2\6
\&{if} ${}(\\{next\_control}\E\\{begin\_C}){}$\5
${}\{{}$\C{ \.{@c} or \.{@p} }\1\6
${}\|p\K\\{name\_dir};{}$\6
\&{break};\6
\4${}\}{}$\2\6
\&{if} ${}(\\{next\_control}\E\\{section\_name}){}$\5
${}\{{}$\C{ \.{@<} or \.{@(} }\1\6
${}\|p\K\\{cur\_section\_name};{}$\6
\X94:If section is not being defined, \PB{\&{continue}}\X\6
\&{break};\6
\4${}\}{}$\2\6
\&{return};\C{ \.{@\ } or \.{@*} }\6
\4${}\}{}$\2\6
${}\\{no\_where}\K\\{print\_where}\K\\{false};{}$\6
\X95:Scan the \CEE/ part of the current section\X\6
\4${}\}{}$\2\par
\fi

\M{91}\B\X8:Predeclaration of procedures\X${}\mathrel+\E{}$\5
\&{static} \&{void} \\{scan\_section}(\&{void});\par
\fi

\M{92}At the top of this loop, if \PB{$\\{next\_control}\E\\{section\_name}$},
the
section name has already been scanned (see \PB{$\X75:Get control code and
possible section name\X$}).  Thus, if we encounter \PB{$\\{next\_control}\E%
\\{section\_name}$} in the
skip-ahead process, we should likewise scan the section name, so later
processing will be the same in both cases.

\Y\B\4\X92:Skip ahead until \PB{\\{next\_control}} corresponds to \.{@d}, %
\.{@<}, \.{@\ } or the like\X${}\E{}$\6
\&{while} ${}(\\{next\_control}<\\{definition}{}$)\C{ \PB{\\{definition}} is
the lowest of the ``significant'' codes }\1\6
\&{if} ${}((\\{next\_control}\K\\{skip\_ahead}(\,))\E\\{section\_name}){}$\5
${}\{{}$\1\6
${}\\{loc}\MRL{-{\K}}\T{2};{}$\6
${}\\{next\_control}\K\\{get\_next}(\,);{}$\6
\4${}\}{}$\2\2\par
\U90.\fi

\M{93}\B\X93:Scan a definition\X${}\E{}$\6
\&{while} ${}((\\{next\_control}\K\\{get\_next}(\,))\E\.{'\\n'}){}$\1\5
;\C{ allow newline before definition }\2\6
\&{if} ${}(\\{next\_control}\I\\{identifier}){}$\5
${}\{{}$\1\6
\\{err\_print}(\.{"!\ Definition\ flushe}\)\.{d,\ must\ start\ with\ i}\)%
\.{dentifier"});\6
\&{continue};\6
\4${}\}{}$\2\6
\\{store\_id}(\|a);\C{ append the lhs }\6
\&{if} ${}({*}\\{loc}\I\.{'('}){}$\5
${}\{{}$\C{ identifier must be separated from replacement text }\1\6
\\{app\_repl}(\\{string});\6
\\{app\_repl}(\.{'\ '});\6
\\{app\_repl}(\\{string});\6
\4${}\}{}$\2\6
\\{scan\_repl}(\\{macro});\6
${}\\{cur\_text}\MG\\{text\_link}\K\\{macro}{}$;\par
\U90.\fi

\M{94}If the section name is not followed by \.{=} or \.{+=}, no \CEE/
code is forthcoming: the section is being cited, not being
defined.  This use is illegal after the definition part of the
current section has started, except inside a comment, but
\.{CTANGLE} does not enforce this rule; it simply ignores the offending
section name and everything following it, up to the next significant
control code.

\Y\B\4\X94:If section is not being defined, \PB{\&{continue}}\X${}\E{}$\6
\&{while} ${}((\\{next\_control}\K\\{get\_next}(\,))\E\.{'+'}){}$\1\5
;\C{ allow optional \.{+=} }\2\6
\&{if} ${}(\\{next\_control}\I\.{'='}\W\\{next\_control}\I\\{eq\_eq}){}$\1\5
\&{continue};\2\par
\U90.\fi

\M{95}\B\X95:Scan the \CEE/ part of the current section\X${}\E{}$\6
\X96:Insert the section number into \PB{\\{tok\_mem}}\X\6
\\{scan\_repl}(\\{section\_name});\C{ now \PB{\\{cur\_text}} points to the
replacement text }\6
\X97:Update the data structure so that the replacement text is accessible\X\par
\U90.\fi

\M{96}\B\X96:Insert the section number into \PB{\\{tok\_mem}}\X${}\E{}$\6
$\\{store\_two\_bytes}((\&{sixteen\_bits})(\T{\~150000}+\\{section%
\_count})){}$;\C{ \PB{$\T{\~150000}\E\T{\~320}*\T{\~400}$} }\par
\U95.\fi

\M{97}\B\X97:Update the data structure so that the replacement text is
accessible\X${}\E{}$\6
\&{if} ${}(\|p\E\\{name\_dir}\V\|p\E\NULL){}$\5
${}\{{}$\C{ unnamed section, or bad section name }\1\6
${}\\{last\_unnamed}\MG\\{text\_link}\K\\{cur\_text}-\\{text\_info};{}$\6
${}\\{last\_unnamed}\K\\{cur\_text};{}$\6
\4${}\}{}$\2\6
\&{else} \&{if} ${}(\|p\MG\\{equiv}\E{}$(\&{void} ${}{*}){}$ \\{text\_info})\1\5
${}\|p\MG\\{equiv}\K{}$(\&{void} ${}{*}){}$ \\{cur\_text};\C{ first section of
this name }\2\6
\&{else}\5
${}\{{}$\1\6
${}\|q\K{}$(\&{text\_pointer}) \|p${}\MG\\{equiv};{}$\6
\&{while} ${}(\|q\MG\\{text\_link}<\\{section\_flag}){}$\1\5
${}\|q\K\|q\MG\\{text\_link}+\\{text\_info}{}$;\C{ find end of list }\2\6
${}\|q\MG\\{text\_link}\K\\{cur\_text}-\\{text\_info};{}$\6
\4${}\}{}$\2\6
${}\\{cur\_text}\MG\\{text\_link}\K\\{section\_flag}{}$;\C{ mark this
replacement text as a nonmacro }\par
\U95.\fi

\M{98}\B\1\1\&{static} \&{void} \\{phase\_one}(\&{void})\2\2\6
${}\{{}$\1\6
${}\\{phase}\K\T{1};{}$\6
${}\\{section\_count}\K\T{0};{}$\6
\\{reset\_input}(\,);\6
\\{skip\_limbo}(\,);\6
\&{while} ${}(\R\\{input\_has\_ended}){}$\1\5
\\{scan\_section}(\,);\2\6
\\{check\_complete}(\,);\6
\4${}\}{}$\2\par
\fi

\M{99}\B\X8:Predeclaration of procedures\X${}\mathrel+\E{}$\5
\&{static} \&{void} \\{phase\_one}(\&{void});\par
\fi

\M{100}Only a small subset of the control codes is legal in limbo, so limbo
processing is straightforward.

\Y\B\1\1\&{static} \&{void} \\{skip\_limbo}(\&{void})\2\2\6
${}\{{}$\1\6
\&{while} (\\{true})\5
${}\{{}$\1\6
\&{if} ${}(\\{loc}>\\{limit}\W\\{get\_line}(\,)\E\\{false}){}$\1\5
\&{return};\2\6
${}{*}(\\{limit}+\T{1})\K\.{'@'};{}$\6
\&{while} ${}({*}\\{loc}\I\.{'@'}){}$\1\5
${}\\{loc}\PP;{}$\2\6
\&{if} ${}(\\{loc}\PP\Z\\{limit}){}$\5
${}\{{}$\1\6
\&{char} \|c${}\K{*}\\{loc}\PP;{}$\7
\&{switch} (\\{ccode}[(\&{eight\_bits}) \|c])\5
${}\{{}$\1\6
\4\&{case} \\{new\_section}:\5
\&{return};\6
\4\&{case} \\{translit\_code}:\5
\X102:Read in transliteration of a character\X\6
\&{break};\6
\4\&{case} \\{format\_code}:\5
\&{case} \.{'@'}:\5
\&{break};\6
\4\&{case} \\{control\_text}:\6
\&{if} ${}(\|c\E\.{'q'}\V\|c\E\.{'Q'}){}$\5
${}\{{}$\1\6
\&{while} ${}((\|c\K{}$(\&{char}) \\{skip\_ahead}(\,))${}\E\.{'@'}){}$\1\5
;\2\6
\&{if} ${}({*}(\\{loc}-\T{1})\I\.{'>'}){}$\1\5
\\{err\_print}(\.{"!\ Double\ @\ should\ b}\)\.{e\ used\ in\ control\ te}\)%
\.{xt"});\2\6
\&{break};\6
\4${}\}{}$\C{ otherwise fall through }\2\6
\4\&{default}:\5
\\{err\_print}(\.{"!\ Double\ @\ should\ b}\)\.{e\ used\ in\ limbo"});\6
\4${}\}{}$\2\6
\4${}\}{}$\2\6
\4${}\}{}$\2\6
\4${}\}{}$\2\par
\fi

\M{101}\B\X8:Predeclaration of procedures\X${}\mathrel+\E{}$\5
\&{static} \&{void} \\{skip\_limbo}(\&{void});\par
\fi

\M{102}\B\X102:Read in transliteration of a character\X${}\E{}$\6
\&{while} ${}(\\{xisspace}({*}\\{loc})\W\\{loc}<\\{limit}){}$\1\5
${}\\{loc}\PP;{}$\2\6
${}\\{loc}\MRL{+{\K}}\T{3};{}$\6
\&{if} ${}(\\{loc}>\\{limit}\V\R\\{xisxdigit}({*}(\\{loc}-\T{3}))\V\R%
\\{xisxdigit}({*}(\\{loc}-\T{2}))\3{-1}\V({*}(\\{loc}-\T{3})\G\.{'0'}\W{*}(%
\\{loc}-\T{3})\Z\.{'7'})\V\R\\{xisspace}({*}(\\{loc}-\T{1}))){}$\1\5
\\{err\_print}(\.{"!\ Improper\ hex\ numb}\)\.{er\ following\ @l"});\2\6
\&{else}\5
${}\{{}$\1\6
\&{unsigned} \&{int} \|i;\6
\&{char} ${}{*}\\{beg};{}$\7
${}\\{sscanf}(\\{loc}-\T{3},\39\.{"\%x"},\39{\AND}\|i);{}$\6
\&{while} ${}(\\{xisspace}({*}\\{loc})\W\\{loc}<\\{limit}){}$\1\5
${}\\{loc}\PP;{}$\2\6
${}\\{beg}\K\\{loc};{}$\6
\&{while} ${}(\\{loc}<\\{limit}\W(\\{xisalpha}({*}\\{loc})\V\\{xisdigit}({*}%
\\{loc})\V{*}\\{loc}\E\.{'\_'})){}$\1\5
${}\\{loc}\PP;{}$\2\6
\&{if} ${}(\\{loc}-\\{beg}\G\\{translit\_length}){}$\1\5
\\{err\_print}(\.{"!\ Replacement\ strin}\)\.{g\ in\ @l\ too\ long"});\2\6
\&{else}\5
${}\{{}$\1\6
${}\\{strncpy}(\\{translit}[\|i-\T{\~200}],\39\\{beg},\39(\&{size\_t})(\\{loc}-%
\\{beg}));{}$\6
${}\\{translit}[\|i-\T{\~200}][\\{loc}-\\{beg}]\K\.{'\\0'};{}$\6
\4${}\}{}$\2\6
\4${}\}{}$\2\par
\U100.\fi

\M{103}Because on some systems the difference between two pointers is a \PB{%
\&{ptrdiff\_t}}
but not an \PB{\&{int}}, we use \.{\%td} to print these quantities.

\Y\B\1\1\&{void} \\{print\_stats}(\&{void})\2\2\6
${}\{{}$\1\6
\\{puts}(\.{"\\nMemory\ usage\ stat}\)\.{istics:"});\6
${}\\{printf}(\.{"\%td\ names\ (out\ of\ \%}\)\.{ld)\\n"},\39(\&{ptrdiff\_t})(%
\\{name\_ptr}-\\{name\_dir}),\39{}$(\&{long}) \\{max\_names});\6
${}\\{printf}(\.{"\%td\ replacement\ tex}\)\.{ts\ (out\ of\ \%ld)\\n"},\39(%
\&{ptrdiff\_t})(\\{text\_ptr}-\\{text\_info}),\39{}$(\&{long}) \\{max\_texts});%
\6
${}\\{printf}(\.{"\%td\ bytes\ (out\ of\ \%}\)\.{ld)\\n"},\39(\&{ptrdiff\_t})(%
\\{byte\_ptr}-\\{byte\_mem}),\39{}$(\&{long}) \\{max\_bytes});\6
${}\\{printf}(\.{"\%td\ tokens\ (out\ of\ }\)\.{\%ld)\\n"},\39(\&{ptrdiff\_t})(%
\\{tok\_ptr}-\\{tok\_mem}),\39{}$(\&{long}) \\{max\_toks});\6
\4${}\}{}$\2\par
\fi

\N{0}{104}Index.
Here is a cross-reference table for \.{CTANGLE}.
All sections in which an identifier is
used are listed with that identifier, except that reserved words are
indexed only when they appear in format definitions, and the appearances
of identifiers in section names are not indexed. Underlined entries
correspond to where the identifier was declared. Error messages and
a few other things like ``ASCII code dependencies'' are indexed here too.
\fi

\inx
\fin
\con
