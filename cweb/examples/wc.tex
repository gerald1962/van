\input cwebmac
% wc: An example of CWEB by Silvio Levy and Donald E. Knuth

\nocon % omit table of contents
\datethis % print date on listing
\def\SPARC{SPARC\-\kern.1em station}


\N{1}{1}An example of {\tt CWEB}.  This example, based on a program by
Klaus Guntermann and Joachim Schrod [{\sl TUGboat\/ \bf7} (1986),
134--137] presents the ``word count'' program from \UNIX/, rewritten in
\.{CWEB} to demonstrate literate programming in \CEE/.  The level of
detail in this document is intentionally high, for didactic purposes;
many of the things spelled out here don't need to be explained in
other programs.

The purpose of \.{wc} is to count lines, words, and/or characters in a
list of files. The number of lines in a file is the number of newline
characters it contains. The number of characters is the file length in bytes.
A ``word'' is a maximal sequence of consecutive characters other than
newline, space, or tab, containing at least one visible ASCII code.
(We assume that the standard ASCII code is in use.)

This version of \.{wc} has a nonstandard ``silent'' option (\.{-s}),
which suppresses printing except for the grand totals over all files.

\fi

\M{2}Most \.{CWEB} programs share a common structure.  It's probably a
good idea to state the overall structure explicitly at the outset,
even though the various parts could all be introduced in unnamed
sections of the code if we wanted to add them piecemeal.

Here, then, is an overview of the file \.{wc.c} that is defined
by this \.{CWEB} program \.{wc.w}:

\Y\B\X3:Header files to include\X\6
\X4:Global variables\X\6
\X20:Functions\X\6
\X5:The main program\X\par
\fi

\M{3}We must include the standard I/O definitions, since we want to send
formatted output to \PB{\\{stdout}} and \PB{\\{stderr}}.

\Y\B\4\X3:Header files to include\X${}\E{}$\6
\8\#\&{include} \.{<stdio.h>}\par
\U2.\fi

\M{4} The \PB{\\{status}} variable will tell the operating system if the run
was
successful or not, and \PB{\\{prog\_name}} is used in case there's an error
message to
be printed.

\Y\B\4\D\.{OK}\5
\T{0}\C{ \PB{\\{status}} code for successful run }\par
\B\4\D\\{usage\_error}\5
\T{1}\C{ \PB{\\{status}} code for improper syntax }\par
\B\4\D\\{cannot\_open\_file}\5
\T{2}\C{ \PB{\\{status}} code for file access error }\par
\Y\B\4\X4:Global variables\X${}\E{}$\6
\&{int} \\{status}${}\K\.{OK}{}$;\C{ exit status of command, initially \PB{%
\.{OK}} }\6
\&{char} ${}{*}\\{prog\_name}{}$;\C{ who we are }\par
\A14.
\U2.\fi

\M{5}Now we come to the general layout of the \PB{\\{main}} function.

\Y\B\4\X5:The main program\X${}\E{}$\6
$\\{main}(\\{argc},\39\\{argv}){}$\1\1\6
\&{int} \\{argc};\C{ the number of arguments on the \UNIX/ command line }\6
\&{char} ${}{*}{*}\\{argv}{}$;\C{ the arguments themselves, an array of strings
}\2\2\6
${}\{{}$\1\6
\X6:Variables local to \PB{\\{main}}\X\6
${}\\{prog\_name}\K\\{argv}[\T{0}];{}$\6
\X7:Set up option selection\X;\6
\X8:Process all the files\X;\6
\X19:Print the grand totals if there were multiple files\X;\6
\\{exit}(\\{status});\6
\4${}\}{}$\2\par
\U2.\fi

\M{6}If the first argument begins with a `\.{-}', the user is choosing
the desired counts and specifying the order in which they should be
displayed.  Each selection is given by the initial character
(lines, words, or characters).  For example, `\.{-cl}' would cause
just the number of characters and the number of lines to be printed,
in that order. The default, if no special argument is given, is `\.{-lwc}'.

We do not process this string now; we simply remember where it is.
It will be used to control the formatting at output time.

If the `\.{-}' is immediately followed by `\.{s}', only summary totals
are printed.

\Y\B\4\X6:Variables local to \PB{\\{main}}\X${}\E{}$\6
\&{int} \\{file\_count};\C{ how many files there are }\6
\&{char} ${}{*}\\{which}{}$;\C{ which counts to print }\6
\&{int} \\{silent}${}\K\T{0}{}$;\C{ nonzero if the silent option was selected }%
\par
\As9\ET12.
\U5.\fi

\M{7}\B\X7:Set up option selection\X${}\E{}$\6
$\\{which}\K\.{"lwc"}{}$;\C{ if no option is given, print all three values }\6
\&{if} ${}(\\{argc}>\T{1}\W{*}\\{argv}[\T{1}]\E\.{'-'}){}$\5
${}\{{}$\1\6
${}\\{argv}[\T{1}]\PP;{}$\6
\&{if} ${}({*}\\{argv}[\T{1}]\E\.{'s'}){}$\1\5
${}\\{silent}\K\T{1},\39\\{argv}[\T{1}]\PP;{}$\2\6
\&{if} ${}({*}\\{argv}[\T{1}]){}$\1\5
${}\\{which}\K\\{argv}[\T{1}];{}$\2\6
${}\\{argc}\MM;{}$\6
${}\\{argv}\PP;{}$\6
\4${}\}{}$\2\6
${}\\{file\_count}\K\\{argc}-\T{1}{}$;\par
\U5.\fi

\M{8}Now we scan the remaining arguments and try to open a file, if
possible.  The file is processed and its statistics are given.
We use a \PB{\&{do}}~\dots~\PB{\&{while}} loop because we should read from the
standard input if no file name is given.

\Y\B\4\X8:Process all the files\X${}\E{}$\6
$\\{argc}\MM;{}$\6
\&{do}\5
${}\{{}$\1\6
\X10:If a file is given, try to open \PB{${*}(\PP\\{argv})$}; \PB{\&{continue}}
if unsuccessful\X;\6
\X13:Initialize pointers and counters\X;\6
\X15:Scan file\X;\6
\X17:Write statistics for file\X;\6
\X11:Close file\X;\6
\X18:Update grand totals\X;\C{ even if there is only one file }\6
\4${}\}{}$\5
\2\5
\&{while} ${}(\MM\\{argc}>\T{0}){}$;\par
\U5.\fi

\M{9}Here's the code to open the file.  A special trick allows us to
handle input from \PB{\\{stdin}} when no name is given.
Recall that the file descriptor to \PB{\\{stdin}} is~0; that's what we
use as the default initial value.

\Y\B\4\X6:Variables local to \PB{\\{main}}\X${}\mathrel+\E{}$\6
\&{int} \\{fd}${}\K\T{0}{}$;\C{ file descriptor, initialized to \PB{\\{stdin}}
}\par
\fi

\M{10}\B\D\.{READ\_ONLY}\5
\T{0}\C{ read access code for system \PB{\\{open}} routine }\par
\Y\B\4\X10:If a file is given, try to open \PB{${*}(\PP\\{argv})$}; \PB{%
\&{continue}} if unsuccessful\X${}\E{}$\6
\&{if} ${}(\\{file\_count}>\T{0}\W(\\{fd}\K\\{open}({*}(\PP\\{argv}),\39\.{READ%
\_ONLY}))<\T{0}){}$\5
${}\{{}$\1\6
${}\\{fprintf}(\\{stderr},\39\.{"\%s:\ cannot\ open\ fil}\)\.{e\ \%s\\n"},\39%
\\{prog\_name},\39{*}\\{argv});{}$\6
${}\\{status}\MRL{{\OR}{\K}}\\{cannot\_open\_file};{}$\6
${}\\{file\_count}\MM;{}$\6
\&{continue};\6
\4${}\}{}$\2\par
\U8.\fi

\M{11}\B\X11:Close file\X${}\E{}$\6
\\{close}(\\{fd});\par
\U8.\fi

\M{12}We will do some homemade buffering in order to speed things up:
Characters
will be read into the \PB{\\{buffer}} array before we process them.
To do this we set up appropriate pointers and counters.

\Y\B\4\D\\{buf\_size}\5
\.{BUFSIZ}\C{ \.{stdio.h}'s \PB{\.{BUFSIZ}} is chosen for efficiency}\par
\Y\B\4\X6:Variables local to \PB{\\{main}}\X${}\mathrel+\E{}$\6
\&{char} \\{buffer}[\\{buf\_size}];\C{ we read the input into this array }\6
\&{register} \&{char} ${}{*}\\{ptr}{}$;\C{ the first unprocessed character in %
\PB{\\{buffer}} }\6
\&{register} \&{char} ${}{*}\\{buf\_end}{}$;\C{ the first unused position in %
\PB{\\{buffer}} }\6
\&{register} \&{int} \|c;\C{ current character, or number of characters just
read }\6
\&{int} \\{in\_word};\C{ are we within a word? }\6
\&{long} \\{word\_count}${},{}$ \\{line\_count}${},{}$ \\{char\_count};\C{
number of words, lines,     and characters found in the file so far }\par
\fi

\M{13}\B\X13:Initialize pointers and counters\X${}\E{}$\6
$\\{ptr}\K\\{buf\_end}\K\\{buffer};{}$\6
${}\\{line\_count}\K\\{word\_count}\K\\{char\_count}\K\T{0};{}$\6
${}\\{in\_word}\K\T{0}{}$;\par
\U8.\fi

\M{14}The grand totals must be initialized to zero at the beginning of the
program. If we made these variables local to \PB{\\{main}}, we would have to
do this initialization explicitly; however, \CEE/'s globals are automatically
zeroed. (Or rather, ``statically zeroed.'') (Get it?)

\Y\B\4\X4:Global variables\X${}\mathrel+\E{}$\6
\&{long} \\{tot\_word\_count}${},{}$ \\{tot\_line\_count}${},{}$ \\{tot\_char%
\_count};\C{ total number of words, lines, and chars }\par
\fi

\M{15}The present section, which does the counting that is \.{wc}'s {\it raison
d'\^etre}, was actually one of the simplest to write. We look at each
character and change state if it begins or ends a word.

\Y\B\4\X15:Scan file\X${}\E{}$\6
\&{while} (\T{1})\5
${}\{{}$\1\6
\X16:Fill \PB{\\{buffer}} if it is empty; \PB{\&{break}} at end of file\X;\6
${}\|c\K{*}\\{ptr}\PP;{}$\6
\&{if} ${}(\|c>\.{'\ '}\W\|c<\T{\~177}){}$\5
${}\{{}$\C{ visible ASCII codes }\1\6
\&{if} ${}(\R\\{in\_word}){}$\5
${}\{{}$\1\6
${}\\{word\_count}\PP;{}$\6
${}\\{in\_word}\K\T{1};{}$\6
\4${}\}{}$\2\6
\&{continue};\6
\4${}\}{}$\2\6
\&{if} ${}(\|c\E\.{'\\n'}){}$\1\5
${}\\{line\_count}\PP;{}$\2\6
\&{else} \&{if} ${}(\|c\I\.{'\ '}\W\|c\I\.{'\\t'}){}$\1\5
\&{continue};\2\6
${}\\{in\_word}\K\T{0}{}$;\C{ \PB{\|c} is newline, space, or tab }\6
\4${}\}{}$\2\par
\U8.\fi

\M{16}Buffered I/O allows us to count the number of characters almost for free.

\Y\B\4\X16:Fill \PB{\\{buffer}} if it is empty; \PB{\&{break}} at end of file%
\X${}\E{}$\6
\&{if} ${}(\\{ptr}\G\\{buf\_end}){}$\5
${}\{{}$\1\6
${}\\{ptr}\K\\{buffer};{}$\6
${}\|c\K\\{read}(\\{fd},\39\\{ptr},\39\\{buf\_size});{}$\6
\&{if} ${}(\|c\Z\T{0}){}$\1\5
\&{break};\2\6
${}\\{char\_count}\MRL{+{\K}}\|c;{}$\6
${}\\{buf\_end}\K\\{buffer}+\|c;{}$\6
\4${}\}{}$\2\par
\U15.\fi

\M{17}It's convenient to output the statistics by defining a new function
\PB{\\{wc\_print}}; then the same function can be used for the totals.
Additionally we must decide here if we know the name of the file
we have processed or if it was just \PB{\\{stdin}}.

\Y\B\4\X17:Write statistics for file\X${}\E{}$\6
\&{if} ${}(\R\\{silent}){}$\5
${}\{{}$\1\6
${}\\{wc\_print}(\\{which},\39\\{char\_count},\39\\{word\_count},\39\\{line%
\_count});{}$\6
\&{if} (\\{file\_count})\1\5
${}\\{printf}(\.{"\ \%s\\n"},\39{*}\\{argv}){}$;\C{ not \PB{\\{stdin}} }\2\6
\&{else}\1\5
\\{printf}(\.{"\\n"});\C{ \PB{\\{stdin}} }\2\6
\4${}\}{}$\2\par
\U8.\fi

\M{18}\B\X18:Update grand totals\X${}\E{}$\6
$\\{tot\_line\_count}\MRL{+{\K}}\\{line\_count};{}$\6
${}\\{tot\_word\_count}\MRL{+{\K}}\\{word\_count};{}$\6
${}\\{tot\_char\_count}\MRL{+{\K}}\\{char\_count}{}$;\par
\U8.\fi

\M{19}We might as well improve a bit on \UNIX/'s \.{wc} by displaying the
number of files too.

\Y\B\4\X19:Print the grand totals if there were multiple files\X${}\E{}$\6
\&{if} ${}(\\{file\_count}>\T{1}\V\\{silent}){}$\5
${}\{{}$\1\6
${}\\{wc\_print}(\\{which},\39\\{tot\_char\_count},\39\\{tot\_word\_count},\39%
\\{tot\_line\_count});{}$\6
\&{if} ${}(\R\\{file\_count}){}$\1\5
\\{printf}(\.{"\\n"});\2\6
\&{else}\1\5
${}\\{printf}(\.{"\ total\ in\ \%d\ file\%s}\)\.{\\n"},\39\\{file\_count},\39%
\\{file\_count}>\T{1}\?\.{"s"}:\.{""});{}$\2\6
\4${}\}{}$\2\par
\U5.\fi

\M{20}Here now is the function that prints the values according to the
specified options.  The calling routine is supposed to supply a
newline. If an invalid option character is found we inform
the user about proper usage of the command. Counts are printed in
8-digit fields so that they will line up in columns.

\Y\B\4\D\\{print\_count}$(\|n)$\5
$\\{printf}(\.{"\%8ld"},\39\|n{}$)\par
\Y\B\4\X20:Functions\X${}\E{}$\6
$\\{wc\_print}(\\{which},\39\\{char\_count},\39\\{word\_count},\39\\{line%
\_count}){}$\1\1\6
\&{char} ${}{*}\\{which}{}$;\C{ which counts to print }\6
\&{long} \\{char\_count}${},{}$ \\{word\_count}${},{}$ \\{line\_count};\C{
given totals }\2\2\6
${}\{{}$\1\6
\&{while} ${}({*}\\{which}){}$\1\6
\&{switch} ${}({*}\\{which}\PP){}$\5
${}\{{}$\1\6
\4\&{case} \.{'l'}:\5
\\{print\_count}(\\{line\_count});\6
\&{break};\6
\4\&{case} \.{'w'}:\5
\\{print\_count}(\\{word\_count});\6
\&{break};\6
\4\&{case} \.{'c'}:\5
\\{print\_count}(\\{char\_count});\6
\&{break};\6
\4\&{default}:\6
\&{if} ${}((\\{status}\AND\\{usage\_error})\E\T{0}){}$\5
${}\{{}$\1\6
${}\\{fprintf}(\\{stderr},\39\.{"\\nUsage:\ \%s\ [-lwc]\ }\)\.{[filename\ ...]%
\\n"},\39\\{prog\_name});{}$\6
${}\\{status}\MRL{{\OR}{\K}}\\{usage\_error};{}$\6
\4${}\}{}$\2\6
\4${}\}{}$\2\2\6
\4${}\}{}$\2\par
\U2.\fi

\M{21}Incidentally, a test of this program against the system \.{wc}
command on a \SPARC\ showed that the ``official'' \.{wc} was slightly
slower. Furthermore, although that \.{wc} gave an appropriate error
message for the options `\.{-abc}', it made no complaints about the
options `\.{-labc}'! Dare we suggest that the system routine might have been
better if its programmer had used a more literate approach?

\fi

\N{1}{22}Index.
Here is a list of the identifiers used, and where they appear. Underlined
entries indicate the place of definition. Error messages are also shown.
\fi

\inx
\fin
\con
