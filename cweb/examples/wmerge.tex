\input cwebmac
% Modified 16 Jan 2002 to agree with COMMON version 3.64

\def\9#1{} % this hack is explained in CWEB manual Appendix F11


\N{1}{1}Introduction.  This file contains the program \.{wmerge},
which takes two or more files and merges them according
to the conventions of \.{CWEB}. Namely, it takes an ordinary \.{.w}
file and and optional \.{.ch} file and sends the corresponding
\.{.w}-style file to standard output (or to a named file),
expanding all ``includes''
that might be specified by \.{@i} in the original \.{.w} file.
(A more precise description appears in the section on ``command line
arguments'' below.)

\Y\B\8\#\&{include} \.{<stdio.h>}\6
\8\#\&{include} \.{<stdlib.h>}\C{ declaration of \PB{\\{getenv}} }\6
\8\#\&{include} \.{<ctype.h>}\C{ definition of \PB{\\{isalpha}}, \PB{%
\\{isdigit}} and so on }\6
\X2:Definitions\X\6
\X3:Predeclarations of functions\X\6
\X6:Functions\X\7
${}\\{main}(\\{ac},\39\\{av}){}$\1\1\6
\&{int} \\{ac};\6
\&{char} ${}{*}{*}\\{av};\2\2{}$\6
${}\{{}$\1\6
${}\\{argc}\K\\{ac};{}$\6
${}\\{argv}\K\\{av};{}$\6
\X31:Set the default options\X;\6
\X41:Scan arguments and open output file\X;\6
\\{reset\_input}(\,);\6
\&{while} (\\{get\_line}(\,))\1\5
\\{put\_line}(\,);\2\6
\\{fflush}(\\{out\_file});\6
\\{check\_complete}(\,);\6
\\{fflush}(\\{out\_file});\6
\&{return} \\{wrap\_up}(\,);\6
\4${}\}{}$\2\par
\fi

\M{2}\B\X2:Definitions\X${}\E{}$\6
\&{typedef} \&{short} \&{boolean};\6
\&{typedef} \&{unsigned} \&{char} \&{eight\_bits};\6
\&{typedef} \&{char} \&{ASCII};\C{ type of characters inside \.{WEB} }\par
\As5, 7, 8, 23, 30\ETs40.
\U1.\fi

\M{3}We predeclare some standard string-handling functions here instead of
including their system header files, because the names of the header files
are not as standard as the names of the functions. (There's confusion
between \.{<string.h>} and \.{<strings.h>}.)

\Y\B\4\X3:Predeclarations of functions\X${}\E{}$\6
\&{extern} \&{size\_t} \\{strlen}(\,);\C{ length of string }\6
\&{extern} \&{char} ${}{*}\\{strcpy}(\,){}$;\C{ copy one string to another }\6
\&{extern} \&{int} \\{strncmp}(\,);\C{ compare up to $n$ string characters }\6
\&{extern} \&{char} ${}{*}\\{strncpy}(\,){}$;\C{ copy up to $n$ string
characters }\par
\As4, 24\ETs32.
\U1.\fi

\M{4}\B\X3:Predeclarations of functions\X${}\mathrel+\E{}$\par
\fi

\M{5}The lowest level of input to the \.{WEB} programs
is performed by \PB{\\{input\_ln}}, which must be told which file to read from.
The return value of \PB{\\{input\_ln}} is 1 if the read is successful and 0 if
not (generally this means the file has ended).
The characters of the next line of the file
are copied into the \PB{\\{buffer}} array,
and the global variable \PB{\\{limit}} is set to the first unoccupied position.
Trailing blanks are ignored. The value of \PB{\\{limit}} must be strictly less
than \PB{\\{buf\_size}}, so that \PB{$\\{buffer}[\\{buf\_size}-\T{1}]$} is
never filled.

Some of the routines use the fact that it is safe to refer to
\PB{${*}(\\{limit}+\T{2})$} without overstepping the bounds of the array.

\Y\B\4\D\\{buf\_size}\5
\T{4096}\par
\Y\B\4\X2:Definitions\X${}\mathrel+\E{}$\6
\&{ASCII} \\{buffer}[\\{buf\_size}];\C{ where each line of input goes }\6
\&{ASCII} ${}{*}\\{buffer\_end}\K\\{buffer}+\\{buf\_size}-\T{2}{}$;\C{ end of %
\PB{\\{buffer}} }\6
\&{ASCII} ${}{*}\\{limit}{}$;\C{ points to the last character in the buffer }\6
\&{ASCII} ${}{*}\\{loc}{}$;\C{ points to the next character to be read from the
buffer }\par
\fi

\M{6}In the unlikely event that your standard I/O library does not
support \PB{\\{feof}}, \PB{\\{getc}} and \PB{\\{ungetc}}, you may have to
change things here.

Incidentally, here's a curious fact about \.{CWEB} for those of you
who are reading this file as an example of \.{CWEB} programming.
The file \.{stdio.h} includes a typedef for
the identifier \PB{\&{FILE}}, which is not, strictly speaking, part of \CEE/.
It turns out \.{CWEAVE} knows that \PB{\&{FILE}} is a reserved word (after all,
\PB{\&{FILE}} is almost as common as \PB{\&{int}}); indeed, \.{CWEAVE} knows
all
the types of the ISO standard \CEE/ library. But
if you're using other types like {\bf caddr\_t},
which is defined in \.{/usr/include/sys/types.h}, you should let
\.{WEAVE} know that this is a type, either by including the \.{.h} file
at \.{WEB} time (saying \.{@i /usr/include/sys/types.h}), or by
using \.{WEB}'s format command (saying \.{@f caddr\_t int}).  Either of
these will make {\bf caddr\_t} be treated in the same way as \PB{\&{int}}.

\Y\B\4\X6:Functions\X${}\E{}$\6
\\{input\_ln}(\\{fp})\C{ copies a line into \PB{\\{buffer}} or returns 0 }\6
\1\1\6
\&{FILE} ${}{*}\\{fp}{}$;\C{ what file to read from }\2\2\6
${}\{{}$\1\6
\&{register} \&{int} \|c${}\K\.{EOF}{}$;\C{ character read; initialized so some
compilers won't complain }\6
\&{register} \&{char} ${}{*}\|k{}$;\C{ where next character goes }\7
\&{if} (\\{feof}(\\{fp}))\1\5
\&{return} (\T{0});\C{ we have hit end-of-file }\2\6
${}\\{limit}\K\|k\K\\{buffer}{}$;\C{ beginning of buffer }\6
\&{while} ${}(\|k\Z\\{buffer\_end}\W(\|c\K\\{getc}(\\{fp}))\I\.{EOF}\W\|c\I\.{'%
\\n'}){}$\1\6
\&{if} ${}(({*}(\|k\PP)\K\|c)\I\.{'\ '}){}$\1\5
${}\\{limit}\K\|k;{}$\2\2\6
\&{if} ${}(\|k>\\{buffer\_end}){}$\1\6
\&{if} ${}((\|c\K\\{getc}(\\{fp}))\I\.{EOF}\W\|c\I\.{'\\n'}){}$\5
${}\{{}$\1\6
${}\\{ungetc}(\|c,\39\\{fp});{}$\6
${}\\{loc}\K\\{buffer};{}$\6
\\{err\_print}(\.{"!\ Input\ line\ too\ lo}\)\.{ng"});\6
\4${}\}{}$\2\2\6
\&{if} ${}(\|c\E\.{EOF}\W\\{limit}\E\\{buffer}){}$\1\5
\&{return} (\T{0});\C{ there was nothing after     the last newline }\2\6
\&{return} (\T{1});\6
\4${}\}{}$\2\par
\As9, 13, 15, 17, 22, 25, 28\ETs33.
\U1.\fi

\M{7}Now comes the problem of deciding which file to read from next.
Recall that the actual text that \.{CWEB} should process comes from two
streams: a \PB{\\{web\_file}}, which can contain possibly nested include
commands \.{@i}, and a \PB{\\{change\_file}}, which might also contain
includes.  The \PB{\\{web\_file}} together with the currently open include
files form a stack \PB{\\{file}}, whose names are stored in a parallel stack
\PB{\\{file\_name}}.  The boolean \PB{\\{changing}} tells whether or not we're
reading
from the \PB{\\{change\_file}}.

The line number of each open file is also kept for error reporting.

\Y\B\F\\{line}\5
\|x\C{ make \PB{\\{line}} an unreserved word }\par
\B\4\D\\{max\_include\_depth}\5
\T{10}\C{ maximum number of source files open   simultaneously, not counting
the change file }\par
\B\4\D\\{max\_file\_name\_length}\5
\T{60}\par
\B\4\D\\{cur\_file}\5
\\{file}[\\{include\_depth}]\C{ current file }\par
\B\4\D\\{cur\_file\_name}\5
\\{file\_name}[\\{include\_depth}]\C{ current file name }\par
\B\4\D\\{cur\_line}\5
\\{line}[\\{include\_depth}]\C{ number of current line in current file }\par
\B\4\D\\{web\_file}\5
\\{file}[\T{0}]\C{ main source file }\par
\B\4\D\\{web\_file\_name}\5
\\{file\_name}[\T{0}]\C{ main source file name }\par
\Y\B\4\X2:Definitions\X${}\mathrel+\E{}$\6
\&{int} \\{include\_depth};\C{ current level of nesting }\6
\&{FILE} ${}{*}\\{file}[\\{max\_include\_depth}]{}$;\C{ stack of non-change
files }\6
\&{FILE} ${}{*}\\{change\_file}{}$;\C{ change file }\6
\&{char} \\{file\_name}[\\{max\_include\_depth}][\\{max\_file\_name\_length}];%
\C{ stack of non-change file names }\6
\&{char} \\{change\_file\_name}[\\{max\_file\_name\_length}];\C{ name of change
file }\6
\&{char} \\{alt\_web\_file\_name}[\\{max\_file\_name\_length}];\C{ alternate
name to try }\6
\&{int} \\{line}[\\{max\_include\_depth}];\C{ number of current line in the
stacked files }\6
\&{int} \\{change\_line};\C{ number of current line in change file }\6
\&{int} \\{change\_depth};\C{ where \.{@y} originated during a change }\6
\&{boolean} \\{input\_has\_ended};\C{ if there is no more input }\6
\&{boolean} \\{changing};\C{ if the current line is from \PB{\\{change\_file}}
}\6
\&{boolean} \\{web\_file\_open}${}\K\T{0}{}$;\C{ if the web file is being read
}\par
\fi

\M{8}When \PB{$\\{changing}\K\T{0}$}, the next line of \PB{\\{change\_file}} is
kept in
\PB{\\{change\_buffer}}, for purposes of comparison with the next
line of \PB{\\{cur\_file}}. After the change file has been completely input, we
set \PB{$\\{change\_limit}\K\\{change\_buffer}$},
so that no further matches will be made.

Here's a shorthand expression for inequality between the two lines:

\Y\B\4\D\\{lines\_dont\_match}\5
$(\\{change\_limit}-\\{change\_buffer}\I\\{limit}-\\{buffer}\V\\{strncmp}(%
\\{buffer},\39\\{change\_buffer},\39\\{limit}-\\{buffer}){}$)\par
\Y\B\4\X2:Definitions\X${}\mathrel+\E{}$\6
\&{char} \\{change\_buffer}[\\{buf\_size}];\C{ next line of \PB{\\{change%
\_file}} }\6
\&{char} ${}{*}\\{change\_limit}{}$;\C{ points to the last character in \PB{%
\\{change\_buffer}} }\par
\fi

\M{9}Procedure \PB{\\{prime\_the\_change\_buffer}} sets \PB{\\{change\_buffer}}
in
preparation for the next matching operation. Since blank lines in the change
file are not used for matching, we have
\PB{$(\\{change\_limit}\E\\{change\_buffer}\W\R\\{changing})$} if and only if
the change file is exhausted. This procedure is called only when
\PB{\\{changing}} is 1; hence error messages will be reported correctly.

\Y\B\4\X6:Functions\X${}\mathrel+\E{}$\6
\1\1\&{void} \\{prime\_the\_change\_buffer}(\,)\2\2\6
${}\{{}$\1\6
${}\\{change\_limit}\K\\{change\_buffer}{}$;\C{ this value is used if the
change file ends }\6
\X10:Skip over comment lines in the change file; \PB{\&{return}} if end of file%
\X;\6
\X11:Skip to the next nonblank line; \PB{\&{return}} if end of file\X;\6
\X12:Move \PB{\\{buffer}} and \PB{\\{limit}} to \PB{\\{change\_buffer}} and %
\PB{\\{change\_limit}}\X;\6
\4${}\}{}$\2\par
\fi

\M{10}While looking for a line that begins with \.{@x} in the change file, we
allow lines that begin with \.{@}, as long as they don't begin with \.{@y},
\.{@z} or \.{@i} (which would probably mean that the change file is fouled up).

\Y\B\4\X10:Skip over comment lines in the change file; \PB{\&{return}} if end
of file\X${}\E{}$\6
\&{while} (\T{1})\5
${}\{{}$\1\6
${}\\{change\_line}\PP;{}$\6
\&{if} ${}(\R\\{input\_ln}(\\{change\_file})){}$\1\5
\&{return};\2\6
\&{if} ${}(\\{limit}<\\{buffer}+\T{2}){}$\1\5
\&{continue};\2\6
\&{if} ${}(\\{buffer}[\T{0}]\I\.{'@'}){}$\1\5
\&{continue};\2\6
\&{if} (\\{isupper}(\\{buffer}[\T{1}]))\1\5
${}\\{buffer}[\T{1}]\K\\{tolower}(\\{buffer}[\T{1}]);{}$\2\6
\&{if} ${}(\\{buffer}[\T{1}]\E\.{'x'}){}$\1\5
\&{break};\2\6
\&{if} ${}(\\{buffer}[\T{1}]\E\.{'y'}\V\\{buffer}[\T{1}]\E\.{'z'}\V\\{buffer}[%
\T{1}]\E\.{'i'}){}$\5
${}\{{}$\1\6
${}\\{loc}\K\\{buffer}+\T{2};{}$\6
\\{err\_print}(\.{"!\ Missing\ @x\ in\ cha}\)\.{nge\ file"});\6
\4${}\}{}$\2\6
\4${}\}{}$\2\par
\U9.\fi

\M{11}Here we are looking at lines following the \.{@x}.

\Y\B\4\X11:Skip to the next nonblank line; \PB{\&{return}} if end of file\X${}%
\E{}$\6
\&{do}\5
${}\{{}$\1\6
${}\\{change\_line}\PP;{}$\6
\&{if} ${}(\R\\{input\_ln}(\\{change\_file})){}$\5
${}\{{}$\1\6
\\{err\_print}(\.{"!\ Change\ file\ ended}\)\.{\ after\ @x"});\6
\&{return};\6
\4${}\}{}$\2\6
\4${}\}{}$\2\5
\&{while} ${}(\\{limit}\E\\{buffer}){}$;\par
\U9.\fi

\M{12}\B\X12:Move \PB{\\{buffer}} and \PB{\\{limit}} to \PB{\\{change\_buffer}}
and \PB{\\{change\_limit}}\X${}\E{}$\6
${}\{{}$\1\6
${}\\{change\_limit}\K\\{change\_buffer}+(\\{limit}-\\{buffer});{}$\6
${}\\{strncpy}(\\{change\_buffer},\39\\{buffer},\39\\{limit}-\\{buffer}+%
\T{1});{}$\6
\4${}\}{}$\2\par
\Us9\ET13.\fi

\M{13}The following procedure is used to see if the next change entry should
go into effect; it is called only when \PB{\\{changing}} is 0.
The idea is to test whether or not the current
contents of \PB{\\{buffer}} matches the current contents of \PB{\\{change%
\_buffer}}.
If not, there's nothing more to do; but if so, a change is called for:
All of the text down to the \.{@y} is supposed to match. An error
message is issued if any discrepancy is found. Then the procedure
prepares to read the next line from \PB{\\{change\_file}}.

This procedure is called only when \PB{$\\{buffer}<\\{limit}$}, i.e., when the
current line is nonempty.

\Y\B\4\X6:Functions\X${}\mathrel+\E{}$\6
\1\1\&{void} \\{check\_change}(\,)\C{ switches to \PB{\\{change\_file}} if the
buffers match }\2\2\6
${}\{{}$\1\6
\&{int} \|n${}\K\T{0}{}$;\C{ the number of discrepancies found }\7
\&{if} (\\{lines\_dont\_match})\1\5
\&{return};\2\6
\&{while} (\T{1})\5
${}\{{}$\1\6
${}\\{changing}\K\T{1};{}$\6
${}\\{change\_line}\PP;{}$\6
\&{if} ${}(\R\\{input\_ln}(\\{change\_file})){}$\5
${}\{{}$\1\6
\\{err\_print}(\.{"!\ Change\ file\ ended}\)\.{\ before\ @y"});\6
${}\\{change\_limit}\K\\{change\_buffer};{}$\6
${}\\{changing}\K\T{0};{}$\6
\&{return};\6
\4${}\}{}$\2\6
\&{if} ${}(\\{limit}>\\{buffer}+\T{1}\W\\{buffer}[\T{0}]\E\.{'@'}){}$\5
${}\{{}$\1\6
\&{char} \\{xyz\_code}${}\K\\{isupper}(\\{buffer}[\T{1}])\?\\{tolower}(%
\\{buffer}[\T{1}]):\\{buffer}[\T{1}];{}$\7
\X14:If the current line starts with \.{@y}, report any discrepancies and \PB{%
\&{return}}\X;\6
\4${}\}{}$\2\6
\X12:Move \PB{\\{buffer}} and \PB{\\{limit}} to \PB{\\{change\_buffer}} and %
\PB{\\{change\_limit}}\X;\6
${}\\{changing}\K\T{0};{}$\6
${}\\{cur\_line}\PP;{}$\6
\&{while} ${}(\R\\{input\_ln}(\\{cur\_file})){}$\5
${}\{{}$\C{ pop the stack or quit }\1\6
\&{if} ${}(\\{include\_depth}\E\T{0}){}$\5
${}\{{}$\1\6
\\{err\_print}(\.{"!\ CWEB\ file\ ended\ d}\)\.{uring\ a\ change"});\6
${}\\{input\_has\_ended}\K\T{1};{}$\6
\&{return};\6
\4${}\}{}$\2\6
${}\\{include\_depth}\MM;{}$\6
${}\\{cur\_line}\PP;{}$\6
\4${}\}{}$\2\6
\&{if} (\\{lines\_dont\_match})\1\5
${}\|n\PP;{}$\2\6
\4${}\}{}$\2\6
\4${}\}{}$\2\par
\fi

\M{14}\B\X14:If the current line starts with \.{@y}, report any discrepancies
and \PB{\&{return}}\X${}\E{}$\6
\&{if} ${}(\\{xyz\_code}\E\.{'x'}\V\\{xyz\_code}\E\.{'z'}){}$\5
${}\{{}$\1\6
${}\\{loc}\K\\{buffer}+\T{2};{}$\6
\\{err\_print}(\.{"!\ Where\ is\ the\ matc}\)\.{hing\ @y?"});\6
\4${}\}{}$\2\6
\&{else} \&{if} ${}(\\{xyz\_code}\E\.{'y'}){}$\5
${}\{{}$\1\6
\&{if} ${}(\|n>\T{0}){}$\5
${}\{{}$\1\6
${}\\{loc}\K\\{buffer}+\T{2};{}$\6
${}\\{fprintf}(\\{stderr},\39\.{"\\n!\ Hmm...\ \%d\ "},\39\|n);{}$\6
\\{err\_print}(\.{"of\ the\ preceding\ li}\)\.{nes\ failed\ to\ match"});\6
\4${}\}{}$\2\6
${}\\{change\_depth}\K\\{include\_depth};{}$\6
\&{return};\6
\4${}\}{}$\2\par
\U13.\fi

\M{15}The \PB{\\{reset\_input}} procedure gets the program ready to read the
user's \.{WEB} input.

\Y\B\4\X6:Functions\X${}\mathrel+\E{}$\6
\1\1\&{void} \\{reset\_input}(\,)\2\2\6
${}\{{}$\1\6
${}\\{limit}\K\\{buffer};{}$\6
${}\\{loc}\K\\{buffer}+\T{1};{}$\6
${}\\{buffer}[\T{0}]\K\.{'\ '};{}$\6
\X16:Open input files\X;\6
${}\\{include\_depth}\K\T{0};{}$\6
${}\\{cur\_line}\K\T{0};{}$\6
${}\\{change\_line}\K\T{0};{}$\6
${}\\{change\_depth}\K\\{include\_depth};{}$\6
${}\\{changing}\K\T{1};{}$\6
\\{prime\_the\_change\_buffer}(\,);\6
${}\\{changing}\K\R\\{changing};{}$\6
${}\\{limit}\K\\{buffer};{}$\6
${}\\{loc}\K\\{buffer}+\T{1};{}$\6
${}\\{buffer}[\T{0}]\K\.{'\ '};{}$\6
${}\\{input\_has\_ended}\K\T{0};{}$\6
\4${}\}{}$\2\par
\fi

\M{16}The following code opens the input files.

\Y\B\4\X16:Open input files\X${}\E{}$\6
\&{if} ${}((\\{web\_file}\K\\{fopen}(\\{web\_file\_name},\39\.{"r"}))\E%
\NULL){}$\5
${}\{{}$\1\6
${}\\{strcpy}(\\{web\_file\_name},\39\\{alt\_web\_file\_name});{}$\6
\&{if} ${}((\\{web\_file}\K\\{fopen}(\\{web\_file\_name},\39\.{"r"}))\E%
\NULL){}$\1\5
${}\\{fatal}(\.{"!\ Cannot\ open\ input}\)\.{\ file\ "},\39\\{web\_file%
\_name});{}$\2\6
\4${}\}{}$\2\6
${}\\{web\_file\_open}\K\T{1};{}$\6
\&{if} ${}((\\{change\_file}\K\\{fopen}(\\{change\_file\_name},\39\.{"r"}))\E%
\NULL){}$\1\5
${}\\{fatal}(\.{"!\ Cannot\ open\ chang}\)\.{e\ file\ "},\39\\{change\_file%
\_name}){}$;\2\par
\U15.\fi

\M{17}The \PB{\\{get\_line}} procedure is called when \PB{$\\{loc}>\\{limit}$};
it puts the next
line of merged input into the buffer and updates the other variables
appropriately. A space is placed at the right end of the line.
This procedure returns \PB{$\R\\{input\_has\_ended}$} because we often want to
check the value of that variable after calling the procedure.

\Y\B\4\X6:Functions\X${}\mathrel+\E{}$\6
\1\1\&{int} \\{get\_line}(\,)\C{ inputs the next line }\2\2\6
${}\{{}$\1\6
\4\\{restart}:\6
\&{if} ${}(\\{changing}\W\\{include\_depth}\E\\{change\_depth}){}$\1\5
\X21:Read from \PB{\\{change\_file}} and maybe turn off \PB{\\{changing}}\X;\2\6
\&{if} ${}(\R\\{changing}\V\\{include\_depth}>\\{change\_depth}){}$\5
${}\{{}$\1\6
\X20:Read from \PB{\\{cur\_file}} and maybe turn on \PB{\\{changing}}\X;\6
\&{if} ${}(\\{changing}\W\\{include\_depth}\E\\{change\_depth}){}$\1\5
\&{goto} \\{restart};\2\6
\4${}\}{}$\2\6
\&{if} (\\{input\_has\_ended})\1\5
\&{return} \T{0};\2\6
${}\\{loc}\K\\{buffer};{}$\6
${}{*}\\{limit}\K\.{'\ '};{}$\6
\&{if} ${}(\\{buffer}[\T{0}]\E\.{'@'}\W(\\{buffer}[\T{1}]\E\.{'i'}\V\\{buffer}[%
\T{1}]\E\.{'I'})){}$\5
${}\{{}$\1\6
${}\\{loc}\K\\{buffer}+\T{2};{}$\6
${}{*}\\{limit}\K\.{'"'};{}$\6
\&{while} ${}({*}\\{loc}\E\.{'\ '}\V{*}\\{loc}\E\.{'\\t'}){}$\1\5
${}\\{loc}\PP;{}$\2\6
\&{if} ${}(\\{loc}\G\\{limit}){}$\5
${}\{{}$\1\6
\\{err\_print}(\.{"!\ Include\ file\ name}\)\.{\ not\ given"});\6
\&{goto} \\{restart};\6
\4${}\}{}$\2\6
\&{if} ${}(\\{include\_depth}\G\\{max\_include\_depth}-\T{1}){}$\5
${}\{{}$\1\6
\\{err\_print}(\.{"!\ Too\ many\ nested\ i}\)\.{ncludes"});\6
\&{goto} \\{restart};\6
\4${}\}{}$\2\6
${}\\{include\_depth}\PP{}$;\C{ push input stack }\6
\X19:Try to open include file, abort push if unsuccessful, go to \PB{%
\\{restart}}\X;\6
\4${}\}{}$\2\6
\&{return} \T{1};\6
\4${}\}{}$\2\7
\1\1\&{void} \\{put\_line}(\,)\2\2\6
${}\{{}$\1\6
\&{char} ${}{*}\\{ptr}\K\\{buffer};{}$\7
\&{while} ${}(\\{ptr}<\\{limit}){}$\1\5
${}\\{putc}({*}\\{ptr}\PP,\39\\{out\_file});{}$\2\6
${}\\{putc}(\.{'\\n'},\39\\{out\_file});{}$\6
\4${}\}{}$\2\par
\fi

\M{18}When an \.{@i} line is found in the \PB{\\{cur\_file}}, we must
temporarily
stop reading it and start reading from the named include file.  The
\.{@i} line should give a complete file name with or without
double quotes.
If the environment variable \.{CWEBINPUTS} is set, or if the compiler flag
of the same name was defined at compile time,
\.{CWEB} will look for include files in the directory thus named, if
it cannot find them in the current directory.
(Colon-separated paths are not supported.)
The remainder of the \.{@i} line after the file name is ignored.

\Y\B\4\D\\{too\_long}$()$\6
${}\{{}$\1\6
${}\\{include\_depth}\MM;{}$\6
\\{err\_print}(\.{"!\ Include\ file\ name}\)\.{\ too\ long"});\6
\&{goto} \\{restart};\6
\4${}\}{}$\2\par
\fi

\M{19}\B\X19:Try to open include file, abort push if unsuccessful, go to \PB{%
\\{restart}}\X${}\E{}$\6
${}\{{}$\1\6
\&{char} \\{temp\_file\_name}[\\{max\_file\_name\_length}];\6
\&{char} ${}{*}\\{cur\_file\_name\_end}\K\\{cur\_file\_name}+\\{max\_file\_name%
\_length}-\T{1};{}$\6
\&{char} ${}{*}\|k\K\\{cur\_file\_name},{}$ ${}{*}\\{kk};{}$\6
\&{int} \|l;\C{ length of file name }\7
\&{if} ${}({*}\\{loc}\E\.{'"'}){}$\5
${}\{{}$\1\6
${}\\{loc}\PP;{}$\6
\&{while} ${}({*}\\{loc}\I\.{'"'}\W\|k\Z\\{cur\_file\_name\_end}){}$\1\5
${}{*}\|k\PP\K{*}\\{loc}\PP;{}$\2\6
\&{if} ${}(\\{loc}\E\\{limit}){}$\1\5
${}\|k\K\\{cur\_file\_name\_end}+\T{1}{}$;\C{ unmatched quote is `too long' }\2%
\6
\4${}\}{}$\2\6
\&{else}\1\6
\&{while} ${}({*}\\{loc}\I\.{'\ '}\W{*}\\{loc}\I\.{'\\t'}\W{*}\\{loc}\I\.{'"'}%
\W\|k\Z\\{cur\_file\_name\_end}){}$\1\5
${}{*}\|k\PP\K{*}\\{loc}\PP;{}$\2\2\6
\&{if} ${}(\|k>\\{cur\_file\_name\_end}){}$\1\5
\\{too\_long}(\,);\2\6
${}{*}\|k\K\.{'\\0'};{}$\6
\&{if} ${}((\\{cur\_file}\K\\{fopen}(\\{cur\_file\_name},\39\.{"r"}))\I%
\NULL){}$\5
${}\{{}$\1\6
${}\\{cur\_line}\K\T{0};{}$\6
\&{goto} \\{restart};\C{ success }\6
\4${}\}{}$\2\6
${}\\{kk}\K\\{getenv}(\.{"CWEBINPUTS"});{}$\6
\&{if} ${}(\\{kk}\I\NULL){}$\5
${}\{{}$\1\6
\&{if} ${}((\|l\K\\{strlen}(\\{kk}))>\\{max\_file\_name\_length}-\T{2}){}$\1\5
\\{too\_long}(\,);\2\6
${}\\{strcpy}(\\{temp\_file\_name},\39\\{kk});{}$\6
\4${}\}{}$\2\6
\&{else}\5
${}\{{}$\6
\8\#\&{ifdef} \.{CWEBINPUTS}\1\6
\&{if} ${}((\|l\K\\{strlen}(\.{CWEBINPUTS}))>\\{max\_file\_name\_length}-%
\T{2}){}$\1\5
\\{too\_long}(\,);\2\6
${}\\{strcpy}(\\{temp\_file\_name},\39\.{CWEBINPUTS});{}$\6
\8\#\&{else}\6
${}\|l\K\T{0};{}$\6
\8\#\&{endif}\C{ \PB{\.{CWEBINPUTS}} }\6
\4${}\}{}$\2\6
\&{if} ${}(\|l>\T{0}){}$\5
${}\{{}$\1\6
\&{if} ${}(\|k+\|l+\T{2}\G\\{cur\_file\_name\_end}){}$\1\5
\\{too\_long}(\,);\2\6
\&{for} ( ; ${}\|k\G\\{cur\_file\_name};{}$ ${}\|k\MM){}$\1\5
${}{*}(\|k+\|l+\T{1})\K{*}\|k;{}$\2\6
${}\\{strcpy}(\\{cur\_file\_name},\39\\{temp\_file\_name});{}$\6
${}\\{cur\_file\_name}[\|l]\K\.{'/'}{}$;\C{ \UNIX/ pathname separator }\6
\&{if} ${}((\\{cur\_file}\K\\{fopen}(\\{cur\_file\_name},\39\.{"r"}))\I%
\NULL){}$\5
${}\{{}$\1\6
${}\\{cur\_line}\K\T{0};{}$\6
\&{goto} \\{restart};\C{ success }\6
\4${}\}{}$\2\6
\4${}\}{}$\2\6
${}\\{include\_depth}\MM;{}$\6
\\{err\_print}(\.{"!\ Cannot\ open\ inclu}\)\.{de\ file"});\6
\&{goto} \\{restart};\6
\4${}\}{}$\2\par
\U17.\fi

\M{20}\B\X20:Read from \PB{\\{cur\_file}} and maybe turn on \PB{\\{changing}}%
\X${}\E{}$\6
${}\{{}$\1\6
${}\\{cur\_line}\PP;{}$\6
\&{while} ${}(\R\\{input\_ln}(\\{cur\_file})){}$\5
${}\{{}$\C{ pop the stack or quit }\1\6
\&{if} ${}(\\{include\_depth}\E\T{0}){}$\5
${}\{{}$\1\6
${}\\{input\_has\_ended}\K\T{1};{}$\6
\&{break};\6
\4${}\}{}$\2\6
\&{else}\5
${}\{{}$\1\6
\\{fclose}(\\{cur\_file});\6
${}\\{include\_depth}\MM;{}$\6
\&{if} ${}(\\{changing}\W\\{include\_depth}\E\\{change\_depth}){}$\1\5
\&{break};\2\6
${}\\{cur\_line}\PP;{}$\6
\4${}\}{}$\2\6
\4${}\}{}$\2\6
\&{if} ${}(\R\\{changing}\W\R\\{input\_has\_ended}){}$\1\6
\&{if} ${}(\\{limit}-\\{buffer}\E\\{change\_limit}-\\{change\_buffer}){}$\1\6
\&{if} ${}(\\{buffer}[\T{0}]\E\\{change\_buffer}[\T{0}]){}$\1\6
\&{if} ${}(\\{change\_limit}>\\{change\_buffer}){}$\1\5
\\{check\_change}(\,);\2\2\2\2\6
\4${}\}{}$\2\par
\U17.\fi

\M{21}\B\X21:Read from \PB{\\{change\_file}} and maybe turn off \PB{%
\\{changing}}\X${}\E{}$\6
${}\{{}$\1\6
${}\\{change\_line}\PP;{}$\6
\&{if} ${}(\R\\{input\_ln}(\\{change\_file})){}$\5
${}\{{}$\1\6
\\{err\_print}(\.{"!\ Change\ file\ ended}\)\.{\ without\ @z"});\6
${}\\{buffer}[\T{0}]\K\.{'@'};{}$\6
${}\\{buffer}[\T{1}]\K\.{'z'};{}$\6
${}\\{limit}\K\\{buffer}+\T{2};{}$\6
\4${}\}{}$\2\6
\&{if} ${}(\\{limit}>\\{buffer}){}$\5
${}\{{}$\C{ check if the change has ended }\1\6
${}{*}\\{limit}\K\.{'\ '};{}$\6
\&{if} ${}(\\{buffer}[\T{0}]\E\.{'@'}){}$\5
${}\{{}$\1\6
\&{if} (\\{isupper}(\\{buffer}[\T{1}]))\1\5
${}\\{buffer}[\T{1}]\K\\{tolower}(\\{buffer}[\T{1}]);{}$\2\6
\&{if} ${}(\\{buffer}[\T{1}]\E\.{'x'}\V\\{buffer}[\T{1}]\E\.{'y'}){}$\5
${}\{{}$\1\6
${}\\{loc}\K\\{buffer}+\T{2};{}$\6
\\{err\_print}(\.{"!\ Where\ is\ the\ matc}\)\.{hing\ @z?"});\6
\4${}\}{}$\2\6
\&{else} \&{if} ${}(\\{buffer}[\T{1}]\E\.{'z'}){}$\5
${}\{{}$\1\6
\\{prime\_the\_change\_buffer}(\,);\6
${}\\{changing}\K\R\\{changing};{}$\6
\4${}\}{}$\2\6
\4${}\}{}$\2\6
\4${}\}{}$\2\6
\4${}\}{}$\2\par
\U17.\fi

\M{22}At the end of the program, we will tell the user if the change file
had a line that didn't match any relevant line in \PB{\\{web\_file}}.

\Y\B\4\X6:Functions\X${}\mathrel+\E{}$\6
\1\1\&{void} \\{check\_complete}(\,)\2\2\6
${}\{{}$\1\6
\&{if} ${}(\\{change\_limit}\I\\{change\_buffer}){}$\5
${}\{{}$\C{ \PB{\\{changing}} is 0 }\1\6
${}\\{strncpy}(\\{buffer},\39\\{change\_buffer},\39\\{change\_limit}-\\{change%
\_buffer}+\T{1});{}$\6
${}\\{limit}\K\\{buffer}+(\&{int})(\\{change\_limit}-\\{change\_buffer});{}$\6
${}\\{changing}\K\T{1};{}$\6
${}\\{change\_depth}\K\\{include\_depth};{}$\6
${}\\{loc}\K\\{buffer};{}$\6
\\{err\_print}(\.{"!\ Change\ file\ entry}\)\.{\ did\ not\ match"});\6
\4${}\}{}$\2\6
\4${}\}{}$\2\par
\fi

\N{1}{23}Reporting errors to the user.
A global variable called \PB{\\{history}} will contain one of four values
at the end of every run: \PB{\\{spotless}} means that no unusual messages were
printed; \PB{\\{harmless\_message}} means that a message of possible interest
was printed but no serious errors were detected; \PB{\\{error\_message}} means
that
at least one error was found; \PB{\\{fatal\_message}} means that the program
terminated abnormally. The value of \PB{\\{history}} does not influence the
behavior of the program; it is simply computed for the convenience
of systems that might want to use such information.

\Y\B\4\D\\{spotless}\5
\T{0}\C{ \PB{\\{history}} value for normal jobs }\par
\B\4\D\\{harmless\_message}\5
\T{1}\C{ \PB{\\{history}} value when non-serious info was printed }\par
\B\4\D\\{error\_message}\5
\T{2}\C{ \PB{\\{history}} value when an error was noted }\par
\B\4\D\\{fatal\_message}\5
\T{3}\C{ \PB{\\{history}} value when we had to stop prematurely }\par
\B\4\D\\{mark\_harmless}\6
${}\{{}$\1\6
\&{if} ${}(\\{history}\E\\{spotless}){}$\1\5
${}\\{history}\K\\{harmless\_message};{}$\2\6
\4${}\}{}$\2\par
\B\4\D\\{mark\_error}\5
$\\{history}\K{}$\\{error\_message}\par
\Y\B\4\X2:Definitions\X${}\mathrel+\E{}$\6
\&{int} \\{history}${}\K\\{spotless}{}$;\C{ indicates how bad this run was }\par
\fi

\M{24}The command `\PB{\\{err\_print}(\.{"!\ Error\ message"})}' will report a
syntax error to
the user, by printing the error message at the beginning of a new line and
then giving an indication of where the error was spotted in the source file.
Note that no period follows the error message, since the error routine
will automatically supply a period. A newline is automatically supplied
if the string begins with \PB{\.{"!"}}.

The actual error indications are provided by a procedure called \PB{\&{error}}.

\Y\B\4\X3:Predeclarations of functions\X${}\mathrel+\E{}$\6
\&{void} \\{err\_print}(\,);\par
\fi

\M{25}
\Y\B\4\X6:Functions\X${}\mathrel+\E{}$\6
\1\1\&{void} \\{err\_print}(\|s)\C{ prints `\..' and location of error message
}\6
\&{char} ${}{*}\|s;\2\2{}$\6
${}\{{}$\1\6
\&{char} ${}{*}\|k,{}$ ${}{*}\|l{}$;\C{ pointers into \PB{\\{buffer}} }\7
${}\\{fprintf}(\\{stderr},\39{*}\|s\E\.{'!'}\?\.{"\\n\%s"}:\.{"\%s"},\39%
\|s);{}$\6
\&{if} (\\{web\_file\_open})\1\5
\X26:Print error location based on input buffer\X\2\6
\&{else}\1\5
${}\\{putc}(\.{'\\n'},\39\\{stderr});{}$\2\6
\\{update\_terminal};\6
\\{mark\_error};\6
\4${}\}{}$\2\par
\fi

\M{26}The error locations can be indicated by using the global variables
\PB{\\{loc}}, \PB{\\{cur\_line}}, \PB{\\{cur\_file\_name}} and \PB{%
\\{changing}},
which tell respectively the first
unlooked-at position in \PB{\\{buffer}}, the current line number, the current
file, and whether the current line is from \PB{\\{change\_file}} or \PB{\\{cur%
\_file}}.
This routine should be modified on systems whose standard text editor
has special line-numbering conventions.

\Y\B\4\X26:Print error location based on input buffer\X${}\E{}$\6
${}\{{}$\1\6
\&{if} ${}(\\{changing}\W\\{include\_depth}\E\\{change\_depth}){}$\1\5
${}\\{fprintf}(\\{stderr},\39\.{".\ (l.\ \%d\ of\ change\ }\)\.{file)\\n"},\39%
\\{change\_line});{}$\2\6
\&{else} \&{if} ${}(\\{include\_depth}\E\T{0}){}$\1\5
${}\\{fprintf}(\\{stderr},\39\.{".\ (l.\ \%d)\\n"},\39\\{cur\_line});{}$\2\6
\&{else}\1\5
${}\\{fprintf}(\\{stderr},\39\.{".\ (l.\ \%d\ of\ include}\)\.{\ file\ \%s)%
\\n"},\39\\{cur\_line},\39\\{cur\_file\_name});{}$\2\6
${}\|l\K(\\{loc}\G\\{limit}\?\\{limit}:\\{loc});{}$\6
\&{if} ${}(\|l>\\{buffer}){}$\5
${}\{{}$\1\6
\&{for} ${}(\|k\K\\{buffer};{}$ ${}\|k<\|l;{}$ ${}\|k\PP){}$\1\6
\&{if} ${}({*}\|k\E\.{'\\t'}){}$\1\5
${}\\{putc}(\.{'\ '},\39\\{stderr});{}$\2\6
\&{else}\1\5
${}\\{putc}({*}\|k,\39\\{stderr}){}$;\C{ print the characters already read }\2%
\2\6
${}\\{putc}(\.{'\\n'},\39\\{stderr});{}$\6
\&{for} ${}(\|k\K\\{buffer};{}$ ${}\|k<\|l;{}$ ${}\|k\PP){}$\1\5
${}\\{putc}(\.{'\ '},\39\\{stderr}){}$;\C{ space out the next line }\2\6
\4${}\}{}$\2\6
\&{for} ${}(\|k\K\|l;{}$ ${}\|k<\\{limit};{}$ ${}\|k\PP){}$\1\5
${}\\{putc}({*}\|k,\39\\{stderr}){}$;\C{ print the part not yet read }\2\6
${}\\{putc}(\.{'\\n'},\39\\{stderr});{}$\6
\4${}\}{}$\2\par
\U25.\fi

\M{27}When no recovery from some error has been provided, we have to wrap
up and quit as graciously as possible.  This is done by calling the
function \PB{\\{wrap\_up}} at the end of the code.

\Y\B\4\D\\{fatal}$(\|s,\|t)$\6
${}\{{}$\1\6
${}\\{fprintf}(\\{stderr},\39\|s);{}$\6
\\{err\_print}(\|t);\6
${}\\{history}\K\\{fatal\_message};{}$\6
\\{exit}(\\{wrap\_up}(\,));\6
\4${}\}{}$\2\par
\fi

\M{28}Some implementations may wish to pass the \PB{\\{history}} value to the
operating system so that it can be used to govern whether or not other
programs are started. Here, for instance, we pass the operating system
a status of 0 if and only if only harmless messages were printed.

\Y\B\4\X6:Functions\X${}\mathrel+\E{}$\6
\\{wrap\_up}(\,)\1\1\2\2\6
${}\{{}$\1\6
\X29:Print the job \PB{\\{history}}\X;\6
\&{if} ${}(\\{history}>\\{harmless\_message}){}$\1\5
\&{return} (\T{1});\2\6
\&{else}\1\5
\&{return} (\T{0});\2\6
\4${}\}{}$\2\par
\fi

\M{29}\B\X29:Print the job \PB{\\{history}}\X${}\E{}$\6
\&{switch} (\\{history})\5
${}\{{}$\1\6
\4\&{case} \\{spotless}:\6
\&{if} (\\{show\_happiness})\1\5
${}\\{fprintf}(\\{stderr},\39\.{"(No\ errors\ were\ fou}\)\.{nd.)\\n"});{}$\2\6
\&{break};\6
\4\&{case} \\{harmless\_message}:\5
${}\\{fprintf}(\\{stderr},\39\.{"(Did\ you\ see\ the\ wa}\)\.{rning\ message\
above?}\)\.{)\\n"});{}$\6
\&{break};\6
\4\&{case} \\{error\_message}:\5
${}\\{fprintf}(\\{stderr},\39\.{"(Pardon\ me,\ but\ I\ t}\)\.{hink\ I\ spotted\
somet}\)\.{hing\ wrong.)\\n"});{}$\6
\&{break};\6
\4\&{case} \\{fatal\_message}:\5
${}\\{fprintf}(\\{stderr},\39\.{"(That\ was\ a\ fatal\ e}\)\.{rror,\ my\
friend.)\\n"});{}$\6
\4${}\}{}$\C{ there are no other cases }\2\par
\U28.\fi

\N{1}{30}Command line arguments.
The user calls \.{wmerge} with arguments on the command line.
These are either file names or flags to be turned off (beginning with \PB{%
\.{"-"}})
or flags to be turned on (beginning with \PB{\.{"+"}}.
The following globals are for communicating the user's desires to the rest
of the program. The various file name variables contain strings with
the names of those files. Most of the 128 flags are undefined but available
for future extensions.

\Y\B\4\D\\{show\_banner}\5
\\{flags}[\.{'b'}]\C{ should the banner line be printed? }\par
\B\4\D\\{show\_happiness}\5
\\{flags}[\.{'h'}]\C{ should lack of errors be announced? }\par
\Y\B\4\X2:Definitions\X${}\mathrel+\E{}$\6
\&{int} \\{argc};\C{ copy of \PB{\\{ac}} parameter to \PB{\\{main}} }\6
\&{char} ${}{*}{*}\\{argv}{}$;\C{ copy of \PB{\\{av}} parameter to \PB{%
\\{main}} }\6
\&{char} \\{out\_file\_name}[\\{max\_file\_name\_length}];\C{ name of \PB{%
\\{out\_file}} }\6
\&{boolean} \\{flags}[\T{128}];\C{ an option for each 7-bit code }\par
\fi

\M{31}The \PB{\\{flags}} will be initially 1.

\Y\B\4\X31:Set the default options\X${}\E{}$\6
$\\{show\_banner}\K\\{show\_happiness}\K\T{1}{}$;\par
\U1.\fi

\M{32}We now must look at the command line arguments and set the file names
accordingly.  At least one file name must be present: the \.{WEB}
file.  It may have an extension, or it may omit it to get \PB{\.{'.w'}}
added.

If there is another file name present among the arguments, it is the
change file, again either with an extension or without one to get \PB{%
\.{'.ch'}}
An omitted change file argument means that \PB{\.{'/dev/null'}} should be used,
when no changes are desired.

If there's a third file name, it will be the output file.

\Y\B\4\X3:Predeclarations of functions\X${}\mathrel+\E{}$\6
\&{void} \\{scan\_args}(\,);\par
\fi

\M{33}
\Y\B\4\X6:Functions\X${}\mathrel+\E{}$\6
\1\1\&{void} \\{scan\_args}(\,)\2\2\6
${}\{{}$\1\6
\&{char} ${}{*}\\{dot\_pos}{}$;\C{ position of \PB{\.{'.'}} in the argument }\6
\&{register} \&{char} ${}{*}\|s{}$;\C{ register for scanning strings }\6
\&{boolean} \\{found\_web}${}\K\T{0},{}$ \\{found\_change}${}\K\T{0},{}$ %
\\{found\_out}${}\K\T{0}{}$;\C{ have these names have been seen? }\6
\&{boolean} \\{flag\_change};\7
\&{while} ${}(\MM\\{argc}>\T{0}){}$\5
${}\{{}$\1\6
\&{if} ${}({*}{*}(\PP\\{argv})\E\.{'-'}\V{*}{*}\\{argv}\E\.{'+'}){}$\1\5
\X37:Handle flag argument\X\2\6
\&{else}\5
${}\{{}$\1\6
${}\|s\K{*}\\{argv}{}$;\5
${}\\{dot\_pos}\K\NULL;{}$\6
\&{while} ${}({*}\|s){}$\5
${}\{{}$\1\6
\&{if} ${}({*}\|s\E\.{'.'}){}$\1\5
${}\\{dot\_pos}\K\|s\PP;{}$\2\6
\&{else} \&{if} ${}({*}\|s\E\.{'/'}){}$\1\5
${}\\{dot\_pos}\K\NULL,\39\PP\|s;{}$\2\6
\&{else}\1\5
${}\|s\PP;{}$\2\6
\4${}\}{}$\2\6
\&{if} ${}(\R\\{found\_web}){}$\1\5
\X34:Make \PB{\\{web\_file\_name}}\X\2\6
\&{else} \&{if} ${}(\R\\{found\_change}){}$\1\5
\X35:Make \PB{\\{change\_file\_name}} from \PB{\\{fname}}\X\2\6
\&{else} \&{if} ${}(\R\\{found\_out}){}$\1\5
\X36:Override output file name\X\2\6
\&{else}\1\5
\X38:Print usage error message and quit\X;\2\6
\4${}\}{}$\2\6
\4${}\}{}$\2\6
\&{if} ${}(\R\\{found\_web}){}$\1\5
\X38:Print usage error message and quit\X;\2\6
\&{if} ${}(\R\\{found\_change}){}$\1\5
${}\\{strcpy}(\\{change\_file\_name},\39\.{"/dev/null"});{}$\2\6
\4${}\}{}$\2\par
\fi

\M{34}We use all of \PB{${*}\\{argv}$} for the \PB{\\{web\_file\_name}} if
there is a \PB{\.{'.'}} in it,
otherwise we add \PB{\.{".w"}}. If this file can't be opened, we prepare an
\PB{\\{alt\_web\_file\_name}} by adding \PB{\.{"web"}} after the dot.
The other file names come from adding other things
after the dot.  We must check that there is enough room in
\PB{\\{web\_file\_name}} and the other arrays for the argument.

\Y\B\4\X34:Make \PB{\\{web\_file\_name}}\X${}\E{}$\6
${}\{{}$\1\6
\&{if} ${}(\|s-{*}\\{argv}>\\{max\_file\_name\_length}-\T{5}){}$\1\5
\X39:Complain about argument length\X;\2\6
\&{if} ${}(\\{dot\_pos}\E\NULL){}$\1\5
${}\\{sprintf}(\\{web\_file\_name},\39\.{"\%s.w"},\39{*}\\{argv});{}$\2\6
\&{else}\5
${}\{{}$\1\6
${}\\{strcpy}(\\{web\_file\_name},\39{*}\\{argv});{}$\6
${}{*}\\{dot\_pos}\K\T{0}{}$;\C{ string now ends where the dot was }\6
\4${}\}{}$\2\6
${}\\{sprintf}(\\{alt\_web\_file\_name},\39\.{"\%s.web"},\39{*}\\{argv});{}$\6
${}{*}\\{out\_file\_name}\K\.{'\\0'}{}$;\C{ this will print to stdout }\6
${}\\{found\_web}\K\T{1};{}$\6
\4${}\}{}$\2\par
\U33.\fi

\M{35}\B\X35:Make \PB{\\{change\_file\_name}} from \PB{\\{fname}}\X${}\E{}$\6
${}\{{}$\1\6
\&{if} ${}(\|s-{*}\\{argv}>\\{max\_file\_name\_length}-\T{4}){}$\1\5
\X39:Complain about argument length\X;\2\6
\&{if} ${}(\\{dot\_pos}\E\NULL){}$\1\5
${}\\{sprintf}(\\{change\_file\_name},\39\.{"\%s.ch"},\39{*}\\{argv});{}$\2\6
\&{else}\1\5
${}\\{strcpy}(\\{change\_file\_name},\39{*}\\{argv});{}$\2\6
${}\\{found\_change}\K\T{1};{}$\6
\4${}\}{}$\2\par
\U33.\fi

\M{36}\B\X36:Override output file name\X${}\E{}$\6
${}\{{}$\1\6
\&{if} ${}(\|s-{*}\\{argv}>\\{max\_file\_name\_length}-\T{5}){}$\1\5
\X39:Complain about argument length\X;\2\6
\&{if} ${}(\\{dot\_pos}\E\NULL){}$\1\5
${}\\{sprintf}(\\{out\_file\_name},\39\.{"\%s.out"},\39{*}\\{argv});{}$\2\6
\&{else}\1\5
${}\\{strcpy}(\\{out\_file\_name},\39{*}\\{argv});{}$\2\6
${}\\{found\_out}\K\T{1};{}$\6
\4${}\}{}$\2\par
\U33.\fi

\M{37}\B\X37:Handle flag argument\X${}\E{}$\6
${}\{{}$\1\6
\&{if} ${}({*}{*}\\{argv}\E\.{'-'}){}$\1\5
${}\\{flag\_change}\K\T{0};{}$\2\6
\&{else}\1\5
${}\\{flag\_change}\K\T{1};{}$\2\6
\&{for} ${}(\\{dot\_pos}\K{*}\\{argv}+\T{1};{}$ ${}{*}\\{dot\_pos}>\.{'%
\\0'};{}$ ${}\\{dot\_pos}\PP){}$\1\5
${}\\{flags}[{*}\\{dot\_pos}]\K\\{flag\_change};{}$\2\6
\4${}\}{}$\2\par
\U33.\fi

\M{38}\B\X38:Print usage error message and quit\X${}\E{}$\6
${}\{{}$\1\6
${}\\{fatal}(\.{"!\ Usage:\ wmerge\ web}\)\.{file[.w]\ [changefile}\)\.{[.ch]\
[outfile[.out]}\)\.{]]\\n"},\39\.{""}){}$\6
\4${}\}{}$\2\par
\U33.\fi

\M{39}\B\X39:Complain about argument length\X${}\E{}$\6
$\\{fatal}(\.{"!\ Filename\ too\ long}\)\.{\\n"},\39{*}\\{argv}){}$;\par
\Us34, 35\ETs36.\fi

\N{1}{40}Output. Here is the code that opens the output file:

\Y\B\4\X2:Definitions\X${}\mathrel+\E{}$\6
\&{FILE} ${}{*}\\{out\_file}{}$;\C{ where output goes }\par
\fi

\M{41}\B\X41:Scan arguments and open output file\X${}\E{}$\6
\\{scan\_args}(\,);\6
\&{if} ${}(\\{out\_file\_name}[\T{0}]\E\.{'\\0'}){}$\1\5
${}\\{out\_file}\K\\{stdout};{}$\2\6
\&{else} \&{if} ${}((\\{out\_file}\K\\{fopen}(\\{out\_file\_name},\39\.{"w"}))%
\E\NULL){}$\1\5
${}\\{fatal}(\.{"!\ Cannot\ open\ outpu}\)\.{t\ file\ "},\39\\{out\_file%
\_name}){}$;\2\par
\U1.\fi

\M{42}The \PB{\\{update\_terminal}} procedure is called when we want
to make sure that everything we have output to the terminal so far has
actually left the computer's internal buffers and been sent.

\Y\B\4\D\\{update\_terminal}\5
\\{fflush}(\\{stderr})\C{ empty the terminal output buffer }\par
\fi

\N{1}{43}Index.
\fi

\inx
\fin
\con
