\input cwebmac
\datethis

\N{1}{1}Introduction. This program is a simple filter that inputs \TEX/ or %
\.{CWEB}
files and outputs its best guess at the ``words'' they contain. The word
list can then be passed to a spelling check routine such as {\tt wordtest}.

If this program is invoked with the name `{\tt excweb}', it will apply
special rules based on the syntax of \.{CWEB} files. Otherwise it will
use only the \TEX/ conventions. (Note that \UNIX/'s {\tt ln} command
allows a program to be invoked with more than one name although it
appears only once in the computer's memory.)

The \TEX/ conventions adopted here say that words are what remain
after you remove nonletters, control sequences,
comments triggered by \.\% marks, and material enclosed
within \.{\$...\$} or \.{\$\$...\$\$}. However, an apostrophe within
a word will be retained. The plain \TEX/ control
sequences for accented characters and special text characters, namely
$$\vbox{\halign{&\.{\\#}\hfil\qquad\cr
'&`&\relax\^&"&\relax\~&=&.&u&v\cr
H&t&c&d&b&oe&OE&ae&AE\cr
aa&AA&o&O&l&L&ss&i&j\cr}}$$
will also be retained, so that users can treat them as parts of words.
A blank space
following any of the alphabetic control sequences in this list will be carried
along too. If any of these control sequences is followed by \.\{, everything
up to the next \.\} will also be retained. Thus, for example, the
construction `\.{m\\=\{\\i\}n\\u\ us}' will be considered a single word,
in spite of the control sequences and the space between the two u's.
Discretionary hyphens `\.{\char`\\-}' are treated in the same way as accents.

The \.{CWEB} conventions are essentially the same as the \TEX/ conventions,
in the \TEX/ parts of a \.{CWEB} file. The \CEE/ parts of the file
are blanked out.

No attempt is made to reach a high level of artificial intelligence,
which would be able to truly understand the input file. Tricky users can
confuse us. But we claim that devious tricks are their problem, not ours.

\fi

\M{2}So here goes. The main idea is to keep a one-character lookahead
buffer, called \PB{\|c}, which is set to zero when the character has been
processed. A giant switch to various cases, depending on the value of~\PB{\|c},
keeps everything moving.

If you don't like \PB{\&{goto}} statements, don't read this. (And don't read
any other programs that simulate finite-state automata.)

\Y\B\8\#\&{include} \.{<stdio.h>}\6
\8\#\&{include} \.{<ctype.h>}\6
\8\#\&{include} \.{<string.h>}\6
\&{extern} \&{void} \\{exit}(\,);\C{ system routine that terminates execution }%
\7
\X3:Global variables\X\6
\X6:Procedures\X\7
\1\1\&{int} ${}\\{main}(\\{argc},\39\\{argv}){}$\6
\&{int} \\{argc};\C{ the number of arguments (should be 1, but this isn't
checked) }\6
\&{char} ${}{*}\\{argv}[\,]{}$;\C{ the arguments (\PB{${*}\\{argv}$} is the
program name) }\2\2\6
${}\{{}$\1\6
\X4:Local variables\X;\6
\&{if} ${}(\\{strlen}({*}\\{argv})\G\T{6}\W\\{strcmp}({*}\\{argv}+%
\\{strlen}({*}\\{argv})-\T{6},\39\.{"excweb"})\E\T{0}){}$\5
${}\{{}$\1\6
${}\\{web}\K\T{1};{}$\6
\X15:Adjust tables for \.{CWEB} mode\X;\6
\4${}\}{}$\2\6
\&{else}\1\5
${}\\{web}\K\T{0};{}$\2\6
${}\\{comment}\K\\{skipping}\K\|c\K\T{0};{}$\6
\4\\{main\_cycle}:\6
\&{if} (\|c)\1\5
\&{goto} \\{big\_switch};\2\6
\4\\{restart}:\5
${}\|c\K\\{get}(\,);{}$\6
\4\\{big\_switch}:\6
\&{switch} (\|c)\5
${}\{{}$\1\6
\X5:Special cases of the giant switch where we don't just discard \PB{\|c}\X\6
\4\&{case} \.{EOF}:\5
\\{exit}(\T{0});\6
\4\&{default}:\5
\&{goto} \\{restart};\6
\4${}\}{}$\2\6
\X8:Labeled code segments, which exit by explicit \PB{\&{goto}}\X;\6
\4${}\}{}$\2\par
\fi

\M{3}\B\X3:Global variables\X${}\E{}$\6
\&{int} \|c;\C{ one-character look-see buffer }\par
\As14\ET17.
\U2.\fi

\M{4}\B\X4:Local variables\X${}\E{}$\6
\&{int} \\{web};\C{ are we looking for \.{CWEB} constructs? }\6
\&{int} \\{comment};\C{ are we inside a \CEE/ comment in a \.{CWEB} document? }%
\6
\&{int} \\{skipping};\C{ are we skipping \CEE/ code in a \.{CWEB} document? }\6
\&{int} \\{save\_skipping};\C{ value of \PB{\\{skipping}} outside current \CEE/
mode }\6
\&{register} \&{int} \\{cc};\C{ temporary buffer }\par
\U2.\fi

\N{1}{5}Simple cases.
Let's do some of the easiest things first, in order to get the hang of
this program. Several special characters will cause us to ignore everything
until the first appearance of something else.

\Y\B\4\D\\{discard\_to}$(\|x)$\6
${}\{{}$\5
\1\&{while} ${}(\\{get}(\,)\I\|x){}$\1\5
;\5
\2${}\}{}$\2\par
\B\4\D\\{discard\_to\_dol}\6
${}\{{}$\5
\1\&{for} ${}(\\{cc}\K\|c,\39\|c\K\\{get}(\,);{}$ ${}\|c\I\.{'\$'}\V\\{cc}\E%
\.{'\\\\'};{}$ ${}\\{cc}\K\|c,\39\|c\K\\{get}(\,)){}$\1\6
\&{if} ${}(\\{cc}\E\.{'\\\\'}\W\|c\E\\{cc}){}$\1\5
${}\|c\K\.{'\\0'}{}$;\5
\2\2${}\}{}$\2\par
\Y\B\4\X5:Special cases of the giant switch where we don't just discard \PB{%
\|c}\X${}\E{}$\6
\4\&{case} \.{'\%'}:\5
\\{discard\_to}(\.{'\\n'});\5
\&{goto} \\{restart};\6
\4\&{case} \.{'\$'}:\5
${}\|c\K\\{getchar}(\,);{}$\6
\&{if} ${}(\|c\I\.{'\$'}){}$\1\5
\\{discard\_to\_dol}\2\6
\&{else}\5
${}\{{}$\C{ after \.{\$\$} we discard everything to the next \.{\$\$} }\1\6
\&{do}\5
\\{discard\_to\_dol}\5
\&{while} ${}(\\{getchar}(\,)\I\.{'\$'});{}$\6
\4${}\}{}$\2\6
\&{goto} \\{restart};\par
\As7, 11\ETs16.
\U2.\fi

\M{6}The `\PB{\\{get}}' procedure in the code above is like \Cee's standard
`\PB{\\{getchar}}', except that it immediately terminates execution at the end
of
the input file. Otherwise malformed input files could lead to
infinite loops.

\Y\B\4\X6:Procedures\X${}\E{}$\6
\1\1\&{int} \\{get}(\,)\2\2\6
${}\{{}$\5
\1\&{register} \&{int} \|x;\7
${}\|x\K\\{getchar}(\,);{}$\6
\&{if} ${}(\|x\E\.{EOF}){}$\1\5
\\{exit}(\T{0});\2\6
\&{return} \|x;\6
\4${}\}{}$\2\par
\A12.
\U2.\fi

\M{7}More complex behavior is handled by jumping out of the \PB{\&{switch}}
statement
to one of the routines following it. None of the cases say \PB{\&{break}}, so
the code following the switch statement is accessible only via \PB{\&{goto}}.

\Y\B\4\X5:Special cases of the giant switch where we don't just discard \PB{%
\|c}\X${}\mathrel+\E{}$\6
\4\&{case} \.{'a'}:\5
\&{case} \.{'A'}:\5
\&{case} \.{'b'}:\5
\&{case} \.{'B'}:\5
\&{case} \.{'c'}:\5
\&{case} \.{'C'}:\5
\&{case} \.{'d'}:\5
\&{case} \.{'D'}:\5
\&{case} \.{'e'}:\5
\&{case} \.{'E'}:\5
\&{case} \.{'f'}:\5
\&{case} \.{'F'}:\5
\&{case} \.{'g'}:\5
\&{case} \.{'G'}:\5
\&{case} \.{'h'}:\5
\&{case} \.{'H'}:\5
\&{case} \.{'i'}:\5
\&{case} \.{'I'}:\5
\&{case} \.{'j'}:\5
\&{case} \.{'J'}:\5
\&{case} \.{'k'}:\5
\&{case} \.{'K'}:\5
\&{case} \.{'l'}:\5
\&{case} \.{'L'}:\5
\&{case} \.{'m'}:\5
\&{case} \.{'M'}:\5
\&{case} \.{'n'}:\5
\&{case} \.{'N'}:\5
\&{case} \.{'o'}:\5
\&{case} \.{'O'}:\5
\&{case} \.{'p'}:\5
\&{case} \.{'P'}:\5
\&{case} \.{'q'}:\5
\&{case} \.{'Q'}:\5
\&{case} \.{'r'}:\5
\&{case} \.{'R'}:\5
\&{case} \.{'s'}:\5
\&{case} \.{'S'}:\5
\&{case} \.{'t'}:\5
\&{case} \.{'T'}:\5
\&{case} \.{'u'}:\5
\&{case} \.{'U'}:\5
\&{case} \.{'v'}:\5
\&{case} \.{'V'}:\5
\&{case} \.{'w'}:\5
\&{case} \.{'W'}:\5
\&{case} \.{'x'}:\5
\&{case} \.{'X'}:\5
\&{case} \.{'y'}:\5
\&{case} \.{'Y'}:\5
\&{case} \.{'z'}:\5
\&{case} \.{'Z'}:\5
\&{goto} \\{out\_word};\par
\fi

\M{8}When letters appear in \PB{\\{stdin}}, we pass them immediately through to
\PB{\\{stdout}}
with little further ado.
An apostrophe is rejected unless it is immediately followed by a letter.

\Y\B\4\X8:Labeled code segments, which exit by explicit \PB{\&{goto}}\X${}\E{}$%
\6
\4\\{out\_word}:\5
\\{putchar}(\|c);\6
\4\\{continue\_word}:\5
${}\|c\K\\{getchar}(\,);{}$\6
\4\\{checkout\_word}:\6
\&{if} (\\{isalpha}(\|c))\1\5
\&{goto} \\{out\_word};\2\6
\&{if} ${}(\|c\E\.{'\\''}){}$\5
${}\{{}$\1\6
${}\|c\K\\{getchar}(\,);{}$\6
\&{if} (\\{isalpha}(\|c))\5
${}\{{}$\1\6
\\{putchar}(\.{'\\''});\5
\&{goto} \\{out\_word};\6
\4${}\}{}$\2\6
\&{goto} \\{end\_word};\6
\4${}\}{}$\2\6
\&{if} ${}(\|c\E\.{'\\\\'}\W\\{controlseq}(\,)){}$\1\5
\&{goto} \\{control\_seq\_in\_word};\2\6
\4\\{end\_word}:\5
\\{putchar}(\.{'\\n'});\6
\&{goto} \\{main\_cycle};\par
\As10, 18\ETs19.
\U2.\fi

\N{1}{9}Control sequences.  The \PB{\\{controlseq}(\,)} function is the only
delicate part of this program.  After a backslash has been scanned,
\PB{\\{controlseq}} looks to see if the next characters define one of the
special plain \TEX/ macros listed above. If so, the control sequence
and its immediately following argument (if any) are output and
\PB{\\{controlseq}} returns a nonzero value. If not, nothing is output and
\PB{\\{controlseq}} returns zero. In both cases the value of \PB{\|c} will be
nonzero if and only if \PB{\\{controlseq}} has had to look ahead at a
character it decided not to process.

\fi

\M{10}\B\X8:Labeled code segments, which exit by explicit \PB{\&{goto}}\X${}%
\mathrel+\E{}$\6
\4\\{control\_seq\_in\_word}:\6
\&{if} ${}(\R\|c){}$\1\5
\&{goto} \\{continue\_word};\2\6
\&{goto} \\{checkout\_word};\par
\fi

\M{11}\B\X5:Special cases of the giant switch where we don't just discard \PB{%
\|c}\X${}\mathrel+\E{}$\6
\4\&{case} \.{'\\\\'}:\6
\&{if} (\\{controlseq}(\,))\1\5
\&{goto} \\{control\_seq\_in\_word};\2\6
\&{goto} \\{main\_cycle};\par
\fi

\M{12}\B\X6:Procedures\X${}\mathrel+\E{}$\6
\1\1\&{int} \\{controlseq}(\,)\2\2\6
${}\{{}$\1\6
\&{int} \|l;\C{ number of letters in the control sequence }\6
\&{char} \|a${},{}$ \|b;\C{ the first two characters after `\.\\' }\7
${}\|l\K\T{0};{}$\6
${}\|a\K\|c\K\\{getchar}(\,);{}$\6
\&{while} (\\{isalpha}(\|c))\5
${}\{{}$\1\6
${}\|l\PP;{}$\6
${}\|c\K\\{getchar}(\,);{}$\6
\&{if} ${}(\|l\E\T{1}){}$\1\5
${}\|b\K\|c;{}$\2\6
\4${}\}{}$\2\6
\&{if} ${}(\|l\E\T{0}){}$\1\5
${}\|c\K\\{getchar}(\,);{}$\2\6
\X13:Check for special plain \TEX/ control sequences; output them and \PB{%
\&{return} \T{1}} if found\X;\6
\&{return} \T{0};\6
\4${}\}{}$\2\par
\fi

\M{13}\B\D\\{pair}$(\|x,\|y)$\5
$(\|a\E\|x\W\|b\E\|y{}$)\par
\Y\B\4\X13:Check for special plain \TEX/ control sequences; output them and %
\PB{\&{return} \T{1}} if found\X${}\E{}$\6
\&{if} ${}((\|a\G\.{'"'}\W\|a\Z\.{'\~'}\W\\{ptab}[\|a-\.{'"'}]\E\|l)\V(\|l\E%
\T{2}\W(\\{pair}(\.{'a'},\39\.{'e'})\V\\{pair}(\.{'A'},\39\.{'E'})\3{-1}\V%
\\{pair}(\.{'o'},\39\.{'e'})\V\\{pair}(\.{'O'},\39\.{'E'})\3{-1}\V\\{pair}(%
\.{'a'},\39\.{'a'})\V\\{pair}(\.{'A'},\39\.{'A'})\V\\{pair}(\.{'s'},\39%
\.{'s'})))){}$\5
${}\{{}$\1\6
\\{putchar}(\.{'\\\\'});\6
\\{putchar}(\|a);\6
\&{if} ${}(\|l\E\T{2}){}$\1\5
\\{putchar}(\|b);\2\6
\&{if} ${}(\|l\W\|c\E\.{'\ '}){}$\5
${}\{{}$\1\6
\\{putchar}(\.{'\ '});\C{ optional space after alphabetic control sequence }\6
${}\|c\K\\{getchar}(\,);{}$\6
\4${}\}{}$\2\6
\&{if} ${}(\|c\E\.{'\{'}){}$\5
${}\{{}$\1\6
\&{do}\5
${}\{{}$\1\6
\\{putchar}(\|c);\6
${}\|c\K\\{get}(\,);{}$\6
\4${}\}{}$\5
\2\5
\&{while} ${}(\|c\I\.{'\}'}){}$;\C{ optional argument after special control
sequence }\6
\\{putchar}(\|c);\6
${}\|c\K\T{0};{}$\6
\4${}\}{}$\2\6
\&{return} \T{1};\6
\4${}\}{}$\2\par
\U12.\fi

\M{14}The \PB{\\{ptab}} entries for nonletters are 0 when the control sequence
is
special, otherwise~1; the conventions for letters are reversed.

\Y\B\4\X3:Global variables\X${}\mathrel+\E{}$\6
\&{char} \\{ptab}[\,]${}\K\{\T{0},\39\T{1},\39\T{1},\39\T{1},\39\T{1},\39%
\T{0},{}$\C{ \.{\\"} and \.{\\'} }\6
\T{1}${},\39\T{1},\39\T{1},\39\T{1},\39\T{1},\39\T{0},\39\T{0},\39\T{1},{}$\C{ %
\.{\\-} and \.{\\.} }\6
\T{1}${},\39\T{1},\39\T{1},\39\T{1},\39\T{1},\39\T{1},\39\T{1},\39\T{1},\39%
\T{1},\39\T{1},\39\T{1},\39\T{1},\39\T{1},\39\T{0},\39\T{1},\39\T{1},{}$\C{ \.{%
\\=} }\6
\T{1}${},\39\T{0},\39\T{0},\39\T{0},\39\T{0},\39\T{0},\39\T{0},\39\T{0},\39%
\T{1},\39\T{0},\39\T{0},\39\T{0},\39\T{1},\39\T{0},\39\T{0},\39\T{1},{}$\C{ \.{%
\\H}, \.{\\L}, \.{\\O} }\6
\T{0}${},\39\T{0},\39\T{0},\39\T{0},\39\T{0},\39\T{0},\39\T{0},\39\T{0},\39%
\T{0},\39\T{0},\39\T{0},\39\T{1},\39\T{1},\39\T{1},\39\T{0},\39\T{1},{}$\C{ \.{%
\\\^} }\6
\T{0}${},\39\T{0},\39\T{1},\39\T{1},\39\T{1},\39\T{0},\39\T{0},\39\T{0},{}$\C{ %
\.{\\`}, \.{\\b}, \.{\\c}, \.{\\d} }\6
\T{0}${},\39\T{1},\39\T{1},\39\T{0},\39\T{1},\39\T{0},\39\T{0},\39\T{1},{}$\C{ %
\.{\\i}, \.{\\j}, \.{\\l}, \.{\\o} }\6
\T{0}${},\39\T{0},\39\T{0},\39\T{0},\39\T{1},\39\T{1},\39\T{1},\39\T{0},{}$\C{ %
\.{\\t}, \.{\\u}, \.{\\v} }\6
\T{0}${},\39\T{0},\39\T{0},\39\T{1},\39\T{1},\39\T{1},\39\T{0}\}{}$;\C{ \.{%
\\\~} }\par
\fi

\M{15}In \.{CWEB} the \TEX/ control sequence `\.{\\.}' denotes the typewriter
font used for strings, not the dot-over accent. We must modify
\PB{\\{ptab}} to reflect this unfortunate (but too-late-too-change) design
decision.

\Y\B\4\X15:Adjust tables for \.{CWEB} mode\X${}\E{}$\6
$\\{ptab}[\T{12}]\K\T{1}{}$;\par
\U2.\fi

\N{1}{16}CWEB considerations.
We're finished now with all that would be needed if we only wanted to
handle \TEX/. For \.{CWEB} a bit more should be done.

The \.{CWEB} escape character is \.{@}, and the character following
it tells us what mode we should enter. In \TEX/ mode we should not
only do what we normally do for \TEX/ files, we should also ignore
material delimited by \.{\char"7C...\char"7C}, being careful to recognize when
\PB{\.{'|'}} is part of a string (as it just was). And we should stop \TEX/
mode if we scan \.{*/} while in a \CEE/ comment.

\Y\B\4\X5:Special cases of the giant switch where we don't just discard \PB{%
\|c}\X${}\mathrel+\E{}$\6
\4\&{case} \.{'@'}:\6
\&{if} (\\{web})\1\5
\&{goto} \\{do\_web};\2\6
\&{goto} \\{restart};\6
\4\&{case} \.{'|'}:\6
\&{if} ${}(\\{web}>\T{1}){}$\5
${}\{{}$\1\6
${}\\{save\_skipping}\K\\{skipping};{}$\6
\&{goto} \\{skip\_C\_prime};\6
\4${}\}{}$\2\6
\&{goto} \\{restart};\6
\4\&{case} \.{'*'}:\6
\&{if} ${}(\R\\{comment}){}$\1\5
\&{goto} \\{restart};\2\6
${}\|c\K\\{getchar}(\,);{}$\6
\&{if} ${}(\|c\E\.{'/'}){}$\5
${}\{{}$\1\6
${}\\{comment}\K\T{0};{}$\6
\&{goto} \\{skip\_C};\6
\4${}\}{}$\2\6
\&{goto} \\{big\_switch};\par
\fi

\M{17}The characters that follow \.@ in a \.{CWEB} file can be classified into
a few types that have distinct implications for {\tt excweb}.

\Y\B\4\D\\{nop}\5
\T{0}\C{ control code that doesn't matter to us }\par
\B\4\D\\{start\_section}\5
\T{1}\C{ control code that begins a \.{CWEB} section }\par
\B\4\D\\{start\_C}\5
\T{2}\C{ control code that begins \CEE/ code }\par
\B\4\D\\{start\_name}\5
\T{3}\C{ control code that begins a section name }\par
\B\4\D\\{start\_index}\5
\T{4}\C{ control code for \.{CWEB} index entry }\par
\B\4\D\\{start\_insert}\5
\T{5}\C{ control code for \CEE/ material ended by `\.{@>}' }\par
\B\4\D\\{end\_item}\5
\T{6}\C{ `\.{@>}' }\par
\B\4\D\\{ignore\_line}\5
\T{7}\C{ `\.{@i}' or `\.{@l}' }\par
\Y\B\4\X3:Global variables\X${}\mathrel+\E{}$\6
\&{char} \\{wtab}[\,]${}\K\{\\{start\_section},\39\\{nop},\39\\{nop},\39%
\\{nop},\39\\{nop},\39\\{nop},\39\\{nop},\39\\{nop},{}$\C{ \.{\ !"\#\$\%\&'} }\6
\\{start\_name}${},\39\\{nop},\39\\{start\_section},\39\\{nop},\39\\{nop},\39%
\\{nop},\39\\{start\_index},\39\\{nop},{}$\C{ \.{()*+,-./} }\6
\\{nop}${},\39\\{nop},\39\\{nop},\39\\{nop},\39\\{nop},\39\\{nop},\39\\{nop},%
\39\\{nop},{}$\C{ \.{01234567} }\6
\\{nop}${},\39\\{nop},\39\\{start\_index},\39\\{nop},\39\\{start\_name},\39%
\\{start\_insert},\39\\{end\_item},\39\\{nop},{}$\C{ \.{89:;<=>?} }\6
\\{nop}${},\39\\{nop},\39\\{nop},\39\\{start\_C},\39\\{start\_C},\39\\{nop},\39%
\\{start\_C},\39\\{nop},{}$\C{ \.{@ABCDEFG} }\6
\\{nop}${},\39\\{ignore\_line},\39\\{nop},\39\\{nop},\39\\{ignore\_line},\39%
\\{nop},\39\\{nop},\39\\{nop},{}$\C{ \.{HIJKLMNO} }\6
\\{start\_C}${},\39\\{start\_insert},\39\\{nop},\39\\{start\_C},\39\\{start%
\_insert},\39\\{nop},\39\\{nop},\39\\{nop},{}$\C{ \.{PQRSTUVW} }\6
\\{nop}${},\39\\{nop},\39\\{nop},\39\\{nop},\39\\{nop},\39\\{nop},\39\\{start%
\_index},\39\\{nop},{}$\C{ \.{XYZ[\\]\^\_} }\6
\\{nop}${},\39\\{nop},\39\\{nop},\39\\{start\_C},\39\\{start\_C},\39\\{nop},\39%
\\{start\_C},\39\\{nop},{}$\C{ \.{`abcdefg} }\6
\\{nop}${},\39\\{ignore\_line},\39\\{nop},\39\\{nop},\39\\{ignore\_line},\39%
\\{nop},\39\\{nop},\39\\{nop},{}$\C{ \.{hijklmno} }\6
\\{start\_C}${},\39\\{start\_insert},\39\\{nop},\39\\{start\_C},\39\\{start%
\_insert}\}{}$;\C{ \.{pqrst} }\par
\fi

\M{18}We do not leave \TEX/ mode until the first \.{WEB} section has begun.

\Y\B\4\X8:Labeled code segments, which exit by explicit \PB{\&{goto}}\X${}%
\mathrel+\E{}$\6
\4\\{do\_web}:\5
${}\|c\K\\{getchar}(\,);{}$\6
\&{if} ${}(\|c<\.{'\ '}\V\|c>\.{'t'}){}$\1\5
\&{goto} \\{restart};\2\6
\&{switch} ${}(\\{wtab}[\|c-\.{'\ '}]){}$\5
${}\{{}$\1\6
\4\&{case} \\{nop}:\5
\&{case} \\{start\_index}:\5
\&{goto} \\{restart};\6
\4\&{case} \\{start\_section}:\5
${}\\{web}\PP{}$;\C{ out of ``limbo'' }\6
${}\\{comment}\K\\{skipping}\K\T{0}{}$;\5
\&{goto} \\{restart};\6
\4\&{case} \\{start\_C}:\6
\&{if} ${}(\\{web}>\T{1}){}$\1\5
\&{goto} \\{skip\_C};\2\6
\&{goto} \\{restart};\6
\4\&{case} \\{start\_name}:\5
\&{case} \\{start\_insert}:\6
\&{if} ${}(\\{web}>\T{1}){}$\1\5
${}\\{skipping}\K\T{1};{}$\2\6
\&{goto} \\{restart};\6
\4\&{case} \\{end\_item}:\6
\&{if} (\\{skipping})\1\5
\&{goto} \\{skip\_C};\2\6
\&{goto} \\{restart};\6
\4\&{case} \\{ignore\_line}:\5
\\{discard\_to}(\.{'\\n'});\6
\&{goto} \\{restart};\6
\4${}\}{}$\2\par
\fi

\M{19}The final piece of program we need is a sub-automaton to pass over
the \CEE/ parts of a \.{CWEB} document. The main subtlety here is that
we don't want to get out of synch while scanning over a supposed
string constant or verbatim insert.

\Y\B\4\X8:Labeled code segments, which exit by explicit \PB{\&{goto}}\X${}%
\mathrel+\E{}$\6
\4\\{skip\_C}:\5
${}\\{save\_skipping}\K\T{2};{}$\6
\4\\{skip\_C\_prime}:\5
${}\\{skipping}\K\T{1};{}$\6
\&{while} (\T{1})\5
${}\{{}$\1\6
${}\|c\K\\{get}(\,);{}$\6
\4\\{C\_switch}:\6
\&{switch} (\|c)\5
${}\{{}$\1\6
\4\&{case} \.{'/'}:\5
${}\|c\K\\{get}(\,);{}$\6
\&{if} ${}(\|c\I\.{'*'}){}$\1\5
\&{goto} \\{C\_switch};\2\6
${}\\{comment}\K\T{1}{}$;\C{ fall through to the next case, returning to \TEX/
mode }\6
\4\&{case} \.{'|'}:\6
\&{if} ${}(\\{save\_skipping}\E\T{2}){}$\1\5
\&{continue};\C{ \PB{\.{'|'}} as \CEE/ operator }\2\6
${}\\{skipping}\K\\{save\_skipping}{}$;\5
\&{goto} \\{restart};\C{ \PB{\.{'|'}} as \.{CWEB} delimiter }\6
\4\&{case} \.{'@'}:\5
${}\|c\K\\{getchar}(\,);{}$\6
\4\\{inner\_switch}:\6
\&{if} ${}(\|c<\.{'\ '}\V\|c>\.{'t'}){}$\1\5
\&{continue};\2\6
\&{switch} ${}(\\{wtab}[\|c-\.{'\ '}]){}$\5
${}\{{}$\1\6
\4\&{case} \\{nop}:\5
\&{case} \\{start\_C}:\5
\&{case} \\{end\_item}:\5
\&{continue};\6
\4\&{case} \\{start\_section}:\5
${}\\{web}\PP;{}$\6
${}\\{comment}\K\\{skipping}\K\T{0}{}$;\5
\&{goto} \\{restart};\6
\4\&{case} \\{start\_name}:\5
\&{case} \\{start\_index}:\5
\&{goto} \\{restart};\6
\4\&{case} \\{start\_insert}:\5
\&{do}\5
\\{discard\_to}(\.{'@'})\5
\5
\&{while} ${}((\|c\K\\{getchar}(\,))\E\.{'@'});{}$\6
\&{goto} \\{inner\_switch};\C{ now \PB{\|c} should equal \PB{\.{'>'}} }\6
\4\&{case} \\{ignore\_line}:\5
\\{discard\_to}(\.{'\\n'});\6
\&{continue};\6
\4${}\}{}$\2\6
\4\&{case} \.{'\\''}:\5
\&{case} \.{'"'}:\6
\&{while} ${}((\\{cc}\K\\{get}(\,))\I\|c\W\\{cc}\I\.{'\\n'}){}$\1\6
\&{if} ${}(\\{cc}\E\.{'\\\\'}){}$\1\5
\\{getchar}(\,);\2\6
\&{else} \&{if} ${}(\\{cc}\E\.{'@'}){}$\5
${}\{{}$\1\6
${}\\{cc}\K\\{getchar}(\,);{}$\6
\&{if} ${}(\\{cc}\I\.{'@'}){}$\5
${}\{{}$\C{ syntax error, we try to recover }\1\6
${}\|c\K\\{cc};{}$\6
\&{goto} \\{inner\_switch};\6
\4${}\}{}$\2\6
\4${}\}{}$\2\2\6
;\6
\&{continue};\C{ \.{CWEB} strings do not extend past one line }\6
\4\&{default}:\5
\&{continue};\6
\4${}\}{}$\2\6
\4${}\}{}$\2\par
\fi

\N{1}{20}Index.
\fi

\inx
\fin
\con
