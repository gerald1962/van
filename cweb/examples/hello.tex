\input cwebmac
% wc: An example of CWEB-


\M{1}Here is the file \.{hello.c}.

<<<<<<< HEAD
\Y\B\X2:Header files\X\6
=======
\Y\B\X2:Header files to include\X\6
>>>>>>> main
\X3:The main program\X\par
\fi

\M{2}We include the standard I/O declarations, since we send output to \PB{%
\\{stdout}}.

<<<<<<< HEAD
\Y\B\4\X2:Header files\X${}\E{}$\6
\8\#\&{include} \.{<stdio.h>}\6
\8\#\&{include} \.{<stdlib.h>}\par
\U1.\fi
=======
\Y\B\4\X2:Header files to include\X${}\E{}$\6
\8\#\&{include} \.{<stdio.h>}\par
\U1.\fi

\M{3}Now we come to the general layout of the \PB{\\{main}} function.
>>>>>>> main

\M{3}Now we come to the layout of the \PB{\\{main}} function.
\Y\B\4\X3:The main program\X${}\E{}$\6
\1\1\&{void} \\{main}(\&{void})\2\2\6
${}\{{}$\1\6
\\{printf}(\.{"Hello.\\n"});\6
\4${}\}{}$\2\par
\U1.\fi

\N{1}{4}Index.
Here is the list of the identifiers used.
\fi

\inx
\fin
\con
