\input cwebmac
\def\covernote{Copyright 1987 Norman Ramsey -- Princeton University}

\def\vbar{\.{|}}

\N{1}{1}Directory Trees.
Our object is to print out a directory hierarchy in some pleasant way.
The program takes output from {\tt find * -type d -print \vbar\ sort}
and produces a nicer-looking listing.
More precisely, our input, which is the output of {\tt find} followed
by {\tt sort}, is a list of fully qualified directory names (parent
and child separated by slashes \PB{\.{'/'}}); everything has already been
sorted nicely into lexicographic order.

The {\tt treeprint} routine takes one option, \PB{\.{"-p"}}, which tells it
to use the printer's line-drawing set, rather than the terminal's.

\Y\B\X12:Global definitions\X\6
\X5:Global include files\X\6
\X2:Global declarations\X\7
${}\\{main}(\\{argc},\39\\{argv}){}$\1\1\6
\&{int} \\{argc};\6
\&{char} ${}{*}{*}\\{argv};\2\2{}$\6
${}\{{}$\1\6
\X3:\PB{\\{main}} variable declarations\X;\6
\X14:Search for options and set special characters on \PB{\.{"-p"}}\X;\6
\X11:Read output from find and enter into tree\X;\6
\X18:Write tree on standard output\X\6
\\{exit}(\T{0});\6
\4${}\}{}$\2\par
\fi

\M{2}
We make all the siblings of a directory a linked list off of its left child,
and the offspring a linked list off the right side.
Data are just directory names.
\Y\B\4\D\\{sibling}\5
\\{left}\par
\B\4\D\\{child}\5
\\{right}\par
\Y\B\4\X2:Global declarations\X${}\E{}$\6
\&{typedef} \&{struct} \&{tnode} ${}\{{}$\1\6
\&{struct} \&{tnode} ${}{*}\\{left},{}$ ${}{*}\\{right};{}$\6
\&{char} ${}{*}\\{data};{}$\2\6
${}\}{}$ \&{TNODE};\par
\As10, 13\ETs15.
\U1.\fi

\M{3}\B\X3:\PB{\\{main}} variable declarations\X${}\E{}$\6
\&{struct} \&{tnode} ${}{*}\\{root}\K\NULL{}$;\par
\U1.\fi

\N{1}{4}Input.
Reading the tree is simple---we read one line at a time, and call on the
recursive \PB{\\{add\_tree}} procedure.

\Y\B$\\{read\_tree}(\\{fp},\39\\{rootptr}){}$\1\1\6
\&{FILE} ${}{*}\\{fp};{}$\6
\&{struct} \&{tnode} ${}{*}{*}\\{rootptr};\2\2{}$\6
${}\{{}$\1\6
\&{char} \\{buf}[\T{255}]${},{}$ ${}{*}\|p;{}$\7
\&{while} ${}((\\{fgets}(\\{buf},\39\T{255},\39\\{fp}))\I\NULL){}$\5
${}\{{}$\1\6
\X6:If \PB{\\{buf}} contains a newline, make it end there\X;\6
${}\\{add\_tree}(\\{rootptr},\39\\{buf});{}$\6
\4${}\}{}$\2\6
\4${}\}{}$\2\par
\fi

\M{5}\B\X5:Global include files\X${}\E{}$\6
\8\#\&{include} \.{<stdio.h>}\par
\U1.\fi

\M{6}Depending what system you're on, you may or may not get a newline in \PB{%
\\{buf}}.
\Y\B\4\X6:If \PB{\\{buf}} contains a newline, make it end there\X${}\E{}$\6
$\|p\K\\{buf};{}$\6
\&{while} ${}({*}\|p\I\.{'\\0'}\W{*}\|p\I\.{'\\n'}){}$\1\5
${}\|p\PP;{}$\2\6
${}{*}\|p\K\.{'\\0'}{}$;\par
\U4.\fi

\M{7}
To add a string, we split off the first part of the name and insert it into
the sibling list. We then do the rest of the string as a child of the new node.

\Y\B$\\{add\_tree}(\\{rootptr},\39\|p){}$\1\1\6
\&{struct} \&{tnode} ${}{*}{*}\\{rootptr};{}$\6
\&{char} ${}{*}\|p;\2\2{}$\6
${}\{{}$\1\6
\&{char} ${}{*}\|s;{}$\6
\&{int} \\{slashed};\7
\&{if} ${}({*}\|p\E\.{'\\0'}){}$\1\5
\&{return};\2\6
\X8:Break up the string so \PB{\|p} is the first word, \PB{\|s} points at
null-begun remainder, and \PB{\\{slashed}} tells whether \PB{${*}\|s\E\.{'/'}$}
on entry\X;\6
\&{if} ${}({*}\\{rootptr}\E\NULL){}$\5
${}\{{}$\1\6
\X9:Allocate new node to hold string of size \PB{\\{strlen}(\|p)}\X;\6
${}\\{strcpy}(({*}\\{rootptr})\MG\\{data},\39\|p);{}$\6
\4${}\}{}$\2\6
\&{if} ${}(\\{strcmp}(({*}\\{rootptr})\MG\\{data},\39\|p)\E\T{0}){}$\5
${}\{{}$\1\6
\&{if} (\\{slashed})\1\5
${}\PP\|s;{}$\2\6
${}\\{add\_tree}({\AND}(({*}\\{rootptr})\MG\\{child}),\39\|s);{}$\6
\4${}\}{}$\2\6
\&{else}\5
${}\{{}$\1\6
\&{if} (\\{slashed})\1\5
${}{*}\|s\K\.{'/'};{}$\2\6
${}\\{add\_tree}({\AND}(({*}\\{rootptr})\MG\\{sibling}),\39\|p);{}$\6
\4${}\}{}$\2\6
\4${}\}{}$\2\par
\fi

\M{8}We perform some nonsense to cut off the string \PB{\|p} so that \PB{\|p}
just
holds the first word of a multiword name. Variable \PB{\|s} points at what
was either the end of \PB{\|p} or a slash delimiting names. In either case
\PB{${*}\|s$} is made \PB{\.{'\\0'}}.  Later, depending on whether we want to
pass the
whole string or the last piece, we will restore the slash or advance
\PB{\|s} one character to the right.

\Y\B\4\X8:Break up the string so \PB{\|p} is the first word, \PB{\|s} points at
null-begun remainder, and \PB{\\{slashed}} tells whether \PB{${*}\|s\E\.{'/'}$}
on entry\X${}\E{}$\6
\&{for} ${}(\|s\K\|p;{}$ ${}{*}\|s\I\.{'\\0'}\W{*}\|s\I\.{'/'};{}$ \,)\1\5
${}\|s\PP;{}$\2\6
\&{if} ${}({*}\|s\E\.{'/'}){}$\5
${}\{{}$\1\6
${}\\{slashed}\K\T{1};{}$\6
${}{*}\|s\K\.{'\\0'};{}$\6
\4${}\}{}$\2\6
\&{else}\1\5
${}\\{slashed}\K\T{0}{}$;\2\par
\U7.\fi

\M{9}Node allocation is perfectly standard \dots
\Y\B\4\X9:Allocate new node to hold string of size \PB{\\{strlen}(\|p)}\X${}%
\E{}$\6
${*}\\{rootptr}\K{}$(\&{struct} \&{tnode} ${}{*}){}$ \\{malloc}(\&{sizeof}(%
\&{struct} \&{tnode}));\6
${}({*}\\{rootptr})\MG\\{left}\K({*}\\{rootptr})\MG\\{right}\K\NULL;{}$\6
${}({*}\\{rootptr})\MG\\{data}\K\\{malloc}(\\{strlen}(\|p)+\T{1}){}$;\par
\U7.\fi

\M{10}
\Y\B\4\X2:Global declarations\X${}\mathrel+\E{}$\6
\&{char} ${}{*}\\{malloc}(\,){}$;\par
\fi

\M{11}In this simple implementation, we just read from standard input.
\Y\B\4\X11:Read output from find and enter into tree\X${}\E{}$\6
$\\{read\_tree}(\\{stdin},\39{\AND}\\{root}){}$;\par
\U1.\fi

\N{1}{12}Output.
We begin by defining some lines, tees, and corners.
The \PB{\|s} stands for screen and the \PB{\|p} for printer.
You will have to change this for your line-drawing set.

\Y\B\4\X12:Global definitions\X${}\E{}$\6
\8\#\&{define} \\{svert}\5\.{'|'}\6
\8\#\&{define} \\{shoriz}\5\.{'-'}\6
\8\#\&{define} \\{scross}\5\.{'+'}\6
\8\#\&{define} \\{scorner}\5\.{'\\\\'}\C{ lower left corner }\6
\8\#\&{define} \\{pvert}\5\.{'|'}\6
\8\#\&{define} \\{phoriz}\5\.{'-'}\6
\8\#\&{define} \\{pcross}\5\.{'+'}\6
\8\#\&{define} \\{pcorner}\5\.{'\\\\'}\C{ lower left corner }\par
\U1.\fi

\M{13}The default is to use the terminal's line drawing set.
\Y\B\4\X2:Global declarations\X${}\mathrel+\E{}$\6
\&{char} \\{vert}${}\K\\{svert};{}$\6
\&{char} \\{horiz}${}\K\\{shoriz};{}$\6
\&{char} \\{cross}${}\K\\{scross};{}$\6
\&{char} \\{corner}${}\K\\{scorner}{}$;\par
\fi

\M{14}With option \PB{\.{"-p"}} use the printer character set.
\Y\B\4\X14:Search for options and set special characters on \PB{\.{"-p"}}\X${}%
\E{}$\6
\&{while} ${}(\MM\\{argc}>\T{0}){}$\5
${}\{{}$\1\6
\&{if} ${}({*}{*}\PP\\{argv}\E\.{'-'}){}$\5
${}\{{}$\1\6
\&{switch} ${}({*}\PP({*}\\{argv})){}$\5
${}\{{}$\1\6
\4\&{case} \.{'p'}:\5
${}\\{vert}\K\\{pvert};{}$\6
${}\\{horiz}\K\\{phoriz};{}$\6
${}\\{cross}\K\\{pcross};{}$\6
${}\\{corner}\K\\{pcorner};{}$\6
\&{break};\6
\4\&{default}:\5
${}\\{fprintf}(\\{stderr},\39\.{"treeprint:\ bad\ opti}\)\.{on\ -\%c\\n"},%
\39{*}{*}\\{argv});{}$\6
\&{break};\6
\4${}\}{}$\2\6
\4${}\}{}$\2\6
\4${}\}{}$\2\par
\U1.\fi

\M{15}We play games with a character stack to figure out when to put in
vertical
bars.
A vertical bar connects every sibling with its successor, but the last sibling
in a list is followed by blanks, not by vertical bars. The state of
bar-ness or space-ness for each preceding sibling is recorded in the
\PB{\\{indent\_string}} variable, one character (bar or blank) per sibling.

\Y\B\4\X2:Global declarations\X${}\mathrel+\E{}$\6
\&{char} \\{indent\_string}[\T{100}]${}\K\.{""}{}$;\par
\fi

\M{16} Children get printed
before siblings.
We don't bother trying to bring children up to the same line as their parents,
because the \UNIX/ filenames are so long.

We define a predicate telling us when a sibling is the last in a series.
\Y\B\4\D\\{is\_last}$(\|S)$\5
$(\|S\MG\\{sibling}\E\NULL{}$)\par
\Y\B$\\{print\_node}(\\{fp},\39\\{indent\_string},\39\\{node}){}$\1\1\6
\&{FILE} ${}{*}\\{fp};{}$\6
\&{char} ${}{*}\\{indent\_string};{}$\6
\&{struct} \&{tnode} ${}{*}\\{node};\2\2{}$\6
${}\{{}$\1\6
\&{char} \\{string}[\T{255}];\6
\&{int} \|i;\6
\&{char} ${}{*}\|p,{}$ ${}{*}\\{is};{}$\7
\&{if} ${}(\\{node}\E\NULL){}$\5
${}\{\,\}{}$\6
\&{else}\5
${}\{{}$\1\6
${}{*}\\{string}\K\.{'\\0'};{}$\6
\&{for} ${}(\|i\K\\{strlen}(\\{indent\_string});{}$ ${}\|i>\T{0};{}$ ${}\|i%
\MM){}$\1\5
${}\\{strcat}(\\{string},\,\39\.{"\ |\ \ "});{}$\2\6
${}\\{strcat}(\\{string},\39\hbox{\ \ }\.{"\ +--"});{}$\6
\X17:Replace chars in \PB{\\{string}} with chars from line-drawing set and from
\PB{\\{indent\_string}}\X;\6
${}\\{fprintf}(\\{fp},\39\.{"\%s\%s\\n"},\39\\{string},\39\\{node}\MG%
\\{data}){}$;\C{ Add vertical bar or space for this sibling (claim \PB{${*}%
\\{is}\E\.{'\\0'}$}) }\6
${}{*}\\{is}\PP\K(\\{is\_last}(\\{node})\?\.{'\ '}:\\{vert});{}$\6
${}{*}\\{is}\K\.{'\\0'};{}$\6
${}\\{print\_node}(\\{fp},\39\\{indent\_string},\39\\{node}\MG\\{child}){}$;\C{
extended \PB{\\{indent\_string}} }\6
${}{*}\MM\\{is}\K\.{'\\0'};{}$\6
${}\\{print\_node}(\\{fp},\39\\{indent\_string},\39\\{node}\MG\\{sibling}){}$;%
\C{ original \PB{\\{indent\_string}} }\6
\4${}\}{}$\2\6
\4${}\}{}$\2\par
\fi

\M{17}For simplicity, we originally wrote connecting lines with \PB{\.{'|'}}, %
\PB{\.{'+'}}, and
\PB{\.{'-'}}.
Now we replace those characters with appropriate characters from the
line-drawing set.
We take the early vertical bars and replace them with characters from
\PB{\\{indent\_string}}, and we replace the other characters appropriately.
We are sure to put a \PB{\\{corner}}, not a \PB{\\{cross}}, on the last sibling
in
a group.
\Y\B\4\X17:Replace chars in \PB{\\{string}} with chars from line-drawing set
and from \PB{\\{indent\_string}}\X${}\E{}$\6
$\\{is}\K\\{indent\_string};{}$\6
\&{for} ${}(\|p\K\\{string};{}$ ${}{*}\|p\I\.{'\\0'};{}$ ${}\|p\PP){}$\1\6
\&{switch} ${}({*}\|p){}$\5
${}\{{}$\1\6
\4\&{case} \.{'|'}:\5
${}{*}\|p\K{*}\\{is}\PP;{}$\6
\&{break};\6
\4\&{case} \.{'+'}:\5
${}{*}\|p\K(\\{is\_last}(\\{node})\?\\{corner}:\\{cross});{}$\6
\&{break};\6
\4\&{case} \.{'-'}:\5
${}{*}\|p\K\\{horiz};{}$\6
\&{break};\6
\4\&{default}:\5
\&{break};\6
\4${}\}{}$\2\2\par
\U16.\fi

\M{18}For this simple implementation, we just write on standard output.

\Y\B\4\X18:Write tree on standard output\X${}\E{}$\6
$\\{print\_node}(\\{stdout},\39\\{indent\_string},\39\\{root}){}$;\par
\U1.\fi

\N{1}{19}Index.
\fi

\inx
\fin
\con
