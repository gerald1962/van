\documentstyle[11pt,german,twoside,a4,makeidx]{report}
\pagestyle{headings}
\makeindex
%
% Neue Schriften
%
\newfont{\itg}{cmu10 scaled 1000}
\newfont{\ninerm}{cmr9 scaled 1000}
\newfont{\eightrm}{cmr8 scaled 1000}
\newfont{\sevenrm}{cmr7 scaled 1000}
\newfont{\fiverm}{cmr5 scaled 1000}
\newfont{\grossfiverm}{cmr5 scaled 2000}
\newfont{\bigtenrm}{cmr10 scaled\magstep1}
\newfont{\Bigtenrm}{cmr10 scaled\magstep2}
\newfont{\Mf}{cmex10 scaled 1000}
\newfont{\manual}{manfnt scaled 1000}
%
% Ma\ss{}einheit f\"ur Bilder
%
\unitlength1cm
%
\begin{document}
\title{Das kleine\\[1ex] {\Huge\bf \TeX Buch}}
\author{Fritz Cremer}
\date{5.7.1993}
\maketitle
\chapter*{Vorwort}
\section*{Warum dieses Buch?}
Ich habe mich entschlossen dieses Buch zu schreiben, weil in vielen Texten zum
Programm \TeX\ das \TeX book angesprochen wird, es aber kaum ein vern\"unftiges
Buch zu \TeX\ in deutscher Sprache gibt. Au\ss{}erdem wollte ich den Text in der
DF\"U zur Verf\"ugung stellen. Daraus ergaben sich verschiedene Probleme. Zum
einen mu\ss{}te ich die Urheberrechte ber\"uck\-sich\-tigen, die auf dem \TeX book
liegen, zum anderen durfte der Text nicht zu lang werden. Ich habe mich daher
f\"ur folgendes Vorgehen entschlossen. Ich habe das \TeX book nicht einfach
linear \"ubersetzt, das w\"are auch fast nicht m\"oglich, da dieses Buch viel zu
viele Wortspiele u.a.\ enth\"alt. Ich habe vielmehr das \TeX book Abschnitt f\"ur
Abschnitt gelesen und in meinen eigenen Worten wiedergegeben. Dabei konnte ich viel
k\"urzen. So enth\"alt dieser Text keine \"Ubungsaufgaben mehr. Ich habe die
Aufgaben, soweit sie vorkamen mit in den Text eingearbeitet, so da\ss{} keine
Information verloren ging.
\section*{An wen richtet sich das Buch?}
Dieser Text richtet sich an die Benutzer von \TeX\ und nicht so sehr
an die \index{Makro!Design}
Makrodesigner f\"ur \TeX, wenn auch dieses Kapitel keineswegs
ausgenommen wurde. Ich habe mich nur entschlossen die Teile des Buches
wegzulassen, die so speziell auf das Design von Makros ausgerichtet
sind, da\ss{} sie normalerweise nicht von Interesse sein k\"onnen. So fehlt
z.B.\ der Anhang `Dirty tricks', in dem un\"ublicher Gebrauch von Teilen von 
\TeX\ beschrieben wird.
\section*{Copyright und \"Ahnliches}
Dieser Text kann beliebig weitergegeben werden und auch in
\index{File!Netze}Filenetzen
vertrieben werden. Ich bitte sogar ausdr\"ucklich darum, da\ss{} dieses Buch
m\"oglichst vielen Mitgliedern der DF\"U zu\-g\"ang\-lich gemacht wird. Die
Rechte an diesem Text bleiben allerdings bei mir.
\section*{Fehler}
Dieser Text entstand im Laufe vieler Monate. Es bleibt nicht aus, da\ss{}
im Verlaufe dieser Zeit bessere und weniger gute Abschnitte
entstanden. Auch werden sich noch etliche Recht\-schreib\-feh\-ler in diesem
Buch befinden. Verbesserungsvorschl\"age und Korrekturhinweise bitte ich
mir zukommen zu lassen.

\begin{center}
\fbox{\fbox{\parbox{12cm}{
Wem dieses Buch gef\"allt, der empfehle und kopiere es weiter,
wems nicht gef\"allt, der schweige stille.
}}}
\end{center}

\vspace*{2cm}
\hspace{\fill}{\bf Fritz Cremer}, S\"udstr. 45, 52064 Aachen, Deutschland
\tableofcontents
\chapter{Einf\"uhrung}
Wenn sie einmal ein Buch durchbl\"attern, dann werden sie feststellen,
da\ss{} es verschiedene \index{Bindestrich}
Bindestriche gibt, je nachdem, ob W\"orter durch
Bindestriche verbunden werden (wie bei Tren-nung-en), ob
\index{Seite!Bereich}Seitenbereiche angegeben werden (Seiten: 2--13), ob
\index{Gedankenstrich}Gedankenstriche
--- ge\-schrie\-ben werden sollen, oder ob in einer mathematischen
\index{Formel!Minuszeichen}Formel ein \index{Minuszeichen}
Minuszeichen auftaucht ($1-2$).

Auch k\"onnte ihnen auffallen, da\ss{} manche Buchstaben wie {\em ein}
Zeichen aussehen, so z.B.\ die Kombination von `f' und `i', die dann
als `fi' erscheint. Die Kombination von Zeichen zu {\em einem} Zeichen
hei\ss{}t \"ubrigens \index{Ligatur}
{\em Ligatur}. Derartige Ligaturen, werden f\"ur viele
Zeichenkombinationen benutzt, wie z.B.\ f\"ur `ff', `fl' oder `ff'.

Aber es gibt noch weitere Merkw\"urdigkeiten. Sie finden z.B.\ auf der
Tastatur ihres Com\-put\-ers keine Taste f\"ur ein Paar
\index{Doppelt!Anfuehrungszeichen@Doppelt!Anf\"uhrungszeichen}doppelter
\index{Anfuehrungszeichen@Anf\"uhrungszeichen}
Anf\"uhrungszeichen. Das liegt daran, da\ss{} es es im \index{ASCII}
ASCII-Zeichensatz
keine derartigen Zeichen gibt. Dennoch tauchen in B\"uchern diese
Zeichen ``auf''. In \TeX\ k\"onnen sie das erreichen, indem sie einfach
zweimal ein einfaches Anf\"uhrungszeichen, bzw.\
\index{Abfuehrungszeichen@Abf\"uhrungszeichen}``Abf\"uhrungszeichen''
benutzen. Falls es auf ihrer Tastatur kein einfaches Anf\"uhrungszeichen
gibt, k\"onnen sie \"ubrigens auch den Befehl \index{zzzlq@{\tt\char92 lq}}
`\verb|\lq|' f\"ur das linke,
bzw.\ `\verb|\rq|' f\"ur das rechte An\-f\"uhr\-ungs\-zei\-chen benutzen. Sie
m\"ussen in diesem Fall nur ein `\verb|\|' hinter den Befehl schreiben,
wenn dem `\verb|\lq|' oder `\verb|\rq|' ein \index{Leer!Zeichen}
Leerzeichen folgt. Mit:
\begin{verbatim*}
\lq\lq Haken\rq\rq\ 
\end{verbatim*}
Erhalten sie:
\begin{quote}
\lq\lq Haken\rq\rq\ 
\end{quote}
Der Trick ist nat\"urlich, da\ss{} auch die beiden Anf\"uhrungszeichen zu
einem Zeichen zusammen\-ge\-fa\ss{}t werden, genau wie bei dem `fi'. Da nur
zwei Anf\"uhrungszeichen zusammengefa\ss{}t werden, ergibt die Kombination
von drei Anf\"uhrungszeichen: ```. So kann man also ein einfaches
Anf\"uhrungszeichen innerhalb von einem doppelten schreiben.

\"Ubrigens, folgendes (`\thinspace``) erreichen sie nicht so einfach,
dazu ist ein weiterer Befehl n\"otig, geben sie:
\index{zzzthinspace@{\tt\char92 thinspace}}
\begin{verbatim*}
`\thinspace``
\end{verbatim*}
ein, um das gew\"unschte Resultat zu erhalten.

Schlie\ss{}lich finden sich Zeichen, die {\em untereinandergeschoben}
scheinen. So zu sehen, wenn ein `V' hinter einem `A' steht. Das sieht
dann so: `VA' aus. Dieses Untereinanderschieben nennt man
\index{Kerning}{\em Kerning}.

Jetzt die gute Nachricht: Mit \TeX\ brauchen sie sich um derartige
Kleinigkeiten nicht mehr zu k\"ummern; die verschiedenen
\index{Bindestrich}Bindestriche,
die \index{Ligatur}Ligaturen und auch das \index{Kerning}
Kerning wird von \TeX\ v\"ollig selbst\"andig
erledigt. Sie k\"onnen sich also ganz auf ihre eigentliche Arbeit, das
Schreiben des Textes konzentrieren.
\chapter{Bis zum ersten Text ...}
\section{\TeX\ kontrollieren}
Die Tastatur ihres Computers bietet viel weniger M\"oglichkeiten, als
sie f\"ur ein \index{Schrift!Stueck@Schrift!St\"uck}
Schriftst\"uck brauchen. Daher mu\ss{} ein Zeichen als
sogenanntes \index{Escape}
{\em Escape}zeichen definiert werden, mit dem sie dann
Befehle, die die Form oder das Aussehen des Textes ver\"andern, oder die
nur ein weiteres Zeichen zur Verf\"ugung stellen. In \TeX\ ist es zwar
prinzipiell m\"oglich jedes Zeichen dazu zu verwenden, \"ublicherweise
wird aber der \index{Gegenschraegstrich@Gegenschr\"agstrich}
Gegenschr\"agstrich, `\verb|\|', dazu verwendet. Der Grund
hierf\"ur ist einfach, dieses Zeichen kommt in normalen Texten sogut wie
nie vor,

\TeX\ versteht jede \index{Eingabe}
Eingabe, die mit diesem \index{Gegenschraegstrich@Gegenschr\"agstrich}
Gegenschr\"agstrich beginnt
als eine \index{Kontrollsequenz} {\em Kon\-troll\-se\-quenz}. Die Eingabe von:
\index{zzzinput@{\tt\char92 input}}
\begin{verbatim*}
\input neu
\end{verbatim*}
z.B.\ bedeutet, da\ss{} `\verb|\input|' eine \index{Kontrollsequenz}
Kontrollsequenz ist, und
`\verb|neu|' sein \index{Argument}
Argument (Tats\"achlich veranla\ss{}t dieser Befehl \TeX\
als n\"achstes den \index{File!Einlesen}
File `\verb|neu.tex|' einzulesen und diesen zu be\-ar\-bei\-ten).

In der Zeile:
\begin{verbatim*}
Neue Eintr\\"age
\end{verbatim*}
findet sich die \index{Kontrollsequenz}
Kontrollsequenz `\verb|\\"|'. Diese sorgt daf\"ur, da\ss{}
\"uber das folgende Zeichen zwei Strichelchen geschrieben werden, das
Ergebnis sieht also so `\"a' aus.

Im folgenden wird zwischen zwei Arten von \index{Kontrollsequenz}
Kontrollsequenzen
unterschieden. Zum einen die, bei denen hinter dem
\index{Gegenschraegstrich@Gegenschr\"agstrich}Gegenschr\"agstrich
eine Reihe von Buchstaben folgt (Buchstaben sind: `\verb|A...Z|' und
`\verb|a...z|', aber nicht `\verb|0...9|'). Die obige Sequenz
\index{zzzinput@{\tt\char92 input}}
`\verb|\input|' geh\"ort z.B.\ zu dieser Kategorie; sie wird {\em
Kontrollwort} genannt. Zum anderen die, bei denen hinter dem
\index{Gegenschraegstrich@Gegenschr\"agstrich}
Gegenschr\"agstrich nur ein Zeichen folgt, das {\em kein} Buchstabe ist.
Zur letzten Kategorie geh\"ort auch das `\verb|\\"|'. Diese Sequenzen
hei\ss{}en \index{Kontrollzeichen}{\em Kontrollzeichen}.

Folgt einem Kontrollwort ein \index{Leer!Zeichen}
Leerzeichen, dann wird es ignoriert.
Dieses Leerzeichen dient quasi nur der Begrenzung des Kontrollwortes.
Was nun aber, wenn sie ein Leerzeichen hinter einem Kontrollwort
schreiben wollen? Da \TeX\ mehrere Leerzeichen immer zu einem
zusammenfa\ss{}t, w\"are es falsch einfach zwei, statt des einen
Leerzeichens zu schreiben. Statt dessen gibt es eine
\index{Kontrollsequenz}Kontrollsequenz,
die genau das gew\"unscht Ergebnis bereith\"alt:\footnote{Das Zeichen {\tt\char32}
bezeichnet dabei ein Leerzeichen.}
\begin{quote}
{\tt\char92\char32}
\end{quote}
Da, wie schon eben gesagt, \TeX\ mehrere Leerzeichen immer zu einem
zusammenfa\ss{}t, m\"u\ss{}ten sie, wollten sie drei Leerzeichen hinter einem
Kontrollwort erzeugen schon: `\verb*|\ \ \ |' schrei\-ben, um das
Gew\"unschte zu erhalten.

\"Ubrigens, statt des \index{Leer!Zeichen}
Leerzeichens kann auch ein Tabulatorzeichen, oder
ein Zeilentrenner stehen, da \TeX\ letztere in Leerzeichen umwandelt.

Sie fragen sich sicher schon, wieso man hinter einem Kontrollwort ein
Leerzeichen schrei\-ben will? Nun, die Antwort ist ganz einfach: Um z.B.
die Zeichenfolge `\TeX' zu erhalten, gibt es den Befehl:
`\verb|\TeX|'. Wenn sie nun einen Satz wie:
\begin{quote}
\TeX\ gef\"allt mir.
\end{quote}
schreiben wollen, dann m\"ussen sie eingeben:
\begin{verbatim*}
\TeX\ gef\"allt mir.
\end{verbatim*}
W\"urden sie `\verb|\TeX|' nicht mit einem
\index{Gegenschraegstrich@Gegenschr\"agstrich}Gegenschr\"agstrich
abschlie\ss{}en, dann s\"ahe das Ergebnis so aus:
\begin{quote}
\TeX gef\"allt mir.
\end{quote}
Und das gefiele uns dann nicht mehr. Es darf allerdings ein
Kontrollwort nicht in jedem Fall mit einem
\index{Gegenschraegstrich@Gegenschr\"agstrich}Gegenschr\"agstrich beendet
werden. Um z.B.\ `\TeX' zu erhalten, darf nicht:
\begin{verbatim*}
`\TeX\'
\end{verbatim*}
eingegeben werden, das Ergebnis w\"are:
\begin{quote}
`\TeX\'.
\end{quote}
Sie k\"onnen sich vermutlich schon denken, woher der
\index{Fehler!Kontrollzeichen}Fehler stammt,
`\verb|\'|' ist ein \index{Kontrollzeichen}Kontroll\-zei\-chen, was die gew\"unschte
\index{Ausgabe}Ausgabe verhinderte.

Derartige \index{Kontrollsequenz}
Kontrollsequenzen gibt es einige in \TeX, so z.B.\ `\verb|\pi|'
f\"ur den Buchstaben `$\pi$', `\verb|\Pi|' f\"ur `$\Pi$' und so weiter.
Sie sehen hier schon eine wichtige Eigenschaft von \TeX. Die
Kontrollsequenzen sind alle so gew\"ahlt, da\ss{} sie m\"oglichst {\em
klingende} Namen haben. Dies gew\"ahrleistet, da\ss{} sie die wichtigsten
schnell lernen k\"onnen.

Die \index{Kontrollsequenz}
Kontrollsequenzen werden weiterhin unterschieden in sogenannte:
\index{Primitiv}
{\em Primitive} und {\em zusammengesetzte} \index{Kontrollsequenz}
Kontrollsequenzen.
Primitive sind nicht aus anderen \index{Kontrollsequenz}
Kontrollsequenzen zusammengesetzt. Zu
dieser Kategorie geh\"oren \index{zzzinput@{\tt\char92 input}}
`\verb|\input|' oder \index{zzzaccent@{\tt\char92
accent}}`\verb|\accent|'.
Wohingegen `\verb|\\"|' ein zusammengesetzte \index{Kontrollsequenz}
Kontrollsequenz ist, die
u.a.\ auf `\verb|\accent|' zur\"uckgreift. Sie werden sich fragen, warum
man \"uberhaupt zusammengesetzte \index{Kontrollsequenz}
Kontrollsequenzen zur Verf\"ugung stellt.
Nun ganz einfach, die anderen sind einfach zu \index{Primitiv}
{\bf primitiv}. Es w\"are
zu m\"uhsam einen Text nur mit Primitiven zu schreiben, ausserdem kann
man mit der \"Anderung {\em einer} zusammengesetzten
\index{Kontrollsequenz}Kontrollsequenz das
Aussehen im ganzen Text beeinflu\ss{}en. Wie man aus \TeX\
Kontrollsequenzen neue zusammensetzt, und wie man die Sequenzen
\"andert, lesen sie in einem sp\"ateren Kapitel.

Insgesamt bietet ein nicht schon ver\"andertes \TeX\ ca.\ 300
\index{Primitiv}Primitive
und 600 zu\-sam\-men\-ge\-setz\-te \index{Kontrollsequenz}
Kontrollsequenzen an. Erschrecken sie nicht
vor der gro\ss{}en Zahl. Mit den meisten werden sie nie zu tun haben, und
die, die sie brauchen haben gr\"o\ss{}tenteils selbsterkl\"arende Namen.

Wie kann man nun unterscheiden, ob eine \TeX\ \index{Kontrollsequenz}
Kontrollsequenz
zusammengesetzt, oder \index{Primitiv}
primitiv ist? Nun, man kann nat\"urlich in diesem
Buch nachlesen, wo die Befehle erkl\"art werden, man kann aber auch
\TeX\ veranlassen die Antwort zu geben. Dazu gibt es eine weitere
Kontrollsequenz:
\index{zzzshow@{\tt\char92 show}}
\begin{verbatim*}
\show
\end{verbatim*}
Wenn sie in einen \TeX\ File \index{zzzinput@{\tt\char92 input}}
`\verb|\show\input|' schreiben, dann gibt
\TeX\ auf dem \index{Bildschirm}Bildschirm aus:
\begin{verbatim*}
> \input=\input.
\end{verbatim*}
Sie erinnern sich daran, da\ss{} `\verb|\input|' eine \index{Primitiv}primitive
\index{Kontrollsequenz}
Kontrollsequenz ist. Anders hingegen die Reaktion, wenn sie auf
\"ahnliche Art den Befehl: \index{zzzthinspace@{\tt\char92 thinspace}}
`\verb|\thinspace|' untersuchen. Sie
erhielten das Ergebnis:
\index{zzzkern@{\tt\char92 kern}}
\begin{verbatim}
\> thinspace=macro:
->\kern .16667em.
\end{verbatim}
Hierbei ist `\verb|\kern|' allerdings wieder ein \index{Primitiv}Primitiv.

\section{Die Zeichens\"atze}
\subsection{Die Schrifttypen}
Mitunter wollen sie auf einen anderen Zeichensatz umschalten, z.B.
etwas \index{Fettschrift}{\bf fett} schreiben, oder durch {\sl schr\"agstellen}
hervorheben. Um dies zu erreichen m\"ussen sie einfach einige Befehle in
den Text schreiben, die die Schrift umschalten. Mit der
\index{Eingabe}Eingabe von:
\begin{verbatim}
z.B.\ etwas \bf fett \rm schreiben, oder durch \sl schr\"agstellen \rm her..
\end{verbatim}
erhalten sie obiges Ergebnis. \TeX\ arbeitet mit Zeichen, die zu
\index{Gruppe!Von Zeichen}
Gruppen von 256 Zeichen zusammengefa\ss{}t sind, die sogenannten
Zeichens\"atze. \TeX\ bietet folgende Befehle um auf einen anderen
Zeichensatz umzuschalten:

\begin{quote}\index{Italic}\index{Schreibmaschinenschrift}
\index{zzzit@{\tt\char92 it}}
\begin{tabular}{ll}
{\tt\char92 rm} schaltet auf die Schriftart \index{Roman}`Roman' & Roman \\
{\tt\char92 sl} schaltet auf die geneigte Roman Schrift & {\sl Geneigt} \\
{\tt\char92 it} schaltet auf die Schrift `Italic' & {\it Italic} \\
{\tt\char92 tt} schaltet auf eine Schreibmaschinenschrift & {\tt Schreibmaschine} \\
{\tt\char92 bf} schaltet auf fette Schrift um & {\bf Fett}
\end{tabular}
\end{quote}

Es gibt einen Unterschied zwischen der geneigten und er
Italic-Schrift. Der Unterschied liegt darin, da\ss{} es sich bei der
geneigten Schrift um die Schriftart \index{Roman}`Roman' handelt, die einfach nur
geneigt wurde. Die Schriftart `Italic' ist dagegen eine ganz andere
Schrift. {\itg Am augenf\"alligsten wird das, wenn sie sich mal die
ungeneigte Italic Schrift ansehen}\footnote{Wie soeben geschehen.}.
Erst mit dem Fortschritt in der
Erarbeitung von Zeichens\"atzen, wurde der Unterschied zwischen der
geneigten und der Italic Schrift relevant. Fr\"uher wurde kein
Unterschied zwischen Texten, die geneigt werden sollten und
mathematischen Zeichen gemacht. Mittlerweile hat es sich eingeb\"urgert,
Texte in der geneigten, mathematische \index{Formel!Schrift}
Formeln aber in der Schrift
`Italic' zu schreiben. Sie kennen die `Italic' Schrift vermutlich aus
mathematischen B\"uchern, als Bezeichnung f\"ur Variablen.

\TeX\ beginnt die Arbeit \"ublicherweise in der Schriftart \index{Roman}
`Roman'. Der
Grund hierf\"ur liegt darin, da\ss{} die Schriftart `Roman' speziell auf
gute Lesbarkeit ausgelegt ist. Sie sollten sich angew\"ohnen, die
Texthervorhebungen nur in geringem Ma\ss{}e zu benutzen. L\"angere Texte in
einer anderen Schriftart, als \index{Roman}Roman, sind nicht gut lesbar.

Nachdem sie auf eine andere Schriftart umgeschaltet haben, m\"ussen sie
normalerweise erst wieder auf die Schriftart \index{Roman}
`Roman' zur\"uckschalten,
bevor sie weiter schreiben k\"onnen. Dies kann auch einfacher gel\"ost
werden, wenn sie den Text, der hervorgehoben werden soll, zusammen mit
dem Zeichensatzbefehl in geschweifte \index{Klammer!Geschweift}
Klammern schreiben. Das obige
Beispiel h\"atte man also auch schreiben k\"onnen, als:\footnote{Bei dieser
Art der Schriftart\"anderung handelt es sich um ein Beispiel f\"ur die
Gruppierung, die im n\"achsten Abschnitt vorgestellt wird.}
\begin{verbatim}
z.B.\ etwas {\bf fett} schreiben, oder durch {\sl schr\"agstellen} her..
\end{verbatim}
Die geneigten Schriften, also die geneigte \index{Roman}
Roman Schrift und die Schrift \index{Italic}
Italic, werfen ein kleines Problem auf. Betrachten sie z.B.
den folgenden Satz, der aus geraden und geneigten Zeichen gemischt ist:
\begin{quote}
Ich sehe {\sl f\"unf} Indianer.
\end{quote}
Dadurch, da\ss{} das Wort `f\"unf' in einer geneigten Schrift dargestellt
wird, erscheint der \index{Abstand}
Abstand zwischen dem zweiten `f' des Wortes `f\"unf'
und dem `I' von `Indianer' zu klein. F\"ur diesen Fall bietet \TeX\
einen besonderen Befehl `\verb|\/|'. Mit diesem Befehl wird ein
zus\"atzlicher \index{Zwischenraum}
Zwischenraum beim Umschalten von einer geneigten Schrift
auf eine gerade Schrift eingef\"ugt. Dieser Zwischenraum ist abh\"angig
vom Buchstaben, also f\"ur ein `f' \index{Groesse!Zeichen@Gr\"o\ss{}e!Zeichen}
gr\"o\ss{}er, als z.B.\ f\"ur ein `e'. Wird der obige Satz als:
\begin{verbatim}
Ich sehe {\sl f\"unf\/} Indianer.
\end{verbatim}
eingegeben, dann erscheint er so:
\begin{quote}
Ich sehe {\sl f\"unf\/} Indianer.
\end{quote}
Tats\"achlich gibt es f\"ur jedes Zeichen, also auch f\"ur die Zeichen der
geraden Schriften eine sogenannte \index{Italic!Korrektur}
{\em Italic Korrektur}, nur ist sie
bei den meisten Zeichen gleich Null.

Noch eine Nachbemerkung zu den verschiedenen Schriften. Fr\"uher galt es
als guter \index{Stil!Schreib}Schriftstil, einen \index{Punkt}
Punkt oder ein \index{Komma}Komma in derselben
Schriftart zu schreiben, in der auch das Wort davor geschrieben war.
Sp\"atestens bei einem `Italic' Semikolon sieht das aber ziemlich
l\"acherlich aus. Es hat sich daher durchgesetzt, die
\index{Satz!Zeichen}Satzzeichen in der
normalen Schriftart zu schreiben.

Schlie\ss{}lich gibt es bei \TeX\ noch einen Befehl f\"ur eine Schriftart,
die es gar nicht gibt. \index{Font!Null}\index{zzznullfont@{\tt\char92 nullfont}}
`\verb|\nullfont|' schaltet auf einen {\em
leeren}\index{Leer!Zeichensatz} Zeichensatz um, der keine Zeichen enth\"alt. Dieser wird
gebraucht, wenn man als \index{Argument}
Argument einen Zeichensatz angeben mu\ss{}, aber
gar keine Zeichen ausgeben will.

\subsection{Die Schriftgr\"o\ss{}en}
Schriften k\"onnen sich in der Gr\"o\ss{}e genauso voneinander unterscheiden,
wie in der Form. Die Schrift, die sie hier lesen hat die Gr\"o\ss{}e von
10pt\footnote{Das pt-Ma\ss{}system wird sp\"ater noch genauer vorgestellt.
Hier reicht es zu wissen, da\ss{} die \index{Klammer!Hoehe@Klammer!H\"ohe}
Klammern der Schrift 10pt hoch und
der \index{Gedankenstrich}
Gedankenstrich 10pt breit ist (\"Ubrigens nicht hier in der
\index{Fussnote!Schriftgroesse@Fu\ss{}note!Schriftgr\"o\ss{}e}Fu\ss{}note,
die Schrift ist kleiner)}.

Jeder Zeichensatz in \TeX\ steht in Zusammenhang mit einem
Kontrollwort. F\"ur die 10pt \index{Roman}Roman Schrift gilt z.B.\ der Befehl
`\verb|\tenrm|', f\"ur die entsprechende 9pt Schrift `\verb|\ninerm|'.
Die Kontrollworte f\"ur die entsprechenden geneigten Schriften hei\ss{}en
`\verb|\tensl|' bzw.\ `\verb|\ninesl|'. Diese Kontrollworte sind nicht
in \TeX\ {\em eingebaut}, sie sind keine \index{Primitiv}
Primitive, sie dienen nur der
besseren Ansprechbarkeit der Schriftgr\"o\ss{}en.

Wenn verschiedengro\ss{}e Schriften zusammen gebraucht werden, dann
orientiert sich \TeX\ an einer \index{Grundlinie}
{\em Grundlinie}. Die \index{Eingabe}Eingabe von:
\begin{verbatim}
\tenrm kleiner \ninerm und kleiner
\eightrm und kleiner \sevenrm und kleiner
\sixrm und kleiner \fiverm und kleiner \tenrm
\end{verbatim}
ergibt im Text: \tenrm kleiner \ninerm und kleiner \eightrm und
kleiner \sevenrm und kleiner \sixrm und kleiner \fiverm und
kleiner \tenrm.

Wir kennen jetzt zwei Befehle um auf die 10pt \index{Roman}Roman Schrift
umzuschalten `\verb|\tenrm|' und \index{zzzrm@{\tt\char92 rm}}
`\verb|\rm|'. Worin besteht der
Unterschied? Nun, `\verb|\rm|' schaltet gar nicht auf die 10pt Schrift
um, sondern in die derzeit aktuelle Gr\"o\ss{}e der Schriftart \index{Roman}
Roman. Der
Sinn dieses Befehls liegt darin, da\ss{} man ihn benutzen kann, ohne an
die augenblicklich g\"ultige Gr\"o\ss{}e der Schrift denken zu m\"ussen.
`\verb|\tenrm|' schaltet dagegen tats\"achlich auf die 10pt
\index{Roman}Roman
Schriftart um.

Wie bringt man \TeX\ nun bei, welchen Zeichensatz man benutzen m\"ochte.
Die Zeichens\"atze existieren ja als \index{File!Zeichensatz}
Zeichensatzfiles in ihrem System.
Nun, sie geben die \index{Definition}Definition:
\index{zzzfont@{\tt\char92 font}}
\begin{verbatim}
\font\ninerm=cmr9
\end{verbatim}
ein, und ab sofort wei\ss{} \TeX\ was unter dem Kontrollwort zu verstehen
ist, n\"amlich da\ss{} ab da der Zeichensatz zu verwenden ist, der in der
Datei {\em cmr9.xxx} steht (um das `xxx' k\"ummern wir uns
sp\"ater)\footnote{`cmr9' bedeutet \"ubrigens: {\em Computer modern
Roman} mit 9pt}.

Au\ss{}er den verschieden gro\ss{}en Zeichens\"atzen, kann man einen einzelnen
Zeichensatz auch noch vergr\"o\ss{}ert oder verkleinert benutzen. Das
definiert man:
\index{zzzfont@{\tt\char92 font}}
\begin{verbatim}
\font\cs=<externer Zeichensatzname> at <gew\"unschte Gr\"o\ss{}e>
\end{verbatim}
Mit `\verb|\cs|' steht dann der vergr\"o\ss{}erte Zeichensatz zur Verf\"ugung,
vorausgesetzt, er existiert in ihrem System. Mit:
\begin{verbatim}
\font\grossfiverm=cmr5 at 10pt
\end{verbatim}
erhalten sie z.B.\ die Schrift `cmr5' auf 10pt vergr\"o\ss{}ert, also in
\index{Doppelt!Schrift}doppelter Gr\"o\ss{}e.

Sie werden sich jetzt sicher fragen, wieso man denn dann \"uberhaupt
noch eine 10pt Schrift (cmr10) braucht? Der Grund ist einfach, die
beiden Schriften sehen unterschiedlich aus\footnote{Der Grund f\"ur das
unterschiedliche Aussehen liegt einfach darin, da\ss{} z.B.\ in der Schrift
cmr10 noch Striche von der St\"arke eins m\"oglich sind, bei der
vergr\"o\ss{}erten cmr5 Schrift, w\"are hingegen ein Strich der St\"arke eins
schon auf die St\"arke 2 verdoppelt worden.}. Sie sollten die 10pt
Schrift immer dann verwenden, wenn sie davon ausgehen k\"onnen, da\ss{} ihr
Text in Originalgr\"o\ss{}e gelesen werden wird, wenn ihr Text aber, z.B.
auf fotographischem Wege, verkleinert werden soll, dann w\"ahlen sie die
vergr\"o\ss{}erte 5pt Schrift. Um den Unterschied deutlich zu machen ein
Beispiel:
\begin{quote}
Hier in 10pt {\grossfiverm und hier vergr\"o\ss{}ert}
\end{quote}

Es gibt noch eine andere M\"oglichkeit den Vergr\"o\ss{}erungsfaktor einer
Schrift anzugeben. Mit der \index{Eingabe}Eingabe von:
\index{zzzfont@{\tt\char92 font}}
\begin{verbatim}
\font\grossfiverm=cmr5 scaled 2000
\end{verbatim}
erreichen sie dasselbe Ergebnis wie mit der Angabe von `\verb|at
10pt|', n\"amlich die \index{Doppelt!Schrift}
Verdoppelung. Die Zahlenangabe hinter dem Wort
`\verb|scaled|' gibt den Vergr\"o\ss{}erungsfaktor in \index{Promille}
Promille an. Die Basis
der Berechnung ist hier also 1000. 1500 w\"are die eineinhalbfache
Vergr\"o\ss{}erung.

Es hat sich bew\"ahrt Zeichens\"atze mit den
\index{Potenz!Vergroesserung@Potenz!Vergr\"o\ss{}erung}Potenzen von $1.2$ zu
vergr\"o\ss{}ern. Daher gibt es in \TeX\ auch \index{Kontrollsequenz}
Kontrollsequenzen, die diese
Vergr\"o\ss{}erungen einfach bereitstellen. Mit:
\index{zzzfont@{\tt\char92 font}}\index{zzzmagstep@{\tt\char92 magstep}}
\begin{verbatim}
\font\bigtenrm=cmr10 scaled\magstep2
\end{verbatim}
wird die 10pt Schrift in $1.2 \times 1.2$-facher Vergr\"o\ss{}erung
bereitgestellt. Die \index{Abkuerzung@Abk\"urzung}
abk\"urzende \index{Kontrollsequenz}Kontrollsequenz hei\ss{}t also
`\verb|\magstepn|', wobei `n' \"ublicherweise f\"ur: $0, 1, 2, 3, 4, 5$
steht. Die Vergr\"o\ss{}erungen sehen dann so aus:\footnote{Es gibt auch noch
die Vergr\"o\ss{}erungsstufe `{\tt\char92 magstephalf}', die eine Schrift um
den Faktor: $\sqrt{1.2}$ vergr\"o\ss{}ert, also genau zwischen 0 und 1 liegt.}
\begin{quote}
Hier normal `\verb|\magstep0|'\\
{\bigtenrm einmal um $1.2$ vergr\"o\ss{}ert `\verb|\magstep1|'}\\
{\Bigtenrm und zweimal um $1.2$ vergr\"o\ss{}ert `\verb|\magstep2|'}
\end{quote}

\section{Gruppierungen}
Mitunter ist es erw\"unscht, wenn ein Teil eines Textes, wie eine
Einheit behandelt wird. Dazu stellt \TeX\ die Zeichen `\verb|{|' und
`\verb|}|' bereit. Alles, was zwischen diesen Zeichen steht wird als
\index{Einheit!Text}Einheit behandelt. Die beiden \index{Klammer!Gruppe}
Klammern sind also \"ahnlich spezielle
Zeichen, wie das `\verb|\|'.

Die \index{Gruppe}Gruppen in \TeX\ haben eine \"ahnliche Aufgabe, wie die sogenannte
\index{Block!Struktur}
{\em Blockstruktur} bei Pro\-gram\-mier\-spra\-chen. \index{Definition}
Definitionen, die
innerhalb einer Gruppe getroffen wurden, gelten nur innerhalb der
Gruppe, und verlieren ihre G\"ultigkeit, sobald die Gruppe endet. In der
Sprache der Pro\-gram\-mier\-spra\-chen sind sie {\em lokal}.

Im letzten Abschnitt wurde schon ein Beispiel f\"ur Gruppen geliefert,
hier folgt ein weiteres. Wir haben schon gelernt, da\ss{} man bei dem
\index{Logo}Logo
`\TeX' unterscheiden mu\ss{}, welche \index{Kontrollsequenz}
Kontrollsequenz man schreibt, je
nachdem, ob ein \index{Leer!Zeichen}
Leerzeichen folgen soll, oder nicht. Man kann dieses
Problem auch l\"osen, indem man:
\begin{verbatim}
{\TeX}
\end{verbatim}
schreibt. Die zweite schlie\ss{}ende Klammer beendet die kleine Gruppe,
und danach wird alles wieder normal behandelt, d.h.\ ein
\index{Leer!Zeichen}Leerezeichen
ist dann nicht mehr \index{Begrenzer!Kontrollsequenz}
Begrenzer einer \index{Kontrollsequenz}Kontrollsequenz, sondern wieder
ein ganz normales Leerzeichen. Auch:
\begin{verbatim}
\TeX{}
\end{verbatim}
h\"atte denselben Erfolg gebracht, da sich nun der
\index{Kontrollsequenz}Kontrollsequenz
`\verb|\TeX|' eine leere Gruppe anschlie\ss{}t.

Als weiteres Beispiel, was uns dann weiter f\"uhren wird dient
folgendes:
\index{zzzcenterline@{\tt\char92 centerline}}
\begin{verbatim}
\centerline{Diese Information ist zentriert}
\end{verbatim}
Das Beispiel benutzt die \index{Kontrollsequenz}
Kontrollsequenz `\verb|\centerline|'. Diese
Kontrollsequenz \index{Zentrierung}
zentriert das fol\-gen\-de Zeichen in einer eigenen Zeile.
Die \index{Klammer!Gruppe}
Gruppenklammern sind n\"otig, da sonst nur das `D' von `Diese'
zentriert w\"urde, der Rest aber in die n\"achste Zeile geschrieben w\"urde,
--- nicht im Sinne des Erfinders.

Es ist nat\"urlich auch m\"oglich \index{Gruppe!Schachtelung}
Gruppen ineinanderzuschachteln. Die \index{Eingabe}Eingabe von:
\index{zzzit@{\tt\char92 it}}
\begin{verbatim}
\centerline{Diese Information ist {\it zentriert}}
\end{verbatim}
ergibt als Ergebnis:\medskip

\index{Zentrierung}\index{zzzcenterline@{\tt\char92 centerline}}
\centerline{Diese Information ist {\it zentriert}}\medskip

Ich habe oben geschrieben, da\ss{} \index{Definition}
Definitionen, die innerhalb von \index{Gruppe!Definitionen}Gruppen
getroffen werden, nur innerhalb der Gruppe G\"ultigkeit haben. Das gilt
auch f\"ur alle anderen Befehle innerhalb der Gruppe. Beim letzten
Beispiel betraf die Umschaltung auf \index{Italic}
Italic nur das Wort \index{Zentrierung}`zentriert'. Es
gibt mitunter Situationen, wo man innerhalb einer Gruppe etwas
ver\"andern m\"ochte und diese Ver\"anderung \"uber die \index{Gruppe!Grenze}
Gruppengrenzen hinaus wirken soll. Das ist m\"oglich, indem man den Befehl
\index{zzzglobal@{\tt\char92 global}}`\verb|\global|' vor den
Befehl schreibt. H\"atten wir im letzten Beispiel:
\index{zzzit@{\tt\char92 it}}
\begin{verbatim}
\centerline{Diese Information ist {\global\it zentriert}}
\end{verbatim}
geschrieben, dann w\"are auch aller nachfolgender Text in \index{Italic}
Italic
geschrieben worden, bis zu einem anderen, auf eine andere Schriftart
umschaltenden Befehl. \index{zzzglobal@{\tt\char92 global}}
`\verb|\global|' bewirkt, das der unmittelbar
darauf folgende Befehl f\"ur alle existierenden Gruppen wirksam wird.

Eine weitere M\"oglichkeit eine Gruppe zu begrenzen besteht mit den
\index{Primitiv} Primitiven \index{zzzbegingroup@{\tt\char92 begingroup}}
\index{zzzendgroup@{\tt\char92 endgroup}}
`\verb|\begingroup|' und `\verb|\endgroup|'. Diese
machen es leichter m\"oglich eine Gruppe in einer
\index{Kontrollsequenz}Kontrollsequenz
beginnen und in einer anderen enden zu lassen. Man mu\ss{} nur darauf
achten, da\ss{} sie die \index{Gruppe!Ueberlappung@Gruppe!\"Uberlappung}
Gruppen nicht \"uberlappen. Konstruktionen, wie:
\index{zzzbegingroup@{\tt\char92 begingroup}}\index{zzzendgroup@{\tt\char92 endgroup}}
\begin{verbatim}
{ \begingroup } \endgroup
\end{verbatim}
sind unzul\"assig.

\section{Die Arbeit mit \TeX}
Sie wissen nun schon genug, um mit \TeX, vom Brief angefangen bis zum
Buch alles schreiben zu k\"onnen (wirklich). Das Einzige, was noch
fehlt, ist der eigentliche Umgang mit dem Programm \TeX\ selber. In
diesem Abschnitt sollen ihnen einige Beispiele den Umgang mit dem
Programm \TeX\ verdeutlichen und ihnen das notwendige R\"ustzeug zu
eigenen Experimenten geben.

\subsection{Der erste Text}
Rufen sie \TeX\ auf\index{Betriebssystem}\footnote{Der Aufruf eines Programms ist von
Betriebssystem zu Betriebssystem unterschiedlich, der Pro\-gramm\-name von
\TeX\ lautet i.a.\ {\tt virtex}}. Es m\"u\ss{}te sich mit einer Meldung,
nicht un\"ahnlich dieser:
\begin{verbatim}
This is TeX, Version 3.0 (preloaded format=plain 89.7.15)
**
\end{verbatim}
melden. F\"ur den Anfang geben sie bitte
\index{zzzrelax@{\tt\char92 relax}}
`\verb|\relax|' ein, was \TeX\
veranla\ss{}t nichts zu tun, \TeX\ m\"u\ss{}te sich nun mit {\em einem}
\index{Stern!Prompt}
Sternchen wiedermelden. Versuchen sie nun `\verb|Hallo?|', oder etwas
\"ahnliches. Und zum Schlu\ss{} schreiben sie noch `\verb|\end|'.

\TeX\ sollte nun mit einem `\verb|[1]|' antworten (Es wurde soeben die
erste \index{Seite!Erste}
Seite ihres ersten Textes ausgegeben. Der Inhalt steht in einer
Datei mit dem Namen: {\tt texput.dvi}\footnote{{\tt dvi} hei\ss{}t {\em
device independent}, also \index{Geraeteunabhaengig@Ger\"ateunabh\"angig}
ger\"ateunabh\"angig. Fragen sie jemanden, der
sich damit auskennt, wie man dieses Ergebnis ansehen, besser noch
\index{Ausdruck}ausdrucken kann.}.

\subsection{Nun die Eingabe aus einem File}
Sie k\"onnen \TeX\ nat\"urlich {\em on line} benutzen, das ist aber nicht
sehr effektiv. Besser ist es ihre Eingabe in einen File zu schreiben,
sie k\"onnen dazu einen \index{Editor}
Editor ihrer Wahl benutzen. Diesen File werden
wir dann sp\"ater an \TeX\ \"ubergeben.

Schreiben sie als erstes den folgenden, aus dem \TeX Book hinl\"anglich
bekannten Text:
\index{zzzcenterline@{\tt\char92 centerline}}\index{zzzeject@{\tt\char92 eject}}
\index{zzzhrule@{\tt\char92 hrule}}\index{zzzvfill@{\tt\char92 vfill}}
\index{zzzvskip@{\tt\char92 vskip}}
\begin{verbatim}
\hrule
\vskip 1in
\centerline{\bf A SHORT STORY}
\vskip 8pt
\centerline{\sl by A. U. Thor}
\vskip .6cm
Once upon the time, in a distant
  galaxy called \\"O\\"o\c c,
there lived a computer
named R. J. Drofnats

Mr.~Drofnats---or ``R. J.,'' as
he preferred to be called---
was happiest when he was at work
typesetting beautiful documents.
\vskip 1in
\hrule
\vfill\eject
\end{verbatim}
Schreiben sie den Text genauso, wie sie ihn hier sehen, ich beziehe
mich im weiteren auf die Zeilen-num\-mern.

Sehen wir uns den Text doch noch etwas an, bevor wir ihn an \TeX\
\"ubergeben. In Zeile 1 und 17 steht jeweils der Befehl `\verb|\hrule|',
der eine d\"unne \index{Linie!Quer}
Querlinie mit der Breite einer \index{Seite!Querstrich}
Seite zieht. Dann gibt
es in den Zeilen 2, 4, 6 und 16 noch den Befehl \index{zzzvskip@{\tt\char92 vskip}}
`\verb|\vskip|', der
einen zus\"atzlichen vertikalen \index{Zwischenraum}
Zwischenraum einf\"ugt. Die
\index{Groesse!Zwischenraum@Gr\"o\ss{}e!Zwischenraum}Gr\"o\ss{}e des
Zwischenraums steht unmittelbar hinter dem Befehl, aber um die
\index{Einheit!Mass@Einheit!Ma\ss{}}
Ma\ss{}einheiten k\"ummern wir uns sp\"ater. Weitere Befehle kennen sie ja
schon, so z.B.\ den Befehl \index{zzzcenterline@{\tt\char92 centerline}}
`\verb|\centerline|'. Auch die Befehle
`\verb|\\"|' sollten ihnen schon bekannt vorkommen, es handelt sich um
Umlautbefehle, so da\ss{} der Ausdruck: `\verb|\\"O\\"o\c c|' als: `\\"O\\"o\c
c' erscheint.

Bei genauerem Hinsehen fallen allerdings noch ein paar Kleinigkeiten
auf: Zum einen taucht zwischen manchen W\"ortern, statt eines
\index{Leer!Zeichen}
Leerzeichens das Zeichen: `\verb|~|' auf. Dieses wird von \TeX\ wie
ein Leerzeichen behandelt, nur da\ss{} dort die Zeile nicht umgebrochen
werden darf. N\"utzlich z.B.\ bei \index{Initialen}
Initialen. Weiterhin ist die leere
Zeile 11 bemerkenswert. Mit einer derartigen Leerzeile wird \TeX\
mitgeteilt, da\ss{} ein Abschnitt zu Ende ist, und danach ein
\index{Neu!Abschnitt}neuer
anf\"angt. Die \"ubliche Neuformatierung, die sie vielleicht von
Textverarbeitungsprogrammen gew\"ohnt sind, entf\"allt.

Am Schlu\ss{} unseres Textes stehen die Befehle:
\index{zzzvfill@{\tt\char92 vfill}}`\verb|\vfill|' und
\index{zzzeject@{\tt\char92 eject}}
`\verb|\eject|'. Diese bewirken, da\ss{} die laufende
\index{Seite!Laufende}Seite mit
\index{Leer!Zeile}leeren
Zeilen aufgef\"ullt wird (\verb|\vfill|) und dann ausgegeben wird
(\verb|\eject|).

Jetzt wird es Zeit \TeX\ wieder zu starten, aber nachdem es uns mit
den beiden \index{Stern!Prompt}
Sternchen begr\"u\ss{}t, geben wir jetzt nicht wieder
\index{zzzrelax@{\tt\char92 relax}}
`\verb|\relax|' ein, sondern `\verb|story|'\footnote{Ich gehe davon
aus, da\ss{} sie ihren Text in einer Datei mit dem Namen {\tt story.tex}
abgespeichert haben.}.

Sie haben sich vielleicht schon gewundert, wieso \TeX\ bei der ersten
Meldung zwei, sp\"ater aber nur ein Sternchen ausgibt. Der Grund hierf\"ur
ist einfach. Die beiden Sternchen bedeuten, da\ss{} jede Eingabe, die
nicht mit einem \index{Gegenschraegstrich@Gegenschr\"agstrich}
Gegenschr\"agstrich oder einem \index{Kaufmannsund}Kaufmannsund (`\verb|&|')
beginnt, als Filename verstanden wird und \TeX\ deshalb ein
\index{zzzinput@{\tt\char92 input}} `\verb|\input|' vorher ausf\"uhrt (Sie
h\"atten also auch `\verb|\input story|' schreiben k\"onnen; `\verb|\input|'
beginnt mit einem \index{Gegenschraegstrich@Gegenschr\"agstrich}
Gegenschr\"agstrich).\footnote{Wenn dem doppelten
\index{Stern!Prompt!Doppelt}Sternchen ein\index{Kaufmannsund}
Kaufmannsund folgt, dann nimmt \TeX\ an, da\ss{} es sich um die Angabe
eines Formatfiles handelt. Wenn bei ihrem \TeX\ kein Formatfile
mitgeladen wird, dann m\"u\ss{}te ihre Eingabe lauten: {\tt\char38 plain
\char92 input story}.} Bei manchen Systemen ist es auch m\"oglich all
diese Angaben sofort beim Aufruf von \TeX\ in die Kommandozeile zu
schreiben. Der Aufruf s\"ahe dann ungef\"ahr so aus:\footnote{Bei manchen
Betriebssystemen ist es dar\"uber hinaus notwendig das Argument in
Anf\"uhrungszeichen zu schreiben.}
\begin{verbatim}
virtex &plain \input story
\end{verbatim}
Und \TeX\ w\"urde sofort mit seiner Arbeit beginnen.

W\"ahrend der Bearbeitung von \TeX\ taucht jetzt die \index{Ausgabe}Ausgabe:
\begin{verbatim}
(story.tex [1])
\end{verbatim}
auf, und nachdem sie beim abschlie\ss{}end wieder auftretenden Sternchen
wieder ein `\verb|\end|' eingegeben habe, k\"onnen sie den
\index{Neu!File}neuen File
`\verb|story.dvi|' wieder ausgeben lassen und sich das Ergebnis
ansehen.
\subsection{Fehler in der Eingabe}
Damit ihnen sp\"ater, wenn es dann mal zu {\em echten} Fehlern kommt,
jede Erfahrung im Umgang mit dieser Situation abgeht, jetzt mal einige
Fehlersituationen.\footnote{\index{Knuth. D.}
D. Knuth schreibt hierzu, da\ss{} man sich
immer vor Augen halten soll, da\ss{} man mit Fehlern nicht etwa den
Computer beleidigt, so da\ss{} er sp\"ater die Mitarbeit verweigert, Fehler
sind dazu da gemacht zu werden um aus ihnen zu lernen.}
\subsubsection{Die Sache mit der Textbreite}
Starten sie \TeX\ und auf die Eingabeaufforderung (\verb|**|) geben
sie ein:
\index{zzzhsize@{\tt\char92 hsize}}\index{zzzinput@{\tt\char92 input}}
\begin{verbatim}
\hsize=4in \input story
\end{verbatim}
Die \index{Kontrollsequenz}
Kontrollsequenz `\verb|\hsize|' regelt die bedruckbare Breite des
Textes, normalerweise wird sie in {\em Plain} vordefiniert, hier aber
auf die Breite von 4 Inch reduziert.\footnote{Ein Inch entspricht etwa
2.54 cm.} Wenn jetzt wieder das Sternchen auftaucht, geben sie ein:
\index{zzzhsize@{\tt\char92 hsize}}\index{zzzinput@{\tt\char92 input}}
\begin{verbatim}
\hsize=3in \input story
\end{verbatim}
und nachdem sie \TeX\ dar\"uber informiert hat, da\ss{} es die zweite
\index{Seite!Zweite}Seite
des Dokuments auch geschrieben hat (die erste Seite war ja 4 Inch
breit, und wird nicht \"uberschrieben, sondern jede weitere Seite wird
einfach angeh\"angt), geben sie der Reihe nach auch noch ein:
\index{zzzhsize@{\tt\char92 hsize}}\index{zzzinput@{\tt\char92 input}}
\begin{verbatim}
\hsize=2.5in \input story
\hsize=2in \input story
\end
\end{verbatim}

\"Ubergehen sie zun\"achst einmal alle Meldungen, die \"uber ihren
\index{Bildschirm}
Bildschirm huschen; f\"uhren sie einfach erst mal alles so aus, wie hier
beschrieben. Sie m\"ussen keine Angst haben, da\ss{} ihnen etwas entgeht.
Alle Fehlermeldungen, die sie auf dem Bildschirm sehen, werden
au\ss{}erdem noch in einem sogenannten \index{File!Transscript}
Transscriptfile, oder \index{File!Log}Logfile
gesichert, so da\ss{} sie sich dort sp\"ater in Ruhe ansehen
k\"onnen.\footnote{Tats\"achlich ist die Fehlermeldung im Logfile sogar
noch etwas ausf\"uhrlicher.}

Schauen wir uns doch mal eine der Fehlermeldungen an, die sie erhalten
haben d\"urften. Wenn sie nicht genau so aussehen, wie die hier
beschriebenen, dann macht das nichts, die Struktur der Fehlermeldung
ist die gleiche.\index{Box!Overfull}\index{zzzhbox@{\tt\char92 hbox}}
\begin{verbatim}
Overfull \hbox (0.98807pt too wide) in paragraph at lines 7--11
\tenrm tant galaxy called []O^^?o^^Xc, there lived|
Overfull \hbox (0.4325pt too wide) in paragraph at lines 7--11
\tenrm a com-puter named R. J. Drof-nats. ||
Overfull \hbox (5.32132pt too wide) in paragraph at lines 12--16
\tenrm he pre-ferred to be called---was hap-||
\end{verbatim}
Jede \"ubervolle Box liegt an einer bestimmten Stelle im urspr\"unglichen
File (beim erste Fall zwischen den Zeilen 7 und 11) und au\ss{}erdem l\"a\ss{}t
sich ablesen, wieviel die Box zu gro\ss{} ist (im ersten Fall 0.98807pt).
Die Fehlermeldung gibt auch noch einen Teil der Box aus, die zu lang
ist; hier Zeichen im Zeichensatz {\em tenrm}. Die beiden
\index{Klammer!Eckige}Klammern vor
dem `O' bedeuten, da\ss{} hier etwas kompliziertes, wie eine
\index{Absatz!Einrueckung@Absatz!Einr\"uckung}
Absatzeinr\"uckung, oder wie in diesem Fall die Konstruktion von
`\\"O\\"o\c c' vorliegt. Die Meldung gibt auch noch an, an welchen
Stellen \TeX\ trennen wollte, bevor der Fehler auftrat. Beim zweiten
Fehler z.B.\ wollte \TeX\ `Drof-nats' trennen. Der Trennalgorithmus von
\TeX\ ist ausserordentlich gut, aber nat\"urlich nicht perfekt, so wurde
z.B.\ nicht erkannt, da\ss{} `galaxy' h\"atte hinter dem `l' getrennt werden
k\"onnen.\footnote{Man kann \TeX\ etwas auf die Spr\"unge helfen, indem
man entweder an der entsprechenden Stelle `{\tt gal\char92 -axy}'
schreibt, oder, wenn man es \index{Global!Trennung}
global f\"ur den ganzen Text haben will, zu
Beginn des Textes: \index{zzzhyphenation@{\tt\char92 hyphenation}}
`{\tt\char92 hyphenation\char123 gal-axy\char125}'.}

Jetzt aber zum Grund f\"ur die Fehlermeldungen. \TeX\ l\"a\ss{}t zwischen
Worten variablen Wortabstand\index{Abstand!Worte}
 zu. Betrachten sie die folgenden
Zeilen:\medskip

\hbox spread-5.5pt{Hier mal mit ganz kleinen Abst\"anden}

\hbox spread+1.679895em{und hier mal mit ganz gro\ss{}en}\medskip

Es ist deutlich, da\ss{} die Wortabst\"ande unterschiedlich sind. \TeX\
berechnet nun f\"ur jede Zeile einen Wert, genannt \index{Badness}
`\verb|badness|'.
Dieser gibt an, wie gut oder schlecht die Wort in der Zeile verteilt
sind. Dabei geht \TeX\ von einem Idealabstand\index{Abstand!Ideal}
aus. Wird dieser unter-
oder \"uberschritten, dann erh\"oht sich demgem\"a\ss{} der Wert von
\verb|badness|. Normalerweise l\"a\ss{}t \TeX\ nur Zeilen mit einer badness
von weniger als 200 ohne Fehlermeldung zu. Dieser Wert (200) steht in
einer Variablen names `\verb|tolerance|'. Setzen sie diesen Wert doch
einfach einmal h\"oher. Starten sie \TeX\ erneut und geben sie
\index{zzzhsize@{\tt\char92 hsize}}\index{zzzinput@{\tt\char92 input}}
\index{zzztolerance@{\tt\char92 tolerance}}
\begin{verbatim}
\hsize=2in \tolerance=1600 \input story
\end{verbatim}
ein. Es wird ihnen keine `\verb|Overfull hbox|' mehr gemeldet. Ok, ok,
es gibt da noch ein paar Meldungen wegen `\verb|Underful hbox|'; das
liegt daran, da\ss{} \TeX\ alle \index{Box!Underfull}Boxen meldet, die eine badness \"uber einem
Wert haben, der in \index{zzzhbadness@{\tt\char92 hbadness}}
`\verb|\hbadness|' steht. Der vordefinierte Wert
f\"ur `\verb|\hbadness|' ist 1000.

Auch den letzten Fehler k\"onnen wir noch beseitigen, indem der Befehl
\index{zzzraggedright@{\tt\char92 raggedright}}
`\verb|\raggedright|' mit\-an\-ge\-ge\-ben wird. Er veranla\ss{}t
\TeX\ sich nicht mehr um den rechten \index{Rand!Ausgleich}
Randausgleich zu k\"ummern. Bis zum
rechten Rand wird mit \index{Leer!Zeichen}Leerzeichen aufgef\"ullt.

Abschlie\ss{}ende Bemerkungen zum Thema. Man kann sich z.B.\ fragen, ob es
besser ist, die {\tt tolerance} oder den Wert f\"ur {\tt hbadness}
hochzusetzen. Tats\"achlich haben beide M\"oglichkeiten ihre Vor- und
Nachteile. Wenn sie wirklich ihren Text noch korregieren wollen, dann
w\"ahlen sie lieber die erste Alternative, \TeX\ gibt ihnen dann noch
an, wieweit die Boxen das erlaubte Ma\ss{} \"uberschreiten, und sie k\"onnen
darauf reagieren. Geht es ihnen darum einfach ihren Text m\"oglichst
schnell und fehlerfrei zu setzen, dann w\"ahlen sie die zweite Methode.

Es gibt aber auch noch eine Variante: \TeX\ meckert nicht jede zu voll
geratene Box an. Wenn eine Box nur {\em etwas} zu gro\ss{} ist, erhalten
sie keine Fehlermeldung und auch \TeX\ ignoriert das herausragen der
Zeile. Der Wert, um den ein Text herausragen darf steht in
\index{zzzhfuzz@{\tt\char92 hfuzz}}
`\verb|\hfuzz|', und er ist \"ublicherweise mit 0.1pt vorbesetzt. Sie
k\"onnen ihn aber auch \"andern, wie alle Werte in \TeX.
\subsubsection{Und jetzt mal {\em richtige} Fehler}
Im letzten Teil dieses Abschnitts wollen wir mal ein paar {\em
richtige} Fehler begehen. \"Andern sie in ihrem File {\tt story.tex} die
Zeile 3 in:
\index{zzzcenterline@{\tt\char92 centerline}}\index{zzzERROR@{\tt\char92 ERROR}}
\begin{verbatim}
\centerline{\bf A SHORT \ERROR STORY}
\end{verbatim}
in der zweiten Zeile schreiben sie statt
\index{zzzvskip@{\tt\char92 vskip}}`\verb|\vskip|' einfach
\index{zzzvship@{\tt\char92 vship}}
`\verb|\vship|'. Um das Ma\ss{} voll zu machen, schreiben sie statt des
\index{File!Name}
Filenamens {\tt story} auch noch {\tt sorry} in die Eingabe von \TeX.

\TeX\ wird sie auffordern, einen andern Filenamen einzugeben, da er
den von ihnen angegebenen nicht finden kann. Versuchen sie nur sich durch
ein neues Dr\"ucken der Returntaste aus dieser Lage zu befreien,
letztlich ist es doch besser, sie schreiben den richtigen Filenamen.
\TeX\ wird dann mit der Arbeit beginnen und nach verh\"altnism\"a\ss{}ig
kurzer Zeit mit der Fehlermeldung:
\begin{verbatim}
! Undefined control sequence.
l.2 \vship
           1in
?
\end{verbatim}
aufwarten. Die Fehlermeldung beginnt mit einem `\verb|!|' und der
Angabe, worin der Fehler besteht. Es folgt die Angabe der
Zeilennummer, und dem Text, soweit \TeX\ ihn bisher bearbeitet hat.
Eine Zeile tiefer erscheint der Text, der als n\"achstes zu bearbeiten
w\"are. Abgeschlossen wird das Ganze durch ein \index{Fragezeichen}
Fragezeichen, was
anzeigt, da\ss{} \TeX\ eine Eingabe von ihnen erwartet, wie es weiter
gehen soll. Sie haben nat\"urlich keine Ahnung (und wenn sie schon
welche haben, dann haben sie vergessen) wie es weitergehen soll.
Tippen sie deshalb auch mal ein Fragezeichen, und \TeX\ wird ihnen
folgenderma\ss{}en antworten:
\begin{verbatim}
Type <return> to proceed, S to scroll future error messages,
R to run without stopping, Q to run quietly,
I to insert something, E to edit your file,
1 or ... or 9 to ignore the next 1 to 9 tokens of input,
H for help, X to quit.
\end{verbatim}
Es handelt sich dabei um eine Menuauswahl, wobei sie jetzt nur noch
entscheiden m\"ussen, wie sie reagieren wollen. Sie haben die folgenden
M\"oglichkeiten:
\begin{enumerate}
\item Dr\"ucken sie einfach die Returntaste. \TeX\ versucht dann soweit
es m\"oglich ist alleine aus den Schwierigkeiten heraus zu kommen.
\item Dr\"ucken sie die Taste `S'. Sie wechseln damit in einen anderen
\index{Modus!Fehler}
Modus. Ab sofort h\"alt \TeX\ bei Fehlern nicht mehr an, die
Fehlermeldungen sausen nur so \"uber den \index{Bildschirm}
Bildschirm (eventuell) und sie
k\"onnen sie nur noch sp\"ater im \index{File!Log}Logfile richtig lesen. \TeX\ verh\"alt
sich also so, als w\"urden sie bei diesem und jedem weiteren Fehler
immer wieder die Returntaste dr\"ucken.

Denselben Effekt k\"onnen sie auch erreichen, wenn sie
\index{zzzscrollmode@{\tt\char92 scrollmode}}
`\verb|\scrollmode|'in ihren File als Befehl hineinschreiben.
\item Dr\"ucken sie die Taste `R'. \TeX\ reagiert wie im letzten Fall,
allerdings noch etwas rabiater. \TeX\ h\"alt jetzt nicht einmal mehr an,
wenn ein falscher \index{File!Name}Filename eingegeben wurde.

Mit dem Befehl \index{zzznonstopmode@{\tt\char92 nonstopmode}}
`\verb|\nonstopmode|' im Text erreichen sie denselben Effekt.
\item Die Taste `Q'. Auch hier wieder eine \"ahnliche Reaktion, wie bei
`R', nur das dar\"uberhinaus auch noch alle Meldungen auf dem
\index{Bildschirm}Bildschirm
unterdr\"uckt werden.

Hierf\"ur gibt es im Text den Befehl \index{zzzbatchmode@{\tt\char92 batchmode}}
`\verb|\batchmode|'
\item Mit `I' kann etwas in den Text eingef\"ugt werden. Schreiben sie
dazu das, was sie einf\"ugen wollen unmittelbar hinter das `I'. \TeX\
tut so, als st\"unde dieser so eingef\"ugte Text vor dem, den es noch
lesen mu\ss{}. Das Return, das sie zur Beendigung der Eingabe eingeben, wird {\em
nicht} als \index{Leer!Zeichen}Leerzeichen interpretiert.
\item Wenn sie eine Nummer, die kleiner als 100 sein mu\ss{}, eingeben,
dann ignoriert \TeX\ die ent\-sprech\-en\-de Anzahl von Buchstaben, bzw.
\index{Kontrollsequenz}
Kontrollsequenzen in ihrem Text. Damit sind die gemeint, die dem
Fehler folgen!
\item Mit `H' erhalten sie eine weitergehende Hilfe. \TeX\
unterscheidet zwischen einer formalen und einer informellen
Fehlermeldung. Normalerweise erhalten sie nur die formale, in den
\index{File!Log}Logfile wird allerdings sofort auch die informelle geschrieben. Mit
`H' erhalten sie die volle Fehlermeldung.
\item Mit `X' verlassen sie \TeX\ an dieser Stelle. Dies bietet sich
an, wenn es einfach schon zu viele Fehler geworden sind, oder wenn es
zuviele Folgefehler gibt.
\item Auch `E' verl\"a\ss{}t \TeX, bereitet aber den Computer auf eine
Korrektur des \index{File!Eingabe}Eingabefiles vor. Dieses Feature wird nicht von allen
Rechnern und \TeX\ Versionen unterst\"utzt.
\end{enumerate}
Ach ja, Ich hatte oben von den Befehlen geschrieben, mit denen die
Modi auch im Text gesetzt werden k\"onnen. Zur\"uckgesetzt werden sie
Modi durch den Befehl: \index{zzzerrorstopmode@{\tt\char92 errorstopmode}}
`\verb|\errorstopmode|'.

Wie begegnen wir nun unserer ersten Fehlermeldung? Schreiben sie doch
einfach:
\index{zzzvskip@{\tt\char92 vskip}}
\begin{verbatim}
I\vskip
\end{verbatim}
Die unbekannte \index{Kontrollsequenz}
Kontrollsequenz \index{zzzvship@{\tt\char92 vship}}
`\verb|\vship|' wird ignoriert, und
statt dessen die gewollte eingef\"ugt (allerdings nicht in ihrer
Textdatei, da m\"ussen sie sie sp\"ater erst noch wieder \"andern). Sie
h\"atten auch `\verb|3|' eingeben k\"onnen, dann w\"are die Angabe
`\verb|1in|' auch ignoriert worden. H\"atten sie einfach Return
gedr\"uckt, st\"unde ein seltsames `1in' in ihrem Text.

Nun zum n\"achsten Fehler. \TeX\ beantwortet ihn folgenderma\ss{}en:
\index{zzzcenterline@{\tt\char92 centerline}}\index{zzzERROR@{\tt\char92 ERROR}}
\index{zzzhss@{\tt\char92 hss}}\index{zzzline@{\tt\char92 line}}
\begin{verbatim}
! Undefined control sequence.
<argument> \bf A SHORT \ERROR
                              STORY
\centerline #1->\line {\hss #1
                              \hss }
l.3 \centerline{\bf A SHORT \ERROR STORY}
|null
?
\end{verbatim}
Diese Fehlermeldung ist etwas komplizierter, da der Text:
\index{zzzERROR@{\tt\char92 ERROR}}
\begin{verbatim}
\bf A SHORT \ERROR STORY
\end{verbatim}
als \index{Argument}
Argument an den Befehl `\verb|\centerline|' \"ubergeben worden ist.
Der Fehler tritt auf, bei der Bearbeitung von `\verb|\centerline|'.
Dieser Befehl ist in {\em Plain} \TeX\ folgenderma\ss{}en definiert:
\index{zzzdef@{\tt\char92 def}}\index{zzzhss@{\tt\char92 hss}}
\index{zzzline@{\tt\char92 line}}
\begin{verbatim}
\def\centerline#1{\line{\hss#1\hss}}
\end{verbatim}
Das `\verb|#1|' steht dabei stellvertretend f\"ur das \"ubergebene
\index{Argument}
Argument.\footnote{Um die Bedeutung der Befehle: {\tt\char92 hss} und
{\tt\char92 line} brauchen wir uns jetzt noch nicht zu k\"ummern.} Ich
denke mit diesem Wissen braucht die Fehlermeldung nicht weiter erkl\"art
zu werden. Die Verschachtelung sogenannter
\index{Makro!Verschachtelung}Makros kann sehr tief sein.
Sie brauchen sich aber nicht alle Verschachtelungen in der
Fehlermeldung anzusehen, die ihnen meist sowieso nichts sagt, da sie
die \index{Definition}Makrodefinitionen nicht kennen. Mit dem Befehl:
\index{zzzerrorcontextlines@{\tt\char92 errorcontextlines}}
\begin{verbatim}
\errorcontextlines=...
\end{verbatim}
k\"onnen sie angeben, wieviele {\em Zwischen}zeilen sie angezeigt haben
m\"ochten. Ist der Wert z.B.\ gleich 2, dann werden nur die obersten
beiden Verschachtelungen angezeigt.
\chapter{Die internen Zeichen}
\section{Wie lie\ss{}t \TeX, was sie schreiben?}
Es d\"urfte ihnen schon klar geworden sein, da\ss{} ihre Texteingabe nur
ungef\"ahr etwas mit der \index{Ausgabe}
Ausgabe zu tun hat. Zeilenumbr\"uche etc.\ werden
von \TeX\ gesetzt. Au\ss{}erdem wird:
\begin{itemize}
\item Ein Return wie ein \index{Leer!Zeichen}Leerzeichen behandelt,
\item Zwei Leerzeichen wie ein Leerzeichen behandelt und
\item Eine leere Zeile beendet einen Absatz.
\end{itemize}\index{Absatz}
Wenn sie sich die drei \index{Punkt}
Punkte genau ansehen, wird ihnen ein Fehler
auffallen. Die Regeln wieder\-spre\-chen sich. Eine \index{Leer!Zeile}
Leerzeile besteht aus
zwei Returns hintereinander, die wie zwei Leerzeichen hintereinander
zu einem Leerzeichen zusammengefasst werden m\"u\ss{}ten. Vielleicht
interessieren sie sich irgendwann einmal f\"ur die {\em tats\"achlichen}
Regeln, Hier zun\"achst ein kleiner Einblick, was mit ihrer Eingabe
alles geschieht, bis sie zur \index{Ausgabe}Ausgabe geworden ist.

Die 256 verschiedenen Zeichen, die innerhalb \TeX\ m\"oglich sind
geh\"oren 16 verschiedenen \index{Kategorie!Zeichen}
Ka\-te\-gori\-en an. Diese Kategorien sind:\medskip

\index{Block!Anfang}\index{Block!Ende}\index{Hochstellung}
\index{Satz!Zeichen}\index{Kommentar}\index{Spalte!Tabelle}\index{Ziffer}
\begin{tabular}{|l|l|l|} \hline
{\em Kategorie} & {\em Bedeutung} & {\em Standard} \\ \hline\hline
0 & Befehlsanfang & {\tt\char92} \\ \hline
1 & Blockanfang & {\tt\char123} \\ \hline
2 & Blockende & {\tt\char125} \\ \hline
3 & Math. Umschaltbefehl & {\tt\char36} \\ \hline
4 & Tabellenspalte & {\tt\char38} \\ \hline
5 & Zeilenende & Return (ASCII CR) \\ \hline
6 & Makroargument & {\tt\char35} \\ \hline
7 & Hochstellung & {\tt\char94} \\ \hline
8 & Tiefstellung & {\tt\char95} \\ \hline
9 & Ignorieren & ASCII NULL \\ \hline
10 & Leerzeichen & {\tt\char32} \\ \hline
11 & Buchstabe & {\tt A..Z} und {\tt a..z} \\ \hline
12 & Sonstige Zeichen & Ziffern, Satzzeichen, etc. \\ \hline
13 & Aktives Zeichen & {\tt\char126} Als Makroname verwendbar \\ \hline
14 & Kommentarzeichen & {\tt\char37} Kommentaranfang \\ \hline
15 & Ung\"ultiges Zeichen & ASCII DEL \\ \hline
\end{tabular}\medskip\index{Aktiv!Zeichen}\index{Argument}\index{ASCII}

In diesem Text sind schon mehrere Male Zeichen aufgetaucht, die nicht
als normale Zeichen von \TeX\ verstanden werden. Nun haben wir alle
diese Zeichen auf einem Blick (und es sind dann auch wirklich alle).
Aus {\em Plain} \TeX\ heraus sind etliche dieser Zeichen, die man auch gerne
drucken w\"urde, nicht so einfach druckbar, da sie eine andere Bedeutung
haben. Es sind dies die Zeichen:
\begin{verbatim}
\ { } $ & # ^ _ % ~
\end{verbatim}
Einige davon lassen sich dennoch im Text ausgeben, nat\"urlich durch
\index{Kontrollsequenz}Kontrollsequenzen:
\begin{quote}
`\verb|\&|' f\"ur \&, `\verb|\%|' f\"ur \%, `\verb|\&|' f\"ur \&, `\verb|\#|'
f\"ur \# und `\verb|\_|' f\"ur \_.
\end{quote}
Innerhalb des mathematischen \index{Modus!Mathematisch}
Modus k\"onnen au\ss{}erdem die
\index{Kontrollsequenz}
Kontrollsequenzen `\verb|\{|' und `\verb|\}|' f\"ur $\{$ und $\}$
benutzt werden. Und schlie\ss{}lich gibt es nicht die \index{Akzent}
Akzentbefehle:
`\verb|\~|' und `\verb|\^|' z.B.\ f\"ur die Akzente \~e und \^a.

Jede Eingabezeile wird von \TeX\ in sogenannte Token zerlegt. Diese
Token bestehen aus Zeichen, Kontrollsequenzen, \index{Ziffer}
Ziffern etc.\ So wird
die Eingabe von: \index{zzzhskip@{\tt\char92 hskip}}
`\verb|{\hskip 36 pt}|' zerlegt in:
\[\{_1\quad\mbox{\fbox{hskip}}\quad 3_{12}\quad
6_{12}\quad\mbox{\tt\char32}_10\quad p_{11}\quad t_{11}\quad \}_2\]
Die Indizes deuten dabei die \index{Kategorie!Zeichen}
Kategorie an, der das jeweilige Token
angeh\"ort. \fbox{hskip} hat keinen Index, da eine
\index{Kontrollsequenz}Kontrollsequenz
keiner Kategorie angeh\"ort. Auch das \index{Leer!Zeichen}
Leerzeichen hinter der
Kontrollsequenz wird nicht behandelt, da es nur die Kontrollsequenz
abschlie\ss{}t. Nach der Zerlegung in Token, erscheint \TeX\ ihre Eingabe,
wie eine unheimlich lange Zeile von Token, es gibt also gar keine
Abs\"atze und sonstiges mehr, nur noch Token.

\index{Kontrollsequenz}
Kontrollsequenzen sind also nur {\em ein} Token lang, so da\ss{} es
\"uberhaupt nichts ausmacht, wie lang der Name der Kontrollsequenz ist.
Die Kategorie, die einem Token zukommt, kann ge\"andert werden. Wenn sie
z.B.\ INITEX starten, das Programm, mit dem die Formatdateien erstellt
werden k\"onnen etc., dann gibt es f\"ur manche Kategorien noch gar keine
Zeichen. Die Kategorie k\"onnen sie einem Zeichen zuweisen, resp.\ die
Kategorie eines Zeichens andern, mit dem Befehl:
\index{zzzcatcode@{\tt\char92 catcode}}
\begin{verbatim}
\catcode`\
\end{verbatim}
Hinter diesem Befehl m\"ussen sie dann das Zeichen tippen, dessen
\index{Kategorie!Aendern@Kategorie!\"Andern}
Kategorie sie bestimmen oder \"andern wollen.
\begin{verbatim}
\catcode`\a=14
\end{verbatim}
wuerde z.B.\ daf\"ur sorgen, da\ss{} \index{Kommentar!Einleitung}
Kommentare mit einem `a' eingeleitet
werden (nicht sehr sinnvoll).

\index{Kontrollsequenz}
Kontrollsequenzen werden ja als {\em ein} Token behandelt. Trotzdem
bietet \TeX\ eine M\"oglichkeit eine Kontrollsequenz in ihre Zeichen zu
zerlegen. Der entsprechende Befehl hei\ss{}t \index{zzzstring@{\tt\char92 string}}
`\verb|\string|'. Mit der
Eingabe von:
\begin{verbatim}
\string\TeX
\end{verbatim}
erhalten sie:
\[ \mbox{\tt\char92}_{12}\quad \mbox{\tt T}_{12}\quad \mbox{\tt
e}_{12}\quad \mbox{\tt X}_{12}\]

Umgekehrt kann nat\"urlich auch aus beliebigen Token eine
\index{Kontrollsequenz}
Kontrollsequenz aufgebaut werden. Der zugeh\"orige Befehl lautet:
\index{zzzcsname@{\tt\char92 csname}}\index{zzzendcsname@{\tt\char92 endcsname}}
\begin{verbatim}
\csname TeX\endcsname
\end{verbatim}
Hier werden die Token:
\[ \mbox{\tt T}_{12}\quad \mbox{\tt e}_{12}\quad \mbox{\tt X}_{12}\]
zu der \index{Kontrollsequenz}Kontrollsequenz: `\verb|\TeX|' zusammengef\"ugt.

Leider gibt es eine Einschr\"ankung. Da zwischen
`\verb|\csname|' und `\verb|\endcsname|' zwar beliebige Token, also
auch wieder Kontrollsequenzen, stehen d\"urfen, aber diese m\"ussen sich
alle zu Zeichen, und nicht zu \index{Primitiv}
Primitiven aufl\"osen lassen. Somit ist
die Eingabe von:
\index{zzzcsname@{\tt\char92 csname}}\index{zzzendcsname@{\tt\char92 endcsname}}
\begin{verbatim}
\csname\TeX\endcsname
\end{verbatim}
falsch, da `\verb|\TeX|' auch auf das \index{Primitiv}
Primitiv \index{zzzkern@{\tt\char92 kern}}`\verb|\kern|'
zur\"uckgreift.

Wenn wir schon bei den F\"ahigkeiten von \TeX\ sind etwas umzuwandeln.
Nat\"urlich ist auch das \index{Escape}
Escapezeichen `\verb|\|' nichts gottgegebenes
in \TeX. Tats\"achlich steht in der Variablen
\index{zzzescapechar@{\tt\char92 escapechar}}`\verb|\escapechar|' der
Wert des Zeichens, das dann das Escapezeichen ausmacht.

Und noch mehr F\"ahigkeiten zur Umwandlung:
Es gibt noch die beiden Befehle: \index{zzznumber@{\tt\char92 number}}
`\verb|\number|' und\index{zzzromannumber@{\tt\char92 romannumber}}
`\verb|\romannumber|', welche beide eine Zahl als \index{Ziffer}Ziffernfolge
ausgeben.
\begin{verbatim}
\romannumber24
\end{verbatim}
produziert z.B.\ `\verb|xxiv|'. `\verb|\number|' gibt den Wert einfach
als Zahl aus. Hier ist es nat\"urlich nicht sinnvoll eine Zahl
anzugeben, aber `\verb|\number|' arbeitet auch mit den
\index{Register}Registern von
\TeX, deren Inhalt man sich mit `\verb|\number|' ausgeben lassen kann.
\section{Die Sache mit den Zeichen}
Etwas weiter vorne konnten sie lesen, da\ss{} \TeX\ intern 256 Zeichen
verarbeiten kann. Diese gro\ss{}e Menge von Zeichen kann allerdings auf
den meisten Tastaturen nicht direkt eingegeben werden. Es stellt sich
die Frage, wie man denn am besten vorgeht.
Um z.B.\ das 37te Zeichen eines Zeichensatzes ansprechen zu k\"onnen,
geben sie einfach:
\begin{verbatim}
\char37
\end{verbatim}
ein, also `\verb|\char|' gefolgt von der gew\"unschten Nummer. Auf diese
Art lassen sich alle Zeichen ansprechen. So ergibt z.B.:
\begin{verbatim}
\char70 ritz
\end{verbatim}
meinen Vornamen: `\char70ritz'. Sie vermuten ganz richtig, da\ss{} das `F' das
70. Zeichen im aktuell g\"ultigen Zeichensatz ist. Die Nummerierung der
Zeichen h\"angt vom sogenannten \index{ASCII}
`ASCII' Zeichensatz ab, nach dem sich
\TeX\ orientiert. Diese Orientierung findet \"ubrigens {\em innerhalb}
von \TeX\ statt, unabh\"angig davon, welchen Zeichensatz ihr Rechner
benutzt.

Innerhalb der Tabelle von Zeichen, die zur Verf\"ugung stehen, werden
die Zeichen oft mit ihrem \index{Oktal}
oktalen, oder \index{Hexadezimal}hexadezimalen Wert
angesprochen. In \TeX\ k\"onnen diese Werte auch angegeben werden, wenn
sie die \index{Dezimal!Wert}
dezimalen nicht kennen oder ausrechnen wollen. Vor den \index{Oktal}
oktalen
Zahlen mu\ss{} ein (rechtes) H\"akchen `\verb|'|' stehen, vor den
\index{Hexadezimal}hexadezimalen ein \index{Anfuehrungszeichen@Anf\"uhrungszeichen}
Anf\"uhrungszeichen `\verb|\"|'. Die folgenden Angaben
sind also \"aquivalent:
\begin{verbatim}
\char98  \char'142  \char\"62
\end{verbatim}
Tats\"achlich brauchen sie nicht einmal die Nummer des Zeichens zu
kennen, dessen Wert sie angeben wollen. Das Token $`_{12}$ (linkes
H\"akchen) liefert den Wert des unmittelbar folgenden Buchstaben oder
der \index{Kontrollsequenz}
Kontrollsequenz, die nur aus einem Buchstaben besteht (dann
nat\"urlich den Wert dieses Buchstabens). Man kann also angeben:
`\verb|\char`b|' oder `\verb|\char`\b|'.\footnote{Auf diese Art wird
\"ubrigens das \index{Kommentar!Zeichen}
Kommentarzeichen ausgegeben. Man definiert die zugeh\"orige
\index{Kontrollsequenz}
Kontrollsequenz als: {\tt\char92def\char92\char37\char123\char92
char\char96\char92\char37\char125}. Etwas eleganter geht es aber auch
mit dem Befehl: {\tt\char92chardef}.}

Den `\verb|\char|' Befehl d\"urfen sie \"uberall benutzen um ein Zeichen
anzusprechen, allerdings nicht innerhalb einer Kontrollsequenz. Wozu
ist dieser Befehl denn nun eigentlich n\"utzlich? Nun, es gibt Zeichen,
wie z.B.\ das Zeichen `\char0', welches im normalen Text nicht
geschrieben werden kann. Da es aber das 0. Zeichen des Zeichensatzes
ist, kann es einfach als `\verb|\char0|' angesprochen werden.

Es gibt noch zwei weitere interessante M\"oglichkeiten ein Zeichen aus
einem Zeichensatz an\-zu\-spre\-chen und zwar durch:
\begin{verbatim}
^^
\end{verbatim}
Wenn sie hinter diese beiden Zeichen ein weiteres Zeichen schreiben,
dann reagiert \TeX\ folgenderma\ss{}en: Hat das folgende Zeichen einen
inneren Wert zwischen 64 und 127, dann werden 64 subtrahiert, hat es
einen Wert zwischen 0 und 63, dann werden 64 addiert. Um das obige
Zeichen darzustellen h\"atte ich also auch:
\begin{verbatim}
^^@
\end{verbatim}
schreiben k\"onnen, da `@' den inneren Wert 64 hat. `\verb|^^@|' wird
also als das 0. Zeichen gelesen.

Folgt den beiden `\verb|^^|' eine {\em kleingeschriebene}
\index{Hexadezimal}hexadezimale
Zahl, dann wird das entsprechende Zeichen, dem diese hexadezimale Zahl
entspricht ausgegeben. `\verb|^^7f|' ergibt also das 127. Zeichen.

Die Realisation der Zeichenordnung innerhalb von \TeX\ erlaubt eine
ungeheure Sprachunabh\"angigkeit. So ist es z.B.\ m\"oglich auch noch mit
Tastaturen zu arbeiten, die Zeichen, wie `\char26' bereitstellen, doch
hier verweise ich auf das \TeX Book, das diese M\"oglichkeit viel
genauer beschreibt.
\chapter{\TeX's Ausgabe}
Dieses Kapitel besch\"aftigt sich mit dem, was \TeX\ in die {\tt dvi}
\index{File!DVI}Files ausgibt. Dazu geh\"oren, neben den Zeichen, auch die
\index{Einheit!Mass@Einheit!Ma\ss{}}Ma\ss{}einheiten,
um Positionierungen durchf\"uhren zu k\"onnen.
\section{\TeX's Roman Zeichens\"atze}
Ich habe schon fr\"uher die m\"oglichen Zeichens\"atze angesprochen, die
\TeX\ ausgeben kann. Hier nun ein etwas systematischerer Zugang.
Zun\"achst einmal die Zeichen des Roman Zeichensatzes. Da sind zun\"achst
folgende Zeichen m\"oglich:
\begin{quote}\index{Satz!Zeichen}
Die Buchstaben: {\tt A} bis {\tt Z} und {\tt a} bis {\tt z}\\
die \index{Ziffer}Ziffern {\tt 0} bis {\tt 9}\\
die Satzzeichen: {\tt : ; ! ? ( ) [ ] ` ' - * / . , @}
\end{quote}
Au\ss{}erdem gibt es die \index{Ligatur}Ligaturen:\medskip

\begin{tabular}{*{4}{lcl@{$\qquad$}}}
{\tt ff} & $\rightarrow$ & ff &
{\tt ffi} & $\rightarrow$ & ffi &
{\tt ``} & $\rightarrow$ & `` &
$<$ & $\rightarrow$ & < \\
{\tt fi} & $\rightarrow$ & fi &
{\tt ffl} & $\rightarrow$ & ffl &
{\tt ''} & $\rightarrow$ & '' &
$>$ & $\rightarrow$ & > \\
{\tt fl} & $\rightarrow$ & fl &
{\tt --} & $\rightarrow$ & -- &
{\tt ---} & $\rightarrow$ & --- & & &
\end{tabular}\medskip

Auch die Zeichen `+' und `=' sind m\"oglich, sollten aber besser im
mathematischen \index{Modus!Mathematisch}
Modus verwendet werden, der durch ein
\index{Dollarzeichen}Dollarzeichen
`\verb|$|' ein- bzw.\ ausgeleitet wird. Sie sehen dann besser aus. Das
gleiche gilt f\"ur `-' und `/'.

Weiterhin gibt es die reservierten Zeichen:
\begin{verbatim}
 $  #  %  &
\end{verbatim}
die aber durch die \index{Kontrollzeichen}Kontrollzeichen:
\begin{verbatim}
\$  \#  \%  \&
\end{verbatim}
erhalten werden k\"onnen. Au\ss{}erdem reserviert {\em Plain} \TeX\ noch die
Zeichen:
\begin{verbatim}
\  {  }  ^  _  ~
\end{verbatim}
Schlie\ss{}lich bleiben im sichtbaren \index{ASCII}
ASCII-Code noch vier Zeichen \"ubrig,
die sie aber in normalem Text nicht verwenden sollten, resp.\ durch
andere Zeichen(kombinationen) ersetzen sollten:
\begin{verbatim}
\"  |  <  >
\end{verbatim}
\TeX\ bietet aber auch einiges, was sie in der normalen
\index{ASCII}
ASCII-Codierung nicht finden. An erster Stelle die \index{Akzent}
Akzente:\medskip

\begin{tabular}{ll}
{\em Eingabe} & {\em Ausgabe} \\
{\tt\char92\char96 o} & \`o \\
{\tt\char92\char39 o} & \'o \\
{\tt\char92\char94 o} & \^o \\
{\tt\char92\char34 o} & \\"o \\
{\tt\char92\char126 o} & \~o \\
{\tt\char92\char61 o} & \=o \\
{\tt\char92\char46 o} & \.o \\
{\tt\char92 u o} & \u o \\
{\tt\char92 v o} & \v o \\
{\tt\char92 H o} & \H o \\
{\tt\char92 t oo} & \t oo
\end{tabular}\medskip

Die Akzente lassen sich nat\"urlich auch auf andere Buchstaben, als das
`o' setzen. Achten sie allerdings darauf, da\ss{} bei den letzten vier
Beispielen ein \index{Zwischenraum}
Zwischenraum zwischen dem \index{Kontrollzeichen}
Kontrollzeichen und dem
\index{Argument}
Argument steht. Alternativ k\"onnen sie auch `\verb|\H{o}|' schreiben.

Weiterhin kennt \TeX\ auch Akzente {\em unter} den Buchstaben:\medskip

\begin{tabular}{ll}
{\em Eingabe} & {\em Ausgabe} \\
{\tt\char92 c o} & \c o \\
{\tt\char92 d o} & \d o \\
{\tt\char92 b o} & \b o
\end{tabular}\medskip

Dar\"uberhinaus kennt \TeX\ auch noch einige \index{Spezialzeichen}
Spezialzeichen europ\"aischer
\index{Schrift!Europaeisch@Schrift!Europ\"aisch}Schriften:\medskip

\index{Kreis}\index{Lateinisch}\index{Polnisches L}
\begin{tabular}{lll}
{\em Eingabe} & {\em Ausgabe} & {\em Bezeichnung} \\
{\tt\char92 oe, \char92 OE} & \oe, \OE & Franz\"osische \index{Ligatur}Ligatur \\
{\tt\char92 ae, \char92 AE} & \ae, \AE & Lateinische, skandinavische
\index{Ligatur}Ligatur \\
{\tt\char92 aa, \char92 AA} & \aa, \AA & Skandinavisches A mit Kreis \\
{\tt\char92 o, \char92 O} & \o, \O & Skandinavisches O mit Strich \\
{\tt\char92 l, \char92 L} & \l, \L & Polnisches L mit Strich \\
{\tt\char92 ss} & \ss & Deutsches Es-Zet
\end{tabular}\medskip

Die Roman \index{Schrift!Roman}
Schrift enth\"alt au\ss{}erdem noch die Buchstaben `i' und `j'
ohne den \index{Punkt}
Punkt. So ergeben `\verb|\i|': \i\ und `\verb|\j|': \j. Der
Grund hierf\"ur ist, da\ss{} der Punkt bei diesen Zeichen entfernt werden
mu\ss{}, bevor ein anderer \index{Akzent}
Akzent auf die Zeichen gesetzt werden kann.

Die Schriften: `\verb|\sl|', `\verb|\it|' und `\verb|\bf|' haben die
gleichen Zeichen. Nur der Zeichensatz `\verb|\tt|' verh\"alt sich etwas
anders. Zum einen kennt dieser Zeichensatz keine \index{Ligatur}
Ligaturen. Selbst
{\tt ---} wird als {\tt ---} ausgegeben. Die \index{Akzent}Akzente:
\index{zzzldots@{\tt\char92 ldots}}
`\verb|\H|'$\ldots$`\verb|\L|' k\"onnen nicht benutzt werden, daf\"ur aber
die Zeichen: {\tt \", |, <, >}.

Schlie\ss{}lich gibt es noch vier Zeichen, die in allen Zeichens\"atzen
gleich aussehen:\medskip

\index{zzzdag@{\tt\char92 dag}}\index{zzzddag@{\tt\char92 ddag}}
\begin{tabular}{ll}
{\em Eingabe} & {\em Ausgabe} \\
{\tt\char92 dag} & \dag \\
{\tt\char92 ddag} & \ddag \\
{\tt\char92 S} & \S \\
{\tt\char92 P} & \P
\end{tabular}\medskip

Im Anhang wird noch genauer dargestellt, da\ss{} alle \index{Akzent}
Akzente von \TeX\
auf das \index{Primitiv}
Primitiv: \index{zzzaccent@{\tt\char92 accent}}
`\verb|\accent|' zugreifen. So ist z.B.\ `\verb|\'#1|'
\"aquivalent zu: `\verb|{\accent19 #1}|'. Generell wird der
\index{Akzent}Akzent mit
einer Nummer angesprochen. Die Akzente werden immer richtig \"uber die
nachfolgenden Zeichen gesetzt. So w\"urde ein Akzent \"uber einem `o'
niedriger gesetzt, als \"uber einem `f', aber auch Akzente \"uber gar
keinem Zeichen sind m\"oglich. Ahnen sie wie? `\verb|\'{}|' ergibt:
\'{}.
\section{Die Ma\ss{}einheiten}
\subsection{Die eigentlichen Ma\ss{}einheiten}
Im englischsprachigen Raum haben sich die {\em klassischen}
Ma\ss{}einheiten \index{Point}
`Point' und \index{Pica}`Pica' f\"ur Drucker durchgesetzt. Neben diesen
kennt \TeX\ aber auch noch einige andere, um die Benutzung von \TeX\
komfortabeler zu gestalten.\medskip

\index{Cicero}\index{Didot}\index{Pica}\index{Zentimeter}
\begin{tabular}{ll}
{\em Abk\"urzung} & {\em Bedeutung} \\
{\tt pt} & Point \\
{\tt pc} & Pica (1pc=12pt) \\
{\tt in} & Inch (1in=72.27pt) \\
{\tt bp} & Big Point (72bp=1in) \\
{\tt cm} & Zentimeter (2.54cm=1in) \\
{\tt mm} & Millimeter (10mm=1cm) \\
{\tt dd} & Didot Punkt (1157dd=1238pt) \\
{\tt cc} & Cicero (1cc = 12dd) \\
{\tt sp} & Skalierter Punkt (65536sp=1pt)
\end{tabular}\medskip\index{Abkuerzung@Abk\"urzung}

Die Ma\ss{}einheiten werden immer nach folgenden Regeln angegeben:
\begin{quote}
$<$optionales Vorzeichen$><$Nummer$><$Ma\ss{}einheit$>$, oder\\
$<$optionales Vorzeichen$><$Ziffern$>.<$Ziffern$><$Ma\ss{}einheit$>$
\end{quote}
Dabei k\"onnen zwischen der Zahl und der Ma\ss{}einheit \index{Leer!Zeichen}
Leerzeichen
eingef\"ugt werden. Leerzeichen zwischen den \index{Ziffer}
Ziffern der Zahl, oder den
Buchstaben der Ma\ss{}einheit sind dagegen nicht erlaubt.

Intern arbeitet \TeX\ mit dem {\tt sp}. Diese Ma\ss{}einheit ist so
klein,\footnote{Die Wellenl\"ange des sichtbaren Lichtes inst ungef\"ahr
100sp} da\ss{} auch trotz der Verwendung von \index{Integer!Arithmetik}
Integerarithmetik keine
erkennbaren \index{Fehler!Integerarithmetik}
Fehler auftreten. Allerdings begrenzt sich dadurch die
maximal angebbare \index{Groesse@Gr\"o\ss{}e}Gr\"o\ss{}e in \TeX\ auf $2^{30}$sp, was aber auch nicht
weiter st\"ort, da diese Gr\"o\ss{}e einer Strecke von $5.7583$ Metern
entspricht.

Im weiteren werden des \"ofteren \index{Dimension!Angabe}
Dimensionsangaben gebraucht. Statt
dort immer eine explizite Angabe zu verwenden, wird die
Abk\"urzung:\index{Abkuerzung@Abk\"urzung}
$<$dimen$>$ verwendet, z.B.:
\begin{quote}
\verb|\hsize=|$<$dimen$>$
\end{quote}
\subsection{Ma\ss{}einheiten und Vergr\"o\ss{}erung}
Zu Beginn eines Textes k\"onnen sie eine
\index{Global!Vergroesserung@Global!Vergr\"o\ss{}erung}globale Vergr\"o\ss{}erung oder
Verkleinerung angeben. Dies geschieht mit dem Befehl:
\index{zzzmagnifikation@{\tt\char92 magnifikation}}
\begin{verbatim}
\magnifikation=1200
\end{verbatim}
Die Zahl am Ende des Befehls variiert nat\"urlich von Fall zu Fall. Bei
der Zahlangabe handelt es sich um eine \index{Promille}
Promilleangabe, so da\ss{} eine
Angabe von 1000 zu keiner Ver\"anderung f\"uhrt. Der Befehl:
\begin{verbatim}
\magnifikation=2000
\end{verbatim}
\index{Doppelt!Vergroesserung@Doppelt!Vergr\"o\ss{}erung}
verdoppelt die Gr\"o\ss{}e ihrer \index{Ausgabe}
Ausgabe. Vergewissern sie sich allerdings {\em
vorher}, ob die entsprechenden Zeichens\"atze auf ihrem System verf\"ugbar
sind.

Mit der Vergr\"o\ss{}erung oder Verkleinerung des Textes werden auch alle
Ma\ss{}angaben ge\"andert, mit Ausnahme des {\tt sp}. Wenn sie in ihrem
urspr\"unglichen Text eine L\"ucke von 2cm gelassen hatten, dann erhalten
sie jetzt eine 4cm gro\ss{}e L\"ucke (entsprechend dem letzten Beispiel).
Wenn sie das verhindern wollen, dann m\"ussen sie {\em vor} der
Ma\ss{}einheit das Wort `\verb|true|' einf\"ugen. Haben sie obige L\"ucke mit:
\index{zzzvskip@{\tt\char92 vskip}}
\begin{verbatim}
\vskip 2 true cm
\end{verbatim}
definiert, dann ist sie auch in der Vergr\"o\ss{}erung nur 2cm gro\ss{}. Der
Grund f\"ur dieses Verhalten liegt darin, da\ss{} alle Ma\ss{}einheiten mit
einer Grundgr\"o\ss{}e in \index{zzzmag@{\tt\char92 mag}}
`\verb|\mag|' verrechnet werden. Diese Berechnung
verhindern sie mit der Angabe von `\verb|true|'.
\subsection{Relative Ma\ss{}einheiten}
Jeder Zeichensatz verf\"ugt \"uber zwei relative Ma\ss{}einheiten:
\begin{quote}
{\tt em} ist die Breite eines \index{zzzquad@{\tt\char92 quad}}
`\verb|\quad|' der entsprechenden
Schrift\\
{\tt ex} ist die \index{Hoehe!x@H\"ohe!x}
H\"ohe des `x' in der entsprechenden Schrift.
\end{quote}
Die tats\"achlichen Ausma\ss{}e dieser Ma\ss{}einheiten ist von Zeichensatz zu
Zeichensatz verschieden.
\subsection{Abschlie\ss{}ende Bemerkungen}
Ihnen ist vielleicht aufgefallen, da\ss{} die Ma\ss{}einheiten nicht mit einem
\index{Gegenschraegstrich@Gegenschr\"agstrich}
Gegenschr\"agstrich eingeleitet werden. Das liegt daran, da\ss{} \TeX\
einige \index{Schluesselwort@Schl\"usselwort}
Schl\"usselworte kennt. Die Ma\ss{}einheiten geh\"oren auch dazu. Hier
eine vollst\"andige \index{Liste!Schluesselworte@Liste!Schl\"usselworte}
Liste der Schl\"usselworte von \TeX:
{\tt at}, {\tt bp}, {\tt by}, {\tt cc}, {\tt cm}, {\tt dd}, {\tt
depth}, {\tt em}, {\tt ex}, {\tt fil}, {\tt height}, {\tt in}, {\tt l},
{\tt minus}, {\tt mm}, {\tt mu}, {\tt pc}, {\tt plus},
{\tt pt}, {\tt scaled}, {\tt sp}, {\tt spread}, {\tt to}, {\tt true},
{\tt width}.
\chapter{Interna des Textsatzes}
In diesem Kapitel werden die wichtigsten Elemente des Textsatzes von
\TeX\ dargestellt. F\"ur die normale Anwendung ist das Wissen um diese
Zusammenh\"ange nicht unbedingt n\"otig, wenn sie also nur {\em normale}
Texte mit \TeX\ schreiben wollen, k\"onnen sie dieses Kapitel
\"uberspringen.
\section{Boxen}
Auch, wenn sich \TeX\ des modernen Hilfsmittel {\em Computer} bedient,
beruht die Methode, mit der Texte gesetzt werden letztlich auf der
guten alten \index{Satz!Kunst}
Satzkunst, die sp\"atestens mit Guttenberg ihren Einzug in
die sch\"onen K\"unste gefunden hat. Fr\"uher war es n\"otig einzelne Zeichen,
die als erhabene Grate auf Bleikl\"otzchen vorgegeben waren, auf einer
Schiene zusammenzusetzen, und diese Schienen dann ihrerseits wieder zu
gr\"o\ss{}eren Komponenten zusammenzuf\"ugen. An dieser Idee hat \TeX\ nichts
ge\"andert, nur da\ss{} statt der Bleikl\"otzchen nun sogenannte `Boxen'
verwendet werden.\footnote{Nat\"urlich nur intern. Der {\em normale}
Anwender bekommt davon nichts mit.} Es gibt zwei Arten von Boxen: die
einen k\"onnen Zeichen enthalten, die anderen nicht, sind insofern also
noch einfacher.
\subsection{Die {\em Rule}boxen}
Die Ruleboxen\footnote{Der Begriff kommt von der \"ublichen Verwendung
dieser Boxen. Mit ihnen werden normalerweise \index{Linie}
Linien gezogen.} sind
eigentlich nur gef\"ullte Rechtecke. Je nachdem, ob sie breiter als
hoch, oder umgekehrt sind, werden sie mit den Befehlen:
\index{zzzhrule@{\tt\char92 hrule}}`\verb|\hrule|'
oder \index{zzzvrule@{\tt\char92 vrule}}
`\verb|\vrule|' angesprochen. \index{Quadrat}
Quadratische Boxen dieser Art k\"onnen
mit beiden Befehlen erzeugt werden. Der generelle Aufruf der Befehle
lautet:
\index{zzzxrule@{\tt\char92 xrule}}
\begin{verbatim}
\xrule width <dimen> height <dimen> depth <dimen>
\end{verbatim}
Zu den \index{Groesse!Einer Box@Gr\"o\ss{}e!Einer Box}
Gr\"o\ss{}en `\verb|width|', `\verb|height|' und `\verb|depth|' komme
ich im n\"achsten Abschnitt (Das `x' mu\ss{} nat\"urlich gegen `v' oder `h'
ausgetauscht werden). Zu diesem Boxtyp abschlie\ss{}end ein Beispiel. Die
\index{Eingabe}Eingabe von:
\index{zzzhrule@{\tt\char92 hrule}}
\begin{verbatim}
\hrule width 10pt height 2pt
\end{verbatim}
ergibt: \vrule width 10pt height 2pt
\subsection{Die Zeichenboxen}
Jedes Zeichen, mit dem \TeX\ arbeitet, stellt sich f\"ur das Programm
als eine Box dar, deren Abmessungen vom Zeichensatzdesigner festgelegt
wurden.\footnote{Genauste Darstellung im \index{METAFONT}
{\manual METAFONT}Book.} Bei
den Abmessungen werden folgende Werte unterschieden: Breite (width),
\index{Hoehe!Box@H\"ohe!Box}
H\"ohe (height) und Tiefe (depth). Eine Box hat somit folgendes
Aussehen:

\begin{picture}(10,6)
\put(1,1.5){\makebox(0,0){Referenzpunkt}}	% Text
\put(2.3,1.5){\vector(1,0){0.5}}		% Zeiger
\put(3,1.5){\circle*{0.1}}			% Referenzpunkt
\put(3,0.5){\line(1,0){3}}			% untere Linie
\put(3,0.5){\line(0,1){5}}			% links senkrecht
\put(6,0.5){\line(0,1){5}}			% rechts senkrecht
\put(3,5.5){\line(1,0){3}}			% obere Linie
\put(3,1.5){\line(1,0){3}}			% Bezugslinie
\put(4.5,1.7){\makebox(0,0){Basislinie}}	% Text
\put(4.5,0){\makebox(0,0){Breite}}		% Text
\put(7,1){\makebox(0,0){Tiefe}}			% Text
\put(7,3.5){\makebox(0,0){H\"ohe}}		% Text
\put(4.5,0.3){\vector(1,0){1.5}}		% Breiten-
\put(4.5,0.3){\vector(-1,0){1.5}}		% Vector
\put(6.2,1){\vector(0,1){0.5}}			% Tiefen-
\put(6.2,1){\vector(0,-1){0.5}}			% Vektor
\put(6.2,3.5){\vector(0,1){2}}			% H\"ohen-
\put(6.2,3.5){\vector(0,-1){2}}			% Vektor
\end{picture}

Die Bedeutung der \index{Linie!Basis}
Basislinie und des Referenzpunktes wird sofort
deutlich. Wenn n\"amlich die Zeichen zu einer Zeile zusammengef\"ugt
werden, dann befinden sich die Basislinien aller Zeichen auf einer
H\"ohe.\footnote{Die \index{Leer!Zeichen}
Leerzeichen sind {\em keine} Boxen, doch dazu im
n\"achsten Abschnitt mehr.} Die Zeichen werden dabei zu einer
\"ubergeordneten Box, einer sogenannten `{\tt hbox}' (horizontale Box)
zusammengef\"ugt. Die einzelnen Zeilen, jetzt zusammengef\"ugt in ihren
{\tt hboxen} werden dann \"ubereinandergef\"ugt, und zwar so, da\ss{} die
Referenzpunkte der {\tt hboxen} jeweils auch wieder genau \"ubereinander
liegen. \TeX\ w\"are nicht \TeX, wenn sich so etwas nicht auch {\em
manuell} bewerkstelligen lie\ss{}e. Um eine {\em fertige} {\tt hbox} zu
erhalten braucht man nur den richtigen Befehl und den darin
enthaltenen Test. Die Sequenz:
\index{zzzhbox@{\tt\char92 hbox}}
\begin{verbatim}
\hbox{Eine Zeile mit Text}
\end{verbatim}
ergibt eine vollst\"andige {\tt hbox}. Um deutlicher zu machen, wie
\TeX\ das nun setzt, sollen die Boxen einmal mitgezeichnet
werden:\medskip

\thicklines
\fbox{\fbox{E}\fbox{i}\fbox{n}\fbox{e}
\fbox{Z}\fbox{e}\fbox{i}\fbox{l}\fbox{e} \fbox{m}\fbox{i}\fbox{t}
\fbox{T}\fbox{e}\fbox{x}\fbox{t}}\medskip

Der entsprechende Befehl f\"ur die \"ubereinandergesetzten {\tt hboxen}
hei\ss{}t \"ubrigens {\tt vbox} (f\"ur vertikale Box, wie kaum anders zu
erwarten war. Der entsprechende Befehl daf\"ur lautet:
\index{zzzhbox@{\tt\char92 hbox}}
\index{zzzvbox@{\tt\char92 vbox}}
\begin{verbatim}
\vbox{\hbox{Zwei Zeilen}\hbox{purer Text}}
\end{verbatim}
und das gesetzte Ergebnis sieht dann folgenderma\ss{}en aus:\medskip

\fbox{\vbox{\hbox{\fbox{\fbox{Z}\fbox{w}\fbox{e}\fbox{i}
\fbox{Z}\fbox{e}\fbox{i}\fbox{l}\fbox{e}\fbox{n}}}
\hbox{\fbox{\fbox{p}\fbox{u}\fbox{r}\fbox{e}\fbox{r}
\fbox{T}\fbox{e}\fbox{x}\fbox{t}}}}}\medskip

Bei diesem zweiten Beispiel ist auch eine Teilbox erkennbar, die eine
{\em echte} Tiefe besitzt. Sehen sie sich dazu bitte den Rahmen um das
`p' des Wortes `purer' einmal genau an. Sie k\"onnen sehen, da\ss{} der
Rahmen tiefer hinunterreicht, als die Rahmen der anderen Buchstaben.
Auch die, je nach Buchstabe, unterschiedliche Breite der Boxen ist
deutlich erkennbar.

Normalerweise brauchen sie die Boxbefehl nie, sie erlauben aber
absolute Kontrolle \"uber das Aussehen des Ergebnisses. Es sind z.B.
auch negative \index{Dimension!Angabe}
Dimensionsangaben f\"ur die Breite einer Box zul\"assig, so
da\ss{} auch Buchstaben \"ubereinander gedruckt werden k\"onnen.

Boxen k\"onnen auch in internen \index{Register!Box}
Registern gespeichert werden. Die Angabe
von:
\index{zzzhbox@{\tt\char92 hbox}}\index{zzzkern@{\tt\char92 kern}}
\index{zzzlower@{\tt\char92 lower}}
\index{zzzsetbox@{\tt\char92 setbox}}
\begin{verbatim}
\setbox0=\hbox{T\kern-.1667em\lower.5ex\hbox{E}\kern-.125em X}
\end{verbatim}
speichert das \TeX\ \index{Logo}Logo in der {\tt box0}, einem der internen
\index{Register!Intern}
Register von \TeX. Sie k\"onnen sich auch den Inhalt einer derartigen
Box ansehen. Mit \index{zzzshowbox@{\tt\char92 showbox}}
`\verb|\showbox0|' erhalten sie:
\index{zzzendtt@{\tt\char92 endtt}}\index{zzzhbox@{\tt\char92 hbox}}
\index{zzzkern@{\tt\char92 kern}}
\begin{verbatim}
\hbox(6.83331+2.15277)x18.6108
.\tenrm T
.\kern -1.66702
.\hbox(6.83331+0.0)x6.80557, shifted 2.15277
..\tenrm E
.\kern -1.25
.\tenrm X
\end{verbatim}
allerdings nur in ihrem \index{File!Log}
Logfile. Die Angaben im Einzelnen. Zun\"achst
sagt ihnen \TeX, da\ss{} es sich um eine Box mit der
\index{Hoehe!Box@H\"ohe!Box}H\"ohe 6.83331pt, Tiefe
2.15277pt und der Breite 18.6108pt handelt. In der folgenden Zeile
zeigt der f\"uhrende \index{Punkt}
Punkt an, da\ss{} sie sich sozusagen schon im Inneren
der Box befinden. Der erste Inhalt ist das Zeichen `T' in
`\verb|\tenrm|' als \index{Schrift!Art}
Schriftart. Jetzt folgt ein negativer
\index{Abstand}Abstand von
-1.66702pt, das `E' wird etwas an das `T' heranger\"uckt. Gefolgt von
einer weiteren Box, die 6.83331pt hoch ist, 0pt tief und 6.80557pt
breit, au\ss{}erdem ist sie um 2.15277pt nach unten verschoben, was die
{\tt shifted} Angabe aussagt. Der Inhalt dieser Box (angezeigt durch
die {\em zwei} \index{Punkt}
Punkte) ist das `E'. Den Rest k\"onnen sie sich jetzt
sicher selber erkl\"aren. Eine derartige Aufschl\"usselung ist sicher nur
zu \index{Diagnose}Diagnosezwecken sinnvoll.

Jetzt kennen sie fast schon den ganzen Weg von ihrer \index{Eingabe}
Eingabe bis zur \index{Ausgabe}
Ausgabe durch \TeX. Sie beginnen mit Token, die intern in Boxen
umgewandelt werden. Diese Boxen werden zu einer gr\"o\ss{}eren {\tt hbox}
zusammengef\"ugt, die {\tt hboxen} zu einer, oder mehreren {\tt vbox}
und diese wieder zu einer \index{Seite!Aus Boxen}
Seite, die nat\"urlich auch wieder eine Box
ist. Alles was ihnen noch zum Verst\"andnis fehlt ist das, was die Boxen
zusammenh\"alt, der {\em Leim}, englisch: {\em Glue}.
\section{Leim}
\subsection{Eigenschaften von Leim}
Neben den Boxen gibt es noch etwas in Texten, die mit \TeX\ gesetzt
wurden, und das hei\ss{}t {\em Leim}\footnote{Dieser Leim d\"urfte Lesern
der B\"ucher von \index{Kopka}H.\ Kopka bekannt sein, der Leim hei\ss{}t dort:
\index{Elastische Masse@Elastische
Ma\ss{}e}\index{Masse!Elastisch@Ma\ss{}e!Elastisch}{\em
elastische Ma\ss{}e}. Die Wichtigkeit dieser elastischen Ma\ss{}e wird erst
hier richtig klar.}. Die Zeilen dieses Textes haben z.B.\ einen
derartigen \index{Zwischenraum}Zwischenraum, da\ss{} der \index{Abstand}
Abstand der \index{Grundlinie}Grundlinien zweier Zeilen
immer 12pt betr\"agt.

Was macht nun diesen Leim aus? Er hat drei Eigenschaften: Eine {\em
nat\"urliches} Ausma\ss{}, ein \index{Schrumpf!Mass@Schrumpf!Ma\ss{}}
{\em Schrumpfma\ss{}} und ein {\em Streckma\ss{}}.
Wenn z.B.\ kleinere Boxen zu einer gr\"o\ss{}eren horizontalen Box
zusammengefa\ss{}t werden, dann wird zwischen den kleinen Boxen Leim
eingef\"ugt. Um zu verstehen, wie das funktioniert, ein kleines
Beispiel:

\thinlines
\begin{center}
\begin{picture}(10.4,7)
\put(0,2){\framebox(1,4){}}		% erste Box
\put(2.8,3.2){\framebox(1.2,1){}}	% zweite Box
\put(5.8,3){\framebox(.6,3){}}		% dritte Box
\put(8.8,2){\framebox(1.6,4){}}		% vierte Box
\put(0,4){\line(1,0){1}}		% Bezugslinie
\put(2.8,4){\line(1,0){1.2}}		% Bezugslinie
\put(5.8,4){\line(1,0){.6}}		% Bezugslinie
\put(8.8,4){\line(1,0){1.6}}		% Bezugslinie
\put(1,4){\dashbox{.1}(1.8,0){}}	% Verbindungslinie
\put(4,4){\dashbox{.1}(1.8,0){}}	% Verbindungslinie
\put(6.4,4){\dashbox{.1}(2.4,0){}}	% Verbindungslinie
\put(0,6.5){Breite 5}
\put(2.8,6.5){Breite 6}
\put(5.8,6.5){Breite 3}
\put(8.8,6.5){Breite 8}
\put(1.1,3.5){\makebox(0,0)[tl]{\parbox{1.8cm}{Platz 9\\Streck 3\\Schrumpf 1}}}
\put(4.1,3.5){\makebox(0,0)[tl]{\parbox{1.8cm}{Platz 9\\Streck 6\\Schrumpf 2}}}
\put(6.5,3.5){\makebox(0,0)[tl]{\parbox{1.8cm}{Platz 12\\Streck 0\\Schrumpf 0}}}
\put(5.2,1){\makebox(0,0){Breite 52}}	% Gesamtbreite
\put(4,1){\vector(-1,0){4}}		% Anzeiger
\put(6.4,1){\vector(1,0){4}}
\put(0,4){\circle*{.1}}			% Bezugs-
\put(2.8,4){\circle*{.1}}		% punkte
\put(5.8,4){\circle*{.1}}
\put(8.8,4){\circle*{.1}}
\end{picture}
\end{center}

Das erste Leimst\"uck hat z.B.\ eine nat\"urliche Breite von 9pt, eine
\index{Schrumpf!Moeglichkeit@Schrumpf!M\"oglichkeit}
Schrumpfm\"oglichkeit von 1pt und eine Streckm\"oglichkeit von 3pt. Die
\index{Gesamt!Breite}
Gesamtbreite der ganzen, gro\ss{}en \index{Box!Breite}
Box ergibt sich normalerweise als die
Breite der Einzelboxen, plus die nat\"urliche Breite des Leimes, also:
\[5+9+6+9+3+12+8=52\]
Angenommen, \TeX\ w\"urde aufgefordert diese Box auf eine Breite von
58pt zu strecken (z.B.\ wegen der \index{Rechtsbuendig@Rechtsb\"undig}
Rechtsb\"undigkeit), d.h.\ die Box um
6pt zu strecken. Nun, es gibt insgesamt $3+6+0=9$pt als
\index{Gesamt!Streckmoeglichkeit@Gesamt!Streckm\"oglichkeit}
Gesamtstreckm\"oglichkeit. \TeX\ multipliziert nun jede
Streckm\"oglichkeit mit $6/9$ um die neuen
\index{Groesse!Leim@Gr\"o\ss{}e!Leim}Leimgr\"o\ss{}en zu errechnen.
Damit ergibt sich f\"ur die erste L\"ucke: $9+(6/9)\times 3=11$pt und f\"ur
die zweite L\"ucke: $9+(6/9)\times 6=13$pt. Die letzte L\"ucke ist nicht
weiter streckbar, so da\ss{} sie $12$pt breit bleibt. Das
\index{Gesamt!Bild}Gesamtbild sieht
nun folgenderma\ss{}en aus:

\begin{center}
\begin{picture}(11.6,7)
\put(0,2){\framebox(1,4){}}		% erste Box
\put(3.2,3.2){\framebox(1.2,1){}}	% zweite Box
\put(7,3){\framebox(.6,3){}}		% dritte Box
\put(10,2){\framebox(1.6,4){}}		% vierte Box
\put(0,4){\line(1,0){1}}		% Bezugslinie
\put(3.2,4){\line(1,0){1.2}}		% Bezugslinie
\put(7,4){\line(1,0){.6}}		% Bezugslinie
\put(10,4){\line(1,0){1.6}}		% Bezugslinie
\put(1,4){\dashbox{.1}(2.2,0){}}	% Verbindungslinie
\put(4.4,4){\dashbox{.1}(2.6,0){}}	% Verbindungslinie
\put(7.6,4){\dashbox{.1}(2.4,0){}}	% Verbindungslinie
\put(1.1,3.5){$9+2$}
\put(4.5,3.5){$9+4$}
\put(7.7,3.5){$12+0$}
\put(5.8,1){\makebox(0,0){Breite 58}}	% Gesamtbreite
\put(4.6,1){\vector(-1,0){4.6}}		% Anzeiger
\put(7,1){\vector(1,0){4.6}}
\put(0,4){\circle*{.1}}			% Bezugs-
\put(3.2,4){\circle*{.1}}		% punkte
\put(7,4){\circle*{.1}}
\put(10,4){\circle*{.1}}
\end{picture}
\end{center}

W\"are \TeX, umgekehrt angewiesen worden die Box nur 51pt breit zu
machen, dann w\"urde sie erste L\"ucke um $1/3$, die zweite um $2/3$ ihrer
\index{Schrumpf!Moeglichkeit@Schrumpf!M\"oglichkeit}
Schrumpfm\"oglichkeit verkleinert.

Egal wie, der Leim definiert wurde, so hat \TeX\ dennoch nicht die
M\"oglichkeit eine Box beliebig weit zusammenzuschrumpfen. Im vorherigen
Beispiel ist daher $49$pt die geringste Breite, die die horizontale
Box annehmen kann. Hingegen kann der Leim beliebig weit gestreckt
werden, wenn man ihm die entsprechenden Angaben beigibt.

Es fragt sich nur, welche Angaben gibt man denn dem Leim? Nun, man
sollte die Ma\ss{}e immer so w\"ahlen, da\ss{} der Text mit dem nat\"urlichen Ma\ss{}
des Leimes am besten aussieht, und die maximale
\index{Schrumpf!Mass@Schrumpf!Ma\ss{}}Schrumpf- bzw.
\index{Masse!Streck@Ma\ss{}e!Streck}
Streckma\ss{}e so, da\ss{} er gerade eben noch gut aussieht. Normalerweise
brauchen sie sich allerdings nicht darum zu k\"ummern, da \TeX\ ihnen
diese Arbeit abnimmt. Um z.B.\ die Abst\"ande zwischen Abs\"atzen zu
bestimmen, gibt es in {\em Plain} \TeX\ den Befehl:
\index{zzzsmallskip@{\tt\char92 smallskip}}`\verb|\smallskip|'.
Dieser Befehl wird automatisch immer hinter einem \index{Absatz}
Absatz eingef\"ugt,
sie kennen also schon genug Beispiele f\"ur die Anwendung. Es ist immer
besser einen derartigen Befehl einzusetzen, als den \index{Abstand}
Abstand explizit
anzugeben. Nur, wenn es unbedingt sein mu\ss{}, k\"onnen sie auf den Leim
zur\"uckgreifen. Stellt sich die Frage, {\em wie} man den nun eine
Leimangabe macht. \"Ublicherweise wird Leim in \TeX\ folgenderma\ss{}en
spezifiziert:
\begin{quote}
$<$dimen$>$ {\tt plus} $<$dimen$>$ {\tt minus} $<$dimen$>$
\end{quote}
Die Angaben von {\tt plus} $<$dimen$>$ und {\tt minus} $<$dimen$>$
sind optional, und werden als Null angenommen, wenn sie nicht
angegeben werden. Die \index{Groesse!Leim@Gr\"o\ss{}e!Leim}
Gr\"o\ss{}e, die hinter {\tt plus} steht repr\"asentiert
sie Streckbarkeit, die hinter {\tt minus} die
\index{Schrumpf!Schrumpfbarkeit}Schrumpfbarkeit. Die
nat\"urliche Gr\"o\ss{}e {\em mu\ss{}} angegeben werden, auch wenn sie Null ist.
Der Befehl \index{zzzmedskip@{\tt\char92 medskip}}
`\verb|\medskip|' wird z.B.\ in {\em Plain} \TeX\ folgenderma\ss{}en
definiert:
\index{zzzvskip@{\tt\char92 vskip}}
\begin{verbatim}
\vskip6pt plus2pt minus2pt
\end{verbatim}
Auch \index{Horizontal!Platz}horizontaler Platz wird entsprechend definiert. Der Befehl
\index{zzzenskip@{\tt\char92 enskip}}
`\verb|\enskip|' wird z.B.\ definiert als:\index{zzzhskip@{\tt\char92 hskip}}
\index{zzzrelax@{\tt\char92 relax}}
\begin{verbatim}
\hskip.5em\relax
\end{verbatim}
Es wird also ein halbes `em' Platz gelassen, allerdings ohne die
M\"oglichkeit der \index{Schrumpfung}Schrumpfung oder Streckung. Der Befehl
\index{zzzrelax@{\tt\char92 relax}}`\verb|\relax|'
hinter der Angabe sorgt daf\"ur, da\ss{} der Befehl auch dann richtig
arbeitet, wenn im Text hinter dem Befehl eine `plus' oder `minus'
steht.

Eine wichtige Eigenschaft des Leimes ist, da\ss{} er auch {\em unendlich}
weit gestreckt werden kann. Nehmen sie z.B.\ an, im letzten Beispiel
w\"are der zweite Leimabschnitt unendlich weit streckbar. Dann w\"urde nur
der zweite \index{Zwischenraum}
Zwischenraum gestreckt, w\"arend die andern unver\"andert
bleiben w\"urden.

Das wird erst interessant, wenn sie daran denken, da\ss{}
der Befehl \index{zzzcenterline@{\tt\char92 centerline}}
`\verb|\centerline|' nichts anderes ist, als
eine \index{Box!Hbox}hbox von der Breite \index{zzzhsize@{\tt\char92 hsize}}
`\verb|\hsize|', deren Text von zwei
unendlich streckbaren Leimst\"ucken eingerahmt ist. Auch in der kleinen
Geschichte, die wir in einem fr\"uheren Kapitel kennengelernt haben
wurde derartiger Leim benutzt. Die Geschichte endete mit den Befehlen:
\index{zzzvfill@{\tt\char92 vfill}}
`\verb|\vfill|' und \index{zzzeject@{\tt\char92 eject}}
`\verb|\eject|'. `\verb|\eject|' sorgt nur f\"ur die
\index{Ausgabe}
Ausgabe des Textes an den \index{Ausgabe!File}
Ausgabefile, `\verb|\vfill|' hingegen
bedeutet: F\"uge vertikalen \index{Zwischenraum}
Zwischenraum mit der nat\"urlichen L\"ange 0
ein, der aber unendlich weit streckbar ist. Mit anderen Worten, f\"ulle
die restliche \index{Seite!Auffuellen@Seite!Auff\"ullen}
Seite mit \index{Leer!Raum}Leerraum auf.

Es gibt nun in \TeX\ verschiedene Arten von {\em Unendlichkeit}, die
sich voneinander unterscheiden. Bei der vertikalen Streckung gibt es
die Befehle: \index{zzzvfil@{\tt\char92 vfil}}
`\verb|\vfil|' und `\verb|\vfill|', wobei der zweite
wesentlich st\"arker wirkt. Wenn `\verb|\vfil|' alleine benutzt wird,
dann wird es soweit, wie n\"otig gestreckt. Tritt es allerdings zusammen
mit `\verb|\vfill|' auf, dann wird es nicht gestreckt, sondern es wird
da gestreckt, wo `\verb|\vfill|' steht. Denken sie sich das einfach
so, als h\"atte `\verb|\vfil|' eine Streckf\"ahigkeit von einem Kilometer,
und `\verb|\vfill|' von einer Million Kilometer.

Analog gibt es auch noch die Befehl
\index{zzzhfil@{\tt\char92 hfil}}`\verb|\hfil|' und
\index{zzzhfill@{\tt\char92 hfill}}`\verb|\hfill|'
und die Befehle \index{zzzhss@{\tt\char92 hss}}
`\verb|\hss|' und \index{zzzvss@{\tt\char92 vss}}
`\verb|\vss|', die sowohl Streckung,
als auch unendliche \index{Schrumpfung}Schrumpfung zulassen.

In diesen Zusammenhang geh\"ort auch noch der Befehl
\index{zzzhfilneg@{\tt\char92 hfilneg}}`\verb|\hfilneg|',
bzw.\ \index{zzzvfilneg@{\tt\char92 vfilneg}}
`\verb|\vfilneg|'. Diese Befehle heben die Streckbarkeit von
\index{zzzhfil@{\tt\char92 hfil}}
`\verb|\hfil|' und `\verb|\vfil|' auf. Den Grund f\"ur einen derartigen
Befehl werden wir erst sp\"ater kennenlernen.

Um derartige unendlich streckbare \index{Masse!Streck@Ma\ss{}e!Streck}
Ma\ss{}e erstelle zu k\"onnen, k\"onnen sie
auf die \index{Primitiv}
Primitive `\verb|fil|', `\verb|fill|' oder `\verb|filll|'
zur\"uckgreifen. Die Befehle `\verb|\vfil|', `\verb|\vfill|',
`\verb|\vss|' und `\verb|\vfilneg|' werden z.B.\ definiert als:
\index{zzzvskip@{\tt\char92 vskip}}
\begin{verbatim}
\vskip 0pt plus 1fil
\vskip 0pt plus 1fill
\vskip 0pt plus 1fil minus 1fil
\vskip 0pt plus -1fil
\end{verbatim}
Die dritte Alternative: `\verb|filll|' sollten sie nicht benutzen,
auch {\em Plain} \TeX\ macht keinen Gebrauch davon. Sie ist f\"ur absolute
Notf\"alle vorbehalten.
\subsection{Verwendung des Leims durch \TeX}
Wie wird der Leim von {\em Plain} \TeX\ verwendet? Es gibt einige
interessante Anwendungen. \TeX\ f\"ugt z.B.\ hinter einem Satz einen
etwas \index{Groesse!Zwischenraum@Gr\"o\ss{}e!Zwischenraum}
gr\"o\ss{}eren \index{Zwischenraum}Zwischenraum ein, als zwischen normalen Worten.
Au\ss{}erdem wird die Streckbarkeit des Zwischenraums an diesen Stellen
erh\"oht. Betrachten wir folgenden Satz:\footnote{Originalbeispiel aus
dem \TeX Book.}
\begin{verbatim}
``Oh, oh!'' cried Baby Sally. Dick and Jane laughed.
\end{verbatim}
In nat\"urlicher Breite sieht der Satz so aus:\medskip

``Oh, oh!'' cried Baby Sally. Dick and Jane laughed.\medskip

Was aber passiert, wenn man ihn um 5pt, 10pt, 15pt oder mehr
auseinanderzieht? Sehen wir es uns an:\medskip

\hbox spread 5pt{``Oh, oh!'' cried Baby Sally. Dick and Jane laughed.}

\hbox spread 10pt{``Oh, oh!'' cried Baby Sally. Dick and Jane laughed.}

\hbox spread 15pt{``Oh, oh!'' cried Baby Sally. Dick and Jane laughed.}

\hbox spread 20pt{``Oh, oh!'' cried Baby Sally. Dick and Jane laughed.}\medskip

Der Leim hinter dem \index{Komma}Komma wird um das $1\frac{1}{4}$fache des Normalen
gestreckt. Hinter dem \index{Punkt}
Punkt und hinter dem `\verb|!''|' sogar um das
dreifache. Zwischen zusammengeh\"origen Zeichen ist kein Leim, so da\ss{}
die einzelnen Worte ihr Aussehen behalten. Wie siehts nun aus, wenn
dieser Satz auf das Minimum zusammengestaucht wird?\medskip

{\hfuzz 1000pt
\hbox to 0pt{``Oh, oh!'' cried Baby Sally. Dick and Jane laughed.}}
\medskip

Der \index{Zwischenraum}Zwischenraum nach dem \index{Komma}
Komma wird nur 80\% dessen, der Zwischenraum
nach dem \index{Punkt}
Punkt, bzw.\ dem Ausrufezeichen sogar nur ein Drittel dessen
kleiner, wie der normale Zwischenraum zwischen den Worten.

F\"ur den Normalfall ist das auch so in Ordnung, was aber, wenn ein
\index{Punkt}
Punkt \"uberhaupt nicht ein Wortende markiert? Dazu gibt es mehrere
M\"oglichkeiten:
\begin{enumerate}
\item Wenn sie drei \index{Punkt}
Punkte schreiben wollen, um die Fortsetzung eines
Satzes anzuzeigen $\ldots$, dann sollten sie statt der drei Punkte
lieber die Sequenz \index{zzzldots@{\tt\char92 ldots}}
`\verb|$\ldots$|' schreiben. Damit schalten sie
kurzfristig in den mathematischen \index{Modus!Mathematisch}
Modus um und geben dort mit dem
Befehl `\verb|\ldots|' die gew\"unschten Punkte an. \"Ubrigens sieht diese
Version auch besser aus. Drei Punkte ergeben n\"amlich normalerweise, ohne
die Umschaltung in den mathematischen Modus: `...'.
\item \index{Punkt}Punkte hinter \index{Abkuerzung@Abk\"urzung}
Abk\"urzungen beenden \"ublicherweise auch keinen
Satz. Hier gibt es zwei M\"oglichkeiten. Entweder, sie ersetzen das
\index{Leer!Zeichen}
Leerzeichen hinter dem \index{Punkt}
Punkt durch die Tilde: `\verb|~|', oder sie
setzen ein Kontrollleerzeichen: `\verb*+\ +', je nachdem, ob sie eine
Trennung an dieser Stelle zulassen wollen oder nicht. Die Tilde wird
durch ein Leerzeichen ersetzt, an dem aber kein Zeilenumbruch
stattfinden kann.
\item Sie k\"onnen sich diese \"Uberlegungen sparen, wenn der
\index{Punkt}Punkt hinter
einem \index{Grossbuchstaben@Gro\ss{}buchstaben}
Gro\ss{}buchstaben steht. \TeX\ nimmt dann sowieso an, da\ss{} es sich
um eine \index{Abkuerzung@Abk\"urzung}
Abk\"urzung handelt und behandelt den \index{Punkt}Punkt nicht wie einen
\index{Satz!Ende}Satz-Ende-Punkt.
\end{enumerate}
Letztlich k\"onnen sie alle \"Uberlegungen zu \index{Zwischenraum}
Zwischenr\"aumen vergessen,
wenn sie zu Beginn ihres Textes den Befehl:
\index{zzzfrenchspacing@{\tt\char92 frenchspacing}}`\verb|\frenchspacing|'
schreiben. Mit diesem Befehl bewirken sie, da\ss{} alle Zwischenr\"aume
gleich gro\ss{} sind, egal, wie und wo sie stehen. Unser Beispiel s\"ahe
damit folgenderma\ss{}en aus:\medskip

{\frenchspacing
``Oh, oh!'' cried Baby Sally. Dick and Jane laughed.}\medskip

Mit `\verb|\nonfrenchspacing|' wird wieder zur\"uckgeschaltet.
Angenommen, wir h\"atten den Befehl gegeben, normal mit
\index{Zwischenraum}Zwischenraum
umzugehen, dann k\"onnten wir uns den Leim ansehen, indem wir den Satz
in eine Box schreiben und diese dann ansehen:
\index{zzzglue@{\tt\char92 glue}}\index{zzzkern@{\tt\char92 kern}}
\begin{verbatim}
.\tenrm \ (ligature ``)
.\tenrm O
.\tenrm h
.\tenrm ,
.\glue 3.33333 plus 2.08331 minus 0.88889
.\tenrm o
.\tenrm h
.\tenrm !
.\tenrm \" (ligature '')
.\glue 4.44444 plus 4.99997 minus 0.37036
.\tenrm c
.\tenrm r
.\tenrm i
.\tenrm e
.\tenrm d
.\glue 3.33333 plus 1.66666 minus 1.11111
.\tenrm B
.\tenrm a
.\tenrm b
.\kern-0.27779
.\tenrm y
.\glue 3.33333 plus 1.66666 minus 1.11111
.\tenrm S
.\tenrm a
.\tenrm l
.\tenrm l
.\tenrm y
.\kern-0.83334
.\tenrm .
.\glue 4.44444 plus 4.99997 minus 0.37036
\end{verbatim}
Normalerweise hat der Leim bei der \index{Schrift!Art}
Schriftart `\verb|\tenrm|' die
Werte: 3.33333pt plus 1.66666 pt minus 1.11111 pt, aber achten sie
darauf, wie sich bei der \index{Interpunktion}
Interpunktion der Leim \"andert. Sie k\"onnen an
diesem Beispiel noch mehr sehen. Zum einen wandelt \TeX\ die beiden
`\verb|``|' in eine \index{Ligatur}
Ligatur (also {\em ein} Zeichen) und ausserdem
werden noch einige `\verb|\kern|' Befehle eingef\"ugt. `\verb|\kern|'
verh\"at sich \"ahnlich wie Leim, nur das er weder streck- noch
\index{Schrumpf!Schrumpfbarkeit}
schrumpfbar ist und es wird hinter einem `\verb|\kern|' Befehl {\em
nie} ein Zeilenumbruch durchgef\"uhrt, es sei denn es folgt unmittelbar
ein Leimbefehl.
\subsection{Die genauen Regeln f\"ur den Leim}
Ihnen d\"urfte eventuell schon aufgefallen sein, da\ss{} die oben
beschriebenen Regeln nicht {\em hunderprozentig} richtig sind. So
w\"urden die obigen Regeln nicht erkl\"aren, wieso das Ausrufezeichen
richtig behandelt wird, obwohl noch die beiden
\index{Anfuehrungszeichen@Anf\"uhrungszeichen}Anf\"uhrungsstriche
folgen.

Es gibt einen weiteren wichtigen Wert, den \index{Zwischenraum!Faktor}
Zwischenraumfaktor ($f$).
Dieser ist \"ublicherweise $1000$. Wenn der Faktor gr\"o\ss{}er als $2000$
wird, wird der \index{Zwischenraum}
Zwischenraum folgenderma\ss{}en berechnet: Addiere zum
normalen Zwischenraum den sog. Extrazwischenraum\footnote{Zu jedem
Zeichensatz sind die folgenden Angaben vorgegeben: Normaler
Zwischenraum, normaler Streckwert, normaler \index{Schrumpf!Wert}
Schrumpfwert und eben
dieser Extrazwischenraum. F\"ur {\tt cmr10} sind die Werte: 3.33333pt,
1.66666pt, 1.11111pt und 1.11111pt} und multipliziere den Streckwert
mit $f/1000$, den Schrumpfwert mit $1000/f$.

Es gibt nun f\"ur sie einfache M\"oglichkeiten den normalen Zwischenraum
zu \"uberschreiben. Mit den Befehlen: \index{zzzspaceskip@{\tt\char92 spaceskip}}
`\verb|\spaceskip|' und `\verb|\xspaceskip|' wird folgenderma\ss{}en verfahren:
Ist der
Zwischenraumfaktor ($f$) gr\"o\ss{}er als 2000, dann wird entweder der
Leimwert von \index{zzzxspaceskip@{\tt\char92 xspaceskip}}
`\verb|\xspaceskip|' genommen, wenn dieser ungleich Null
ist, oder der von `\verb|\spaceskip|', wobei die Streck- und
\index{Schrumpf!Komponenten}
Schrumpfkomponenten mit $f/1000$ bzw.\ $1000/f$ multipliziert werden.
Das \index{zzzraggedright@{\tt\char92 raggedright}}
`\verb|\raggedright|' Makro z.B.\ benutzt die beiden Befehle, um
jedes Strecken oder \index{Schrumpfung}Schrumpfen zu verhindern.

Wenn eine \index{Horizontal!Liste}horizontale \index{Liste!Horizontal}
Liste begonnen wird, wird der \index{Zwischenraum!Faktor}
Zwischenraumfaktor
auf 1000 gesetzt. Ebenso nach jedem \index{Nichtbuchstaben}
Nichtbuchstaben oder einer
mathematischen Formel. Jedes andere Zeichen, was an die Liste angef\"ugt
wird, hat einen eigenen Zwischenraumfaktor, der dann den alten Wert
ersetzt. Dabei gibt es allerdings noch zwei Regeln:
\begin{enumerate}
\item Ist der Faktor des Zeichens 0, dann wird der alte Wert
beibehalten,
\item Ist der Wert von $1000$ verschieden, und es gilt z.B.: $f<1000<g$
($g$ der Faktor des Zeichens), dann erh\"alt $f$ den Wert $1000$. Mit
andern Worten, $f$ {\em \"uberspringt} nicht die $1000$er Grenze.
\end{enumerate}
Der gr\"o\ss{}tm\"ogliche Zwischenraumfaktor ist $32767$, und damit gr\"o\ss{}er,
als jemals n\"otig.

Wenn {\tt INITEX} seine Arbeit aufnimmt, haben alle Zeichen den
Zwischenraumfaktor 1000, au\ss{}er den
\index{Grossbuchstaben@Gro\ss{}buchstaben}Gro\ss{}buchstaben, die den Wert 999
haben.\footnote{Erraten sie, warum \index{Punkt}
Punkte hinter Gro\ss{}buchstaben anders
behandelt werden?} {\em Plain} \TeX\ \"andert nun einige Werte mit dem
\index{Primitiv}\index{zzzcatcode@{\tt\char92 catcode}}
Primitiv: \index{zzzsfcode@{\tt\char92 sfcode}}
`\verb|\sfcode|', der \"ahnlich dem `\verb|\catcode|' Befehl
arbeitet.
\begin{verbatim}
\sfcode`)=0   \sfcode`.=3000
\end{verbatim}
\"andert den Faktor f\"ur die schlie\ss{}ende \index{Klammer!Platz um}
Klammer so, da\ss{} er ignoriert
wird, f\"ur den \index{Punkt}
Punkt so, da\ss{} zus\"atzlicher \index{Zwischenraum}Zwischenraum eingef\"ugt wird.

Bei \index{Ligatur}Ligaturen berechnet sich der
\index{Zwischenraum!Faktor}Zwischenraumfaktor aus den
Einzelkomponenten, aus denen die Ligatur besteht.

Nun zu einem Beispiel, wie \TeX\ {\em wirklich} eine Zeile
zusammensetzt: Die nat\"urliche Breite einer \index{Box!Breite}
Box setzt sich zusammen aus
allen Zeichenboxbreiten plus den nat\"urlichen Breiten der Leimst\"ucke.
Ist die nat\"urliche Breite gleich der gew\"unschten Breite, passiert
nichts. Ist hingegen die gew\"unschte Breite kleiner oder gr\"o\ss{}err, als
die nat\"urliche Breite, dann wird ein Faktor berechnet, mit dem jeder
\index{Schrumpf!Faktor}
Schrumpf- bzw.\ Streckfaktor der Leimst\"ucke multipliziert wird. Das
Ergebnis ist eine Box, die genau die gew\"unschte Breite hat.

Um einer \index{Box!Hbox}{\tt hbox} eine feste Breite zu geben benutzt man den Befehl:
\index{zzzhbox@{\tt\char92 hbox}}
\begin{verbatim}
\hbox to <dimen>{ ... }
\end{verbatim}
\index{zzzcenterline@{\tt\char92 centerline}}
Das Makro `\verb|\centerline|' beginnt z.B.: `\verb|\hbox to\hsize|'.
\"Ahnlich arbeitet die Angabe von:
\index{zzzhbox@{\tt\char92 hbox}}
\begin{verbatim}
\hbox spread <dimen>{ ... }
\end{verbatim}
Die angegebene \index{Dimension}
Dimension besagt, um wieviel gr\"o\ss{}er, als die nat\"urliche
Breite ($<$dimen$> > 0$), bzw.\ um wieviel kleiner als die nat\"urliche
\index{Groesse!Natuerliche@Gr\"o\ss{}e!Nat\"urliche}
Gr\"o\ss{}e ($<$dimen$> < 0$) die {\tt hbox} werden soll.
\section{Die Feinheiten einer vertikalen Box}\index{Box!Vbox}
Die \index{Grundlinie}
Grundlinie einer horizontalen Box ist gleich der gemeinsamen
Grundlinie aller Elemente, soweit diese nicht angehoben oder abgesenkt
wurden. Die \index{Hoehe!Box@H\"ohe!Box}
H\"ohe einer derartigen Box ist gleich der H\"ohe des h\"ochsten
Elements, die Tiefe gleich der tiefsten Einzeltiefe. Somit kann eine
horizontale Box nur positive H\"ohe und Tiefe haben, wohl aber eine
negative Breite.

Wenn nun horizontale Boxen zu einer vertikalen Box zusammengef\"ugt
werden sollen, dann funktioniert das \"ahnlich, wie bei den horizontalen
Boxen, mit einem Unterschied. Die horizontalen Boxen sollen sich nicht
ber\"uhren. Dazu gibt es drei Werte, die vorher festgelegt werden
m\"ussen:
\index{zzzbaselineskip@{\tt\char92 baselineskip}}
\index{zzzlineskip@{\tt\char92 lineskip}}
\index{zzzlineskiplimit@{\tt\char92 lineskiplimit}}
\begin{verbatim}
\baselineskip=<Leim>
\lineskip=<Leim>
\lineskiplimit=<dimen>
\end{verbatim}\index{Baseline!Skip}
Wird nun eine neue horizonatale \index{Box!In Liste}Box an eine vertikale
\index{Liste!Vertikal}Liste angef\"ugt,
dann wird der \index{Abstand!Zeilen}
Zeilenabstand so berechnet, da\ss{} der Abstand der
Basislinie der neuen \index{Box!Basislinie}
Box, von der Basislinie der letzten Box genau den
\index{zzzbaselineskip@{\tt\char92 baselineskip}}
Wert von `\verb|\baselineskip|' hat. Ist das Ergebnis so, da\ss{} die
Zeilen mit dieser L\"osung zu nahe aneinanderliegen, genauer, n\"aher als
der Wert in `\verb|\lineskiplimit|', dann wird statt dessen der Wert
\index{zzzlineskip@{\tt\char92 lineskip}}`\verb|\lineskip|' benutzt.

Als Ausnahme von dieser Regel dienen nur die Befehle:
\index{zzzvskip@{\tt\char92 vskip}}`\verb|\vskip|'
und \index{zzzkern@{\tt\char92 kern}}
`\verb|\kern|', und alle daraus zusammengesetzten Befehle. Hiermit
ist es m\"oglich einen anderen, als den normalen \index{Abstand!Zeilen}
Zeilenabstand zu
erzeugen. Z.B.\ der Befehl
\index{zzzsmallskip@{\tt\char92 smallskip}}`\verb|\smallskip|' arbeitet so. Eine
weitere Ausnahme besteht darin, da\ss{} hinter
\index{zzzxrule@{\tt\char92 xrule}}`\verb|\xrule|'-Befehlen
kein \index{Zwischenraum!Zeile}
Zeilenzwischenraum eingef\"ugt wird. Sie k\"onnen den \index{Zwischenraum}
Zwischenraum
aber auch mit dem Befehl: \index{zzznointerlineskip@{\tt\char92 nointerlineskip}}
`\verb|\nointerlineskip|' selber verhindern.

Als Regel sollte sie sich folgendes angew\"ohnen: Wenn ihr Text \"uber
mehrere Zeilen geht, dann sollte \index{zzzbaselineskip@{\tt\char92 baselineskip}}
`\verb|\baselineskip|' {\em keine}
\index{Schrumpf!Angabe}
Schrumpf- oder Streckangaben enthalten, um so das Aussehen der
\index{Seite!Aussehen}Seiten
gleichm\"a\ss{}iger zu gestalten. Bei Texten, die nur \"uber eine Seite gehen,
ist dagegen die Angabe dieser Werte sinvoll, um eine bessere
Ausnutzung der Seite zu erhalten.

Zur Berechnung des \index{Zwischenraum!Zeile}
Zeilenzwischenraums greift \TeX\ \"ubrigens auf ein
\index{Primitiv}
Primitiv mit Namen:
\index{zzzprevdepth@{\tt\char92 prevdepth}}
`\verb|\prevdepth|' zur\"uck. Normalerweise die
Tiefe der letzten horizontalen \index{Box!Tiefe}
Box, die angef\"ugt wurde. Der Wert wird
allerdings auf $-1000$ gesetzt, am Anfang einer neuen vertikalen
\index{Liste!Vertikal}Liste, oder nach einem
\index{zzzrule@{\tt\char92 rule}}`\verb|\rule|'-Befehl. Sie k\"onnen den Wert aber
auch {\em von Hand} \"andern, wenn sie es wollen.

Schlie\ss{}lich noch die exakten Regeln, nach denen eine neue horizontale
Box an eine vertikale Box angeh\"angt wird: Angenommen die neu Box hat
eine \index{Hoehe!Box@H\"ohe!Box}
H\"ohe von $h$, und der Wert von `\verb|\prevdepth|' ist $p$.
Weiterhin sollen sein:
\index{zzzbaselineskip@{\tt\char92 baselineskip}}
\index{zzzlineskiplimit@{\tt\char92 lineskiplimit}}
\begin{verbatim}
\baselineskip=b plus y minus z
\lineskiplimit=l
\end{verbatim}
Wenn $p\le -1000$pt wird kein \index{Zwischenraum!Zeile}
Zeilenzwischenraum eingef\"ugt. Ist
andererseits: $b-p-h\ge l$ dann wird der \index{Zwischenraum}
Zwischenraum $(b-p-h)
\mbox{ plus } x \mbox{ minus } y)$ eingef\"ugt, ansonsten {\tt\char92
lineskip}.

F\"ur die vertikalen Boxen gelten dieselben Befehle, wie f\"ur die
horizontalen Boxen, mit dem Unterschied, da\ss{} der Befehl: `\verb|\vbox|'
\index{zzzvbox@{\tt\char92 vbox}}
verwendet wird. Es sind nat\"urlich auch die Befehle: `\verb|\vbox to
...|' und `\verb|\vbox spread ...|' erlaubt.

Der tiefste vorkommende Referenzpunkt innerhalb einer vertikalen
\index{Box!Referenzpunkt}Box
wird als Referenzpunkt f\"ur die ganze vertikale Box genommen.
Normalerweise werden vertikale Boxen so untereinander gesetzt, da\ss{}
ihre Referenzpunkte senkrecht untereinander liegen, mit den Befehlen:
\index{zzzmoveleft@{\tt\char92 moveleft}}
\index{zzzmoveright@{\tt\char92 moveright}}
\begin{verbatim}
\moveright<dimen><box>
\moveleft<dimen><box>
\end{verbatim}
k\"onnen sie aber auch nach rechts oder links verschoben werden. Dies
entspricht den Befehlen
\index{zzzraise@{\tt\char92 raise}}
`\verb|\raise|' und \index{zzzlower@{\tt\char92 lower}}
`\verb|\lower|' in
horizontalen Boxen.
\section{Sonderf\"alle}
Neben dem Befehl f\"ur eine vertikale \index{Box!Vbox}
{\tt vbox} gibt es auch den
Befehl: \index{zzzvtop@{\tt\char92 vtop}}
`\verb|\vtop|'. Dieser erzeugt auch eine vertikale Box, und
die zugeh\"origen Befehle arbeiten komplett analog zu denen f\"ur
\index{zzzvbox@{\tt\char92 vbox}}
`\verb|\vbox|'. Der Unterschied besteht darin, da\ss{} die Basislinie der
{\em obersten} enthaltenen horizontalen Box als Referenzpunkt genommen
wird. Damit sind Konstruktionen wie die folgende m\"oglich:
\index{zzzhbox@{\tt\char92 hbox}}
\begin{verbatim}
\hbox{Hier sind \vtop{\hbox{zwei Zeilen}\hbox{mit Text}}}
\end{verbatim}
Das ergibt:\medskip

\hbox{Hier sind \vtop{\hbox{zwei Zeilen}\hbox{mit Text}}}\medskip

Ein weiteres Element bei der Arbeit mit vertikalen Boxen stellt der
Befehl \index{zzzstrut@{\tt\char92 strut}}
`\verb|\strut|' dar. Mit ihm wird eine Box der
\index{Hoehe!Box@H\"ohe!Box}H\"ohe 8.5pt, der
Tiefe 3.5pt und der Breite Null bereitgestellt. Sie k\"onnen diesen
Befehl benutzen, um immer korrekte Abst\"ande zu erzwingen. Schauen sie
sich z.B.\ im Anhang das `\verb|\footnote|' Makro an, das daf\"ur sorgt,
da\ss{} mehrere \index{Fussnote!Absetzen@Fu\ss{}note!Absetzen}
Fu\ss{}noten auch voneinander abgesetzt werden.

Letztlich sind noch die Befehle \index{zzzrlap@{\tt\char92 rlap}}
`\verb|\rlap|' und
\index{zzzllap@{\tt\char92 llap}}`\verb|\llap|'
interessant. Sie sorgen daf\"ur, da\ss{} der Text, auf den sich die Befehle
beziehen folgenderma\ss{}en verh\"alt. Das Makro `\verb|\rlap|' gibt den
zugeh\"origen Text aus, nimmt aber keinen Platz weg. Genauer, es wird
eine Box der Breite Null erzeugt, an den sich der zugeh\"orige Text
anschlie\ss{}t. Man kann das auch so verstehen, als w\"urde der Text
ausgegeben, und dann um die Breite des Textes zur\"uckgegangen. Somit
ist das Zeichen: `$\neq$' auch produzierbar mit dem Befehl:
`\verb|\rlap=/|'. Analog arbeitet der Befehl `\verb|\llap|' nur, da\ss{}
hier der Text sozusagen {\em nach links} ausgegeben wird. Das obige
Zeichen lie\ss{}e sich also auch mit. `\verb|/\llap=|' erzeugen.
\chapter{Zusammensetzung der Seiten}
\section{Modi}
\TeX\ kennt sechs verschiedene
Bearbeitungsmodi:\index{Horizontal!Modus}
\begin{itemize}
\item Den {\em vertikalen Modus}, in dem die \"ausserste vertikale
\index{Liste!Vertikal}Liste
bearbeitet wird, die zur \index{Ausgabe}Ausgabe der Seiten f\"uhrt.
\item Den {\em internen vertikalen Modus} zum Aufbau von
`\verb|\vbox|'en.\index{zzzvbox@{\tt\char92 vbox}}
\item Den {\em horizontalen Modus} um aus Zeilen Paragraphen
aufzubauen.
\item Den {\em eingeschr\"ankten horizontalen Modus} zum Aufbau von
\index{zzzhbox@{\tt\char92 hbox}}`\verb|\hbox|'en.
\item Den {\em mathematischen Modus} mit dem mathematische Formeln in
eine horizontale \index{Liste!Horizontal}Liste eingef\"ugt werden.
\item Den {\em abgesetzten mathematischen Modus}, mit dem Formeln in
eine eigene Zeile gesetzt werden, die den momentanen Paragraphen
unterbricht.
\end{itemize}\index{Formel!abgesetzte}
Normalerweise braucht sie der Modus, in dem \TeX\ sich gerade befindet
nicht zu interessieren, h\"ochstens bei der \index{Fehler!Suche}
Fehlersuche kann es
hilfreich sein die verschiedenen Modi zu kennen. Grunds\"atzlich reicht
es zu wissen, da\ss{} es die drei Grundtypen: vertikaler, horizontaler und
mathematischer \index{Modus!Mathematisch}
Modus gibt. Manche Befehle haben, je nach momentanem
Modus unterschiedliche Wirkung. Der Befehl \index{zzzkern@{\tt\char92 kern}}
`\verb|\kern|' bewirkt
z.B.\ einen vertikalen resp.\ horizontalen \index{Zwischenraum}
Zwischenraum, je nach
Modus, in dem er angetroffen wird.

Betrachten wir mal wieder unser Lieblingsbeispiel: die kleine Story.
Zu Bearbeitungsbeginn befindet sich \TeX\ immer im vertikalen Modus.
Die \index{Abstand!Anweisung}
Abstandsanweisung, sowie der \index{Horizontal!Strich}
horizontale Strich, werden einfach an
die vertikale \index{Liste!Horizontal}
Liste angef\"ugt. Auch die \index{Box!Centerline}
Box, die von \index{zzzcenterline@{\tt\char92 centerline}}
`\verb|\centerline|' erzeugt wird, wird einfach an die Liste
angeh\"angt, sie erfordert aber etwas mehr Vorarbeit. Die Box, die `{\tt
A SHORT STORY}' enth\"alt wird im eingeschr\"ankten horizontalen Modus
erstellt. Die beiden folgenden Paragraphen werden dann im normalen
horizontalen Modus erstellt.

Wenn sich \TeX\ im vertikalen Modus befindet, sorgt jeder auftretende
Buchstabe f\"ur ein Umschalten in den horizontalen Modus. Das gilt aber
auch f\"ur die Befehle: `\verb|\char|', \index{zzzaccent@{\tt\char92 accent}}
`\verb|\accent|',\index{zzzhskip@{\tt\char92 hskip}}
`\verb|\hskip|', `\verb*|\ |' und
\index{zzzvrule@{\tt\char92 vrule}}`\verb|\vrule|' oder die Umschaltung
in den mathematischen Modus (\$). Ausserdem l\"a\ss{}t sich mit den
Befehlen: \index{zzzindent@{\tt\char92 indent}}
`\verb|\indent|' oder \index{zzznoindent@{\tt\char92 noindent}}
`\verb|\noindent|' in den horizontalen
Modus umschalten. Diese Befehle werden zu Beginn eines Paragraphen
gebraucht um eine \index{Einrueckung@Einr\"uckung}
Einr\"uckung der Gr\"o\ss{}e \index{zzzparindent@{\tt\char92 parindent}}
`\verb|\parindent|' zu
bewirken, oder eben nicht.

Der horiontale Modus wird beendet durch:
\index{zzzpar@{\tt\char92 par}}
\begin{enumerate}
\item Zwei \index{Leer!Zeile}Leerzeilen,
\item den Befehl: `\verb|\par|'
\item einen inkompatibelen Befehl wie z.B.:\
\index{zzzvskip@{\tt\char92 vskip}}`\verb|\vskip|', der im
horizontalen Modus keinen Sinn erg\"abe.
\end{enumerate}

Der mathematische Modus wird durch ein \index{Dollarzeichen}
Dollarzeichen (\$) eingeleitet
und solange beibehalten, bis wieder ein schlie\ss{}endes Dollarzeichen
gefunden wird. Die Dollarzeichen wirken also wie eine
\index{Klammer!Dollarzeichen als}Klammer.

Mit einem doppelten \index{Dollarzeichen!Doppelt}
Dollarzeichen wird in den
abgesetzten\index{Formel!abgesetzte}
mathematischen Modus geschaltet (\$\$). Der laufende Paragraph wird
unterbrochen, die mathematische Formel in einer eigenen Zeile gesetzt
und nach dem schlie\ss{}enden doppelten Dollarzeichen wird der Paragraph
wieder fortgesetzt. So ergibt:
\index{zzzapprox@{\tt\char92 approx}}
\begin{verbatim}
Die Nummer $$\pi \approx 3.1415926536$$ ist wichtig.
\end{verbatim}
Die Nummer $$\pi \approx 3.1415926536$$ ist wichtig.

Im vertikalen Modus werden \index{Leer!Zeichen}
Leerzeichen und -zeilen ignoriert. Eine
Ausnahme bildet nur das kontrollierte Leerzeichen (`\verb*|\ |'), das
einen neuen Paragraphen einleitet, nachdem zuvor die
\index{Einrueckung@Einr\"uckung}Einr\"uckung
vorgenommen wurde.

Um einen Text zu beenden sollte man am besten den Befehl:
\index{zzzbye@{\tt\char92 bye}}`\verb|\bye|'
verwenden, der eine \index{Abkuerzung@Abk\"urzung}
Abk\"urzung f\"ur die Befehlsfolge: \index{zzzvfill@{\tt\char92 vfill}}
`\verb|\vfill|'
(auff\"ullen der letzten Seite mit Leerplatz),
\index{zzzeject@{\tt\char92 eject}}`\verb|\eject|'
(\index{Ausgabe}Ausgabe
der Seite) und `\verb|\end|' (Beendigung der Arbeit von \TeX)
darstellt.

Der interne vertikale Modus unterscheidet sich nur geringf\"ugig vom
normalen vertikalen Modus und der eingeschr\"ankte
\index{Horizontal!Modus}horizontale Modus
noch weniger vom normalen horizontalen Modus. Dennoch gibt es kleine
Unterschiede, da die Zielsetzung ein wenig anders ist.

\TeX\ schaltet h\"aufig zwischen den Modi hin und her. Dabei ist der
\"ausserste Modus immer der vertikale Modus. Wird er unterbrochen, dann
wird nach der Unterbrechung der vertikale Modus wiederaufgenommen.
Alle Modi k\"onnen getestet werden, wenn man folgenden \index{File!Test}
File eingibt:
\index{zzznoindent@{\tt\char92 noindent}}
\index{zzzshowlists@{\tt\char92 showlists}}
\index{zzztracingcommands@{\tt\char92 tracingcommands}}
\index{zzzvbox@{\tt\char92 vbox}}
\begin{verbatim}
\tracingcommands=1
\hbox{
$
\vbox{
\noindent$$
x\showlists
$$}$}\bye
\end{verbatim}
Der erste Befehl in diesem File bewirkt, da\ss{} alle weiteren Aktionen
genauer mitprotokolliert werden. Der Logfile zu dieser Datei sieht
damit folgenderma\ss{}en aus:
\index{Display!Mode}
\index{zzzvbox@{\tt\char92 vbox}}
\begin{verbatim}
{vertical mode: \hbox}
{restricted horizontal mode: blank space}
{math shift character $}
{math mode: blank space}
{\vbox}
{internal vertical mode: blank space}
{\noindent}
{horizontal mode: math shift character $}
{display math mode: blank space}
{the letter x}
\end{verbatim}
Versuchen sie diese \index{Ausgabe}
Ausgabe selber zu verstehen, so schwierig, wie es
auf den ersten Blick aussieht ist es gar nicht. Denken sie nur daran,
da\ss{} \TeX\ die \index{Markierung!Zeilenende}
Zeilenendemarkierung in ein \index{Leer!Zeichen}
Leerzeichen umwandelt. Der\index{zzzshowlists@{\tt\char92 showlists}}
Befehl `\verb|\showlists|' soll hier zun\"achst ignoriert werden.

\section{Wie macht \TeX\ aus Zeilen Paragraphen?}
\subsection{Das Vorgehen und Sonderf\"alle}
Grunds\"atzlich werden Zeilenumbr\"uche innerhalb eines Paragraphen immer
f\"ur den ganzen Paragraphen berechnet. Die Vorgehensweise ist dabei
so, da\ss{} die Umbr\"uche gesucht werden, bei denen die \index{Badness}
{\em badness} f\"ur
den ganzen Paragraphen am geringsten ist. \TeX\ kann nat\"urlich nicht
f\"ur {\em psychologisch} schlechte Umbr\"uche verantwortlich gemacht
werden. In ganz schlimmen F\"allen\footnote{Ich erinnere hier immer
wieder gerne an das Beispiel aus der {\em Context} Dokumentation: Man
sollte Urinstinkt nicht zu: Urin -- stinkt trennen.} mu\ss{} man wohl per
Hand nachhelfen. Es gibt aber f\"ur viele F\"alle eine einfache
M\"oglichkeit Zeilenumbr\"uche zu verhindern: Die Tilde (`\verb|~|').

Die Tilde ersetzt ein \index{Leer!Zeichen}
Leerzeichen, \TeX\ setzt an die Stelle auch ein
Leerzeichen, bricht aber an dieser Stelle die Zeile nicht um. Man
sollte die Tilde z.B.\ in folgenden F\"allen
benutzen:\index{Kapitel!Nummer}

1. \index{Referenz}Referenzen in Texten:
\begin{verbatim}
Kapitel~1          Satz~12
Anhang~A           Figur~13
\end{verbatim}

2. Zwischen Vornamen und zwischen den Teilen von
Nachnamen:\index{Gogh}\index{Knuth. D.}
\begin{verbatim}
Donald~E. Knuth
Vincent van~Gogh
Fritz~der~Gro\ss{}e
\end{verbatim}

3. Zwischen mathematischen Symbolen und ihren Bezeichnern:
\begin{verbatim}
Breite~$b$         H\"ohe~$h$
\end{verbatim}

4. Zwischen Symbolen in Serie:
\index{zzzldots@{\tt\char92 ldots}}
\begin{verbatim}
1,~2 oder~3
1,~2 $\ldots$~$n$
\end{verbatim}
Auch \index{Box!Hbox}\index{zzzhbox@{\tt\char92 hbox}}
`\verb|\hboxen|' werden von \TeX\ nicht gebrochen. Wenn sie z.B.\
einen Seitenbereich angeben wollen, dann schreiben sie am besten:
`\verb|Seiten \hbox{1--3}|'. Seien sie aber bitte mit dieser Variante
vorsichtig. Es ist z.B.\ besser `\verb|Kapitel~12|', als
`\verb|\hbox{Kapitel 12}|' zu schreiben, da im ersten Falle das
\index{Leer!Zeichen}
Leerzeichen noch gestreckt oder geschrumpft werden kann und au\ss{}erdem
das Wort `\verb|Kapitel|' noch getrennt werden kann.

Einen Zeilenumbruch an einer bestimmten Stelle erzwingen
sie mit dem Befehl \index{zzzbreak@{\tt\char92 break}}`\verb|\break|'.
Er\hfil kann\hfil allerdings\hfil dazu\hfil f\"uhren,\break da\ss{} ihre
Zeile etwas in die L\"ange gezogen wird. Wenn sie wollen, da\ss{} \TeX\
zun\"achst die Zeile mit \index{Leer!Platz}
leerem Platz auff\"ullt,\hfil\break dann
benutzen sie den Befehl \index{zzzhfil@{\tt\char92 hfil}}
`\verb|\hfil\break|' um genau das zu erreichen.

Mitunter, besonders in Gedichten wollen sie vielleicht, da\ss{} die Zeilen
immer da umgebrochen werden, wo sie auch in der \index{Eingabe!Umbruch}
Eingabe umgebrochen
sind. Sie k\"onnen das erreichen, indem sie hinter jede Zeile ein
\index{zzzpar@{\tt\char92 par}}
`\verb|\par|' schreiben, einfacher ist es allerdings, wenn sie
innerhalb einer \index{Gruppe}
Gruppe den Befehl \index{zzzobeylines@{\tt\char92 obeylines}}
`\verb|\obeylines|' benutzen. Auch
hier wieder ein Beispiel aus dem Original:
\index{zzzsmallskip@{\tt\char92 smallskip}}
\begin{verbatim}
{\obeylines\smallskip
Roses are red,
\quad Violets are blue;
Rhymes can be typeset
\quad With boxes and glue.
\smallskip}
\end{verbatim}
\subsection{Wie gehts genau?}
\subsubsection{Die horizontale Liste}
Wie macht \TeX\ es denn nun, wenn es einen Paragraphen in Zeilen
bricht? Zun\"achst ist ein Paragraph f\"ur \TeX\ nur eine lange Reihe von
W\"ortern. Genauer gesagt nicht aus W\"ortern, sondern aus einer Reihe von
Elementen, die folgendes sein k\"onnen:
\begin{enumerate}\index{Box}
\item Ein Box (Also ein Zeichen, eine \index{Ligatur}
Ligatur, eine Linie, eine hbox
oder eine vbox).
\item Eine m\"ogliche Trennung (wird gleich erkl\"art).
\item Ein ``WasDenn'' (kommt sp\"ater).
\item Vertikales Material (von \index{zzzmark@{\tt\char92 mark}}
`\verb|\mark|', \index{zzzvadjust@{\tt\char92 vadjust}}`\verb|\vadjust|' oder
\index{zzzinsert@{\tt\char92 insert}}`\verb|\insert|').
\item Etwas \index{Leim}\index{zzzleaders@{\tt\char92 leaders}}
Leim (oder `\verb|\leaders|', wie auch sp\"ater noch erkl\"art
wird).
\item Ein Kern (so was wie Leim, der aber nicht gestreckt oder
geschrumpft werden kann).
\item Strafpunkte (die anzeigen, ob hier ein Zeilenumbruch g\"unstig
oder ung\"unstig ist).
\item Ein Zeichen, da\ss{} den Beginn oder das Ende des mathematischen
Modus anzeigt.
\end{enumerate}
Die letzten vier M\"oglichkeiten hei\ss{}en \"ubrigens {\em entfernbar}, da
sie bei einem Zeilenumbruch ver\"andert werden k\"onnen, oder sogar ganz
wegfallen.
\subsubsection{Die m\"ogliche Trennung}
Immer, wenn ein Wort auf eine besondere Art getrennt werden soll, dann
gibt es in \TeX\ die M\"oglichkeit diese Trennung gesondert anzugeben.
Besonders in der deutschen Sprache ist diese M\"oglichkeit wichtig, da
im Deutschen h\"ochst seltsame Trennregeln existieren. Aber auch bei
ungew\"ohnlichen Trennungen (z.B.\ Trennung von `st'), kann dieser
Befehl benutzt werden. Die allgemeine Form lautet dabei
folgenderma\ss{}en:
\index{zzzdiscretionary@{\tt\char92 discretionary}}
\begin{verbatim}
\discretionary{<Vor-Bruch Text>}{<Nach-Bruch Text>}{<Nicht-Bruch Text>}
\end{verbatim}
Im Deutschen wird z.B.\ jedes `ck' in `k-k' getrennt. Sollte es einmal
wichtig werden, dann schreiben sie doch einfach:
\begin{verbatim}
dr\"u\discretionary{k-}{k}{ck}en
\end{verbatim}
wenn sie das Wort `dr\"ucken' richtig trennen wollen. So kompliziert
braucht es aber nur in solchen Sonderf\"allen zu sein. F\"ur die m\"ogliche
Trennung:
\begin{verbatim}
\discretionary{-}{}{}
\end{verbatim}
gibt es auch die \index{Abkuerzung@Abk\"urzung}Abk\"urzung: `\verb|\-|'
\TeX\ macht \"ubrigens eigentlich genau das Gleiche, wenn es Worte trennen
will. Es werden einfach in ein Wort derartige m\"ogliche Trennungen
eingef\"ugt. Aus dem Wort `\verb|Trennung|' wird z.B.\ `\verb|Tren\-nung|'.
\subsubsection{Die Berechnung der Trennungen}
Zun\"achst versucht \TeX\ einen Paragraphen {\em ohne} Trennungen zu
setzen. Die dabei auftretenden Streckungen und Stauchungen m\"ussen
dabei unter dem Wert von
\index{zzzpretolerance@{\tt\char92 pretolerance}}
`\verb|\pretolerance|' bleiben. Erst im
zweiten Durchlauf, sollte er denn n\"otig sein, werden W\"orter getrennt
und dann m\"ussen die Werte von \index{Badness}
badness unter `\verb|\tolerance|'\index{zzztolerance@{\tt\char92 tolerance}}
bleiben. Voreingestellt sind \"ubrigens:
\begin{verbatim}
\pretolerance=100
\tolerance=200
\end{verbatim}

Zun\"achst einmal: Wo darf \"uberhaupt getrennt werden? Es darf an den
folgenden Stellen im Text getrennt werden:
\begin{enumerate}
\item Bei \index{Leim}Leim, sofern dieser Leim nicht unmittelbar hinter einem
entfernbaren Element steht und er nicht in einer mathematischen Formel
auftritt. Gebrochen wird am linken
\index{Rand!Leimstueck@Rand!Leimst\"uck}Rand des Leimst\"ucks.
\item Bei Kern, vorausgesetzt, da\ss{} diesem Kern etwas Leim folgt und er
nicht in einer mathematischen Formel steht.
\item An einem Mathe-Ende-Zeichen, dem sofort Leim folgt.
\item Bei Strafpunkten, die eventuell automatisch eingef\"ugt wurden.
\item Bei einer m\"oglichen Trennung.
\end{enumerate}
Zu jeder m\"oglichen Trennung geh\"ort eine {\em \"asthetische
\index{Strafe!Aesthetische@Strafe!\"Asthetische}Strafe}. In
den F\"allen: 1,~2 und~3 ist die Strafe 0, im vierten Fall ist die
Strafe exakt angegeben, im f\"unften Fall ist es etwas komplizierter.
Ist der Text vor der Trennung nicht-leer, dann wird der Wert von
\index{zzzhyphenpenalty@{\tt\char92 hyphenpenalty}}
`\verb|\hyphenpenalty|' (voreingestellt mit 50), ist er leer der Wert
von \index{zzzexhyphenpenalty@{\tt\char92 exhyphenpenalty}}
`\verb|\exhyphenpenalty|' (voreingestellt mit 50) verwendet.

Wenn sie explizit an einer Stelle in ihrem Text angeben:
\index{zzzpenalty@{\tt\char92 penalty}}
`\verb|\penalty 100|', dann kann dort zwar noch gebrochen werden, es
tritt aber eine Strafe von 100 auf\footnote{\index{Penalty}Penalty=Strafpunkte}.
Wenn sie hingegen `\verb|\penalty-100|' schreiben, dann geben sie damit an,
da\ss{} an dieser Stelle besonders gut gebrochen werden kann, da {\em
negative} Strafpunkte die Trennung f\"ordern. Eine Strafe von $10000$ ist
z.B. so hoch, da\ss{} \TeX\ hier nie eine Zeile umbrechen wird, eine
Strafe von $-10000$ so niedrig, da\ss{} \TeX\ hier immer die Zeile
umbrechen wird. Der Befehl \index{zzznobreak@{\tt\char92 nobreak}}
`\verb|\nobreak|' ist nur eine
\index{Abkuerzung@Abk\"urzung}Abk\"urzung
f\"ur `\verb|\penalty 10000|' und die Tilde wird einfach als:
`\verb*|\nobreak\ |' realsiert.

Nach dem Zeilenumbruch werden alle entfernbaren Elemente entfernt und
zwar bis zum ersten nicht-entfernbaren Element, oder bis zum n\"achsten
Zeilenumbruch.

Die genaue Berechnung der \index{Badness}
badness einer Zeile braucht hier nicht zu
interessieren, sie ist ungef\"ahr das Hundertfache der dritten
\index{Potenz!Badnessberechnung}Potenz
der Streckung oder \index{Schrumpf!Berechnung}
Schrumpfung der Zeile. Wenn eine Zeile z.B.\ 10pt
zu Streckung zur Verf\"ugung hat, die Zeile tats\"achlich um 9pt gestreckt
werden mu\ss{}, um auf das notwendige Ma\ss{} zu kommen, dann ist die badness:
\index{zzzapprox@{\tt\char92 approx}}
$100\times (9/10)^3\approx 73$.

Zeilen werden nun entsprechend ihrer badness klassifiziert. Ist die
badness kleiner als 13, dann gilt die Zeile als {\em anst\"andig}, liegt
sie zwischen 13 und 99, dann hei\ss{}t sie {\em dicht}, wenn sie
geschrumpft wurde, {\em lose}, wenn sie gestreckt wurde. Ist die
badness gr\"o\ss{}er als 100, dann hei\ss{}t die Zeile {\em sehr dicht} oder
{\em sehr lose}. Zwei Zeilen sind optisch inkompatibel, wenn mehr als
ein Bereich \"ubersprungen wird, wenn z.B.\ eine anst\"andige Zeile vor
einer sehr losen kommt.

\TeX\ berechnet nun f\"ur eine Folge von Zeilenumbr\"uchen die
\index{Minuspunkte}Minuspunkte
des Paragraphen, indem es die Minuspunkte aller Zeilen aufaddiert. Die
Minuspunkte einer Zeile ergeben sich dabei nach folgender
\index{Formel!Fuer Minuspunkte@Formel!F\"ur Minuspunkte}Formel
(dabei ist: $d$ die \index{Minuspunkte}
Minuspunkte, $p$ die Strafpunkte, $b$ die badness,
$l$ die sogenannte \index{Strafe!Zeile}
Zeilenstrafe, die von \TeX\ \"ublicherweise mit 10
voreingestellt ist \index{zzzlinepenalty@{\tt\char92 linepenalty}}
(`\verb|\linepenalty|'). Sie kann erh\"oht werden,
wenn man erreichen will, da\ss{} die Paragraphen mit m\"oglichst geringer
Zeilenzahl gesetzt werden sollen):
\index{zzzleq@{\tt\char92 leq}}
\[
d=\left\{\begin{array}{ll}
(l+b)^2+p^2 & \mbox{wenn\quad} 0\leq p <10000 \\
(l+b)^2-p^2 & \mbox{wenn\quad} -10000 < p < 0 \\
(l+b)^2 & \mbox{wenn\quad} p \leq -10000 \end{array} \right. \]
Sind zwei aufeinanderfolgende Zeilen optisch inkompatibel, dann wird
au\ss{}erdem noch der Wert von \index{zzzadjdemerits@{\tt\char92 adjdemerits}}
`\verb|\adjdemerits|' hinzugef\"ugt, enden
zwei aufeinanderfolgende Zeilen mit einer Trennung, dann wird der Wert
von \index{zzzdoublehyphendemerits@{\tt\char92 doublehyphendemerits}}
`\verb|\doublehyphendemerits|' addiert und endet die vorletzte
Zeile eines Paragraphen mit einer Trennung, dann wird noch
\index{zzzfinalhyphendemerits@{\tt\char92 finalhyphendemerits}}
`\verb|\finalhyphendemerits|' addiert. Dabei sind die folgenden Werte
voreingestellt:
\index{zzzadjdemerits@{\tt\char92 adjdemerits}}
\index{zzzdoublehyphendemerits@{\tt\char92 doublehyphendemerits}}
\begin{verbatim}
\adjdemerits=10000
\doublehyphendemerits=10000
\finalhyphendemerits=5000
\end{verbatim}
Die genaue Berechnung der einzelnen Durchl\"aufe l\"a\ss{}t sich
verfolgen, wenn
\index{zzztracingparagraphs@{\tt\char92 tracingparagraphs}}
`\verb|\tracingparagraphs=1|' gesetzt wird,
dessen Erkl\"arung aber hier zu weit f\"uhren w\"urde.

\subsubsection{Die letzte Zeile}
{\parfillskip=0pt Bisher d\"urfte noch nicht klar sein, wieso die letzte
Zeile eines Paragraphen k\"urzer sein kann, als die letzten. Die L\"osung
ist aber ganz einfach: Bevor \TeX\ anf\"angt die besten Zeilenumbr\"uche
zu suchen unternimmt es noch zwei wichtige Dinge: (1) Steht am Ende
eines Paragraphen ein St\"uck \index{Leim}
Leim, so wird dieser entfernt. (2) Werden
drei Befehle an den Paragraphen angeh\"angt: (a)
\index{zzzpenalty@{\tt\char92 penalty}}
`\verb|\penalty10000|',
was einen Zeilenumbruch vor den letzten Befehlen verhindert, (b)
\index{zzzparfillskip@{\tt\char92 parfillskip}}
`\verb|\hskip\parfillskip|', was den \index{Leer!Platz}
leeren Platz am Ende der Zeile
bewirkt, `\verb|\parfillskip|' ist dabei \"ublicherweise als {\tt 0pt
plus 1fil} definiert und (c) der Befehl `\verb|\penalty-10000|', was
den letzten Zeilenumbruch bewirkt. Mit dem Wert f\"ur
`\verb|\parfillskip|' kann \"ubrigens auch gespielt werden. Diese
\index{Absatz}Absatz
wurde mit dem Wert {\tt 0pt} gesetzt, so da\ss{} am Ende des Abschnitts
kein Platz gelassen wird. Das ist allerdings nur bei l\"angeren
Abschnitten problemlos m\"oglich.}

\subsubsection{Lange Paragraphen}
Was tun, wenn ein Paragraph {\em wirklich} lang wird? \TeX\ h\"alt immer
den ganzen Paragraphen im \index{Speicher!Paragraph}
Speicher, und daher kann es vorkommen, da\ss{}
lange Paragraphen den \index{Speicher!Platz}
Speicherplatz, der zur Verf\"ugung steht,
sprengen. Man kann das umgehen, indem man in einen Paragraphen die
Befehlsfolge:
\index{zzznoindent@{\tt\char92 noindent}}
\index{zzzparfillskip@{\tt\char92 parfillskip}}
\index{zzzparskip@{\tt\char92 parskip}}
\begin{verbatim}
{\parfillskip0pt\par\parskip0pt\noindent}
\end{verbatim}
einf\"ugt. Auf die Art wird der Paragraph in kleinere Paragraphen
unterteilt, ohne da\ss{} der Leser es merkt.
\subsection{Feinheiten und Tricks}
\subsubsection{Einr\"uckungen und Flattersatz}
\begin{quote}
Bei dem Satz eines Paragraphen werden au\ss{}er den \"ublichen
Gr\"o\ss{}en auch noch die Werte von \index{zzzrightskip@{\tt\char92 rightskip}}
`\verb|\rightskip|' und
\index{zzzleftskip@{\tt\char92 leftskip}}
`\verb|\leftskip|' ber\"ucksichtigt. Es handelt sich dabei um
\index{Leim}Leimst\"ucke, die vor, bzw.\ nach jeder Zeile eingef\"ugt werden.
Normalerweise sind die Werte einfach Null, aber z.B.\ das Makro
\index{zzznarrower@{\tt\char92 narrower}}
`\verb|\narrower|' erh\"oht die beiden Werte auf den Wert von
\index{zzzparindent@{\tt\char92 parindent}}
`\verb|\parindent|', so da\ss{} alle Zeilen rechts und links einger\"uckt
werden. Dieser Absatz wurde mit dem Makro gesetzt.
\end{quote}\index{Absatz}

{\raggedright Auch das Makro `\verb|\raggedright|' wird mit
`\verb|\rightskip|' realisiert, aber nicht einfach mit:}
\index{zzzraggedright@{\tt\char92 raggedright}}
\index{zzzrightskip@{\tt\char92 rightskip}}
\begin{verbatim}
\rightskip 0pt plus 1fil
\end{verbatim}
{\raggedright wie man vielleicht annehmen k\"onnte, da diese L\"osung dazu
f\"uhren w\"urde, da\ss{} auch ganz kurze Zeilen noch f\"ur \TeX\ akzeptabel
w\"aren. Statt dessen sollte man die Wortabst\"ande konstant w\"ahlen und
\index{zzzrightskip@{\tt\char92 rightskip}}
`\verb|\rightskip|' zwar hoch, aber auch nicht {\em zu} hoch setzen.}

\subsubsection{Totale Kontrolle}
\parshape 16
0pc 26pc
0pc 26pc
0pc 24.69pc
0pc 23.51pc
0pc 22.73pc
0pc 22.20pc
0pc 21.85pc
0pc 21.65pc
0pc 21.58pc
0pc 21.65pc
0pc 21.85pc
0pc 22.20pc
0pc 22.73pc
0pc 23.51pc
0pc 24.69pc
0pc 29pc
Es gibt schlie\ss{}lich noch eine {\em sehr} eindrucksvolle M\"oglichkeit
die L\"ange von Zeilen im Text zu beeinflu\ss{}en. Den Befehl
\index{zzzparshape@{\tt\char92 parshape}}
`\verb|\parshape|'. Ihm folgen zun\"achst eine einfache Zahl, und dann
Paare von Werte, die die folgende Bedeutung haben: Die erste Zahl
gibt an, auf wieviele Zeilen sich der Befehl bezieht (in diesem
Paragraphen sind es 16 Zeilen). Bei den folgenden Wertepaaren gibt der
erste Wert an, wieviel die Zeile vom linken
\index{Rand!Einrueckung@Rand!Einr\"uckung}Rand her einger\"uckt werden
soll, und der zweite Wert, wie lang die jeweilige Zeile sein soll. In
diesem Paragraphen mu\ss{}ten also 16 Wertepaare angegeben werden. Ist der
Paragraph k\"urzer, als die angegebene Anzahl von Zeilen, dann werden
die \"uberfl\"u\ss{}igen Angaben einfach ignoriert, ist er l\"anger, dann gilt
die Angabe f\"ur die letzte Zeile. Haben sie einen Paragraphen
geschrieben, und wissen nicht, ob ihre Angaben nicht eventuell mit in
den n\"achsten Abschnitt hineinragen, dann k\"onnen sie die letzte Angabe
einfach dadurch unwirksam machen, da\ss{} sie `\verb|\parshape 0|'
eingeben. So schalten sie wieder auf die urspr\"unglichen Werte von
Zeilenl\"ange. Eine wichtige \index{Abkuerzung@Abk\"urzung}
Abk\"urzung f\"ur den Befehl `\verb|\parshape|'
stellt \TeX\ mit den Befehlen \index{zzzhangindent@{\tt\char92 hangindent}}
`\verb|\hangindent|' und
\index{zzzhangafter@{\tt\char92 hangafter}}
`\verb|\hangafter|' bereit. Diese werden viel \"ofters gebraucht, als
der eigentliche Befehl `\verb|\parshape|', der nur in wirklich
seltenen F\"allen zur Anwendung kommt.

\hangafter=-3
\hangindent=\parindent
\noindent Mit den Befehlen \index{zzzhangindent@{\tt\char92 hangindent}}
`\verb|\hangindent|' und
\index{zzzhangafter@{\tt\char92 hangafter}}
`\verb|\hangafter|' k\"onnen an den vier {\em Ecken} eines
\index{Absatz}Absatzes
rechteckige Fl\"achen freigelassen werden. `\verb|\hangafter|' dient
dabei der Angabe, wieviele Zeilen von der
\index{Einrueckung@Einr\"uckung}Einr\"uckung betroffen sind,
und zwar sind die Zeilen $n+1$, $n+2 \ldots$ betroffen, wenn $n$ ein
positiver Wert von `\verb|\hangafter|' ist, ist $n$ negativ, dann sind
die Zeilen $1$, $2$, $|n|$ betroffen. Mit `\verb|\hangindent|' wird
nun angegeben, wie gro\ss{} die Einr\"uckung ist. Bei positivem Wert, wird
links, andernfalls rechts einger\"uckt. Dieser Paragraph wurde mit:
\index{zzzhangafter@{\tt\char92 hangafter}}
\index{zzzhangindent@{\tt\char92 hangindent}}
\index{zzzparindent@{\tt\char92 parindent}}
\begin{verbatim}
\hangafter=-3
\hangindent=\parindent
\end{verbatim}
gesetzt.
Die beiden Befehle werden \"ubrigens f\"ur die Makros `\verb|\item|' und
`\verb|\itemitem|' benutzt, die f\"ur \index{Aufzaehlung@Aufz\"ahlung}
Aufz\"ahlungen zust\"andig sind. Wenn
sie z.B. schreiben:\index{Ebene!Aufzaehlung@Ebene!Aufz\"ahlung}
\begin{verbatim}
\item{1.} Dies ist die erste Aufz\"ahlungsebene
\itemitem{(a)} Hier nun die Unterebene,
\itemitem{(b)} und ein weiterer Eintrag in die Unterebene
\item{2.} Und zum Schlu\ss{} noch ein Eintrag in der h\"oheren Eintragebene,
um das Ganze etwas interessanter zu gestalten.
\end{verbatim}
dann erhalten sie:

\begin{enumerate}
\item Dies ist die erste Aufz\"ahlungsebene
\begin{enumerate}
\item Hier nun die Unterebene,
\item und ein weiterer Eintrag in die Unterebene
\end{enumerate}
\item Und zum Schlu\ss{} noch ein Eintrag in der h\"oheren Eintragebene,
um das Ganze etwas interessanter zu gestalten.
\end{enumerate}

Die Werte von \index{zzzparshape@{\tt\char92 parshape}}
`\verb|\parshape|', \index{zzzhangindent@{\tt\char92 hangindent}}
`\verb|\hangindent|' und
\index{zzzhangafter@{\tt\char92 hangafter}}
`\verb|\hangafter|' werden \"ubrigens am Ende eines Paragraphen
zur\"uckgesetzt.
\subsubsection{Die H\"ohe von Paragraphen}
Eine \index{Formel!abgesetzte}
abgesetzte Formel in einem Paragraphen nimmt den Platz von drei
Zeilen ein. Wenn ihr Paragraph also vier Zeilen vor und zwei Zeilen
nach der Formel hat, dann ist der gesamte Paragraph $4+3+2=9$ Zeilen
hoch. Intern wird die Anzahl der Zeilen, die schon bearbeitet wurden
in der Variablen \index{zzzprefgraf@{\tt\char92 prefgraf}}
`\verb|\prefgraf|' festgehalten. Dies k\"onnen sie dazu
ausnutzen, da\ss{} sie den Wert bei h\"oheren Formeln \"andern. Ist ihre
Formel z.B.\ sehr hoch, dann k\"onnten sie in dem vorigen Beispiel vor
den beiden letzten Zeilen den Wert von `\verb|\prefgraf|' auf 8
setzen, so da\ss{} \TeX\ denkt, es h\"atte insgesamt einen Paragraphen mit
10 Zeilen gesetzt.

Ein weiteres Mittel gibt es um die H\"ohe von Paragraphen zu
beeinflu\ss{}en. Dazu verwenden sie den Befehl
\index{zzzlooseness@{\tt\char92 looseness}}
`\verb|\looseness|'. Setzen sie diesen Wert
z.B.\ auf 1, dann macht \TeX\ den Paragraphen um eine Zeile l\"anger,
als es ihn normalerweise machen w\"urde. Mit negativen Werten kann ein
Paragraph auch k\"urzer gemacht werden. Der Wert von `\verb|\looseness|'
wird auch am Ende des Paragraphen zur\"uckgesetzt.

\subsection{Der Satz von Paragraphen}
Schlie\ss{}lich, wenn alle Berechnungen zum Zeilenumbruch beendet sind,
m\"ussen die Zeilen noch in die vertikale Liste der Seite geschrieben
werden. Dabei geht \TeX\ folgenderma\ss{}en vor: Unmittelbar vor dem
Paragraphen wird ein spezielles \index{Leim}Leimst\"uck in die vertikale Liste
eingef\"ugt. Den \index{Abstand!Paragraphen}
Abstand zweier aufeinanderfolgender Paragraphen:
\index{zzzparskip@{\tt\char92 parskip}}
`\verb|\parskip|'. Plain \TeX\ setzt diesen Abstand auf:
\begin{verbatim}
\parskip=0pt plus 1pt
\end{verbatim}
Danach folgen die einzelnen Zeilen, die schon umgebrochen wurden.
Zwischen die Zeilen werden ggf.\ noch besondere Strafpunkte
geschrieben, die sp\"ater f\"ur den Seitenumbruch gebraucht werden.
Insbesondere wird zwischen die ersten beiden Zeilen und unmittelbar
vor die letzte Zeile eine besondere \index{Strafe!Zeile}
Strafe geschrieben, um
\index{Schusterjunge}``Schusterjungen'' und \index{Hurenkind}
``Hurenkinder'' zu vermeiden\footnote{Mit Schusterjungen und Hurenkindern
bezeichnen die Drucker einzelne Zeilen eines Abschnittes, die durch Seitenumbruch
vom Rest des Abschnittes getrennt wurde.}. Normalerweise wird zwischen die Zeilen
der Wert von \index{zzzinterlinepenalty@{\tt\char92 interlinepenalty}}
`\verb|\interlinepenalty|'
geschrieben, handelt es sich um den \index{Zwischenraum}
Zwischenraum zwischen erster und
zweiter Zeile, dann wird der Wert von \index{zzzclubpenalty@{\tt\char92 clubpenalty}}
`\verb|\clubpenalty|' addiert.
Handelt es sich um den \index{Zwischenraum}
Zwischenraum vor der letzten Zeile, dann wird
\index{zzzwidowpenalty@{\tt\char92 widowpenalty}}
`\verb|\widowpenalty|' eingef\"ugt, wenn keine \index{Formel!abgesetzte}
abgesetzte Formel davor
stand, sonst \index{zzzdisplaywidowpenalty@{\tt\char92 displaywidowpenalty}}
`\verb|\displaywidowpenalty|'. Fand in der letzten Zeile
eine Trennung statt, dann wird auch noch
\index{zzzbrokenpenalty@{\tt\char92 brokenpenalty}}
`\verb|\brokenpenalty|'
addiert. Die Werte werden in Plain \TeX\ folgenderma\ss{}en gesetzt:
\index{zzzdisplaywidowpenalty@{\tt\char92 displaywidowpenalty}}
\index{zzzinterlinepenalty@{\tt\char92 interlinepenalty}}
\begin{verbatim}
\interlinepenalty=0
\clubpenalty=150
\widowpenalty=150
\displaywidowpenalty=50
\brokenpenalty=100
\end{verbatim}

Neben den \index{Horizontal!Linie}
horizontalen Linien kann man auch aus einem Paragraphen
heraus andere Dinge in die vertikale Liste schreiben. Dazu dienen die
Befehl \index{zzzinsert@{\tt\char92 insert}}
`\verb|\insert|', \index{zzzmark@{\tt\char92 mark}}
`\verb|\mark|' und \index{zzzvadjust@{\tt\char92 vadjust}}`\verb|\vadjust|'. Die
ersten beiden werden sp\"ater beschrieben, der dritte f\"ugt das
\index{Argument}Argument
des Befehls unmittelbar nach der Zeile, in der der Befehl steht, in
die vertikale Liste ein. \index{zzzvadjust@{\tt\char92 vadjust}}
\vadjust{\kern2pt}Hier habe ich z.B.\ mit dem
Befehl `\verb|\vadjust{\kern2pt}|' einen zus\"atzlichen
\index{Zwischenraum}Zwischenraum in
den Paragraphen eingef\"ugt.
\subsection{Die restlichen Befehle}
\subsubsection{Ausf\"uhrungen f\"ur jeden Paragraphen}
Der Befehl
\index{zzzeverypar@{\tt\char92 everypar}}`\verb|\everypar|' gestattet es einen Befehl zu Beginn {\em
jedes} Paragraphen auszuf\"uhren. In der einfachsten Form sieht der
Befehl folgenderma\ss{}en aus: `\verb|\everypar{A}|'. Jedem Paragraphen
w\"urde nun ein `A' vorangestellt. Nicht sehr sinnvoll, aber denken sie
einmal daran, wie in der \LaTeX\ Umgebung `itemize' die
\index{Aufzaehlung!Markierung@Aufz\"ahlung!Markierung}
\index{Markierung!Aufzaehlung@Markierung!Aufz\"ahlung}
Aufz\"ahlungsmarkierung realisiert wird.
\subsubsection{Leere Zeilen}
Einen leeren Paragraphen erhalten sie mit der Befehlsfolge
\index{zzznoindent@{\tt\char92 noindent}}
`\verb|\noindent\par|', wenn `\verb|\everypar|' leer ist. Es wird dann
nur \index{zzzparskip@{\tt\char92 parskip}}
`\verb|\parskip|' in die vertikale Liste eingef\"ugt.
\subsubsection{Das letzte Hilfsmittel}
Wenn \"uberhaupt nichts mehr geht, und sie sich vor lauter
\index{Fehler!Meldung}
Fehlermeldungen nicht mehr retten k\"onnen, dann benutzen sie den Befehl
\index{zzzemergencystretch@{\tt\char92 emergencystretch}}
`\verb|\emergencystretch|'. Dieser Befehl l\"a\ss{}t auch ungew\"ohnlich gro\ss{}e
Streckungen zu. Ihr Text wird dann allerdings nicht mehr so gut
aussehen.
\section{Wie macht \TeX\ aus Paragraphen Seiten?}
\subsection{Der Normalfall}
Sie k\"onnen es \TeX\ einfach machen die besten Stellen f\"ur den
Seitenumbruch zu finden, wenn sie viele abgesetze Formeln in ihrem
Text verwenden, oder wenn sie oft die Befehl
\index{zzzsmallskip@{\tt\char92 smallskip}}`\verb|\smallskip|',
\index{zzzmedskip@{\tt\char92 medskip}}
`\verb|\medskip|' oder \index{zzzbigskip@{\tt\char92 bigskip}}
`\verb|\bigskip|' verwenden. Haben die diese
M\"oglichkeit nicht, und sie sind mit dem Umbruchalgorithmus von \TeX\
nicht einverstanden, dann k\"onnen sie auch jederzeit selbst einen
Seitenumbruch bewirken, indem sie die Befehle
\index{zzzeject@{\tt\char92 eject}}\index{zzzvfill@{\tt\char92 vfill}}
`\verb|\vfill\eject|'
verwenden. Die Verwendung von `\verb|\eject|' alleine ist nicht
unbedingt zu empfehlen, dann dann die Zeilen der Seite auf das
erforderliche Ma\ss{} auseinandergezogen w\"urden. Der Seitenumbruch
geschieht sehr \"ahnlich, dem Zeilenumbruch, der im letzten Paragraphen
besprochen wurde. Aus \index{Speicher!Platz}
Speicherplatzgr\"unden wird allerdings nur {\em
lokal} optimiert. Es wird also nicht der gesamte Text im Speicher
gelassen, bevor die besten Stellen f\"ur den Seitenumbruch gesucht
werden.
\subsection{Die gehts genau?}
Die Inhalte einer Seite, werden zun\"achst auch einfach als Liste, hier
aber als vertikale Liste repr\"asentiert. In dieser Liste k\"onnen
folgende Elemente auftreten:
\begin{enumerate}\index{Box}\index{Einfuegung@Einf\"ugung}
\item Eine Box (hier eine hbox, eine vbox oder eine rulebox).
\item Ein ``WasDenn'' (kommt wieder sp\"ater).
\item Eine Markierung (wird auch sp\"ater erkl\"art).
\item Eine Einf\"ugung (auch das erst sp\"ater).
\item Etwas \index{Leim}Leim (oder \index{zzzleaders@{\tt\char92 leaders}}`\verb|\leaders|').
\item Ein Kern.
\item Eine Strafe.
\end{enumerate}
Die drei letzten Elemente sind dabei wieder {\em entfernbare}
Elemente, entsprechend den Ausf\"uhrungen im letzen Abschnitt. Es kann
ebenfalls nicht an beliebiger Stelle eine Seitenumbruch erfolgen, das
geht nur an:
\begin{enumerate}
\item Bei einem \index{Leim}Leimst\"uck, vorausgesetzt, da\ss{} dem Leim ein
nicht-entfernbares Element unmittelbar vorangeht.
\item Bei einem Kern, vorausgesetzt, dem Kern folgt sofort etwas Leim.
\item Bei Strafpunkten.
\end{enumerate}
Die Regeln f\"ur Strafpunkte entsprechen dabei wieder denen f\"ur den
Zeilenumbruch. Im letzten Abschnitt wurde ja auch schon erl\"autert, an
welchen Stellen einer vertikalen Liste Strafpunkte auftreten k\"onnen.

Den Befehlen `\verb|\small-|', `\verb|\med-|' und \index{zzzbigskip@{\tt\char92 bigskip}}
`\verb|\bigskip|' in
der \index{Horizontal!Liste}horizontalen Liste entsprechen jetzt die Befehle:
\index{zzzsmallbreak@{\tt\char92 smallbreak}}
`\verb|\smallbreak|', \index{zzzmedbreak@{\tt\char92 medbreak}}
`\verb|\medbreak|' und \index{zzzbigbreak@{\tt\char92 bigbreak}}
`\verb|\bigbreak|', mit
denen $-50$, $-100$ oder $-200$ Strafpunkte vergeben werden k\"onnen, um
anzuzeigen, da\ss{} ein Seitenumbruch an einer bestimmten Stelle besonders
gut oder schlecht zu machen ist. Au\ss{}erdem gibt es noch den Befehl
\index{zzzgoodbreak@{\tt\char92 goodbreak}}
`\verb|\goodbreak|', der eine \index{Abkuerzung@Abk\"urzung}
Abk\"urzung f\"ur die Befehle\index{zzzpar@{\tt\char92 par}}
\index{zzzpenalty@{\tt\char92 penalty}}
`\verb|\par\penalty-200|' darstellt. Dieser Befehl sollte am Ende
eines Paragraphen verwendet werden, wenn dort ein Seitenumbruch
erw\"unscht ist, ohne da\ss{} sofort mit dem drastischen Befehl
\index{zzzeject@{\tt\char92 eject}}
`\verb|\eject|' gearbeitet werden soll. Schlie\ss{}lich gibt es noch den
Befehl \index{zzzfilbreak@{\tt\char92 filbreak}}
`\verb|\filbreak|'. Dieser Befehl schlie\ss{}t die Seite ab und
f\"ullt sie mit \index{Leer!Raum}
Leerraum auf, wenn das vertikale Material bis zum
n\"achsten `\verb|\filbreak|' nicht mehr auf die Seite pa\ss{}t.

Das Analog zum Befehl `\verb|\raggedright|' hei\ss{}t
\index{zzzraggedbottom@{\tt\char92 raggedbottom}}
`\verb|\raggedbottom|' und dieser Befehl bewirkt, da\ss{} die Seiten nicht
mehr gleichm\"a\ss{}ig aufgef\"ullt werden, sondern der \index{Abstand!Zeilen}
Zeilenabstand immer
konstant bleibt.

Zur Berechnung des Seitenumbruchs kommt es nun folgenderma\ss{}en. Zu
Beginn kennt \TeX\ die Gr\"o\ss{}en \index{zzzvsize@{\tt\char92 vsize}}
`\verb|\vsize|', sowie
\index{zzzmaxdepth@{\tt\char92 maxdepth}}
`\verb|\maxdepth|'. Aus diesen Werten berechnet sich \TeX\ das {\em
Ziel} der Berechnung: \index{zzzpagegoal@{\tt\char92 pagegoal}}
`\verb|\pagegoal|'.\footnote{Zus\"atzlich wird noch
der Wert von {\tt\char92 topskip} ausgewertet, der angibt, wieviel am
oberen \index{Rand!Oben}
Rand der Seite freigelassen werden soll.} Die aktuelle
\index{Hoehe!Seite@H\"ohe!Seite}H\"ohe der
Seite wird in
\index{zzzpagetotal@{\tt\char92 pagetotal}}
`\verb|\pagetotal|' gespeichert. Nun wird Zeile an Zeile
in die Seite geschrieben, bis die Seite voll ist, also
\index{zzzpagegoal@{\tt\char92 pagegoal}}
`\verb|\pagegoal|' gleich `\verb|\pagetotal|' ist. Das passiert
nat\"urlich nur im Idealfall. W\"urde die Seite mit der letzten vertikalen
\index{Box!Vbox}
Box, die eingef\"ugt werden soll zu voll, dann wird sie sozusagen in
eine Warteposition gestellt, dann wird versucht noch
\index{Einfuegung@Einf\"ugung}Einf\"ugungen auf
der Seite unterzubringen\footnote{Kommt gleich.} bevor sie dann
ausgegeben wird; genauer an die Outputroutine \"ubergeben wird.
Seitenzahlen und \"ahnliches wird erst {\em nach} Fertigstellung der
Seite hinzugef\"ugt.
\subsection{Einf\"ugungen}
\subsubsection{Illustrationen}
\index{Fussnote@Fu\ss{}note}Fu\ss{}noten und Illustrationen werden in \TeX\ als Einf\"ugungen
realisiert. Der Rest des Kapitels besch\"af\-tigt sich mit der Frage, wie
diese Einf\"ugungen mit dem Seitenumbruchalgorithmus kooperieren.
Zun\"achst besch\"aftigen wir uns dabei mit den komplizierteren Befehlen,
die Plain \TeX\ zur Verf\"ugung stellt, sp\"ater sehen wir uns dann an,
wie die eigentliche Organisation vonstatten geht.
\subsubsection{Normale Einf\"ugungen}
Die einfachste Art etwas in einen Text einzuf\"ugen besteht in dem
Anweisungspaar:
\index{zzzendinsert@{\tt\char92 endinsert}}
\index{zzztopinsert@{\tt\char92 topinsert}}
\begin{verbatim}
\topinsert <vertikales Material> \endinsert
\end{verbatim}
Das `vertikale Material' ist dabei das, was eingef\"ugt werden soll.
\TeX\ versucht dieses Material am Anfang der laufenden Seite
unterzubringen. Ist dort kein Platz, dann wird es an den Anfang der
n\"achsten Seite geschrieben. Man kann z.B.\ folgendes schreiben:
\index{zzzendinsert@{\tt\char92 endinsert}}\index{zzzhsize@{\tt\char92 hsize}}
\index{zzznoindent@{\tt\char92 noindent}}
\index{zzztopinsert@{\tt\char92 topinsert}}
\index{zzzvskip@{\tt\char92 vskip}}
\begin{verbatim}
\topinsert\vskip2in\hsize=3in\noindent
{\bf Figur 3} Dies ist die Unterschrift unter der dritten Figur meines
Textes. Ich habe vor dem Text 2 Inch Platz gelassen, damit ich dort
sp\"ater meine Illustration einkleben kann \endinsert
\end{verbatim}
\TeX\ f\"ugt hinter dem eingef\"ugten Material automatisch einen
\index{zzzbigskip@{\tt\char92 bigskip}}`\verb|\bigskip|' ein.

Analog funktioniert der Befehl
\index{zzzpageinsert@{\tt\char92 pageinsert}}
`\verb|\pageinsert|', nur da\ss{} das
vertikale Material, das dem Befehl folgt auf eine eigene Seite
geschrieben wird, \"ublicherweise auf die n\"achste Seite. und schlie\ss{}lich
gibt es noch den Befehl \index{zzzmidinsert@{\tt\char92 midinsert}}
`\verb|\midinsert|', der versucht das Material
an der Stelle des Auftretens unterzubringen. Dann ist der Effekt wie:
\index{zzzbigbreak@{\tt\char92 bigbreak}}\index{zzzbigskip@{\tt\char92 bigskip}}
\index{zzzvbox@{\tt\char92 vbox}}
\begin{verbatim}
\bigskip\vbox{<vertikales Material>}\bigbreak
\end{verbatim}
ist nicht genug Platz vorhanden, dann wird das Material wie bei
\index{zzztopinsert@{\tt\char92 topinsert}}`\verb|\topinsert|' behandelt.

Benutzen sie die Befehl nicht innerhalb von horizontalen Boxen,
sondern nur an Stellen, an denen sich \TeX\ im vertikalen Modus
befindet.

Wenn sie viele der obigen Befehl hintereinander benutzen, kann es
passieren, da\ss{} \TeX\ die Inhalte der Befehl auf viele folgende Seiten
verteilen mu\ss{}. Die Einf\"ugungen behalten dabei aber ihre vorgegebene
Reihenfolge. Wenn sie verhindern wollen, da\ss{} Einf\"ugungen in ein
sp\"ateres Kapitel \"ubernommen werden, dann benutzen sie die Befehlsfolge
\index{zzzsupereject@{\tt\char92 supereject}}\index{zzzvfill@{\tt\char92 vfill}}
`\verb|\vfill\supereject|'. Damit werden alle noch vorhandenen
Einf\"ugungen, nach \index{Leer!Raum}
Leerraum ausgegeben, bevor neuer Text gesetzt wird.

\subsubsection{Fu\ss{}noten}
Neben der M\"oglichkeit mit `\verb|\topinsert|' etwas an den Anfang
einer Seite zu schreiben, wird mit dem Befehl `\verb|\footnote|' etwas
ans Ende einer Seite geschrieben. Au\ss{}erdem kann dieser Befehl auch
innerhalb eines Paragraphen benutzt werden. Fu\ss{}noten werden mit zwei
\index{Argument}Argumenten aufgerufen, das erste Argument stellt dabei die
\index{Markierung!Fussnote@Markierung!Fu\ss{}note}
Fu\ss{}notenmarkierung, das zweite den Fu\ss{}notentext dar.\footnote{In
diesem Text wird das Fu\ss{}notenmakro von \LaTeX\ benutzt. In {\em Plain} \TeX\
sieht die Sache etwas anders aus. Zwischen den Fu\ss{}noten wird kein
zus\"atzlicher \index{Zwischenraum}
Zwischenraum eingef\"ugt, und die Fu\ss{}noten werden auch in
normaler \index{Groesse!Schrift@Gr\"o\ss{}e!Schrift}
\index{Schrift!Groesse@Schrift!Gr\"o\ss{}e}
Schriftgr\"o\ss{}e gesetzt. Die Fu\ss{}notenmarkierung wird in einen
\index{Zwischenraum}
Zwischenraum der Gr\"o\ss{}e {\tt\char92 textindent} gesetzt. Das Makro
{\tt\char92 textindent} arbeitet \"ahnlich, wie {\tt\char92 item},
allerdings ohne eine h\"angende \index{Einrueckung@Einr\"uckung}Einr\"uckung.}

Fu\ss{}noten k\"onnen umgebrochen werden und auf der folgenden Seite
weitergef\"uhrt werden. Dies wird allerdings soweit m\"oglich vermieden.
Einf\"ugungen k\"onnen nicht innerhalb von Einf\"ugungen verwendet werden.
Es ist also nicht m\"oglich innerhalb eines
\index{zzztopinsert@{\tt\char92 topinsert}}`\verb|\topinsert|'s eine
Fu\ss{}note zu benutzen. Sollten sie es doch einmal brauchen, dann k\"onnen
sie den Befehl \index{zzzvfootnote@{\tt\char92 vfootnote}}
`\verb|\vfootnote|' verwenden. Er ist dazu da im
vertikalen Modus eine Fu\ss{}note zu schreiben, und sie m\"ussen dann die
Fu\ss{}notenmarkierung in der Einf\"ugung per Hand anbringen, und den
`\verb|\vfootnote|' Befehl irgendwo auf die Seite schreiben, auf der
die Einf\"ugung sp\"ater erscheinen wird. Sie merken schon, nicht ganz
elegant, aber das sind Fu\ss{}noten sowieso nicht.

Bevor wir uns nun mit den Feinheiten von Einf\"ugungen besch\"aftigen
k\"onnen m\"ussen wir lernen, was \TeX\ unter einem \index{Register}
{\em Register}
versteht.
\subsection{Register}
\subsubsection{Einfache Register}
\TeX\ kennt ein Konstrukt, da\ss{} sich Register nennt und das in anderen
Programmiersprachen wohl am ehesten den Variablen gleichkommt. Es gibt
verschiedene Registertypen, und von jedem Registertyp 256 Register.
Die einfachen Registertypen sind das Zahlenregister, das
\index{Dimension!Register}
Dimensionsregister, das Skipregister und das Muskipregister. Die
Zahlregister hei\ss{}en \index{zzzcount@{\tt\char92 count}}
`\verb|\count0|' $\ldots$ `\verb|\count255|', die
Dimensionsregister \index{zzzdimen@{\tt\char92 dimen}}
`\verb|\dimen0|' $\ldots$ `\verb|\dimen255|'. Der
Zahlbereich der Zahlregister reicht von $-2^{31}$ bis $2^{31}$. Die
Dimensionsregister k\"onnen jeden g\"ultigen \index{Dimension}
Dimensionswert annehmen, die
Skip- und Muskipregister jeden g\"ultigen Wert f\"ur \index{Leim}Leim. Die
\index{Zuweisung}Zuweisung erfolgt als:
\index{zzzcount@{\tt\char92 count}}\index{zzzdimen@{\tt\char92 dimen}}
\index{zzzmuskip@{\tt\char92 muskip}}\index{zzzskip@{\tt\char92 skip}}
\begin{verbatim}
\count<number> = <number>
\dimen<number> = <dimen>
\skip<number> = <glue>
\muskip<number> = <muglue>
\end{verbatim}
Mit diesen Register k\"onnen sie auch rechnen. \index{Addition}Addition und
\index{Subtraktion}Subtraktion
erfolgen mit dem gleichen Befehl:
\index{zzzadvance@{\tt\char92 advance}}\index{zzzcount@{\tt\char92 count}}
\index{zzzmuskip@{\tt\char92 muskip}}
\begin{verbatim}
\advance\count<number> by <number>
\advance\dimen<number> by <dimen>
\advance\skip<number> by <glue>
\advance\muskip<number> by <muglue>
\end{verbatim}
Die Anweisung: \index{zzzhsize@{\tt\char92 hsize}}
`\verb|\dimen8=\hsize \advance\dimen8 by 1in|' bewirkt
z.B., da\ss{} der Inhalt des 9ten \index{Dimension!Register}
Dimensionsregister eine L\"ange enth\"alt,
die um ein Inch gr\"o\ss{}er ist, als die augenblickliche Textbreite.

{\em Unendliche} \index{Masse!Unendlich@Ma\ss{}e!Unendlich}
Ma\ss{}e verdr\"angen endliche Ma\ss{}e. Nach den Anweisungen:
\begin{verbatim}
\skip2 = 0pt plus 2fill minus 3fill
\advance\skip2 by 4pt plus 1fil minus 2filll
\end{verbatim}
Hat das Register den Wert:
\begin{verbatim}
4pt plus 2fill minus 2filll
\end{verbatim}

Auch die \index{Division}
Division und \index{Multiplikation!Register}
Multiplikation von Registern ist m\"oglich,
allerdings nur ganzzahligen Werten. Bei den Zahlregistern ist die
Sache einfach. Bei den anderen Registern werden alle Angaben mit der
entsprechenden Zahl multipliziert, bzw.\ dividiert.
\index{zzzmultiply@{\tt\char92 multiply}}
`\verb|\multiply\count2 by 2|' \index{Doppelt!Register}
verdoppelt beispielsweise den Inhalt
des 3ten Zahlregisters, \index{zzzdivide@{\tt\char92 divide}}
`\verb|\divide\skip2 by 2|' halbiert alle
Werte im 3ten Skipregister. Da es sich nur um ganzzahlige
Rechenoperationen handelt, wird bei der Division etwaiger Rest
vernachl\"assigt. Das Vorzeichen des Ergebnisses bei der Division h\"angt
von den Vorzeichen der Operanden ab. Bei ungleichen Vorzeichen ist das
Ergebnis negativ, sonst positiv.\footnote{Vermeiden sie die Division
durch Null und \index{Multiplikation!Overflow}
Multiplikationen, die die Kapazit\"at des Registers
\"uberschreiten.}

Sie k\"onnen Register \"uberall da verwenden, wo sie auch explizite
Angaben verwenden d\"urfen. Die Angabe von \index{zzzdimen@{\tt\char92 dimen}}
\index{zzzhskip@{\tt\char92 hskip}}`\verb|\hskip\dimen2|' ist
ebenso erlaubt, wie die Angabe von \index{zzzadvance@{\tt\char92 advance}}
`\verb|\advance\count10|
\verb|by| \verb|\count10|'.

\index{Dimension!Register}
Dimensionsregister k\"onnen auch als Zahlregister, und Skipregister
k\"onnen ihrerseits sowohl als Dimensions- oder auch als Zahlregister
verwendet werden. Bei der Verwendung eines Skipregisters als
Dimensionsregister, wird etwaige Angaben f\"ur
\index{Schrumpf!Schrumpfung}Schrumpfung oder
Streckung ignoriert, bei der Verwendung eines Dimensionsregisters
wird als \index{Einheit!Scaled Point}Einheit der {\it sp} (Scaled Point) verwendet.

Normalerweise haben Register ihren Wert nur innerhalb der
\index{Gruppe!Registerwert}Gruppe, in
der sie benutzt werden,\footnote{Daraus ergibt sich, da\ss{} \TeX\
wesentlich mehr Register hat, als oben angegeben.} es ist aber m\"oglich
Register auch f\"ur alle Gruppen zu \"andern, indem man den Befehl
\index{zzzglobal@{\tt\char92 global}}
`\verb|\global|' voranstellt. Die Anweisungen:
\index{zzzadvance@{\tt\char92 advance}}\index{zzzcount@{\tt\char92 count}}
\begin{verbatim}
\count1=10 \count2=20 \count3=30
{\count1=20 \global\advance\count1 by 12 \global\count2=2 \count2=4}
\end{verbatim}
haben zur Folge, da\ss{} die Register folgende Werte haben:
\begin{verbatim}
\count1 : 32
\count2 :  2
\count3 : 30
\end{verbatim}
\index{zzzcount@{\tt\char92 count}}
Die ersten zehn Zahlregister (`\verb|\count0|' $\ldots$
`\verb|\count9|') sind reserviert und sollten nicht benutzt werden.
Die Inhalte dieser Register werden bei der \index{Ausgabe!Auf dem
Bildschirm} Ausgabe auf dem Bildschirm
ausgegeben und durch \index{Dezimal!Punkt}\index{Punkt}
Dezimalpunkte getrennt. F\"uhrende \index{Leer!Register}leere Register
werden dabei nicht ausgegeben. {\em Plain} \TeX\ benutzt nur das erste
Zahlregister f\"ur die Seitenzahl, weshalb sie auch meist nur die
Ausgabe: `\verb|[0] [1] [2] ...|' sehen. W\"are z.B.\ `\verb|\count5=3|'
und `\verb|\count7=4|' dann w\"are die Ausgabe: `\verb|[3.0.4]|'.

Es ist in \TeX\ einfach m\"oglich symbolische Namen f\"ur Register zu
vergeben. Dazu gibt es z.B. den Befehl \index{zzzcountdef@{\tt\char92 countdef}}
`\verb|\countdef|'. Nach der
Anweisung:
\index{zzzchapno@{\tt\char92 chapno}}
\begin{verbatim}
\countdef\chapno=28
\end{verbatim}
kann der Befehl `\verb|\chapno|' als \index{Abkuerzung@Abk\"urzung}
Abk\"urzung f\"ur `\verb|\count28|'
verwendet werden. Analog arbeiten die Befehle:
\index{zzzdimendef@{\tt\char92 dimendef}}`\verb|\dimendef|',
\index{zzzskipdef@{\tt\char92 skipdef}}
`\verb|\skipdef|' und \index{zzzmuskipdef@{\tt\char92 muskipdef}}
`\verb|\muskipdef|'
\subsubsection{Boxregister}\index{Box!Register}
Neben den numerischen Registern kennt \TeX\ auch noch die sogenannten
Boxregister (`\verb|\box0|' $\ldots$ `\verb|\box255|'). In diesen
Boxregistern kann eine beliebige hbox, vbox oder Rulebox gespeichert
werden. Die Anweisung:
\index{zzzhbox@{\tt\char92 hbox}}
\index{zzzsetbox@{\tt\char92 setbox}}
\begin{verbatim}
\setbox0=\hbox{A}
\end{verbatim}
hat zur Folge, da\ss{} das erste Boxregister (`\verb|\box0|') den
Buchstaben `A' zum Inhalt hat. Die Abmessungen des Inhalts eines
derartigen Registers k\"onnen abgerufen werden. Mit den Befehlen:
\index{zzzdp@{\tt\char92 dp}}\index{zzzht@{\tt\char92 ht}}
\index{zzzwd@{\tt\char92 wd}}
\begin{verbatim}
\wd<nummer>
\ht<nummer>
\dp<nummer>
\end{verbatim}
werden die Breite, \index{Hoehe!Box@H\"ohe!Box}H\"ohe und Tiefe der entsprechenden
\index{Box!Masse@Box!Ma\ss{}e}Box angesprochen.

Einen wichtigen Unterschied zwischen Boxregistern und numerischen
Registern gibt es: Boxregister werden beim Gebrauch geleert. Die
Anweisung \index{zzzraise@{\tt\char92 raise}}
`\verb|\raise\2pt\box3|' in einer \index{Horizontal!Liste}
horizontalen Liste bewirkt
erstens, da\ss{} der Inhalt der 4ten Box, um 2pt noch oben verschoben, in
die Liste aufgenommen wird, zweitens ist aber auch die 4te Box nach
dieser Anweisung leer. Wenn sie den Inhalt dieser Box nocheinmal
verwenden wollen, benutzen sie den Befehl \index{zzzcopy@{\tt\char92 copy}}
`\verb|\copy|', satt des
Befehls \index{zzzbox@{\tt\char92 box}}`\verb|\box|'. Im obigen Beispiel w\"are dann die Anweisung
\index{zzzraise@{\tt\char92 raise}}`\verb|\raise2pt\copy3|' richtig gewesen.

Eine weitere Anweisung, die mit \index{Box!Register}
Boxregistern durchgef\"uhrt werden kann
ist das `{\em Unboxen}'. Damit k\"onnen sie eine \index{Ebene!Boxen}
Ebene von Boxen aufheben. Die Anweisungen:
\index{zzzsetbox@{\tt\char92 setbox}}
\begin{verbatim}
\setbox3=\hbox{A} \setbox3=\hbox{\box3 B}
\setbox4=\hbox{A} \setbox4=\hbox{\unhbox4 B}
\end{verbatim}
bewirken, da\ss{} in `\verb|\box3|' `\verb|\hbox{\hbox{A} B}|' steht, und
in `\verb|\box4|' `\verb|\hbox{AB}|'. Analog arbeitet der Befehl
\index{zzzunvbox@{\tt\char92 unvbox}}
`\verb|\unvbox|' f\"ur vertikale Boxen. Der Vorteil dieses Befehls liegt
auf der Hand: Weniger \index{Speicher!Bedarf}
Speicherbenutzung und schnellere Bearbeitung.
Ebenso gibt es die Befehle
\index{zzzunhcopy@{\tt\char92 unhcopy}}\index{zzzunvcopy@{\tt\char92 unvcopy}}
`\verb|\unhcopy|' und `\verb|\unvcopy|',
deren Verwendung eigentlich sofort klar sein d\"urfte. Interessant ist
nur noch das Verhalten von \index{Leim}Leim beim Unboxen. Nach
\index{zzzsetbox@{\tt\char92 setbox}}
\begin{verbatim}
\setbox5=\hbox{A \hbox{B C}}
\setbox6=\hbox to 1.05\wd5{\unhcopy5}
\end{verbatim}
ist `\verb|\box6|' um 5 \index{Prozent}
Prozent breiter, als `\verb|\box5|', es wird
aber nur zwischen dem `A' und dem `B' gestreckt, die {\em innere} Box
bleibt unber\"uhrt.

Es gibt einen Unterschied zwischen einem leeren Boxregister und einem
Boxregister, das eine leere Box enth\"alt. Enth\"alt ein Register eine
leere hbox (`\verb|\hbox{}|'), dann k\"onnen auf dieses Register nur die
Befehl \index{zzzunhbox@{\tt\char92 unhbox}}\index{zzzunhcopy@{\tt\char92 unhcopy}}
`\verb|\unhbox|' und `\verb|\unhcopy|' angewendet werden. die
vertikalen \"Aquivalente sind nicht mehr zul\"a\ss{}ig. Bei einer
\index{Leer!Box}leeren Box
durfen alle Befehle verwendet werden.

\subsubsection{Allgemeine Registerbefehle}
Wenn sie sich unschl\"ussig sind, wie \TeX\ mit Registern arbeitet, dann
k\"onnen sie verschiedene `\verb|\show|'-Befehle benutzen.
\index{zzzcount@{\tt\char92 count}}\index{zzzdimen@{\tt\char92 dimen}}
\index{zzzshowthe@{\tt\char92 showthe}}
\begin{verbatim}
\showthe\count1 \showthe\dimen2 \showthe\skip3
\end{verbatim}
gibt z.B.\ die Inhalte des 2ten Zahl-, 3ten \index{Dimension!Register}
Dimensions- und 4ten
Skipregisters aus. Mit dem Befehl \index{zzzshowbox@{\tt\char92 showbox}}
`\verb|\showbox4|' k\"onnen sie sich den Inhalt
des 5ten Boxregisters ansehen.\footnote{Die Ausgabe erfolgt \"ubrigens
nur in den \index{File!Log}Logfile, es sei denn, sie setzen {\tt\char92
tracingonline=1}.}

Wenn viele Leute an einem \index{Makro!Paket}
Makropaket f\"ur \TeX\ arbeiten, w\"are es
fatal, wenn verschiedene Menschen dasselbe Register f\"ur
unterschiedliche Zwecke benutzen wollen. \TeX\ stellt deshalb den
Befehl \index{zzznewcount@{\tt\char92 newcount}}
`\verb|\newcount|' bereit, um ein noch unbenutztes Register
ansprechen zu k\"onnen. Sie k\"onnen einfach mit
\index{zzzmeinzaehler@{\tt\char92 meinzaehler}}
\begin{verbatim}
\newcount\meinzaehler
\end{verbatim}
ein \index{Neu!Zahlregister}
neues Zahlregister bereitstellen lassen, da\ss{} dann mit
`\verb|\meinzaehler|' angesprochen werden kann. Analog gibt es auch
die Befehle:
\index{zzznewbox@{\tt\char92 newbox}}
\index{zzznewdimen@{\tt\char92 newdimen}}
\index{zzznewmuskip@{\tt\char92 newmuskip}}
\index{zzznewskip@{\tt\char92 newskip}}
\begin{verbatim}
\newdimen
\newskip
\newmuskip
\newbox
\end{verbatim}
Au\ss{}erdem stellt {\em Plain} \TeX\ auch noch die Befehle:
\index{zzznewfam@{\tt\char92 newfam}}\index{zzznewinsert@{\tt\char92 newinsert}}
\index{zzznewread@{\tt\char92 newread}}\index{zzznewtoks@{\tt\char92 newtoks}}
\index{zzznewwrite@{\tt\char92 newwrite}}
\begin{verbatim}
\newtoks
\newread
\newwrite
\newfam
\newinsert
\end{verbatim}
bereit, deren Bedeutung sp\"ater erkl\"art wird.

Wenn sie den Inhalt eines numerischen Registers in den Text \"ubernehmen
wollen, benutzen sie den Befehl \index{zzzthe@{\tt\char92 the}}
`\verb|\the|'. Mit
`\verb|\the\meinzaehler|' wird der augenblickliche Inhalt des
Zahlregisters `\verb|\meinzaehler|' in den Text \"ubernommen.

F\"ur lokalen Gebrauch hat es sich durchgesetzt die Register
\index{zzzcount@{\tt\char92 count}}\index{zzzdimen@{\tt\char92 dimen}}
`\verb|\count255|', `\verb|\dimen255|' usw.\ zu benutzen. Dabei
sollten sie allerdings nicht das Register \index{Box!255}
`\verb|\box255|' benutzen,
da es eine besondere Bedeutung hat. Die Register $0\ldots9$ haben auch
spezielle Bedeutungen und werden auch nicht von den
\index{zzznew@{\tt\char92 new}}`\verb|\new|$\ldots$' Befehlen zur\"uckgegeben.

\subsection{Die Details der Einf\"ugung}
Es gibt 255 \index{Klasse!Einfuegung@Klasse!Einf\"ugung}
Klassen von Einf\"ugungen. \index{zzzinsert@{\tt\char92 insert}}
`\verb|\insert0|' $\ldots$
`\verb|\insert255|'. Jede Einf\"ugung korrespondiert mit einigen der
entsprechenden bekannten \index{Register!Korrespondierende}
Registern. `\verb|\insert100|' korrespondiert
mit \index{zzzcount@{\tt\char92 count}}\index{zzzdimen@{\tt\char92 dimen}}
`\verb|\count100|', `\verb|\dimen100|', `\verb|\skip100|' und
`\verb|\box100|'. {\em Plain} \TeX\ stellt Befehle zur Verf\"ugung, diese
Register sofort zu reservieren. Mit dem Befehl
\index{zzzfootins@{\tt\char92 footins}}\index{zzznewinsert@{\tt\char92 newinsert}}
\begin{verbatim}
\newinsert\footins
\end{verbatim}
Wird z.B.\ die Einf\"ugungsnummer f\"ur \index{Fussnote@Fu\ss{}note}Fu\ss{}noten festegelegt. Mit
\index{zzzcount@{\tt\char92 count}}
`\verb|\count\footins|' kann jetzt das entsprechende Zahlregister
angesprochen werden. Tats\"achlich sind die \index{Klasse!Nummer}
Klassennummern
festgelegt,\footnote{Fu\ss{}noten sind Klasse 254, Topeinf\"ugungen die
Klasse 253} aber im weiteren braucht uns das zun\"achst nicht zu st\"oren.
Nehmen wir einfach an es ginge um die Einf\"ugung $n$. Dann gilt f\"ur
diese Klasse von Einf\"ugung:

\begin{quote}
\begin{tabular}{lp{10cm}}
{\tt\char92 box}$n$ & Enth\"alt das Material, das bei der Ausgabe
eingef\"ugt werden soll.\\
{\tt\char92 count}$n$ & Enth\"alt den Vergr\"o\ss{}erungsfaktor, der f\"ur den
Seitenumbruch gebraucht wird.\\
{\tt\char92 dimen}$n$ & Enth\"alt die maximale Einf\"ugungsgr\"o\ss{}e pro
Seite.\\
{\tt\char92 skip}$n$ & Enth\"alt den Platz, der auf einer Seite
zus\"atzlich bereitgestellt werden soll.
\end{tabular}
\end{quote}\index{Ausgabe}

Wenn die Einf\"ugung einfach nur ihre normale
\index{Hoehe!Einfuegung@H\"ohe!Einf\"ugung}H\"ohe hat, kann der
\index{Groesse!Faktor@Gr\"o\ss{}e!Faktor}
Vergr\"o\ss{}erungsfaktor auf 1000 gesetzt werden, auf dem er normalerweise
auch steht. Eine 10pt \index{Fussnote!Platzbedarf@Fu\ss{}note!Platzbedarf}
Fu\ss{}note braucht auch 10pt Platz. Es kann nun
aber sein, da\ss{} diese Fu\ss{}note in zwei
\index{Spalte!Fussnote@Spalte!Fu\ss{}note}Spalten gesetzt werden soll, dann
braucht sie in der H\"ohe nur den halben Platz. In diesem Falle w\"are
\index{zzzcount@{\tt\char92 count}}
`\verb|\count|$n$' auf 500 zu setzen. Der Vergr\"o\ss{}erungsfaktor wird
also gebraucht um auch in schwierigen F\"allen die Seitenberechnung
richtig durchf\"uhren zu k\"onnen.

Die erste Fu\ss{}note auf einer Seite braucht etwas mehr Platz, als sie
selber einnimmt, da wir ja die Fu\ss{}noten etwas vom \"ubrigen Text
absetzen wollen und auch noch eine \index{Horizontal!Linie}
horizontale Linie einf\"ugen wollen.
Dieser Extraplatz steht in dem entsprechenden Skipregister. {\em Plain}
\TeX\ setzt z.B.
\index{zzzbigskipamount@{\tt\char92 bigskipamount}}
\index{zzzfootins@{\tt\char92 footins}}
\index{zzzskip@{\tt\char92 skip}}
\begin{verbatim}
\skip\footins=\bigskipamount
\end{verbatim}
Damit wird der notwendige Platz bereitgestellt.

Mitunter ist es w\"unschenswert den maximalen Platz f\"ur Einf\"ugungen zu
begrenzen. Z.B.\ will niemand eine ganze Seite voller Fu\ss{}noten lesen.
Der Maximale Paltz, den alle Einf\"ugungen einer
\index{Klasse!Einfuegung@Klasse!Einf\"ugung}Klasse einnehmen d\"urfen
steht im entsprechenden \index{Dimension!Register}
Dimensionsregister. F\"ur Fu\ss{}noten etwa,
definiert {\em Plain} \TeX:
\index{zzzdimen@{\tt\char92 dimen}}\index{zzzfootins@{\tt\char92 footins}}
\begin{verbatim}
\dimen\footins=8 in
\end{verbatim}
Somit k\"onnen maximal 8 Inch einer Seite mit Fu\ss{}noten aufgef\"ullt
werden, der Rest wird auf weitere Seiten \"ubernommen.

Zur Zerlegung einer Einf\"ugung gibt es einen Befehl:
\index{zzzvsplit@{\tt\char92 vsplit}}`\verb|\vsplit|'.
Mit der Anweisung
\index{zzzsetbox@{\tt\char92 setbox}}
\begin{verbatim}
\setbox200=\vsplit100 to 50pt
\end{verbatim}
geht \TeX\ folgenderma\ss{}en um: Zun\"achst wird der Anfang des Inhaltes
von \index{Box!Register}
Box100 nach Box200 kopiert, bis die Box200 die geforderte
\index{Hoehe!Box@H\"ohe!Box}H\"ohe von
50pt hat.\footnote{Hierbei werden dieselben Regeln wirksam, wie beim
Seitenumbruch.} Danach wird aus der Box100 alles entfernt, was jetzt
in der Box200 steht. So wird auch verfahren, wenn eine Einf\"ugung \"uber
mehr als eine Seite verteilt werden soll.
\chapter{Mathematische Formeln}
\section{Eingabe einfacher Formeln}
\subsection{Zeichen im mathematischen Modus}
Mathematische Formeln werden in \TeX\ mit zwei \index{Dollarzeichen}
Dollarzeichen (\$)
\index{Klammer!Mathematische Formeln}
eingeklammert. Alles was zwischen diesen Klammern steht wird als
mathematische Formel verstanden. \TeX\ verf\"ahrt dabei so, da\ss{}
komplexere Formeln aus einfacheren Formeln modular zusammengesetzt
werden.

Innerhalb des mathematischen Modus von \TeX\ verh\"alt sich \TeX\
anders, als im Textmodus. Die normalen Zeichen (`A'$\ldots$`Z' und
`a'$\ldots$`z') werden in \index{Italic}\index{Schrift!Italic}
Italicschrift gesetzt. Die \index{Punkt}Punkte. \index{Komma}Kommata
u.\"a.\ werden normal in \index{Roman}Roman gesetzt, aber z.B.\ das
\index{Minuszeichen}Minuszeichen
sieht anders aus, als der einfache \index{Bindestrich}
Bindestrich. \index{Leer!Zeichen}Leerzeichen werden im
mathematischen Modus g\"anzlich ignoriert, statt dessen verwendet \TeX\
eigene Abst\"ande f\"ur die Formeln. Die \index{Funktion!Des Leerzeichens}
Funktion des Leerzeichens als
Beendigung einer \index{Kontrollsequenz}
Kontrollsequenz bleibt allerdings erhalten. Die in
der \index{Mathematik}Mathematik oft gebrauchten \index{Griechische Buchstaben}
griechischen Buchstaben stehen sowohl
als Klein, wie auch als \index{Grossbuchstaben@Gro\ss{}buchstaben}
Gro\ss{}buchstaben zur Verf\"ugung. Die Eingabe von \index{zzzalpha@{\tt\char92
alpha}}\index{zzzbeta@{\tt\char92 beta}}\index{zzzdelta@{\tt\char92 delta}}
`\verb|$$\alpha,| \verb|\beta,| \index{zzzgamma@{\tt\char92 gamma}}
\verb|\gamma,| \verb|\delta$$|' ergibt:
\[\alpha, \beta, \gamma, \delta\]
Und durch eine kleine \"Anderung wird mit `\verb|$\Gamma$|': $\Gamma$.
Einige griechische Buchstaben stehen sogar in verschiedenen
Ausf\"uhrungen bereit.

\index{zzzepsilon@{\tt\char92 epsilon}}\index{zzzphi@{\tt\char92 phi}}
\index{zzzrho@{\tt\char92 rho}}\index{zzztheta@{\tt\char92 theta}}
`\verb|$\phi,\theta,\epsilon,\rho$|' ergibt: $\phi,\theta,\epsilon,\rho$

\index{zzzvarepsilon@{\tt\char92 varepsilon}}\index{zzzvarphi@{\tt\char92 varphi}}
\index{zzzvarrho@{\tt\char92 varrho}}\index{zzzvartheta@{\tt\char92 vartheta}}
`\verb|$\varphi,\vartheta,\varepsilon,\varrho$|' ergibt:
$\varphi,\vartheta,\varepsilon,\varrho$

Dar\"uber hinaus gibt es noch viele Zeichen, die meist nur in der
\index{Mathematik}
Mathematik verwendet werden, wie \index{zzzapprox@{\tt\char92 approx}}
`$\approx$' (`\verb|\approx|') oder
`$\mapsto$' \index{zzzmapsto@{\tt\char92 mapsto}}(`\verb|\mapsto|').

\subsection{Potenzen und Indizes}
Nun soll etwas $^{\mbox{\tiny hochgestellt}}$ oder $_{\mbox{\tiny
tiefgestellt}}$ werden. In \TeX\ verwendet man dazu die Zeichen
`\verb|^|' und `\verb|_|'. Im einfachsten Fall, wenn nur ein Zeichen
hoch- bzw.\ tiefgestellt werden soll, sieht das folgenderma\ss{}en aus:

\medskip\begin{tabular}{ll}
{\em Eingabe} & {\em Ausgabe}\\
{\tt\char36 x\char94 2\char36} & $x^2$\\
{\tt\char36 x\char95 2\char36} & $x_2$\\
{\tt\char36 2\char94 x\char36} & $2^x$\\
{\tt\char36 x\char94 2y \char94 2\char36} & $x^2y^2$\\
{\tt\char36 x\ \char94\ 2y\ \char94\ 2\char36} & $x ^ 2y ^ 2$\\
{\tt\char36 \char95 2F\char94 3\char36} & $_2F_3$
\end{tabular}\medskip

Wenn mehrere Zeichen hoch- bzw.\ tiefgestellt werden sollen, dann
m\"ussen diese Zeichen zu einer \index{Gruppe!Hoch- Tiefstellung}
Gruppe zusammengefa\ss{}t werden:

\medskip\begin{tabular}{ll}
{\tt\char36 x\char94\char123 2y\char125\char26} & $x^{2y}$\\
{\tt\char36 2\char94\char123 2\char94 x\char125\char36} & $2^{2^x}$\\
{\tt\char36 y\char95\char123 x\char94 2\char125\char36} & $y_{x^2}$\\
\end{tabular}\medskip

Wie es auch bei {\em guten} \index{Mathematik}
Mathematikern der Fall ist, sind
Kombinationen, wie `\verb|x^y^z|' oder `\verb|x_y_z|'
verboten.\footnote{Der Grund liegt in der Nichteindeutigkeit der
Schreibweise. $9^{9^9}$ kann sowohl als $9^{(9^9)}$ verstanden werden,
als auch als ${(9^9)}^9 = 9^{81}$} Verwenden sie {\em immer}
\index{Klammer!Gruppe}
Gruppenklammern, um ihre Intention deutlich zu machen, also z.B.\
`\verb|{x_y}_z|' oder `\verb|x^{y^z}|'. Aus \"ahnlichem Grunde ist es
auch schlechter \index{Stil!Index}
Stil `\verb|_2F_3|' zu schreiben. Setzen sie den
ersten Index besser an eine \index{Leer!Gruppe}leere \index{Gruppe!Leere}
Gruppe: `\verb|{}_2F_3|'.

Die Reihenfolge der Indizierung und Potenzierung eines Zeichens ist
egal, wie sie bei den folgenden Beispielen sehen k\"onnen:

\medskip\begin{tabular}{ll}
{\tt\char36 x\char94 2\char95 3\char36} & $x^2_3$\\
{\tt\char36 x\char95 3\char94 2\char36} & $x_3^2$\\
{\tt\char36 x\char94\char123 31415\char125\char95\char123 92\char125
+\char92 pi\char36} & $x^{314151}_{92}+\pi$\\
{\tt\char36 x\char95\char123 y\char94 a\char95b\char125\char94\char123
z\char95 c\char94 d\char125\char36} & $x_{y^a_b}^{z_c^d}$
\end{tabular}\medskip

Das die normalen Zeichen in \index{Italic}
Italic ausgegeben werden, werden Indizes
etwas an das Zeichen heranger\"uckt, der Index steht also nicht genau
unter der Potenz. Will man diesen Effekt vermeiden, mu\ss{} man wieder
eine \index{Leer!Gruppe}leere Gruppe verwenden:

\medskip\begin{tabular}{ll}
{\tt\char36 P\char95 2\char94 2\char36} & $P_2^2$\\
{\tt\char36 P\char123\char125\char95 2\char94 2\char36} & $P{}_2^2$
\end{tabular}\medskip

Sehr oft wird als hochgestelltes Zeichen in der \index{Mathematik}
Mathematik ein
einfacher Strich benutzt. \TeX\ stellt hierf\"ur den Befehl
\index{zzzprime@{\tt\char92 prime}}`\verb|\prime|' zur Verf\"ugung.

\medskip\begin{tabular}{ll}
{\tt\char36 y\char95 1\char94\char92 prime\char36} & $y_1^\prime$\\
{\tt\char36 y\char95 2\char94\char123\char92 prime\char92
prime\char125\char36} & $y_2^{\prime\prime}$\\
{\tt\char36 y\char95 2\char94\char123\char92 prime\char92
prime\char92 prime\char125\char36} & $y_2^{\prime\prime\prime}$\\
\end{tabular}\medskip

\subsection{Wurzeln und \"Ahnliches}
In \TeX\ gibt es einige Zeichen, die sich in ihrer
\index{Groesse!Automatisch@Gr\"o\ss{}e!Automatisch}Gr\"o\ss{}e automatisch
den Erfordernissen anpassen. Dazu geh\"oren das Wurzelzeichen
\index{zzzsqrt@{\tt\char92 sqrt}}\index{zzzunderline@{\tt\char92 underline}}
(`\verb|\sqrt|'), die Unterstreichung (`\verb|\underline|') und die
\index{Ueberstreichung@\"Uberstreichung}
\"Uberstreichung \index{zzzoverline@{\tt\char92 overline}}
(`\verb|\overline|'). Die Eingaben von
\index{zzzalpha@{\tt\char92 alpha}}
\index{zzzsqrt@{\tt\char92 sqrt}}
\index{zzzunderline@{\tt\char92 underline}}
\begin{verbatim}
$\sqrt2$
$\sqrt{x+2}$
$\underline4$
$\overline{x+y}$
$\overline x+\overline y$
$x^{\underline n}$
$x^{\overline{n+m}}$
$\sqrt{x^3+\sqrt\alpha}$
\end{verbatim}
ergeben der Reihe nach:

$\sqrt2$, $\sqrt{x+2}$, $\underline4$, $\overline{x+y}$,
$\overline x+\overline y$, $x^{\underline n}$, $x^{\overline{n+m}}$,
$\sqrt{x^3+\sqrt\alpha}$

Auch andere, als die zweite Wurzeln, lassen sich in \TeX\ realisieren:
\index{zzzroot@{\tt\char92 root}}

\medskip\begin{tabular}{ll}
{\tt\char36\char92 root\ 3\ \char92 of\ 2\char36} & $\root 3 \of 2$\\
{\tt\char36\char92 root\ n\ \char92 of\ \char123 x\char94 n+y\char94
n\char125\char36} & $\root n \of {x^n+y^n}$\\
{\tt\char36\char92 root\ n+1\ \char92 of a\char36} & $\root n+1 \of a$
\end{tabular}\medskip

Wurzeln, \"Uber- und Unterstreichungen passen sich nicht nur der L\"ange
des \index{Argument}Arguments an, sondern auch der
\index{Hoehe!Argument@H\"ohe!Argument}H\"ohe. Deutlich wird das am
Unterschied zwischen \index{zzzoverline@{\tt\char92 overline}}
`\verb|\overline l|' ($\overline l$) und
`\verb|\overline m|' ($\overline m$). Oder auch der Folge: $\sqrt
a+\sqrt d+\sqrt y$. Wenn sie diese unterschiedlichen H\"ohen vermeiden
wollen, dann gibt es den Befehl \index{zzzmathstrut@{\tt\char92 mathstrut}}
`\verb|\mathstrut|', der dem\index{zzzstrut@{\tt\char92 strut}}
`\verb|\strut|' Befehl weitgehend entspricht, nur f\"ur den
mathematischen Modus. Die Eingabe von
\index{zzzsqrt@{\tt\char92 sqrt}}
\begin{verbatim}
$\sqrt{\mathstrut a}+\sqrt{\mathstrut d}+\sqrt{\mathstrut y}$
\end{verbatim}
ergibt: $\sqrt{\mathstrut a}+\sqrt{\mathstrut d}+\sqrt{\mathstrut y}$

\subsection{Die Zeichen im mathematischen Modus}
Ich habe oben schon angedeutet, da\ss{} im mathematischen Modus manche
Zeichen anders ausgegeben werden, als im Textmodus. Hier nun ein etwas
systematischerer \"Uberblick, welche Zeichen sich wie \"andern.

Die normalen Buchstaben (`A' bis `Z' und `a' bis `z') werden in
\index{Italic}Italic
gesetzt, da damit \"ublicherweise Variablen bezeichnet werden.
\[A\ldots Z\mbox{\quad und\quad}a\ldots z\]
F\"ur den Buchstaben `l' gibt es sogar zwei Zeichen, da in
\index{Schrift!Hand}
handschriftlichen Texten das `l' oft mit der `1' verwechselt wird. Der
zugeh\"orige Befehl hei\ss{}t \index{zzzell@{\tt\char92 ell}}
`\verb|\ell|' und er ergibt $\ell$.

Die 18 Zeichen
\begin{verbatim}
0 1 2 3 4 5 6 7 8 9 ! ? . | / ` @ \"
\end{verbatim}
sind einfache Symbole, f\"ur sie wird kein extra Platz bereitgestellt.
Man achte allerdings auf den Unterschied zwischen der Null ($0$) und
dem kleinen oder gro\ss{}en `O'.

Die drei Zeichen `\verb|-|', `\verb|+|' und `\verb|*|' hei\ss{}en bin\"are
\index{Operator!Binaer@Operator!Bin\"ar}
Operatoren, da sie sich \"ublicherweise auf zwei Teile einer Formel
beziehen. Sie werden mit etwas zus\"atzlichem \index{Zwischenraum}
Zwischenraum gesetzt:

\medskip\begin{tabular}{ll}
{\em Eingabe} & {\em Ausgabe}\\
{\tt\char36 x+y-z\char35} & $x+y-z$\\
{\tt\char36 x+y*z\char36} & $x+y*z$\\
{\tt\char36 x*y/z\char36} & $x*y/z$
\end{tabular}\medskip

Beachten sie auch, da\ss{} der \index{Bindestrich}
Bindestrich, im Textmodus `-', zum
\index{Minuszeichen}
Minuszeichen ($-$) wird, und das \index{Stern!Textmodus}
Sternchen, im Textmodus `*' tiefer
gesetzt wird ($*$). Beim \index{Schraegstrich@Schr\"agstrich}
Schr\"agstrich wird {\em kein} zus\"atzlicher
Platz gelassen, auch wenn meist zwei Teile durcheinander geteilt
werden.

\TeX\ stellt noch mehr zweiseitige \index{Operator!Zweiseitig}
Operatoren bereit:

\index{zzzbullet@{\tt\char92 bullet}}\index{zzzcap@{\tt\char92 cap}}
\index{zzzcirc@{\tt\char92 circ}}\index{zzzvee@{\tt\char92 vee}}
\index{zzzwedge@{\tt\char92 wedge}}
\medskip\begin{tabular}{ll}
{\tt\char36 c\char92 times y\char92 cdot z\char36} & $x\times y\cdot
z$\\
{\tt\char36 x\char92 circ y\char92 bullet z\char36} & $x\circ y\bullet
z$\\
{\tt\char36 x\char92 cup y\char92 cap z\char36} & $x\cup y\cap z$\\
{\tt\char36 x\char92 sqcup y\char92 sqcap z\char36} & $x\sqcup y\sqcap z$\\
{\tt\char36 x\char92 vee y\char92 wedge z\char36} & $x\vee y\wedge
z$\\
{\tt\char36 x\char92 mp y\char92 pm z\char36} & $x\mp y\pm z$
\end{tabular}\medskip

In manchen F\"allen wird auch bei einem zweiseitigen
\index{Operator!Zweiseitig}Operator {\em kein}
zus\"atzlicher \index{Zwischenraum}Zwischenraum eingef\"ugt:
\begin{enumerate}
\item Wenn nur auf einer Seite des Operators etwas steht:

\medskip\begin{tabular}{ll}
{\tt\char36 x=+1\char36} & $x=+1$\\
{\tt\char36 3.142-\char36} & $3.142-$\\
{\tt\char36(D*)\char36} & $(D*)$
\end{tabular}\medskip

\item Bei Indizes und \index{Potenz}Potenzen
\index{zzzbullet@{\tt\char92 bullet}}\index{zzzcap@{\tt\char92 cap}}
\index{zzzmapsto@{\tt\char92 mapsto}}

\medskip\begin{tabular}{ll}
{\tt\char36 K\char95 n\char94 +,K\char95 n\char94 -\char36} &
$K_n^+,K_n^-$\\
{\tt\char36 z\char94 *\char95\char123 ij\char125\char36} &
$z^*_{ij}$\\
{\tt\char36 g\char94\char92 circ\ \char92 mapsto\ g\char94\char92
bullet\char36} & $g^\circ \mapsto g^\bullet$\\
{\tt\char36 f\char94 *(x)\ \char92 cap\ f\char95 *(y)\char36} &
$f^*(x) \cap f_*(y)$
\end{tabular}\medskip
\end{enumerate}

Neben den einfachen zweiseitigen Operatoren stellt \TeX\ auch noch
\index{Operator!Relation}\index{Relation}
Relationsoperatoren bereit, die {\em etwas} anders gesetzt werden. Die
einfachsten Relationsoperatoren sind `\verb|=|', `\verb|<|', `\verb|>|'
und `\verb|:|', wie die folgenden Beispiele aber zeigen, nicht die
einzigen.

\medskip\begin{tabular}{ll}
{\tt\char36x=y\char62 z\char36} & $x=y>z$\\
{\tt\char36 x:=y\char36} & $x:=y$\\
{\tt\char36 x\char92 le y\char92 ne z\char36} & $x\le y\ne z$\\
{\tt\char36 x\char92 sim y\char92 simeq z\char36} & $x\sim y\simeq
z$\\
{\tt\char36 x\char92 equiv y\char92 not\char92 equiv z\char36}&
$x\equiv y\not\equiv z$\\
{\tt\char36 x\char92 subset y\char92 subseteq z\char36} & $x\subset
y\subseteq z$
\end{tabular}\medskip

Die \index{Interpunktion!Zeichen}
Interpunktionszeichen \index{Komma}
Komma (,) und Semikolon (;) werden mit etwas
zus\"atzlichem Platz {\em hinter} dem Zeichen, aber nicht {\em vor} dem
Zeichen gesetzt

\medskip\begin{tabular}{ll}
{\tt\char36 f(x,y;z)\char36} & $f(x,y;z)$
\end{tabular}\medskip

Der \index{Punkt}Punkt wird als normales Zeichen behandelt, anders, als der
\index{Doppelt!Punkt}Doppelpunkt. Soll dieser als normales Zeichen gesetzt werden, dann kann
das mit dem Befehl \index{zzzcolon@{\tt\char92 colon}}`\verb|\colon|' geschehen.

\medskip\begin{tabular}{ll}
{\tt\char36 f:A\char92 to\ B\char36} & $f:A\to B$\\
{\tt\char36 f\char92 colon A\char92 to\ B\char36} & $f\colon A\to B$
\end{tabular}\medskip

Wenn das \index{Komma}
Komma als einfaches Symbol gesetzt werden soll, schreiben sie
es in geschweifte \index{Klammer!Geschweifte}
Klammern. Alles, was in geschweiften Klammern steht
wird als einfaches Symbol betrachtet.

\medskip\begin{tabular}{lll}
{\tt\char36 12,345x\char36} & $12,345x$ & (falsch) \\
{\tt\char36 12\char123 ,\char125 345x\char36} & $12{,}345x$ & (richtig)
\end{tabular}\medskip

Nun gibt es nur noch einige Tasten zu besprechen. Die Zeichen
`\verb|(|' und `\verb|[|' werden `\"Offner' genannt, die Zeichen
`\verb|)|' und `\verb|]|' \index{Schliesser@Schlie\ss{}er}
`Schlie\ss{}er'. Das Zeichen `\verb|'|' kennen
wir ja schon, und es bleiben nur noch die reservierten Zeichen:
\begin{verbatim}
\ $ % # & ~ { } _ ^
\end{verbatim}

Die besonderen Zeichen f\"ur Hoch- und Tiefstellung, sowie einige der
anderen mathematischen Befehle, wie \index{zzzapprox@{\tt\char92 approx}}
`\verb|\approx|' oder
\index{zzzalpha@{\tt\char92 alpha}}
`\verb|\alpha|' k\"onnen nicht in normalem Text verwendet werden. \TeX\
benutzt diese Zeichen, um einen m\"oglichen \index{Fehler!Mathematische
Modus}Fehler ihrerseits finden zu
k\"onnen. Wird ein derartiges Zeichen oder ein solcher Befehl au\ss{}erhalb
des mathematischen Modus angetroffen, geht \TeX\ davon aus, da\ss{} die
ein `\verb|$|' vergessen haben, oder eins zuviel geschrieben haben.

\subsection{Die mathematischen Akzente}\index{Akzent!Mathematische}
\TeX\ stellt neben den oben beschriebenen Zeichen auch noch eine Reihe
von Akzenten bereit, die auch nur im mathematischen Modus ansprechbar
sind. Hier eine vollst\"andige Liste:

\index{zzzacute@{\tt\char92 acute}}\index{zzzbar@{\tt\char92 bar}}
\index{zzzbreve@{\tt\char92 breve}}\index{zzzcheck@{\tt\char92 check}}
\index{zzzddot@{\tt\char92 ddot}}\index{zzzdot@{\tt\char92 dot}}
\index{zzzgrave@{\tt\char92 grave}}\index{zzzhat@{\tt\char92 hat}}
\index{zzztilde@{\tt\char92 tilde}}\index{zzzvec@{\tt\char92 vec}}
\medskip\begin{tabular}{ll}
{\tt\char36\char92 hat a\char36} & $\hat a$\\
{\tt\char36\char92 check a\char36} & $\check a$\\
{\tt\char36\char92 tilde a\char36} & $\tilde a$\\
{\tt\char36\char92 acute a\char36} & $\acute a$\\
{\tt\char36\char92 grave a\char36} & $\grave a$\\
{\tt\char36\char92 dot a\char36} & $\dot a$\\
{\tt\char36\char92 ddot\char36} & $\ddot a$\\
{\tt\char36\char92 breve\char36} & $\breve a$\\
{\tt\char36\char92 bar\char36} & $\bar a$\\
{\tt\char36\char92 vec\char36} & $\vec a$
\end{tabular}\medskip

Die Akzente lassen sich nat\"urlich auch auf andere Buchstaben als das
`a' setzen. Die ersten neun Akzente sind auch im Textmodus verwendbar,
haben dort allerdings eine andere \index{Kontrollsequenz}Kontrollsequenz.
Wenn sie einen der Buchstaben mit einem bestimmten Akzent
\"ofters brauchen, emfiehlt sich eine \index{Definition}
Definition zu Beginn des Textes
\index{zzzdef@{\tt\char92 def}}\index{zzzhat@{\tt\char92 hat}}
\begin{verbatim}
\def\Ahat{{\hat A}}
\def\chat{{\hat c}}
\end{verbatim}
Die Buchstaben `i' und `j' stehen im mathematischen Modus auch ohne
\index{Punkt}
Punkt,mit den Befehlen \index{zzzimath@{\tt\char92 imath}}
`\verb|\imath|' und `\verb|\jmath|', zur Verf\"ugung.
Diese eignen sich besser zur Akzentuierung.

Auch \index{Doppelt!Akzent}doppelte \index{Akzent!Doppelter}
Akzente, jeder \index{Mathematik}Mathematiker h\"atte seine helle Freude,
lassen sich in \TeX\ realsieren, es bedarf allerdings etwas M\"uhe. Das
Symbol \index{zzzhat@{\tt\char92 hat}}
`$\skew6\hat{\hat A}$' l\"a\ss{}t sich z.B. mit dem Befehl
\index{zzzskew@{\tt\char92 skew}}
`\verb|\skew6\hat\Ahat|' erreichen. Die `6' wurde durch probieren
gefunden, mit dieser Angabe wird der Akzent nach rechts oder links
verschoben. `5' h\"atte den Akzent etwas zu weit nach links gesetzt, `7'
etwas zu weit nach rechts. Der entscheidende Befehl hei\ss{}t
`\verb|\skew|'.

Abschlie\ss{}ende kennt \TeX\ noch zwei \index{Akzent!wachsende}
Akzente, die mit dem \index{Argument}Argument
mitwachsen (zumindest in gewissem Ma\ss{}e). Es sind dies die Akzente
\index{zzzwidehat@{\tt\char92 widehat}}\index{zzzwidetilde@{\tt\char92 widetilde}}
`\verb|\widehat|' und `\verb|\widetilde|'

\medskip\begin{tabular}{ll}
{\tt\char36\char92 widehat x\char36} & $\widehat x$\\
{\tt\char36\char92 widetilde x\char36} & $\widetilde x$\\
{\tt\char36\char92 widehat\char123 xy\char125\char36} &
$\widehat{xy}$\\
{\tt\char36\char92 widetilde\char123 xy\char125\char36} &
$\widetilde{xy}$\\
{\tt\char36\char92 widehat\char123 xyz\char125\char36} &
$\widehat{xyz}$\\
{\tt\char36\char92 widetilde\char123 xyz\char125\char36} &
$\widetilde{xyz}$
\end{tabular}\medskip

\section{Weitere Konstruktionselemente}
\subsection{Gestapelte Formeln}
Die beliebtesten Formeln aller \index{Mathematik}
Mathematiker sind die \index{Bruch}Br\"uche und
verwandte Formeln. Die Konstruktion von
\[\frac{1}{2}\mbox{\quad oder\quad}
\frac{n+1}{3}\mbox{\quad oder\quad}
\left(\begin{array}{c}n+1\\3\end{array}\right)\mbox{\quad oder\quad}
\sum_{n=1}^3 Z^2_n\]
sind in \TeX\ m\"oglich durch die Befehle
\index{zzzchoose@{\tt\char92 choose}}
\index{zzzover@{\tt\char92 over}}
\index{zzzsum@{\tt\char92 sum}}
\begin{verbatim}
$$1\over2$$
$$n+1\over3$$
$$n+1\choose3$$
$$\sum_{n=1}^3 Z_n^2$$
\end{verbatim}
\subsubsection{Br\"uche}\index{Bruch}
Br\"uche werden in \TeX\ durch den Befehl `\verb|\over|' realisiert. Er
bezieht sich auf {\em alles} was vor resp. nach ihm steht, es sei denn
sie gruppieren die gew\"unschten Zeichen.

\index{Display!Stil}
\medskip\begin{tabular}{ll}
{\tt\char36\char36 x+y\char94 2\char92 over k+1\char36\char36} &
$\displaystyle{\frac{x+y^2}{k+1}}$\\
{\tt\char36\char36\char123 x+y\char94 2\char92 over k\char125
+1\char36\char36} & $\displaystyle{\frac{x+y^2}{k}+1}$\\
{\tt\char36\char36 x+\char123 y\char94 2\char92 over k\char125
+1\char36\char36} & $\displaystyle{x+\frac{y^2}{k}+1}$\\
{\tt\char36\char36 x+\char123 y\char94 2\char92 over
k+1\char125\char36\char36} & $\displaystyle{x+\frac{y^2}{k+1}}$\\
{\tt\char36\char36 x+y\char94\char123 2\char92 over
k+1\char125\char36\char36} & $\displaystyle{x+y^{\frac{2}{k+1}}}$\\
\end{tabular}\medskip

Bei \index{Bruch}\index{Mehrfachbrueche@Mehrfachbr\"uche}
Mehrfachbr\"uchen m\"ussen sie sogar geschweifte
\index{Klammer!Bei Bruechen@Klammer!Bei Br\"uchen}Klammern setzen

\medskip\begin{tabular}{ll}
{\tt\char36\char36\char123 a\char92 over b\char125\char92 over
2\char36\char36} & $\displaystyle{\frac{\frac{a}{b}}{2}}$\\
{\tt\char36\char36 a\char92 over\char123 b\char92 over
2\char125\char36\char36} & $\displaystyle{\frac{a}{\frac{b}{2}}}$
\end{tabular}\medskip

Bei derartigen \index{Bruch}\index{Mehrfachbrueche@Mehrfachbr\"uche}
Mehrfachbr\"uchen empfiehlt sich aber sowieso die
Verwendung des \index{Schraegstrich@Schr\"agstrich}
Schr\"agstriches. Die obigen Br\"uche s\"ahen dann wie
$\frac{a/b}{2}$ und $\frac{a}{b/2}$ aus. Denken sie nur daran, da\ss{} bei
Br\"uchen mit \index{Schraegstrich@Schr\"agstrich}
Schr\"agstrichen etwas andere Regeln gelten. Sie m\"ussen
eventuell ein paar Klammern mehr schreiben.
\subsection{Die verschiedenen Stile}
Bei den Beispielen d\"urften ihnen schon aufgefallen sein, da\ss{} \TeX\ die
Zeichen im mathematischen Modus in unterschiedlichen
\index{Groesse!Von Zeichen@Gr\"o\ss{}e!Von Zeichen}Gr\"o\ss{}en und Stilen
setzt. Es gibt acht verschiedene Stile in \TeX\ eine Formel zu setzen:

\index{Display!Stil}\index{Ebene!Hoch- Tiefstellung}\index{Skript!Stil}
\medskip\begin{tabular}{ll}
Display Stil & F\"ur abgesetzte Formeln\\
Text Stil & F\"ur Formeln im Text\\
Skript Stil & F\"ur Hoch- bzw.\ Tiefstellungen\\
Skriptskript Stil & F\"ur die zweite Ebene von Hoch- bzw.\
Tiefstellungen
\end{tabular}\medskip\index{Formel!abgesetzte}

Au\ss{}erdem gibt es noch zu jedem Stil einen ``geklammerten'' Stil, der
sich nicht wesentlich unterscheidet von dem Originalstil, nur da\ss{} die
\index{Exponent}
Exponenten nicht so hoch gesetzt werden. Im weiteren werden die Stile
durch:
\[D, D', T, T', S, S', SS, SS'\]
abgek\"urzt. Wird eine Formel in einfachen \index{Dollarzeichen}
Dollarzeichen eingeschlossen,
dann erscheint sie in Textstil, ist sie in doppelten
\index{Dollarzeichen!Doppelt}Dollarzeichen
eingeschlossen, erscheint sie im \index{Display!Stil}
Displaystil. Aus der Stilart folgt
die Gr\"o\ss{}e, in der die Formel, oder die Unterformel, gesetzt wird.

Es gibt keinen $SSS$-Stil. Dieser w\"are nicht mehr lesbar. Die
Zusammenh\"ange der Stilarten ergeben sich wie folgt:

\index{Hochstellung}
\medskip\begin{tabular}{lll}
{\em Die Formel} & {\em Hochstellung} & {\em Tiefstellung} \\
$D,T$ & $S$ & $S'$\\
$D',T'$ & $S'$ & $S'$\\
$S,SS$ & $SS$ & $SS'$\\
$S',SS'$ & $SS'$ & $SS'$
\end{tabular}\medskip

Wenn z.B.\ die Formel `\verb|x^{a_b}|' im $D$-Stil gesetzt werden
soll, dann wird das `\verb|a|' in $S$ und das `\verb|b|' in $SS'$
gesetzt. Das Ergebnis ist: $\displaystyle{x^{a_b}}$.

Der Hauptunterschied zwischen \index{Display!Stil}
Displaystil und Textstil zeigt sich bei
\index{Bruch}Br\"uchen. Die Zusammenh\"ange zeigt die folgende Tabelle:

\index{Bruch}\index{Nenner}
\medskip\begin{tabular}{lll}
 & \multicolumn{2}{c}{\em Bruch} \\
{\em Formel} & {\em Nenner} & {\em Z\"ahler} \\
$D$ & $T$ & $T'$\\
$D'$ & $T'$ & $T'$\\
$T$ & $S$ & $S'$\\
$T'$ & $S'$ & $S'$\\
$S,SS$ & $SS$ & $SS'$\\
$S',SS'$ & $SS'$, & $SS'$
\end{tabular}\medskip

Am deutlichsten immer noch im Beispiel. Der \index{Bruch}
Bruch `einhalb' erscheint
im Text als: $\frac{1}{2}$ und in der abgesetzten
Formel\index{Formel!abgesetzte}
als:\[\frac{1}{2}\]

\"Ubrigens, auch die Befehle \index{zzzoverline@{\tt\char92 overline}}
`\verb|\overline|' und \index{zzzsqrt@{\tt\char92 sqrt}}`\verb|\sqrt|'
\"andern den $D$-Stil in den $D'$-Stil. Die Stilarten sind nat\"urlich
nicht gottgegeben. Sie k\"onnen sie mit den Befehlen
\index{zzzdisplaystyle@{\tt\char92 displaystyle}}
\index{zzzscriptstyle@{\tt\char92 scriptstyle}}
\index{zzztextstyle@{\tt\char92 textstyle}}
\begin{verbatim}
\displaystyle
\textstyle
\scriptstyle
\scriptscriptstyle
\end{verbatim}
jederzeit selber frei w\"ahlen. Dazu zun\"achst ein etwas {\em dummes}
Beispiel: Die Eingabe von
\begin{verbatim}
$$n+\scriptstyle n+\scriptscriptstyle n$$
\end{verbatim}
erzeugt \[n+\scriptstyle n+\scriptscriptstyle n\]
Man sieht aber schon, da\ss{} auch das \index{Pluszeichen}
Pluszeichen mit verkleinert wird.
Etwas besser geeignet um die Unterschiede zu zeigen ist das folgende
Beispiel.\footnote{Der Befehl {\tt\char92 strut} wird benutzt, um die
Z\"ahler etwas gr\"o\ss{}er zu machen. In einem sp\"ateren Kapitel besch\"aftigen
wir uns noch mehr mit dem {\em sch\"onen} Satz von Formeln.} Die Eingabe
von
\index{zzzdisplaystyle@{\tt\char92 displaystyle}}
\index{zzzover@{\tt\char92 over}}
\index{zzzstrut@{\tt\char92 strut}}
\begin{verbatim}
$$a_0+{1\over\displaystyle a_1+
          {\strut 1\over\displaystyle a_2+
            {\strut 1\over\displaystyle a_3+
              {\strut 1\over a_4}}}}$$
\end{verbatim}
erzeugt
\[a_0+{1\over\displaystyle a_1+
          {\strut 1\over\displaystyle a_2+
            {\strut 1\over\displaystyle a_3+
              {\strut 1\over a_4}}}}\]
Ohne die Befehle `\verb|\displaystyle|' und `\verb|\strut|' sa\"ahe das
Ergebnis
\[a_0+{1\over a_1+
          {1\over a_2+
            {1\over a_3+
              {1\over a_4}}}}\]
aus. Die \index{Nenner}
Nenner werden automatisch \index{Zentrierung}
zentriert. Will man das verhindern,
dann sollte man den Befehl
\index{zzzhfill@{\tt\char92 hfill}}`\verb|\hfill|' benutzen. Damit sieht unser
\index{Bruch}Serienbruch so
\[a_0+{1\hfill\over\displaystyle a_1+
          {\strut 1\hfill\over\displaystyle a_2+
            {\strut 1\hfill\over\displaystyle a_3+
              {\strut 1\hfill\over a_4}}}}\]
aus.
\subsection{Weitere Befehle zur Stapelung}
Neben dem Befehl
\index{zzzover@{\tt\char92 over}}
`\verb|\over|' zur Bildung von Br\"uchen existiert in
\TeX\ auch noch der Befehl \index{zzzatop@{\tt\char92 atop}}
`\verb|\atop|', der einen \index{Bruch}
`Bruch' ohne Bruchstrich ausgibt.

\medskip\begin{tabular}{ll}
{\tt\char36\char36x\char92 atop y+2\char36\char36} &
$\displaystyle{x\atop y+2}$\\
\end{tabular}\medskip

Weiterhin existiert der Befehl \index{zzzchoose@{\tt\char92 choose}}
`\verb|\choose|'\footnote{Die
\index{Binomealkoeffizienten}
Binomealkoeffizienten werden meist in der \index{Statistik}
Statistik gebraucht, um die
Anzahl der M\"oglichkeiten einer Wahl auszudr\"ucken.} der einen
Binomealkoeffizienten ausgibt.

\medskip\begin{tabular}{ll}
{\tt\char36\char36 n\char92 choose k\char36\char36} &
$\displaystyle{n\choose k}$\\
\end{tabular}\medskip

Auch Mischungen dieser Befehle sind erlaubt, vorausgesetzt, sie halten
sich an die Regeln f\"ur Gruppierungen innerhalb von Formeln.
\[{n\choose k}\over 2\]
\[n\choose{k\over2}\]

Es gibt noch einen Befehl, der zum Satz von Br\"uchen herangezogen
werden kann, bei dem allerdings die St\"arke des \index{Bruch}Bruchstrichs extra
angegeben werden mu\ss{}. Der Befehl hei\ss{}t \index{zzzabove@{\tt\char92 above}}
`\verb|\above|' und wird z.B.\
folgenderma\ss{}en angewendet
\index{zzzdisplaystyle@{\tt\char92 displaystyle}}
\begin{verbatim}
$$\displaystyle{a\over b}\above1pt\displaystyle{c\over d}$$
\end{verbatim}
Das Ergebnis sieht dann so aus:
\[\displaystyle{a\over b}\above1pt\displaystyle{c\over d}\]

\subsection{Summen und Integrale}
Summen und Integrale werden im \index{Stil!Text}
Textstil und im \index{Display!Stil}
\index{Stil!Display}Displaystil in
unterschiedlichen Gr\"o\ss{}en gesetzt, sie hei\ss{}en auch {\em gro\ss{}e}
Symbole.\footnote{Es gibt au\ss{}er diesen beiden noch einige andere, die
aber hier nicht weiter besprochen werden.}

\medskip\begin{tabular}{llll}
{\tt\char36\char92 sum x\char95 n\char36} & ergibt & $\sum x_n$ &
($T$-Stil)\\
{\tt\char36\char36\char92 sum x\char95 n\char36\char36} & ergibt &
$\displaystyle{\sum x_n}$ & ($D$-Stil)
\end{tabular}\medskip

Bei den \index{Summe!Zeichen}Summationszeichen werden meist Grenzen angegeben
\[\sum_{n=1}^m x_n\]
Die Grenzen werden in \index{Formel!abgesetzte}
abgesetzten Formeln \"uber und unter das
\index{Summe!Zeichen}
Summensymbol gesetzt. in Textformeln werden sie neben das Symbol
gesetzt: $\sum_{n=1}^m x_n$. Bei Integralen werden die
Integrationsgrenzen immer neben das Integrationszeichen gesetzt.

\index{zzzinfty@{\tt\char92 infty}}\index{zzzint@{\tt\char92 int}}
\medskip\begin{tabular}{ll}
$\int_{-\infty}^{+\infty}$ & Im $T$-Stil\\
$\displaystyle{\int_{-\infty}^{+\infty}}$ & Im $D$-Stil
\end{tabular}\medskip

Mit \verb|\limits| und \verb|\nolimits| l\"\ss{}t sich das Setzen der Grenzen
aber auch ganz individuell gestalten.\index{zzzlimits@{\tt\char92 limits}}
\index{zzznolimits@{\tt\char92 nolimits}}
Die Eingabe von:
\index{zzzsum@{\tt\char92 sum}}
\begin{verbatim}
$\sum\limits_{n=1}^m x_n
\end{verbatim}
erzeugt in Textformeln: $\sum\limits_{n=1}^m x_n$ und die Eingabe von:
\index{zzzint@{\tt\char92 int}}\index{zzzlimits@{\tt\char92 limits}}
\begin{verbatim}
$$\int\limits_{-\infty}^{+\infty}$$
\end{verbatim}
erzeugt: \[\int\limits_{-\infty}^{+\infty}\]
Andererseits ergibt
\index{zzznolimits@{\tt\char92 nolimits}}
\index{zzzsum@{\tt\char92 sum}}
\begin{verbatim}
$$\sum\nolimits_{n=1}^3 x_n$$
\end{verbatim}
\[\sum\nolimits_{n=1}^3 x_n\]
Mit dem Befehl \index{Display!Limits}
\index{zzzdisplaylimits@{\tt\char92 displaylimits}}
`\verb|\displaylimits|' werden die Grenzen {\em jedes}
gro\ss{}en Symbols nur noch im $D$-Stil ober- resp.\ unterhalb des Symbols
gesetzt, sonst immer dahinter.
\subsubsection{Mehrzeilige Grenzen}
Bei \index{Mathematik}
Mathematikern kommt es auch vor, da\ss{} die Angabe der Grenzen sich
\"uber mehrere Zeilen erstreckt. Auch das ist in \TeX\ m\"oglich. Die
Formel:\index{zzzatop@{\tt\char92 atop}}
\[\sum_{\scriptstyle0\le i\le m\atop\scriptstyle0<j<n}P(i,j)\]
wurde z.B. mit der Befehlsfolge
\index{zzzsum@{\tt\char92 sum}}
\begin{verbatim}
$$\sum_{\scriptstyle0\le i\le m\atop\scriptstyle0<j<n}P(i,j)$$
\end{verbatim}
erzeugt.
\subsection{Gr\"o\ss{}enanpassung von Symbolen}
Manche mathematischen Formeln sind ecklig lang. \TeX\ bietet daher
die M\"oglichkeit manche Symbole in ihrer Gr\"o\ss{}e anzupassen. Die Eingabe
von
\index{zzzsqrt@{\tt\char92 sqrt}}
\begin{verbatim}
$$\sqrt{1+\sqrt{1+\sqrt{1+
            \sqrt{1+\sqrt{1+\sqrt{1+\sqrt{1+x}}}}}}}$$
\end{verbatim}
erzeugt z.B.:
$$\sqrt{1+\sqrt{1+\sqrt{1+
            \sqrt{1+\sqrt{1+\sqrt{1+\sqrt{1+x}}}}}}}$$
Die drei gr\"o\ss{}ten Wurzelzeichen sind dabei untereinander identisch, bis
auf die Tatsache, da\ss{} der Aufstrich des Wurzelzeichens durch
unterschiedlich lange `\verb+|+' verl\"angert wurde. Die anderen Wurzeln
sind eigene Zeichen. Analog ist eine Gr\"o\ss{}enanpassung von
\index{Klammer!Groessenanpassung@Klammer!Gr\"o\ss{}enanpassung}Klammern
m\"oglich.
\index{zzzBiggl@{\tt\char92 Biggl}}\index{zzzbiggl@{\tt\char92 biggl}}
\index{zzzbiggr@{\tt\char92 biggr}}\index{zzzBigl@{\tt\char92 Bigl}}
\index{zzzbigl@{\tt\char92 bigl}}\index{zzzbigr@{\tt\char92 bigr}}
\[
\displaystyle
\left(\vbox to 27pt{}\left(\vbox to 24pt{}\left(\vbox to 21pt{}
\Biggl(\biggl(\Bigl(\bigl(({\scriptstyle({\scriptscriptstyle(\hskip3pt
)})})\bigr)\Bigr)\biggr)\Biggr)\right)\right)\right)
\left\{\vbox to 27pt{}\left\{\vbox to 24pt{}\left\{\vbox to 21pt{}
\Biggl\{\biggl\{\Bigl\{\bigl\{\{{\scriptstyle\{{\scriptscriptstyle\{\hskip3pt
\}}\}}\}\bigr\}\Bigr\}\biggr\}\Biggr\}\right\}\right\}\right\}
\]
Klammern sind in Formeln sehr wichtig, deshalb bietet \TeX\ eine ganze
Reihe von ihnen an:

\index{Klammer}\index{Doppelt!Strich}\index{Doppelt!Pfeil}\index{Pfeil}
\index{Gegenschraegstrich@Gegenschr\"agstrich}\index{Schraegstrich@Schr\"agstrich}
\index{zzzbackslash@{\tt\char92 backslash}}
\index{zzzdownarrow@{\tt\char92 downarrow}}
\index{zzzlangle@{\tt\char92 langle}}\index{zzzlceil@{\tt\char92 lceil}}
\index{zzzlfloor@{\tt\char92 lfloor}}\index{zzzrangle@{\tt\char92 rangle}}
\index{zzzrceil@{\tt\char92 rceil}}\index{zzzrfloor@{\tt\char92 rfloor}}
\index{zzzuparrow@{\tt\char92 uparrow}}
\index{zzzupdownarrow@{\tt\char92 updownarrow}}
\medskip\begin{tabular}{lll}
{\em Eingabe} & {\em Bedeutung} & {\em Ausgabe}\\
{\tt (} & linke Klammer & $($ \\
{\tt )} & rechte Klammer & $)$\\
{\tt\char91\ oder \char92 lbrack} & linke eckige Klammer & $[$\\
{\tt\char93\ oder \char92 rbrack} & rechte eckige Klammer & $]$\\
{\tt\char92\char123\ oder \char92 lbrace} & linke geschweifte Klammer & $\{$\\
{\tt\char92\char125\ oder \char92 rbrace} & rechte geschweifte Klammer & $\}$\\
{\tt\char92 lfloor} & linke Bodenklammer & $\lfloor$\\
{\tt\char92 rfloor} & rechte Bodenklammer & $\rfloor$\\
{\tt\char92 lceil} & linke Deckenklammer & $\lceil$\\
{\tt\char92 rceil} & rechte Deckenklammer & $\rceil$\\
{\tt\char92 langle} & linke einfach eckige Klammer & $\langle$\\
{\tt\char92 rangle} & rechte einfach eckige Klammer & $\rangle$\\
{\tt /} & Schr\"agstrich & $/$\\
{\tt\char92 backslash} & Gegenschr\"agstrich & $\backslash$\\
{\tt\char124\ oder \char92 vert} & Gerader Strich & $|$\\
{\tt\char92\char124\ oder \char92 Vert} & Doppelter gerader Strich & $\|$\\
{\tt\char92 uparrow} & Pfeil nach oben & $\uparrow$\\
{\tt\char92 downarrow} & Pfeil nach unten & $\downarrow$\\
{\tt\char92 Uparrow} & Doppelpfeil nach oben & $\Uparrow$\\
{\tt\char92 Downarrow} & Doppelpfeil nach unten & $\Downarrow$\\
{\tt\char92 updownarrow} & Pfeil nach oben und unten & $\updownarrow$\\
{\tt\char92 Updownarrow} & Doppelpfeil nach oben und unten & $\Updownarrow$
\end{tabular}\medskip

Um eine gr\"o\ss{}ere Fassung der gezeigten Symbole zu erreichen mu\ss{} ihnen
nur der Befehl \index{zzzbig@{\tt\char92 big}}\index{zzzbigl@{\tt\char92 bigl}}
`\verb|\bigl|' (f\"ur linke Klammersymbole) bzw.\
\index{zzzbigr@{\tt\char92 bigr}}
`\verb|\bigr|' (f\"ur rechte) vorangestellt werden. Die Eingabe von
\begin{verbatim}
$\bigl(x-s(x)\bigr)\bigl(y-s(y)\bigr)$
\end{verbatim}
erzeugt: $\bigl(x-s(x)\bigr)\bigl(y-s(y)\bigr)$, die Eingabe von
\index{zzzlfloor@{\tt\char92 lfloor}}\index{zzzrfloor@{\tt\char92 rfloor}}
\index{zzzsqrt@{\tt\char92 sqrt}}
\begin{verbatim}
$\bigl\lfloor\sqrt A\bigr\rfloor$
\end{verbatim}
erzeugt: $\bigl\lfloor\sqrt A\bigr\rfloor$.

Sehen wir uns doch einmal die verschiedenen Symbol im Vergleich an:
Zun\"achst die normalen mit den \index{zzzbig@{\tt\char92 big}}
ge`\verb|\big|'ten:
\index{zzzbackslash@{\tt\char92 backslash}}
\index{zzzbigl@{\tt\char92 bigl}}
\index{zzzbigm@{\tt\char92 bigm}}
\index{zzzdownarrow@{\tt\char92 downarrow}}
\index{zzzlangle@{\tt\char92 langle}}
\index{zzzlceil@{\tt\char92 lceil}}
\index{zzzlfloor@{\tt\char92 lfloor}}
\index{zzzrangle@{\tt\char92 rangle}}
\index{zzzrceil@{\tt\char92 rceil}}
\index{zzzrfloor@{\tt\char92 rfloor}}
\index{zzzVert@{\tt\char92 Vert}}
\[\begin{array}{l}
(\,)\,[\,]\,\{\,\}\,\lfloor\,\rfloor\,\lceil\,\rceil\,\langle\,\rangle
\,/\,\backslash\,\vert\,\Vert\,\uparrow\,\Uparrow\,\downarrow\,\Downarrow
\,\updownarrow\,\Updownarrow\\
\bigl(\,\bigr)\,\bigl[\,\bigr]\,\bigl\{\,\bigr\}\,\bigl\lfloor
\,\bigr\rfloor\,\bigl\lceil\,\bigr\rceil\,\bigl\langle\,\bigr\rangle
\,\big/\,\big\backslash\,\big\vert\,\big\Vert\,\bigm\uparrow\,\bigm\Uparrow
\,\bigm\downarrow\,\bigm\Downarrow\,\bigm\updownarrow\,\bigm\Updownarrow
\end{array}\]
Jetzt k\"onnen sie auch noch `\verb|\Big|'x sagen (das `x' steht f\"ur `l'
oder `r'):\index{zzzBigl@{\tt\char92 Bigl}}
\[\Bigl(\,\Bigr)\,\Bigl[\,\Bigr]\,\Bigl\{\,\Bigr\}\,\Bigl\lfloor
\,\Bigr\rfloor\,\Bigl\lceil\,\Bigr\rceil\,\Bigl\langle\,\Bigr\rangle
\,\Big/\,\Big\backslash\,\Big\vert\,\Big\Vert\,\Bigm\uparrow\,\Bigm\Uparrow
\,\Bigm\downarrow\,\Bigm\Downarrow\,\Bigm\updownarrow\,\Bigm\Updownarrow\]
Wenn ihnen das noch nicht gen\"ugt, gibt es die Befehle
\index{zzzbigg@{\tt\char92 bigg}}
\index{zzzbiggl@{\tt\char92 biggl}}
\index{zzzbiggm@{\tt\char92 biggm}}
\index{zzzbiggr@{\tt\char92 biggr}}
`\verb|\bigg|'x
\[\biggl(\,\biggr)\,\biggl[\,\biggr]\,\biggl\{\,\biggr\}\,\biggl\lfloor
\,\biggr\rfloor\,\biggl\lceil\,\biggr\rceil\,\biggl\langle\,\biggr\rangle
\,\bigg/\,\bigg\backslash\,\bigg\vert\,\bigg\Vert\,\biggm\uparrow
\,\biggm\Uparrow\,\biggm\downarrow\,\biggm\Downarrow\,\biggm\updownarrow
\,\biggm\Updownarrow\]
Und zu guter Letzt noch `\verb|\Bigg|'x
\index{zzzBiggl@{\tt\char92 Biggl}}
\index{zzzBiggm@{\tt\char92 Biggm}}
\[\Biggl(\,\Biggr)\,\Biggl[\,\Biggr]\,\Biggl\{\,\Biggr\}\,\Biggl\lfloor
\,\Biggr\rfloor\,\Biggl\lceil\,\Biggr\rceil\,\Biggl\langle\,\Biggr\rangle
\,\Bigg/\,\Bigg\backslash\,\Bigg\vert\,\Bigg\Vert\,\Biggm\uparrow
\,\Biggm\Uparrow\,\Biggm\downarrow\,\Biggm\Downarrow\,\Biggm\updownarrow
\,\Biggm\Updownarrow\]
Neben den `l' und `r' Formen dieser Befehle bietet \TeX\ auch noch
eine `m' Variante an, also z.B.\ \index{zzzbigm@{\tt\char92 bigm}}
`\verb|\bigm|' oder `\verb|\Biggm|'.
Bei dieser Befehlsform wird das entsprechend folgende Symbol nicht wie
eine Klammer, \"offnend oder schlie\ss{}end, gesetzt, sondern wie eine
\index{Relation}
Relation. Welches Ergebnis am besten aussieht h\"angt vom Einzelfall ab.
Als Beispiel diene hier die Eingabe von
\index{zzzbigl@{\tt\char92 bigl}}
\index{zzzbigm@{\tt\char92 bigm}}
\index{zzzbigr@{\tt\char92 bigr}}
\begin{verbatim}
$$\bigl(x\in A(n)\bigm|\||x\in B(n)\bigr)$$
\end{verbatim}
was
\[\bigl(x\in A(n)\bigm|\||x\in B(n)\bigr)\]
erzeugt. Au\ss{}erdem gibt es noch die M\"oglichkeit die Befehl {\em ohne}
Links-, oder Rechtsangabe zu benutzen. Dann wird das Symbol wie eine
normale Variable gesetzt. Dies empfiehlt sich meist nur beim Schr\"ag-
oder \index{Gegenschraegstrich@Gegenschr\"agstrich}Gegenschr\"agstrich.
\index{zzzover@{\tt\char92 over}}
\begin{verbatim}
$${a+1\over b}\bigg/{c+1\over d}$$
\end{verbatim}
ergibt
$${a+1\over b}\bigg/{c+1\over d}$$
\subsubsection{Die Rechts-Links-Befehle}
\TeX\ bietet noch eine einfache M\"oglichkeit die
\index{Klammer!Groessenanpassung@Klammer!Gr\"o\ss{}enanpassung}
Klammersymbole der
Gr\"o\ss{}e der Formel anzupassen. Die Befehl \index{zzzleft@{\tt\char92 left}}
`\verb|\left|' und\index{zzzrightarrow@{\tt\char92 rightarrow}}
`\verb|\right|' gefolgt von einem Klammersymbol sorgen selbst\"andig f\"ur
die richtige Gr\"o\ss{}e des Klammersymbols. Die Eingabe von
\index{zzzleft@{\tt\char92 left}}
\begin{verbatim}
$$1+\left(1\over1-x^2\right)^3$$
\end{verbatim}
ergibt in der \index{Ausgabe}Ausgabe
$$1+\left(1\over1-x^2\right)^3$$
In einer einfachen Formel, wie \index{zzzleft@{\tt\char92 left}}
`\verb|$\left(x\right)$|', werden nur
die einfachen Klammersymbole gew\"ahlt und die Ausgabe ist
$\left(x\right)$. Die Befehle `\verb|\left|' und `\verb|\right|'
m\"ussen immer in einer Gruppe stehen. Konstruktionen wie
\begin{verbatim}
$$\left( ... { ... \right) ... }$$
\end{verbatim}
sind nicht erlaubt. Die Befehle stellen allerdings {\em ihrerseits}
selber eine Gruppierung dar, wie das obige Beispiel zeigt. Das `$1+$'
wird nicht in den \index{Bruch}
Bruch einbezogen. Daraus folgt auch, da\ss{} immer zu
jedem `\verb|\left|'-Befehl ein entsprechender `\verb|\right|'-Befehl
geschrieben werden mu\ss{}. Die Klammersymbole m\"ussen allerdings nicht
zueinander passen, wie das folgende Beispiel zeigt.
\index{zzzatop@{\tt\char92 atop}}\index{zzzleft@{\tt\char92 left}}
\begin{verbatim}
$$\left(1+n\atop 1-n\right[$$
\end{verbatim}
ist erlaubt und ergibt
$$\left(1+n\atop 1-n\right[$$
Jetzt fragt man sich, wieso man \"uberhaupt noch die Einzelbefehle f\"ur
die Gr\"o\ss{}e der Klammersymbole benutzen soll, wenn doch \TeX\ das so
sch\"on selber kann. Nun, es k\"onnen drei Situationen auftreten, wo die
Links-Rechts-Befehle eben doch nicht so hundertprozentig
passen.\index{Klammer!Groessenwahl, manuell@Klammer!Gr\"o\ss{}enwahl,
manuell}\index{Eingabe!Groesse@Eingabe!Gr\"o\ss{}e}
\begin{enumerate}
\item Manchmal w\"ahlt \TeX\ eine Klammer, die f\"ur diese Situation zu
klein ist. Wenn man z.B.
\index{zzzleft@{\tt\char92 left}}\index{zzzrightarrow@{\tt\char92 rightarrow}}
\begin{verbatim}
$\left|\left|x\right|+\left|y\right|\right|$
\end{verbatim}
eingibt, dann ist das Ergebnis
$\left|\left|x\right|+\left|y\right|\right|$, obwohl man durch
explizite Eingabe der Gr\"o\ss{}en, das wesentlich bessere Ergebnis
\index{zzzbigl@{\tt\char92 bigl}}\index{zzzbigr@{\tt\char92 bigr}}
$\bigl|\left|x\right|+\left|y\right|\bigr|$ bevorzugen w\"urde.
\item Andererseits kann es aber auch vorkommen, da\ss{} \TeX\ zu gro\ss{}e
Symbole aussucht. Dies passiert z.B.\ bei gro\ss{}en Symbolen. Vergleichen
sie einmal
\index{zzzleft@{\tt\char92 left}}
\index{zzzsum@{\tt\char92 sum}}
\begin{verbatim}
$$\left(\sum_{k=1}^n A_k\right)$$
\end{verbatim}
mit\index{zzzbiggl@{\tt\char92 biggl}}\index{zzzbiggr@{\tt\char92 biggr}}
\begin{verbatim}
$$\biggl(\sum_{k=1}^n A_k\biggr)$$
\end{verbatim}
Und nun die Ausgabe
$$\left(\sum_{k=1}^n A_k\right)\qquad\biggl(\sum_{k=1}^n A_k\biggr)$$
\item Schlie\ss{}lich kann es vorkommen, da\ss{} sie eine Formel \"uber mehr als
eine Zeile schreiben m\"ussen. Sie werden dann Probleme bekommen die
zusammengeh\"origen Befehle innerhalb einer Gruppe zu realisieren. Auch
in diesem Fall m\"ussen sie die Gr\"o\ss{}enanpassung selber vornehmen.
\end{enumerate}\index{Ausgabe}
Es gibt \"ubrigens f\"ur die Links-Rechts-Befehle auch eine
\index{Leer!Klammer}{\em leere}
\index{Klammer!Leere}Klammer, die bei Anwendungen, wie
$$|x|=\left\{\begin{array}{ll}
x & \mbox{f\"ur\quad}x\ge0\\
-x & \mbox{f\"ur\quad}x<0\end{array}\right.$$
n\"otig ist. Zu jedem `\verb|\left|'-Befehl mu\ss{} ja auch der passende
\index{zzzrightarrow@{\tt\char92 rightarrow}}
`\verb|\right|'-Befehl geschrieben werden. Man gibt hier bei dem
`\verb|\right|'-Befehl statt eines Klammersymbols einen \index{Punkt}
Punkt an. Im
obigen Beispiel stand am Ende einfach
\begin{verbatim}
...x<0\right.$$
\end{verbatim}
Manchen sie sich keine Sorgen \"uber die zweizeilige Formel, wie so
etwas geht wird auch noch besprochen. Die {\em leere} Klammer ist
\"ubrigens nicht leer, statt dessen wird von \TeX\ noch der
\index{Zwischenraum}
Zwischenraum, der in der Variablen \index{Delimiterspace}
\index{zzznulldelimiterspace@{\tt\char92 nulldelimiterspace}}
`\verb|\nulldelimiterspace|' steht,
eingef\"ugt (Voreingestellt ist 1.2pt).
\subsubsection{Teilklammern}\index{Klammer!Teilklammern}
Weiter oben wurden einige Symbole gezeigt, die aus Teilzeichen
zusammengesetzt waren. Diese Teilzeichen kann man nat\"urlich auch
einzeln verwenden. Mit den Befehlen
\index{zzzarrowvert@{\tt\char92 arrowvert}}
\index{zzzbracevert@{\tt\char92 bracevert}}
\begin{verbatim}
\arrowvert
\Arrowvert
\bracevert
\end{verbatim}
werden einfache, bzw. \index{Doppelt!Strich}
doppelte senkrechte Striche gezogen, mit den
Befehlen
\index{zzzlgroup@{\tt\char92 lgroup}}
\index{zzzlmoustache@{\tt\char92 lmoustache}}
\index{zzzrgroup@{\tt\char92 rgroup}}
\index{zzzrmoustache@{\tt\char92 rmoustache}}
\begin{verbatim}
\lgroup
\rgroup
\lmoustache
\rmoustache
\end{verbatim}
werden Teile der geschweiften Klammer benutzt. Die Klammern sehen
folgenderma\ss{}en aus
\index{zzzarrowvert@{\tt\char92 arrowvert}}
\index{zzzbracevert@{\tt\char92 bracevert}}
\index{zzzlgroup@{\tt\char92 lgroup}}
$$\Big\arrowvert\ldots\Big\Arrowvert\ldots\Big\bracevert
\ldots\Big\lgroup\ldots\Big\rgroup\ldots\Big\lmoustache
\ldots\Big\rmoustache$$
Diese Klammern sind nur als ziemlich gro\ss{}e Symbole (gr\"o\ss{}er als Big)
verf\"ugbar.

Mitunter m\"ochten sie selber irgendwelche Symbole definieren, und in
unterschiedlichen Situationen benutzen. Daf\"ur bietet \TeX\ den Befehl
\index{zzzmathchoice@{\tt\char92 mathchoice}}
\begin{verbatim}
\mathchoice{<Teil>}{<Teil>}{<Teil>}{<Teil>}
\end{verbatim}
Die `\verb|<Teil>|'e sind dabei Teilformeln, die folgenderma\ss{}en
benutzt werden: Die erste Teilformel wird benutzt wenn der Stil $D$
oder $D'$ \index{Aktiv!Stil}
aktiv ist, die zweite, wenn der Stil $T$ oder $T'$ aktiv
ist, wie es weitergeht k\"onnen sie sich denken. Beachten sie im Anhang
auch den Befehle \index{zzzmathpalette@{\tt\char92 mathpalette}}
`\verb|\mathpalette|', der in diesem Zusammenhang
interessant ist.
\subsection{Die Achse einer Formel}\index{Achse einer Formel}
Wenn sie sich einmal eine Formel mit Klammern genau ansehen, dann
werden sie feststellen, da\ss{} die Klammern bez\"uglich einer unsichtbaren
\index{Achse!einer Formel}
Achse \index{Zentrierung}
zentriert sind. Diese Achse hei\ss{}t `Achse der Formel' und liegt
bei Textformeln so: $\hskip 2em\over$. Z.B.\ jeder \index{Bruch}
Bruchstrich liegt
auf dieser Achse. Mitunter ist es notwendig eine vertikale
\index{Box!Vcenter}Box auf
dieser Achse zu \index{Zentrierung}
zentrieren.\footnote{Bei dem $|x|=\{\ldots$ Beispiel
von eben war das z.B.\ n\"otig} \TeX\ bietet nat\"urlich auch hierf\"ur
einfache Befehle. Geben sie einfach
\index{zzzvcenter@{\tt\char92 vcenter}}
\begin{verbatim}
\vcenter{<vertikales Material>}
\end{verbatim}
ein. Das vertikale Material schreiben sie so, als handele es sich um
eine \index{zzzvbox@{\tt\char92 vbox}}
\verb|\vbox|. Den Rest erledigt \TeX. Auch solche Konstruktionen
wie `\verb|\vcenter to...|' oder `\verb|\vcenter spread...|' sind
m\"oglich. \"Ubrigens, auch alle anderen \index{Box!Konstruktion}Boxkonstruktionen, inklusive
\index{zzzraise@{\tt\char92 raise}}
`\verb|\raise|' oder \index{zzzlower@{\tt\char92 lower}}
`\verb|\lower|' sind im mathematischen Modus
m\"oglich und erlaubt.
\section{Die Interna}
Dieser Abschnitt ist zum eigentlichen Verst\"andnis von \TeX\ im
mathematischen Modus nicht notwendig, er besch\"aftigt sich mit den
Interna der Zeichenauswahl in mathematischen Formeln.
\subsection{Familien}
Jedes Zeichen, das im mathematischen Modus ausgegeben wird, geh\"ort zu
einer von sechzehn Familien, die ihrerseits aus drei Zeichens\"atzen
bestehen. Mit den Befehlen \index{Font!Wahl}
\index{zzztextfont@{\tt\char92 textfont}}`\verb|\textfont|',
\index{zzzscriptfont@{\tt\char92 scriptfont}}`\verb|\scriptfont|'
und `\verb|scriptscriptfont|' w\"ahlen dabei jeweils eine dieser
Zeichens\"atze. Die Familie 0 wird in {\em Plain} \TeX\ f\"ur die
\index{Roman}Roman
Zeichens\"atze verwendet und die \index{Zuweisung}Zuweisungen lauten
\begin{verbatim}
\textfont0=\tenrm
\scriptfont0=\sevenrm
\scriptscriptfont0=\fiverm
\end{verbatim}
Wird kein Zeichensatz angegeben, verwendet \TeX\
\index{Font!Null}\index{zzznullfont@{\tt\char92 nullfont}}
`\verb|\nullfont|'.\footnote{Haben sie sich nicht schon gefragt, wof\"ur
dieser Zeichensatz ben\"otigt wird?} Die \index{Schrift!Familie}
Schriftfamilie wird erst {\em
nach} Abarbeitung der Formel ausgewertet. Angenommen, sie haben auf
ihrem Rechner eine \index{Helevetica}\index{Schrift!Helevetica}
Heleveticaschrift zur Verf\"ugung, dann erhalten sie
nach der Anweisung
\index{zzzhelevetica@{\tt\char92 helevetica}}
\index{zzztextfont@{\tt\char92 textfont}}
\begin{verbatim}
$\textfont0=\tenrm 9 \textfont0=\helevetica 9$
\end{verbatim}
zwei Neuen in der Schriftart Helevetica, da diese Anweisung am Ende
der Formel g\"ultig ist. Anders sieht es aus, wenn sie
\index{zzzhelevetica@{\tt\char92 helevetica}}
\index{zzztextfont@{\tt\char92 textfont}}
\begin{verbatim}
$\textfont0=\tenrm 9\hbox{$9\textfont0=\helevetica$}$
\end{verbatim}
eingeben, dann erhalten sie die erste Neun in `tenrm' und die zweite
in `helevetica'.
\subsection{Zeichenklassen}
Zu jedem mathematischen Zeichen gibt es einen Code zwischen $0$ und
$4096$, der sich aus der Positionsnummer innerhalb des Zeichensatzes
plus dem $256$fachen der \index{Familie!Nummer}
Familiennummer ergibt. Am einfachsten wird
das \index{Hexadezimal}hexadezimal notiert. Die Kodierung `\verb|24A|' bedeutet, da\ss{} es
sich um das {\tt 2A}te Zeichen in der Familie {\tt 2} handelt. Au\ss{}er
der Familienzugeh\"origkeit und der Position innerhalb eines
Zeichensatzes gibt es aber auch noch eine Klassifizierung der Zeichen.

\index{Interpunktion}\index{Schliesser@Schlie\ss{}er}
\medskip\begin{tabular}{lll}
{\em Klasse} & {\em Bedeutung} & {\em Beispiel} \\
0 & Normal & $/$\\
1 & Gro\ss{}es Symbol & $\sum$\\
2 & Bin\"arer Operator & $+$\\
3 & Relation & $=$\\
4 & `\"Offner' & $($\\
5 & `Schlie\ss{}er' & $)$\\
6 & Interpunktion & $,$\\
7 & Variable Familie & $x$
\end{tabular}\medskip

Die Klassen $0$ bis $6$ besagen, welchem Sprachteil das Zeichen
zugeh\"ort, die Klasse $7$ wird sofort besprochen. Die Klassenkodierung
wird mit $4096$ multipliziert der Zeichenkodierung zuaddiert, so da\ss{}
sich eine vierstellige \index{Hexadezimal}hexadezimale Zahl ergibt. {\tt 1350} bedeutet:
Klasse {\tt 1} (Gro\ss{}es Symbol), in der \index{Familie}
Familie {\tt 3}, und dort das
Zeichen {\tt 50} (Alle Zahlangeben hexadezimal).

Die Klasse $7$ erlaubt es Zeichen in verschiedenen Familien
aufzutauchen. Normalerweise verhalten sich Zeichen der Klasse $7$ wie
die der Klasse $0$, es sein denn die \index{Integer!Variable}
Integervariable \index{zzzfam@{\tt\char92 fam}}`\verb|\fam|' hat
einen g\"ultigen Familienwert ($0\ldots15$). Normalerweise wird der Wert
auf $-1$ gesetzt, kann aber ge\"andert werden. {\em Plain} \TeX\ benutzt diese
M\"oglichkeit f\"ur den Befehl \index{zzzrm@{\tt\char92 rm}}
`\verb|\rm|'. Er ist eine
\index{Abkuerzung@Abk\"urzung}
Abk\"urzung f\"ur die
Befehle `\verb|\fam=0|' und `\verb|\tenrm|'. Die Familie aller Zeichen
(haben normalerweise die Klasse $7$) wird also $0$ und die
\index{Schrift!Familie}Schrift
{\tt tenrm} wird als voreingestellte \index{Schrift!Voreingestellt}
Schrift verwendet. Mit diesem
Befehl ist es dann m\"oglich innerhalb mathematischer Formeln normalen
Text zu setzen.
\subsection{Mathcode}
Von einer Tabelle mit $256$ ``mathcode''-Werten h\"angt die Interpretation
der Zeichen im mathematischen Modus ab. Der zu einem Zeichen geh\"orige Wert kann
mit dem Befehl \index{zzzmathcode@{\tt\char92 mathcode}}
`\verb|\mathcode|' ge\"andert werden.\footnote{Entspricht
soweit dem Befehl {\tt\char92 catcode}, der schon behandelt wurde.}
Der Befehl
\begin{verbatim}
\mathcode`<=\"313C
\end{verbatim}
bedeutet, da\ss{} dem Zeichen `\verb|<|' der Code `\verb|313C|' zukommt,
also $3$te \index{Klasse!Zeichencode}
Klasse in der $1$ten Familie und dort Zeichen Nummer
`\verb|3C|'. Der Buchstabe `b' wird z.B.\ mit `\verb|7162|' kodiert.
Es gibt noch den `mathcode' `\verb|8000|', der zur Folge hat, da\ss{} das
Zeichen wie ein \index{Aktiv!Zeichen}
aktives Zeichen (catcode=13) behandelt wird.

Analog zu dem Befehl `\verb|\char|' gibt es auch einen Befehl um jedes
mathematische Zeichen ansprechen zu k\"onnen, und der hei\ss{}t, wie nicht
anders zu erwarten war, \index{zzzmathchar@{\tt\char92 mathchar}}
`\verb|\mathchar|'. Die \index{Definition}
Definition des
\index{Summe!Zeichen}Summenzeichens w\"are also m\"oglich mit der Befehlsfolge
\index{zzzdef@{\tt\char92 def}}
\index{zzzsum@{\tt\char92 sum}}
\begin{verbatim}
\def\sum=\mathchar\"1350
\end{verbatim}
(Klasse 1, gro\ss{}es Symbol; Familie 3, Zeichen 50) Aber es gibt auch
noch eine bessere M\"oglichkeit, denn genau, wie es den Befehl
\index{zzzchardef@{\tt\char92 chardef}}
`\verb|\chardef|' gibt, gibt es auch den Befehl
\index{zzzmathchardef@{\tt\char92 mathchardef}}`\verb|\mathchardef|'
und damit reduziert sich die Aufgabe zu
\index{zzzsum@{\tt\char92 sum}}
\begin{verbatim}
\mathchardef\sum=\"1350
\end{verbatim}

\TeX\ kann auch einer ganzen Teilformel eine \index{Klasse!Einer
Teilformel}Klasse zukommen lassen.
Dazu dienen die Befehle
\index{zzzmathbin@{\tt\char92 mathbin}}\index{zzzmathclose@{\tt\char92 mathclose}}
\index{zzzmathop@{\tt\char92 mathop}}\index{zzzmathopen@{\tt\char92 mathopen}}
\index{zzzmathord@{\tt\char92 mathord}}\index{zzzmathpunkt@{\tt\char92 mathpunkt}}
\index{zzzmathrel@{\tt\char92 mathrel}}
\begin{verbatim}
\mathord
\mathop
\mathbin
\mathrel
\mathopen
\mathclose
\mathpunkt
\end{verbatim}
Sie beziehen sich immer auf das Zeichen, oder die gruppierte
Teilformel, die dahinter steht. Die Anweisung
\begin{verbatim}
$G\mathbin:H$
\end{verbatim}
sorgt daf\"ur, da\ss{} der \index{Doppelt!Punkt}
Doppelpunkt wie ein Zeichen eines bin\"aren
\index{Operator!Binaer@Operator!Bin\"ar}
Operators verwendet wird. Der Befehl `\verb|\mathord|' ist eigentlich
\"uberfl\"u\ss{}ig, da alles in \index{Klammer!In Formel}Klammern wie ein normales Symbol verwendet
wird. Die beiden folgenden Anweisungen sind also gleichwertig
\index{zzzmathord@{\tt\char92 mathord}}
\begin{verbatim}
$1\mathord,2345$
$1{,}2345$
\end{verbatim}
Es gibt noch eine achte Klassifizierung: \index{zzzmathinner@{\tt\char92 mathinner}}
`\verb|\mathinner|', die f\"ur
spezielle Zwecke gebraucht wird. \index{Bruch}Br\"uche und
Rechts-Links-Konstruktionen werden als {\em inner} bezeichnet, und mit
etwas zus\"atzlichem \index{Leer!Raum}
Leerraum umgeben. Normalerweise werden Teilformeln
als normales Symbol angesehen.
\subsection{Delcode}\index{Klammer!Code}
Klammern werden noch weiter spezifiziert, sie haben neben dem {\tt
catcode} und dem {\tt mathcode} auch noch einen sechstelligen,
\index{Hexadezimal}hexadezimalen {\tt delcode}.\footnote{Vom englischen \index{Delimiter}
{\em Delimiter}.}
Die ersten drei \index{Ziffer}
Ziffern geben dabei das {\em kleine}, die zweiten drei
Ziffern das {\em gro\ss{}e} Symbol an. Die Anweisung
\index{zzzdelcode@{\tt\char92 delcode}}
\begin{verbatim}
\delcode`(=\"028300 \delcode`.=0
\end{verbatim}
besagt, da\ss{} f\"ur die \"offnende Klammer als kleines Symbol das {\tt 28}te
Zeichen aus der \index{Familie}
Familie {\tt0} in Frage kommt, und die gro\ss{}e Variante
in Familie {\tt3} an der Position {\tt0} zu finden ist. Im Falle des
\index{Punkt}
Punktes wird keine gro\ss{}e und keine kleine Variante angegeben, so da\ss{}
es auch zu keiner \index{Ausgabe}
Ausgabe kommt.\footnote{Erinnern sie sich an die
Bedeutung des Punktes in Zusammenhang mit Klammern?} Eine Klammer kann
auch direkt mit dem Befehle \index{Delimiter!Befehl}
\index{zzzdelimiter@{\tt\char92 delimiter}}
`\verb|\delimiter|' angegeben werden. Der
zugeh\"orige Zahlwert mu\ss{} nur kleiner als `\verb|\"7FFFFFF|' sein. Die
Angabe enth\"alt sofort auch die \index{Klasse!Angabe}Klassenangabe.
\index{zzzdef@{\tt\char92 def}}\index{zzzdelimiter@{\tt\char92 delimiter}}
\begin{verbatim}
\def\langle{\delimiter\"426830A}
\end{verbatim}
Bedeutet, da\ss{} \index{zzzlangle@{\tt\char92 langle}}
`\verb|\langle|' ein \"offnendes Symbol ist, dessen kleine
Variante bei `\verb|268|' zu finden ist, und die gro\ss{}e Variante bei
`\verb|30A|'. Tritt der `\verb|\delimiter|' Befehl nach einem
\index{zzzleft@{\tt\char92 left}}\index{zzzrightarrow@{\tt\char92 rightarrow}}
`\verb|\left|' oder einem `\verb|\right|' Befehl auf, dann wird die
Klassenangabe ignoriert, in anderem Kontext werden die letzten drei
Stellen gestrichen und die ersten vier \index{Ziffer}Ziffern wie bei einer
\index{zzzmathchar@{\tt\char92 mathchar}}
`\verb|\mathchar|'  Angabe verwendet. Die folgenden Angaben sind also
\"aquivalent
\index{zzzlangle@{\tt\char92 langle}}
\begin{verbatim}
$\langle x$
$\mathchar\"4268 x$
\end{verbatim}
\subsubsection{Andere \"ahnliche Befehle}
Wie auch der `\verb|\delimiter|' Befehl arbeiten die Befehle
\index{zzzradical@{\tt\char92 radical}}
`\verb|\radical|' und \index{zzzmathaccent@{\tt\char92 mathaccent}}
`\verb|\mathaccent|', die dazu dienen
Wurzelzeichen, bzw.\ mathematische \index{Akzent!mathematische}
Akzente zu definieren. Die
Kodierung ist auch hier wieder dieselbe, wie schon in den vorherigen
Beispielen.
\subsection{Familienzugeh\"origkeit nach {\em Plain} \TeX}
{\em Plain} \TeX\ setzt die \index{Familie!Plain TeX@Familie!{\it
Plain} \TeX}Familien normalerweise wie folgt: Familie $1$
f\"ur die \index{Schrift!Italic}\index{Italic}
Italic-Schriften, Familie $2$ f\"ur normale mathematische
Symbole, Familie $3$ f\"ur gro\ss{}e Symbole. Die Familien $2$ und $3$ sind
meist mit den Zeichens\"atzen `\verb|cmsy|' und `\verb|cmex|' verbunden,
die noch weitere Informationen zum Satz beinhalten.
\subsection{Elemente einer mathematischen Liste}
Sie k\"onnen sich vermutlich schon denken, da\ss{} auch mathematische Formeln
von \TeX\ zun\"achst wie eine lange Liste von Elementen angesehen wird,
wie es auch schon bei der \index{Horizontal!Liste}
horizontalen und der vertikalen Liste der
Fall war. In einer mathematischen Liste k\"onnen nun folgende Elemente
auftreten.\index{Bruch}\index{Display!Stil}
\begin{enumerate}
\item Ein Atom (Siehe unten)
\item Horizontales Material (Ein Strich, eine Trennung, eine Strafe
oder ein ``WasDenn'')
\item Vertikales Material
\item Etwas \index{Leim}Leim
\item Ein Kern
\item Ein \index{Stil!Wechsel}
Stilwechsel (von den Befehlen {\tt\char92 displaystyle} o.\"a.)
\item Ein Bruch o.\"a.
\item Ein B\"undel (meist eine Teilformel, die z.B. in Klammern gesetzt
ist)
\item Eine Mathe-Auswahl von {\tt\char92 mathchoise}
\end{enumerate}\index{Atom}
\subsection{Atome}
Das wichtigste Element in einer mathematischen Liste ist das sog.\
Atom. Es besteht aus einem {\em K\"orper}, einer \index{Hochstellung}
{\em Hochstellung} und
einer {\em Tiefstellung}. Wenn sie z.B.\ eingeben
\index{zzzoverline@{\tt\char92 overline}}
\begin{verbatim}
(x_i+y)^{\overline{n+1}}
\end{verbatim}
dann erhalten sie eine Folge von f\"unf Atomen: $($, $x_i$, $+$, $y$ und
$)^{\overline{n+1}}$. Die K\"orper dieser Atome sind: $($, $x$, $+$, $y$
und $)$, ihre \index{Hochstellung}Hochstellungen sind leer, bis auf die 
Hochstellung des
letzten Atoms, die aus $\overline{n+1}$ besteht, ihre Tiefstellungen
sind auch leer, bis auf die des zweiten Atoms, welche aus $i$ besteht.
Es gibt verschiedene Atomtypen.

\index{Box!Vbox}\index{Interpunktion!Atom}\index{Punct}
\index{zzzhat@{\tt\char92 hat}}\index{zzzoverline@{\tt\char92 overline}}
\medskip\begin{tabular}{ll}
Ord & Ein einfaches Atom wie $x$\\
Op  & Ein gro\ss{}es Atom wie $\sum$\\
Bin & Ein bin\"ares Atom wie $+$\\
Rel & Eine Relation wie $=$\\
Open & Ein \"offnendes Atom wie $($\\
Close & Ein schlie\ss{}endes Atom wie $)$\\
Punct & Ein Interpunktionsatom wie $,$\\
Inner & Ein inneres Atom wie $\frac{1}{2}$\\
Over & Eine \"uberstrichenes Atom wie $\overline{x}$\\
Under & Ein unterstrichenes Atom wie $\underline{x}$\\
Acc & Ein Akzentatom wie $\hat a$\\
Rad & Ein Wurzelatom wie $\sqrt{2}$\\
Vcent & Eine vertikale Box, von {\tt\char92 vcenter}
\end{tabular}\medskip\index{Atom!Typen}

Der K\"orper, die \index{Hochstellung}
Hochstellung und die Tiefstellung eines Atoms hei\ss{}en
{\em Felder}. F\"ur diese Felder gibt es vier M\"oglichkeiten, ein Feld
enth\"alt\index{Box}
\begin{enumerate}
\item nichts
\item ein mathematisches Symbol
\item eine Box
\item eine mathematische Liste
\end{enumerate}
Den genauen Aufbau dieser Atome k\"onnen sie sich mit
\index{zzzshowlists@{\tt\char92 showlists}}`\verb|\showlists|'
ansehen. Manche Atome haben \"ubrigens weitere Informationen, so sind
z.B.\ alle `Op' Atome mit der zus\"atzlichen Information \"uber
\index{zzzlimits@{\tt\char92 limits}}\index{zzznolimits@{\tt\char92 nolimits}}
`\verb|\limits|' oder `\verb|\nolimits|' versehen.
\section{Sch\"one Aussichten}
Es sind nun schon fast alle M\"oglichkeiten von \TeX\ besprochen, eine
Formel zu setzen, wenn man allerdings mit der Zeit etwas \"Ubung darin
bekommen hat, zu erkennen, wie eine Formel aussieht, wenn man sie
schreibt, dann kommt der Wunsch auf Formeln nicht nur einfach zu
setzen, sondern ihr Aussehen noch positiv zu beeinflu\ss{}en. Hier nun
einige Regeln daf\"ur.
\subsection{Interpunktion}
Punkt und \index{Komma}Kommata sollten in Textformeln nicht zwischen die
`\verb|$|'s geschrieben werden, wenn sie nicht zur Formel geh\"oren.
Eine Textformel
\begin{verbatim}
f\"ur $x=a, b$, oder $c$
\end{verbatim}
schreibt man besser als
\begin{verbatim}
f\"ur $x=a$, $b$, oder~$c$
\end{verbatim}
Der Grund hierf\"ur liegt darin, da\ss{} {\em in} einer Formel die Abst\"ande
hinter Interpunktionszeichen anders gesetzt werden, als im Text, man
erhielte also unterschiedlich breite \index{Leer!Raum}
Leerr\"aume, was h\"ochst unsch\"on
aussieht. Au\ss{}erdem bricht \TeX\ eine Formel hinter einem
Interpunktionszeichen nur {\em sehr} ungerne um, was aber in diesem
Fall gerade w\"unschenswert w\"are.\footnote{Der Grund f\"ur das Verhalten
von \TeX\ liegt darin, da\ss{} man ja \"ublicherweise nicht eine Formel wie
$f(a,b)$ in $f(a,$ und $b)$ getrennt haben m\"ochte.}

Innerhalb einer \index{Formel!abgesetzte}
abgesetzten Formel sieht die Sache etwas anders aus.
Hier sollte das Interpunktionszeichen {\em in} die Formel geschrieben
werden. Der Grund wird sofort klar, wenn man sich die beiden folgenden
Beispiele ansieht.
\begin{verbatim}
$$f(a,b).$$
\end{verbatim}
ergibt das gew\"unschte
$$f(a,b).$$
wohingegen
\begin{verbatim}
$$f(a,b)$$.
\end{verbatim}
etwas seltsam aussieht:
$$f(a,b)$$.
Achten sie darauf, wo der \index{Punkt}Punkt hingekommen ist.
\subsubsection{Mehr Platz}
Man kann Textformeln auch dadurch hervorheben, da\ss{} man ihnen etwas
mehr Platz verschaftt. \TeX\ bietet diese M\"oglichkeit mit dem Befehl
\index{zzzmathsurround@{\tt\char92 mathsurround}}
`\verb|\mathsurround|'. Die Textformel
\begin{verbatim}
$x=a$, $b$, oder $c$
\end{verbatim}
sieht mit den Vorgaben f\"ur `\verb|\mathsurround|' f\"ur ein, bzw.\ zwei
pt folgenderma\ss{}en aus

{\mathsurround=1pt$x=1$, $b$, oder $c$}

{\mathsurround=2pt$x=1$, $b$, oder $c$}

Normalerweise wird der Wert auf 0pt gesetzt.
\subsection{Buchstaben in Nicht-Italic}
\subsubsection{Generelles}
Normalerweise werden Buchstaben in Formeln immer in der
\index{Schrift!Art}\index{Schrift!Italic}Schriftart
\index{Italic}Italic gesetzt. Bei \index{Funktion!Namen}
Funktionsnamen o.\"a.\ ist dies aber st\"orend. \TeX\
bietet daher einige Befehle, die die entsprechenden Funktionsnamen in
der \index{Schrift!Roman}\index{Schrift!Art}
Schriftart \index{Roman}Roman ausgeben.
\index{zzzarccos@{\tt\char92 arccos}}
\index{zzzarcsin@{\tt\char92 arcsin}}
\index{zzzarctan@{\tt\char92 arctan}}
\index{zzzarg@{\tt\char92 arg}}
\index{zzzcos@{\tt\char92 cos}}
\index{zzzcosh@{\tt\char92 cosh}}
\index{zzzcot@{\tt\char92 cot}}
\index{zzzcoth@{\tt\char92 coth}}
\index{zzzcsc@{\tt\char92 csc}}
\index{zzzdeg@{\tt\char92 deg}}
\index{zzzdet@{\tt\char92 det}}
\index{zzzdim@{\tt\char92 dim}}
\index{zzzexp@{\tt\char92 exp}}
\index{zzzgcd@{\tt\char92 gcd}}
\index{zzzhom@{\tt\char92 hom}}
\index{zzzinf@{\tt\char92 inf}}
\index{zzzker@{\tt\char92 ker}}
\index{zzzlim@{\tt\char92 lim}}
\index{zzzliminf@{\tt\char92 liminf}}
\index{zzzlog@{\tt\char92 log}}
\index{zzzmax@{\tt\char92 max}}
\index{zzzmin@{\tt\char92 min}}
\index{zzzsec@{\tt\char92 sec}}
\index{zzzsin@{\tt\char92 sin}}
\index{zzzsinh@{\tt\char92 sinh}}
\index{zzzsup@{\tt\char92 sup}}
\index{zzztan@{\tt\char92 tan}}
\index{zzztanh@{\tt\char92 tanh}}
\index{zzzlimsup@{\tt\char92 limsup}}
\begin{verbatim}
\arccos  \cos   \csc   \exp   \ker     \limsup  \min  \sinh
\arcsin  \cosh  \deg   \gcd   \lg      \ln      \Pr   \sup
\arctan  \cot   \det   \hom   \lim     \log     \sec  \tan
\arg     \coth  \dim   \inf   \liminf  \max     \sin  \tanh
\end{verbatim}
Hier einige Beispiele, die auch zeigen, da\ss{} immer der richtige
\index{Leer!Raum}Leerraum um die Funktionsnamen gesetzt wird.

\index{zzzover@{\tt\char92 over}}
\medskip\begin{tabular}{ll}
{\em Eingabe} & {\em Ausgabe}\\
{\tt\char36\char92 sin2\char92 theta=2\char92 sin\char92 theta\char92
cos\char92 theta\char36} & $\sin2\theta=2\sin\theta\cos\theta$\\
{\tt\char36 O(n\char92 log n\char92 log \char92 log n)\char36} &
$O(n\log n\log\log n)$\\
{\tt\char36\char92 Pr(X\char62 x)=\char92 exp(-x/\char92 mu)\char36} &
$\Pr(X>x)=\exp(-x/\mu)$\\
{\tt\char36\char36\char92 max\char95\char123 1\char92 le n\char92 le
m\char125\char92 log\char95 2P\char95 n\char36\char36} &
$\displaystyle{\max_{1\le n\le m}\log_2P_n}$\\
{\tt\char36\char36\char92 lim\char95\char123 x\char92
to0\char125\char123\char92 sin x\char92 over x\char125
=1\char36\char36} & $\displaystyle{\lim_{x\to0}{\sin x\over x}=1}$
\end{tabular}\medskip

Wie man am Beispiel des \index{zzzmax@{\tt\char92 max}}
`\verb|\max|'-Operators sieht werden manche
dieser \index{Kontrollsequenz}Kontrollsequenzen als gro\ss{}e Symbole behandelt.

Mitunter reichen diese Befehle nicht aus. Sollten sie nur ab und zu
einen anderen Befehle benutzen m\"ussen, dann sollten sie es per Hand
machen, tritt die Situation allerdings h\"aufiger auf, dann sollten sie
sich einen entsprechenden Befehl schaffen. Hier zun\"achst die
M\"oglichkeit als solche Text in Formeln in einer anderen
\index{Schrift!Art}Schriftart zu setzen.

Am einfachsten erreichen sie das gew\"unschte Ziel, wenn sie den Befehl
\index{zzzrm@{\tt\char92 rm}}`\verb|\rm|' verwenden. Die Eingaben von
\index{zzzsqrt@{\tt\char92 sqrt}}
\begin{verbatim}
$$\sqrt{{\rm Var}(X)}$$
$$x_{\rm max}-x_{\rm min}$$
\end{verbatim}
f\"uhren zu dem Ergebnis
$$\sqrt{{\rm Var}(X)}$$
und
$$x_{\rm max}-x_{\rm min}$$
Denken sie allerdings daran, da\ss{} sie \index{Leer!Zeichen}
Leerzeichen in derartigen Texten
mit der \index{Kontrollsequenz}
Kontrollsequenz `\verb*|\ |' schreiben, da sie sonst ignoriert
w\"urden. Es geht auch mit dem Befehle \index{Box!Hbox}
`\verb|\hbox|', aber Achtung, es k\"onnen \index{Fehler!Bei Hbox}Fehler auftreten.
\index{Groesse!Schrift@Gr\"o\ss{}e!Schrift}\index{Schrift!Groesse@Schrift!Gr\"o\ss{}e}
\index{Schrift!Art}
\begin{enumerate}
\item Wenn sie den {\tt\char92 hbox} Befehl verwenden, wird die
Schriftgr\"o\ss{}e nicht ge\"andert. Die \index{Eingabe}Eingabe von {\tt\char36
x\char95\char123\char92 hbox\char123 min\char125\char125\char36}
ergibt die \index{Ausgabe}Ausgabe: $x_{\hbox{min}}$.
\item Es ist nicht unbedingt sichergestellt, da\ss{} sie auch wirklich die
Schriftart erreichen, die sie haben wollen. Der Text in der
{\tt\char92 hbox} wird in der Schrift gesetzt, die au\ss{}erhalb der
Formel g\"ultig ist. Wenn sie also z.B.\ die Schriftart \index{Boldface}Boldface
benutzen, dann w\"urde auch ihr Text in Boldface gesetzt.
\end{enumerate}
\subsubsection{Moduln}
F\"ur Modulnangaben existieren zwei \index{Kontrollsequenz}
Kontrollsequenzen. Einmal der Befehl
\index{zzzbmod@{\tt\char92 bmod}}`\verb|\bmod|', der wie ein bin\"arer
\index{Operator!Binaer@Operator!Bin\"ar}Operator gesetzt wird und
innerhalb von Formeln verwendet werden sollte, und andererseits der
Befehl \index{zzzpmod@{\tt\char92 pmod}}
`\verb|\pmod|', der am Ende von Formeln verwendet werden
sollte, der wie eine \index{Klammer!Bei Moduln}
Klammer gesetzt wird. Der Befehl `\verb|\pmod|'
setzt \"ubrigens eigene Klammern, die Eingabe von
\begin{verbatim}
$$x\equiv y+1\pmod{m^2}$$
\end{verbatim}
ergibt
$$x\equiv y+1\pmod{m^2}$$
\subsubsection{\"Andern der Schriftart}
\"Anderungen der Schriftart durch die normalen Befehle wirkt sich nur
auf die normalen Buchstaben, die \index{Griechische Buchstaben}
griechischen \index{Grossbuchstaben@Gro\ss{}buchstaben}Gro\ss{}buchstaben und die
\index{Akzent!mathematische}
Akzente aus. Die \index{Dollarzeichen}
Dollarzeichen, die eine Formel umschlie\ss{}en wirken
dabei wie eine \index{Gruppe!Formel}
Gruppenklammer, so da\ss{} die Schriftart\"anderung lokal bleibt.

\index{zzzphi@{\tt\char92 phi}}
{\tt\char36\char92 bf a+b=\char92 Phi\char95 m\char36} ergibt $\bf
a+b=\Phi_m$

Neben den \"ublichen Umschaltbefehlen existiert im mathematischen Modus
auch Befehl
\index{zzzcal@{\tt\char92 cal}}`\verb|\cal|', der \index{Kalligraphie}
kalligraphische Buchstaben erzeugt,
allerdings {\em nur} \index{Grossbuchstaben@Gro\ss{}buchstaben}
Gro\ss{}buchstaben. Die Angabe von `\verb|$\cal A$|'
ergibt $\cal A$.

Ein weiterer Befehl ist \index{zzzmit@{\tt\char92 mit}}
`\verb|\mit|', der f\"ur \index{Italic}``mathematisches
Italic'' steht. Mit diesem Befehle werden auch die griechischen
\index{Grossbuchstaben@Gro\ss{}buchstaben}
Gro\ss{}buchstaben in \index{Italic}Italic ausgegeben. Auf die normalen Buchstaben hat
dieser Befehl keinen Einflu\ss{}, genau, wie der Befehl
\index{zzzrm@{\tt\char92 rm}}`\verb|\rm|' auf
die griechischen Gro\ss{}buchstaben keinen Einflu\ss{} hat. Worin besteht nun
der Unterschied zwischen mathematischem Italic und normalem Italic, am
besten zeigt das ein Beispiel

\medskip\begin{tabular}{ll}
{\tt\char36 Dies\char92\ ist\char92\ mathematisches\char92\
Italic\char36} & $Dies\ ist\ mathematisches\ Italic$\\
{\tt\char123\char92 it Das ist Text Italic\char125} & {\it Das ist
Text Italic}
\end{tabular}\medskip

Die mathematischen Buchstaben sind etwas breiter und sie werden mit
anderem \index{Leer!Raum}Leerraum gesetzt.
\subsection{Platz zwischen Formeln}
Mitunter sollen, vor allem in \index{Formel!abgesetzte}
abgesetzten Formeln, mehr als eine
Formel gesetzt werden, die voneinander durch etwas Platz getrennt
werden sollen. Der normale \index{Abstand!In Formeln}
Abstand hilft hier nicht weiter. Der
Formelsatz
$$F_n=F_{n-1}+F_{n-2},n\ge2$$
ist nicht gerade das, was man erwarten w\"urde. Grunds\"atzlich ist es
empfehlenswert derartige Formeln durch Text zu trennen, also
$$F_n=F_{n-1}+F_{n-2},{\rm\ wenn\ }n\ge2$$
es geht aber auch mit \index{Zwischenraum}
Zwischenraum, wenn man sie einer der beiden
Befehle \index{zzzquad@{\tt\char92 quad}}
`\verb|\quad|' oder \index{zzzqquad@{\tt\char92 qquad}}
`\verb|\qquad|' bedient. Der Befehl
`\verb|\quad|' schafft etwas den Platz eines em, der andere Befehl
etwa das \index{Doppelt!Platz}doppelte. Die Formel
\begin{verbatim}
$$F_n=F_{n-1}+F_{n-2},\qquad n\ge2$$
\end{verbatim}
sieht dann so aus
$$F_n=F_{n-1}+F_{n-2},\qquad n\ge2$$
Derartigen Platz m\"ussen sie immer explizit mit solchen Befehlen
angeben.
\subsection{Platz {\em in} Formeln}
Meist ist die Positionierung von Formelteilen in einer Formel
hinreichend gut. Es kann aber vorkommen, da\ss{} die Vorgabe von \TeX\
doch nicht ganz den Erwartungen entspricht. Es w\"are unsinnig mit den
Befehlen `\verb|\quad|', \index{zzzqquad@{\tt\char92 qquad}}
`\verb|\qquad|' oder `\verb*|\ |' Platz zu
schaffen, da diese Befehle viel zu gro\ss{}e \index{Zwischenraum}
Zwischenr\"aume schaffen. In
Formeln werden von \TeX\ drei verschiedene Abst\"ande benutzt. Ein
gro\ss{}er \index{Abstand!Gleichheitszeichen}
Abstand wird rechts und links von Gleichheitszeichen verwendet
($2=3-1$), mittleren \index{Abstand!Binaere Zeichen@Abstand!Bin\"are Zeichen}
Abstand w\"ahlt \TeX\ vor und hinter bin\"aren
Zeichen, wie dem \index{Minuszeichen}
Minuszeichen, der kleinste Abstand ist noch kleiner,
er macht gerade den Unterschied zwischen `${\rm loglog}$' und `${\rm
log\, log}$' aus.

Sie k\"onnen diese \index{Zwischenraum}
Zwischenr\"aume auch selber in Formeln schreiben, wenn
sie die Befehle

\medskip\begin{tabular}{ll}
{\tt\char92 ,} & Kleiner Zwischenraum ($1/6$ eines quad)\\
{\tt\char92\char62} & Mittlerer Zwischenraum ($2/9$ eines quad)\\
{\tt\char92 ;} & Gro\ss{}er Zwischenraum ($5/18$ eines quad)\\
{\tt\char92 !} & negativer kleiner Zwischenraum
\end{tabular}\medskip

Die Gr\"o\ss{}enangaben sind nur im Normalfall richtig, da die Abst\"ande in
\index{Muglue}{\em muglue} definiert sind. Die Bezeichnung \index{Leim}
{\em glue}=Leim, kennen
wir bereits, die Vorsilbe {\em mu} bedeutet `mathematic units'. $18$
mu machen ein em aus, je nach aktueller Schriftart \"andern sich also
auch die Abst\"ande. Die \index{Definition}
Definition der Abst\"ande lautet
\index{zzzmedmuskip@{\tt\char92 medmuskip}}
\index{zzzthickmuskip@{\tt\char92 thickmuskip}}
\index{zzzthinmuskip@{\tt\char92 thinmuskip}}
\begin{verbatim}
\thinmuskip = 3mu
\medmuskip = 4mu plus 2mu minus 4mu
\thickmuskip = 5mu plus 5mu
\end{verbatim}
Sie k\"onnen den Platz, den sie in eine Formel zus\"atzlich hineinbringen
wollen auch sofort in \index{Muglue}muglue angeben, dazu dient der Befehl
\index{zzzmuskip@{\tt\char92 muskip}}
`\verb|\muskip|', der wie der {\tt hskip} Befehl arbeitet, allerdings
verlangt er als \index{Argument!Bei Muskip}Argument {\em immer} mu's.
\subsubsection{Beispiele der Anwendung}
Als erstes Beispiel dient der $dx$ Ausdruck am Ende eines
\index{Integral}Integrals.
Normalerweise w\"urde er zu nahe an den Inhalt des Integrals gesetzt.

\index{zzzinfty@{\tt\char92 infty}}\index{zzzint@{\tt\char92 int}}
\medskip\begin{tabular}{ll}
{\tt\char26\char92 int\char95 0\char94\char92 infty f(x)\char92
,dx\char36} & $\int_0^\infty f(x)\,dx$\\
{\tt\char36 x\char92 ,dy/dx\char36} & $x\,dx/dy$
\end{tabular}\medskip

Im letzten Beispiel ist der Befehl vor dem $dy$ nat\"urlich \"uberfl\"u\ss{}ig.
Auch \index{Einheit!Mass@Einheit!Ma\ss{}}
Einheiten sollten derartig abgesetzt werden

\medskip\begin{tabular}{ll}
{\tt\char36 55\char92 rm\char92 ,km/h\char36} & $55\rm\, km/h$\\
{\tt\char36 g=9,8\char92 rm\char92 ,m/sec\char94 2\char36} &
$g=9,8\rm\, m/sec^2$
\end{tabular}\medskip

Auch hinter \index{Fakultaet@Fakult\"at}
Fakult\"atszeichen, hinter denen noch etwas folgt, sollten
sie etwas zus\"atzlichen \index{Zwischenraum}Zwischenraum setzen

\medskip\begin{tabular}{ll}
{\tt\char36 3!\char92 ,4!\char92 ,5!\char36} & $3!\,4!\,5!$
\end{tabular}\medskip

Weitere Beispiele sehen sie nun
\index{zzzbigl@{\tt\char92 bigl}}\index{zzzbigr@{\tt\char92 bigr}}
\index{zzzint@{\tt\char92 int}}

\medskip\begin{tabular}{ll}
{\tt\char36\char92 sqrt2\char92 ,x\char36} & $\sqrt2\,x$\\
{\tt\char36\char92 sqrt\char123\char92 ,\char92 log x\char125\char36}
& $\sqrt{\,\log x}$\\
{\tt\char36 O\char92 bigl(1/\char92 sqrt\ n\char92 ,\char92
bigr)\char36} & $O\bigl(1/\sqrt n\,\bigr)$\\
{\tt\char36\char91\char92 ,0,1)\char36} & $[\,0,1)$\\
{\tt\char36\char36\char92 int\char92 !\char92 !\char92 !\char92
int\char95 D dx\char92 ,dy\char36\char36} &
$\displaystyle{\int\!\!\!\int_D dx\,dy}$
\end{tabular}\medskip

In der folgenden Tabelle werden alle \index{Zwischenraum}
Zwischenr\"aume erkl\"art, die \TeX\
automatisch zwischen \index{Atom!Einer Formel}
Formelatomen setzt. Dabei stellen $0$, $1$, $2$
und~$3$  keinen, kleinen, mittleren und gro\ss{}en Zwischenraum dar. Die
Angaben stehen in Klammern, wenn der Zwischenraum nur im
\index{Display!Stil}Display- und
im Textstil gesetzt wird und sonst nicht. Wenn ein Stern in der
Tabelle auftaucht, dann ist damit eine Situation gemeint, die nie
auftreten kann.

\medskip\begin{scriptsize}
\begin{tabular}{lc}
 & Rechtes Atom \\
\begin{tabular}{ll}
 & \strut \\
 & \strut Ord \\
 & \strut Op \\
 & \strut Bin \\
Linkes & \strut Rel \\
Atom & \strut Open \\
 & \strut Close \\
 & \strut Punct \\
 & \strut Inner
\end{tabular} & \fbox{
\begin{tabular}{cccccccc}
Ord & Op & Bin & Rel & Open & Close & Punct & Inner \\
0 & 1 & (2) & (3) & 0 & 0 & 0 & (1) \\
1 & 1 & $*$ & (3) & 0 & 0 & 0 & (1) \\
(2) & (2) & $*$ & $*$ & (2) & $*$ & $*$ & (2) \\
(3) & (3) & $*$ & 0 & (3) & 0 & 0 & (3) \\
0 & 0 & $*$ & 0 & 0 & 0 & 0 & 0 \\
0 & 1 & (2) & (3) & 0 & 0 & 0 & (1) \\
(1) & (1) & $*$ & (1) & (1) & (1) & (1) & (1) \\
(1) & 1 & (2) & (3) & (1) & 0 & (1) & (1)
\end{tabular}}
\end{tabular}
\end{scriptsize}\medskip

Die Regeln von \TeX\ sind manchmal etwas \index{Fehler!Der Regeln}
fehlerhaft, besonders, wenn
`\verb+|+' und `\verb+\|+' in einer Formel auftreten, da die Zeichen
$|$ und $\|$ als normale Symbole und nicht als \index{Klammer!Fehler}
Klammern aufgefa\ss{}t werden.

\medskip\begin{tabular}{ll}
{\em Eingabe} & {\em Ausgabe}\\
{\tt\char36\char124 -x\char124=\char124 +x\char124\char36} &
$|-x|=|+x|$\\
{\tt\char36\char92 left\char124 -x\char92 right\char124 =\char92
left\char124 +x\char92 right\char124\char36} &
$\left|-x\right|=\left|+x\right|$\\
\end{tabular}\medskip

\subsection{Punkte}
\index{Mathematik}
Mathematiker brauchen \"ofters Punkte in ihren Formeln, um weitergehende
Folgen von Elementen anzuzeigen. Die Eingabe von `\verb|$x...y$|'
f\"uhrt zu dem unsch\"onen Ergebnis $x...y$. \TeX\ bietet statt dessen die
Befehle \index{zzzldots@{\tt\char92 ldots}}
`\verb|\ldots|' und \index{zzzcdots@{\tt\char92 cdots}}
`\verb|\cdots|' an, die drei Punkte auf
der \index{Linie!Basis}
Basislinie, resp.\ etwas dar\"uber setzen. Der erste Fall bietet
sich bei Elementen an, die durch \index{Komma}Kommata getrennt sind, z.B.\
$x_1,\ldots,x_n$, der zweite Befehl dient der Darstellung von
Elementen, die per Operation verkn\"upft sind, wie $x_1+\cdots+x_n$. Sie
sollten allerdings daran denken, da\ss{} diese Befehle nicht gen\"ugend
\index{Leer!Raum}
Leerraum lassen, speziell am Ende einer Formel, oder vor Klammern. Um
das korrekte Ergebnis
$$(1-x)^{-1}=1+x+x^2+\cdots\,.$$
zu erhalten, mu\ss{} man
\index{zzzcdots@{\tt\char92 cdots}}
\begin{verbatim}
$$(1-x)^{-1}=1+x+x^2+\cdots\,.$$
\end{verbatim}
eingeben. Da dieser Fall oft vorkommt, bietet \TeX\ f\"ur die
Befehlsfolge \index{zzzldots@{\tt\char92 ldots}}
`\verb|\ldots\,|' die einfache \index{Abkuerzung@Abk\"urzung}
Abk\"urzung \index{zzzdots@{\tt\char92 dots}}`\verb|\dots|'
an.

\subsection{Zeilenumbr\"uche}
Auch mathematische Formeln k\"onnen in Zeilen umgebrochen werden. \TeX\
trennt dabei nur nach einer \index{Relation}
Relation oder einem bin\"aren Symbol,
solange diese nicht in einer Gruppe oder einem \index{Bruch}
Bruch stehen. Die Formel
\begin{verbatim}
$f(x,y) = x^2-y^2 = (x+y)(x-y)$
\end{verbatim}
wird vorzugsweise bei den \index{Gleichheitszeichen}
Gleichheitszeichen umgebrochen, wenn das
nicht geht, eventuell auch noch nach dem Plus- bzw.\
\index{Minuszeichen}Minuszeichen. Wenn
man erreichen will, da\ss{} {\em nur} bei den Gleichheitszeichen
umgebrochen werden kann, dann mu\ss{} man die Formel als
\begin{verbatim}
$f(x,y) = {x^2-y^2} = {(x+y)(x-y)}$
\end{verbatim}
schreiben. Man kann in Formeln einen \"ahnlichen Befehl wie den
`\verb|\-|' Befehl benutzen, er lautet `\verb|\*|', allerdings
schreibt \TeX\ im Trennungsfall keinen \index{Bindestrich}
Bindestrich, sondern ein
\index{Multiplikation!Symbol}\index{zzztimes@{\tt\char92 times}}
Multiplikationssymbol ($\times$). Ebenso kann man, au\ss{}erhalb einer
Gruppierung, auch den Befehl \index{zzzallowbreak@{\tt\char92 allowbreak}}
`\verb|\allowbreak|' verwenden, um \TeX\
einen Hinweis zu geben, wo am besten umgebrochen wird.

Die \index{Strafe!Zeile!Formel}
Strafen f\"ur Zeilenumbr\"uche in Formeln sind

\medskip\begin{tabular}{llp{8cm}}
{\em Befehl} & {\em Strafe} & {\em Bedeutung} \\
{\tt\char92 relpenalty} & 500 & Umbruch nach einem Rel-Atom\\
{\tt\char92 binoppenalty} & 700 & Umbruch nach einem Bin-Atom\\
\end{tabular}\medskip\index{Atom}

Sie k\"onnen auch jederzeit in einer Formel die Strafe mit dem Befehl
\index{zzzpenalty@{\tt\char92 penalty}}
`\verb|\penalty<Nummer>|' selber angeben. Auch der Befehl
\index{zzznobreak@{\tt\char92 nobreak}}`\verb|\nobreak|' ist erlaubt.

\subsection{Geschweifte Klammern}\index{Klammer!Geschweifte}
Geschweifte Klammern werden in der \index{Mathematik}
Mathematik z.B.\ ben\"otigt, um
Mengen darzustellen. Bei einfachen Mengendarstellungen ist das auch
problemlos m\"oglich, denken sie nur daran, da\ss{} sie die geschweiften
Klammern mit den Befehlen `\verb|\{|' und `\verb|\}|' schreiben. Somit
ergibt `\verb|$\{a,b,c\}$|' das gew\"unschte Resultat $\{a,b,c\}$. Etwas
anders sieht es aus, wenn die Mengendarstellungen komplizierter
werden. Um die korrekte Abst\"ande in der Angabe
$$\{\,x\mid x>5\,\}$$
zu erreichen, m\"ussen sie
\begin{verbatim}
$$\{\,x\mid x>5\,\}$$
\end{verbatim}
eingeben. Noch etwas schwieriger wird die Sache, wenn die Klammern
gr\"o\ss{}er werden. Die \index{Ausgabe}Ausgabe
\index{zzzbigl@{\tt\char92 bigl}}\index{zzzbigm@{\tt\char92 bigm}}
\index{zzzbigr@{\tt\char92 bigr}}
$$\bigl\{\,\bigl(x,f(x)\bigr)\bigm|x\in D\,\bigr\}$$
wurde mit der Anweisung
\begin{verbatim}
$$\bigl\{\,\bigl(x,f(x)\bigr)\bigm|x\in D\,\bigr\}$$
\end{verbatim}
erreicht.

Ein anderes Beispiel, in dem geschweifte Klammern ben\"otigt werden ist
die Auswahl aus verschiedenen Alternativen. Mit
\index{zzzcases@{\tt\char92 cases}}
\begin{verbatim}
$$|x|=\cases{x,&wenn$x\ge0$;\cr -x,&anderenfalls.\cr}$$
\end{verbatim}
erreichen sie das Ergebnis
$$|x|=\cases{x,&wenn$x\ge0$;\cr -x,&anderenfalls.\cr}$$
Sie sehen hier erstmals die Verwendung des
`\verb|&|'-Zeichens.\footnote{Dieses Zeichen wird in
\index{Tabelle!Zeichen}Tabellen
verwendet. Da es sich hier um etwas \"ahnliches handelt, ist die
Verwendung dieses Zeichens hier notwendig. Auch die andern, in einer
Tabelle erlaubten Befehle sind hier erlaubt.} Mit dem Zeichen werden
zwei Spalten hinter der \"offnenden geschweiften Klammer, die von dem
Befehl automatisch eingesetzt wird, getrennt. Alles {\em vor} diesem
Zeichen wird automatisch in `\verb|$|' gesetzt, steht also im
mathematischen Modus. Alles hinter diesem Zeichen wird als normaler
Text behandelt. Die Zeilen, die schlie\ss{}lich die Auswahl erst
ausmachen, werden durch den Befehl `\verb|\cr|' getrennt. Es sind
\"ubrigens beliebig viele Zeilen m\"oglich.

Schlie\ss{}lich gibt es noch geschweifte \index{Klammer!Geschweifte, unter
Formeln}Klammern \"uber, bzw.\ unter
Formeln. Diese werden mit den Befehlen
\index{zzzoverbrace@{\tt\char92 overbrace}}`\verb|\overbrace|' und
\index{zzzunderbrace@{\tt\char92 underbrace}}
`\verb|\underbrace|' erreicht. Da diese Befehle wie gro\ss{}e Symbole
behandelt werden, ist es m\"oglich noch {\em Grenzen} anzugeben.
\index{zzzcdots@{\tt\char92 cdots}}\index{zzzoverbrace@{\tt\char92 overbrace}}
\begin{verbatim}
$$\overbrace{x+\cdots+x}^{k\rm\;mal}$$
$$\underbrace{x+y+z}_{>\,0}$$
\end{verbatim}
ergeben
$$\overbrace{x+\cdots+x}^{k\rm\;mal}$$
$$\underbrace{x+y+z}_{>\,0}$$

\subsection{Matrizen}
Eine weitere Anwendung, die einer Tabelle sehr \"ahnlich ist, sind die
Matrizen. Da auch sie \"ofter vorkommen, bietet \TeX\ einige Befehle zur
Konstruktion dieser Matrizen. Die \index{Matrix}Matrix
\index{zzzmatrix@{\tt\char92 matrix}}
\index{zzzlambda@{\tt\char92 lambda}}
$$A=\left(\matrix{x-\lambda&1&0\cr
                  0&x-\lambda&1\cr
                  0&0&x-\lambda\cr}\right)$$
wurde mit
\begin{verbatim}
$$A=\left(\matrix{x-\lambda&1&0\cr
                  0&x-\lambda&1\cr
                  0&0&x-\lambda\cr}\right)$$
\end{verbatim}
gesetzt. Es werden wieder die `\verb|&|' Zeichen ben\"otigt, um die
\index{Spalte!Tabelle}
Spalten, und die `\verb|\cr|' Befehle ben\"otigt, um die Zeilen zu
trennen. \TeX\ setzt die Matrize nicht automatisch in
\index{Klammer!Matrix}Klammern, da
aber Matrizen meist in runde Klammern gesetzt werden, gibt es den
Befehl \index{zzzpmatrix@{\tt\char92 pmatrix}}
`\verb|\pmatrix|', der dann die Klammern automatisch setzt.

Die einzelnen Eintr\"age der \index{Matrix}
Matrix werden normalerweise \index{Zentrierung}zentriert, und
durch ein quad getrennt, man kann die Ausrichtung aber durch
\index{zzzhfill@{\tt\char92 hfill}}
`\verb|\hfill|' Befehle \"andern. Die Eintr\"age werden einzeln gesetzt,
man sollte also nicht versuchen zwei Eintr\"age in eine Gruppenklammer
zu setzen.

Vielfach werden Matrizen nicht ganz ausgeschrieben. Neben den
\index{Punkt}
Punktbefehlen, die weiter oben beschrieben wurden, gibt es noch die
Befehle \index{zzzvdots@{\tt\char92 vdots}}
`\verb|\vdots|' f\"ur vertikale und `\verb|\ddots|' f\"ur
diagonale \index{Punkt}Punkte.\index{zzzddots@{\tt\char92 ddots}}
$$A=\pmatrix{a_{11}&a_{12}&\ldots&a_{1n}\cr
             a_{21}&a_{22}&\ldots&a_{2n}\cr
             \vdots&\vdots&\ddots&\vdots\cr
             a_{m1}&a_{m2}&\ldots&a_{mn}\cr}$$
erreicht man durch
\begin{verbatim}
$$A=\pmatrix{a_{11}&a_{12}&\ldots&a_{1n}\cr
             a_{21}&a_{22}&\ldots&a_{2n}\cr
             \vdots&\vdots&\ddots&\vdots\cr
             a_{m1}&a_{m2}&\ldots&a_{mn}\cr}$$
\end{verbatim}

Neben den normalen Matrizen kennt \TeX\ noch eine Sonderform, bei der
Zeilen und \index{Spalte!Matrix}Spalten der \index{Matrix}
Matrix beschriftet werden. Der entsprechende
Befehl hei\ss{}t \index{zzzbordermatrix@{\tt\char92 bordermatrix}}
`\verb|\bordermatrix|' und sieht in der Anwendung
folgenderma\ss{}en aus.
$$M=\bordermatrix{&C&I&C'\cr C&1&0&0\cr I&b&1-b&0\cr C'&0&a&1-a\cr}$$
wird realisiert durch
\begin{verbatim}
$$M=\bordermatrix{&C&I&C'\cr C&1&0&0\cr I&b&1-b&0\cr C'&0&a&1-a\cr}$$
\end{verbatim}
Jeweils die erste Zeile und die erste \index{Spalte!Matrix}
Spalte der \index{Matrix}
Matrix werden dann
nicht \index{Klammer!Bordermatrix}
geklammert. Beachten sie, da\ss{} der erste Spalteneintrag der
ersten Zeile leer ist, was durchaus erlaubt ist, und das die Klammern
wieder automatisch gesetzt wurden.

Wenn sie in einer Textformel Matrizen schreiben wollen, dann sollten
sie sie anders konstruieren, da die \index{Matrix}
Matrixbefehle zu gro\ss{}e Matrizen\index{zzzchoose@{\tt\char92 choose}}
erzeugen. $1\,0\choose0\,1$ erzeugt man z.B.\ mit
\begin{verbatim}
$1\,0\choose0\,1$|
\end{verbatim}
und eine \index{Matrix}Matrix wie
\index{zzzatop@{\tt\char92 atop}}\index{zzzbigl@{\tt\char92 bigl}}
\index{zzzbigr@{\tt\char92 bigr}}
$\bigl({a\atop l}{b\atop m}{c\atop n}\bigr)$ erzeugt man durch die
Befehle
\begin{verbatim}
$\bigl({a\atop l}{b\atop m}{c\atop n}\bigr)$
\end{verbatim}
\subsection{Vertikaler Leerraum}
Wir haben schon einige Befehle kennengelernt, die es erlauben die
\index{Hoehe!Formel@H\"ohe!Formel}H\"ohe
einer Formel so zu ver\"andern, da\ss{} sie anders als normal gesetzt wird.
Dazu geh\"oren die Befehle
\index{zzzstrut@{\tt\char92 strut}}
`\verb|\strut|' und \index{zzzmathstrut@{\tt\char92 mathstrut}}
`\verb|\mathstrut|'. Es
gibt aber noch andere M\"oglichkeiten. Als erstes den Befehl
\index{zzzphantom@{\tt\char92 phantom}}
\begin{verbatim}
\phantom{<Teilformel>}
\end{verbatim}
\TeX\ belegt bei Benutzung dieses Befehls zwar den Platz, den die
Teilformel einnehmen w\"urde, gibt die Formel selber aber nicht aus. Die
Angabe von `\verb|\phantom{0}2|' w\"urde als Ergebnis nur die Zwei
liefern, aber den Platz von der Null und der Zwei beanspruchen. Wenn
sie also z.B.\ ein Symbol in der
\index{Groesse!Summenzeichen@Gr\"o\ss{}e!Summenzeichen}
Gr\"o\ss{}e eines \index{Summe!Zeichen}Summenzeichens selber
sp\"ater per Hand einf\"ugen wollen, dann schreiben sie
\index{zzzmathop@{\tt\char92 mathop}}
\index{zzzphantom@{\tt\char92 phantom}}
\index{zzzsum@{\tt\char92 sum}}
\begin{verbatim}
\mathop{\phantom\sum}
\end{verbatim}
Der `\verb|\mathop|' Befehl sorgt daf\"ur, das der Platz wie f\"ur ein
gro\ss{}es Symbol gesetzt wird.

Weiterhin gibt es die beiden \index{Subbefehl}
Subbefehle \index{zzzhphantom@{\tt\char92 hphantom}}`\verb|\hphantom|' und
\index{zzzvphantom@{\tt\char92 vphantom}}
`\verb|\vphantom|'. Bei der ersten Variante wird die Breite der
nachfolgenden Teilformel freigehalten, \index{Hoehe!Formel@H\"ohe!Formel}
H\"ohe und Tiefe sind aber Null,
bei der zweiten Alternative, wird die H\"ohe und die Tiefe der
Teilformel freigehalten, die Breite aber ist Null. Der
\index{zzzmathstrut@{\tt\char92 mathstrut}}
`\verb|\mathstrut|' Befehl wird einfach mit
\begin{verbatim}
\vphantom(
\end{verbatim}
erzeugt.

{\em Plain} \TeX\ kennt noch den Befehl
\index{zzzsmash@{\tt\char92 smash}}`\verb|\smash{<Teilformel>}|', der
zwar die Teilformel ganz normal ausgibt, aber die H\"ohe und die Tiefe
der \index{Ausgabe}
Ausgabe auf Null setzt. Somit ist es m\"oglich jeder Ausgabe jede
beliebige H\"ohe und Tiefe zu geben. Mit
\index{zzzliminf@{\tt\char92 liminf}}\index{zzzlimsup@{\tt\char92 limsup}}
\index{zzzmathop@{\tt\char92 mathop}}
\index{zzzsmash@{\tt\char92 smash}}
\index{zzzvphantom@{\tt\char92 vphantom}}
\begin{verbatim}
\mathop{\smash\limsup\vphantom\liminf}
\end{verbatim}
wird das gro\ss{}e Symbol `$\limsup$' ausgegeben, allerdings mit den
H\"ohenabmessungen von `$\liminf$', also mit der Tiefe Null.

Mit den \index{Box!Befehle}
Boxbefehlen \index{zzzraise@{\tt\char92 raise}}
`\verb|\raise|' und \index{zzzlower@{\tt\char92 lower}}
`\verb|\lower|' k\"onnen
Teilformeln auch noch h\"oher oder tiefergestellt werden. Mit der
Eingabe
\index{zzzscriptstyle@{\tt\char92 scriptstyle}}
\begin{verbatim}
$2^{\raise1pt\hbox{$\scriptstyle n$}}$
\end{verbatim}
erreicht man, da\ss{} der \index{Exponent}Exponent um ein pt h\"oher gesetzt wird, als
normal, also $2^{\raise1pt\hbox{$\scriptstyle n$}}$ statt $2^n$. Der
Befehl `\verb|\scriptstyle|' ist n\"otig, da normlerweise in einer hbox
die \index{Schrift!Art}
Schriftart gew\"ahlt wird, die die mathematische Umgebung umgibt.

\subsection{Spezielle Befehle f\"ur besondere Aufgaben}
\subsubsection{\tt\char92 nonscript}
Folgt dem Befehl \index{zzznonscript@{\tt\char92 nonscript}}
`\verb|\nonscript|' unmittelbar etwas \index{Leim}
Leim, oder ein
Kern, dann wird dieser nicht entsprechend dem Script- oder
\index{Stil!Scriptscript}
Scriptscriptstil ausgegeben, sondern in der f\"ur den Text- oder
\index{Display!Stil}
Displaystil vorgegebenen Gr\"o\ss{}e. Der Befehl `\verb|\nonscript\;|'
erzeugt also genau den \index{Zwischenraum}
Zwischenraum, der in obiger Tabelle mit $(3)$
angegeben ist.
\subsubsection{\tt\char92 every...}
Mit den beiden Befehlen \index{zzzeverymath@{\tt\char92 everymath}}
`\verb|\everymath|' und
\index{zzzeverydisplay@{\tt\char92 everydisplay}}`\verb|\everydisplay|'
kann jeder mathematishcen Umgebung, Textformel, bzw.\
\index{Formel!abgesetzte}abgesetzte
Formel eine Befehlsgruppe \"ubergeben werden. Die Befehle entsprechen
somit dem Befehl \index{zzzeverypar@{\tt\char92 everypar}}
`\verb|\everypar|' f\"ur Abs\"atze.
\section{Abgesetzte Formeln}
\subsection{Einzeilige Formeln}
Wie abgesetzt Formeln aufgerufen werden, d\"urfte bis hier schon klar
sein, auch, wie man normalen Text in eine derartige Formel schreibt.
Wir haben auch schon gesehen, wie man mehr als eine Formel in eine
derartige Zeile schreibt. Hier jetzt noch einmal ein zusammenfassendes
Beispiel. Die Eingabe von
\index{zzzqquad@{\tt\char92 qquad}}
\begin{verbatim}
$$X_n=X_k\qquad\hbox{wenn, und nur wenn}\qquad
  Y_n=Y_k\quad\hbox{und}\quad Z_n=Z_k$$
\end{verbatim}
erzeugt
$$X_n=X_k\qquad\hbox{wenn, und nur wenn}\qquad
  Y_n=Y_k\quad\hbox{und}\quad Z_n=Z_k$$
Achten sie auf die unterschiedlichen \index{Zwischenraum}
Zwischenr\"aume, bei dem {\em wenn,
und nur wenn} und dem {\em und}. Wenn sie \index{Box!Hbox}
hboxen in mathematischen
Formeln verwenden, k\"onnen sie sich auch noch f\"ur eine andere Art des
\index{Leer!Raum}
Leerraums entscheiden. Beenden sie ihren Text in der hbox einfach mit
einem \index{Leer!Zeichen}
Leerzeichen, das erscheint dann in der Formel. Mit der Eingabe
von
\begin{verbatim}
$$... \hbox{f\"ur alle }n\ge0$$
\end{verbatim}
bekommen sie
$$\ldots\hbox{f\"ur alle }n\ge0$$

Sie haben sich sicher schon gefragt, wie man in \TeX\ denn derartige
abgesetzte Formeln nummerieren kann. Die L\"osung des Problems liegt in
dem Befehl \index{zzzeqno@{\tt\char92 eqno}}`\verb|\eqno|'. Er wird
\begin{verbatim}
$$<Formel 1>\eqno<Formel 2>$$
\end{verbatim}
angewendet, und alles hinter dem Befehl wird als Formelnummer
verwendet. Mit der Eingabe von
\begin{verbatim}
$$x^2-y^2=(x+y)(x-y)\eqno(15)$$
\end{verbatim}
erhalten sie
$$x^2-y^2=(x+y)(x-y)\eqno(15)$$
\"Ahnlich arbeitet der Befehl \index{zzzleqno@{\tt\char92 leqno}}
`\verb|\leqno|', nur da\ss{} dann die
Formelnummer links von der Formel erscheint, sie mu\ss{} trotzdem als {\em
zweites} \index{Argument!Bei Formelnummern}Argument angegeben werden.
\begin{verbatim}
$$x^2-y^2=(x+y)(x-y)\leqno(16)$$
\end{verbatim}
erscheint also als
$$x^2-y^2=(x+y)(x-y)\leqno(16)$$
An den Beispielen k\"onnen sie auch sehen, da\ss{} die Formel unabh\"angig von
den Formelnummern \index{Zentrierung}
zentriert wird. Nur wenn die Formel so lang w\"are,
da\ss{} sie bei der \index{Zentrierung}
Zentrierung mit der Nummer in Ber\"uhrung k\"ame, w\"urde
sie etwas zur Seite gesetzt. Wenn sie dann immer noch nicht pa\ss{}t, dann
k\"onnen sie noch einen der folgenden Tricks verwenden. Setzen sie die
Formelnummer in einen \index{zzzrlap@{\tt\char92 rlap}}
`\verb|\rlap|' oder einen \index{zzzllap@{\tt\char92 llap}}
`\verb|\llap|' Befehl,
sie erscheint dann in einer eigenen Zeile.\footnote{Welchen der beiden
Befehle sie verwenden m\"ussen ergibt sich daraus, nach welcher Seite
die Formelnummer \"uberlappen mu\ss{}.}
\index{zzzeqno@{\tt\char92 eqno}}
\begin{verbatim}
$$x^2-y^2=(x+y)(x-y)\eqno\llap{(17)}$$
\end{verbatim}
Erzeugt
$$x^2-y^2=(x+y)(x-y)\eqno\llap{(17)}$$
\subsection{Mehrzeilige Formeln}
F\"ur mehrzeilige Formeln stellt \TeX\ einige n\"utzliche Befehle bereit.
Der erste Befehl dieser Art ist der Befehl
\index{zzzeqalign@{\tt\char92 eqalign}}`\verb|\eqalign|', der es
mit einer \"ahnlichen Konstruktion wie schon bei
\index{zzzmatrix@{\tt\char92 matrix}}
`\verb|\matrix|' und
\index{zzzcases@{\tt\char92 cases}}
`\verb|\cases|' erlaubt, mehrere Formeln zu setzen. Der
grunds\"atzliche Aufbau ist
\index{zzzeqalign@{\tt\char92 eqalign}}
\begin{verbatim}
\eqalign{<linke Seite 1> & <rechte Seite 1> \cr
         <linke Seite 2> & <rechte Seite 2> \cr
            .
            .
         <linke Seite n> & <rechte Seite n>\cr}
\end{verbatim}
Dabei wird die linke Seite immer \index{Rechtsbuendig@Rechtsb\"undig}
rechtsb\"undig, die rechte immer
\index{Linksbuendig@Linksb\"undig}
linksb\"undig gesetzt, so da\ss{} beide Teile eine gemeinsame Formel
ergeben. Da h\"aufig \index{Gleichung}Gleichungen gesetzt werden m\"ussen, bietet sich
dieses Befehlsformat an. Man kann z.B.\ das \index{Gleichheitszeichen}
Gleichheitszeichen immer
als erstes Zeichen der rechten Seite w\"ahlen, und erreicht damit, da\ss{}
die Gleichheitszeichen genau \"ubereinander stehen.
\index{zzzcdots@{\tt\char92 cdots}}\index{zzzeqalign@{\tt\char92 eqalign}}
\begin{verbatim}
$$\eqalign{X_1+\cdots+X_n&=m,\cr
           Y_1+\cdots+Y_m&=n.\cr}$$
\end{verbatim}
Ergibt somit
\begin{eqnarray*}
X_1+\cdots+X_n & = & m,\\
Y_1+\cdots+Y_m & = & n.
\end{eqnarray*}
Der Befehl erzeugt eine \index{Box!Gleichungen}Box, die vertikal
\index{Zentrierung}zentriert wird, es k\"onnen
also auch mehrere \index{Gleichung}
Gleichungsketten nebeneinander dargestellt werden.
\index{zzzalpha@{\tt\char92 alpha}}\index{zzzbeta@{\tt\char92 beta}}
$$
\left\{\begin{array}{r@{=}l}
\displaystyle{\alpha} & \displaystyle{f(z)}\\
\displaystyle{\beta} & \displaystyle{f(z^2)}\\
\displaystyle{\gamma} & \displaystyle{f(z^3)}\end{array}
\right\}\qquad\left\{
\begin{array}{r@{=}l}
\displaystyle{x} & \displaystyle{\alpha^2-\beta}\\
\displaystyle{y} & \displaystyle{2\gamma}\end{array}\right\}.$$
wurde mit
\index{zzzeqalign@{\tt\char92 eqalign}}
\begin{verbatim}
$$\left\{\eqalign{
\alpha&=f(z)\cr \beta&=f(z^2)\cr \gamma&=f(z^3)\cr}
\right\}\qquad\left\{
\eqalign{x&=\alpha^2-\beta\cr y&=2\gamma\cr}\right\}.$$
\end{verbatim}
erzeugt.

Bei derartigen Folgen von \index{Gleichung}Gleichungen, aber auch bei anderen
Konstruktionen, bei denen mehrere Formeln untereinander stehen, ist es
mitunter w\"unschenswert die Zeilen zu gruppieren. Um etwas zus\"atzlichen
Platz zwischen zwei Formeln zu erhalten, kann hinter einem `\verb|\cr|'
der Befehl
\index{zzznoalign@{\tt\char92 noalign}}
\index{zzzvskip@{\tt\char92 vskip}}
\begin{verbatim}
\noalign{\vskip3pt}
\end{verbatim}
geschrieben werden, um 3pt mehr Platz zu erhalten. Statt 3pt kann
nat\"urlich jede \index{Leim}
Leimangabe stehen, die gerade gebraucht
wird.\footnote{Es kann mit diesem Befehl alles m\"ogliche zwischen zwei
Zeilen geschrieben werden, wie gleich in einem weiteren Beispiel zu
sehen ist.}

Um Zeilen in einer mehrzeiligen Umgebung nummerieren zu k\"onnen gibt es
die zwei verwandten Befehle \index{zzzeqalignno@{\tt\char92 eqalignno}}
`\verb|\eqalignno|' und \index{zzzleqalignno@{\tt\char92 leqalignno}}
`\verb|\leqalignno|', wobei der
zweite wieder nur die Formelnummer auf der linken Seite ausgibt. Wenn
einzelne Zeilen nicht nummeriert werden sollen, dann braucht auch das
`\verb|&|' nicht geschrieben zu werden. Die Eingabe von
\index{zzzeqalignno@{\tt\char92 eqalignno}}
\begin{verbatim}
$$\eqalignno{(x+y)(x-y)&=x^2-xy+yx-y^2\cr
    &=x^2-y^2;&(7.1)\cr
  (x+y)^2&=x^2+2xy+y^2.&(7.2)\cr}$$
\end{verbatim}
ergibt
\begin{eqnarray}
(x+y)(x-y)&=&x^2-xy+yx-y^2\nonumber\\
    &=&x^2-y^2;\\
  (x+y)^2&=&x^2+2xy+y^2.\end{eqnarray}
was auch zeigt, da\ss{} durchaus auch einzelne Teile der Formel ganz
\index{Leer!Formel}leer
bleiben k\"onnen.

Es gibt noch einen wichtigen Unterschied zwischen den Befehlen, die
eine Formelzeile ausgeben, und denen, die das nicht tun. Mit dem
`\verb|\eqalignno|' Befehl werden Zeilen ausgegeben, die \"uber eine
ganze Zeile gehen, mit dem
\index{zzzeqalign@{\tt\char92 eqalign}}`\verb|\eqalign|' Befehl werden Boxen
erzeugt, die nur ihre nat\"urliche Breite haben, und die vertikal
\index{Zentrierung}
zentriert werden. Daraus ergeben sich einige Konsequenzen:
\begin{enumerate}
\item Der {\tt\char92 eqalignno} Befehl kann nur {\em einmal} in einer
Formel verwendet werden, der {\tt\char92 eqalign} Befehl, wie oben
gezeigt, mehrfach.
\item Die Zeilen, die aus einem {\tt\char92 eqalign} Befehl erzeugt
wurden k\"onnen nicht zwischen zwei \index{Seite!Umbruch}
Seiten umgebrochen werden, da dieser
Befehl ja eine Box erzeugt. Der {\tt\char92 eqalignno} Befehl erzeugt
dagegen einzelne Zeilen, die sehr wohl auf zwei Seiten aufgeteilt
werden k\"onnen.
\item Weiterhin k\"onnen die Zeilen, die mit dem
\index{zzzeqalignno@{\tt\char92 eqalignno}}`\verb|\eqalignno|'
Befehl angeordnet werden, durch Text unterbrochen werden, {\em ohne}
da\ss{} die Ausrichtung dabei verloren geht. Eine Anordnung, wie
\begin{eqnarray}
x&=&y+z\\ \noalign{\hbox{und}}
y^2+z^2&=&x^2\end{eqnarray}
wird mit einem \index{zzznoalign@{\tt\char92 noalign}}
`\verb|\noalign{\hbox{und}}|'
hinter dem ersten `\verb|\cr|' erzeugt.
\end{enumerate}

Eine weitere M\"oglichkeit mehrere Zeilen untereinander zu schreiben,
besteht in dem Befehl \index{zzzdisplaylines@{\tt\char92 displaylines}}
`\verb|\displaylines|'. Das generelle Format
dieses Befehls lautet
\begin{verbatim}
$$\displaylines{<erste Zeile>\cr
               <zweite Zeile>\cr
                      .
                      .
               <letzte Zeile>\cr}$$
\end{verbatim}
Die einzelnen Zeilen werden durch ein \index{zzzhfil@{\tt\char92 hfil}}
`\verb|\hfil|'
\index{Zentrierung}zentriert, was
also durch ein \index{zzzhfill@{\tt\char92 hfill}}
`\verb|\hfill|' \"uberschrieben werden kann. Mit
\index{zzzdisplaylines@{\tt\char92 displaylines}}\index{zzzllap@{\tt\char92 llap}}
\begin{verbatim}
$$\displaylines{\hfill x\equiv x;\hfill\llap{(1)}\cr
   \hfill\hbox{wenn}\quad x\equiv y\quad\hbox{dann}\quad
      y\equiv x;\hfill\llap{(2)}\cr
   \hfill\hbox{wenn}\quad x\equiv y\quad\hbox{und}\quad
      y\equiv z\quad\hbox{dann}\quad
      x\equiv z.\hfill\llap{(3)}\cr}$$
\end{verbatim}
erhalten sie
$$\displaylines{\hfill x\equiv x;\hfill\llap{(1)}\cr
   \hfill\hbox{wenn}\quad x\equiv y\quad\hbox{dann}\quad
      y\equiv x;\hfill\llap{(2)}\cr
   \hfill\hbox{wenn}\quad x\equiv y\quad\hbox{und}\quad
      y\equiv z\quad\hbox{dann}\quad
      x\equiv z.\hfill\llap{(3)}\cr}$$

Sie haben sicher schon bemerkt, da\ss{} der Abstand zwischen zwei und mehr
Zeilen \index{Groesse!Zeilenabstand@Gr\"o\ss{}e!Zeilenabstand}
gr\"o\ss{}er ist, als der Abstand der Zeilen im Text. Dies geschieht
zum Zwecke der besseren Lesbarkeit. Der Befehl, der diese `\"Offnung'
der Zeilen bewirkt, wird automatisch nach jedem der Mehrzeilenbefehle
eingef\"ugt, er lautet \index{zzzjot@{\tt\char92 jot}}
\index{zzzopenup@{\tt\char92 openup}}
`\verb|\openup1\jot|'. Wenn sie nicht mit \TeX's
Befehlen arbeiten wollen, sondern die Anordnung mit dem
\index{zzzhalign@{\tt\char92 halign}}`\verb|\halign|'
Befehl selber bewerkstelligen wollen, dann sollten sie den
\index{Oeffnungsbefehl@\"Offnungsbefehl}
\"Offnungsbefehl selber einf\"ugen. Aber beachten sie dabei folgendes. Die
`\"Offnung' der Zeilen wirkt nur innerhalb der
\index{Gruppe!Oeffnung@Gruppe!\"Offnung}Gruppe, die durch die
`\verb|$$|' erzeugt wird, dadurch gibt es einen kleinen
\index{Nebeneffekt}Nebeneffekt.
Die letzte Formelzeile steht ja wieder vor einer Textzeile, und da ist
der \index{Abstand!Hinter Formeln}
Abstand wieder der normale, sie m\"ussen also am Ende der
Formelzeilen {\em per Hand} noch etwas zus\"atzlichen Abstand einf\"ugen.
\subsection{Lange Formeln}
Alle Regeln, die sie bisher \"uber Formelsatz gelernt haben, n\"utzen
ihnen nichts, wenn sie eine Formel haben, die einfach l\"anger ist, als
eine Zeile. \index{Knuth. D.}
D.~Knuth empfiehlt in solch einem Falle die Verwendung des
`\verb|eqalign|' Befehls. Man kann dabei eine lange Formel z.B. so
setzen
\begin{eqnarray*}
\sigma(2^{34}-1,2^{35},1)&=&-3+(2^{34}-1)/2^{35}+2^{35}\!/(2^{34}-1)\\
 & &{}+7/2^{35}(2^{34}-1)-\sigma(2^{35},2^{34}-1,1).\end{eqnarray*}
Diese Formel wurde mit
\index{zzzeqalign@{\tt\char92 eqalign}}
\index{zzzqquad@{\tt\char92 qquad}}
\index{zzzsigma@{\tt\char92 sigma}}
\begin{verbatim}
$$\eqalign{\sigma(2^{34}-1,2^{35},1)
  &=-3+(2^{34}-1)/2^{35}+2^{35}\!/(2^{34}-1)\cr
  &\qquad+7/2^{35}(2^{34}-1)-\sigma(2^{35},2^{34}-1,1).\cr}$$
\end{verbatim}
erzeugt. \TeX\ kennt selber {\em keine} Regeln, f\"ur den Umbruch einer
derartigen Formel, dazu mu\ss{} man den Inhalt der Formel nicht nur
kennen, sondern sie auch verstehen, um den besten Platz f\"ur den
Umbruch zu erkennen. Statt dessen stehen hier einige kleine Regeln,
die ihnen die \index{Umbruch von Formeln}
Entscheidung etwas erleichtern sollen.
\begin{enumerate}
\item Brechen sie eine lange Formel {\em vor} einem bin\"aren Symbol,
oder einem \index{Relation}
Relationssymbol um, also anders, als in Textformeln.
\item R\"ucken sie den zweiten Zeil der Formel etwas ein. Sie k\"onnen
auch den ersten Teil der Formel mit einem {\tt\char92 hfill} ganz nach
links, den zweiten mit einem weiteren {\tt\char92 hfill} ganz nach
rechts schieben.
\end{enumerate}

Achten sie zum Abschlu\ss{} darauf, da\ss{} sie keine
\index{zzzleft@{\tt\char92 left}}
`\verb|\left|',
`\verb|\right|' Konstruktionen \"uber mehrere Zeilen verwenden. Bei
einer langen Formel m\"ussen sie die Gr\"o\ss{}enwahl der
\index{Klammer!Groessenanpassung@Klammer!Gr\"o\ss{}enanpassung}Klammer schon selber
vornehmen.
\chapter{Definitionen oder Makros}
\section{Generelles}
Wenn sie einen Text schreiben, der oft dieselbe Folge von Zeichen
enth\"alt, dann k\"onnen sie sich mit \TeX\ die Arbeit erleichtern.
Angenommen, sie schreiben einen mathematischen Text, der oft den
Vektor `$(x_1,\ldots,x_n)$' enth\"alt. Dann k\"onnen sie mit der Anweisung
\index{zzzdef@{\tt\char92 def}}
\index{zzzxvek@{\tt\char92 xvek}}
\begin{verbatim}
\def\xvek{(x_1,\ldots,x_n)}
\end{verbatim}
einen \index{Neu!Befehl}
neuen Befehl `\verb|\xvek|' bereitstellen, der ab sofort als
\index{Abkuerzung@Abk\"urzung}
Abk\"urzung f\"ur `\verb|(x_1,\ldots,x_n)|' steht. Eine komplizierte
\index{Formel!Abkuerzung@Formel!Abk\"urzung}Formel, wie
\index{zzzbigl@{\tt\char92 bigl}}\index{zzzbigr@{\tt\char92 bigr}}
$$\sum_{(x_1,\ldots,x_n)\ne(0,\ldots,0)}
\bigl(f(x_1,\ldots,x_n)+g(x_1,\ldots,x_n)\bigr)$$
kann dann mit der einfachen Anweisung
\index{zzzsum@{\tt\char92 sum}}
\index{zzzxvek@{\tt\char92 xvek}}
\begin{verbatim}
$$\sum_{\xvek\ne(0,\ldots,0)} \bigl(f\xvek+g\xvek\bigr)$$
\end{verbatim}
geschrieben werden. Derartige \index{Abkuerzung@Abk\"urzung}
Abk\"urzungen bieten sich nat\"urlich nur
an, wenn eine Zeichenfolge wirklich \"ofters vorkommt. Sie bietet dann
aber auch eine gewisse Gew\"ahr gegen \index{Fehler!Schreibfehler}
Tippfehler, die auch nicht zu vernachl\"a\ss{}igen ist.

\TeX\ expandiert ihre neu geschaffene \index{Kontrollsequenz}
Kontrollsequenz, bis nichts mehr
zu expandieren ist, sprich nur noch \index{Primitiv}
Primitive \"ubrigbleiben. Der Befehl
\index{zzzxvek@{\tt\char92 xvek}}
\begin{verbatim}
\xvek
\end{verbatim}
wird zun\"achst zu
\index{zzzldots@{\tt\char92 ldots}}
\begin{verbatim}
(x_1,\ldots,x_n)
\end{verbatim}
expandiert, und dieser Ausdruck dann zu
\index{zzzldotp@{\tt\char92 ldotp}}
\index{zzzmathinner@{\tt\char92 mathinner}}
\begin{verbatim}
(x_1,\mathinner{ldotp\ldotp\ldotp},x_n)
\end{verbatim}
da auch \index{zzzldots@{\tt\char92 ldots}}
`\verb|\ldots|' ein Makro\footnote{Statt des Begriffs
`Definition' hat sich auch der Begriff `Makro' durchgesetzt, da quasi
eine mikroskopisch kleine Anweisung einen makroskopischen Effekt haben
kann.} ist. Die jetzt noch verbliebenen \index{Kontrollsequenz}
Kontrollsequenzen sind keine
Makros mehr, werden also auch nicht weiter expandiert.

Es empfiehlt sich eine eigene \index{Bibliothek}
Bibliothek solcher Makrodefinitionen
anzulegen, in denen alle Definitionen gespeichert werden, die sie
\"ofters brauchen. Achten sie aber darauf, da\ss{} die Bibliothek nicht {\em
zu} gro\ss{} wird, da sie f\"ur jeden Text mit
\index{zzzinput@{\tt\char92 input}}
\begin{verbatim}
\input makro
\end{verbatim} eingelesen werden mu\ss{},\footnote{Ich gehe hier davon aus,
da\ss{} sie alle Makros in einer Datei {\tt makro.tex} gespeichert haben,
w\"ahlen sie einen anderen Dateinamen, dann lautet auch der Name hinter
dem {\tt\char92 input} Befehl anders.} was jedesmal etwas Zeit dauert.

Definitionen dieser Art sind nat\"urlich auch au\ss{}erhalb des
mathematischen Modus erlaubt, und sie haben auch dort dieselbe Syntax.
Eine Definition hat {\em immer} den folgenden Aufbau
\index{Ersatztext}\index{zzzdef@{\tt\char92 def}}
\begin{verbatim}
\def\<name>{<Ersatztext>}
\end{verbatim}
sie wird also immer mit der \index{Kontrollsequenz}
Kontrollsequenz `\verb|\def|' eingeleitet,
der dann der Name der neu zu definierenden Kontrollsequenz, inklusive
\index{Gegenschraegstrich@Gegenschr\"agstrich}
Gegenschr\"agstrich, folgt. Zum Schlu\ss{} dann in geschweiften
\index{Klammer!In Definition}Klammern der
Text, der aus der Kontrollsequenz expandiert werden soll. {\em
Achtung}, die geschweiften Klammern haben hier ausnahmsweise einmal
{\em nicht} die Wirkung einer Gruppierungsklammer. Wenn sie erreichen
wollen, da\ss{} eine Kontrollsequenz zu `\verb|{\bf x}|' expandiert werden
soll, dann mu\ss{} die Definition
\index{zzzdef@{\tt\char92 def}}
\begin{verbatim}
\def\bfx{{\bf x}}
\end{verbatim}
lauten. Anderenfalls w\"urde der gesamte nachfolgende Text
\index{Schrift!Fett}\index{Fettschrift}fett gesetzt.
\section{Parameterisierte Definitionen}
Angenommen, sie wollten au\ss{}er `$(x_1,\ldots,x_n)$' auch \"ofters
\"ahnliche Konstruktionen, nur mit anderen Buchstaben, also z.B.\
`$(y_1,\ldots,y_n)$', schreiben. Sie m\"ussen daf\"ur nicht jeweils eine
Extradefinition schreiben, \TeX\ bietet die M\"oglichkeit Makros auch
mit Parametern zu definieren und aufzurufen. Die Definition
\index{zzzdef@{\tt\char92 def}}
\index{zzzvektor@{\tt\char92 vektor}}
\begin{verbatim}
\def\vektor#1{(#1_1,\ldots,#1_n)}
\end{verbatim}
erwartet bei Aufruf ein \index{Argument!Einer Definition}
Argument, da\ss{} dan an die entsprechende Stelle
gesetzt wird. Der Aufruf `\verb|\vektor x|' erzeugt
`$(x_1,\ldots,x_n)$', der Aufruf `\verb|\vektor y|' erzeugt
`$(y_1,\ldots,y_n)$'. Auch mehrere Parameter sind m\"oglich. Wenn z.B.\
der h\"ochste Wert der Vektoren nicht immer $n$ sein soll, dann
empfiehlt sich die Definition
\index{zzzdef@{\tt\char92 def}}
\index{zzzvektorn@{\tt\char92 vektorn}}
\begin{verbatim}
\def\vektorn#1{(#1_1,\ldots,#1_#2))}
\end{verbatim}
Der Aufruf `\verb|\vektorn am|' erzeugt dann `$(a_1,\ldots,a_m)$'. Sie
k\"onnen bis zu neun Argumente an ein Makro \"ubergeben (\verb|#1| bis
\verb|#9|). Die Reihenfolge des Auftretens im \index{Ersatztext}
Ersatztext ist dabei
beliebig, nicht aber hinter der Angabe des Namens. Sie k\"onnen z.B.\
kein `\verb|#5|' benutzen, wenn sie nicht vorher alle Parameterzeichen
bis `\verb|#4|' verwendet haben.\footnote{Ab hier d\"urfte auch klar
sein, wieso das Zeichen {\tt\char35} normalerweise verboten ist.}

Wenn sie eine \index{Kontrollsequenz}
Kontrollsequenz mehrfach definieren, dann ist immer nur
die letzte Definition g\"ultig. Die Anweisungen
\index{zzzdef@{\tt\char92 def}}
\begin{verbatim}
\def\a{Hallo} \def\a{wie geht's}
\end{verbatim}
bewirkt, da\ss{} bei Aufruf von `\verb|\a|' `wie geht's' ausgegeben wird.
Allerdings lassen sich Definitionen lokal, in einer Gruppe, \"andern,
ohne, da\ss{} die urspr\"ungliche Bedeutung verloren geht.

Sie k\"onnen sich entscheiden, ob sie die Definition mit
\index{Leer!Zeichen}Leerzeichen
\"ubersichtlicher machen wollen, oder nicht. Die Definitionen
\index{zzzdef@{\tt\char92 def}}
\begin{verbatim}
\def\a #1 #2{#1 oder #2}
\def\a#1#2{#1 oder #2}
\end{verbatim}
sind gleichwertig. Auch im Aufruf ist es gleichg\"ultig, ob sie
`\verb|\a A B|' oder `\verb|\aAB|' verwenden, Das Ergebnis ist immer
`A oder B'.

Normalerweise folgt einem Parameterzeichen nur eine \index{Ziffer}
Ziffer zwischen 1
und 9, man kann aber durch weitere Zeichen erreichen, da\ss{} auch gr\"o\ss{}ere
\index{Argument!Einer Definition}
Argumente leicht \"ubergeben werden k\"onnen.\footnote{Von dieser
M\"oglichkeit sollte man nur sehr sparsam Gebrauch machen, und nur, wenn
einem die Definition eines Makros {\em immer} klar ist.} Die
Definition
\index{zzzdef@{\tt\char92 def}}
\begin{verbatim}
\def\cs #1. #2\par{...}
\end{verbatim}
definiert einen neuen Befehl mit zwei \index{Argument!Mehrere
Definitionsargumente}Argumenten, wobei das erste
Argument durch das Vorkommen eines \index{Punkt}
Punktes, gefolgt von einem
\index{Leer!Zeichen}Leerzeichen, hier ist das 
Leerzeichen wichtig, das zweite Argument von
dem Befehl \index{zzzpar@{\tt\char92 par}}
`\verb|\par|' beendet wird. Diese Beender, gelten auch beim
Aufruf. Wenn man diesen Befehl mit
\begin{verbatim}
\cs Sie schulden \$5.00 DM. Zahlen Sie sie.\par
\end{verbatim}
aufruft, dann wird `{\em Sie schulden \$5.00 DM}' zum ersten Argument;
der erste \index{Punkt}Punkt wird nicht als \index{Begrenzer!Makroargument}
Begrenzer erkannt, da ihm keine
\index{Leer!Stelle}
Leerstelle folgt, das zweite Argument ist dann `{\em Zahlen Sie sie.}'
Weiterhin wird auch dann ein Argument weiter angenommen, wenn bei
Auftreten eines Argumentbegrenzers noch geschweifte Klammern ge\"offnet
sind. H\"atte man obigen Befehl mit
\index{zzzdef@{\tt\char92 def}}
\begin{verbatim}
\def\cs #1.#2\par{...}
\end{verbatim}
definiert, dann m\"u\ss{}te der Aufruf
\index{zzzpar@{\tt\char92 par}}
\begin{verbatim}
\cs Sie schulden {\$5.00} DM. Zahlen Sie sie.\par
\end{verbatim}
lauten, um das gew\"unschte Ergebnis zu erhalten. Man kann sich diese
Form der Parameterbegrenzung zunutze machen, um z.B.\ mathematische
Theoreme {\em sch\"on} zu setzen.

\proclaim Theorem 1. \TeX\ hat gro\ss{}e Makrof\"ahigkeiten.\par

Tats\"achlich kennt {\em Plain} \TeX\ einen Befehl
\index{zzzproclaim@{\tt\char92 proclaim}}
`\verb|\proclaim|', mit dem
die Beispielzeile gesetzt wurde. Er ist folgenderma\ss{}en definiert
\index{zzzdef@{\tt\char92 def}}\index{zzzenspace@{\tt\char92 enspace}}
\index{zzzmedbreak@{\tt\char92 medbreak}}\index{zzznoindent@{\tt\char92 noindent}}
\index{zzzpar@{\tt\char92 par}}
\begin{verbatim}
\def\proclaim #1. #2\par{\medbreak
  \noindent{\bf#1.\enspace}{\sl#2}\par\medbreak}
\end{verbatim}
Wenn sie diese Definition \"andern, was sie ohne weiteres durch eine
\index{Neu!Definition}
Neudefinition zu Beginn ihres Textes tun k\"onnen, \"andern sie damit das
Aussehen {\em aller} Theoreme in ihrem gesamten Text.
\subsection{Die Regeln der Parameterisierung}
Jede Makrodefinition hat die folgende Form\index{Ersatztext}
\begin{verbatim}
\def<Kontrollsequenz><Parameterliste>{<Ersatztext>}
\end{verbatim}
Wobei die Parameterliste keine geschweiften
\index{Klammer!Parameterliste}Klammern enthalten darf,
au\ss{}erdem das \index{Spezialzeichen}
Spezialzeichen `\verb|#|' von \index{Ziffer}Ziffern 1 bis (maximal) 9
gefolgt wird, die der Reihe nach auftreten m\"ussen; der Ersatztext
enth\"alt nur paarweise geschweifte Klammern und die entsprechenden
Parameterzeichen `\verb|#|' haben nur Ziffern als n\"achstes Zeichen, es
sei denn, ihnen folgt ein weiteres `\verb|#|', was das einfache
Zeichen `\verb|#|' darstellt. Zur weiteren Betrachtung dient ein
Beispiel einer Definition, die zu nichts anderem gut ist, als uns die
Regeln, nach denen \TeX\ vorgeht zu verdeutlichen.
\index{zzzdef@{\tt\char92 def}}
\begin{verbatim}
\def\cs AB#1#2C$#3\$ {#3{ab#1}#1 c##\x #2}
\end{verbatim}
Diese Definition hat eine Parameterliste, bestehend aus neun Token
\begin{verbatim}
A, B, #1, #2, C, $, #3, \$, Leerzeichen
\end{verbatim}
und einen \index{Ersatztext}Ersatztext, bestehend aus zw\"olf Token
\begin{verbatim}
#3, {, a, b, #1, }, #1, Leerzeichen, c, # \x, #2
\end{verbatim}
Beim Aufruf erwartet \TeX\ zun\"achst die beiden Zeichen `\verb|AB|'.
Falls diese beim Aufruf nicht mit angegeben werden, erhalten sie eine
\index{Fehler!Meldung}
Fehlermeldung, die besagt, da\ss{} der Aufruf des Makros nicht mit seiner
Definition \"ubereinstimmt. Dann folgen die Argumente 1 und 2, $\ldots$

Woran merkt \TeX\ nun, da\ss{} ein Parameter beendet ist? Es gibt zwei
F\"alle, einmal kann ein Parameter noch von einem \index{Begrenzer!Makro}
Parameterbegrenzer
gefolgt werden, oder es folgt ein n\"achster Parameter. Schlie\ss{}lich wird
das Ende der Parameterliste erreicht, wenn die ge\"offnete, geschweifte
\index{Klammer!Vor Ersatztext}Klammer den Beginn des
\index{Ersatztext}Ersatztextes anzeigt. Der Aufruf
\index{zzzLook@{\tt\char92 Look}}
\begin{verbatim}
\cs AB {\Look}C${And\$ }{look}\$ 5.
\end{verbatim}
bewirkt nun folgendes. Das erste \index{Argument!Einer Definition}
Argument ist `\verb|\Look|', von dem
zuvor die Klammern entfernt wurden. Das zweite Argument bleibt
\index{Leer!Argument}leer,
da das `\verb|$C|' sofort folgt. Das dritte Argument bekommt den {\em
Inhalt} `\verb|{And\$ }{look}|', da das erste Auftreten von `\verb|\$|'
in einer geschweiften Klammer steht. Auch die Klammern um das gesamte
Argument werden {\em nicht} entfernt, da dann die beiden inneren
Klammern alleine \"ubrigblieben, was auch wieder zu einer nicht
richtigen Verteilung der geschweiften Klammern f\"uhren w\"urde. Die
\index{Ausgabe}Ausgabe dieses Aufrufs w\"are
\index{zzzLook@{\tt\char92 Look}}
\begin{verbatim*}
{And\$ }{look}{ab\Look}\Look c#\x5.
\end{verbatim*}
Das \index{Leer!Zeichen}
Leerzeichen wird nicht entfernt, auch wenn es hinter dem
Kontrollwort `\verb|\Look|' steht, da \TeX\ Leerzeichen hinter
Kontrollworten sozusagen nur auf den ersten Rutsch entfernt, aber
nicht bei der \index{Expansion}Expansion von Makros.

Noch einen Spezialfall. Wenn sie als letztes Zeichen vor der \"offnenden
Klammer, die den \index{Ersatztext}
Ersatztext einleitet ein `\verb|#|' schreiben, dann
verh\"alt sich \TeX\ so, als st\"unde die \"offnende Klammer beim Aufruf vor
dem Argument. Wenn sie also
\index{zzzdef@{\tt\char92 def}}
\begin{verbatim}
\def\a#1#{\hbox to #1}
\end{verbatim}
definieren, dann ergibt der Aufruf `\verb|\a3pt{x}|' als Ergebnis
\begin{verbatim}
\hbox to 3pt{x}
\end{verbatim}
und nicht nur
\begin{verbatim}
\hbox to 3
\end{verbatim}
wie man zun\"achst erwarten w\"urde. Was \TeX\ genau macht, wenn es die
Parameter einer Makrodefinition auswertet kann man sich nat\"urlich auch
anzeigen lassen, es geht, wie nicht anders zu erwarten mit dem Befehl
`\verb|\tracingmacros=1|'.\index{zzztracingmacros@{\tt\char92 tracingmacros}}

Jeder Mensch macht \index{Fehler!Schreibfehler}
Fehler. Der h\"aufigste, in Zusammenhang mit
Makrodefinitionen und Makroaufrufen d\"urfte eine fehlende schlie\ss{}ende
\index{Klammer!Makro}Klammer bei der \index{Argument!Uebergabe@Argument!\"Ubergabe}
Argument\"ubergabe an eine \index{Kontrollsequenz}Kontrollsequenz sein. \TeX\
tritt dem entgegen, indem es bei dem ersten Auftreten des Befehls
\index{zzzpar@{\tt\char92 par}}
`\verb|\par|' das weitere Einlesen eines Arguments beendet, es sei
denn, sie haben \TeX\ ausdr\"ucklich angewiesen `\verb|\par|' zu
akzeptieren. Dies geschieht, indem sie den Befehl \index{zzzlong@{\tt\char92 long}}
`\verb|\long|' vor
das \index{zzzdef@{\tt\char92 def}}
`\verb|\def|' schreiben. Mit der Anweisung
\begin{verbatim}
\long\def\bftext{{\bf #1}}
\end{verbatim}
k\"onnen sie auch mehrere Paragraphen \index{Schrift!Fett}
\index{Fettschrift}fett ausgeben
lassen.\footnote{Allerdings verbrauchen sie mit dieser Methode {\em
sehr} viel \index{Speicher!Bedarf}
Speicher --- es gibt bessere L\"osungen.} Etwas schwieriger
sieht die Sache aus, wenn sie die schlie\ss{}ende Klammer des
\index{Ersatztext}Ersatztextes
vergessen haben. Hier w\"are eine Begrenzung durch
\index{zzzpar@{\tt\char92 par}}
`\verb|\par|' nicht
w\"unschenswert, da dieser Befehl in Ersatztexten oft gebraucht wird. Es
gibt aber dennoch eine M\"oglichkeit sich wenigstens etwas abzusichern.
Wenn sie der \index{zzzdef@{\tt\char92 def}}
`\verb|\def|' Anweisung ein \index{zzzouter@{\tt\char92 outer}}
`\verb|\outer|' voranstellen,
dann kann die neudefinierte \index{Kontrollsequenz}
Kontrollsequenz nur in der \"aussersten
vertikalen Liste verwendet werden. Sie kann also nicht in einer
Tabelle stehen, noch kann sie in einem \index{Ersatztext}
Ersatztext auftauchen o.\"a.
\TeX\ hat es in diesem Fall leichter einen \index{Fehler!Suche}
Fehler zu finden.

Schlie\ss{}lich kann einer \index{zzzdef@{\tt\char92 def}}
`\verb|\def|' Anweisung noch ein
\index{zzzglobal@{\tt\char92 global}}
`\verb|\global|' vorangestellt werden. Der Effekt ist wie man ihn
erwarten w\"urde. Da diese Situation oft auftritt, gibt es aber auch die
\index{Abkuerzung@Abk\"urzung}Abk\"urzung \index{zzzgdef@{\tt\char92 gdef}}
`\verb|\gdef|', die das gleiche bewirkt.
\section{Ein weiterer Zuweisungsbefehl}
Bisher haben wir schon einige Befehle kennengelernt, die einer
\index{Kontrollsequenz}Kontrollsequenz etwas zuweisen. Z.B.\
\index{zzzfont@{\tt\char92 font}}`\verb|\font|',
\index{zzzchardef@{\tt\char92 chardef}}\index{zzzcountdef@{\tt\char92 countdef}}
`\verb|\chardef|', `\verb|\countdef|' oder \index{zzzdef@{\tt\char92 def}}
`\verb|\def|'. Dar\"uber
hinaus gibt es noch den Befehl \index{zzzlet@{\tt\char92 let}}
`\verb|\let|', der einer
Kontrollsequenz eine Reihe von Token zuweist. Mit der Anweisung
\index{zzzdef@{\tt\char92 def}}\index{zzzlet@{\tt\char92 let}}
\begin{verbatim}
\let\a=\def
\end{verbatim}
k\"onnte man erreichen, da\ss{} \index{Neu!Definition}
Neudefinitionen nicht mehr mit `\verb|\def|'
eingeleitet werden brauchen, sondern statt dessen mit dem Befehl
`\verb|\a|'. Ist das \index{Argument}Argument der `\verb|\let|' Anweisung ein
einzelnes Zeichen, dann verh\"alt sich der neu geschaffene Befehl wie
dieses Zeichen, allerdings mit ein paar Ausnahmen. Wenn man z.B.\ mit
\index{zzzzero@{\tt\char92 zero}}
`\verb|\let\zero=0|' den Befehl `\verb|\zero|' schafft, dann kann
dieser dennoch nicht in numerischen Konstanten verwendet werden, da
dort nur \index{Ziffer}
Ziffern erlaubt sind, und `\verb|\zero|' {\em kein} Makro
ist, das expandiert wird. Letztlich gibt es auch noch den Befehl
\index{zzzfuturelet@{\tt\char92 futurelet}}
\begin{verbatim}
\futurelet\cs=<Token1><Token2>
\end{verbatim}
was sich genauso verh\"alt, wie die Anweisung
\index{zzzlet@{\tt\char92 let}}
\begin{verbatim}
\let\cs=<Token2><Token1><Token2>
\end{verbatim}
\section{Die Entscheidungsbefehle}
\TeX\ bietet eine Reihe von Befehlen, mit denen das Verhalten von
\TeX\ gesteuert werden kann, d.h.\ da\ss{} je nach Ausgangssituation
unterschiedliche Ergebnisse herauskommen. Diese M\"oglichkeit findet
nat\"urlich auch besonders in Makros eine Anwendung. Der generelle
Aufbau dieser \index{Bedingung}Bedingungspr\"ufungen ist
\index{zzzelse@{\tt\char92 else}}\index{zzzfi@{\tt\char92 fi}}
\begin{verbatim}
\if<Bedingung><Wahrtext>\else<Falschtext>\fi
\end{verbatim}
Wenn bei Erreichen dieser Anweisung die Bedingung {\em wahr} ist, dann
wird der {\tt Wahrtext} ausgef\"uhrt, andernfalls der {\tt Falschtext}.
Der Befehl\index{Seite!Rechts-Links}
\index{zzzcount@{\tt\char92 count}}\index{zzzelse@{\tt\char92 else}}
\index{zzzfi@{\tt\char92 fi}}\index{zzzifodd@{\tt\char92 ifodd}}
\index{zzzlinkeseite@{\tt\char92 linkeseite}}
\index{zzzrechteseite@{\tt\char92 rechteseite}}
\begin{verbatim}
\ifodd\count0 \rechteseite \else \linkeseite \fi
\end{verbatim}
sorgt daf\"ur, da\ss{} je nach \index{Seite!Zahl}
Seitenzahl, die bei \TeX\ \"ublicherweise im
Z\"ahlregister $0$ steht, entweder die \index{Kontrollsequenz}
Kontrollsequenz f\"ur rechte, oder
linke Seiten ausgef\"uhrt wird. Achten sie bei numerischen Vergleichen
in den Abfragen darauf, da\ss{} die numerischen Konstanten immer richtig
abgeschlossen werden. Bei der Anweisung
\index{zzzifnum@{\tt\char92 ifnum}}
\begin{verbatim}
\ifnum\zaehler=0...
\end{verbatim}
sollten sie hinter der $0$ ein Leerzeichen schreiben, um
sicherzustellen, da\ss{} die Konstante richtig erkannt wird, auch wenn der
nachfolgende Text auch wieder mit \index{Ziffer}Ziffern anf\"angt.

Hier nun eine vollst\"andige Liste der
Vergleichsbefehle.\index{Box!If}\index{Dimension!If}\index{Gleichheit!If}
\index{Horizontal!Modus}
\begin{enumerate}
\item \index{zzzifnum@{\tt\char92 ifnum}}
`\verb|\ifnum<Nummer1><Relation><Nummer2>|'\\Vergleicht zwei
Zahlen miteinander. Als \index{Relation}
Relation sind dabei nur die Vergleiche
`\verb|<|', `\verb|=|' und `\verb|>|' zugelassen.
\item \index{zzzifdim@{\tt\char92 ifdim}}
`\verb|\ifdim<Dimension1><Relation><Dimension2>|'\\Genau wie der
letzte Befehl, nur da\ss{} zwei Dimensionen vergliechen werden.
\item \index{zzzifodd@{\tt\char92 ifodd}}
`\verb|\ifodd<Zahl>|'\\Testet, ob die \"ubergebene Zahl ungerade
ist.
\item \index{zzzifvmode@{\tt\char92 ifvmode}}
`\verb|\ifvmode|'\\Testet, ob sich \TeX\ gerade im vertikalen,
oder im internen vertikalen Modus befindet.
\item \index{zzzifhmode@{\tt\char92 ifhmode}}
`\verb|\ifhmode|'\\Testet, ob sich \TeX\ gerade im horizontalen,
bzw.\ im eingeschr\"ankten horizontalen Modus befindet.
\item \index{zzzifmmode@{\tt\char92 ifmmode}}
`\verb|\ifmmode|'\\Testet, ob sich \TeX\ im mathematischen Modus
befindet.
\item `\verb|\if<Token1><Token2>|'\\Die beiden \"ubergebenen Token
werden miteinander auf Gleichheit vergleichen. Allerdings nicht
uneingeschr\"ankt. Tats\"achlich werden die \"ubergebenen Token expandiert,
falls das erforderlich ist, und beim ersten nichtexpandierbaren
Zeichen wird der Zeichencode getestet.
\item \index{zzzifcat@{\tt\char92 ifcat}}
`\verb|\ifcat<Token1><Token2>|'\\Arbeitet wie der letzte Befehl,
nur da\ss{} jetzt nicht die Zeichencodes vergleichen werden, sondern die
\index{Kategorie!Vergleich}Kategoriecodes.
\item \index{zzzifx@{\tt\char92 ifx}}
`\verb|\ifx<Token1><Token2>|'\\Jetzt wird der Vergleich
ausgef\"uhrt, {\em ohne} da\ss{} die Token zuvor expandiert werden. Handelt
es sich um Zeichen, dan wird Zeichen- und Kategoriecode verglichen,
sind es beides Makros, dann wird ihr Status verglichen.
\item \index{zzzifhbox@{\tt\char92 ifhbox}}\index{zzzifvbox@{\tt\char92 ifvbox}}
\index{zzzifvoid@{\tt\char92 ifvoid}}
`\verb|\ifvoid<Nummer>  \ifhbox<Nummer>  \ifvbox<Nummer>|'\\ Die
\"ubergebene Nummer ist eine Zahl zwischen $0$ und $255$, und die
\index{Bedingung}
Bedingung ist wahr, wenn das entsprechende Boxregister
\index{Leer!Register}leer ist, bzw.
eine hbox oder vbox enth\"alt.
\item \index{zzzifeof@{\tt\char92 ifeof}}
`\verb|\ifeof|'\\Testet auf Fileende.\index{File!Test auf Ende}
\item \index{zzziffalse@{\tt\char92 iffalse}}\index{zzziftrue@{\tt\char92 iftrue}}
`\verb|\iftrue  \iffalse|'\\Sind immer wahr, bzw.\ immer falsch.
\end{enumerate}
Schlie\ss{}lich gibt es noch einen Befehl, der den einfachen
Vergleichsbefehlen verwandt ist.
\index{zzzelse@{\tt\char92 else}}\index{zzzfi@{\tt\char92 fi}}
\index{zzzifcase@{\tt\char92 ifcase}}
\begin{verbatim}
\ifcase<Nummer><Text Fall 0>\or
               <Text Fall 1>\or
                     ...
               <Text Fall n>\else
               <Text in allen anderen Situationen>\fi
\end{verbatim}
Wenn die \"ubergebene Nummer den Wert $0$ hat wird der erste Text
ausgef\"uhrt, ist der Wert $1$, dann der zweite, usf. Pa\ss{}t keiner der
angegebene F\"alle zu der Zahl, dann wird der Text hinter dem
`\verb|\else|' ausgef\"uhrt.

Das Auftreten von `\verb|\if|' und dem beendenden
\index{zzzfi@{\tt\char92 fi}}`\verb|\fi|' ist
{\em immer} paarweise, allerdings ist es unabh\"angig von
\index{Klammer!Gruppe}Gruppierungsklammern.\footnote{Achtung, hier
gibt es die M\"oglichkeit einige echte Sauereien zu programmieren!}

Wenn man eine eigene Bedingung erst schaffen will, dann gibt es noch
den Befehl \index{zzznewif@{\tt\char92 newif}}`\verb|\newif|'. Mit
\index{zzzifabc@{\tt\char92 ifabc}}
\begin{verbatim}
\newif\ifabc
\end{verbatim}
werden die Befehl `\verb|\ifabc|', \index{zzzabctrue@{\tt\char92 abctrue}}
`\verb|\abctrue|' und
\index{zzzabcfalse@{\tt\char92 abcfalse}}
`\verb|\abcfalse|' geschaffen. Mit dem ersten kann die Bedingung
abgefragt werden, mit den beiden anderen kann die Bedingung gesetzt
werden.

F\"ur die \index{Uebergabe@\"Ubergabe}
\"Ubergabe von Token an eine Definition oder einen Vergleich
kennt \TeX\ noch $256$ Tokenregister, die
\index{zzztoks@{\tt\char92 toks}}`\verb|\toks0|' bis
`\verb|\toks255|' hei\ss{}en. Die \index{Zuweisung}
Zuweisung an ein Tokenregister erfolgt mit
\begin{verbatim}
\toks<Nummer>={Token} oder
\toks<Nummer>=\toks<Nummer}
\end{verbatim}
Es gibt nat\"urlich auch hier wieder einen \index{zzznewtoks@{\tt\char92 newtoks}}
`\verb|\newtoks|' Befehl, der
entsprechend den anderen \index{zzznew@{\tt\char92 new}}
`\verb|\new|' Befehlen arbeitet.
\section{Expansionen}
Wenn \TeX\ einen Text abarbeitet, dann wird Token nach Token
eingelesen. Ist eines dieser Token ein Makro, dann wird dieses erst
expandiert, bevor \TeX\ seine Arbeit fortsetzt.\footnote{Als Ausnahme
gibt es nat\"urlich Befehle, wie {\tt\char92 def} oder
\index{zzzifx@{\tt\char92 ifx}}{\tt\char92
ifx}.} Es gibt unterschiedliche \index{Kontrollsequenz}
Kontrollsequenzen, die expandiert
werden m\"ussen.\index{Bedingung}\index{Bildschirm}\index{Ersatztext}
\begin{enumerate}
\item {\em Makros} Wenn ein Makro expandiert wird, dann sucht \TeX\
ggf.\ erst nach Argumenten und ersetzt dann die
\index{Kontrollsequenz}
Kontrollsequenz durch
den angegebenen Ersatztext.
\item {\em Vergleiche} \TeX\ entscheidet, ob die Bedingung zutrifft,
oder nicht und ersetzt dann die Bedingung durch den der Bedingung
entsprechenden Text.
\item \index{zzznumber@{\tt\char92 number}}
`\verb|\number<Nummer>|' Die Nummer, die hinter dem
`\verb|\number|' Befehl steht wird in eine \index{Ziffer}
Ziffernfolge expandiert, je
nach Wert der Nummer noch mit einem f\"uhrenden \index{Minuszeichen}
Minuszeichen.
\item \index{zzzromannumeral@{\tt\char92 romannumeral}}
`\verb|\romannumeral<Nummer>|' Komplett analog zum vorherigen
Befehl, nur da\ss{} jetzt die Zahl in r\"omische Zahlzeichen expandiert
wird.
\item \index{zzzstring@{\tt\char92 string}}
`\verb|\string<Token>|' Steht hinter dem Befehl eine
\index{Kontrollsequenz}
Kontrollsequenz, dann wird der Befehl zu der Zeichenkette expandiert,
aus der die Kontrollsequenz besteht (inkl. dem f\"uhrenden Zeichen,
entsprechend \index{Escape}\index{zzzescapechar@{\tt\char92 escapechar}}
`\verb|\escapechar|'). Handelt es sich nur um ein
einfaches Zeichen, dann wird dieses Zeichen als Ergebnis der Expansion
ausgegeben.
\item \index{zzzjobname@{\tt\char92 jobname}}
`\verb|\jobname|' Dieser Befehl wird zu dem augenblicklichen
Arbeitsnamen expandiert. Wenn \TeX\ z.B.\ in die \index{Ausgabe!File}
Ausgabefiles
{\em Papier.dvi} und {\em Papier.log} schreibt, dann wird
`\verb|\jobname|' zu {\em Papier} expandiert.
\item \index{zzzfontname@{\tt\char92 fontname}}
`\verb|\fontname|' wird zu dem angegebenen \index{Font!Name}Fontnamen expandiert.
Die Angabe von\\`\verb|\fontname\tenrm|'\\w\"urde z.B.\ im
Normalfall die Expansion {\em cmr10} zur Folge haben.
\item \index{zzzmeaning@{\tt\char92 meaning}}
`\verb|\meaning<Token>|' gibt das aus, was auf dem Bildschirm
mit der Anweisung\\\index{zzzlet@{\tt\char92 let}}
`\verb|\let\test=<Token>|'\\und\\\index{zzzshow@{\tt\char92 show}}
`\verb|\show\test|'\\
ausgegeben w\"urde. Wenn es sich um den Buchstaben `A' handelt,
w\"are das Ergebnis: `{\em The letter A}'
\item \index{zzzcsname@{\tt\char92 csname}}
\index{zzzendcsname@{\tt\char92 endcsname}}
`\verb|\csname|'$\ldots$`\verb|\endcsname|' Dieses Befehlspaar
ist das Gegenteil des \index{zzzstring@{\tt\char92 string}}
`\verb|\string|' Befehls. Alle Token zwischen
den beiden Befehlen werden solange expandiert, bis nur noch Buchstaben
\"ubrig sind. Diese werden dann als \index{Kontrollsequenz}
Kontrollsequenz interpretiert.
\item \index{zzzexpandafter@{\tt\char92 expandafter}}
`\verb|\expandafter<Token>|' \TeX\ lie\ss{}t bei diesem Befehl das
Token, das unmittlebar hinter dem Befehl steht ein und expandiert es
nicht. Das folgende Token wird expandiert und schlie\ss{}lich das nicht
expandierte wieder davor geschrieben.
\item \index{zzznoexpand@{\tt\char92 noexpand}}
`\verb|\noexpand<Token>|' Die Expansion ist das Token selber,
das aber behandelt wird, wie der Befehl
\index{zzzrelax@{\tt\char92 relax}}`\verb|\relax|'.
\item \index{zzzbotmark@{\tt\char92 botmark}}
\index{zzzfirstmark@{\tt\char92 firstmark}}
\index{zzzsplitbotmark@{\tt\char92 splitbotmark}}
\index{zzzsplitfirstmark@{\tt\char92 splitfirstmark}}
\index{zzztopmark@{\tt\char92 topmark}}
`\verb|\topmark, \firstmark, \botmark, \splitfirstmark,
\splitbotmark|'\\Die Expansionen sind der Inhalt des entsprechenden
\index{Markregister}Markregisters, wie noch besprochen wird.
\item \index{zzzinput@{\tt\char92 input}}
`\verb|\input<Filename>|' Die Expansion ist
\index{Leer!Expansion}leer, aber \TeX\
bereitet sich darauf vor aus den angegebenen File zu lesen.
\item \index{zzzendinput@{\tt\char92 endinput}}
`\verb|\endinput|' Die Expansion ist wieder leer, aber \TeX\
beendet die \index{Eingabe!Vom File}
Eingabe vom augenblicklichen File, sobald die aktuelle
Zeile bearbeitet wurde.
\item \index{zzzthe@{\tt\char92 the}}
`\verb|\the<Interne Gr\"o\ss{}e>|' Der `\verb|\the|' Befehl gibt eine
Reihe von Token aus, die dem Wert der internen Gr\"o\ss{}e entsprechen. Die
Anweisung `\verb|\the\skip5|' k\"onnte z.B.\ {\em 5.0pt plus 2.0fil}
sein. Der `\verb|\the|' Befehl hat viele Unterarten, die jetzt
zusammen vorgestellt werden sollen.
\index{Horizontal!Modus}
\begin{enumerate}\index{Delcode}\index{Dimension}
\item `\verb|\the<Parameter>|' Wobei \index{Parameter!Intern}
Parameter eines der internen
Parameter von \TeX\ ist. Dies k\"onnen\\ \index{Parameter!Ganzzahl}
Ganzzahlparameter
(`\verb|\the\widowpeanlty|'),\\ \index{Parameter!Dimension}
Dimensionsparameter\index{zzzparindent@{\tt\char92 parindent}}
(`\verb|\the\parindent|')\\ \index{Leim}
\index{Parameter!Leim}Leimparameter \index{zzzleftskip@{\tt\char92 leftskip}}
(`\verb|\the\leftskip|') oder\\
\index{Muglue}\index{Parameter!Muglue}
Muglueparameter\index{zzzthinmuskip@{\tt\char92 thinmuskip}}
(`\verb|\the\thinmuskip|') sein.
\item `\verb|\the<register>|' womit die Inhalte der Zahl-,
Dimensions-, Leim- und Mugluregister ausgegeben werden k\"onnen.
\item `\verb|\the<Codename><8-Bit Zahl>|' Hierbei steht
\index{zzzcatcode@{\tt\char92 catcode}}
`\verb|<Codename>|' f\"ur\\`\verb|\catcode|',\\
\index{zzzmathcode@{\tt\char92 mathcode}}`\verb|\mathcode|',\\
\index{zzzlccode@{\tt\char92 lccode}}
\index{zzzuccode@{\tt\char92 uccode}}
`\verb|\lccode|',\\`\verb|\uccode|',\\
\index{zzzsfcode@{\tt\char92 sfcode}}`\verb|\sfcode|', oder\\
\index{zzzdelcode@{\tt\char92 delcode}}
`\verb|\delcode|'.\\Die Anweisung `\verb|\the\mathcode`/|' gibt den
augenblich g\"ultigen mathematischen Codewert des Schr\"ag\-strichs aus.
\item `\verb|\the<Spezialregister>|' Als Spezialregister kommen dabei
die folgenden in Frage:\\\index{zzzdeadcycles@{\tt\char92 deadcycles}}
\index{zzzinsertpenalties@{\tt\char92 insertpenalties}}
\index{zzzprevgraf@{\tt\char92 prevgraf}}
`\verb|\prevgraf, \deadcycles,
\insertpenalties|',\\ \index{zzzbadness@{\tt\char92 badness}}
\index{zzzinputlineno@{\tt\char92 inputlineno}}
\index{zzzparshape@{\tt\char92 parshape}}
\verb|\inputlineno, \badness, \parshape|'\\oder
eine der Dimensionen\\
\index{zzzpagegoal@{\tt\char92 pagegoal}}
\index{zzzpagestretch@{\tt\char92 pagestretch}}
\index{zzzpagetotal@{\tt\char92 pagetotal}}
`\verb|\pagetotal, \pagegoal, \pagestretch,
\pagefilstretch|'\\\index{zzzpagedepth@{\tt\char92 pagedepth}}
\index{zzzpagefilllstretch@{\tt\char92 pagefillstretch}}
\index{zzzpagefillstretch@{\tt\char92 pagefillstretch}}
\index{zzzpagefilstretch@{\tt\char92 pagefilstretch}}
\index{zzzpageshrink@{\tt\char92 pageshrink}}
`\verb|\pagefillstretch, \pagefilllstretch,
\pageshrink, \pagedepth|'\\Im horizontalen Modus kann au\ss{}erdem noch
die Dimension \index{zzzspacefactor@{\tt\char92 spacefactor}}
`\verb|\spacefactor|' und im vertikalen Modus die
Dimension \index{zzzprevdepth@{\tt\char92 prevdepth}}
`\verb|\prevdepth|' ausgegeben werden.
\item `\verb|\the\fontdim<Parameternummer><Font>|' Jeder
\index{Font!Parameter}Font
(Zeichensatz) hat seine eigenen \index{Parameter!Zeichensatz}
Parameter. Der sechste Parameter ist
der Wert von `em'. Mit\\`\verb|\the\fontdim6\tenrm|'\\erhielte man
z.B.\ {\em 10.0pt}.
\item `\verb|\the\hyphenchar<Font> \the\skewchar<Font>|'
\index{zzzskewchar@{\tt\char92 skewchar}}
\"ubergibt das
entsprechende Zeichen, das f\"ur diesen Font definiert wurde.
\item `\verb|\the\lastpenalty \the\lastkern \the\lastskip|' ergibt die
letzte Strafe, den letzten Kern oder den letzten \index{Zwischenraum}
Zwischenraum,
vorausgesetzt, da\ss{} das letzte Element der aktuellen Liste auch genau
das entsprechende Element war.
\item `\verb|\the<Definiertes Zeichen>|' gibt den ganzzahligen Wert
einer \index{Kontrollsequenz}
Kontrollsequenz zu\-r\"uck, die mit einem der `\verb|\...def|'
Befehle zuvor gesetzt wurde.
\end{enumerate}\index{Badness}
Normalerweise produziert der `\verb|\the|' Befehl eine Zeichenkette
aus \index{ASCII}ASCII-Zeichen. Nur in einigen F\"allen gilt das nicht.
\begin{enumerate}
\item `\verb|\the<Font>|' w\"ahlt den angegebenen Font aus.
\index{zzzfont@{\tt\char92 font}}
`\verb|\the\font|' w\"ahlt z.B.\ den augenblicklichen Font.
\item `\verb|\the<Tokenvariable>|' \"ubergibt eine Kopie des Inhalts des
Tokenregisters.
\end{enumerate}
Neben dem `\verb|\the|' Befehl existiert auch noch der
\index{zzzshowthe@{\tt\char92 showthe}}`\verb|\showthe|'
Befehl, der absolut analog arbeitet, nur da\ss{} die \index{Ausgabe!Auf
dem Bildschirm}Ausgabe auf dem Bildschirm erfolgt.
\end{enumerate}
\subsection{Wann wird nicht expandiert}
Es gibt Situationen, in denen eine \index{Kontrollsequenz}
Kontrollsequenz oder ein Makro
nicht expandiert wird. Hier nun eine vollst\"andige Liste all dieser
Situationen. Manche \index{Primitiv}
Primitive, die hier angesprochen werden sind noch
unbekannt, sie werden sp\"ater
besprochen.\index{Dollarzeichen}\index{Ersatztext}
\begin{enumerate}
\item Wenn ein Token w\"ahrend des Versuchs einen Fehler zu beheben,
gel\"oscht wird.
\item Wenn ein Token in einer Vergleichsanweisung steht, die aufgrund
der Bedingung nicht ausgef\"uhrt wird.
\item Wenn \TeX\ die Argumente eines Makros liest.
\item Wenn \TeX\ eine \index{Kontrollsequenz}
Kontrollsequenz einliest, die mit einem der
\index{zzzlet@{\tt\char92 let}}
`\verb|\let|' oder einem der \index{zzzdef@{\tt\char92 def}}
`\verb|\def|' Befehle definiert werden soll.
\item Wenn \TeX\ Argumente f\"ur einen der Befehle
\index{zzzafterassignment@{\tt\char92 afterassignment}}
\index{zzzaftergroup@{\tt\char92 aftergroup}}
\index{zzzexpandafter@{\tt\char92 expandafter}}
\index{zzzfuturelet@{\tt\char92 futurelet}}
\index{zzzifx@{\tt\char92 ifx}}
\index{zzzlet@{\tt\char92 let}}
\index{zzzmeaning@{\tt\char92 meaning}}
\index{zzznoexpand@{\tt\char92 noexpand}}
\index{zzzshow@{\tt\char92 show}}
\index{zzzstring@{\tt\char92 string}}\\
`\verb|\expandafter|',\\
`\verb|\noexpand|',\\
`\verb|\string|',\\
`\verb|\meaning|',\\
`\verb|\let|',\\
`\verb|\futurelet|',\\
`\verb|\ifx|',\\
`\verb|\show|',\\
`\verb|\afterassignment|',\\
`\verb|\aftergroup|'\\einliest.
\item Beim Einlesen einer \index{Parameter!Definition}
Parameterliste der \index{zzzdef@{\tt\char92 def}}`\verb|\def|' Befehle.
\item Beim Einlesen eines Ersatztextes, oder beim Einlesen einer
Tokenliste, dies geschieht z.B. bei der Bearbeitung von
\index{zzzlowercase@{\tt\char92 lowercase}}`\verb|\lowercase|'.
\item Beim Einlesen des Vorspanns einer Tabelle.
\item Wenn ein Dollarzeichen eingelesen wurde.
\item Wenn ein `\verb|`|' einen numerischen Wert einleitet.
\end{enumerate}
Wenn sie erreichen wollen, da\ss{} in einer Makrodefinition der
\index{Ersatztext}Ersatztext
nach den obigen Regeln expandiert werden soll, dann verwenden sie den
Befehl \index{zzzedef@{\tt\char92 edef}}
`\verb|\edef|', statt des normalen `\verb|\def|' Befehls, oder
den Befehl \index{zzzxdef@{\tt\char92 xdef}}
`\verb|\xdef|', der wie der `\verb|\edef|' Befehl arbeitet,
nur \index{Global!Definition}global. Mit der Anweisung
\begin{verbatim}
\def\doppelt#1{#1#1}
\edef\a{\doppelt{xy}}
\end{verbatim}
w\"urden sie erreichen, da\ss{} bei Aufruf von `\verb|\a|' {\em xyxy}
ausgegeben w\"urde. Mit \index{zzznoexpand@{\tt\char92 noexpand}}
`\verb|\noexpand|' wird die
Expandierung des folgenden Tokens unterbunden. So k\"onnen auch in einer
expandierenden Definition Teile vor der Expansion gesch\"utzt werden.

Bei den Befehlen
\index{zzzerrmessage@{\tt\char92 errmessage}}\index{zzzmark@{\tt\char92 mark}}
\index{zzzmessage@{\tt\char92 message}}
\index{zzzspecial@{\tt\char92 special}}
\index{zzzwrite@{\tt\char92 write}}
\begin{verbatim}
\mark{...}
\message{...}
\errmessage{...}
\special{...}
\write{...}
\end{verbatim}
wird der \"ubergebene Text expandiert, genau wie bei einer
\index{zzzedef@{\tt\char92 edef}}`\verb|\edef|'
Definition. Allerdings braucht das \index{Parameter!Zeichen}
Parameterzeichen `\verb|#|' {\em
nicht} doppelt verwendet werden, um es einmal zu erhalten, es gen\"ugt,
im Gegensatz zur Verwendung in Makros, die einmalige Verwendung.
\subsection{Fileeingabe}
Neben der M\"oglichkeit mit dem \index{zzzinput@{\tt\char92 input}}
`\verb|\input|' Befehl weitere Files
einzulesen kann \TeX\ gleichzeitig bis zu $16$ Files zur Eingabe
verwalten. Um einen derartigen File bereitzustellen, sollten sie ihn
mit
\index{zzzopenin@{\tt\char92 openin}}
\begin{verbatim}
\openin<Nummer>=<Filename>
\end{verbatim}
\"offnen. Die Nummern liegen hierbei wieder zwischen $0$ und $15$, wobei
es auch wieder einen \index{zzznewread@{\tt\char92 newread}}
`\verb|\newread|' Befehl gibt, der absolut analog
zu den entsprechenden Befehlen arbeitet.\footnote{Normalerweise wird
von den meisten Implementationen das {\tt .tex} automatisch angeh\"angt,
wenn keine spezielle \index{Extension}
Extension angegeben ist.} Wird der File nicht
gefunden, oder kann aus einem anderen Grunde nicht gelesen werden,
dann gibt \TeX\ {\em keine} \index{Fehler!Meldung}
Fehlermeldung aus. Sie m\"ussen mit
\index{zzzifeof@{\tt\char92 ifeof}}
`\verb|\ifeof|' selber pr\"ufen, ob die Eingabe durch den File m\"oglich
ist. Nach dem Einlesen, sollten sie den File mit
\index{zzzclosein@{\tt\char92 closein}}
\begin{verbatim}
\closein<Nummer>
\end{verbatim}
wieder schlie\ss{}en. Eine Eingabe aus dem File erhalten sie nun mit der
Anweisung
\index{zzzread@{\tt\char92 read}}
\begin{verbatim}
\read<Nummer>to<Kontrollsequenz>
\end{verbatim}
Dabei wird die n\"achste Zeile des Files eingelesen, oder mehr, falls
sich sonst unpaarige geschweifte \index{Klammer!Unpaarig}
Klammern ergeben w\"urden, und in der
\index{Kontrollsequenz}
Kontrollsequenz gespeichert. Diese Definition der Kontrollsequenz ist
lokal, solange dem `\verb|\read|' kein \index{zzzglobal@{\tt\char92 global}}
`\verb|\global|' vorangestellt
wird. Mit Zahlen au\ss{}erhalb den Zul\"assigen, kann ein Dialog mit dem
Benutzer gef\"uhrt werden.
\index{zzzmessage@{\tt\char92 message}}\index{zzzmyname@{\tt\char92 myname}}
\index{zzzread@{\tt\char92 read}}
\begin{verbatim}
\message{Tippen sie ihren Namen}
\read16 to \myname
\message{Hallo, \myname}
\end{verbatim}
fordert den Benutzer auf seinen Namen einzugeben, und gibt diesen
danach, hinter dem Wort `Hallo' wieder aus.
\subsection{Schleifen}
\TeX\ bietet auch die M\"oglichkeit Schleifen zu programmieren. Mit der
Anweisung
\index{zzzloop@{\tt\char92 loop}}\index{zzzrepeat@{\tt\char92 repeat}}
\begin{verbatim}
\loop <Teil 1> \if... <Teil 2> \repeat
\end{verbatim}
wird folgendes bewirkt. Zun\"achst wird der erste Teil abgearbeitet, und
dann die \index{Bedingung}
Bedingung vom `\verb|\if|' getestet.\footnote{Dies ist die
einzige Variante, bei der {\tt\char92 if} {\em ohne} passendes
{\tt\char92 fi} auftritt.} Wenn die Bedingung wahr ist, wird der
zweite Teil ausgef\"uhrt und dann die ganze Schleife wiederholt. Ist die
Bedingung falsch, dann wird der zweite Teil nicht mehr ausgef\"uhrt und
die Schleife sofort verlassen.\footnote{Das \TeX book liefert an
dieser Stelle einige bemerkenswerte Beispiele, wie man \TeX\ als
\index{Taschenrechner}
Taschenrechner, oder zur Berechnung von \index{Primzahlen}
Primzahlen einsetzen kann, ich
habe mir die \"Ubernahme erspart :-)}
\chapter{Weitere Boxbefehle}\index{Box!Befehle}
In diesem Kapitel sollen die letzten, noch unbesprochenen Boxbefehle
erl\"autert werden. Es stellt insofern nur eine Erg\"anzung der schon
bekannten Befehle dar.
\section{Die Strichboxen}
Um eine Strichbox, genauer eine rechteckige, geschw\"arzte Fl\"ache zu
erzeugen, benutzen sie im vertikalen Modus den Befehl `\verb|\hbox|'
und im \index{Horizontal!Modus}
horizontalen Modus den Befehl \index{zzzvbox@{\tt\char92 vbox}}
`\verb|\vbox|', inklusive der
\index{Dimension!Angabe}Dimensionsangaben, also z.B.
\index{zzzvrule@{\tt\char92 vrule}}
\begin{verbatim}
\vrule height4pt width3pt depth2pt
\end{verbatim}
in einem Paragraphen, und sie erhalten `\vrule height4pt width3pt
depth2pt' als Ergebnis. Wenn eine der Dimensionen
\index{Doppelt!Dimensionsangabe}doppelt angegeben
wird, gilt die zweite Angabe, wird einer der Dimensionen ausgelassen,
dann gelten folgende Standardwerte:

\medskip\begin{tabular}{lcc}
 & {\tt\char92 hrule} & {\tt\char92 vrule} \\
Breite & $*$ & 0.4pt\\
H\"ohe & 0.4pt & $*$\\
Tiefe & 0.0pt & $*$
\end{tabular}\medskip

Der `$*$' bedeutet, da\ss{} sich die Rulebox an der \index{Dimension}
Dimension der kleinsten umschlie\ss{}enden Box orientiert.\medskip

\hrule\medskip

Wenn sie z.B.\ einfach \index{zzzhrule@{\tt\char92 hrule}}
`\verb|\hrule|' schreiben, dann erhalten sie
obiges Ergebnis, da die kleinste umschlie\ss{}ende Box die Seite selber
ist. Mit
\index{zzzhrule@{\tt\char92 hrule}}
\index{zzzvskip@{\tt\char92 vskip}}
\begin{verbatim}
\hrule width5cm height1pt \vskip1pt \hrule width6cm
\end{verbatim}
erhalten sie das Ergebnis {\em unter} diesem Abschnitt. Zwischen
Ruleboxen wird {\em kein} zus\"atzlicher \index{Zwischenraum}
Zwischenraum eingef\"ugt. Die
Zeilen sind genau ein pt voneinander entfernt.\medskip

\hrule width5cm height1pt \vskip1pt \hrule width6cm\medskip

Sie m\"ussen sich schon an die Vorgabe halten, im vertikalen Modus nur
horizontale Boxen zu erzeugen (`\verb|\hrule|') und im
\index{Horizontal!Modus}horizontalen
Modus nur vertikale \index{zzzvrule@{\tt\char92 vrule}}
(`\verb|\vrule|'). Auch wenn die
\index{Dimension!Angabe}Dimensionsangaben
\"ubereinstimmen und deshalb sich die Boxen durch nichts unterscheiden,
passiert folgendes. Wenn sie im vertikalen Modus den Befehl
`\verb|\vrule|' verwenden, beginnt \TeX\ einen \index{Neu!Paragraphen}
neuen Paragraphen.
Verwenden sie \index{zzzhrule@{\tt\char92 hrule}}
`\verb|\hrule|' im horizontalen Modus, dann beendet
\TeX\ den augenblicklichen Paragraphen und geht in den vertikalen
Modus \"uber.

Negative Angaben bei einer Rulebox sind zul\"a\ss{}ig und sinnvoll. Die Box
`\vrule width3cm depth-2pt' hat beispielsweise eine Tiefe von $-2$pt.
Allerdings mu\ss{} die Breite positiv sein und auch die Summe von
\index{Hoehe!Box@H\"ohe!Box}H\"ohe und
Tiefe mu\ss{} positiv sein, damit \"uberhaupt eine Box gezeichnet wird.
\section{Boxtypen}
Es gibt verschiedene Typen von Boxen: (1) Jedes Zeichen stellt f\"ur sich
eine Box dar. (2) Die Befehle `\verb|\hrule|' und `\verb|\vrule|'
erzeugen Boxen und (3) lassen sich Boxen auch explizit angeben
\index{zzzbox@{\tt\char92 box}}\index{zzzcopy@{\tt\char92 copy}}
\index{zzzlastbox@{\tt\char92 lastbox}}
\index{zzzvbox@{\tt\char92 vbox}}
\index{zzzvsplit@{\tt\char92 vsplit}}
\index{zzzvtop@{\tt\char92 vtop}}
\begin{verbatim}
\hbox<Box Spez.>{<Horizontales Material>}
\vbox<Box Spez.>{<Vertikales Material>}
\vtop<Box Spez.>{<Vertikales Material>}
\box<Registernummer>
\copy<Registernummer>
\vsplit<Registernummer> to <Dimension>
\lastbox
\end{verbatim}
Die `\verb|<Box. Spez.>|' sind dabei die Angaben \index{Dimension!Box}
`\verb|to <Dimension>|'
oder `\verb|spread <Dimension>|', sie k\"onnen nat\"urlich
auch wegfallen. Der letzte Befehl ist bisher noch unbekannt. Au\ss{}er man
befindet sich im normalen vertikalen Modus, enth\"alt `\verb|\lastbox|'
immer dann eine Box, wenn das letzte Element der aktuellen Liste eine
Box war, sonst, und auch im mathematischen Modus ist sie leer.

Der \index{zzzunskip@{\tt\char92 unskip}}
`\verb|\unskip|' Befehl \"ahnelt dem `\verb|\lastbox|' Befehl, au\ss{}er
da\ss{} er sich auf \index{Leim}
Leim und nicht auf Boxen bezieht. Im vertikalen Modus
erbringt der Befehl `\verb|\unskip|' nicht die gew\"unschte Wirkung,
also die Entfernung des letzten Leimst\"uckes, aber mit
\index{zzzlastskip@{\tt\char92 lastskip}}
\index{zzzvskip@{\tt\char92 vskip}}
\begin{verbatim}
\vskip-\lastskip
\end{verbatim}
kann derselbe Effekt erreicht werden.

In den zusammenfassenden Kapiteln am Ende dieses Buches wird unter
einer Box immer eine der explizit angegebenen verstanden, da
Zeichen(boxen) und Ruleboxen so speziell sind, da\ss{} sie nicht wie die
anderen Boxen behandelt werden.
\section{Leaders}
Was sie in dieser Zeile sehen \xleaders\hbox to 6pt{$\hss\cdot\hss$}\hfil\
wird Leader genannt.\footnote{Vom englischen {\em lead} f\"ur leiten, da
sie die Augen entlang einer Zeile leiten sollen.} Leader sind der
\"ubergeordnete Begriff f\"ur \index{Leim}Leim. Leim f\"ullt einen
\index{Zwischenraum}Zwischenraum mit
\index{Leer!Zeichen}
Leerzeichen, bei Leadern kann der \index{Fuellbuchstabe@F\"ullbuchstabe}
F\"ullbuchstabe angegeben werden. Der Befehl f\"ur Leader lautet
\index{zzzhskip@{\tt\char92 hskip}}\index{zzzleaders@{\tt\char92 leaders}}
\begin{verbatim}
\leaders<Box oder Rule>\hskip<Leim>
\end{verbatim}
Der Effekt ist genauso, als h\"atten sie nur `\verb|\hskip<Leim>|'
angegeben, mit der Ausnahme, da\ss{} der \index{Zwischenraum}
Zwischenraum mit Kopien der Box
gef\"ullt wird. Die Eingabe von
\index{zzzalpha@{\tt\char92 alpha}}\index{zzzdef@{\tt\char92 def}}
\index{zzzhfill@{\tt\char92 hfill}}\index{zzzhss@{\tt\char92 hss}}
\index{zzzleaderfill@{\tt\char92 leaderfill}}
\index{zzzline@{\tt\char92 line}}
\begin{verbatim}
\def\leaderfill{\leaders\hbox to 1em{\hss.\hss}\hfill}
\line{$\alpha$\leaderfill $\omega$}
\line{Anfang\leaderfill Ende}
\end{verbatim}
ergibt

$$\vbox{
\def\leaderfill{\leaders\hbox to 1em{\hss.\hss}\hfill}
\hbox to \hsize{$\alpha$\leaderfill $\omega$}
\hbox to \hsize{Anfang\leaderfill Ende}}$$

Die Angabe von `\verb|\hbox to 1em{\hss .\hss}|' erzeugt eine Box mit
einem em Breite, in dessen Mitte ein \index{Punkt}
Punkt steht. Mit diesen Boxen
wird dann die Zeile aufgef\"ullt. Der Grund daf\"ur, da\ss{} die Punkte genau
\"ubereinander liegen ist darin zu suchen, da\ss{} der Leaderbefehl
sozusagen nur einen kleinen Ausschnitt von einer ganzen Reihe von
Kopien dieser Box zeigt. Dies erm\"oglicht die Justierung der einzelnen
Boxen auch \"ubereinander. Leider hat dieses Verfahren auch einen
kleinen Nachteil. Vor dem ersten und hinter dem letzten \index{Punkt}
Punkt kann ein
etwas zu gro\ss{}er \index{Zwischenraum}
Zwischenraum stehen. Insbesondere, wenn ein Leader
alleine auftaucht, kann das etwas st\"orend sein. Dazu gibt es die
Befehle \index{zzzcleaders@{\tt\char92 cleaders}}
`\verb|\cleaders|' und `\verb|xleadres|'. Mit dem ersten
Befehl werden die Boxen insgesamt in dem Freiraum \index{Zentrierung}
zentriert, mit dem
zweiten werden die Boxen einzeln auf die passende L\"ange gezogen.

Statt einer Box kann auch eine Rulebox angegeben werden, evtl.\ mit
Angaben \"uber die \index{Hoehe!Rulebox@H\"ohe!Rulebox}
H\"ohe, Breite und Tiefe. Hier macht dann auch eine
`\verb|\hbox|' im \index{Horizontal!Modus}
horizontalen Modus Sinn, da so eine \index{Horizontal!Linie}horizontale
Linie ausgegeben werden kann. H\"atten wir die vorige Definition als:
\index{zzzdef@{\tt\char92 def}}\index{zzzhfill@{\tt\char92 hfill}}
\index{zzzhrule@{\tt\char92 hrule}}\index{zzzleaderfill@{\tt\char92 leaderfill}}
\index{zzzleaders@{\tt\char92 leaders}}
\begin{verbatim}
\def\leaderfill{\leaders\hbox to 1em{\hrule}\hfill}
\end{verbatim}
geschrieben, dann w\"are das Ergebnis

$\alpha$\hrulefill $\omega$

Anfang\hrulefill Ende

gewesen.

Leaders k\"onnen nat\"urlich genauso im vertikalen Modus verwendet werden,
wobei, wie auch im horizontalen Modus kein \index{Zwischenraum}
Zwischenraum zwischen den
einzelnen Boxen gelassen wird. Wenn sie Boxen angeben, die eine
negative Breite, oder eine negative \index{Gesamt!Hoehe@Gesamt!H\"ohe}
Gesamth\"ohe (H\"ohe plus Tiefe)
haben, ignoriert \TeX\ die Angabe und verwendet einfach \index{Leim}Leim.

\subsection{Anwendungen}
\subsubsection{Liegende geschweifte Klammern}\index{Klammer!Leader}
Mit den Befehlen
\index{zzzupbracefill@{\tt\char92 upbracefill}}
`\verb|\upbracefill|' und \index{zzzdownbracefill@{\tt\char92 downbracefill}}
`\verb|\downbracefill|'
k\"onnen ganze Textteile durch eine geschweifte Klammer \"uberzogen, oder
unterstrichen werden. Die vier Symbole `{\Mf\char122}',
`{\Mf\char123}', {\Mf\char124}' und `{\Mf\char125}' werden dabei durch
Ruleleader verbunden.
$$\vbox{\kern4pt
\hbox to 100pt{\downbracefill}\hbox to 50pt{\upbracefill}}$$
\subsubsection{Pfeile}
Auch Pfeile lassen sich nat\"urlich auf diese Art verl\"angern. W\"urde man
eine Verl\"angerung eines Pfeilsymbols mit \index{Minuszeichen}
Minuszeichen oder
\index{Gleichheitszeichen}Gleichheitszeichen versuchen, dann w\"urden die einzelnen Zeichen durch
kleine \index{Leer!Stelle}Leerstellen getrennt. Statt dessen dient der Befehl
\index{zzzrightarrowfill@{\tt\char92 rightarrowfill}}
\begin{verbatim}
\hbox to 100pt{\rightarrowfill}
\end{verbatim}
zu dem Ergebnis `\hbox to 100pt{\rightarrowfill}'
\subsubsection{Vertikale Anwendung}
Um auch die Anwendung vertikaler Leader einmal zu zeigen dient diese
Textur\index{zzzcleaders@{\tt\char92 cleaders}}
$$\hbox to 2.5in{\cleaders
     \vbox to .5in{\cleaders\hbox{\TeX}\vfil}\hfil}$$
Sie wurde mit
\index{zzzvbox@{\tt\char92 vbox}}
\index{zzzvfil@{\tt\char92 vfil}}
\begin{verbatim}
$$\hbox to 2.5in{\cleaders
     \vbox to .5in{\cleaders\hbox{\TeX}\vfil}\hfil}$$
\end{verbatim}
erzeugt.
\section{WasDenn}
Schon oft wurden sie erw\"ahnt, hier nun endlich soll gekl\"art werden,
worum es sich dabei handelt. Es handelt sich dabei um eine M\"oglichkeit
Eigenschaften von \TeX\ zu \"andern, ohne dabei den Sourcecode von \TeX\
selber allzu sehr zu modifizieren. Von dieser M\"oglichkeit sollte
allerdings nicht allzuviel Gebrauch gemacht werden, um die
Kompatibilit\"at der Texte nicht zu gef\"ahrden. In allen \TeX\
Implementationen sind allerdings mindestens zwei `WasDenn' mit
eingebaut\index{File!Ausgabe}
\begin{enumerate}
\item Die Fileausgabe.
\item Der {\tt\char92 special} Befehl.
\end{enumerate}
\subsection{Fileausgabe}\index{Ausgabe!File}
Die M\"oglichkeit der Fileausgabe erlaubt die Verwendung der Ausgabe
durch andere Programme, inklusive \TeX\ selber. Sie wird z.B.\ f\"ur
\index{Inhaltsverzeichnis}
Inhaltsverzeichnisse und \"ahnliche Anwendungen gebraucht. Analog zur
\index{Eingabe!Vom File}Fileeingabe gibt es die Befehle
\index{zzzcloseout@{\tt\char92 closeout}}\index{zzznewwrite@{\tt\char92 newwrite}}
\index{zzzopenout@{\tt\char92 openout}}
\index{zzzwrite@{\tt\char92 write}}
\begin{verbatim}
\openout<Nummer>=<Filename>
\closeout<Nummer>
\write<Nummer>{<Tokenliste>}
\newwrite
\end{verbatim}
Ist die Nummer gr\"o\ss{}er als 15 oder kleiner als Null, dann wird wieder
in den Logfile, bzw.\ auf dem \index{Bildschirm}
Bildschirmen ausgegeben. Es gibt aber
einen wesentlichen Unterschied, zu den entsprechenden Eingabebefehlen.
Die Ausgabe erfolgt nicht beim Auftreten des Befehls, sondern bei dem
Auftreten des WasDenn bei der eigentlichen Ausgabe der
\index{Seite!Ausgabe}Seite an den
\index{File!DVI}
Dvi-File. Eine sofortige Ausgabe kann durch unmittlebares voranstellen
des Befehls \index{zzzimmediate@{\tt\char92 immediate}}
`\verb|\immediate|' vor dem Schreibbefehl erreicht werden.
Die Anweisung
\index{zzzwrite@{\tt\char92 write}}
\begin{verbatim}
\immediate\write16{Auf Wiedersehen!}
\end{verbatim}
Bewirkt die Ausgabe von `Auf Wiedersehen!' auf dem \index{Bildschirm}
Bildschirm,
\"ahnlich, wie auch der \index{zzzmessage@{\tt\char92 message}}
`\verb|\message|' Befehl, allerdings wird mit
dem `\verb|\write|' Befehl immer eine eigene Zeile ausgegeben.
\subsection{Der {\tt\char92 special} Befehl}
Der Befehl \index{zzzspecial@{\tt\char92 special}}
`\verb|\special|' erlaubt es eine (expandierte) Tokenliste
in den Dvi-File zu schreiben. Au\ss{}erdem wird im Dvi-File auch vermerkt,
an welcher Stelle im Originalfile der Befehl stand. Diese expandierte
Tokenliste kann dann z.B.\ durch ein Ausgabeprogramm ausgewertet
werden. Bislang gibt es {\em keine} Standardisierung f\"ur die
{\tt\char92 special} Befehle.
\chapter{Tabellen und Ausrichtungen}
Die so oft vorkommenden Tabellen k\"onnen in \TeX\ auf zwei verschiedene
Arten gesetzt werden. Zum einen kennt \TeX\ eine Tabelle, und dann das
Werkzeug der \index{Horizontal!Ausrichtung}
horizontalen Ausrichtung. Zun\"achst die Tabellen.
\section{Tabellen}
\subsection{Gleiche Spaltenbreite}
Mit der Anweisung \index{zzzcolumns@{\tt\char92 columns}}
\index{zzzsettabs@{\tt\char92 settabs}}
`\verb|\settabs n \columns|' wird die laufende Zeile
in $n$ Spalten gebrochen. Jede der so geschaffenen Spalten ist gleich
breit.\footnote{Einen entsprechenden Befehl gibt es in \LaTeX\ nicht!}
Die einzelnen Zeilen dieser Tabelle werden dann mit:
\begin{verbatim}
\+ <Text 1> & <Text 2> & .... \cr
\end{verbatim}
eingegeben. Die Eingabe von
\index{zzzcolumns@{\tt\char92 columns}}
\index{zzzsettabs@{\tt\char92 settabs}}
\begin{verbatim}
\settabs 4 \columns
\+&&Text in der dritten Spalte\cr
\+&Text in der zweiten Spalte\cr
\+\it Text in der ersten Spalte&&&und in der vierten Spalte\cr
\end{verbatim}
hat folgende Wirkung:

\medskip\noindent\begin{tabular}{*{4}{p{30mm}}}
 & & Text 3. Spalte & \\
 & Text 2. Spalte & & \\
\multicolumn{2}{l}{\it Text in der 1. Spalte} & & und 4. Spalte
\end{tabular}\medskip

Dieses Beispiel zeigt einige wichtige Eigenschaften der
Tabulatorbefehle von \TeX.
\begin{enumerate}
\item Die {\tt\char38} entsprechen den Tabulatorstopps einer normalen
Schreibmaschine.
\item Der Unterschied besteht darin, da\ss{} es {\em immer} genausoviele
Tabstopps in einer Zeile gibt, wie es {\tt\char38} Zeichen gibt,
besonders gut an der untersten Zeile zu sehen.
\item Jede Zeile wird mit einem `\verb|\cr|' beendet. Dies kann auch
geschehen, wenn noch nicht alle Spalten mit {\tt\char38} getrennt
sind. Es kann also weniger als die angegebene Anzahl von Spalten geben.
\item Leerzeichen hinter `\verb|&|' und `\verb|\+|' werden ignoriert.
\item Jeder \index{Spalte!Eintrag}Spalteneintrag stellt eine eigene
\index{Gruppe!Tabelleneintrag}Gruppe dar. Das ist
besonders an der ersten Spalte der letzten Zeile zu sehen, bei der f\"ur
die lokale \index{Schrift!Umschaltung}
Schriftumschaltung {\em keine} Gruppenklammern verwendet
werden brauchten.
\end{enumerate}

Normalerweise bleiben die Spalten, die sie mit
\index{zzzsettabs@{\tt\char92 settabs}}`\verb|\settabs|'
eingerichtet haben bestehen, auch wenn sie danach wieder normalen Text
schreiben, der in normale Paragraphen umgebrochen wird. Sie k\"onnen die
Tabulatoren nat\"urlich auch innerhalb einer \index{Gruppe!Tabelle}Gruppe definieren, dann
gelten sie nur dort. Eine Globalisierung durch \index{zzzglobal@{\tt\char92 global}}
`\verb|\global|' ist
{\em nicht} zul\"a\ss{}ig.

Normalerweise werden Tabellen zwischen Paragraphen gesetzt. Es ist
aber auch sinnvoll sie in \index{Box!Vbox}
\verb|vbox|en zu gebrauchen. Damit sind z.B.\
Tabellen in \index{Formel!abgesetzte}abgesetzten Formeln m\"oglich. Mit
\index{zzzcolumns@{\tt\char92 columns}}
\index{zzzsettabs@{\tt\char92 settabs}}
\index{zzzvbox@{\tt\char92 vbox}}
\begin{verbatim}
$$\vbox{\settabs 3 \columns
  \+Dies ist&ein seltsames&Beispiel\cr
  \+mit einem&drei Spalten&Format.\cr}$$
\end{verbatim}
erhalten sie

$$\begin{array}{p{4cm}@{\qquad}p{4cm}@{\qquad}p{4cm}}
Dies ist&ein seltsames&Beispiel\\
mit einem&drei Spalten&Format.\\
\end{array}$$

\subsection{Unterschiedliche Spaltenbreite}
Es ist nat\"urlich nicht immer w\"unschenswert, da\ss{} die Spalten alle
dieselbe Breite haben. F\"ur diesen Fall mu\ss{} der Tabulatorbefehl etwas
abge\"andert werden. Dazu dient eine \index{Muster!Zeile}
{\em Musterzeile}, die unmittelbar
hinter den `\verb|\settabs|' Befehl geschrieben wird und auch mit
`\verb|\cr|' abgeschlossen wird.
\index{zzzindent@{\tt\char92 indent}}
\begin{verbatim}
\settabs\+\indent&Horizontale Liste\quad&\cr % sample line
\+&Horizontale Liste&Kapitel 14\cr
\+&Vertikale Liste&Kapitel 15\cr
\+&Math. Liste&Kapitel 17\cr
\end{verbatim}
Ist ein Beispiel f\"ur eine derartige Liste. Die \index{Muster!Zeile}
Musterzeile wird dabei
{\em nicht} ausgegeben, so da\ss{} das Ergebnis so aussieht

\medskip\begin{tabular}{ll}
Horizontale Liste&Kapitel 14\\
Vertikale Liste&Kapitel 15\\
Math. Liste&Kapitel 17
\end{tabular}\medskip

Bei der Konstruktion einer l\"angeren Tabelle mit dieser Methode mu\ss{}
man darauf achten, da\ss{} man den jeweils l\"angsten Spalteneintrag in die
\index{Muster!Zeile}Musterzeile schreibt.

Mit \index{Einfuegung!Fuellmaterial@Einf\"ugung!F\"ullmaterial}
Einf\"ugung von \index{zzzhfill@{\tt\char92 hfill}}
`\verb|\hfill|' an geeigneten Stellen kann der Text
auch andere Ausrichtungen, als die \index{Linksbuendig@Linksb\"undig}
Linksb\"undigkeit erhalten. In {\em Plain}
\TeX\ wird definiert, da\ss{} jeder Spalteneintrag in eine \index{Box!Hbox}
Hbox mit der
Breite einer Spalte geschrieben wird, gefolgt von dem Befehl
\index{zzzhss@{\tt\char92 hss}}
`\verb|\hss|'. \index{zzzhfill@{\tt\char92 hfill}}
`\verb|\hfill|' ist der m\"achtigere Befehl, so da\ss{} damit
die Wirkung von `\verb|\hss|' unterbunden werden kann. Eine einfache
\index{Zentrierung}
Zentrierung ist allerdings mit einem vorgestellten `\verb|\hss|'
m\"oglich, das nachgestellte wird ja automatisch eingef\"ugt. Dies geht
allerdings nur in den ersten Spalten. In der letzten Spalte wird der
Text in eine Hbox geschrieben, die die nat\"urliche Breite des Textes
hat.

Eine sehr gute Anwendungsm\"oglichkeit dieser variablen Tabulatorstops
bietet die Dokumentation von Programmsourcen. Stellen sie sich vor,
sie sollten folgendes Computerprogramm in der vorgestellten Weise
setzen.

\medskip\begin{tabbing}
\qquad\= {\bf if}\= $\quad n<r\quad$\= {\bf then}\= $\quad n=n+1$\\
\> \> \> {\bf else} \> {\bf begin} $print;\quad n:=0;$\\
\> \> \> \> {\bf end;}\\
\> {\bf while} $p>0$ {\bf do} \> \> \> \\
\> \> {\bf begin} $\quad q:=link(p);\quad free(p);\quad p:=q;$ \> \> \\
\> \> {\bf end;} \> \> \\
\end{tabbing}\medskip

Es w\"are ziemlich m\"uhsam f\"ur jeder der vorkommenden Tabulatorpositionen
einen Tabulatorstop zu setzen. Es ist nun zwar so, da\ss{} man durch
einf\"ugen eines weiteren `\verb|&|' jederzeit eine
\index{Neu!Spaltenposition}neue Position f\"ur
eine Spalte setzen kann, aber auch diese Methode w\"are noch ziemlich
un\"ubersichtlich. Es gibt einen noch einfacheren Weg das obige Ergebnis
zu erreichen. Die gesetzten Tabulatorpositionen k\"onnen mit dem Befehl
\index{zzzcleartabs@{\tt\char92 cleartabs}}
`\verb|\cleartabs|' auch alle wieder gel\"oscht werden, um dann neue zu
setzen. Somit konnte das obige Programm einfach mit
\index{zzzvbox@{\tt\char92 vbox}}
\begin{verbatim}
$$\vbox{\+\bf if $n<r$ \cleartabs&\bf then $n=n+1$\cr
\+&\bf else & {\bf begin} ${\it print}$; $n:=0$;\cr
\+&&{\bf end};\cr
\+\cleartabs {\bf while} $p>0$ {\bf do}\cr
\+\quad&{\bf begin} $q:=link(p)$; $free(p)$; $p:=q$;\cr
\+&\{\bf end};\cr
\end{verbatim}
gesetzt werden.

Es d\"urfen keine `\verb|\+|' Zeilen geschachtelt werden. Sie k\"onnen
also sozusagen keine Tabellen innerhalb von Tabellen definieren. Da
diese Tabellen in \index{Box!Tabellen}
Boxen gleicher Breite aufgebaut werden, k\"onnen sie
mit der Befehlsfolge \index{zzzshowbox@{\tt\char92 showbox}}
`\verb|\showbox\tabs|' genau ausgeben lassen, was
bei der Konstruktion von Tabellen geschieht.

\section{Ausrichtungen}
Es gibt noch eine weitere M\"oglichkeit in \TeX\ Tabellen zu setzen. Sie
basiert auf der Idee einer \index{Schablone} Schablone. Mit der Eingabe
von
\index{zzzhalign@{\tt\char92 halign}}\index{zzzindent@{\tt\char92 indent}}
\begin{verbatim}
\halign{\indent#\hfil&\quad#\hfil\cr
Horizontale Liste&Kapitel 14\\
Vertikale Liste&Kapitel 15\\
Math. Liste&Kapitel 17\cr}
\end{verbatim}
erh\"alt man dasselbe Ergebnis, wie schon zuvor mit dem `\verb|\+|'
Befehl, n\"amlich:

\medskip\begin{tabular}{ll}
Horizontale Liste&Kapitel 14\\
Vertikale Liste&Kapitel 15\\
Math. Liste&Kapitel 17
\end{tabular}\medskip

Dieses Beispiel zeigt aber doch im wesentlichen, wie der
\index{zzzhalign@{\tt\char92 halign}}
`\verb|\halign|' Befehl funktioniert.\footnote{halign kommt von
`horizontal alignment', was soviel beeutet, wie
\index{Horizontal!Ausrichtung}`horizontale
Ausrichtung'.} Die erste Zeile, nicht im physikalischen Sinne, sondern
der Text bis zum ersten `\verb|\cr|', stellt die \index{Schablone}
Schablone f\"ur die
weiteren Aktivit\"aten dar. Dieses Beispiel enth\"alt zwei
\index{Muster!Eintrag}Mustereintr\"age
in der Schablone, n\"amlich

\medskip{\tt\char92 indent\char35\char92 hfil} und {\tt\char92
quad\char35\char92 hfil}\medskip

Jeder dieser \index{Muster!Eintrag}Mustereintr\"age enth\"alt genau ein `\verb|#|' Zeichen, an
der sp\"ater der Text der Spalte eingef\"ugt wird. In der ersten Zeile
wird so der Eintrag
\index{zzzhfil@{\tt\char92 hfil}}\index{zzzindent@{\tt\char92 indent}}
\begin{verbatim}
\indent Horizontale Liste \hfil
\end{verbatim}
eingetragen. Nun zur Frage, was denn das `\verb|\hfil|' in dem Eintrag
soll. Nun. Im \index{zzzhalign@{\tt\char92 halign}}
`\verb|\halign|' Befehl werden zun\"achst {\em alle}
Spalten eingelesen, und \TeX\ setzt dann die Breite der Spalten nach
dem breitesten vorkommenden Eintrag. Da ein aktueller Eintrag ggf.\
schmaler ist, als der breiteste, mu\ss{} mit einem `\verb|\hfil|' der
restliche Platz aufgef\"ullt werden, um keine \index{Fehler!Meldung}
Fehlermeldung zu erhalten.

Es gibt einige Unterschiede zwischen dem
\index{zzzhalign@{\tt\char92 halign}}`\verb|\halign|' Befehl und
dem `\verb|\+|'~Befehl.
\index{Eingabe!Tabellen}\index{Horizontal!Ausrichtung}
\begin{enumerate}
\item Beim dem Befehl zur horizontalen Ausrichtung wird die Breite der
Spalten nach dem breitesten Eintrag automatisch berechnet.
\item Jeder `\verb|\halign|' Befehl berechnet seine Spaltenbreiten f\"ur
sich. Wenn sie in zwei verschiedenen Tabellen die selbe Ausrichtung
haben wollen, dann m\"ussen sie sich schon einiger Tricks bedienen.
\item Gro\ss{}e Tabellen lassen sich schlecht mit `\verb|\halign|'
erstellen, da die ganze Tabelle im \index{Speicher!Tabelle}
Speicher gehalten werden mu\ss{} und es
deshalb zu Problemen kommen kann.
\item Der `\verb|\halign|' Befehl ist wesentlich schneller, als der
`\verb|\+|'~Befehl, da es sich um eine eingebaute Funktion von \TeX\
und nicht um ein Makro handelt.
\item Man kann sich die Eingabe vereinfachen. Statt der obigen Eingabe
w\"are z.B.

\medskip
{\tt\char92 halign\char123\char92 indent\char35\ Liste\char92
hfil\char38\char92 quad Kapitel \char35\char92 cr}\\{\tt
Horizontale\char38 14\char92 cr Vertikale\char38 15\char92 cr
Math.\char38 17\char92 cr\char125}\medskip

auch m\"oglich gewesen. Man sollte sich allerdings nicht {\em zu} viel
M\"uhe mit der \index{Optimierung}Optimierung geben (man h\"atte die `1' der
\index{Kapitel!Nummer}Kapitelnummer
noch mit in die \index{Schablone}
Schablone \"ubernehmen k\"onnen), da die Zeit, die man
daf\"ur aufwenden mu\ss{} in keinem Verh\"altnis zum Erfolg steht.
\item Wenn die Spaltenbreiten h\"aufig wechseln, so wie in dem
Programmbeispiel, sollte man lieber mit dem `\verb|\+|' Befehl
arbeiten.
\end{enumerate}
\subsection{Beispiel zur Anwendung}
Der wichtigste Punkt bei der Erstellung von Tabellen mit dem
\index{zzzhalign@{\tt\char92 halign}}
`\verb|\halign|' Befehl besteht in der g\"unstigen Konstruktion der
\index{Schablone}
Schablone. Hier sollte man ruhig etwas Zeit investieren, die sich
sp\"ater erst auszahlt. Wenn sie z.B.\ die folgende Tabelle setzen
wollen

\medskip\begin{tabular}{rccl}
{\em Name} & {\em Geschlecht} & {\em Alter} & {\em Schul-} \\
 & & & {\em bildung} \\[1ex]
\bf Karl & \em m\"annlich & 20 & Hauptschule\\
\bf Tina & \em weiblich & 22 & Abitur\\
\bf Otto & \em m\"annlich & 21 & Mittl. Reife\\
\bf Jupp & \em m\"annlich & 20 & Abitur
\end{tabular}\medskip

Dann w\"unscht man sich nat\"urlich eine einfache \index{Eingabe!Tabellen}
Eingabe. Die zweite Zeile sollte am besten mit
\begin{verbatim}
Tina & weiblich & 22 & Abitur\cr
\end{verbatim}
eingegeben werden k\"onnen. Die oberste Zeile mu\ss{} nat\"urlich gesondert
behandelt werden, aber es soll m\"oglich sein, da\ss{} die rechte
Ausrichtung der ersten Spalte\index{Spalte!Tabelle!Ausrichtung},
die linke der letzten und die
\index{Zentrierung}
Zentrierung der mittleren Spalten automatisch vorgenommen wird. Ebenso
w\"are es w\"unschenswert, wenn die \index{Schrift!Umschaltung}
Schriftumschaltung in der ersten und
der zweiten Spalte automatisch erfolgen w\"urde. Dies ist tats\"achlich
m\"oglich mit folgender \index{Schablone}Schablone
\index{zzzhalign@{\tt\char92 halign}}\index{zzhfil@{\tt\char92 hfil}}
\begin{verbatim}
\halign{\hfil\bf#&\quad\hfil\it#\hfil&\quad\hfil#\hfil&\quad#\hfil\cr
\end{verbatim}
Jetzt brauchen sie nur noch die Zeilen, entsprechend der `Tina'-Zeile
einzugeben, und sie erhalten das gew\"unschte Ergebnis. In der
Titelzeile ist allerdings bei jedem Eintrag die Angabe der
\index{Schrift!Form}Schriftform
notwendig, um die Vorgaben zu \"uberschreiben.

Die Tabelle kann \"ubrigens mit dem Befehl \index{zzzopenup@{\tt\char92 openup}}
`\verb|\openup|' etwas
auseinandergezogen werden. Mit der Angabe `\verb|\openup2pt|' entsteht

\medskip\begin{tabular}{rccl}
{\em Name} & {\em Geschlecht} & {\em Alter} & {\em Schul-} \\[2pt]
 & & & {\em bildung} \\[1ex]
\bf Karl & \em m\"annlich & 20 & Hauptschule\\[2pt]
\bf Tina & \em weiblich & 22 & Abitur\\[2pt]
\bf Otto & \em m\"annlich & 21 & Mittl. Reife\\[2pt]
\bf Jupp & \em m\"annlich & 20 & Abitur
\end{tabular}\medskip

Der zus\"atzliche \index{Zwischenraum}
Zwischenraum zwischen der Titelzeile und dem Rest der
Tabelle wurde mit der Angabe von
\index{zzznoalign@{\tt\char92 noalign}}
\index{zzzsmallskip@{\tt\char92 smallskip}}
\begin{verbatim}
\noalign{\smallskip}
\end{verbatim}
erreicht. Grunds\"atzlich ist es immer m\"oglich mit diesem Befehl
vertikale Elemente in eine Tabelle einzuf\"ugen, die nicht zu der
Tabelle selber geh\"oren. Von dieser M\"oglichkeit kann man Gebrauch
machen, wenn man etwas zwischen zwei Zeilen schreiben m\"ochte. Mit der
Eingabe von
\index{zzzhalign@{\tt\char92 halign}}\index{zzzhfil@{\tt\char92 hfil}}
\index{zzznoalign@{\tt\char92 noalign}}
\begin{verbatim}
\halign{#\hfil&\quad#\hfil\cr
Erste & Spalte \cr\noalign{Dazwischen}
Zweite & Spalte\cr}
\end{verbatim}
erhalten sie z.B.

\medskip\begin{tabular}{ll}
\qquad Erste & Zeile \\
Dazwischen\\
\qquad Zweite & Zeile
\end{tabular}\medskip

\subsection{Weitere Befehle}
Es ist auch m\"oglich, da\ss{} sie \TeX\ entscheiden lassen, wieviel Platz
zwischen die \index{Spalte!Tabelle}
Tabellenspalten gesetzt werden soll. Sie k\"onnen also auf
den \index{zzzquad@{\tt\char92 quad}}
`\verb|\quad|' Befehl zwischen den Spalten verzichten. Statt
dessen geben sie an, wieviel {\tt tabskipglue} zwischen den Spalten
verwendet werden soll. Wenn sie also statt der obigen
\index{Schablone}Schablone\index{zzzhalign@{\tt\char92 halign}}
\index{zzzhfil@{\tt\char92 hfil}}
\index{zzztabskip@{\tt\char92 tabskip}}
\begin{verbatim}
\tabskip=1em plus 2em minus.5em
\halign{\hfil\bf#&\hfil\it#\hfil&\hfil#\hfil&#\hfil\cr
\end{verbatim}
schreiben, dann erhalten sie trotzdem die Tabelle in der Form, wie sie
sie schon gesehen haben, als

\medskip\begin{tabular}{rccl}
{\em Name} & {\em Geschlecht} & {\em Alter} & {\em Schul-} \\[2pt]
 & & & {\em bildung} \\[1ex]
\bf Karl & \em m\"annlich & 20 & Hauptschule\\[2pt]
\bf Tina & \em weiblich & 22 & Abitur\\[2pt]
\bf Otto & \em m\"annlich & 21 & Mittl. Reife\\[2pt]
\bf Jupp & \em m\"annlich & 20 & Abitur
\end{tabular}\medskip

Der \index{Leim}
Tabskip-Leim wird auch vor sie erste und hinter die letzte Spalte
gesetzt, so da\ss{} sie mit
\index{zzzhalign@{\tt\char92 halign}}
\begin{verbatim}
\halign to...        \halign spread
\end{verbatim}
genauso verfahren k\"onnen, wie bei den entsprechenden Boxbefehlen.

Die Befehlsfolge
\index{zzzhsize@{\tt\char92 hsize}}`\verb|\halign| \verb|to| \verb|\hsize|' wird
allerdings nur dann richtig verstanden, wenn der Tabskip-Leim auch die
n\"otige F\"ahigkeit zur Streckung besitzt. Falls dies nicht der Fall ist
wird eine \index{Fehler!Meldung}Fehlermeldung auf dem \index{Bildschirm}
Bildschirm und im \index{File!Log}Logfile ausgegeben.
Das geschieht auch, wenn ein Eintrag so gro\ss{} oder klein ist, da\ss{} die
angegebene \index{Groesse!Leim@Gr\"o\ss{}e!Leim}
Gr\"o\ss{}e des Leims nicht ausreicht um eine passende Spalte
daraus zu machen.

Die Werte f\"ur den Tabskip-Leim k\"onnen auch in der \index{Schablone}
Schablone ge\"andert
werden. Es m\"ussen also nicht f\"ur alle Spalten die gleichen Werte
verwendet werden. Damit k\"onnen sie einzelne Spalten von den anderen
absetzen.

\TeX\ \"ubernimmt die \index{Schablone}
Schablone $1:1$, d.h.\ da\ss{} nichts in der Schablone
expandiert wird. Die einzigen Befehle, die \TeX\ innerhalb der
Schablone erkennt sind
\index{zzzspan@{\tt\char92 span}}
\index{zzztabskip@{\tt\char92 tabskip}}
\begin{verbatim}
&  #  \cr  \span  \tabskip
\end{verbatim}
Der Befehl `\verb|\span|' sorgt daf\"ur, da\ss{} das n\"achste Token in der
Schablone {\em doch} expandiert wird.\footnote{{\em Innerhalb} der
Tabelle hat der Befehl eine andere Wirkung, die sp\"ater erkl\"art wird.}
Dies f\"uhrt zu einem besonderen Verhalten aller `\verb|\if|' Befehle
innerhalb von Tabellen. Verwenden sie diese Befehle besser nicht
innerhalb von Tabellen, wenn sie Probleme vermeiden wollen.

Es kann vorkommen, da\ss{} alle Spalten, bis auf eine oder zwei eine
\"ahnliche Form haben, so da\ss{} sich entsprechende \index{Muster!Fuer
Schablone@Muster!F\"ur Schablone}Muster in der \index{Schablone}Schablone
empfehlen. F\"ur die Eintr\"age, die dieser Form nicht entsprechen, gibt
es den Befehl \index{zzzomit@{\tt\char92 omit}}
`\verb|\omit|'. Wenn er als erstes im Eintrag steht,
wird das Muster aus der Schablone durch ein einfaches `\verb|#|'
ersetzt.

\subsection{Besondere Tabellen}
\subsubsection{Dezimalpunkte}
Wie geht man vor, wenn man eine Tabelle schreiben will, die z.B.\
Geldwerte enthalten soll, bei denen nat\"urlich der Dezimalpunkt
\"ubereinander stehen soll. Man kann nat\"urlich aus jeder Zahl drei
\index{Spalte!Zahlen}
Spalten machen, und die \index{Zwischenraum}
Zwischenr\"aume so setzen, da\ss{} die mittlere
Spalte den Dezimalpunkt enth\"alt und trotzdem noch gut aussieht. Es
gibt aber auch eine andere M\"oglichkeit. Der Trick besteht darin, da\ss{}
alle \index{Ziffer}
Ziffernzeichen dieselbe Breite haben. Man kann nun ein Zeichen,
das sonst nicht gebraucht wird zu einem \index{Aktiv!Zeichen}
aktiven Zeichen machen und
durch dieses Zeichen einen \index{Leer!Raum}
Leerraum ausgeben lassen, der genau der
Breite eines \index{Ziffer}
Ziffernzeichens entspricht. Wenn man mit der Befehlsfolge
\index{zzzactive@{\tt\char92 active}}\index{zzzcatcode@{\tt\char92 catcode}}
\index{zzzdef@{\tt\char92 def}}\index{zzzdigitwidth@{\tt\char92 digitwidth}}
\index{zzzkern@{\tt\char92 kern}}\index{zzznewdimen@{\tt\char92 newdimen}}
\index{zzzsetbox@{\tt\char92 setbox}}
\begin{verbatim}
\newdimen\digitwidth
\setbox0=\hbox{\rm0}
\digitwidth=\wd0
\catcode`?=\active
\def?{\kern\digitwidth}
\end{verbatim}
Das Zeichen `\verb|?|' zu einem aktiven Zeichen gemacht hat, dann kann
man die Zahlen $123.45$ und $6.789$ als `\verb|123.45?|' und
`\verb|??6.789|' in dem Eintrag eingeben, und sie werden beide genau
mit dem Dezimalpunkt \"ubereinander gesetzt.

\subsubsection{Lange Tabellen}
Was ist mit Tabellen, wie dieser

\index{zzzcal@{\tt\char92 cal}}
\medskip\begin{tabular}{r@{$=$}*{15}{c}}
$n$ & 0 & 1 & 2 & 3 & 4 & 5 & 6 & 7 & 8 & 9 & 10 & 11 & 12 & 13 &
$\ldots$ \\
${\cal G}(n)$ & 1 & 2 & 4 & 3 & 6 & 7 & 8 & 16 & 18 & 25 & 32 & 11 &
64 & 31 & $\ldots$
\end{tabular}\medskip

Mit Sicherheit w\"are es h\"ochst m\"uhsam f\"ur diese Tabelle eine
\index{Schablone}Schablone
zu schreiben, insbesondere, da die Tabelle nur zwei Zeilen hat. Es
gibt aber einen einfacheren Weg. Wenn sie an einer Stelle der
Schablone ein \index{Kaufmannsund}\index{Doppelt!Kaufmannsund}
doppeltes `\verb|&|' schreiben, dann setzt \TeX\ die
folgenden Eintr\"age bis zum `\verb|\cr|' beliebig oft fort.
\index{zzzcdots@{\tt\char92 cdots}}

\medskip
$t_1\,${\tt\char38}$\,t_2\,${\tt\char38}$\,t_3\,${\tt\char38\char38}
$\,t_4\,${\tt\char38}$\,t_5\,${\tt\char92 cr}
\ wird verstanden als \
$t_1\,${\tt\char38}$\,t_2\,${\tt\char38}$\,t_3\,${\tt\char38}$\,t_4\,$
{\tt\char38}$\,t_5\,${\tt\char38}$\,t_4\,${\tt\char38}$\,t_5\,$
{\tt\char38}$\,t_4\,${\tt\char38}$\,\,\cdots\,$

und

{\tt\char38}$t_1\,${\tt\char38}$\,t_2\,${\tt\char38}$\,t_3\,${\tt\char38}
$\,t_4\,${\tt\char38}$\,t_5\,${\tt\char92 cr}
\ wird verstanden als \
$t_1\,${\tt\char38}$\,t_2\,${\tt\char38}$\,t_3\,${\tt\char38}$\,t_4\,$
{\tt\char38}$\,t_5\,${\tt\char38}$\,t_1\,${\tt\char38}$\,t_2\,$
{\tt\char38}$\,t_3\,${\tt\char38}$\,\,\cdots\,.$\medskip

Der evtl.\ gesetzte Tabskip-Leim wird in die Kopie \"ubernommen. Somit
l\"a\ss{}t sich die \index{Schablone}Schablone der obigen Tabelle als
\index{zzzhfil@{\tt\char92 hfil}}
\begin{verbatim}
$\hfil#$ =&&\ \hfil#\hfil\cr
\end{verbatim}
schreiben.

\subsection{Breite Eintr\"age}
Mitunter ist es n\"otig, da\ss{} sich ein Eintrag einer Tabelle \"uber mehr
als eine Spalte erstreckt. F\"ur diesen Fall bieten sich zwei
M\"oglichkeiten an.
\subsubsection{Verstecken der Breite}
In {\em Plain} \TeX\ wird der Befehl
\index{zzzhidewidth@{\tt\char92 hidewidth}}`\verb|\hidewidth|' als
\index{zzzhskip@{\tt\char92 hskip}}
\begin{verbatim}
\hskip-1000pt plus 1fill
\end{verbatim}
definiert. Er hat also eine extreme negative, nat\"urliche Breite, kann
aber beliebig gestreckt werden. Wenn sie diesen Befehl rechts hinter
einen Spalteneintrag schreiben, erscheint es f\"ur \TeX, als habe er
keine Breite, und der Eintrag ragt rechts \"uber seine
\index{Box!Tabellen}Box hinaus.
Analog arbeitet der Befehl, wenn er links vor einen Eintrag
geschrieben wird. Sp\"ater wird noch eine n\"utzliche Eigenschaft dieses
Befehls gezeigt, wenn er rechts {\em und} links vom Eintrag steht.
\subsubsection{Eintr\"age \"uber mehrere Spalten}
Statt der obigen Methode k\"onnen sie auch den Befehl
\index{zzzspan@{\tt\char92 span}}
`\verb|\span|' anstelle des `\verb|&|' verwenden.\footnote{Hier hat
dieser Befehl eine {\em ganz} andere Wirkung, als in der
\index{Schablone}Schablone!}
\TeX\ behandelt die Eintr\"age vor und hinter diesem Befehl genauso, als
wenn dort ein `\verb|&|' st\"unde, schreibt sie aber anschlie\ss{}end in
eine Box, die sich \"uber die Breite zweier Eintr\"age erstreckt.

Sie k\"onnen den Befehl zusammen mit \index{zzzomit@{\tt\char92 omit}}
`\verb|\omit|' verwenden, um
eventuell auftretende automatische Eintr\"age zu unterbinden.
Schlie\ss{}lich gibt es noch den Befehl
\index{zzzmultispan@{\tt\char92 multispan}}`\verb|\multispan|', dem eine
Nummer folgen mu\ss{}, die angibt, wieviele \index{Spalte!Zusammenfassen}
Spalten zusammengefa\ss{}t werden
sollen. Sollen mehr als $9$ Spalten zusammengefa\ss{}t werden, mu\ss{} die
Zahl in geschweiften \index{Klammer!Omit}Klammern angegeben werden.

Sie k\"onnen sich die Wirkung der Befehle noch einmal am folgenden
Beispiel verdeutlichen.

\medskip\begin{tabular}{ccc}
Eins & Zwei & Drei \\
\multicolumn{2}{c}{Eins-Zwei} & Drei\\
Eins &\multicolumn{2}{c}{Zwei-Drei}\\
\multicolumn{3}{c}{Eins-Zwei-Drei}
\end{tabular}\medskip

wurde mit
\index{zzzhalign@{\tt\char92 halign}}\index{zzzhfil@{\tt\char92 hfil}}
\index{zzzmultispan@{\tt\char92 multispan}}
\index{zzzspan@{\tt\char92 span}}
\index{zzztabskip@{\tt\char92 tabskip}}
\index{zzzvbox@{\tt\char92 vbox}}
\begin{verbatim}
$$\tabskip=3em
\vbox{\halign{&\hfil#\hfil\cr
Eins&Zwei&Drei\cr
Eins-\span Zwei&Drei\cr
Eins&Zwei-\span Drei\cr
\multispan3 Eins-Zwei-Drei\cr}}$$
\end{verbatim}
erzeugt.
\subsection{Gerahmte Tabellen}
\index{Knuth. D.}
D.~Knuth spricht davon, da\ss{} die Erstellung von gerahmten Tabellen mit
zum Schwierigsten geh\"ort, was in \TeX\ realisiert werden kann.
Tats\"achlich ergeben sich {\em einige} Schwierigkeiten, wenn man sich
genauer mit diesem Thema besch\"aftigt.\footnote{Die Schwierigkeiten
sind so gro\ss{}, da\ss{} ich jedem nur empfehlen kann, nur im \"aussersten
Notfall {\em nicht} die Tabular-Umgebung von \LaTeX\ zu benutzen.}

Zun\"achst m\"ussen sie den \index{Zwischenraum}
Zwischenraum zwischen den Zeilen ausschalten,
und den n\"otigen \index{Abstand!Zeilen}
Zeilenabstand durch `Struts' in der \index{Schablone}Schablone
aufbauen, da sonst die senkrechten \index{Linie!Senkrecht}
Linien nicht aneinander sto\ss{}en
w\"urden. Dies geschieht mit dem Befehl
\index{zzzoffinterlineskip@{\tt\char92 offinterlineskip}}
`\verb|\offinterlineskip|'. Die
H\"ohe der senkrechte Linie braucht nicht spezifiziert zu werden, da
\TeX\ die Boxh\"ohe vorgibt und sich der unspezifizierte Befehl an die
umgebende Box anpa\ss{}t. Allerdings mu\ss{} jeder senkrechte Strich in eine
eigene Spalte geschrieben werden. Im \TeX Book findet sich folgendes
Beispiel
\index{zzzhalign@{\tt\char92 halign}}
\index{zzzhfil@{\tt\char92 hfil}}
\index{zzzhrule@{\tt\char92 hrule}}
\index{zzzmultispan@{\tt\char92 multispan}}
\index{zzzoffinterlineskip@{\tt\char92 offinterlineskip}}
\index{zzzstrut@{\tt\char92 strut}}
\index{zzzvbox@{\tt\char92 vbox}}
\index{zzzvrule@{\tt\char92 vrule}}
\begin{verbatim}
\vbox{\offinterlineskip
\hrule
\halign{&\vrule#&
  \strut\quad\hfil#\quad\cr
\multispan5\hrulefill\cr
height2pt&\omit&&\omit&\cr
&Year\hfil&&World Population&\cr
height2pt&\omit&&\omit&\cr
\multispan5\hrulefill\cr
height2pt&\omit&&\omit&\cr
&8000\BC&&5,000,000&\cr
&50\AD&&200,000,000&\cr
&1650\AD&&500,000,000&\cr
&1850\AD&&1,000,000,000&\cr
&1945\AD&&2,300,000,000&\cr
&1980\AD&&4,400,000,000&\cr
height2pt&\omit&&\omit&\cr}
\hrule}
\end{verbatim}
An diesem Beispiel k\"onnen sie sehen, wie vergleichsweise k\"ummerlich
die Befehle f\"ur gerahmte Tabellen in {\em Plain} \TeX\ tats\"achlich ist, denn
das Ergebnis ist gerade mal

\medskip\begin{tabular}{|r|r|}\hline
Year\hfil & World Population \\ \hline
8000 B.C. & 5,000,000\\
50 A.D.   & 200,000,000\\
1650 A.D. & 500,000,000\\
1850 A.D. & 1,000,000,000\\
1945 A.D. & 2,300,000,000\\
1980 A.D. & 4,400,000,000\\ \hline
\end{tabular}\medskip

Etwas verwirrend k\"onnten noch die Zeilen
\index{zzzomit@{\tt\char92 omit}}
\begin{verbatim}
height2pt&\omit&&\omit&\cr
\end{verbatim}
sein. Sie bewirken, da\ss{} auch der Anschlu\ss{} zu den waagerechten Zeilen
geschafft wird.
\subsection{Hilfen bei Tabellen}
Wenn bei der Konstruktion einer Tabelle etwas schief geht, empfiehlt
sich folgendes Vorgehen. F\"ugen sie an einer Stelle, an der sie den
\index{Fehler!Behebung}
Fehler vermuten eine unbekannte \index{Kontrollsequenz}
Kontrollsequenz ein, ich benutze f\"ur
diese Zwecke immer `\verb|\fritz|'. Dann k\"onnen sie mit dem Befehl
\index{zzzshowlists@{\tt\char92 showlists}}
`\verb|\showlists|' eine genauer Information dar\"uber bekommen, was
\TeX\ an dieser Stelle gerade macht.

Eine weitere Hilfe interessiert wohl mehr die \index{Makro!Design}
Makrodesigner. Jede
Zeile eines Alignments {\em mu\ss{}} mit dem Befehl `\verb|\cr|'
abgeschlossen werden. Sie k\"onnen durch das Primitv
\index{zzzcrcr@{\tt\char92 crcr}}`\verb|\crcr|' dem
Fall vorbeugen, da\ss{} ein Anwender diesen `\verb|\cr|' Befehl vergi\ss{}t.
Die Wirkung dieses Befehls ist folgende. Steht er unmittelbar hinter
einem `\verb|\cr|', dann wirkt er gar nicht, steht er woanders, dann
wirkt er wie ein `\verb|\cr|' Befehl.

Wenn ihnen selber die andauernde Tipperei der `\verb|\cr|' Befehle
zuviel wird, dann gibt es noch eine M\"oglichkeit. Schlie\ss{}en sie ihre
Tabelle in eine eigene Gruppe ein und stellen sie dann innerhalb der
Gruppe der Tabelle die Befehle
\index{zzzlet@{\tt\char92 let}}\index{zzzobeylines@{\tt\char92 obeylines}}
\begin{verbatim}
\let\par=\cr
\obeylines
\end{verbatim}
voran. Durch den `\verb|\obeylines|' Befehl wird jede Zeile wie ein
eigener \index{Absatz}
Absatz behandelt. Das automatisch eingef\"ugte
\index{zzzpar@{\tt\char92 par}}`\verb|\par|'
wird als `\verb|\cr|' interpretiert. Diese Vorgehensweise empfiehlt
sich aber nur bei l\"angeren Tabellen, die sie aber sowieso mit
`\verb|\+|' ausf\"uhren sollten. Dort funktioniert dieser Trick aber
leider nicht.

\chapter{Ausgaberoutinen}\index{Ausgabe!Routine}
In einem vorhergehenden Kapitel haben sie schon gesehen, wie \TeX\ aus
der langen Reihe von Token zun\"achst Paragraphen und dann Seiten macht.
Diese Seiten bestehen zun\"achst allerdings nur aus dem eigentlichen
\index{Seite!Text}Seitentext und den \index{Einfuegung@Einf\"ugung}
Einf\"ugungen, z.B.\ \index{Fussnote@Fu\ss{}note}Fu\ss{}noten. Was noch fehlt sind
die \index{Kopf!Zeile}Kopf- und \index{Fusszeile@Fu\ss{}note}
Fu\ss{}zeilen.\footnote{Man beachte den Unterschied zwischen
Fu\ss{}noten und Fu\ss{}zeilen. In letzteren stehen \"ublicherweise die
\index{Seite!Nummer}
Seitennummern o.\"a.} Diese Erg\"anzungen werden von einer sogenannten
Outputroutine noch zu der Seite hinzugef\"ugt.
\section{Die Outputroutine von {\em Plain} \TeX}
Normalerweise ist eine \index{Seite!Masse@Seite!Ma\ss{}e}
Seite von {\em Plain} \TeX\ $8\frac{1}{2}$ Inch breit und 11
Inch hoch, dazu ein \index{Rand!Bei Plain@Rand!Bei {\it Plain}}
Rand von einem Inch an allen vier Seiten. Diese
Werte k\"onnen nat\"urlich ver\"andert werden. Dazu dienen die Befehle
\index{zzzhsize@{\tt\char92 hsize}}
`\verb|\hsize|' und \index{zzzvsize@{\tt\char92 vsize}}
`\verb|\vsize|'. {\em Plain} \TeX\ setzt
`\verb|\hsize=6.5in|' und erh\"alt damit eine \index{Seite!Breite}
Seitenbreite von
$8\frac{1}{2}$ Inch (denken sie an die beiden \index{Raender@R\"ander}R\"ander von einem Inch
und `\verb|\vsize=8.9in|'. Nicht, wie erwartet `\verb|9in|', da {\em Plain}
\TeX\ keinen Platz f\"ur die \index{Seite!Nummer}Seitennummer reserviert.

Sie sollten diese beiden Werte nicht zu oft \"andern. Am besten nur ganz
zu Beginn eines Textes, oder wenn sie sicher sind, da\ss{} \TeX\ alle
Seiten ausgegeben hat.

Sie k\"onnen dar\"uber hinaus auch noch die gesamte Seite auf dem Papier
verschieben und zwar mit den Befehlen \index{zzzhoffset@{\tt\char92 hoffset}}
`\verb|\hoffset|' und\index{zzzvoffset@{\tt\char92 voffset}}
`\verb|\voffset|'. Die Angabe
\begin{verbatim}
\hoffset=.5in \voffset=-1in
\end{verbatim}
verschiebt ihre Seite um ein halbes Inch nach rechts und ein Inch nach
oben. Sie sollten aber auf die physikalischen Grenzen ihres Mediums,
meist Papiers, achten, da \TeX\ daf\"ur keine Kontrolle bereitstellt.

Die einfachste Art die Outputroutine von {\em Plain} \TeX\ zu \"andern besteht
in der Angabe von
\index{zzznopagenumbers@{\tt\char92 nopagenumbers}}
\begin{verbatim}
\nopagenumbers
\end{verbatim}
Mit diesem Befehl verhindern sie die \index{Ausgabe!Seitennummern}
Ausgabe von \index{Seite!Nummer}Seitennummern, was
vor allem bei kurzen Texten recht hilfreich sein d\"urfte. Dies ist aber
nur ein Spezialfall. Generell k\"onnen die \index{Kopf!Zeile}Kopf- und die
\index{Fusszeile@Fu\ss{}note}Fu\ss{}zeile einer
Seite mit den Befehlen
\index{zzzfootline@{\tt\char92 footline}}\index{zzzheadline@{\tt\char92 headline}}
\begin{verbatim}
\headline={<Kopfzeile>}
\footline={<Fusszeile>}
\end{verbatim}
gesetzt werden. Mit \index{zzzhrulefill@{\tt\char92 hrulefill}}
`\verb|\headline={\hrulefill}|' w\"urde z.B.\ vor
jede Seite ein waagerechter Strich gezogen. Die generelle Idee ist es,
da\ss{} die Outputroutine von {\em Plain} \TeX\ vor die Seite ein
\index{zzzline@{\tt\char92 line}}
\index{zzzthe@{\tt\char92 the}}
\begin{verbatim}
\line{\the\headline}
\end{verbatim}
und hinter die Seite ein
\index{zzzfootline@{\tt\char92 footline}}
\begin{verbatim}
\line{\the\footline}
\end{verbatim}
schreibt, wobei die \index{Kopf!Zeile}Kopf- bzw.\ \index{Fusszeile@Fu\ss{}note}
Fu\ss{}zeile etwas von der eigentlichen
Seite abgesetzt wird. Das Makro \index{zzznopagenumbers@{\tt\char92 nopagenumbers}}
`\verb|\nopagenumbers|' ist also
einfach eine \index{Abkuerzung@Abk\"urzung}Abk\"urzung f\"ur
\index{zzzfootline@{\tt\char92 footline}}\index{zzzhfil@{\tt\char92 hfil}}
\begin{verbatim}
\footline={\hfil}
\end{verbatim}
Normalerweise ist
\index{zzzfolio@{\tt\char92 folio}}\index{zzzhss@{\tt\char92 hss}}
`\verb|\footline={\hss\tenrm\folio\hss}|' definiert,
wobei `\verb|\folio|' die \index{Seite!Nummer}
Seitennummer ausgibt. Die Umschaltung auf
die \index{Schrift!Umschaltung}
Schrift ist notwendig, da sonst immer die aktuelle Schrift
verwendet w\"urde, was nicht immer w\"unschenswert ist. Das Makro leistet
aber noch mehr. Wenn die \index{Seite!Zahl}
Seitenzahl negativ ist, wird sie in r\"omischen
\index{Ziffer}
Ziffern ausgegeben, ist sie positiv, dann wird sie in normalen
\index{Lateinisch!Ziffern}
lateinischen Ziffern ausgegeben. Die Definition lautet:
\index{zzzelse@{\tt\char92 else}}\index{zzzfi@{\tt\char92 fi}}
\index{zzzifnum@{\tt\char92 ifnum}}\index{zzznumber@{\tt\char92 number}}
\index{zzzpageno@{\tt\char92 pageno}}
\index{zzzromannumeral@{\tt\char92 romannumeral}}
\begin{verbatim}
\ifnum\pageno<0 \romannumeral-\pageno \else\number\pageno \fi
\end{verbatim}

Hier nun mal ein Beispiel einer \index{Seite!Nummer}
Seitennummerierung aus dem \TeX book,
bei dem die Seitennummern in der \index{Kopf!Zeile}Kopfzeile ausgegeben werden, und zwar
bei ungeraden und geraden \index{Seite!Rechts-Links}
Seiten unterschiedlich.
\index{zzzbaselineskip@{\tt\char92 baselineskip}}
\index{zzzdef@{\tt\char92 def}}
\index{zzzelse@{\tt\char92 else}}
\index{zzzfi@{\tt\char92 fi}}
\index{zzzfolio@{\tt\char92 folio}}
\index{zzzheadline@{\tt\char92 headline}}
\index{zzzhfil@{\tt\char92 hfil}}
\index{zzzifodd@{\tt\char92 ifodd}}
\index{zzznopagenumbers@{\tt\char92 nopagenumbers}}
\index{zzzpageno@{\tt\char92 pageno}}
\index{zzzvoffset@{\tt\char92 voffset}}
\begin{verbatim}
\nopagenumbers % suppress footlines
\headline={\ifodd\pageno\rightheadline \else\leftheadline\fi}
\def\rightheadline{\tenrm\hfil RIGHT RUNNING HEAD\hfil\folio}
\def\leftheadline{\tenrm\folio\hfil LEFT RUNNING HEAD\hfil}
\voffset=2\baselineskip
\end{verbatim}\index{Baseline!Skip}
Mit ``Running Head''\footnote{Laufender \index{Kopf!Laufender}
Kopf.} werden in englischen
B\"uchern Kopfzeilen bezeichnet, die auf mehreren Seiten erscheinen. Die
Angabe von `\verb|\voffset|' ist n\"otig, damit auch weiterhin ein
\index{Abstand!Kopfzeile}
Abstand zur eigentlichen Seiten bestehen bleibt.

Wenn sie \index{zzzvsize@{\tt\char92 vsize}}
`\verb|\vsize|' nicht \"andern, dann wird sich auch die
Position der \index{Kopf!Zeile}Kopf- bzw.\ \index{Fusszeile@Fu\ss{}note}
Fu\ss{}zeilen nicht \"andern. Wenn sie z.B.\ den
Befehl
\index{zzzraggedbottom@{\tt\char92 raggedbottom}}
`\verb|\raggedbottom|' verwenden, so da\ss{} nicht mehr alle Seiten
die gleiche \index{Hoehe!Seite@H\"ohe!Seite}
H\"ohe haben, dann \"andert sich dennoch die Position der Fu\ss{}zeile nicht!
\section{Die Interna der Outputroutinen}
Beim Aufruf einer \index{Ausgabe!Routine}
Ausgaberoutine ist in der \index{Box!255}Box255 immer die aktuelle,
zur Ausgabe bereite Seite gespeichert. Der Befehl
\index{zzzoutput@{\tt\char92 output}}`\verb|\output|'
wird wie ein normaler Befehl, so wie
\index{zzzeverypar@{\tt\char92 everypar}}`\verb|\everypar|' o.\"a.\
bearbeitet, mit der Ausnahme, da\ss{} immer eine Gruppierungsklammer um
den Befehl geschrieben wird. Dies dient der Vermeidung von
Seiteneffekten. Die Ausgaberoutinen \"andern z.B.\ oft den Baselineskip,
wobei die \"Anderung nat\"urlich nicht in den weiteren Text flie\ss{}en soll.

Wenn keine spezielle Outputroutine installiert wird
\index{zzzoutput@{\tt\char92 output}}
(`\verb|\output={...}|'), dann verwendet \TeX\ seine eigene, minimale
Ausgaberoutine
\index{zzzshipout@{\tt\char92 shipout}}
\begin{verbatim}
\output={\shipout\box255}
\end{verbatim}
Der Befehl `\verb|\shipout|' schreibt das \index{Argument!Bei Shipout}
Argument in den derzeit
g\"ultigen \index{File!DVI}Dvi-File, zeigt den Inhalt der Zahlregister 0 bis 9 an und
f\"uhrt die anstehenden
\index{zzzwrite@{\tt\char92 write}}
`\verb|\write|'-Befehle aus. Genauere
Informationen k\"onnen auch hier wieder mit dem Befehl
\index{zzztracingoutput@{\tt\char92 tracingoutput}}
`\verb|\tracingoutput|' erhalten werden, wenn der Wert dieser
Variablen gr\"o\ss{}er als Null ist. Achten sie darauf, da\ss{} alle Makros
w\"ahrend eines Shipout genau definiert sind, da sie zu der dann
g\"ultigen Version expandiert werden.

Neben dem 255ten \index{Box!255}Boxregister k\"onnen auch noch andere Registerwerte
Einflu\ss{} auf die Ausgaberoutine haben und von dieser ausgewertet
werden. \TeX\ verwendet z.B. die Werte
\index{zzzinsertpenalties@{\tt\char92 insertpenalties}}
\index{zzzoutput@{\tt\char92 output}}
\index{zzzoutputpenalty@{\tt\char92 outputpenalty}}
\begin{verbatim}
\insertpenalties
\outputpenalty
\end{verbatim}
in folgender Weise. Der erste Wert enth\"alt die Anzahl der
\index{Einfuegung!Anzahl@Einf\"ugung!Anzahl}Einf\"ugungen,
die noch zur\"uckgehalten werden, der zweite Wert ist gleich den
Strafpunkten dieses \index{Seite!Umbruch}
Seitenumbruchs. Die Ausgaberoutine von {\em Plain}
\TeX\ erkennt z.B.\ ein
\index{zzzsupereject@{\tt\char92 supereject}}
`\verb|\supereject|' um alle noch ausstehenden
Einf\"ugungen an den Dvi-File auszugeben. Dies geschieht, indem der
zweite Wert auf $-20000$ gesetzt wird und der zweite Wert ausgewertet
wird um zu entscheiden, wieviele Einf\"ugungen noch vorhanden sind.

Die voreingestellte Ausgaberoutine von {\em Plain} \TeX, die einfach
alles ausgibt hat nat\"urlich auch ein Pendant, eine Ausgaberoutine, die
nichts ausgibt. Diese ist definiert als
\index{zzzoutput@{\tt\char92 output}}
\index{zzzoutputpenalty@{\tt\char92 outputpenalty}}
\index{zzzpenalty@{\tt\char92 penalty}}
\index{zzzunvbox@{\tt\char92 unvbox}}
\begin{verbatim}
\output={\unvbox255 \penalty=\outputpenalty}
\end{verbatim}
Mit dem ersten Befehl wird die Seite wieder zur\"uckgegeben, mit dem
zweiten wieder die richtige Anzahl an Strafpunkten gesetzt. Jetzt
versucht \TeX\ wieder einen \index{Seite!Umbruch}
Seitenumbruch zu finden. Da sich nichts
ge\"andert hat, wird auch wieder derselbe Seitenumbruch gefunden,
allerdings viel schneller, als beim ersten Mal, da z.B.\ die
Paragraphen nicht umgebrochen werden brauchen. Somit befindet sich
\TeX\ in einer Endlosschleife.\index{Schleife!Endlos}

Um derartige Endlosschleifen zu verhindern gibt es in \TeX\ einen
Mechanismus. Bei jedem Aufruf einer Ausgaberoutine wird der Zahler
\index{zzzdeadcycles@{\tt\char92 deadcycles}}
`\verb|\deadcycles|' um eins erh\"oht und erst bei einer tats\"achlichen
Ausgabe wieder auf Null zur\"uckgestellt. Wenn der Wert dieser Variablen
gleich oder gr\"o\ss{}er als
\index{zzzmaxdeadcycles@{\tt\char92 maxdeadcycles}}
`\verb|\maxdeadcycles|' wird, dann wird die
voreingestellte Ausgaberoutine aufgerufen.

Nach dem Aufruf einer Ausgaberoutine mu\ss{} die \index{Box!255}Box255
\index{Leer!Box}leer sein, da
\TeX\ dort die \index{Neu!Seite}\index{Seite!Neu}
neue Seite aufbauen will. Sollte das Register nicht
leer sein, dann geht \TeX\ davon aus, da\ss{} sie das Register
\index{Fehler!Box255}fehlerhaft
benutzen und zerst\"ort den Inhalt.
\section{Beispiele f\"ur Outputroutinen}
\subsection{Die Routine von {\em Plain} \TeX}
Die Ausgaberoutine von {\em Plain} \TeX\ ist definiert als
\index{zzzoutput@{\tt\char92 output}}
\index{zzzplainoutput@{\tt\char92 plainoutput}}
\begin{verbatim}
\output={\plainoutput}
\end{verbatim}
hierbei ist `\verb|\plainoutput|' eine \index{Abkuerzung@Abk\"urzung}
Abk\"urzung f\"ur
\index{zzzadvancepageno@{\tt\char92 advancepageno}}
\index{zzzelse@{\tt\char92 else}}
\index{zzzfi@{\tt\char92 fi}}
\index{zzzifnum@{\tt\char92 ifnum}}
\index{zzzmakefootline@{\tt\char92 makefootline}}
\index{zzzmakeheadline@{\tt\char92 makeheadline}}
\index{zzzoutputpenalty@{\tt\char92 outputpenalty}}
\index{zzzpagebody@{\tt\char92 pagebody}}
\index{zzzshipout@{\tt\char92 shipout}}
\index{zzzvbox@{\tt\char92 vbox}}
\begin{verbatim}
\shipout\vbox{\makeheadline
  \pagebody
  \makefootline}
\advancepageno
\ifnum\outputpenalty>-20000 \else\dosupereject\fi
\end{verbatim}
Wie im \TeX book sollte dieses ``Programm'' Zeile f\"ur Zeile betrachtet
werden, um einen Einblick in die Wirkungsweise dieser Ausgaberoutine
zu gewinnen.

Zun\"achst das Makro `\verb|\makeheadline|'. Dieses Makro erzeugt eine
Box der H\"ohe und Tiefe Null, das die \index{Kopf!Zeile}
Kopfzeile enth\"alt. Definiert ist
es folgenderma\ss{}en
\index{zzzheadline@{\tt\char92 headline}}
\index{zzzline@{\tt\char92 line}}
\index{zzznointerlineskip@{\tt\char92 nointerlineskip}}
\index{zzzthe@{\tt\char92 the}}
\index{zzzvbox@{\tt\char92 vbox}}
\index{zzzvskip@{\tt\char92 vskip}}
\index{zzzvss@{\tt\char92 vss}}
\begin{verbatim}
\vbox to 0pt{\vskip-22.5pt
  \line{\vbox to8.5pt{}\the\headline}\vss}
\nointerlineskip
\end{verbatim}
Der seltsame Wert von $-22.5$pt sorgt daf\"ur, da\ss{} die
\index{Linie!Referenz}Referenzlinie der
\index{Kopf!Zeile}
Kopfzeile genau $24$pt oberhalb der Referenzlinie der obersten
Textzeile der Seite liegt. Sie setzt sich zusammen aus
$$\mbox{topskip} - \mbox{H\"ohe eines Strut} -
2\mbox{Baselineskip}$$\index{Baseline!Skip}
das ist
$$10\mbox{pt}-8.5\mbox{pt}-24\mbox{pt}$$

Das Makro \index{zzzpagebody@{\tt\char92 pagebody}}
`\verb|\pagebody|' ist eine \index{Abkuerzung@Abk\"urzung}Abk\"urzung f\"ur
\index{zzzboxmaxdepth@{\tt\char92 boxmaxdepth}}
\index{zzzmaxdepth@{\tt\char92 maxdepth}}
\index{zzzpagecontents@{\tt\char92 pagecontents}}
\index{zzzvbox@{\tt\char92 vbox}}
\index{zzzvsize@{\tt\char92 vsize}}
\begin{verbatim}
\vbox to\vsize{\boxmaxdepth=\maxdepth \pagecontents}
\end{verbatim}
Der Wert von `\verb|\boxmaxdepth|' wird auf `\verb|\maxdepth|'
gesetzt, wobei die Routine annimmt, da\ss{} \TeX\ die \index{Box!255}
Box255 f\"ur die Seite
verwendet hat. Das Makro `\verb|\pagecontents|' f\"ugt nun die Seite,
inkl.\ aller \index{Einfuegung@Einf\"ugung}
Einf\"ugungen am Anfang und am Ende ein. Es ist definiert als
\index{zzzdimen@{\tt\char92 dimen}}
\index{zzzdp@{\tt\char92 dp}}
\index{zzzelse@{\tt\char92 else}}
\index{zzzfi@{\tt\char92 fi}}
\index{zzzfootins@{\tt\char92 footins}}
\index{zzzfootnoterule@{\tt\char92 footnoterule}}
\index{zzzifraggedbottom@{\tt\char92 ifraggedbottom}}
\index{zzzifvoid@{\tt\char92 ifvoid}}
\index{zzzkern@{\tt\char92 kern}}
\index{zzzunvbox@{\tt\char92 unvbox}}
\index{zzzvfil@{\tt\char92 vfil}}
\index{zzzvskip@{\tt\char92 vskip}}
\begin{verbatim}
\ifvoid\topins \else\unvbox\topins\fi
\dimen0=\dp255 \unvbox255
\ifvoid\footins\else % footnote info is present
  \vskip\skip\footins
  \footnoterule
  \unvbox\footins\fi
\ifraggedbottom \kern-\dimen0 \vfil \fi
\end{verbatim}
\index{zzztopins@{\tt\char92 topins}}
`\verb|\topins|' und `\verb|\footins|' sind die \index{Klasse!Nummer}
Klassennummern f\"ur die
\index{Kopf!Einfuegung@Kopf!Einf\"ugung}
Kopf- und Fu\ss{}einf\"ugungen. Sollten auch noch andere Einf\"ugungsklassen
verwendet werden, ist diese Routine entsprechend abzu\"andern. Mit
`\verb|\footnoterule|' wird ein waagerechter Strich zur Abgrenzung des
Textes von den \index{Fussnote!Abgrenzung@Fu\ss{}note!Abgrenzung}
Fu\ss{}noten ausgegeben. In der letzten Zeile wird
zus\"atzlicher \index{Leim}
Leim eingef\"ugt, falls die Seiten unterschiedliche L\"ange
haben sollen, um den
\index{zzztopskip@{\tt\char92 topskip}}`\verb|\topskip|' nicht \"uberzustrapazieren.

Im \index{Hauptprogramm}
``Hauptprogramm'' folgt nun der Aufruf von
\index{zzzmakefootline@{\tt\char92 makefootline}}
`\verb|\makefootline|'.
Dieser Aufruf setzt die \index{Fusszeile@Fu\ss{}note}Fu\ss{}zeile der Seite
\index{zzzbaselineskip@{\tt\char92 baselineskip}}
\index{zzzfootline@{\tt\char92 footline}}
\index{zzzline@{\tt\char92 line}}
\index{zzzthe@{\tt\char92 the}}
\begin{verbatim}
\baselineskip=24pt
\line{\the\footline}
\end{verbatim}\index{Baseline!Skip}

Das Makro \index{zzzadvancepageno@{\tt\char92 advancepageno}}
`\verb|\advancepageno|' erh\"oht nun die
\index{Seite!Nummer!Erhoehen@Seite!Nummer!Erh\"ohen}Seitenzahl, falls
diese positiv ist, anderenfalls wird sie erniedrigt.\footnote{Daher
erfolgt die Nummerierung der {\em negativen} Seiten richtig, also: i,
ii, iii, iv, $\ldots$} Definiert ist dieses Makro als
\index{zzzadvance@{\tt\char92 advance}}
\index{zzzelse@{\tt\char92 else}}
\index{zzzfi@{\tt\char92 fi}}
\index{zzzglobal@{\tt\char92 global}}
\index{zzzifnum@{\tt\char92 ifnum}}
\index{zzzpageno@{\tt\char92 pageno}}
\begin{verbatim}
\ifnum\pageno<0 \global\advance\pageno by-1
\else \global\advance\pageno by 1 \fi
\end{verbatim}

Schlie\ss{}lich werden {\em alle} \index{Einfuegung@Einf\"ugung}
Einf\"ugungen ausgegeben, wenn nur die
Strafpunkte den entsprechenden Wert haben. Die Definition sieht
folgenderma\ss{}en aus
\index{zzzfi@{\tt\char92 fi}}
\index{zzzifnum@{\tt\char92 ifnum}}
\index{zzzinsertpenalties@{\tt\char92 insertpenalties}}
\index{zzzkern@{\tt\char92 kern}}
\index{zzzline@{\tt\char92 line}}
\index{zzznobreak@{\tt\char92 nobreak}}
\index{zzzsupereject@{\tt\char92 supereject}}
\index{zzztopskip@{\tt\char92 topskip}}
\index{zzzvfill@{\tt\char92 vfill}}
\begin{verbatim}
\ifnum\insertpenalties>0
  \line{} \kern-\topskip \nobreak
  \vfill\supereject\fi
\end{verbatim}
Der Kernbefehl l\"oscht einen evtl. vorhandenen Topskip, der zwischen
den Einf\"ugungen nur st\"orend w\"are.
\subsection{Zweispaltige Ausgabe}
Der Trick bei einer zweispaltigen Ausgabe besteht darin, da\ss{} man \TeX\
einfach sagt, da\ss{} die Seite nur noch halb so breit sein darf, und dann
die Ausgaberoutine nur jede zweite Seite etwas ausgeben l\"a\ss{}t. Insofern
stellt also jede einzelne \index{Spalte!Seite}
Spalte eine eigenst\"andige Seite mit eigenene
Einf\"ugungen und allen anderen Einzelheiten dar. Nur die
\index{Kopf!Zeile}Kopf- und
\index{Fusszeile@Fu\ss{}note}Fu\ss{}zeile teilt sie mit der
\index{Nachbarspalte}Nachbarspalte.

Die Spalten sollen z.B.\ 3.2 Inch breit sein, der \index{Zwischenraum}
Zwischenraum soll
0.1 Inch betragen. F\"ur die \index{Kopf!Zeile}
Kopf- und Fu\ss{}zeile brauchen wir nun eine
neue Gr\"o\ss{}e
\index{zzzdef@{\tt\char92 def}}
\index{zzzfullhsize@{\tt\char92 fullhsize}}
\index{zzzfullline@{\tt\char92 fullline}}
\index{zzzhsize@{\tt\char92 hsize}}
\index{zzznewdimen@{\tt\char92 newdimen}}
\begin{verbatim}
\newdimen\fullhsize
\fullhsize=6.5in \hsize=3.2in
\def\fullline{\hbox to\fullhsize}
\end{verbatim}
Die Makros \index{zzzmakeheadline@{\tt\char92 makeheadline}}
`\verb|\makeheadline|' und \index{zzzmakefootline@{\tt\char92 makefootline}}
`\verb|\makefootline|' sollten
nun `\verb|\fullline|' statt \index{zzzline@{\tt\char92 line}}
`\verb|\line|' benutzen.

Die Ausgaberoutine unterscheidet nun einen neuen Kontrollwert
`\verb|\lr|', der die Werte `L' oder `R' annehmen kann. Bekommt die
Routine eine {\em linke} Spalte, dann wird diese einfach in einem
\index{Register}
Register gespeichert, bekommt die Routine eine rechte Spalte, dann
wird sie zusammen mit der linken ausgegeben und die
\index{Seite!Nummer}Seitennummer
erh\"oht. Die Definition sieht folgenderma\ss{}en aus
\index{zzzadvancepageno@{\tt\char92 advancepageno}}
\index{zzzbox@{\tt\char92 box}}
\index{zzzcolumnbox@{\tt\char92 columnbox}}
\index{zzzdef@{\tt\char92 def}}
\index{zzzdoubleformat@{\tt\char92 doubleformat}}
\index{zzzelse@{\tt\char92 else}}
\index{zzzfi@{\tt\char92 fi}}
\index{zzzfullline@{\tt\char92 fullline}}
\index{zzzglobal@{\tt\char92 global}}
\index{zzzhfil@{\tt\char92 hfil}}
\index{zzzifnum@{\tt\char92 ifnum}}
\index{zzzleftline@{\tt\char92 leftline}}
\index{zzzlet@{\tt\char92 let}}
\index{zzzmakefootline@{\tt\char92 makefootline}}
\index{zzzmakeheadline@{\tt\char92 makeheadline}}
\index{zzznewbox@{\tt\char92 newbox}}
\index{zzzoutput@{\tt\char92 output}}
\index{zzzoutputpenalty@{\tt\char92 outputpenalty}}
\index{zzzpagebody@{\tt\char92 pagebody}}
\index{zzzsetbox@{\tt\char92 setbox}}
\index{zzzshipout@{\tt\char92 shipout}}
\index{zzzvbox@{\tt\char92 vbox}}
\begin{verbatim}
\let\lr=L \newbox\leftcolumn
\output={\if L\lr
    \global\setbox\leftcolumn=\columnbox \global\let\lr=R
  \else \doubleformat \global\let\lr=L\fi
  \ifnum\outputpenalty>-20000 \else\dosupereject\fi}
\def\doubleformat{\shipout\vbox{\makeheadline
    \fullline{\box\leftcolumn\hfil\columnbox}
    \makefootline}
  \advancepageno}
\def\columnbox{\leftline{\pagebody}}
\end{verbatim}
Das `\verb|\columnbox|' Makro stellt sicher, da\ss{} die Seite auch in
eine Spalte pa\ss{}t. Es k\"onnte ja sein, da\ss{} im Originaltext
\index{zzzhsize@{\tt\char92 hsize}}`\verb|\hsize|'
ge\"andert wurde. Es kann nun passieren, da\ss{} ein Text mit einer
gef\"ullten linken, aber leeren rechten Spalte endet. Dieser Text w\"urde
bei der obigen Routine nicht ausgegeben. Hier hilft die Anweisung
\index{zzzeject@{\tt\char92 eject}}
\index{zzzfi@{\tt\char92 fi}}
\index{zzznull@{\tt\char92 null}}
\index{zzzsupereject@{\tt\char92 supereject}}
\index{zzzvfill@{\tt\char92 vfill}}
\begin{verbatim}
\supereject
\if R\lr \null\vfill\eject\fi
\end{verbatim}
\subsection{Markierungen}
Bei der Erstellung eines Textes gibt es die unterschiedlichsten
Anforderungen, die eine \index{Kopf!Zeile}
Kopfzeile erf\"ullen mu\ss{}. Denken sie alleine
einmal an den Unterschied zwischen einem Buch, in dessen Kopfzeile die
aktuelle \index{Kapitel!Ueberschrift@Kapitel!\"Uberschrift}
Kapitel\"uberschrift steht, gegen\"uber einem W\"orterbuch, bei dem
auf der linken \index{Seite!Woerterbuch@Seite!W\"orterbuch}
Seite die ersten Buchstaben des ersten Eintrags der
Seite stehen sollen, auf der rechten Seite aber die
\index{Anfangsbuchstaben}Anfangsbuchstaben
des letzten Eintrags der Seite. Die besondere Art von \TeX\ Seiten an
eine Ausgaberoutine zu \"ubergeben macht es nahezu unm\"oglich im
Vorhinein festzustellen, an welcher Stelle \TeX\ sich befinden wird,
wenn es eine Seite umbricht.

\TeX\ bietet nun die M\"oglichkeit innerhalb des Textes {\em
Markierungen} anzubringen. Generell funktioniert das so, da\ss{} sie
innerhalb eines Textes
\index{zzzmark@{\tt\char92 mark}}
\begin{verbatim}
\mark{<Markierungstext>}
\end{verbatim}
schreiben, wobei der `\verb|<Markierungstext>|' expandiert wird, wie
bei einer \index{zzzedef@{\tt\char92 edef}}
`\verb|\edef|' Definition. \TeX\ schreibt diese Markierung
in die vertikale Liste, wie schon fr\"uher erw\"ahnt.

Fr\"uher haben wir uns auch schon vorgestellt, da\ss{} der Text f\"ur \TeX\
eine lange Liste von Token ist. Stellen sie sich nun weiter vor diese
lange Liste wird in \index{Seite!Markierung}
Seiten gebrochen und an eine Ausgaberoutine
\"ubergeben. Dann werden vorher drei \index{Register}Register gesetzt
\begin{enumerate}
\item {\tt\char92 botmark} Ist der Markierungstext, der am weitesten
unten auf der gerade bearbeiteten Seite gefunden wurde.
\item {\tt\char92 firstmark} Ist der Markierungstext, der am weitesten
oben auf der gerade bearbeiteten Seite gefunden wurde.
\item {\tt\char92 topmark} Hat den Inhalt von {\tt\char92 botmark},
{\em bevor} die aktuelle Seite angefangen wurde.
\end{enumerate}
Angenommen sie haben einen Text geschrieben mit vier Markierungen und
der Text wird so gebrochen, da\ss{} Markierung \index{zzzalpha@{\tt\char92
alpha}}$\alpha$ auf Seite 2,\index{zzzbeta@{\tt\char92 beta}}
Markierung $\beta$ und $\gamma$ auf Seite 4 und Markierung $\delta$
auf Seite 5 steht. Dann sie die \index{Register}Register im Verlaufe der Abarbeitung
folgenderma\ss{}en gesetzt.

\medskip\begin{tabular}{cccc}
Auf Seite & {\tt\char92 topmark} ist & {\tt\char92 firstmark} ist &
{\tt\char92 botmark} ist \\[.5ex]
1 & leer & leer & leer \\
2 & leer & $\alpha$ & $\alpha$ \\
3 & $\alpha$ & $\alpha$ & $\alpha$ \\
4 & $\alpha$ & $\beta$ & $\gamma$ \\
5 & $\gamma$ & $\delta$ & $\delta$ \\
6 &  $\delta$ & $\delta$ & $\delta$
\end{tabular}\medskip

Im vertikalen Modus werden Markierungen einfach an der Stelle, an der
sie auftreten eingef\"ugt, im \index{Horizontal!Modus}
horizontalen Modus werden sie an das Ende
des Paragraphen angeh\"angt. Im internen vertikalen Modus und im
beschr\"ankten horizontalen Modus k\"onnen Markierungen verschwinden und
sollten deshalb dort nicht verwendet werden. Der
\index{zzzvsplit@{\tt\char92 vsplit}}`\verb|\vsplit|'
Befehl behandelt die Markierungen \"ubrigens entsprechend, allerdings
mit einem Unterschied. Die obere bzw.\ untere Markierung wird nun in
\index{zzzsplitbotmark@{\tt\char92 splitbotmark}}
\index{zzzsplitfirstmark@{\tt\char92 splitfirstmark}}
\begin{verbatim}
\splitfirstmark
\splitbotmark
\end{verbatim}
gefunden.

Eine Outputroutine kann nun die Markierungen nutzen um die
entsprechend gew\"unschte \index{Kopf!Zeile}
Kopfzeile zu erstellen. Bei W\"orterb\"uchern
gen\"ugt die Verwendung von\index{zzzfirstmark@{\tt\char92 firstmark}}
`\verb|\firstmark|' bzw.\ \index{zzzbotmark@{\tt\char92 botmark}}
`\verb|\botmark|'
v\"ollig, um den Anforderungen gen\"uge zu tun. Anders sieht das bei
B\"uchern aus, die andere \index{Kopf!Zeile}
Kopfzeilen erfordern. Hier ist es meist
erw\"unscht, da\ss{} in der Kopfzeile {\em das} \index{Kapitel!Kopfzeile}
Kapitel benannt werden soll,
welches zu Beginn einer Seite g\"ultig ist. Sollte auf der Seite ein
\index{Neu!Kapitel}
neues Kapitel beginnen, dann soll dessen \index{Kopf!Zeile}
Kopfzeile erst auf der
n\"achsten Seite stehen.

Angenommen in einem Buch beginnt auf einer Seite ein
\index{Neu!Abschnitt}neuer Abschnitt,
z.B.\ der `Abschnitt 3. Neues aus \TeX Land'. Dieser Abschnitt soll
mit dem Befehl
\index{zzzbeginsection@{\tt\char92 beginsection}}
\begin{verbatim}
\beginsection 3. Neues aus \TeX Land
\end{verbatim}
eingeleitet werden. Wie ist dann der Befehl `\verb|\beginsection|' am
besten zu definieren? Hier eine M\"oglichkeit
\index{zzzbeginsection@{\tt\char92 beginsection}}
\index{zzzdef@{\tt\char92 def}}
\index{zzzleftline@{\tt\char92 leftline}}
\index{zzzmark@{\tt\char92 mark}}
\index{zzznobreak@{\tt\char92 nobreak}}
\index{zzznoindent@{\tt\char92 noindent}}
\index{zzzsectionbreak@{\tt\char92 sectionbreak}}
\index{zzzsectionfont@{\tt\char92 sectionfont}}
\index{zzzsmallskip@{\tt\char92 smallskip}}
\begin{verbatim}
\def\beginsection #1. #2.
  {\sectionbreak
  \leftline{\sectionfont #1. #2}
  \mark{#1}
  \nobreak\smallskip\noindent}
\end{verbatim}
Der `\verb|\sectionbreak|' Befehl soll daf\"ur sorgen, da\ss{} entweder ein
gen\"ugend gro\ss{}er \index{Zwischenraum}
Zwischenraum zum vorherigen Abschnitt gelassen wird,
oder da\ss{} die letzte \index{Seite!Letzte}
Seite aufgef\"ullt wird und eine neue Seite begonnen
wird. Eine m\"ogliche Realisation w\"are
\index{zzzpenalty@{\tt\char92 penalty}}
\index{zzzvskip@{\tt\char92 vskip}}
\begin{verbatim}
\penalty-200 \vskip18pt plus4pt minus6pt
\end{verbatim}
Der letzte Befehl sorgt daf\"ur, da\ss{} der erste Paragraph des Textes
nicht einger\"uckt wird, uns interessiert hier aber die Markierung. Die
hier vorgestellte L\"osung erf\"ullt aber nicht die Erwartungen. Das
Ergebnis ist n\"amlich folgenderma\ss{}en.
\index{zzzfirstmark@{\tt\char92 firstmark}}`\verb|\firstmark|' wird `3.'
sein und \index{zzztopmark@{\tt\char92 topmark}}
`\verb|\topmark|' `2.' unabh\"angig davon, ob der
\index{Absatz}Absatz auf
der Seite beginnt oder nicht. Die L\"osung best\"unde darin die Markierung
{\em vor} den Befehl zu Beginn des Absatzes zu setzen
\index{zzzsectionbreak@{\tt\char92 sectionbreak}}
(`\verb|\sectionbreak|'), statt an die Stelle, an der er jetzt steht.
Dann w\"urde `\verb|\topmark|' immer den Absatz bezeichnen, der an
Beginn der Seite noch g\"ultig ist.

Es gibt aber noch eine bessere L\"osung. Diese sorgt daf\"ur, da\ss{} sich
`\verb|\topmark|' immer auf den Anfang der \index{Seite!Anfang}Seite, und
\index{zzzbotmark@{\tt\char92 botmark}}
`\verb|\botmark|' immer auf das Ende der \index{Seite!Ende}
Seite bezieht. Damit ist eine
noch genauere \index{Kopf!Gestaltung}
Kopfgestaltung, z.B.\ bezogen auf rechte und linke
\index{Seite!Rechts-Links}
Seiten m\"oglich. Eine Definition k\"onnte folgenderma\ss{}en aussehen
\index{zzzbeginsection@{\tt\char92 beginsection}}
\index{zzzcurrentsection@{\tt\char92 currentsection}}
\index{zzzdef@{\tt\char92 def}}
\index{zzzelse@{\tt\char92 else}}
\index{zzzleftline@{\tt\char92 leftline}}
\index{zzzmark@{\tt\char92 mark}}
\index{zzznobreak@{\tt\char92 nobreak}}
\index{zzznoexpand@{\tt\char92 noexpand}}
\index{zzznoindent@{\tt\char92 noindent}}
\index{zzzsectionbreak@{\tt\char92 sectionbreak}}
\index{zzzsectionfont@{\tt\char92 sectionfont}}
\index{zzzsmallskip@{\tt\char92 smallskip}}
\begin{verbatim}
\def\beginsection #1. #2.
  {\mark{\currentsection \noexpand\else #1}
  \sectionbreak
  \leftline{\sectionfont #1. #2}
  \mark{#1\noexpand\else #1} \def\currentsection{#1}
  \nobreak\smallskip\noindent}
\def\currentsection{} % the current section number
\end{verbatim}
Die Idee hier ist folgende: Es werden zwei Markierungen gesetzt, eine
vor und eine hinter dem \index{Absatz!Abstand}
Absatzabstand. Au\ss{}erdem bestehen die
Markierungen aus dem Text `\verb|2.\else 3.|' f\"ur
\index{zzztopmark@{\tt\char92 topmark}}`\verb|\topmark|'
und `\verb|3.\else 3.|' f\"ur \index{zzzbotmark@{\tt\char92 botmark}}
`\verb|\botmark|'. Die linke Komponente
von `\verb|\botmark|' ist somit f\"ur den unteren
\index{Seite!Rand!Unten}
Seitenrand, die rechte
Komponente von `\verb|\topmark|' f\"ur den oberen \index{Seite!Rand!Oben}
Seitenrand zust\"andig.

Im obigen Beispiel wurde zwischen zwei Markierungskomponenten mittels
einer `\verb|if|' Abfrage ausgew\"ahlt. V\"ollig analog kann nat\"urlich
auch ein Markierungstext aus
\begin{verbatim}
a1 \or a2 \or a3 \or ...
\end{verbatim}
bestehen und dann aus mehreren Alternativen ausgew\"ahlt werden.
\chapter{Umgang mit Fehlern}
In einem fr\"uheren Kapitel haben sie schon gesehen, wie \TeX\ darauf
reagiert, wenn eine unbekannte \index{Kontrollsequenz}
Kontrollsequenz auftaucht, meist ein
Fehler in der Schreibweise. \TeX\ kennt aber wesentlich mehr Fehler,
die sie aber vermutlich nie alle kennenlernen werden, da manche dieser
Fehler nur schwer zu produzieren sind. In diesem Kapitel werden nun
einige weitere Fehlersituationen behandelt.
\section{Schreibfehler}
Normalerweise treten Fehler als Schreibfehler auf. Was passiert z.B.\
wenn man eine \index{Einheit!Mass@Einheit!Ma\ss{}}
\index{Masse!Einheit@Ma\ss{}e!Einheit}
Ma\ss{}einheit falsch schreibt? Gehen wir man davon aus, da\ss{}
sie statt
\index{zzzhsize@{\tt\char92 hsize}}
`\verb|\hsize=4in|' `\verb|\hsize=4im|' geschrieben h\"atten.
Die Fehlermeldung, die sie erhalten wird dann ungef\"ahr so aussehen
\index{zzzinput@{\tt\char92 input}}
\begin{verbatim}
! Illegal unit of measure (pt inserted).
<to be read again>
                   i
<to be read again>
                   m
<*> \hsize=4im
               \input story
?
\end{verbatim}
Sie sehen, ich verwende wieder das Beispiel von fr\"uher. Ein einfaches
Return w\"urde hier daf\"ur sorgen, da\ss{} die Buchstaben `i' und `m' einen
neuen Paragraphen ausmachen w\"urden, besser w\"are es eine `2'
einzugeben, um die Buchstaben zu l\"oschen. Danach w\"urde \TeX\ anhalten
um ihnen die Gelegenheit zu geben das weitere Vorgehen zu \"uberdenken.
Die zugeh\"orige Meldung s\"ahe folgenderma\ss{}en aus
\index{zzzinput@{\tt\char92 input}}
\begin{verbatim}
<recently read> m
|indent
<*> \hsize=4im
               \input story
?
\end{verbatim}
O.k.\ das `i' und das `m' sind nun weg. W\"urden sie jetzt versuchen
\TeX\ mit einem einfachen Return zur Weiterarbeit zu veranlassen, dann
w\"urde \TeX\ versuchen den ganzen Text in einer Breite von 4pt zu
setzen. Die Anzahl von Fehlermeldungen wg.\ \"ubervollen \index{Box!Hbox}
{\tt hbox}en k\"onnen
sie sich sicher selber vorstellen. Besser ist es die gew\"unschte Breite
erneut einzuf\"ugen.
\index{zzzhsize@{\tt\char92 hsize}}
\begin{verbatim}
I\hsize=4in
\end{verbatim}

Normalerweise versucht \TeX\ selber Fehler zu beseitigen. Das
geschieht dadurch, da\ss{} Texte eingef\"ugt werden, die offensichtlich
ausgelassen wurden, oder aber Textteile entfernt werden, die
offensichtliche fehlen. Auch hierzu ein Beispiel
\begin{verbatim}
! Missing $ inserted.
<inserted text>
                $
<to be read again>
                   ^
l.11 Die Tatsache 32768=2^
                           {15} ist hier uninteressant
? H
I've inserted a begin-math/end-math symbol since I think
you left one out. Proceed, with fingers crossed.
\end{verbatim}
Der Unterschied zum obigen Beispiel liegt darin, da\ss{} hier der
eingef\"ugte Text (`\verb|$|') noch nicht wirklich eingef\"ugt wurde. Sie
haben also die Chance den \index{Einfuegung!Text@Einf\"ugung!Text}
Einf\"ugungstext zu entfernen, bevor \TeX\ ihn
tats\"achlich zu sehen bekommt.

Wie sollte man in einem solchen Falle vorgehen. Die Sequenz `32768=2'
ist schon gesetzt, und sie haben keine Chance den falschen Leerraum um
das \index{Gleichheitszeichen}Gleichheitszeichen nachtr\"aglich zu \"andern. Ihr Ziel sollte sein
den Rest des Textes m\"oglichst fehlerfrei bearbeitet zu bekommen, um
eventuell auftretende weitere Fehler noch gemeldet zu bekommen.

Ein Return w\"urde den Rest des Paragraphen in \index{Italic}
\index{Schrift!Italic}Italic-Schrift setzen,
was mit Sicherheit nicht w\"unschenswert ist. Es gibt zwei Alternativen
zu reagieren
\begin{enumerate}
\item Mit `6' l\"oschen sie das `\verb|$^{15}|' und f\"ugen dann mit
`\verb|I$^{15}$|' den richtigen Text ein, oder
\item sie geben `2' ein und erhalten $\ldots215$, und der Rest des
Absatzes wird richtig gesetzt.
\end{enumerate}\index{Absatz}

\section{Schlimme Fehler}
Richtig \"argerlich sind allerdings die Fehler, die z.B. in Zeile 10
eines \index{Absatz}
Absatzes verursacht werden, aber erst in Zeile 73 desselben
Absatzes bemerkt werden.\footnote{Meistens handelt es sich um eine
fehlende Gruppenklammer, es ist also ein \index{Editor}
Editor empfehlenswert, der
zueinander passende Klammerpaare findet.} Derartige Fehler haben meist
Folgefehler, dennoch ist es meist m\"oglich in einem Lauf von \TeX\ alle
Fehler zu finden, wenn man nur auf die Fehlermeldungen entsprechend
reagiert. Am schlimmsten sind aber Fehler, bei denen kein
Schreibfehler vorliegt, sondern ein Bedeutungsfehler. Wenn sie z.B.\
mit
\index{zzzdef@{\tt\char92 def}}
\begin{verbatim}
\def\box{...}
\end{verbatim}
eine Definition einrichten und so alle Boxbefehle von \TeX\ unsinnig
werden lassen, dann wird es schwer diesen Fehler zu beheben. Am besten
lesen sie dieses Buch von Anfang bis Ende und vermeiden solche Fehler.

Auch bei der \index{Makro!Definition}
Makrodefinition k\"onnen solche Fehler auftreten. Das \TeX
Book stellt einen solchen Fehler dar. Die Definition lautet
\index{zzzadvance@{\tt\char92 advance}}
\index{zzzdef@{\tt\char92 def}}
\index{zzzfirstnumber@{\tt\char92 firstnumber}}
\index{zzznewcount@{\tt\char92 newcount}}
\index{zzznextnumber@{\tt\char92 nextnumber}}
\index{zzznobreak@{\tt\char92 nobreak}}
\index{zzznumber@{\tt\char92 number}}
\index{zzzserialnumber@{\tt\char92 serialnumber}}
\begin{verbatim}
\newcount\serialnumber
\def\firstnumber{\serialnumber=0 }
\def\nextnumber{\advance \serialnumber by 1
  \number\serialnumber)\nobreak\hskip.2em }
\end{verbatim}
Mit dem Aufruf von
\begin{verbatim}
\firstnumber
\nextnumber xx, \nextnumber yy, and \nextnumber zz
\end{verbatim}
w\"urde \TeX\ dann folgendes setzen: 1)\hskip.2em xx, 2)\hskip.2em yy
und 3)\hskip.2em zz. Ein anderer Benutzer k\"onnte nun folgende
Fehlermeldung erhalten
\index{zzznextnumber@{\tt\char92 nextnumber}}
\begin{verbatim}
! Missing number, treated as zero.
<to be read again>
                   c
l.107 \nextnumber minusc
                        ule chances of error
?
\end{verbatim}
Der Fehler l\"age darin, da\ss{} das Wort `\verb|minus|' ein
\index{Schluesselwort@Schl\"usselwort}Schl\"usselwort
von \TeX\ ist und nat\"urlich eine Zahl erwartet w\"urde. Der Fehler kann
einfach dadurch behoben werden, da\ss{} die \index{Makro!Definition}
Makrodefinition mit einem\index{zzzrelax@{\tt\char92 relax}}
`\verb|\relax|' abgeschlossen wird.
\section{Fehler, die nicht zur Unterbrechung f\"uhren}
Auch wenn ihr Text vollst\"andig und ohne Fehlermeldung \"ubersetzt wird,
kann es noch zu Fehlern kommen. Eine M\"oglichkeit ist die, da\ss{} ein von
ihnen verwendeter Zeichensatz nicht zur Verf\"ugung steht und von ihrem
Dvi-Treiber durch einen anderen ersetzt wird. Dieser Ersatzzeichensatz
kann allerdings ziemlich schlecht aussehen und ihren Anspr\"uchen nicht
gen\"ugen. In diesem Falle m\"ussen sie schon per Hand zur Fehlerkorrektur
greifen, da kann ihnen \TeX\ nicht weiterhelfen.\footnote{Am besten
besorgen sie sich den ben\"otigten Zeichensatz.}

\section{Der Uralttrick}
Wenn sie wirklich einmal nicht mehr weiter wissen und absolut nicht
darauf kommen, wieso \TeX\ an einer Stelle einen Fehler macht, dann
gehen sie folgenderma\ss{}en vor. K\"urzen sie den Text solange, bis er
praktisch nur noch aus dem Fehler besteht. Sie haben eine wesentliche
bessere Chance den Fehler zu finden.

Sehr beliebt sind in diesem Zusammenhang auch fehlende
\index{Leer!Zeichen}Leerzeichen.
Denken sie daran, da\ss{} \TeX\ in zwei F\"allen Leereichen als Trenner
lie\ss{}t und nicht ausgibt
\begin{enumerate}
\item Hinter \index{Kontrollsequenz}Kontrollsequenzen, und
\item hinter Ma\ss{}angaben f\"ur \TeX\ \index{Primitiv}Primitive.
\end{enumerate}
Die Suche nach derartigen Fehlern gestaltet sich besonders bei der
Verwendung von \index{Makro!Fehler, bei}
Makros besonders schwierig. Setzen sie ggf.
\index{zzztracingcommands@{\tt\char92 tracingcommands}}
\begin{verbatim}
\tracingcommands=1
\end{verbatim}
um hinter das Problem zu kommen.
\section{Fatale Fehler}
In manchen F\"allen wei\ss{} auch das gute, alte \TeX\ nicht mehr weiter.
Z.B.\ im \index{zzzbatchmode@{\tt\char92 batchmode}}
`\verb|\batchmode|' wenn \TeX\ eine \index{Eingabe!Tastatur}
Eingabe
von der Tastatur ben\"otigt. \TeX\ macht dann einen \index{Nothalt}
Nothalt, bevor es
f\"ur immer den Geist aufgibt. Hier eine Auswahl der Fehlermeldungen,
die sie u.U.\ zu sehen bekommen.\index{File!Format}
\begin{verbatim}
Fatal format file error; I'm stymied.
\end{verbatim}
Der geladene Formatfile kann nicht benutzt werden, da er f\"ur eine
andere Version von \TeX\ hergestellt wurde.
\begin{verbatim}
That makes 100 errors; please try again.
\end{verbatim}
\TeX\ hat hundert Fehler gefunden, seit der letzte Paragraph beendet
wurde. Vermutlich befindet es sich in einer
Endlosschleife.\index{Schleife!Endlos}
\begin{verbatim}
Interwoven alignment preambles are not allowed.
\end{verbatim}
Wenn sie diese Meldung erhalten, dann werden sie auch verstehen, was
sie bedeutet, und sie werden sie nicht lieben.
\begin{verbatim}
I can't go on meeting you like this.
\end{verbatim}
Wegen eines fr\"uheren Fehlers kann \TeX\ nicht weiter arbeiten, sie
m\"ussen erst den vorigen Fehler beheben.
\begin{verbatim}
This can't happen.
\end{verbatim}
Irgendetwas ist mit ihrem \TeX\ nicht in Ordnung.
\section{Speicherprobleme}
Mitunter kann auch folgende Meldung auftreten
\begin{verbatim}
TeX capacity exceeded, sorry.
\end{verbatim}
Dies ist immer dann der Fall, wenn f\"ur irgendein Teil von \TeX\ mehr
Speicher ben\"otigt wird, als auf ihrem Rechner vorhanden ist.
Zus\"atzlich wird auch angegeben, um welche Art von Speicher es sich
handelt. Es gibt vierzehn M\"oglichkeiten:

\index{Box}\index{Einfuegung!Fehler@Einf\"ugung!Fehler}\index{File!Name}
\index{Font!Daten}
\medskip\begin{tabular}{lp{8cm}}
{\em Meldung} & {\em Gegenstand der Bearbeitung}\\[.5ex]
{\tt number of strings} & Namen der \index{Kontrollsequenz}
Kontrollsequenzen und der Files\\
{\tt pool size} & Die Zeichen in derartigen Namen\\
{\tt main memory size} & Boxen, \index{Leim}Leim, Umbruchpunkte, Tokenlisten,
Zeichen, etc.\\
{\tt hash size} & Namen der Kontrollsequenzen\\
{\tt font memory} & Zeichensatzdaten\\
{\tt exception dictionary} & Trennungsausnahmen\\
{\tt input stack size} & Parallele Eingabedateien\\
{\tt semantic nest size} & Bearbeitung einer unfertigen Liste\\
{\tt parameter stack size} & Makroparameter\\
{\tt buffer size} & Zeichen in einer Eingabezeile\\
{\tt save size} & Werte, die nach einem Gruppenende wieder hergestellt
werden m\"ussen\\
{\tt text input levels} & Inputfiles und Einf\"ugungen\\
{\tt grouping levels} & Unvollst\"andige Gruppen\\
{\tt pattern memory} & Trennungsmuster
\end{tabular}\medskip

Mit \index{zzztracingstats@{\tt\char92 tracingstats}}
`\verb|\tracingstats=1|' k\"onnen nat\"urlich wieder alle
Speicherverwendungen angesehen werden. Bei jedem Shipout wird dann der
Speicherbedarf angegeben, und zwar in der Form `xxx\&xxxx'. Dabei
meint die Zahl vor dem \index{Kaufmannsund}
Kaufmannsund wieviele (Speicher)worte f\"ur gro\ss{}e
Dinge, wie Boxen etc.\ verwendet wurden, und die zweite Zahl wieviele
f\"ur kleine Dinge, wie Zeichen oder Token verwendet wurden.

Was macht man nun, wenn ein derartiger Fehler aufgetreten ist, und man
sonst keine M\"oglichkeit hat f\"ur mehr Speicher zu sorgen. Es gibt zwei
M\"oglichkeiten. Zum einen k\"onnen sie nach gro\ss{}en Paragraphen,
\index{Tabelle!Grosse@Tabelle!Gro\ss{}e}Tabellen
oder Makrodefinitionen suchen und diese verkleinern.\footnote{Tricks
dazu stehen in diesem Text ja reichlich.} Sie k\"onnen aber auch \TeX\
anders konfigurieren. Sie m\"ussten dann f\"ur die Speicherart, bei der
das Problem auftrat, mehr physikalischen Speicher bereitstellen, den
sie bei anderen Arten, die evtl.\ nicht so ausgelastet sind,
einschr\"anken.

Meist handelt es sich aber um einen Fehler in einer Definition, der
leicht zu beheben ist. Bei der Definition von
\index{zzzdef@{\tt\char92 def}}\index{zzzrekurs@{\tt\char92 rekurs}}
\begin{verbatim}
\def\rekurs{(\rekurs)}
\end{verbatim}
erhalten sie in k\"urzester Zeit die Fehlermeldung
\index{zzzrecurse@{\tt\char92 recurse}}
\begin{verbatim}
! TeX capacity exceeded, sorry [input stack size=80].
\recurse ->(\recurse
                     )
\recurse ->(\recurse
                     )
...
\end{verbatim}
was nicht weiter verwundern d\"urfte.
\section{Eine letzte Hilfe}
Wenn ihnen sonst nichts mehr einf\"allt, um einen bestimmten Fehler zu
beheben, dann k\"onnen sie immer noch ein
\index{zzzpausing@{\tt\char92 pausing}}
`\verb|\pausing=1|' einf\"ugen,
was \TeX\ dazu veranla\ss{}t die Zeilen von der Tastatur und nicht mehr
vom \index{Eingabe!Vom File}Eingabefile einzulesen. Sie k\"onnen dann einen der vielen
`\verb|\show...|' Befehle einf\"ugen, oder sonst Befehle eingeben, die
ihnen vielleicht weiterhelfen.

Einen kleinen Wehrmutstropfen zum Schlu\ss{}. Die ganzen Tracingbefehle
sind leider nicht auf allen \TeX\ Implementationen verf\"ugbar.

\begin{appendix}
\chapter{Die Plain \TeX\ Befehle}
\section{Die Konventionen}
\subsection{Spezielle Buchstaben}
F\"ur besonderen Gebrauch sind die Zeichen
\begin{verbatim}
\  {  }  $  &  #  %  ^  _  ~
\end{verbatim}
bestimmt.
\subsection{Die Schriftarten}
\begin{tabular}{cccc}
\verb|\rm roman| & \verb|{\sl geneigt}| & \verb|{\bf fett}| &
\verb|{\it Italic\/} Schrift|\\
roman & {\sl geneigt} & {\bf fett} & {\it Italic\/} Schrift
\end{tabular}
\subsection{Sonderzeichen}
\begin{tabular}{cccccccccc}
\verb|``| & \verb|''| & \verb|--| & \verb|---| & \verb|?`| &
\verb|!`| & \verb|\$| & \verb|\#| & \verb|\&| & \verb|\%|\\
`` & '' & -- & --- & ?` &
!` & \$ & \# & \& & \%\\
\verb|\ae| & \verb|\AE| & \verb|\oe| & \verb|\OE| & \verb|\aa| &
\verb|\AA| & \verb|\ss| & \verb|\o| & \verb|\O| & \verb|\`a|\\
\ae & \AE & \oe & \OE & \aa &
\AA & \ss & \o & \O & \`a\\
\verb|\'e| & \verb|\^o| & \verb|\\"u| & \verb|\=y| & \verb|\~n| &
\verb|\.p| & \verb|\u\i| & \verb|\v s| & \verb|\H\j| & \verb|\t\i u|\\
\'e & \^o & \\"u & \=y & \~n &
\.p & \u\i & \v s & \H\j & \t\i u\\
\verb|\b k| & \verb|\c c| & \verb|\d h| & \verb|\l| & \verb|\L| &
\verb|\dag| & \verb|\ddag| & \verb|\S| & \verb|\P| & \\
\b k & \c c & \d h & \l & \L &
\dag & \ddag & \S & \P & \\
\end{tabular}\medskip

Und au\ss{}erdem noch die Zeichen

\medskip\begin{tabular}{ccccc}
\verb|{\it\$| & \verb|\&}| & \verb|\copyright| & \verb|\TeX| &
\verb|\dots|\\
{\it\$} & {\it\&} & \copyright & \TeX & \dots\\
\end{tabular}
\subsection{Zeilenumbruchbefehle}
Die {\em normalen} Befehle
\begin{verbatim}
\break \nobreak \allowbreak \hbox{Nicht umbrechbar}
\end{verbatim}
und die speziellen Befehle
\begin{verbatim}
dis\-cre\-tion\-ary Tren\-nung
Erlaubte\slash Trennung
\end{verbatim}
\subsection{Horizontale Abst\"ande}
\subsubsection{Mit erlaubtem Zeilenumbruch}
\begin{tabular}{ll}
\verb*|\ | & Normaler Wortabstand\\
\verb|\enskip| & So\enskip viel\\
\verb|\quad| & So\quad viel\\
\verb|\qquad| & So\qquad viel\\
\verb|\hskip| & Angebbarer Platz
\end{tabular}
\subsubsection{Ohne Erlaubnis zum Zeilenumbruch}
\begin{tabular}{ll}
\verb|~| & Normaler Wortabstand\\
\verb|\enspace| & So\enspace viel\\
\verb|\thinspace| & So\thinspace viel\\
\verb|\negthinspace| & So\negthinspace viel\\
\verb|\kern| & Angebbarer Platz
\end{tabular}
\subsection{Vertikale Abst\"ande}
\verb|\smallskip| \vbox{\hrule width 2cm\smallskip\hrule width2cm}\quad
\verb|\medskip| \vbox{\hrule width 2cm\medskip\hrule width2cm}\quad
\verb|\bigskip| \vbox{\hrule width 2cm\bigskip\hrule width2cm}
\subsection{Seitenumbruch}
Seitenumbruchkontrolle ohne vertikalen Platz erlauben die Befehle
\begin{verbatim}
\eject  \supereject  \nobreak  \goodbreak  \filbreak
\end{verbatim}
Mit vertikalem Platz verwendet man
\begin{verbatim}
\smallbreak  \medbreak  \bigbreak
\end{verbatim}
\subsection{Tabellenbefehle}
Zum einen gibt es die Tabellenbefehle
\begin{verbatim}
\settabs  \columns  \+  \cr
\end{verbatim}
und au\ss{}erdem die etwas generelleren Befehle
\begin{verbatim}
\halign  \valign  \omit  \span  \multispan
\end{verbatim}
Zum Auff\"ullen der Zwischenr\"aume in Tabellen dienen die Befehle
\begin{verbatim}
\hrulefill      \dotfill
\leftarrowfill  \rightarrowfill
\upbracefill    \downbracefill
\end{verbatim}
\section{Die Makros von {\tt\bf plain.tex}}
Die Makros aus dem File {\tt plain.tex} sollen ab hier erkl\"art werden.
Tritt dabei ein Befehl auf, der bisher noch nicht behandelt wurde,
dann wird er auch f\"ur normale Anwender erkl\"art, ansonsten richtet sich
die Beschreibung eher an die Designer von Makros.\footnote{Ich habe
die Kommentare im Source in englischer Sprache gehalten, die Bedeutung
sollte aber sowieso klar sein.}
\subsection{Die Kodetabellen}
Zun\"achst m\"ussen die Kategoriekodes mancher Zeichen ge\"andert werden,
damit \"uberhaupt weitere Definitionen m\"oglich sind. Die ersten Zeilen
lauten daher
\begin{verbatim}
\catcode`\{=1 % left brace is begin-group character
\catcode`\}=2 % right brace is end-group character
\catcode`\$=3 % dollar sign is math shift
\catcode`\&=4 % ampersand is alignment tab
\catcode`\#=6 % hash mark is macro parameter character
\catcode`\^=7 \catcode`\^^K=7 % circumflex and uparrow for superscripts
\catcode`\_=8 \catcode`\^^A=8 % underline and downarrow for subscripts
\catcode`\^^I=10 % ASCII tab is treated as a blank space
\chardef\active=13 \catcode`\~=\active % tilde is active
\catcode`\^^L=\active \outer\def^^L{\par} % ASCII form-feed is \outer\par

\message{Preloading the plain format: codes,}
\end{verbatim}\index{ASCII}
Die Zuweisungen f\"ur `\verb|^^K|' und `\verb|^^A|' dienen f\"ur die
Verwendung von anderen Tastaturen. Alle Zeichen au\ss{}er den Buchstaben
erhalten die Kategorie {\em anders}~(12), die Buchstaben die
Kategorie~11. Weiterhin werden die folgenden Zeichenkodes gesetzt
\begin{verbatim}
\catcode `\\ =0   \catcode`\   =10   \catcode `\%=14
\catcode`\^^@=9   \catcode`\^^M=5   \catcode`\^^?=15
\end{verbatim}
Der `\verb|\message|' Befehl gibt eine Meldung \"uber den Stand der
Dinge bis zu diesem Zeitpunkt auf dem \index{Bildschirm}
Bildschirm aus.\footnote{Diese
Meldungen erscheinen {\em immer} bei der Bearbeitung mit INITEX.} Wenn
sie au\ss{}er den Befehlen von {\tt plain.tex} weitere Befehle einbinden
wollen, dann schreiben sie nach der Aufforderung von INITEX einfach
`\verb|&plain meine|', wenn sie ihre Definitionen in einer Datei mit
dem Namen {\tt meine.tex} geschrieben haben. Die {\em Plain} \TeX\
Befehle werden dann auf dem schnelleren, vorkompilierten Wege
eingelesen und ihre Definitionen angef\"ugt.

Schlie\ss{}lich wird das `\verb|@|' Zeichen vor\"ubergehend zu einem
normalen Zeichen gemacht. Auf diese Art ist es m\"oglich, da\ss{} {\em
Plain} \TeX\ seine eigenen Befehle kennt, die sie als Anwender nicht
so einfach benutzen k\"onnen. Sp\"ater wird diesem Zeichen wieder eine
andere Kategorie zugewiesen. Jetzt kommen die Tabellen der
mathematischen Zeichen
\begin{verbatim}
\mathcode`\^^@=\"2201 \mathcode`\^^A=\"3223 \mathcode`\^^B=\"010B
\mathcode`\^^C=\"010C \mathcode`\^^D=\"225E \mathcode`\^^E=\"023A
\mathcode`\^^F=\"3232 \mathcode`\^^G=\"0119 \mathcode`\^^H=\"0115
\mathcode`\^^I=\"010D \mathcode`\^^J=\"010E \mathcode`\^^K=\"3222
\mathcode`\^^L=\"2206 \mathcode`\^^M=\"2208 \mathcode`\^^N=\"0231
\mathcode`\^^O=\"0140 \mathcode`\^^P=\"321A \mathcode`\^^Q=\"321B
\mathcode`\^^R=\"225C \mathcode`\^^S=\"225B \mathcode`\^^T=\"0238
\mathcode`\^^U=\"0239 \mathcode`\^^V=\"220A \mathcode`\^^W=\"3224
\mathcode`\^^X=\"3220 \mathcode`\^^Y=\"3221 \mathcode`\^^Z=\"8000
\mathcode`\^^[=\"2205 \mathcode`\^^\=\"3214 \mathcode`\^^]=\"3215
\mathcode`\^^^=\"3211 \mathcode`\^^_=\"225F \mathcode`\^^?=\"1273
\mathcode`\ =\"8000   \mathcode`\!=\"5021   \mathcode`\'=\"8000
\mathcode`\(=\"4028   \mathcode`\)=\"5029   \mathcode`\*=\"2203
\mathcode`\+=\"202B   \mathcode`\,=\"613B   \mathcode`\-=\"2200
\mathcode`\.=\"013A   \mathcode`\/=\"013D   \mathcode`\:=\"303A
\mathcode`\;=\"603B   \mathcode`\<=\"313C   \mathcode`\==\"303D
\mathcode`\>=\"313E   \mathcode`\?=\"503F   \mathcode`\[=\"405B
\mathcode`\\=\"026E   \mathcode`\]=\"505D   \mathcode`\_=\"8000
\mathcode`\{=\"4266   \mathcode`\|=\"026A   \mathcode`\}=\"5267
\end{verbatim}
Es folgen die Setzungen f\"ur\index{Delcode}
\begin{verbatim}
\uccode  \lccode
\sfcode  \delcode
\end{verbatim}
Abschlie\ss{}end noch einige Definitionen
\begin{verbatim}
\chardef\@ne=1 \chardef\tw@=2 \chardef\thr@@=3
\chardef\sixt@@n=16 \chardef\@cclv=255
\mathchardef\@cclvi=256 \mathchardef\@m=1000
\mathchardef\@M=10000 \mathchardef\@MM=20000
\end{verbatim}
Diese Definitionen sorgen daf\"ur, da\ss{} \TeX\ bei manchen Anwendungen
schneller l\"auft und weniger Speicher ben\"otigt. In diesem Anhang werden
aber dennoch immer die Zahlkonstanten verwendet, da so die Programme
besser lesbar sind.
\subsection{Die Register}
Der zweite Teil von {\tt plain.tex} besch\"aftigt sich mit der
\index{Alloziierung!Register}
Alloziierung von Registern, so da\ss{} sp\"ater die einzelnen Makros
miteinander arbeiten k\"onnen, ohne sich gegenseitig die Register
umzubelegen. Hier die zugeh\"origen Regeln
\begin{enumerate}
\item Die Register mit den Nummern~0 bis~9 sind immer f\"ur den
tempor\"aren Gebrauch frei. Jedes Makro beendet seine Benutzung eines
dieser Register nach seiner Benutzung.\footnote{Die Z\"ahlregister
{\tt\char92 count0} bis {\tt\char92 count9} sind allerdings schon von
\TeX\ vorbelegt. Diese Reservierung gilt also nur f\"ur alle anderen
Register.}
\item Die Register {\tt\char92 count255}, {\tt\char92 dimen255} und
{\tt\char92 skip255} stehen auch nur f\"ur den tempor\"aren Gebrauch zur
Verf\"ugung.
\item Alle Zuweisungen zu ungeraden Registern sollten lokal, alle zu
geraden Registern sollten global sein.\footnote{Damit verhindert man,
da\ss{} zuviele Register am Ende einer Gruppe wieder hergestellt werden
m\"ussen.}
\item Register d\"urfen nat\"urlich innerhalb einer Gruppe frei belegt
werden, wenn sichergestellt ist, da\ss{} sie am Ende der Gruppe wieder
hergestellt werden k\"onnen und wenn keine anderen Makros globale
Zuweisungen auf dieses Register vornehmen.
\item Register, die von vielen Makros, oder f\"ur eine l\"angere Zeit
beansprucht werden, sollten mit den {\tt\char92 new...} Befehle
alloziiert werden.
\end{enumerate}
Einige \index{Abkuerzung@Abk\"urzung}Abk\"urzungen dienen dem besseren Zugriff auf die
Temporalregister
\begin{verbatim}
\countdef\count@=255  \toksdef\toks@=0       \skipdef\skip@=0
\dimendef\dimen@=0    \dimendef\dimen@i=1    \dimendef\dimen@ii=2
\end{verbatim}
Es folgen nun einige Zuweisungen zu Registern, mit denen sp\"ater
festgehalten wird, welche Zahl denn nun mit einem {\tt\char92 new}
Befehl alloziiert wurde. Diese Zahl kann auch von einem alloziierenden
Makro verwendet werden.
\begin{verbatim}
\count10=22 % this counter allocates \count registers 23, 24, 25, ...
\count11=9  % this counter allocates \dimen registers 10, 11, 12, ...
\count12=9  % this counter allocates \skip registers 10, 11, 12, ...
\count13=9  % this counter allocates \muskip registers 10, 11, 12, ...
\count14=9  % this counter allocates \box registers 10, 11, 12, ...
\count15=9  % this counter allocates \toks registers 10, 11, 12, ...
\count16=-1 % this counter allocates input streams 0, 1, 2, ...
\count17=-1 % this counter allocates output streams 0, 1, 2, ...
\count18=3  % this counter allocates math families 4, 5, 6, ...
\count19=0  % this counter allocates language codes 1, 2, 3, ...
\count20=255 % this counter allocates insertions 254, 253, 252, ...
\countdef\insc@unt=20 % nickname for the insertion counter
\countdef\allocationnumber=21 % the most recent allocation
\countdef\m@ne=22 \m@ne=-1 % a handy constant
\def\wlog{\immediate\write-1} % this will write on log file (only)
\end{verbatim}
\begin{verbatim}
\outer\def\newcount{\alloc@0\count\countdef\insc@unt}
\outer\def\newdimen{\alloc@1\dimen\dimendef\insc@unt}
\outer\def\newskip{\alloc@2\skip\skipdef\insc@unt}
\outer\def\newmuskip{\alloc@3\muskip\muskipdef\@cclvi}
\outer\def\newbox{\alloc@4\box\chardef\insc@unt}
\let\newtoks=\relax % this allows plain.tex to be read in twice
\outer\def\newhelp#1#2{\newtoks#1#1=\expandafter{\csname#2\endcsname}}
\outer\def\newtoks{\alloc@5\toks\toksdef\@cclvi}
\outer\def\newread{\alloc@6\read\chardef\sixt@@n}
\outer\def\newwrite{\alloc@7\write\chardef\sixt@@n}
\outer\def\newfam{\alloc@8\fam\chardef\sixt@@n}
\outer\def\newlanguage{\alloc@9\language\chardef\@cclvi}
\end{verbatim}
\begin{verbatim}
\def\alloc@#1#2#3#4#5{\global\advance\count1#1 by 1
  \ch@ck#1#4#2% make sure there's still room
  \allocationnumber=\count1#1   \global#3#5=\allocationnumber
  \wlog{\string#5=\string#2\the\allocationnumber}}
\end{verbatim}
\begin{verbatim}
\outer\def\newinsert#1{\global\advance\insc@unt by-1
  \ch@ck0\insc@unt\count \ch@ck1\insc@unt\dimen
  \ch@ck2\insc@unt\skip  \ch@ck4\insc@unt\box
  \allocationnumber=\insc@unt
  \global\chardef#1=\allocationnumber
  \wlog{\string#1=\string\insert\the\allocationnumber}}
\end{verbatim}
\begin{verbatim}
\def\ch@ck#1#2#3{\ifnum\count1#1<#2%
  \else\errmessage{No room for a new #3}\fi}
\end{verbatim}
Das Makro `\verb|\alloc@|' erledigt hier die meiste Arbeit, inkl.\ der
Meldung in den Logfile etc. Au\ss{}erdem wurde ein Befehl `\verb|\newhelp|'
eingef\"uhrt, der es erlaubt relativ einfach eigene Hilfstexte zu
bearbeiten. Mit der Folge
\begin{verbatim}
\newhelp\helpout{Hilfstext}
\errhelp=\helpout
\end{verbatim}
vor dem Aufruf von `\verb|\errmessage|' wird dieser Text verf\"ugbar
gemacht.\footnote{Dies d\"urfte aber wirklich {\em nur} Makrodesigner
interessieren.}

Nun werden die wichtigen Konstanten \index{Alloziierung!Konstanten}
alloziiert.
\begin{verbatim}
\newdimen\maxdimen \maxdimen=16383.99999pt
\newskip\hideskip \hideskip=-1000pt plus1fill
\newskip\centering \centering=0pt plus 1000pt minus 1000pt
\newdimen\p@ \p@=1pt % this saves macro space and time
\newdimen\z@ \z@=0pt % likewis
\newskip\z@skip \z@skip=0pt plus0pt minus0pt
\newbox\voidb@x % permanently void box register
\end{verbatim}
Die Kontrollsequenz `\verb|\maxdimen|' steht f\"ur die gr\"o\ss{}te zur
Verf\"ugung stehende Dimension. Die Befehle `\verb|\hideskip|' und
`\verb|\centering|' sollten niemals ge\"andert werden, von ihnen wird
sp\"ater noch zu lesen sein. Die konstanten Zuweisungen dienen wieder
einmal der Optimierung.

Abschlie\ss{}end noch einige andere Alloziierungen
\begin{verbatim}
\outer\def\newif#1{\count@=\escapechar \escapechar=-1\parbreak%
  \expandafter\expandafter\expandafter\parbreak%
   \edef\@if#1{true}{\let\noexpand#1=\noexpand\iftrue}%\parbreak%
  \expandafter\expandafter\expandafter\parbreak%
   \edef\@if#1{false}{\let\noexpand#1=\noexpand\iffalse}%\parbreak%
  \@if#1{false}\escapechar=\count@} % the condition starts out false
\def\@if#1#2{\csname\expandafter\if@\string#1#2\endcsname}\parbreak%
{\uccode`1=`i \uccode`2=`f \uppercase{\gdef\if@12{}}} % `if' is required
\end{verbatim}
\subsection{Parameter}
INITEX setzt die meisten Parameter auf Null, leert alle Box- und
sonstigen Register, es gibt aber auch einige Ausnahmen

\medskip\begin{tabular}{ll}
{\em Parameter} & {\em voreingestellter Wert}\\[.5ex]
\verb|\mag| & 1000\\
\verb|\tolerance| & 10000\\
\verb|\maxdeadcycles| & 25\\
\verb|\hangafter| & 1\\
\verb|\escapechar| & \verb|`\\|\\
\verb|\endlinechar| & \verb|`\^^M|
\end{tabular}\medskip

Dar\"uber hinaus setzt {\em Plain} \TeX\ noch weitere Werte
\begin{verbatim}
\pretolerance=100 \tolerance=200 \hbadness=1000 \vbadness=1000
\linepenalty=10   \hyphenpenalty=50 \exhyphenpenalty=50
\binoppenalty=700 \relpenalty=500
\clubpenalty=150 \widowpenalty=150 \displaywidowpenalty=50
\brokenpenalty=100 \predisplaypenalty=10000
\doublehyphendemerits=10000 \finalhyphendemerits=5000 \adjdemerits=10000
\tracinglostchars=1 \uchyph=1 \delimiterfactor=901
\defaulthyphenchar=`\- \defaultskewchar=-1 \newlinechar=-1
\showboxbreadth=5 \showboxdepth=3 \errorcontextlines=5
\end{verbatim}\index{Badness}
\begin{verbatim}
\hfuzz=0.1pt \vfuzz=0.1pt \overfullrule=5pt
\hsize=6.5in \vsize=8.9in \parindent=20pt
\maxdepth=4pt \splitmaxdepth=\maxdimen \boxmaxdepth=\maxdimen
\delimitershortfall=5pt \nulldelimiterspace=1.2pt \scriptspace=0.5pt
\end{verbatim}
\begin{verbatim}
\parskip=0pt plus 1pt
\abovedisplayskip=12pt plus 3pt minus 9pt
\abovedisplayshortskip=0pt plus 3pt
\belowdisplayskip=12pt plus 3pt minus 9pt
\belowdisplayshortskip=7pt plus 3pt minus 4pt
\topskip=10pt \splittopskip=10pt
\parfillskip=0pt plus 1fil
\end{verbatim}
\begin{verbatim}
\thinmuskip=3mu
\medmuskip=4mu plus 2mu minus 4mu
\thickmuskip=5mu plus 5mu
\end{verbatim}
Einige Parameter werden erst zur Laufzeit initialisiert. Die Parameter
\begin{verbatim}
\time \day \month \year
\end{verbatim}
werden erst zu Beginn einer Bearbeitung gesetzt. Der Parameter
`\verb|\outputpenalty|' wird von der Ausgaberoutine gesetzt. Die
Parameter
\begin{verbatim}
\predisplaysize \displaywidth \displayindet
\end{verbatim}
erhalten ihre Werte, unmittelbar bevor eine \index{Formel!abgesetzte}
abgesetzte Formel erstellt
wird, und die Parameter
\begin{verbatim}
\looseness=0
\hangindent=0pt
\hangafter=1
\parshape=0
\end{verbatim}
werden am Ende jedes Paragraphen gesetzt, wenn \TeX\ in den vertikalen
Modus \"ubergeht. Auch die Parameter
\begin{verbatim}
\baselineskip \lineskip \lineskiplimit
\end{verbatim}\index{Baseline!Skip}
werden erst durch ein Makro initialisiert, das erst sp\"ater vorgestellt
wird.

Jetzt kommen einige ``Pseudoparameter'', sie verhalten sich genauso,
wie die Parameter von \TeX, geh\"oren aber eher zum Sprachumfang von
{\em Plain} \TeX, als zu den Primitiven.
\begin{verbatim}
\newskip\smallskipamount % the amount of a \smallskip
  \smallskipamount=3pt plus1pt minus1pt
\newskip\medskipamount % the amount of a \medskip
  \medskipamount=6pt plus2pt minus2pt
\newskip\bigskipamount % the amount of a \bigskip
  \bigskipamount=12pt plus4pt minus4pt
\newskip\normalbaselineskip % normal value of \baselineskip
  \normalbaselineskip=12pt
\newskip\normallineskip % normal value of \lineskip
  \normallineskip=1pt
\newdimen\normallineskiplimit % normal value of \lineskiplimit
  \normallineskiplimit=0pt
\newdimen\jot % unit of measure for opening up displays
  \jot=3pt
\newcount\interdisplaylinepenalty % interline penalty in \displaylines
  \interdisplaylinepenalty=100
\newcount\interfootnotelinepenalty % interline penalty in footnotes
  \interfootnotelinepenalty=100
\end{verbatim}
\subsection{Fontinformationen}
Hier kommen nun die Befehle, die \TeX\ braucht um sich in den ganzen
Zeichens\"atzen zurechtfinden zu k\"onnen. Zun\"achst das {\tt magstep}
Makro
\begin{verbatim}
\def\magstephalf{1095 }
\def\magstep#1{\ifcase#1 1000\or
  1200\or 1440\or 1728\or 2074\or 2488\fi\relax}
\end{verbatim}
Bemerkenswert ist vielleicht, da\ss{} eine Zahl zur\"uck gegeben wird und
nicht mit dem Multiplikationsbefehl gearbeitet wird. Der Grund daf\"ur
liegt darin, da\ss{} der Multiplikationsbefehl eine Zuweisung bedeutet,
und Zuweisungen werden innerhalb von Makros nicht ausgef\"uhrt.

Jetzt kommen die Zeichens\"atze, die schon einmal vorab geladen werden
sollen.
\begin{verbatim}
\font\tenrm=cmr10        \font\preloaded=cmr9     \font\preloaded=cmr8
\font\sevenrm=cmr7       \font\preloaded=cmr6     \font\fiverm=cmr5
\end{verbatim}
\begin{verbatim}
\font\teni=cmmi10        \font\preloaded=cmmi9    \font\preloaded=cmmi8
\font\seveni=cmmi7       \font\preloaded=cmmi6    \font\fivei=cmmi5
\end{verbatim}
\begin{verbatim}
\font\tensy=cmsy10       \font\preloaded=cmsy9    \font\preloaded=cmsy8
\font\sevensy=cmsy7      \font\preloaded=cmsy6    \font\fivesy=cmsy5
\end{verbatim}
\begin{verbatim}
\font\tenex=cmex10
\end{verbatim}
\begin{verbatim}
\font\tenbf=cmbx10       \font\preloaded=cmbx9    \font\preloaded=cmbx8
\font\sevenbf=cmbx7      \font\preloaded=cmbx6    \font\fivebf=cmbx5
\end{verbatim}
\begin{verbatim}
\font\tensl=cmsl10       \font\preloaded=cmsl9    \font\preloaded=cmsl8
\font\tentt=cmtt10       \font\preloaded=cmtt9    \font\preloaded=cmtt8
\font\tenit=cmti10       \font\preloaded=cmti9    \font\preloaded=cmti8
\font\preloaded=cmss10   \font\preloaded=cmssq8
\font\preloaded=cmssi10  \font\preloaded=cmssqi8
\end{verbatim}
\begin{verbatim}
\font\preloaded=cmr7 scaled \magstep4 % for titles
\font\preloaded=cmtt10 scaled \magstep2
\font\preloaded=cmssbx10 scaled \magstep2
\end{verbatim}
\begin{verbatim}
% Additional \preloaded fonts can be specified here.
% (And those that were \preloaded above can be eliminated.)
\let\preloaded=\undefined % preloaded fonts must be declared anew later.
\end{verbatim}
Mit dem `\verb|\preloaded|' Befehl werden die Zeichens\"atze nicht {\em
tatsaechlich} verfuegbar gemacht. Der Befehl bleibt undefiniert. Der
Grund daf\"ur liegt darin, da\ss{} {\em Plain} \TeX\ ein Standardformat ist,
das nicht zu viele Zeichens\"atze enthalten soll. Dennoch werden die
Zeichensatzinformationen hier in \TeX's Speicher geladen und nicht
erst beim Aufruf mit `\verb|\font|'. Das spart sp\"ater eine Menge Zeit.

Die meisten Zeichens\"atze haben ihre eigenen Werte f\"ur `\verb|\/|',
Trennungszeichen u.a. Mathematische Zeichen- und Symbols\"atze haben das
oft nicht, sie m\"ussen erst definiert werden, das geschieht hier.
\begin{verbatim}
\skewchar\teni='177 \skewchar\seveni='177 \skewchar\fivei='177
\skewchar\tensy='60 \skewchar\sevensy='60 \skewchar\fivesy='60
\end{verbatim}
Nachdem nun die Zeichens\"atze geladen sind, m\"ussen sie noch in
Familien, f\"ur den Gebrauch in mathematischen Sequenzen eingeordnet
werden, und es fehlen noch die \index{Abkuerzung@Abk\"urzung}
Abk\"urzungen, wie `\verb|\it|' oder
`\verb|\rm|'
\begin{verbatim}
\textfont0=\tenrm \scriptfont0=\sevenrm \scriptscriptfont0=\fiverm
\def\rm{\fam0 \tenrm}
\textfont1=\teni  \scriptfont1=\seveni  \scriptscriptfont1=\fivei
\def\mit{\fam1 } \def\oldstyle{\fam1 \teni}
\textfont2=\tensy \scriptfont2=\sevensy \scriptscriptfont2=\fivesy
\def\cal{\fam2 }
\textfont3=\tenex \scriptfont3=\tenex   \scriptscriptfont3=\tenex
\end{verbatim}
\begin{verbatim}
\newfam\itfam \def\it{\fam\itfam\tenit} \textfont\itfam=\tenit
\newfam\slfam \def\sl{\fam\slfam\tensl} \textfont\slfam=\tensl
\newfam\bffam \def\bf{\fam\bffam\tenbf} \textfont\bffam=\tenbf
  \scriptfont\bffam=\sevenbf   \scriptscriptfont\bffam=\fivebf
\newfam\ttfam \def\tt{\fam\ttfam\tentt} \textfont\ttfam=\tentt
\end{verbatim}
\subsection{Textmakros}
Hier kommen nun die ersten Makros, zun\"achst die, die nichts mit dem
mathematischen Zeichensatz zu tun haben.
\begin{verbatim}
\def\frenchspacing{\sfcode`\.=1000 \sfcode`\?=1000 \sfcode`\!=1000
  \sfcode`\:=1000 \sfcode`\;=1000 \sfcode`\,=1000 }
\def\nonfrenchspacing{\sfcode`\.=3000 \sfcode`\?=3000 \sfcode`\!=3000
  \sfcode`\:=2000 \sfcode`\;=1500 \sfcode`\,=1250 }
\def\normalbaselines{\lineskip=\normallineskip
  \baselineskip=\normalbaselineskip \lineskiplimit=\normallineskiplimit}
\end{verbatim}\index{Baseline!Normal}
Die nun folgenden Makros sind zwar einfach, aber sie werden oft
eingesetzt. Zun\"achst werden `\verb|<TAB>|' und `\verb|<RETURN>|' so
umdefiniert, da\ss{} sie als normales Leerzeichen verstanden werden. Dann
kommen Makros, f\"ur Leute, die sich schwer tun
\index{Anfuehrungszeichen@Anf\"uhrungszeichen}Anf\"uhrungszeichen und
eckige Klammern zu schreiben. Die Definitionen von `\verb|\endgraf|'
und `\verb|\endline|' sind n\"otig, da es oft sinnvoll ist `\verb|\par|'
und `\verb|\cr|' selber umzudefinieren. Der Rest erkl\"art sich von
selber.
\begin{verbatim}
\def\^^I{\ } \def\^^M{\ }
\def\lq{`} \def\rq{'} \def\lbrack{[} \def\rbrack{]}
\let\endgraf=\par \let\endline=\cr
\def\space{ } \def\empty{} \def\null{\hbox{}}
\let\bgroup={ \let\egroup=}
\end{verbatim}
Nun folgt die Definition des trickreichen Makros `\verb|\obeylines|'
und `\verb|\obeyspaces|'
\begin{verbatim}
\def\obeyspaces{\catcode`\ =\active}
{\obeyspaces\global\let =\space}
{\catcode`\^^M=\active % these lines must end with `%'
  \gdef\obeylines{\catcode`\^^M=\active \let^^M=\par}%
  \global\let^^M=\par} % this is in case ^^M appears in a \write
\end{verbatim}
Die Definition benutzt `\verb|\let|' anstelle von `\verb|\def|' um
sp\"ater mehr Flexibilit\"at zu erreichen.\footnote{Beispiele dazu
im Kapitel \"uber Tabellen.}
Jetzt folgen die {\em Schleifen}makros.
\begin{verbatim}
\def\loop#1\repeat{\def\body{#1}\iterate}
\def\iterate{\body \let\next=\iterate \else\let\next=\relax\fi \next}
\let\repeat=\fi % this makes \loop...\if...\repeat skippable
\end{verbatim}
Nun kommen die Definitionen f\"ur Leerr\"aume. Neben den bekannten
Befehlen mit ihren Eigenschaften werden hier auch `\verb|\hglue\<Leim>|'
und `\verb|\vglue\<Leim>|' definiert, die einen Zwischenraum schaffen,
der {\em nie} entfernt wird.

Die Makros `\verb|\nointerlineskip|' und `\verb|\offinterlineskip|'
bewirken beide, da\ss{} zwischen den n\"achsten Zeilen kein
\index{Abstand!Zeilen}Zeilenabstand
eingef\"ugt wird. Der zweite Befehl ist dabei der weitergehende. Er
bewirkt, da\ss{} zwar Zwischenraum da ist, dieser aber die Ausdehnung Null
hat. Der erste Befehl wirkt nur auf die n\"achste Zeile.
\begin{verbatim}
\def\enskip{\hskip.5em\relax}    \def\enspace{\kern.5em }
\def\quad{\hskip1em\relax}       \def\qquad{\hskip2em\relax}
\def\thinspace{\kern .16667em }  \def\negthinspace{\kern-.16667em }
\end{verbatim}
\begin{verbatim}
\def\hglue{\afterassignment\hgl@\skip@=}
\def\hgl@{\leavevmode \count@=\spacefactor \vrule width0pt
  \nobreak\hskip\skip@ \spacefactor=\count@}
\def\vglue{\afterassignment\vgl@\skip@=}
\def\vgl@{\par \dimen@=\prevdepth \hrule height0pt
  \nobreak\vskip\skip@ \prevdepth=\dimen@}
\def\topglue{\nointerlineskip \vglue-\topskip \vglue} % for top of page
\end{verbatim}
\begin{verbatim}
\def\nointerlineskip{\prevdepth=-1000pt }
\def\offinterlineskip{\baselineskip=-1000pt
  \lineskip=0pt \lineskiplimit=\maxdimen}
\end{verbatim}\index{Baseline!Skip}
\begin{verbatim}
\def\smallskip{\vskip\smallskipamount}
\def\medskip{\vskip\medskipamount}
\def\bigskip{\vskip\bigskipamount}
\end{verbatim}
Nun kommen die Befehle zum Zeilenumbruch u.a.
\begin{verbatim}
\def\break{\penalty-10000 } \def\nobreak{\penalty10000 }
\def\allowbreak{\penalty0 }
\def~{\penalty10000\ }
\def\slash{/\penalty\exhyphenpenalty}
\end{verbatim}
\begin{verbatim}
\def\filbreak{\par\vfil\penalty-200\vfilneg}
\def\goodbreak{\par\penalty-500 }
\def\eject{\par\penalty-10000 }
\def\supereject{\par\penalty-20000 }
\end{verbatim}
\begin{verbatim}
\def\removelastskip{\ifdim\lastskip=0pt \else\vskip-\lastskip\fi}
\def\smallbreak{\par \ifdim\lastskip<\smallskipamount
  \removelastskip \penalty-50 \smallskip \fi}
\def\medbreak{\par \ifdim\lastskip<\medskipamount
  \removelastskip \penalty-100 \medskip \fi}
\def\bigbreak{\par \ifdim\lastskip<\bigskipamount
  \removelastskip \penalty-200 \bigskip \fi}
\end{verbatim}
Als n\"achstes die Boxbefehle\index{Box!Befehle}
\begin{verbatim}
\def\line{\hbox to\hsize}
\def\leftline#1{\line{#1\hss}} \def\rightline#1{\line{\hss#1}}
\def\centerline#1{\line{\hss#1\hss}}
\end{verbatim}
\begin{verbatim}
\def\llap#1{\hbox to 0pt{\hss#1}} \def\rlap#1{\hbox to 0pt{#1\hss}}
\end{verbatim}
\begin{verbatim}
\def\m@th{\mathsurround=0pt }
\def\underbar#1{$\setbox0=\hbox{#1} \dp0=0pt
  \m@th \underline{\box0}$}
\end{verbatim}
Beachten sie, da\ss{} \TeX\ in den mathematischen Modus umschaltet, um
eine Unterstreichung vorzunehmen. Dieser Gebrauch vom mathematischen
Modus kommt noch mehrmals vor.

Zun\"achst aber mal der `\verb|\strut|' Befehl in allen vorkommenden
Feinheiten
\begin{verbatim}
\newbox\strutbox
\setbox\strutbox=\hbox{\vrule height8.5pt depth3.5pt width0pt}
\def\strut{\relax\ifmmode\copy\strutbox\else\unhcopy\strutbox\fi}
\end{verbatim}
Der `\verb|\relax|' Befehl wird hier verwendet, um Seiteneffekte zu
vermeiden.

Und noch einige Tabellenbefehle
\begin{verbatim}
\def\ialign{\everycr={}\tabskip=0pt \halign} % initialized \halign
\def\hidewidth{\hskip\hideskip}
\end{verbatim}
\begin{verbatim}
\newcount\mscount
\def\multispan#1{\omit \mscount=#1 \loop\ifnum\mscount>1 \sp@n\repeat}
\def\sp@n{\span\omit \advance\mscount by -1 }
\end{verbatim}
Die nun folgenden Befehle betreffen alle den Tabellenaufbau. Am besten
sehen sie sich die Wirkung mit `\verb|\tracingall|' an einem einfachen
Beispiel an.
\begin{verbatim}
\newif\ifus@ \newif\if@cr
\newbox\tabs \newbox\tabsyet \newbox\tabsdone
\end{verbatim}
\begin{verbatim}
\def\cleartabs{\global\setbox\tabsyet=\null \setbox\tabs=\null}
\def\settabs{\setbox\tabs=\null \futurelet\next\sett@b}
\let\+=\relax % in case this file is being read in twice
\def\sett@b{\ifx\next\+ \let\next=\relax % turn off \outerness
    \def\next{\afterassignment\s@tt@b\let\next}%
  \else\let\next=\s@tcols\fi\next}
\def\s@tt@b{\let\next=\relax \us@false\m@ketabbox}
\outer\def\+{\tabalign} \def\tabalign{\us@true \m@ketabbox}
\def\s@tcols#1\columns{\count@=#1 \dimen@=\hsize
  \loop \ifnum\count@>0 \@nother \repeat}
\def\@nother{\dimen@ii=\dimen@ \divide\dimen@ii by\count@
  \setbox\tabs=\hbox{\hbox to\dimen@ii{}\unhbox\tabs}%
  \advance\dimen@ by-\dimen@ii \advance\count@ by -1 }
\end{verbatim}
\begin{verbatim}
\def\m@ketabbox{\begingroup
  \global\setbox\tabsyet=\copy\tabs \global\setbox\tabsdone=\null
  \def\cr{\@crtrue\crcr\egroup\egroup
    \ifus@ \unvbox0 \lastbox\fi \endgroup
    \setbox\tabs=\hbox{\unhbox\tabsyet\unhbox\tabsdone}}%
  \setbox0=\vbox\bgroup\@crfalse \ialign\bgroup&\t@bbox##\t@bb@x\crcr}
\end{verbatim}
\begin{verbatim}
\def\t@bbox{\setbox0=\hbox\bgroup}
\def\t@bb@x{\if@cr\egroup % now \box0 holds the column
  \else\hss\egroup \global\setbox\tabsyet=\hbox{\unhbox\tabsyet
      \global\setbox1=\lastbox}% now \box1 holds its size
    \ifvoid1 \global\setbox1=\hbox to\wd0{}%
    \else\setbox0=\hbox to\wd1{\unhbox0}\fi
    \global\setbox\tabsdone=\hbox{\box1\unhbox\tabsdone}\fi
  \box0}
\end{verbatim}
Es folgen die Makros f\"ur die
Aufz\"ahlungen\index{Aufzaehlung!Makros@Aufz\"ahlung!Makros}
\begin{verbatim}
\def\hang{\hangindent\parindent}
\def\item{\par\hang\textindent}
\def\itemitem{\par\indent \hangindent2\parindent \textindent}
\def\textindent#1{\indent\llap{#1\enspace}\ignorespaces}
\def\narrower{\advance\leftskip by\parindent
  \advance\rightskip by\parindent}
\end{verbatim}
Das folgende Makro dient dazu einen neuen Abschnitt in einem Dokument
einzuleiten. Es beginnt eine neue Seite, wenn die letzte nahezu voll
ist, dann wird ein `\verb|\bigskip|' eingef\"ugt, und der
Abschnittstitel in einer eigenen Zeile in Fettschrift ausgegeben.
Dieser Titel erscheint auch bei der Ausgabe auf dem \index{Bildschirm}
Bildschirm.
Au\ss{}erdem wird bei dem ersten Paragraphen keine Einr\"uckung vorgenommen.
\begin{verbatim}
\outer\def\beginsection#1\par{\vskip0pt plus.3\vsize\penalty-250
  \vskip0pt plus-.3\vsize\bigskip\vskip\parskip
  \message{#1}\leftline{\bf#1}\nobreak\smallskip\noindent}
\end{verbatim}
Zum Setzen von Lemmata, S\"atzen u.a.\ dient das folgende Makro
\begin{verbatim}
\outer\def\proclaim #1. #2\par{\medbreak
  \noindent{\bf#1.\enspace}{\sl#2\par}%
  \ifdim\lastskip<\medskipamount \removelastskip\penalty55\medskip\fi}
\end{verbatim}
Hier nun die Definition der Makros f\"ur den Flattersatz. Das zweite
Makro dient f\"ur die Ausgabe in der Teletyperschrift, da dort die
Wortabst\"ande gr\"o\ss{}er sind.
\begin{verbatim}
\def\raggedright{\rightskip=0pt plus2em
  \spaceskip=.3333em \xspaceskip=.5em\relax}
\def\ttraggedright{\tt\rightskip=0pt plus2em\relax}
\end{verbatim}
Schlie\ss{}lich die \index{Akzent!Definition}
Akzente, die ja auch via Makro realisiert werden.
\begin{verbatim}
\chardef\%=`\% \chardef\&=`\& \chardef\#=`\# \chardef\$=`\$
\chardef\ss=\"19
\chardef\ae=\"1A \chardef\oe=\"1B \chardef\o=\"1C
\chardef\AE=\"1D \chardef\OE=\"1E \chardef\O=\"1F
\chardef\i=\"10 \chardef\j=\"11 % dotless letters
\def\aa{\accent'27a} \def\l{\char'40l}
\end{verbatim}
\begin{verbatim}
\def\leavevmode{\unhbox\voidb@x} % begins a paragraph, if necessary
\def\_{\leavevmode \kern.06em \vbox{\hrule width0.3em}}
\def\L{\leavevmode\setbox0=\hbox{L}\hbox to\wd0{\hss\char'40L}}
\def\AA{\leavevmode\setbox0=\hbox{h}\dimen@=\ht0 
  \advance\dimen@ by-1ex
  \rlap{\raise.67\dimen@\hbox{\char'27}}A}
\end{verbatim}
\begin{verbatim}
\def\mathhexbox#1#2#3{\leavevmode
  \hbox{$\m@th \mathchar\"#1#2#3$}}
\def\dag{\mathhexbox279}  \def\ddag{\mathhexbox27A}
\def\S{\mathhexbox278} \def\P{\mathhexbox27B}
\end{verbatim}
\begin{verbatim}
\def\oalign#1{\leavevmode\vtop{\baselineskip0pt \lineskip.25ex
  \ialign{##\crcr#1\crcr}}} % put characters over each other
\def\ooalign{\lineskiplimit-\maxdimen \oalign}
\def\d#1{\oalign{#1\crcr\hidewidth.\hidewidth}}
\def\b#1{\oalign{#1\crcr\hidewidth
    \vbox to.2ex{\hbox{\char'26}\vss}\hidewidth}}
\def\c#1{\setbox0=\hbox{#1}\ifdim\ht0=1ex \accent'30 #1%
  \else{\ooalign{\hidewidth\char'30\hidewidth\crcr\unhbox0}}\fi}
\def\copyright{{\ooalign
    {\hfil\raise.07ex\hbox{c}\hfil\crcr\mathhexbox20D}}}
\end{verbatim}\index{Baseline!Skip}
\begin{verbatim}
\def\dots{\relax\ifmmode\ldots\else$\m@th \ldots\,$\fi}
\def\TeX{T\kern-.1667em \lower.5ex\hbox{E}\kern-.125em X}
\end{verbatim}
\begin{verbatim}
\def\`#1{{\accent\"12 #1}} \def\'#1{{\accent\"13 #1}}
\def\v#1{{\accent\"14 #1}} \def\u#1{{\accent\"15 #1}}
\def\=#1{{\accent\"16 #1}} \def\^#1{{\accent\"5E #1}}
\def\.#1{{\accent\"5F #1}} \def\H#1{{\accent\"7D #1}}
\def\~#1{{\accent\"7E #1}} \def\\"#1{{\accent\"7F #1}}
\def\t#1{{\edef\next{\the\font}\the\textfont1\accent\"7F\next#1}}
\end{verbatim}
Einige spezielle Befehle f\"ur die gl\"ucklichen Menschen, die eine
bessere Tastatur haben, als Normalsterbliche.
\begin{verbatim}
\let\^^_=\v       \let\^^S=\u       \let\^^D=\^
\end{verbatim}
Jetzt jede Menge M\"oglichkeiten Zwischenraum aufzuf\"ullen.
\begin{verbatim}
\def\hrulefill{\leaders\hrule\hfill}
\def\dotfill{\cleaders\hbox{$\m@th \mkern1.5mu . \mkern1.5mu$}\hfill}
\def\rightarrowfill{$\m@th \mathord- \mkern-6mu
  \cleaders\hbox{$\mkern-2mu \mathord- \mkern-2mu$}\hfill
  \mkern-6mu \mathord\rightarrow$}
\def\leftarrowfill
  {$\m@th \mathord\leftarrow \mkern-6mu
  \cleaders\hbox{$\mkern-2mu \mathord- \mkern-2mu$}\hfill
  \mkern-6mu \mathord-$}
\end{verbatim}
\begin{verbatim}
\mathchardef\braceld=\"37A \mathchardef\bracerd=\"37B
\mathchardef\bracelu=\"37C \mathchardef\braceru=\"37D
\def\upbracefill{$\m@th
  \bracelu\leaders\vrule\hfill\bracerd
  \braceld\leaders\vrule\hfill\braceru$}
\def\downbracefill{$\m@th
  \braceld\leaders\vrule\hfill\braceru
  \bracelu\leaders\vrule\hfill\bracerd$}
\end{verbatim}
Die Befehle `\verb|\upbracefill|' und `\verb|\downbracefill|' f\"urfen
nur innerhalb einer \index{Box!Hbox}`\verb|\hbox|' auftreten, oder in einer Tabelle.
Nun kommt noch die Definition von `\verb|\bye|'
\begin{verbatim}
\outer\def\bye{\par\vfill\supereject\end} % the recommended way to stop
\end{verbatim}
\subsection{Makros f\"ur die Mathematik}
Viele der in diesem Abschnitt vertretenen Makros dienen nur der
Darstellung von Sonderzeichen, sie werden hier nicht gesondert
aufgef\"uhrt. Zun\"achst einmal einige einfache Makros, f\"ur Leute, die
kein `\verb|_|' oder `\verb|^|' auf ihrer Tastatur haben gibt es die
Befehle `\verb|\sp|' und `\verb|\sb|'. Dann kommen noch die Makros f\"ur
die Zwischenr\"aume, und andere Kleinigkeiten
\begin{verbatim}
\let\sp=^ \let\sb=_        {\catcode`\_=\active \global\let_=\_}
\def\,{\mskip\thinmuskip}  \def\!{\mskip-\thinmuskip}
\def\>{\mskip\medmuskip}   \def\;{\mskip\thickmuskip}
\def\*{\discretionary{\thinspace\the\textfont2\char2}{}{}}
{\catcode`\^^Z=\active \gdef^^Z{\not=}} % ^^Z is like \ne in math
\end{verbatim}
\begin{verbatim}
{\catcode`\'=\active \gdef'{^\bgroup\prim@s}}
\def\prim@s{\prime\futurelet\next\pr@m@s}
\def\pr@m@s{\ifx'\next\let\nxt\pr@@@s
   \else\ifx^\next\let\nxt\pr@@@t
  \else\let\nxt\egroup\fi\fi \nxt}
\def\pr@@@s#1{\prim@s} \def\pr@@@t#1#2{#2\egroup}
\end{verbatim}
Jetzt kommen die Definitionen f\"ur griechische Buchstaben. Um die
Geschichte nicht allzu lang werden zu lassen werden Folgen von
Befehlen durch drei Punkte angezeigt.
\begin{verbatim}
\mathchardef\alpha=\"010B ... \mathchardef\omega=\"0121
\mathchardef\Gamma=\"7000 ... \mathchardef\Omega=\"700A
\mathchardef\aleph=\"0240 ... \mathchardef\spadesuit=\"027F
\def\hbar{{\mathchar'26\mkern-9muh}}
\def\surd{{\mathchar\"1270}}
\def\angle{{\vbox{\ialign{$\m@th\scriptstyle##$\crcr
      \not\mathrel{\mkern14mu}\crcr \noalign{\nointerlineskip}
      \mkern2.5mu\leaders\hrule height.34pt\hfill\mkern2.5mu\crcr}}}}
\end{verbatim}
Nun die gro\ss{}en Symbole
\begin{verbatim}
\mathchardef\smallint=\"1273
\mathchardef\sum=\"1350 ... \mathchardef\biguplus=\"1355
\mathchardef\intop=\"1352  \def\int{\intop\nolimits}
\mathchardef\ointop=\"1348 \def\oint{\ointop\nolimits}
\end{verbatim}
Auch nichts verwunderliches bei den bin\"aren Operatoren
\begin{verbatim}
\mathchardef\pm=\"2206 ... \mathchardef\amalg=\"2271
\end{verbatim}
Auch die Relationen werden einfach linear definiert, mit Ausnahme der
Relationen, die aus verschiedenen Zeichen zusammengesetzt sind. Dazu
geh\"ort `\verb|\mapsto|' und `\verb|\longmapsto|'
\begin{verbatim}
\mathchardef\leq=\"3214 ... \mathchardef\perp=\"323F
\def\joinrel{\mathrel{\mkern-3mu}}
\def\relbar{\mathrel{\smash-}} \def\Relbar{\mathrel=}
\def\longrightarrow{\relbar\joinrel\rightarrow}
\def\Longrightarrow{\Relbar\joinrel\Rightarrow}
\def\longleftarrow{\leftarrow\joinrel\relbar}
\def\Longleftarrow{\Leftarrow\joinrel\Relbar}
\def\longleftrightarrow{\leftarrow\joinrel\rightarrow}
\def\Longleftrightarrow{\Leftarrow\joinrel\Rightarrow}
\mathchardef\mapstochar=\"322F \def\mapsto{\mapstochar\rightarrow}
\def\longmapsto{\mapstochar\longrightarrow}
\mathchardef\lhook=\"312C \def\hookrightarrow{\lhook\joinrel\rightarrow}
\mathchardef\rhook=\"312D \def\hookleftarrow{\leftarrow\joinrel\rhook}
\end{verbatim}
\begin{verbatim}
\def\neq{\not=}   \def\models{\mathrel|\joinrel=}
\def\bowtie{\mathrel\triangleright\joinrel\mathrel\triangleleft}
\end{verbatim}
Nach der Definition von `\verb|\ldotp|' und `\verb|\cdotp|' ist die
weitere Definition von den im mathematischen Modus gebr\"auchlichen
Punkten keine Schwierigkeit mehr.
\begin{verbatim}
\mathchardef\ldotp=\"613A\mathchardef\cdotp=\"6201\mathchardef\colon=\"603A
\def\ldots{\mathinner{\ldotp\ldotp\ldotp}}
\def\cdots{\mathinner{\cdotp\cdotp\cdotp}}
\def\vdots{\vbox{\baselineskip=4pt \lineskiplimit=0pt
    \kern6pt \hbox{.}\hbox{.}\hbox{.}}}
\def\ddots{\mathinner{\mkern1mu\raise7pt\vbox{\kern7pt\hbox{.}}\mkern2mu
    \raise4pt\hbox{.}\mkern2mu\raise1pt\hbox{.}\mkern1mu}}
\end{verbatim}\index{Baseline!Skip}
Die meisten mathematischen \index{Akzent!mathematische}
Akzente werden mit `\verb|\mathaccent|'
realisiert, die mit variabler Breite m\"ussen einzeln bereitgestellt
werden
\begin{verbatim}
\def\acute{\mathaccent\"7013 } ... \def\ddot{\mathaccent\"707F }
\def\widetilde{\mathaccent\"0365 } \def\widehat{\mathaccent\"0362 }
\def\overrightarrow#1{\vbox{\ialign{##\crcr
     \rightarrowfill\crcr\noalign{\kern-1pt\nointerlineskip}
     $\hfil\displaystyle{#1}\hfil$\crcr}}}
\def\overleftarrow#1{\vbox{\ialign{##\crcr
     \leftarrowfill\crcr\noalign{\kern-1pt\nointerlineskip}
     $\hfil\displaystyle{#1}\hfil$\crcr}}}
\def\overbrace#1{\mathop{\vbox{\ialign{##\crcr\noalign{\kern3pt}
     \downbracefill\crcr\noalign{\kern3pt\nointerlineskip}
     $\hfil\displaystyle{#1}\hfil$\crcr}}}\limits}
\def\underbrace#1{\mathop{\vtop{\ialign{##\crcr
     $\hfil\displaystyle{#1}\hfil$\crcr \noalign{\kern3pt\nointerlineskip}
     \upbracefill\crcr\noalign{\kern3pt}}}}\limits}
\def\skew#1#2#3{{#2{#3\mkern#1mu}\mkern-#1mu}{}}
\end{verbatim}
Nun die 24 \index{Klammer}Klammern, die in der Gr\"o\ss{}e variabel sind
\begin{verbatim}
\def\langle{\delimiter\"426830A }      \def\rangle{\delimiter\"526930B }
\def\lbrace{\delimiter\"4266308 }      \def\rbrace{\delimiter\"5267309 }
\def\lceil{\delimiter\"4264306 }       \def\rceil{\delimiter\"5265307 }
\def\lfloor{\delimiter\"4262304 }      \def\rfloor{\delimiter\"5263305 }
\def\lgroup{\delimiter\"400033A }      \def\rgroup{\delimiter\"500033B }
\def\lmoustache{\delimiter\"4000340 }  \def\rmoustache{\delimiter\"5000341 }
\def\uparrow{\delimiter\"3222378 }     \def\Uparrow{\delimiter\"322A37E }
\def\downarrow{\delimiter\"3223379 }   \def\Downarrow{\delimiter\"322B37F }
\def\updownarrow{\delimiter\"326C33F } \def\arrowvert{\delimiter\"033C000 }
\def\Updownarrow{\delimiter\"326D377 } \def\Arrowvert{\delimiter\"033D000 }  
\def\vert{\delimiter\"026A30C }        \def\Vert{\delimiter\"026B30D }
\def\backslash{\delimiter\"026E30F }   \def\bracevert{\delimiter\"033E000 }
\end{verbatim}
Jetzt k\"onnen auch die `\verb|\big...|' Befehle bereitgestellt werden
\begin{verbatim}
\def\bigl{\mathopen\big} \def\bigm{\mathrel\big} \def\bigr{\mathclose\big}
\def\Bigl{\mathopen\Big} \def\Bigm{\mathrel\Big} \def\Bigr{\mathclose\Big}
\def\biggl{\mathopen\bigg}            \def\Biggl{\mathopen\Bigg}
\def\biggm{\mathrel\bigg}             \def\Biggm{\mathrel\Bigg}
\def\biggr{\mathclose\bigg}           \def\Biggr{\mathclose\Bigg}
\def\big#1{{\hbox{$\left#1\vbox to 8.5pt{}\right.\n@space$}}}
\def\Big#1{{\hbox{$\left#1\vbox to 11.5pt{}\right.\n@space$}}}
\def\bigg#1{{\hbox{$\left#1\vbox to 14.5pt{}\right.\n@space$}}}
\def\Bigg#1{{\hbox{$\left#1\vbox to 17.5pt{}\right.\n@space$}}}
\def\n@space{\nulldelimiterspace=0pt \m@th}
\end{verbatim}
Und jetzt noch ein paar \index{Abkuerzung@Abk\"urzung}
Abk\"urzungen, die sich auf
\index{Klammer!Abkuerzungen@Klammer!Abk\"urzung}Klammern beziehen
\begin{verbatim}
\def\choose{\atopwithdelims()}
\def\brack{\atopwithdelims[]}
\def\brace{\atopwithdelims\{\}}
\def\sqrt{\radical\"270370 }
\end{verbatim}
Nun wird es wieder etwas interessanter. Der `\verb|\mathpalette|'
Befehl reagiert unterschiedlich, je nach Stil, in dem er aufgerufen
wird. Er ist ma\ss{}geblich f\"ur weitere Befehle. Diese Definition kann
auch als Muster daf\"ur dienen, wie andere Definitionen auf die
verschiedenen Stile reagieren kann.
\begin{verbatim}
\def\mathpalette#1#2{\mathchoice{#1\displaystyle{#2}}
  {#1\textstyle{#2}}{#1\scriptstyle{#2}}{#1\scriptscriptstyle{#2}}}
\end{verbatim}
\begin{verbatim}
\newbox\rootbox
\def\root#1\of{\setbox\rootbox=
  \hbox{$\m@th \scriptscriptstyle{#1}$}
  \mathpalette\r@@t}
\def\r@@t#1#2{\setbox0=\hbox{$\m@th #1\sqrt{#2}$}
  \dimen@=\ht0 \advance\dimen@ by-\dp0
  \mkern5mu \raise.6\dimen@\copy\rootbox \mkern-10mu \box0}
\end{verbatim}
\begin{verbatim}
\newif\ifv@ \newif\ifh@
\def\vphantom{\v@true\h@false\ph@nt}
\def\hphantom{\v@false\h@true\ph@nt}
\def\phantom{\v@true\h@true\ph@nt}
\def\ph@nt{\ifmmode\def\next{\mathpalette\mathph@nt}%
  \else\let\next=\makeph@nt\fi \next}
\def\makeph@nt#1{\setbox0=\hbox{#1}\finph@nt}
\def\mathph@nt#1#2{\setbox0=\hbox{$\m@th#1{#2}$}\finph@nt}
\def\finph@nt{\setbox2=\null \ifv@ \ht2=\ht0 \dp2=\dp0 \fi
  \ifh@ \wd2=\wd0 \fi \box2 }
\def\mathstrut{\vphantom(}
\end{verbatim}
\begin{verbatim}
\def\smash{\relax % \relax, in case this comes first in \halign
  \ifmmode\def\next{\mathpalette\mathsm@sh}\else\let\next\makesm@sh
  \fi \next}
\def\makesm@sh#1{\setbox0=\hbox{#1}\finsm@sh}
\def\mathsm@sh#1#2{\setbox0=\hbox{$\m@th#1{#2}$}\finsm@sh}
\def\finsm@sh{\ht0=0pt \dp0=0pt \box0 }
\end{verbatim}
\begin{verbatim}
\def\cong{\mathrel{\mathpalette\@vereq\sim}} % \sim over =
\def\@vereq#1#2{\lower.5pt\vbox{\baselineskip0pt \lineskip-.5pt
    \ialign{$\m@th#1\hfil##\hfil$\crcr#2\crcr=\crcr}}}
\def\notin{\mathrel{\mathpalette\c@ncel\in}}
\def\c@ncel#1#2{\ooalign{$\hfil#1\mkern1mu/\hfil$\crcr$#1#2$}}
\def\rightleftharpoons{\mathrel{\mathpalette\rlh@{}}}
\def\rlh@#1{\vcenter{\hbox{\ooalign{\raise2pt
          \hbox{$#1\rightharpoonup$}\crcr $#1\leftharpoondown$}}}}
\def\buildrel#1\over#2{\mathrel{\mathop{\kern0pt #2}\limits^{#1}}}
\def\doteq{\buildrel\textstyle.\over=}
\end{verbatim}\index{Baseline!Skip}
Hier einige Definitionen f\"ur alternative Namen
\begin{verbatim}
\let\ne=\neq        \let\le=\leq         \let\ge=\geq
\let\{=\lbrace      \let\|=\Vert         \let\}=\rbrace
\let\to=\rightarrow \let\gets=\leftarrow \let\owns=\ni
\let\land=\wedge    \let\lor=\vee        \let\lnot=\neg
\def\iff{\;\Longleftrightarrow\;}
\end{verbatim}
Die meisten Funktionsnamen, die in der Schriftart Roman gesetzt werden
sind schon bekannt, hier folgen nur einige, die noch nicht vorgestellt
wurden
\begin{verbatim}
\def\arccos{\mathop{\rm arccos}\nolimits}
        ... \def\tanh{\mathop{\rm tanh}\nolimits}
\def\det{\mathop{\rm det}} ... \def\sup{\mathop{\rm sup}}
\def\liminf{\mathop{\rm lim\,inf}} \def\limsup{\mathop{\rm lim\,sup}}
\end{verbatim}
\begin{verbatim}
\def\bmod{\mskip-\medmuskip \mkern5mu
  \mathbin{\rm mod} \penalty900 \mkern5mu \mskip-\medmuskip}
\def\pmod#1{\allowbreak \mkern18mu ({\rm mod}\,\,#1)}
\end{verbatim}
Die Definitionen von `\verb|\matrix|' und `\verb|\bordermatrix|'
werden hier bereitgestellt.
\begin{verbatim}
\def\matrix#1{\null\,\vcenter{\normalbaselines\m@th
    \ialign{\hfil$##$\hfil&&\quad\hfil$##$\hfil\crcr
      \mathstrut\crcr\noalign{\kern-\baselineskip}
      #1\crcr\mathstrut\crcr\noalign{\kern-\baselineskip}}}\,}
\end{verbatim}\index{Baseline!Normal}
\begin{verbatim}
\newdimen\p@renwd \setbox0=\hbox{\tenex B} \p@renwd=\wd0
\def\bordermatrix#1{\begingroup \m@th
  \setbox0=\vbox{\def\cr{\crcr\noalign{\kern2pt\global\let\cr=\endline}}
      \ialign{$##$\hfil\kern2pt\kern\p@renwd&\thinspace\hfil$##$\hfil
        &&\quad\hfil$##$\hfil\crcr
        \omit\strut\hfil\crcr\noalign{\kern-\baselineskip}
        #1\crcr\omit\strut\cr}}
  \setbox2=\vbox{\unvcopy0 \global\setbox1=\lastbox}
  \setbox2=\hbox{\unhbox1 \unskip \global\setbox1=\lastbox}
  \setbox2=\hbox{$\kern\wd1\kern-\p@renwd \left( \kern-\wd1
    \global\setbox1=\vbox{\box1\kern2pt}
    \vcenter{\kern-\ht1 \unvbox0 \kern-\baselineskip} \,\right)$}
  \null\;\vbox{\kern\ht1\box2}\endgroup}
\end{verbatim}\index{Baseline!Skip}
Die n\"achsten Definitionen sind wieder etwas einfacher
\begin{verbatim}
\def\cases#1{\left\{\,\vcenter{\normalbaselines\m@th
    \ialign{$##\hfil$&\quad##\hfil\crcr#1\crcr}}\right.}
\def\pmatrix#1{\left( \matrix{#1} \right)}
\end{verbatim}
Zum guten Schlu\ss{} noch die Makros f\"ur die \index{Formel!abgesetzte}
abgesetzten Formeln
\begin{verbatim}
\def\openup{\afterassignment\@penup\dimen@=}
\def\@penup{\advance\lineskip\dimen@
  \advance\baselineskip\dimen@ \advance\lineskiplimit\dimen@}
\def\eqalign#1{\null\,\vcenter{\openup1\jot \m@th
  \ialign{\strut\hfil$\displaystyle{##}$&$\displaystyle{{}##}$\hfil
      \crcr#1\crcr}}\,}
\end{verbatim}
\begin{verbatim}
\newif\ifdt@p
\def\displ@y{\global\dt@ptrue \openup1\jot \m@th
  \everycr{\noalign{\ifdt@p \global\dt@pfalse
      \vskip-\lineskiplimit \vskip\normallineskiplimit
      \else \penalty\interdisplaylinepenalty \fi}}}
\def\@lign{\tabskip=0pt\everycr={}} % restore inside \displ@y
\def\displaylines#1{\displ@y
  \halign{\hbox to\displaywidth{$\hfil\@lign\displaystyle##\hfil$}\crcr
    #1\crcr}}
\end{verbatim}
\begin{verbatim}
\def\eqalignno#1{\displ@y \tabskip=\centering
  \halign to\displaywidth{\hfil$\@lign\displaystyle{##}$\tabskip=0pt
    &$\@lign\displaystyle{{}##}$\hfil\tabskip=\centering
    &\llap{$\@lign##$}\tabskip=0pt\crcr
    #1\crcr}}
\def\leqalignno#1{\displ@y \tabskip=\centering
  \halign to\displaywidth{\hfil$\@lign\displaystyle{##}$\tabskip=0pt
    &$\@lign\displaystyle{{}##}$\hfil\tabskip=\centering
    &\kern-\displaywidth\rlap{$\@lign##$}\tabskip=\displaywidth\crcr
    #1\crcr}}
\end{verbatim}
Der Wert von `\verb|\lineskiplimt|' wird dabei als
`\verb|\normallineskiplimit|' angenommen, nur ver\"an\-dert durch die
``\"Offnung''.

\subsection{Makros f\"ur die Ausgaberoutine}
Zuerst werden einige Dinge bzgl.\ Seitennummern, Kopf- und Fu\ss{}zeilen
geregelt
\begin{verbatim}
\countdef\pageno=0 \pageno=1 % first page is number 1
\newtoks\headline \headline={\hfil} % headline is normally blank
\newtoks\footline \footline={\hss\tenrm\folio\hss}
  % footline is normally a centered page number in font \tenrm
\def\folio{\ifnum\pageno<0 \romannumeral-\pageno \else\number\pageno \fi}
\def\nopagenumbers{\footline={\hfil}} % blank out the footline
\def\advancepageno{\ifnum\pageno<0 \global\advance\pageno by -1
  \else\global\advance\pageno by 1 \fi} % increase |pageno|
\end{verbatim}
\begin{verbatim}
\newif\ifr@ggedbottom
\def\raggedbottom{\topskip10pt plus60pt \r@ggedbottomtrue}
\def\normalbottom{\topskip10pt \r@ggedbottomfalse} % undoes \raggedbottom
\end{verbatim}
Das Fu\ss{}noten Makro benutzt
`\verb|\futurelet|' und `\verb|\aftergroup|', damit die Fu\ss{}note nicht
als Argument an ein anderes Makro \"ubergeben werden mu\ss{}.
\begin{verbatim}
\newinsert\footins
\def\footnote#1{\let\@sf=\empty % parameter #2 (the text) is read later
  \ifhmode\edef\@sf{\spacefactor=\the\spacefactor}\/\fi
  #1\@sf\vfootnote{#1}}
\def\vfootnote#1{\insert\footins\bgroup
  \interlinepenalty=\interfootnotelinepenalty
  \splittopskip=\ht\strutbox % top baseline for broken footnotes
  \splitmaxdepth=\dp\strutbox \floatingpenalty=20000
  \leftskip=0pt \rightskip=0pt \spaceskip=0pt \xspaceskip=0pt
  \textindent{#1}\footstrut\futurelet\next\fo@t}
\def\fo@t{\ifcat\bgroup\noexpand\next \let\next\f@@t
  \else\let\next\f@t\fi \next}
\def\f@@t{\bgroup\aftergroup\@foot\let\next}
\def\f@t#1{#1\@foot}
\def\@foot{\strut\egroup}
\def\footstrut{\vbox to\splittopskip{}}
\skip\footins=\bigskipamount % space added when footnote is present
\count\footins=1000 % footnote magnification factor (1 to 1)
\dimen\footins=8in % maximum footnotes per page
\end{verbatim}
Hier steht nun, wie {\em flie\ss{}ende} Einf\"ugungen realisiert werden
\begin{verbatim}
\newinsert\topins \newif\ifp@ge \newif\if@mid
\def\topinsert{\@midfalse\p@gefalse\@ins}
\def\midinsert{\@midtrue\@ins}
\def\pageinsert{\@midfalse\p@getrue\@ins}
\skip\topins=0pt % no space added when a topinsert is present
\count\topins=1000 % magnification factor (1 to 1)
\dimen\topins=\maxdimen % no limit per page
\def\@ins{\par\begingroup\setbox0=\vbox\bgroup} % start a \vbox
\def\endinsert{\egroup % finish the \vbox
  \if@mid \dimen@=\ht0 \advance\dimen@ by\dp\z@ \advance\dimen@ by12\p@
    \advance\dimen@ by\pagetotal \advance\dimen@ by-\pageshrink
    \ifdim\dimen@>\pagegoal \@midfalse\p@gefalse\fi\fi
  \if@mid \bigskip \box0 \bigbreak
  \else\insert\topins{\penalty100 % floating insertion
    \splittopskip=0pt \splitmaxdepth=\maxdimen \floatingpenalty=0
    \ifp@ge \dimen@=\dp0
     \vbox to\vsize{\unvbox0 \kern-\dimen@} % depth is zero
    \else \box0 \nobreak\bigskip\fi}\fi\endgroup}
\end{verbatim}
Hier die vollst\"andige Ausgaberoutine
\begin{verbatim}
\output={\plainoutput}
\def\plainoutput{\shipout\vbox{\makeheadline\pagebody\makefootline}%
  \advancepageno
  \ifnum\outputpenalty>-20000 \else\dosupereject\fi}
\def\pagebody{\vbox to\vsize{\boxmaxdepth=\maxdepth \pagecontents}}
\def\makeheadline{\vbox to 0pt{\vskip-22.5pt
    \line{\vbox to8.5pt{}\the\headline}\vss}\nointerlineskip}
\def\makefootline{\baselineskip=24pt \line{\the\footline}}
\def\dosupereject{\ifnum\insertpenalties>0 % something is being held over
  \line{}\kern-\topskip\nobreak\vfill\supereject\fi}
\end{verbatim}\index{Baseline!Skip}
\begin{verbatim}
\def\pagecontents{\ifvoid\topins\else\unvbox\topins\fi
  \dimen@=\dp255 \unvbox255
  \ifvoid\footins\else % footnote info is present
    \vskip\skip\footins \footnoterule \unvbox\footins\fi
  \ifr@ggedbottom \kern-\dimen@ \vfil \fi}
\def\footnoterule{\kern-3pt
  \hrule width 2truein \kern 2.6pt} % the \hrule is .4pt high
\end{verbatim}
\subsection{Trennung und dergleichen}
Der letzte Teil von {\tt plain.tex} lie\ss{}t die Trennungsmuster aus dem
File {\tt hyphen.tex} und auch die Ausnahmen
\begin{verbatim}
\lefthyphenmin=2 \righthyphenmin=3 % disallow x- or -xx breaks
\input hyphen % the hyphenation patterns and exceptions
\end{verbatim}
\begin{verbatim}
\def\magnification{\afterassignment\m@g\count@}
\def\m@g{\mag=\count@
  \hsize6.5truein\vsize8.9truein\dimen\footins8truein}
\end{verbatim}
\begin{verbatim}
\def\tracingall{\tracingonline=1 \tracingcommands=2 \tracingstats=2
  \tracingpages=1 \tracingoutput=1 \tracinglostchars=1
  \tracingmacros=2 \tracingparagraphs=1 \tracingrestores=1 
  \showboxbreadth=\maxdimen \showboxdepth=\maxdimen \errorstopmode}
\end{verbatim}
\begin{verbatim}
\def\showhyphens#1{\setbox0=\vbox{\parfillskip0pt \hsize=\maxdimen \tenrm
  \pretolerance=-1 \tolerance=-1 \hbadness=0 \showboxdepth=0 \ #1}}
\end{verbatim}\index{Badness}
\begin{verbatim}
\normalbaselines\rm % select roman font
\nonfrenchspacing % punctuation affects the spacing
\catcode`@=12 % at signs are no longer letters
\end{verbatim}\index{Baseline!Normal}
\begin{verbatim}
\def\fmtname{plain}\def\fmtversion{3.0} % identifies the current format
\end{verbatim}
\chapter{Trennungen}
Es ist sicher besser ein Wort zu trennen, als die \index{Zwischenraum}
Zwischenr\"aume
zwischen einzelnen W\"ortern zu weit zu strecken. Computer sind
normalerweise nicht sehr gut bei der Trennung von Worten. Als die
ersten automatischen Trennungen in Zeitungen auftauchten, machten auch
schnell die ersten Witze die Runde.\footnote{Im Gegensatz zu den
englischen Beispielen aus dem \TeX book erinnere ich hier an die
unvergessenen Beispiele aus der CONTEXT-Dokumentation, wie Klo-ster,
oder noch besser Urin-stinkt. Im weiteren werde ich mich aber an die
Originalbeispiele halten.}

Woher die Schwierigkeiten kommen d\"urfte klar sein. So wird z.B.\ das
Wort `record' einmal als `re-cord' und einmal als `rec-ord' getrennt,
je nachdem, wie es verwendet wird. Auch das englische Wort f\"ur
Trennung: `hy-phen-a-tion' ist nicht ganz einfach zu trennen, so wird
bei dem \"ahnlichen Wort `con-cat-e-na-tion' das `n' mit zu dem `a'
gezogen, im Gegensatz zu dem ersten Beispiel.

Eine gute L\"osung f\"ur das Trennungproblem fand \index{Liang}
Frank~M. Liang zwischen
1980 und 1982 und \TeX\ \"ubernahm die Methode. Diese Methode findet zu
fast allen W\"ortern die legitimen Trennstellen, macht nur selten
\index{Fehler!Bei Trennungen}Fehler, ist schnell, verbraucht nur wenig
\index{Speicherplatz!Trennung}Speicherplatz und ist
flexibel genug um schnell auf andere Sprachen angewendet zu werden,
ja, sie kann sogar simultan auf zwei Sprachen angewendet werden.

Um ein Wort zu trennen schaut \TeX\ zuerst in ein
\index{Register!Ausnahme}Ausnahmenregister,
um das Wort eventuell dort zu finden. Wenn das Wort dort nicht
vorkommt, dann sucht \TeX\ nach \index{Muster!Trennung}
Mustern in dem Wort, und das ist der
Kernpunkt der Methode von \index{Liang}Liang. Hier nun die Vorgehensweise am
Beispiel des Wortes \index{zzzhyphenation@{\tt\char92 hyphenation}}
`hyphenation', vorausgesetzt, \TeX\ arbeitet mit
den englischen Trennmustern.

Zun\"achst wird dem Wort ein Zeichen vor- und eins nachgestellt, die als
\index{Markierung!Trennung}Markierung dienen sollen.
\index{zzzhyphenation@{\tt\char92 hyphenation}}
\begin{verbatim}
.hyphenation.
\end{verbatim}
Hier bei wurde das `\verb|.|' als Markierung verwendet. Dieses Wort
hat die {\em Unter}w\"orter
\begin{verbatim}
. h y p h e n a t i o n .
\end{verbatim}
der L\"ange eins, und die Unterw\"orter der L\"ange zwei
\begin{verbatim}
.h hy yp ph he en na at ti io on n.
\end{verbatim}
und schlie\ss{}lich die der L\"ange drei
\begin{verbatim}
.hy hyp yph phe hen ena nat ati tio ion on.
\end{verbatim}
und so weiter. Jedes Unterwort der L\"ange $k$ definiert $k+1$ kleine
Zahlwerte, die die F\"ahigkeit zur Trennung zwischen zwei Buchstaben
ausdr\"ucken. In den folgenden Beispielen werden diese Wert als Indizes
angegeben.
\newcommand{\ti}[1]{${}_#1$}

`\ti0 h\ti0 e\ti2 n\ti0' meint z.B., da\ss{} bei dem Unterwort `\verb|hen|'
alle Werte Nul sind mit der Ausnahme zwischen den Buchstaben `\verb|e|'
und `\verb|n|', deren Wert zwei ist. Die Werte sind alle Null, es sei
denn \TeX\ findet in seiner \index{Muster!Trennung}
Musterbibliotek ein Muster. Bei unserem
Beispiel werden die folgenden Muster gefunden

\noindent
\ti0 h\ti0 y\ti3 p\ti0 h\ti0\\
\ti0 h\ti0 e\ti2 n\ti0\\
\ti0 h\ti0 e\ti0 n\ti0 a\ti4\\
\ti0 h\ti0 e\ti0 n\ti5 a\ti0 t\ti0\\
\ti1 n\ti0 a\ti0\\
\ti0 n\ti2 a\ti0 t\ti0\\
\ti1 t\ti0 i\ti0 o\ti0\\
\ti2 i\ti0 o\ti0\\
\ti0 o\ti2 n\ti0

\TeX\ berechnet nun das Maximum an Werten, das zwischen zwei
Buchstaben vorkommt. Zwischen `\verb|e|' und `\verb|n|' treten z.B.\
vier signifikante Werte auf. Das Ergebnis dieser Berechnung ist

.\ti0 h\ti0 y\ti3 p\ti0 h\ti0 e\ti2 n\ti5 a\ti4 t\ti2 i\ti0 o\ti2 n\ti0 .

Jetzt kommt der letzte Schritt. Eine Trennung ist akzeptabel an einer
Stelle, an der ein ungerader Wert berechnet wurde. Somit ergeben sich
zwei m\"ogliche Trennungen: `\verb|hy-phen-ation|'. Das Wort
`\verb|concatenation|' ergibt die Muster

\noindent
\ti0 o\ti2 n\ti0\\
\ti0 o\ti0 n\ti1 c\ti0\\
\ti1 c\ti0 a\ti0\\
\ti1 n\ti0 a\ti0\\
\ti0 n\ti2 a\ti0 t\ti0\\
\ti1 t\ti0 i\ti0 o\ti0\\
\ti2 i\ti0 o\ti0\\
\ti0 o\ti2 n\ti0

und das Ergebnis

\ti0 c\ti0 o\ti2 n\ti1 c\ti0 a\ti0 t\ti0 e\ti1 n\ti2 a\ti1 t\ti2 i\ti0
o\ti2 n\ti0

und damit die m\"oglichen Trennungen `\verb|con-cate-na-tion|'. Weitere
Worte, wie das Beispiel aus dem \TeX book
\index{Supercalifragilisticexpialidocious}
`supercalifragilisticexpialidocious' erspare ich mir an dieser Stelle.

{\em Plain} \TeX\ l\"adt 4447 \index{Muster!Trennung}
Muster in seinen Speicher. Angefangen mit
`\ti0 .\ti0 a\ti0 c\ti0 h\ti4' bis `\ti4 z\ti1 z\ti2' und
`\ti0 z\ti4 z\ti0 y\ti0'. Die Werte zwischen den Buchstaben haben
dabei die Werte 0 bis 5. Ein gro\ss{}er ungerader Wert, wie 5 f\"ordert
nat\"urlich eine Trennung an dieser Stelle, ein gro\ss{}er gerader Wert,
wie 4 verhindert praktisch eine Trennung an dieser Stelle.

\index{Liang}Liang fand diese Muster, indem er ein spezielles W\"orterbuch
bearbeitete und auch solche Ausnahmen wie `Af-ghan-i-stan' mit
aufnahm. Diese Trennmuster reichen aus f\"ur nahezu alle vorkommenden
Trennungen, solange es sich um englischsprachige W\"orter handelt. F\"ur
andere Sprachen mu\ss{} nat\"urlich eine eigene \index{Bibliothek}
Bibliothek von W\"ortern
gefunden werden. Normalerweise findet diese Methode nur g\"ultige
Trennungen, es werden aber nicht unbedingt alle m\"oglichen Trennungen
gefunden, wie man auch an den obigen Beispielen sehen kann.

In seltenen F\"allen kann es nat\"urlich vorkommen, da\ss{} \TeX\ ein Wort
nicht richtig trennt. Dann haben sie noch die M\"oglichkeit die Trennung
explizit anzugeben. Sie schaffen damit einen weiteren Ausnahmeeintrag,
der oben schon angesprochen wurde. Das \TeX book wurde mit den
Ausnahmen
\index{zzzhyphenation@{\tt\char92 hyphenation}}
\begin{verbatim}
\hyphenation{man-u-script man-u-scripts ap-pen-dix}
\end{verbatim}
geschrieben, um auch die Worte `manuscript', `manuscripts' und
`appendix' richtig zu trennen. Sie sehen, da\ss{} auch der Plural des
Wortes angegeben ist. Das h\"angt damit zusammen, da\ss{} das Wort
vollst\"andig in der Ausnahmeliste gefunden werden mu\ss{}. Es handelt sich
also tats\"achlich um zwei Eintr\"age.

Wenn sie alle Trennungen sehen wollen, die \TeX\ bei einem Text
findet, dann k\"onnen sie mit der Eingabe
\index{zzzshowhyphens@{\tt\char92 showhyphens}}
\begin{verbatim}
\showhyphens{Text}
\end{verbatim}
eine \index{Box!Hbox}
hbox ausgeben lassen, die alle W\"orter des Textes in getrennter
Form enth\"alt. Diese Box ist allerdings immer zu klein, so da\ss{} sie
immer eine \index{Fehler!Meldung}
Fehlermeldung erhalten werden, k\"ummern sie sich nicht darum.

Um ein Wort in die Ausnahmeliste aufzunehmen mu\ss{} der Befehl
\index{zzzhyphenation@{\tt\char92 hyphenation}}
\begin{verbatim}
\hyphenation{<Worte>}
\end{verbatim}
am besten zu Beginn des Textes angegeben werden. `\verb|<Worte>|' ist
dabei eine Liste von Worten, die durch \index{Leer!Zeichen}
Leerzeichen getrent sind. Diese
Worte enthalten das Zeichen `\verb|-|' als Trennzeichen an den
Stellen, an denen es getrennt werden darf. \TeX\ wandelt alle Worte
zun\"achst in \index{Kleinbuchstabe}
Kleinbuchstaben um, bevor die \index{Gleichheit!Von
Buchstaben}Gleichheit mit einem
aufgefundenen Wort gepr\"uft wird. Trennungen werden nicht nach dem
ersten Buchstaben und vor dem letzten Buchstaben akzeptiert. Ein
Trennungseintrag kann auch keine Trennung enthalten, dann wird \TeX\
das Wort \"uberhaupt nicht mehr trennen.

Die Eintr\"age in die Ausnahmeliste sind \index{Global!Trennung}
global, sie verschwinden nicht
am Ende einer Gruppe. Wir ein Wort mehrfach angegeben, dann gilt immer
die letzte Angabe.

V\"ollig analog funktioniert die Angabe der \index{Muster!Trennung}
Trennmuster, allerdings mit
einem schwerwiegenden Unterschied. Der Befehl
\index{zzzpatterns@{\tt\char92 patterns}}
\begin{verbatim}
\patterns{<Muster>}
\end{verbatim}
darf nur f\"ur INITEX auftreten. Beim Lauf von \TeX\ k\"onnen die Muster
nicht mehr ge\"andert werden.
\end{appendix}
% \begin{theindex}
\printindex
% \end{theindex}
\end{document}
