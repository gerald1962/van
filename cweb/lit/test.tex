\input cwebmac

\M{1}.
Each language has its own grammar and it has undefined symbols like integer,
identifier, string.

The grammar has undefined terminal symbols integer-constant,
character-constant, floating-
constant, identifier, string, and enumeration-constant; the typewriter style
words and
symbols are terminals given literally. This grammar can be transformed
mechanically into
input acceptable for an automatic parser-generator. Besides adding whatever
syntactic
marking is used to indicate alternatives in productions, it is necessary to
expand the ``one of''
constructions, and (depending on the rules of the parser-generator) to
duplicate each
production with an opt symbol, once with the symbol and once without. With one
further
change, namely deleting the production typedef-name: identifier and making
typedef-name a
terminal symbol, this grammar is acceptable to the YACC parser-generator. It
has only one
conflict, generated by the if-else ambiguity.

A function description has this form:

The decimal numbers assigned to the individual
propositions indicate the logical importance of the propo-
sitions, the stress laid on them in my exposition. The
propositions n.1, n.2, n.3, etc. are comments on propo-
sition no. n ; the propositions n.m 1, n.m 2, etc. are com-
ments on proposition no. n.m ; and so on.

\Y\B\&{void} \|x(\,);\par
\fi

\N{1}{2}Index.
\fi

\inx
\fin
\con
