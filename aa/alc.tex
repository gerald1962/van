% Precondition:
% cd /home/gerald/van_development/van/aa
% cp /home/gerald/van_development/van/edu/latex/emoji.sty 
% 
% Build rules:
% pdflatex -shell-escape alc.tex
% evince alc.pdf

% This is the preamble of a LaTeX source code.
\documentclass[10pt,a4paper]{article}

% Import a package for a specific purpose.
\usepackage[textwidth=17cm,textheight=26cm]{geometry}  % Customize the page layou.
\usepackage{tikzsymbols}  % Various emoticon, cooking symbols and trees.
\usepackage{enumitem}     % Control layout of itemize, enumerate, description.
\usepackage{wasysym}      % Provide symbols like \Xbox.
\usepackage{scrextend}    % KOMA-Script class: labeling 

% Initial distance of a paragraphs like enumerate
\setlength{\parskip}{-4pt}%

% All paragraphs are without indentation.
\setlength{\parindent}{0pt}%

% Introduce a paragraph section with a number.
\renewcommand{\theparagraph}{\arabic{paragraph}.}

% Force the numbering of a paragraph text.
\setcounter{secnumdepth}{4}

% Default settings for a paragraph segment.
\newcommand\p[1] {%
  \stepcounter{paragraph}
  {\large\bf\theparagraph \hskip 2pt \large\bf #1 \ \hskip -8pt}
}

% Subparagraph counters.
\newcounter{subc}
\renewcommand{\thesubc}{\bf\arabic{subc}.}

\newcounter{subsubc}
\renewcommand{\thesubsubc}{\bf\arabic{subsubc}.}

% Settings for a subparagraph segment.
\newcommand\subp[1] {%
  \stepcounter{subc}
  {\bf\theparagraph\thesubc}\hskip 3pt\rm #1\hskip 4pt
}

\newcommand\subsubp[1] {%
  \stepcounter{subsubc}
  {\bf\theparagraph\thesubc\thesubsubc}\hskip 3pt\rm #1\hskip 4pt
}

% Elements of a title page.
\newcommand\svthema{Therapie}
\newcommand\svperson{Gerald Schüller}
\newcommand\svdatum{\today}


% Color list:
\definecolor{alizarin}{rgb}{0.82, 0.1, 0.26}
\definecolor{amber(sae/ece)}{rgb}{1.0, 0.49, 0.0}
\definecolor{amethyst}{rgb}{0.6, 0.4, 0.8}
\definecolor{aqua}{rgb}{0.0, 1.0, 1.0}
\definecolor{burntorange}{rgb}{0.8, 0.33, 0.0}
\definecolor{darkbrown}{rgb}{0.4, 0.26, 0.13}
\definecolor{english}{rgb}{0.0, 0.5, 0.0}
\definecolor{indigo(web)}{rgb}{0.29, 0.0, 0.51}
\definecolor{midnightblue}{rgb}{0.1, 0.1, 0.44}


% Document and protocol track
\newcommand\expt[1] {{\color {darkbrown} {\bf #1}}}       % Experiment
\newcommand\info[1] {{\color {midnightblue} {\bf #1}}}    % Informatoin

\newcommand\prop[1] {{\color {alizarin} {\bf #1}}}        % Proposal
\newcommand\draf[1] {{\color {amber(sae/ece)} {\bf #1}}}  % Draft
\newcommand\rele[1] {{\color {english} \bf {#1}}}         % Release
\newcommand\rewo[1] {{\color {aqua} {\bf #1}}}            % Rework

\newcommand\emps[1] {{\color {indigo(web)} {\bf #1}}}     % Emphasis

\newcommand\opti[1] {{\color {amethyst} {\bf #1}}}        % Optional
\newcommand\mand[1] {{\color {burntorange} {\bf #1}}}     % Mandatory


% Start of a LaTeX text, that is to be printed.
\begin{document}

% Front page
\title{ \textbf{\color{blue}\svthema} \Springtree [1.5] }
\author{ \textsl{\color{red}\svperson} --- \svdatum }
\date{}

% You tell LaTeX the information used to produce the title page.
\maketitle

\vskip 8pt
\p{{\info {Document-Track}}}

% Initialize the subparagraph counters.
\setcounter{subc}{0}

\begin{labeling}{Information:} 
  \setlength\itemsep{-3pt}
  
\item[Experiment:]  \textbackslash {\expt {expt}}
\item[Information:] \textbackslash {\info {info}}
\item[Proposal:]    \textbackslash {\prop {prop}}
\item[Draft:]       \textbackslash {\draf {draf}}
\item[Release:]     \textbackslash {\rele {rele}}
\item[Rework:]      \textbackslash {\rewo {rewo}}
\item[Emphasis:]    \textbackslash {\emps {emps}}

\end{labeling}

\vskip 4pt
\p{\draf {Anamnese}}

% Initialize the subparagraph counters.
\setcounter{subc}{0}

\vskip 8pt
\subp{\info {Bibliographie}}

\vskip 8pt
\verb+https://de.wikipedia.org/wiki/Alkoholkrankheit+ \\
\verb+https://de.wikipedia.org/wiki/Psychotrope_Substanz+ \\
\verb+https://de.wikipedia.org/wiki/Psyche+ \\
\verb+https://www.wittgensteinproject.org/w/index.php?title=+ \\
\verb+Philosophische_Untersuchungen#+ \\
\verb+https://de.wikipedia.org/wiki/Ethanol+ \\
\verb+https://www.dimdi.de/static/de/klassifikationen/icd/icd-10-who/kode-suche/htmlamtl2019/+

\vskip 12pt
% Command for \
\verb+https://de.wikibooks.org/wiki/LaTeX-Kompendium:_Sonderzeichen+ \\
% Kästchen zum ankreuzen automatisch multipli choise.
\verb+https://golatex.de/viewtopic.php?t=5926+

\vskip 12pt
\subp{\info {Definitionen}}

% Initialize the subparagraph counters.
\setcounter{subsubc}{0}

\vskip 12pt
\hskip -14pt
\begin{minipage}{0.90\textwidth}  
  \begin{itemize}
    \setlength\itemsep{-3pt}
  
  \item {\emps {Alkoholkrankheit}} ist die Abhängigkeit von der psychotropen
    Substanz Ethanol bzw. Alkohol.
    
  \item {\emps {Psychotrope Substanz}} ist ein Wirkstoff, der die menschliche
    Psyche beeinflusst.

  \item {\emps {Psyche}} ist die Gesamtheit aller geistigen Eigenschaften wie
    Denken, Lernen, Emotionen - Entspannung, Erleichterung, Euphorie $\ldots$ -,
    Wahrnehmen, Empfinden - Tastempfindung, Atmen, Handhaltung, $\ldots$ - ,
    Empathie, Wissen, Intuition und Motivation.

  \end{itemize}
\end{minipage}


\vskip 12pt
\subp{\rele {Symptome}}

% Initialize the subparagraph counters.
\setcounter{subsubc}{0}

\vskip 8pt
Jeden Tag kaufe ich ein bis zwei Flaschen Wein und trinke meist ab dem
späten Nachmittag oder eher selten ab Mittag solange, bis ich einschlafe. Meine
früheren Interessen wie Sport, Treffen mit Freunden $\ldots$ habe ich aufgegeben
ausser Programmieren verknüpft mit Trinken. Wenn ich nicht trinke, bin ich
ungeduldig und wortkarg.


\vskip 8pt
\subp{\rele {Klassifikation nach ICD-10}}

% Initialize the subparagraph counters.
\setcounter{subsubc}{0}

\vskip 12pt
\subsubp{\rele {Abhängigkeitssyndrom}}

\vskip 12pt
\hskip -14pt
\begin{minipage}{0.90\textwidth}
  
  \begin{enumerate}[label={\Square}]
    \setlength\itemsep{-3pt}
  
  \item [\XBox] Zwanghaftes Verlangen, Alkohol zu konsumieren
  \item [\XBox] Verminderte Kontrollfähigkeit bei der Menge
  \item Körperliche Entzugserscheinungen
  \item [\XBox] Nachweis einer Toleranz: zunehmend größere Mengen an Alkohol
  \item [\XBox] Einengung des Denkens auf Alkohol
  \item [\XBox] Anhaltender Substanzkonsum trotz gesundheitlicher und sozialer
    Folgeschäden
    
  \end{enumerate}
  
\end{minipage}

\vskip 8pt
Abhängigkeitssyndrom (F10.2) trifft bei mir zu, da mehr als zwei Kriterien mindestens einen
Monat lang gleichzeitig vorhanden sind.


\vskip 12pt
\subsubp{\rele {Akute Alkoholintoxikation (akuter Alkoholrausch)}}

\vskip 8pt
Eine akute Alkoholintoxikation (F10.0) trifft bei mir zu, da folgende
Verhaltensauffälligkeiten und Merkmale vorliegen:

\vskip 12pt
\hskip -14pt
\begin{minipage}{0.90\textwidth}
  
  \begin{itemize}
    \setlength\itemsep{-3pt}
  
  \item Enthemmung
  \item Aufmerksamkeitsstörung
  \item Einschränkung der Urteilsfähigkeit
  \item verwaschene Sprache
  \item Gesichtsröte (Erröten)
    
  \end{itemize}
  
\end{minipage}


\vskip 12pt
\subp{\rele {Klassifikation nach DSM-5}}

% Initialize the subparagraph counters.
\setcounter{subsubc}{0}


\vskip 12pt
\subsubp{\rele {Störung durch Alkoholkonsum (Alkoholkonsumstörung)}}

\vskip 12pt
\hskip -14pt
\begin{minipage}{0.90\textwidth}
  
  \begin{enumerate}[label={\Square}]
    \setlength\itemsep{-3pt}
  
  \item [\XBox] Alkohol wird in größeren Mengen oder länger als beabsichtigt
    konsumiert.
  \item [\XBox] erfolglose Versuche, den Alkoholkonsum zu verringern oder zu
    kontrollieren    
  \item [\XBox] hoher Zeitaufwand, um Alkohol zu beschaffen, zu konsumieren oder
    sich zu erholen
  \item [\XBox] Craving oder ein starkes Verlangen, Alkohol zu konsumieren
  \item [\XBox] Alkoholkonsum verbunden mit der Nicht-Erfüllung von
    Verpflichtungen: Arbeit und zu Hause
  \item [\XBox] Alkoholkonsum trotz ständiger sozialer oder zwischenmenschlicher
    Probleme
  \item [\XBox] soziale, berufliche oder Freizeitaktivitäten werden aufgrund des
    Alkoholkonsums aufgegeben
  \item Alkoholkonsum verbunden mit einer körperlichen Gefährdung
  \item [\XBox] Alkoholkonsum trotz Kenntnis eines anhaltenden oder
    wiederkehrenden psychischen Problems
  \item [\XBox] Toleranzentwicklung mit Dosissteigerung und verminderter Wirkung
  \item Entzugssymptome
    
  \end{enumerate}
  
\end{minipage}

\vskip 12pt
Alkoholkonsumstörung trifft bei mir zu, da mehr als ein Kriterium über zwölf
Monate vorliegt.

\vskip 8pt
Der Schweregrad ist schwer, da mehr als 5 Symptomkriterien erfüllt sind.


\vskip 10pt
\subp{\rele {Krankheitsverlauf und -bild}}

% Initialize the subparagraph counters.
\setcounter{subsubc}{0}

\vskip 12pt
\subsubp{\rele {Krankheitsverlauf}}

\vskip 8pt
Aktuell befinde ich mich wohl im Übergang von der kritischen zur chronischen
Phase.

\begin{labeling}{{\bf Chronische Phase:}} 
  \setlength\itemsep{-3pt}
  
\item[{\emps {Kritische Phase:}}]  Der Alkoholiker kann sein Trinken nun
  überhaupt nicht mehr kontrollieren. $\ldots$
  
\item[{\emps {Chronische Phase:}}] Der Alkohol beherrscht den Trinker nun
  vollkommen. Seine Persönlichkeit verändert sich. Er trinkt unter der Woche, am
  hellen Tag, häufig schon ab Mittag.  $\ldots$

\end{labeling}

\vskip 4pt
\subsubp{\rele {Ausprägungen der Krankheit}}

\vskip 8pt
Selbst würde ich mich einschätzen als {\emps {Gamma-Typ}} (Rauschtrinker, Alkoholiker),
denn ich trinke täglich. Typisch ist der Kontrollverlust: ich kann nicht
aufhören zu trinken bis ich einschlafe.


\vskip 8pt
\subp{\rele {Biologie der Alkoholsucht}}

% Initialize the subparagraph counters.
\setcounter{subsubc}{0}

\vskip 8pt
Alkohol verändert im Gehirn Rezeptoren, die bei mir die Entspannung verbessern. Wegen
deren Anpassung erhöhe ich die Alkoholmenge. \\
Mit Alkohol wird vermehrt Dopamin und Endorphine produziert. \\
Dopamin ist ein Hormon, das die Motivation fördert. \\
Endorphine sind körpereigene Opioide, die u.a. eine Euphorie hervorrufen.


\vskip 8pt
\subp{\rele {Krankheitsursachen}}

% Initialize the subparagraph counters.
\setcounter{subsubc}{0}

\vskip 12pt
\subsubp{\rele {Genetische Faktoren}}

\vskip 8pt
Meine Mutter war tablettensüchtig und vermutlich gibt es bei mir eine angeborene
Alkoholverträglichkeit.


\vskip 12pt
\subsubp{\rele {Psychologische Faktoren}}

\vskip 8pt
Die schnell eintretenden positiven Wirkungen des Alkohols wie Entspannung und
Glücksgefühle wie Euphorie, Zufriedenheit - Stolz, $\ldots$ -  verstärken das
Suchtverhalten.


\vskip 12pt
\subp{\rele {Rückfall}}

% Initialize the subparagraph counters.
\setcounter{subsubc}{0}

\vskip 8pt
Wenn ich mich nicht total abstinent verhalte, erwarte ich einen
{\bf schweren} {\bf Rückfall} {\bf (relapse)} in alte Trink\-muster, die sich
auf Alkoholmenge, Trinkfrequenz und Trinkdauer beziehen.

\newpage
\p{{\info {Protokoll-Track}}}

% Initialize the subparagraph counters.
\setcounter{subc}{0}

\begin{labeling}{Mandatory elements:} 
  \setlength\itemsep{-3pt}
  
\item[Mandatory elements:]  \textbackslash {\mand {mand}}
\item[Optional elements:]   \textbackslash {\opti {opti}}

\end{labeling}


\vskip 4pt
\subp{\info {Bibliographie}}

% Initialize the subparagraph counters.
\setcounter{subsubc}{0}

\vskip 8pt
\verb+https://de.wikipedia.org/wiki/Therapie+ \\
\verb+https://de.wikipedia.org/wiki/Protokoll_(Niederschrift)+ \\
\verb+https:https://de.wikipedia.org/wiki/Gem%C3%BCt+ \\
\verb+https://de.wikipedia.org/wiki/Selbsthilfegruppe+ \\
\verb+https://www.anonyme-alkoholiker.at/images/E-Medien/2021-docs/+ \\
\verb+2021_09%20-%20SMZ%20Liebenau.pdf+ \\
\verb+https://guttempler.org/+

\vskip 12pt
\verb+https://tex.stackexchange.com/questions/116101/add-bold-enumerate-items+

\vskip 12pt
\subp{\info {Definitionen}}

% Initialize the subparagraph counters.
\setcounter{subsubc}{0}

\vskip 12pt
\hskip -14pt
\begin{minipage}{0.90\textwidth}  
  \begin{itemize}
    \setlength\itemsep{-3pt}
  
  \item {\emps {Therapie}} umfaßt alle Maßnahmen, mit denen körperliche oder
    psychische Funktionen wiederhergestellt werden.
    
  \item {\emps {Protokoll}} ist eine Aufzeichnung, in der Zeitpunkt von
    Zuständen und Vorgänge aufgelistet werden.

  \item {\emps {Gemütszustand}} ist das aktuelle psychische Befinden eines
    Menschen.

  \item {\emps {Selbsthilfegruppe}} ist ein selbstorganisiertes Treffen von
    Menschen, die ein gleiches Problem haben wie Alkoholkrankheit.
    
  \item {\emps {Anonyme Alkoholiker}} oder {\emps {AA}} ist eine
    Selbsthilfegruppe für Alkoholiker mit dem Vorsatz:
    ,,Nur heute, 24 Stunden lang, lasse ich das erste Glas stehen.''
    
  \item {\emps {Guttempler}} ist eine Selbsthilfegruppe, die erwartet, dass man
    weder Alkohol noch andere bewußtseinsverändernde Drogen konsumiert.
    
  \end{itemize}
\end{minipage}

\vskip 12pt
\p{{\rele {Protokoll}}}

% Initialize the subparagraph counters.
\setcounter{subc}{0}


\vskip 12pt
\subp{\rele {KW 36}}

% Initialize the subparagraph counters.
\setcounter{subsubc}{0}

\vskip 10pt
{\mand {Gewicht:}} -

\vskip 10pt
\hrule
\vskip 10pt
{\rele {8.9.2022 - Donnerstag}}

\vskip 4pt
\hskip -6pt
\begin{minipage}{0.95\textwidth}  
  \begin{enumerate}
    \setlength\itemsep{-3pt}
    
  \item {\mand {Gemütszustand:}} \dTongey [1.5][yellow]
  \item {\mand {Abstinenz:}} -
  \item {\mand {Gesundheit:}} Puls: -, Blutdruck: -
  \item {\mand {Selbsthilfegruppe:}} -
  \item {\mand {Freunde:}} -
  \item {\mand {Körperpflege:}} -
  \item {\mand {Essen:}} -
  \item {\mand {Verwaltung:}} -
  \item {\mand {Zazen:}} -
  \item {\mand {Sport:}} -
  \item {\mand {Haus:}} -
  \item {\mand {Garten:}} -
  \item {\mand {Beruf:}} -
  \item {\opti {Beratung:}} \\
    Treffen mit Frau Etter Herzogenauach, die den Zustand meiner
    Alkoholkranheit ermittelt.
    
    {\bf Aktionspunkte:}
    
    \begin{minipage}{0.75\textwidth}  
      \begin{labeling}{Selbsthilfegruppe:} 
        \setlength\itemsep{-3pt}  
      \item[Arztbesuch:]        Leberwerte und Blutbild
      \item[Selbsthilfegruppe:] Treffen mit den Anonymen Alkoholikern oder Guttempler
      \item[Folgetermin:]       Erlangen: 14.09.22 um 14.30      
      \end{labeling}
    \end{minipage}
    
  \end{enumerate}
\end{minipage}


\newpage
{\rele {9.9.2022 - Freitag}}

\vskip 4pt
\hskip -6pt
\begin{minipage}{0.95\textwidth}  
  \begin{enumerate}
    \setlength\itemsep{-3pt}

  \item {\mand {Gemütszustand:}} \dSmiley [1.5][yellow]
  \item {\mand {Abstinenz:}} Geburt
  \item {\mand {Gesundheit:}} Puls: -, Blutdruck: -
  \item {\mand {Selbsthilfegruppe:}} AA in Erlangen von 19.00 - 21.00
  \item {\mand {Freunde:}} -
  \item {\mand {Körperpflege:}} -
  \item {\mand {Essen:}} -
  \item {\mand {Verwaltung:}} -
  \item {\mand {Zazen:}} -
  \item {\mand {Sport:}} -
  \item {\mand {Haus:}} -
  \item {\mand {Garten:}} -
  \item {\mand {Beruf:}} -
  \end{enumerate}
\end{minipage}


\vskip 4pt
\hrule
\vskip 10pt
{\rele {10.9.2022 - Samstag}}

\vskip 4pt
\hskip -6pt
\begin{minipage}{0.95\textwidth}  
  \begin{enumerate}
    \setlength\itemsep{-3pt}

  \item {\mand {Gemütszustand:}} \dSmiley [1.5][yellow]
  \item {\mand {Abstinenz:}} 1. Tag
  \item {\mand {Gesundheit:}} Puls: -, Blutdruck: -
  \item {\mand {Selbsthilfegruppe:}} AA in Nürnberg von 19.00 - 21.00
  \item {\mand {Freunde:}} -
  \item {\mand {Körperpflege:}} -
  \item {\mand {Essen:}} -
  \item {\mand {Verwaltung:}} -
  \item {\mand {Zazen:}} -
  \item {\mand {Sport:}} -
  \item {\mand {Haus:}} -
  \item {\mand {Garten:}} -
  \item {\mand {Beruf:}} -
  \end{enumerate}
\end{minipage}


\vskip 4pt
\hrule
\vskip 10pt
{\rele {11.9.2022 - Sonntag}}

\vskip 4pt
\hskip -6pt
\begin{minipage}{0.95\textwidth}  
  \begin{enumerate}
    \setlength\itemsep{-3pt}
    
  \item {\mand {Gemütszustand:}} \dSmiley [1.5][yellow]
  \item {\mand {Abstinenz:}} 2. Tag
  \item {\mand {Gesundheit:}} Puls: -, Blutdruck: -
  \item {\mand {Selbsthilfegruppe:}} AA in Erlangen von 19.00 - 21.00
  \item {\mand {Freunde:}} -
  \item {\mand {Körperpflege:}} -
  \item {\mand {Essen:}} -
  \item {\mand {Verwaltung:}} -
  \item {\mand {Zazen:}} -
  \item {\mand {Sport:}} -
  \item {\mand {Haus:}} -
  \item {\mand {Garten:}} -
  \item {\mand {Beruf:}} -
  \end{enumerate}
\end{minipage}


\newpage
\subp{\rele {KW 37}}

% Initialize the subparagraph counters.
\setcounter{subsubc}{0}

\vskip 10pt
{\mand {Gewicht:}} -

\vskip 10pt
\hrule
\vskip 10pt
{\rele {12.9.2022 - Montag}}

\vskip 4pt
\hskip -6pt
\begin{minipage}{0.95\textwidth}  
  \begin{enumerate}
    \setlength\itemsep{-3pt}

  \item {\mand {Gemütszustand:}} \dSmiley [1.5][yellow]
  \item {\mand {Abstinenz:}} 3. Tag
  \item {\mand {Gesundheit:}} Puls: -, Blutdruck: -
  \item {\mand {Selbsthilfegruppe:}} Guttempler in Höchstadt von 19.00 - 20.00
  \item {\mand {Freunde:}} -
  \item {\mand {Körperpflege:}} -
  \item {\mand {Essen:}} Ravioli, Wiener, Champignons
  \item {\mand {Verwaltung:}} -
    
  \item {\mand {Zazen:}} \\
    Telefongespräch mit Frau Godt wegen Zen-Meditation in Erlangen
            
    {\bf Aktionspunkt:}
    
    \begin{minipage}{0.75\textwidth}  
      \begin{labeling}{Termin:} 
        \setlength\itemsep{-3pt}  
      \item[Termin:] Erlangen: 20.09.22 um 19.45 Uhr
      \end{labeling}
    \end{minipage}

  \item {\mand {Sport:}} -
  \item {\mand {Haus:}} -
  \item {\mand {Garten:}} -
  \item {\mand {Beruf:}} -
  \item {\opti {Arztbesuch:}} \\
    Treffen mit Dr. Barabasch wegen Blutabnahme.
        
    {\bf Aktionspunkt:}
    
    \begin{minipage}{0.75\textwidth}  
      \begin{labeling}{Folgetermin::} 
        \setlength\itemsep{-3pt}  
      \item[Folgetermin:] Steppach: 13.09.22, vormittags
      \end{labeling}
    \end{minipage}
    
  \end{enumerate}
\end{minipage}


\vskip 4pt
\hrule
\vskip 10pt
{\rele {13.9.2022 - Dienstag}}

\vskip 4pt
\hskip -6pt
\begin{minipage}{0.95\textwidth}  
  \begin{enumerate}
    \setlength\itemsep{-3pt}

  \item {\mand {Gemütszustand:}} \dSmiley [1.5][yellow]
  \item {\mand {Abstinenz:}} -
  \item {\mand {Gesundheit:}} Puls: 64, Blutdruck: -
  \item {\mand {Selbsthilfegruppe:}} {\prop {AA in Erlangen von 19.00 - 21.00}}
  \item {\mand {Freunde:}} -
  \item {\mand {Körperpflege:}} {\prop {Duschen}}
  \item {\mand {Essen:}} {\prop {Blumenkohl, Hühnchen}}
  \item {\mand {Verwaltung:}} -
  \item {\mand {Zazen:}} -
  \item {\mand {Sport:}} -
  \item {\mand {Haus:}} -
  \item {\mand {Garten:}} -
  \item {\mand {Beruf:}} -
  \item {\opti {Arztbesuch:}} {\prop {Treffen mit Dr. Barabasch}}
  \end{enumerate}
\end{minipage}


\vskip 4pt
\hrule
\vskip 10pt
{\rele {14.9.2022 - Mittwoch}}

\vskip 8pt
\hskip -6pt
\begin{minipage}{0.95\textwidth}  
  \begin{enumerate}
    \setlength\itemsep{-3pt}

  \item {\mand {Gemütszustand:}} -
  \item {\mand {Abstinenz:}} -
  \item {\mand {Gesundheit:}} Puls: -, Blutdruck: -
  \item {\mand {Selbsthilfegruppe:}} {\prop {AA in Höchstadt von 19.30 - 21.00}}
  \item {\mand {Freunde:}} -
  \item {\mand {Körperpflege:}} -
  \item {\mand {Essen:}} -
  \item {\mand {Verwaltung:}} -
  \item {\mand {Zazen:}} -
  \item {\mand {Sport:}} -
  \item {\mand {Haus:}} -
  \item {\mand {Garten:}} -
  \item {\mand {Beruf:}} -
  \item {\opti {Beratung:}} {\prop {Treffen mit Frau Etter in Erlangen um 14.30 Uhr}}
  \end{enumerate}
\end{minipage}

\newpage
\subp{\rele {KW 38}}

% Initialize the subparagraph counters.
\setcounter{subsubc}{0}

\vskip 10pt
{\mand {Gewicht:}} -

$\ldots$

\vskip 4pt
\hrule
\vskip 10pt
{\rele {20.9.2022 - Dienstag}}
       
\vskip 4pt
\hskip -6pt
\begin{minipage}{0.95\textwidth}  
  \begin{enumerate}
    \setlength\itemsep{-3pt}

  \item {\mand {Gemütszustand:}} -
  \item {\mand {Abstinenz:}} -
  \item {\mand {Gesundheit:}} Puls: -, Blutdruck: -
  \item {\mand {Selbsthilfegruppe:}} -
  \item {\mand {Freunde:}} -
  \item {\mand {Körperpflege:}} -
  \item {\mand {Essen:}} -
  \item {\mand {Verwaltung:}} -
  \item {\mand {Zazen:}} {\prop {Meditation in Erlangen um 19.45 Uhr}}
  \item {\mand {Sport:}} -
  \item {\mand {Haus:}} -
  \item {\mand {Garten:}} -
  \item {\mand {Beruf:}} -
  \end{enumerate}
\end{minipage}

% End of a LaTeX text, that is to be printed.
\end{document}
